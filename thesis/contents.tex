
\chapter{Introduction(1)}

\chapter{Theory (4)}

\section{The Standard Model of particle physics}

\section[The decay \decay]{The decay \bolddecay}

\chapter{The LHCb Experiment (6)}

\chapter{Outline of the analysis (2)}

\begin{itemize}
  \item We want to determine limit on $B^0\to \overline{D^0}μ^+μ^-$
  \item First measurement
  \item Use control channel $B^0\to J/ψ K^*$ to reduce systematic uncertainties
  \item Blinded analysis
  \item Simple preselection
  \item Vetoes to reject peaking backgrounds: $J/ψ K^*$, $D^{*-}μν$
  \item Apply multivariate classifier to get rid of background
  \item Determine trigger efficiency through TISSTOS on control channel
  \item Determine selection efficiencies trough simulation
  \item Perform fit to control channel to derive control yield
  \item Perform 2-dimensional fit to $B$ and $D$ mass to derive signal yield
  \item Calculate $α$ and use it to transform to branching fraction
  \item Would be neat to write down how to calculate $α$
  \item Use Profile Likelihood Ratio to derive limit on branching fraction
\end{itemize}

\chapter{Selection (18)}

\section{Dataset}

\begin{itemize}
  \item 3 inverse femtobarn from LHCb Run I (2011+2012)
  \item Explain trigger lines used
  \item I used phase space simulation of $B^0\to\overline{D^0}μ^+μ^-$
  \item Which trigger lines do we use for the control channel? (Same, lol)
\end{itemize}

\section{Preselection criteria}

\begin{itemize}
  \item This explains the stripping line cuts
  \item What did we use for the control channel? Same (lol) with some differences: explain
  \item How is the blinding cut applied? Use objective criteria!
  \item How many events end up in the blinded sample?
  \item How many events do we have in our control channel?
\end{itemize}

\section{Vetoes to reject specific physical backgrounds}

\begin{figure}
  \centering
  \missingfigure[figwidth=0.7\textwidth]{The $B\to Dμμ$ dataset after stripping, with blinding}
  \caption{A figure of the $B\to Dμμ$ dataset after stripping, with blinding}
\end{figure}

\begin{figure}
  \centering
  \missingfigure[figwidth=0.7\textwidth]{Reconstructing the $D^*$ invariant mass and spotting background}
  \caption{Figure for $B^0\to D^{*-}μ^+ν$ background veto}
\end{figure}

\begin{itemize}
  \item How does the dataset look after stripping? Dem $J/ψ$ peaks lol
  \item Remove backgrounds with $J/ψ$ through cut on $μμ$ inv mass
  \item Reconstruct $D^*$ mass through $\overline{D}^0μ^-$, plot of the peak, veto $B^0\to D^{*-}μ^+ν$
  \item Treat other part reco as well???
\end{itemize}

\section{Multivariate classification}

\begin{itemize}
  \item What is all this ML anyway?
  \item Way too much background: Looking for N signal events in dataset of K
  \item Use ML methods to classify combine many variables into one discriminant
  \item Choose optimal cut on discriminant according to a criterium
\end{itemize}

\subsection{Gradient Boosted Decision Trees}

\begin{itemize}
  \item Very short explanation of how the algorithm works
\end{itemize}

\subsection{Features used in the classification}

\begin{itemize}
  \item Use phase space signal MC as signal proxy
  \item Use right sideband as bkg proxy. Why?
  \item Dem control plots
  \item Plots of signal MC, sweighted control and control MC
  \item correlation matrix
  \item importances?
\end{itemize}

\subsection{Classification of signal and background}

\begin{itemize}
  \item Plot of classifier output for signal MC, sidebands and control signal as well
  \item Plot of classifier output for control sweighted signal and bkg
  \item Give a KS test value for the control channel that shows that the classifier is unbiased
\end{itemize}

\subsection{Optimization of the Figure of Merit}

\begin{itemize}
  \item Choose an appropriate signal window for this (objectively)
  \item We want to optimize the classifier cut according to an objective criterium
  \item Use punzi figure of merit (formula)
  \item Use signal efficiency for s, don't need total number of signal candidates as it can be factorized and doesn't affect the maximum
  \item For background efficiency, take efficiency on sidebands. For total number of bkg candidates, perform initial fit (show plot)
  \item Show plot of FOM depending on cut point
  \item List maximum value, explain that this was used to cut on clf output
\end{itemize}

\section{Determination of the signal efficiency}

\begin{itemize}
  \item Get stripping efficiency from MC
  \item Get trigger efficiency from TISTOS on control channel
\end{itemize}

\chapter{Determination of the expected signal branching ratio (15)}

\section{Maximum Likelihood estimation}

\section{The Profile Likelihood Ratio}

\section{Signal model}

\section{Background models}

\section{Validation of fit model}

\section{Normalization channel}

\section{Normalization constant}

\section{Calculating the expected branching ratio}

\chapter{Systematic uncertainties (3)}

\chapter{Conclusion and outlook (1)}

% vim ft=latex
