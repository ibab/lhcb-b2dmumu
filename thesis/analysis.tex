
\chapter{Outline of the analysis}

The goal of this analysis is to determine a limit on the branching fraction of the decay $\PBzero\to\APDzero\APmuon\Pmuon$, using events where the $\APDzero$ decays to $\PKplus\Ppiminus$.
This represents the first measurement searching for this decay channel so far.

To minimize experimenter's bias, a blind analysis is performed.
This is achieved by removing a part of the reconstructed mass spectrum of the decay channel that fully contains the signal decay.
The analysis is developed without access to this signal region, which is unblinded (taken into account) only once the analysis is finalized.
A useful result of the analysis before unblinding is the calculation of an \emph{expected limit}, which represents an estimate of the limit that will be achieved after unblinding, under the assumption that there are no signal events in the signal region.

The first step of the analysis is a loose preselection involving cuts on kinematic properties and measures of fit quality for the decay.
Following this, several veto cuts are applied to reject physical backgrounds.
In order to reduce the amount of combinatorial background, a multivariate classifier is employed.
Various signal efficiencies resulting from the trigger, the reconstruction, the preselection, vetos on physical backgrounds and the multivariate classification need to be taken into account.
These are determined using a simulated signal data sample.

To reduce the systematic uncertainty of the branching fraction measurement, it is useful to perform a simultaneous measurement of the branching fraction of a \emph{normalization channel} and to normalize the signal branching fraction to this measurement.
The decay $\PBzero\to\PJpsi\PKstar$, where the $\PJpsi$ decays to $\APmuon\Pmuon$, is an ideal candidate for this, because it features both a large branching fraction, which reduces the uncertainty in the normalization process, and a final state ($\PKplus\Ppiminus\APmuon\Pmuon$) that is identical to the one of the signal decay.

Statistical models are developed for the reconstructed mass of both the signal and and normalization channels.
These can be used to determine the number of signal or normalization events remaining after the selection procedure through a Maximum Likelihood Estimate (MLE).
By calculating the normalization constant $\alpha$ and incorporating it into the fit model, the signal branching fraction can be determined directly as a parameter.

Based on the Likelihood model, the Profile Likelihood Ratio method is used to calculate limits on the branching fraction, taking into account nuisance parameters.
In order to calculate an expected limit for the branching fraction of the signal decay, toy simulations of the blinded signal region are generated using a model of the background distribution, which is determined through a fit to the sidebands of the mass distribution.
Each of the toys is combined with the data outside of the blinded region and a limit is calculated.
The distribution of calculated limits is used to determine an expected limit on the branching fraction of the signal decay.

\chapter{Selection}
\label{selection}

Before an estimate of the number of signal candidates can be extracted effectively, the number of background events in the data has to be reduced.
The following sections describe the different selection steps used in the analysis.

\section{Datasets}

The analyzed data samples correspond to the full integrated luminosity of \SI{3}{fb^{-1}} observed by LHCb during Run I. Throughout the analysis, data samples of the decays $\PBzero\to\APDzero\APmuon\Pmuon$ and $\PBzero\to\PJpsi\PKstar$ are used.

Simulated data samples were generated for both decay channels.
Because no reliable model of $\PBzero\to\APDzero\APmuon\Pmuon$ is available, the decay was simulated using a \emph{phase space} model.
This means that no assumptions about the kinematic dependence of the decay width $\Gamma$ were made outside of the three-body decay phase space factor.
% TODO why is that useful?
% TODO what about B->JpsiK* ?

In order to reduce experimenter bias, the analysis is conducted in a blinded fashion. This is achieved by removing a window of $\pm \SI{50}{MeV}$ in the reconstructed mass $m(\PKplus\Ppiminus\APmuon\Pmuon)$ around the nominal \PBzero mass in the $\PBzero\to\APDzero\APmuon\Pmuon$ dataset.
This way, all signal candidates are removed from the dataset, and a biasing optimization of analysis parameters is prevented.
% TODO why was this window chosen?
% TODO how much MC is in there?
% TODO how many data events do we have (after blinding)?

\section{Preselection criteria}

\subsection{Stripping selection}

As part of the LHCb-wide stripping selection, several loose cuts based on kinematic properties, fit quality and PID variables have been applied.
The stripping line \texttt{B2XMuMu} has been chosen for both of the decays $\PBzero\to\APDzero\APmuon\Pmuon$ and $\PBzero\to\PJpsi\PKstar$.

The selection criteria only differ in the selection of the \APDzero and \PKstar candidates and their decay products.
All selection criteria are listed in table \ref{tab:stripping}.

$\text{IP}\:\chi^2$ is the $\chi^2$ resulting from adding the chosen PV to the track of a particle using the Kalman fit, roughly equivalent to the IP significance $\text{IP}/\sigma(\text{IP})$.
DIRA angle refers to the angle between the momentum of a particle and its associated PV.
Vertex $\chi^2$ is the $\chi^2$ resulting from adding the tracks of all produced particles to the decay vertex.
Flight distance $\chi^2$ is the flight distance significance.
The ghost probability describes the likelihood that a given track (\emph{ghost track}) has been constructed without a corresponding particle track existing in the event.
It is calculated using a specialized ghost probability algorithm. \cite{Ghosts}
The DLL variables characterize the likelihood ratio for two given particle hypotheses (see section \ref{sec:pid}).
The \texttt{hasRICH} variable checks for the existence of RICH information for the given track.
The SPD hits variable correponds to the number of hits in the scintillating pad detector.
% TODO why is this useful?

The efficiency resulting from the limited geometrical acceptance of the LHCb detector is determined to be
\begin{align}
  \varepsilon_\text{geom}(\PBzero\to\APDzero\APmuon\Pmuon) &= \SI{15.79 \pm 0.05}{\percent} \\
  \varepsilon_\text{geom}(\PBzero\to\PJpsi\PKstar) &= \SI{16.050 \pm 0.04}{\percent}
\end{align}
using simulated candidates.

After applying the reconstruction and stripping selection to the two simulated samples, the combined reconstruction and stripping efficiencies are determined to be
\begin{align}
  \varepsilon_\text{reco\&strip}(\PBzero\to\APDzero\APmuon\Pmuon) &= \SI{10.99 \pm 0.03}{\percent} \\
  \varepsilon_\text{reco\&strip}(\PBzero\to\PJpsi\PKstar) &= \SI{9.687 \pm 0.010}{\percent}\:.
\end{align}

\begin{table}
  \centering
  \caption{
     Selection criteria applied to the signal and normalization channels as part of the stripping.
     Definitions of the variables used here are given in the text.
  }
  \begin{tabular}{l l}
    \toprule
    Target & Selection \\
    \midrule
    $\PBzero$ & $\text{IP}\:\chi^2 < 16$ \\
              & $\SI{4600}{MeV} < m(\PKplus\Ppiminus\APmuon\Pmuon) < \SI{7000}{MeV}$ \\
              & $\text{DIRA angle} < \SI{14}{mrad}$ \\
              & $\text{flight distance}\:\chi^2 > 121$ \\
              & $\text{vertex}\:\chi^2/\text{ndf} < 8$\\
    \midrule
    $\PKstar$ & $m(\PKplus\Ppiminus) < \SI{6200}{MeV}$ \\
              & $\text{flight distance}\:\chi^2 > 9$  \\
              & $\text{vertex}\:\chi^2/\text{ndf} < 12$ \\
    \midrule
    $\APDzero$ & $|m(\PKplus\Ppiminus) - m(\PDzero)| < \SI{100}{MeV}$ \\
               & $\text{flight distance}\:\chi^2 > 9$ \\
               & $\text{vertex}\:\chi^2/\text{ndf} < 10$ \\
    \midrule
    $\APmuon\Pmuon$ & $m(\APmuon\Pmuon)$ < \SI{7100}{MeV} \\
                    & $\text{vertex}\:\chi^2$ < 9 \\
    \midrule
    $\mu^\pm$       & \texttt{IsMuon} \\
                    & $\text{DLL}_{\mu\pi} > -3$ \\
    \midrule
    $K^\pm$       & \texttt{hasRICH} \\
                    & $\text{DLL}_{K\pi} > -5$ (only if from \PDzero) \\
    \midrule
    $\pi^\pm$      & \texttt{hasRICH} \\
    \midrule
    Track           & $\text{ghost probability} < 0.4$ \\
                    & $\min(\text{IP}\:\chi^2) > 9$ \\
    \midrule
    GEC             & $\text{SPD multiplicity} < 600$ \\
    \bottomrule
  \end{tabular}
  \label{tab:stripping}
\end{table}

% TODO check table

\begin{figure}
  \centering
  {%% Creator: Matplotlib, PGF backend
%%
%% To include the figure in your LaTeX document, write
%%   \input{<filename>.pgf}
%%
%% Make sure the required packages are loaded in your preamble
%%   \usepackage{pgf}
%%
%% Figures using additional raster images can only be included by \input if
%% they are in the same directory as the main LaTeX file. For loading figures
%% from other directories you can use the `import` package
%%   \usepackage{import}
%% and then include the figures with
%%   \import{<path to file>}{<filename>.pgf}
%%
%% Matplotlib used the following preamble
%%   \usepackage{fontspec}
%%   \setmainfont{DejaVu Serif}
%%   \setsansfont{DejaVu Sans}
%%   \setmonofont{DejaVu Sans Mono}
%%
\begingroup%
\makeatletter%
\begin{pgfpicture}%
\pgfpathrectangle{\pgfpointorigin}{\pgfqpoint{3.660865in}{2.158743in}}%
\pgfusepath{use as bounding box, clip}%
\begin{pgfscope}%
\pgfsetbuttcap%
\pgfsetmiterjoin%
\definecolor{currentfill}{rgb}{1.000000,1.000000,1.000000}%
\pgfsetfillcolor{currentfill}%
\pgfsetlinewidth{0.000000pt}%
\definecolor{currentstroke}{rgb}{1.000000,1.000000,1.000000}%
\pgfsetstrokecolor{currentstroke}%
\pgfsetdash{}{0pt}%
\pgfpathmoveto{\pgfqpoint{0.000000in}{0.000000in}}%
\pgfpathlineto{\pgfqpoint{3.660865in}{0.000000in}}%
\pgfpathlineto{\pgfqpoint{3.660865in}{2.158743in}}%
\pgfpathlineto{\pgfqpoint{0.000000in}{2.158743in}}%
\pgfpathclose%
\pgfusepath{fill}%
\end{pgfscope}%
\begin{pgfscope}%
\pgfsetbuttcap%
\pgfsetmiterjoin%
\definecolor{currentfill}{rgb}{1.000000,1.000000,1.000000}%
\pgfsetfillcolor{currentfill}%
\pgfsetlinewidth{0.000000pt}%
\definecolor{currentstroke}{rgb}{0.000000,0.000000,0.000000}%
\pgfsetstrokecolor{currentstroke}%
\pgfsetstrokeopacity{0.000000}%
\pgfsetdash{}{0pt}%
\pgfpathmoveto{\pgfqpoint{0.597879in}{0.440955in}}%
\pgfpathlineto{\pgfqpoint{3.469480in}{0.440955in}}%
\pgfpathlineto{\pgfqpoint{3.469480in}{2.054978in}}%
\pgfpathlineto{\pgfqpoint{0.597879in}{2.054978in}}%
\pgfpathclose%
\pgfusepath{fill}%
\end{pgfscope}%
\begin{pgfscope}%
\pgfpathrectangle{\pgfqpoint{0.597879in}{0.440955in}}{\pgfqpoint{2.871602in}{1.614023in}} %
\pgfusepath{clip}%
\pgfsetbuttcap%
\pgfsetmiterjoin%
\definecolor{currentfill}{rgb}{0.215686,0.470588,0.749020}%
\pgfsetfillcolor{currentfill}%
\pgfsetlinewidth{0.000000pt}%
\definecolor{currentstroke}{rgb}{0.000000,0.000000,0.000000}%
\pgfsetstrokecolor{currentstroke}%
\pgfsetdash{}{0pt}%
\pgfpathmoveto{\pgfqpoint{0.597879in}{0.440955in}}%
\pgfpathlineto{\pgfqpoint{0.597879in}{1.637484in}}%
\pgfpathlineto{\pgfqpoint{0.609365in}{1.637484in}}%
\pgfpathlineto{\pgfqpoint{0.609365in}{1.643940in}}%
\pgfpathlineto{\pgfqpoint{0.620852in}{1.643940in}}%
\pgfpathlineto{\pgfqpoint{0.620852in}{1.650755in}}%
\pgfpathlineto{\pgfqpoint{0.632338in}{1.650755in}}%
\pgfpathlineto{\pgfqpoint{0.632338in}{1.650934in}}%
\pgfpathlineto{\pgfqpoint{0.643824in}{1.650934in}}%
\pgfpathlineto{\pgfqpoint{0.643824in}{1.657032in}}%
\pgfpathlineto{\pgfqpoint{0.655311in}{1.657032in}}%
\pgfpathlineto{\pgfqpoint{0.655311in}{1.636408in}}%
\pgfpathlineto{\pgfqpoint{0.666797in}{1.636408in}}%
\pgfpathlineto{\pgfqpoint{0.666797in}{1.656135in}}%
\pgfpathlineto{\pgfqpoint{0.678284in}{1.656135in}}%
\pgfpathlineto{\pgfqpoint{0.678284in}{1.669406in}}%
\pgfpathlineto{\pgfqpoint{0.689770in}{1.669406in}}%
\pgfpathlineto{\pgfqpoint{0.689770in}{1.633539in}}%
\pgfpathlineto{\pgfqpoint{0.701256in}{1.633539in}}%
\pgfpathlineto{\pgfqpoint{0.701256in}{1.590319in}}%
\pgfpathlineto{\pgfqpoint{0.712743in}{1.590319in}}%
\pgfpathlineto{\pgfqpoint{0.712743in}{1.615964in}}%
\pgfpathlineto{\pgfqpoint{0.724229in}{1.615964in}}%
\pgfpathlineto{\pgfqpoint{0.724229in}{1.583863in}}%
\pgfpathlineto{\pgfqpoint{0.735716in}{1.583863in}}%
\pgfpathlineto{\pgfqpoint{0.735716in}{1.586911in}}%
\pgfpathlineto{\pgfqpoint{0.747202in}{1.586911in}}%
\pgfpathlineto{\pgfqpoint{0.747202in}{1.566288in}}%
\pgfpathlineto{\pgfqpoint{0.758688in}{1.566288in}}%
\pgfpathlineto{\pgfqpoint{0.758688in}{1.592292in}}%
\pgfpathlineto{\pgfqpoint{0.770175in}{1.592292in}}%
\pgfpathlineto{\pgfqpoint{0.770175in}{1.542436in}}%
\pgfpathlineto{\pgfqpoint{0.781661in}{1.542436in}}%
\pgfpathlineto{\pgfqpoint{0.781661in}{1.558756in}}%
\pgfpathlineto{\pgfqpoint{0.793148in}{1.558756in}}%
\pgfpathlineto{\pgfqpoint{0.793148in}{1.542795in}}%
\pgfpathlineto{\pgfqpoint{0.804634in}{1.542795in}}%
\pgfpathlineto{\pgfqpoint{0.804634in}{1.564674in}}%
\pgfpathlineto{\pgfqpoint{0.816120in}{1.564674in}}%
\pgfpathlineto{\pgfqpoint{0.816120in}{1.544409in}}%
\pgfpathlineto{\pgfqpoint{0.827607in}{1.544409in}}%
\pgfpathlineto{\pgfqpoint{0.827607in}{1.518943in}}%
\pgfpathlineto{\pgfqpoint{0.839093in}{1.518943in}}%
\pgfpathlineto{\pgfqpoint{0.839093in}{1.490249in}}%
\pgfpathlineto{\pgfqpoint{0.850580in}{1.490249in}}%
\pgfpathlineto{\pgfqpoint{0.850580in}{1.529345in}}%
\pgfpathlineto{\pgfqpoint{0.862066in}{1.529345in}}%
\pgfpathlineto{\pgfqpoint{0.862066in}{1.528269in}}%
\pgfpathlineto{\pgfqpoint{0.873552in}{1.528269in}}%
\pgfpathlineto{\pgfqpoint{0.873552in}{1.522351in}}%
\pgfpathlineto{\pgfqpoint{0.885039in}{1.522351in}}%
\pgfpathlineto{\pgfqpoint{0.885039in}{1.495630in}}%
\pgfpathlineto{\pgfqpoint{0.896525in}{1.495630in}}%
\pgfpathlineto{\pgfqpoint{0.896525in}{1.485228in}}%
\pgfpathlineto{\pgfqpoint{0.908012in}{1.485228in}}%
\pgfpathlineto{\pgfqpoint{0.908012in}{1.492401in}}%
\pgfpathlineto{\pgfqpoint{0.919498in}{1.492401in}}%
\pgfpathlineto{\pgfqpoint{0.919498in}{1.486663in}}%
\pgfpathlineto{\pgfqpoint{0.930985in}{1.486663in}}%
\pgfpathlineto{\pgfqpoint{0.930985in}{1.478413in}}%
\pgfpathlineto{\pgfqpoint{0.942471in}{1.478413in}}%
\pgfpathlineto{\pgfqpoint{0.942471in}{1.513025in}}%
\pgfpathlineto{\pgfqpoint{0.953957in}{1.513025in}}%
\pgfpathlineto{\pgfqpoint{0.953957in}{1.525399in}}%
\pgfpathlineto{\pgfqpoint{0.965444in}{1.525399in}}%
\pgfpathlineto{\pgfqpoint{0.965444in}{1.537056in}}%
\pgfpathlineto{\pgfqpoint{0.976930in}{1.537056in}}%
\pgfpathlineto{\pgfqpoint{0.976930in}{1.561266in}}%
\pgfpathlineto{\pgfqpoint{0.988417in}{1.561266in}}%
\pgfpathlineto{\pgfqpoint{0.988417in}{1.625110in}}%
\pgfpathlineto{\pgfqpoint{0.999903in}{1.625110in}}%
\pgfpathlineto{\pgfqpoint{0.999903in}{1.628159in}}%
\pgfpathlineto{\pgfqpoint{1.011389in}{1.628159in}}%
\pgfpathlineto{\pgfqpoint{1.011389in}{1.630669in}}%
\pgfpathlineto{\pgfqpoint{1.022876in}{1.630669in}}%
\pgfpathlineto{\pgfqpoint{1.022876in}{1.657928in}}%
\pgfpathlineto{\pgfqpoint{1.034362in}{1.657928in}}%
\pgfpathlineto{\pgfqpoint{1.034362in}{1.675503in}}%
\pgfpathlineto{\pgfqpoint{1.045849in}{1.675503in}}%
\pgfpathlineto{\pgfqpoint{1.045849in}{1.717289in}}%
\pgfpathlineto{\pgfqpoint{1.057335in}{1.717289in}}%
\pgfpathlineto{\pgfqpoint{1.057335in}{1.739168in}}%
\pgfpathlineto{\pgfqpoint{1.068821in}{1.739168in}}%
\pgfpathlineto{\pgfqpoint{1.068821in}{1.807674in}}%
\pgfpathlineto{\pgfqpoint{1.080308in}{1.807674in}}%
\pgfpathlineto{\pgfqpoint{1.080308in}{1.790637in}}%
\pgfpathlineto{\pgfqpoint{1.091794in}{1.790637in}}%
\pgfpathlineto{\pgfqpoint{1.091794in}{1.827580in}}%
\pgfpathlineto{\pgfqpoint{1.103281in}{1.827580in}}%
\pgfpathlineto{\pgfqpoint{1.103281in}{1.828298in}}%
\pgfpathlineto{\pgfqpoint{1.114767in}{1.828298in}}%
\pgfpathlineto{\pgfqpoint{1.114767in}{1.861295in}}%
\pgfpathlineto{\pgfqpoint{1.126253in}{1.861295in}}%
\pgfpathlineto{\pgfqpoint{1.126253in}{1.927829in}}%
\pgfpathlineto{\pgfqpoint{1.137740in}{1.927829in}}%
\pgfpathlineto{\pgfqpoint{1.137740in}{1.910433in}}%
\pgfpathlineto{\pgfqpoint{1.149226in}{1.910433in}}%
\pgfpathlineto{\pgfqpoint{1.149226in}{1.907923in}}%
\pgfpathlineto{\pgfqpoint{1.160713in}{1.907923in}}%
\pgfpathlineto{\pgfqpoint{1.160713in}{1.926574in}}%
\pgfpathlineto{\pgfqpoint{1.172199in}{1.926574in}}%
\pgfpathlineto{\pgfqpoint{1.172199in}{1.924601in}}%
\pgfpathlineto{\pgfqpoint{1.183685in}{1.924601in}}%
\pgfpathlineto{\pgfqpoint{1.183685in}{1.913303in}}%
\pgfpathlineto{\pgfqpoint{1.195172in}{1.913303in}}%
\pgfpathlineto{\pgfqpoint{1.195172in}{1.934106in}}%
\pgfpathlineto{\pgfqpoint{1.206658in}{1.934106in}}%
\pgfpathlineto{\pgfqpoint{1.206658in}{1.929443in}}%
\pgfpathlineto{\pgfqpoint{1.218145in}{1.929443in}}%
\pgfpathlineto{\pgfqpoint{1.218145in}{1.891603in}}%
\pgfpathlineto{\pgfqpoint{1.229631in}{1.891603in}}%
\pgfpathlineto{\pgfqpoint{1.229631in}{1.909716in}}%
\pgfpathlineto{\pgfqpoint{1.241117in}{1.909716in}}%
\pgfpathlineto{\pgfqpoint{1.241117in}{1.879408in}}%
\pgfpathlineto{\pgfqpoint{1.252604in}{1.879408in}}%
\pgfpathlineto{\pgfqpoint{1.252604in}{1.912585in}}%
\pgfpathlineto{\pgfqpoint{1.264090in}{1.912585in}}%
\pgfpathlineto{\pgfqpoint{1.264090in}{1.915813in}}%
\pgfpathlineto{\pgfqpoint{1.275577in}{1.915813in}}%
\pgfpathlineto{\pgfqpoint{1.275577in}{1.871697in}}%
\pgfpathlineto{\pgfqpoint{1.287063in}{1.871697in}}%
\pgfpathlineto{\pgfqpoint{1.287063in}{1.905053in}}%
\pgfpathlineto{\pgfqpoint{1.298550in}{1.905053in}}%
\pgfpathlineto{\pgfqpoint{1.298550in}{1.877077in}}%
\pgfpathlineto{\pgfqpoint{1.310036in}{1.877077in}}%
\pgfpathlineto{\pgfqpoint{1.310036in}{1.894293in}}%
\pgfpathlineto{\pgfqpoint{1.321522in}{1.894293in}}%
\pgfpathlineto{\pgfqpoint{1.321522in}{1.939665in}}%
\pgfpathlineto{\pgfqpoint{1.333009in}{1.939665in}}%
\pgfpathlineto{\pgfqpoint{1.333009in}{1.918503in}}%
\pgfpathlineto{\pgfqpoint{1.344495in}{1.918503in}}%
\pgfpathlineto{\pgfqpoint{1.344495in}{1.201160in}}%
\pgfpathlineto{\pgfqpoint{1.355982in}{1.201160in}}%
\pgfpathlineto{\pgfqpoint{1.355982in}{0.440955in}}%
\pgfpathlineto{\pgfqpoint{1.367468in}{0.440955in}}%
\pgfpathlineto{\pgfqpoint{1.367468in}{0.440955in}}%
\pgfpathlineto{\pgfqpoint{1.378954in}{0.440955in}}%
\pgfpathlineto{\pgfqpoint{1.378954in}{0.440955in}}%
\pgfpathlineto{\pgfqpoint{1.390441in}{0.440955in}}%
\pgfpathlineto{\pgfqpoint{1.390441in}{0.440955in}}%
\pgfpathlineto{\pgfqpoint{1.401927in}{0.440955in}}%
\pgfpathlineto{\pgfqpoint{1.401927in}{0.440955in}}%
\pgfpathlineto{\pgfqpoint{1.413414in}{0.440955in}}%
\pgfpathlineto{\pgfqpoint{1.413414in}{0.440955in}}%
\pgfpathlineto{\pgfqpoint{1.424900in}{0.440955in}}%
\pgfpathlineto{\pgfqpoint{1.424900in}{0.440955in}}%
\pgfpathlineto{\pgfqpoint{1.436386in}{0.440955in}}%
\pgfpathlineto{\pgfqpoint{1.436386in}{0.440955in}}%
\pgfpathlineto{\pgfqpoint{1.447873in}{0.440955in}}%
\pgfpathlineto{\pgfqpoint{1.447873in}{0.440955in}}%
\pgfpathlineto{\pgfqpoint{1.459359in}{0.440955in}}%
\pgfpathlineto{\pgfqpoint{1.459359in}{0.529547in}}%
\pgfpathlineto{\pgfqpoint{1.470846in}{0.529547in}}%
\pgfpathlineto{\pgfqpoint{1.470846in}{1.832602in}}%
\pgfpathlineto{\pgfqpoint{1.482332in}{1.832602in}}%
\pgfpathlineto{\pgfqpoint{1.482332in}{1.783284in}}%
\pgfpathlineto{\pgfqpoint{1.493818in}{1.783284in}}%
\pgfpathlineto{\pgfqpoint{1.493818in}{1.761047in}}%
\pgfpathlineto{\pgfqpoint{1.505305in}{1.761047in}}%
\pgfpathlineto{\pgfqpoint{1.505305in}{1.728407in}}%
\pgfpathlineto{\pgfqpoint{1.516791in}{1.728407in}}%
\pgfpathlineto{\pgfqpoint{1.516791in}{1.675503in}}%
\pgfpathlineto{\pgfqpoint{1.528278in}{1.675503in}}%
\pgfpathlineto{\pgfqpoint{1.528278in}{1.676400in}}%
\pgfpathlineto{\pgfqpoint{1.539764in}{1.676400in}}%
\pgfpathlineto{\pgfqpoint{1.539764in}{1.650934in}}%
\pgfpathlineto{\pgfqpoint{1.551250in}{1.650934in}}%
\pgfpathlineto{\pgfqpoint{1.551250in}{1.642864in}}%
\pgfpathlineto{\pgfqpoint{1.562737in}{1.642864in}}%
\pgfpathlineto{\pgfqpoint{1.562737in}{1.629055in}}%
\pgfpathlineto{\pgfqpoint{1.574223in}{1.629055in}}%
\pgfpathlineto{\pgfqpoint{1.574223in}{1.613633in}}%
\pgfpathlineto{\pgfqpoint{1.585710in}{1.613633in}}%
\pgfpathlineto{\pgfqpoint{1.585710in}{1.612198in}}%
\pgfpathlineto{\pgfqpoint{1.597196in}{1.612198in}}%
\pgfpathlineto{\pgfqpoint{1.597196in}{1.576689in}}%
\pgfpathlineto{\pgfqpoint{1.608682in}{1.576689in}}%
\pgfpathlineto{\pgfqpoint{1.608682in}{1.552838in}}%
\pgfpathlineto{\pgfqpoint{1.620169in}{1.552838in}}%
\pgfpathlineto{\pgfqpoint{1.620169in}{1.537953in}}%
\pgfpathlineto{\pgfqpoint{1.631655in}{1.537953in}}%
\pgfpathlineto{\pgfqpoint{1.631655in}{1.498858in}}%
\pgfpathlineto{\pgfqpoint{1.643142in}{1.498858in}}%
\pgfpathlineto{\pgfqpoint{1.643142in}{1.528269in}}%
\pgfpathlineto{\pgfqpoint{1.654628in}{1.528269in}}%
\pgfpathlineto{\pgfqpoint{1.654628in}{1.515177in}}%
\pgfpathlineto{\pgfqpoint{1.666115in}{1.515177in}}%
\pgfpathlineto{\pgfqpoint{1.666115in}{1.495092in}}%
\pgfpathlineto{\pgfqpoint{1.677601in}{1.495092in}}%
\pgfpathlineto{\pgfqpoint{1.677601in}{1.460659in}}%
\pgfpathlineto{\pgfqpoint{1.689087in}{1.460659in}}%
\pgfpathlineto{\pgfqpoint{1.689087in}{1.454203in}}%
\pgfpathlineto{\pgfqpoint{1.700574in}{1.454203in}}%
\pgfpathlineto{\pgfqpoint{1.700574in}{1.451692in}}%
\pgfpathlineto{\pgfqpoint{1.712060in}{1.451692in}}%
\pgfpathlineto{\pgfqpoint{1.712060in}{1.448106in}}%
\pgfpathlineto{\pgfqpoint{1.723547in}{1.448106in}}%
\pgfpathlineto{\pgfqpoint{1.723547in}{1.409010in}}%
\pgfpathlineto{\pgfqpoint{1.735033in}{1.409010in}}%
\pgfpathlineto{\pgfqpoint{1.735033in}{1.394843in}}%
\pgfpathlineto{\pgfqpoint{1.746519in}{1.394843in}}%
\pgfpathlineto{\pgfqpoint{1.746519in}{1.402554in}}%
\pgfpathlineto{\pgfqpoint{1.758006in}{1.402554in}}%
\pgfpathlineto{\pgfqpoint{1.758006in}{1.396277in}}%
\pgfpathlineto{\pgfqpoint{1.769492in}{1.396277in}}%
\pgfpathlineto{\pgfqpoint{1.769492in}{1.369198in}}%
\pgfpathlineto{\pgfqpoint{1.780979in}{1.369198in}}%
\pgfpathlineto{\pgfqpoint{1.780979in}{1.352699in}}%
\pgfpathlineto{\pgfqpoint{1.792465in}{1.352699in}}%
\pgfpathlineto{\pgfqpoint{1.792465in}{1.368480in}}%
\pgfpathlineto{\pgfqpoint{1.803951in}{1.368480in}}%
\pgfpathlineto{\pgfqpoint{1.803951in}{1.337635in}}%
\pgfpathlineto{\pgfqpoint{1.815438in}{1.337635in}}%
\pgfpathlineto{\pgfqpoint{1.815438in}{1.345346in}}%
\pgfpathlineto{\pgfqpoint{1.826924in}{1.345346in}}%
\pgfpathlineto{\pgfqpoint{1.826924in}{1.325978in}}%
\pgfpathlineto{\pgfqpoint{1.838411in}{1.325978in}}%
\pgfpathlineto{\pgfqpoint{1.838411in}{1.326695in}}%
\pgfpathlineto{\pgfqpoint{1.849897in}{1.326695in}}%
\pgfpathlineto{\pgfqpoint{1.849897in}{1.314142in}}%
\pgfpathlineto{\pgfqpoint{1.861383in}{1.314142in}}%
\pgfpathlineto{\pgfqpoint{1.861383in}{1.306430in}}%
\pgfpathlineto{\pgfqpoint{1.872870in}{1.306430in}}%
\pgfpathlineto{\pgfqpoint{1.872870in}{1.299077in}}%
\pgfpathlineto{\pgfqpoint{1.884356in}{1.299077in}}%
\pgfpathlineto{\pgfqpoint{1.884356in}{1.278633in}}%
\pgfpathlineto{\pgfqpoint{1.895843in}{1.278633in}}%
\pgfpathlineto{\pgfqpoint{1.895843in}{1.271280in}}%
\pgfpathlineto{\pgfqpoint{1.907329in}{1.271280in}}%
\pgfpathlineto{\pgfqpoint{1.907329in}{1.294953in}}%
\pgfpathlineto{\pgfqpoint{1.918815in}{1.294953in}}%
\pgfpathlineto{\pgfqpoint{1.918815in}{1.282937in}}%
\pgfpathlineto{\pgfqpoint{1.930302in}{1.282937in}}%
\pgfpathlineto{\pgfqpoint{1.930302in}{1.325081in}}%
\pgfpathlineto{\pgfqpoint{1.941788in}{1.325081in}}%
\pgfpathlineto{\pgfqpoint{1.941788in}{1.326336in}}%
\pgfpathlineto{\pgfqpoint{1.953275in}{1.326336in}}%
\pgfpathlineto{\pgfqpoint{1.953275in}{1.344629in}}%
\pgfpathlineto{\pgfqpoint{1.964761in}{1.344629in}}%
\pgfpathlineto{\pgfqpoint{1.964761in}{1.351085in}}%
\pgfpathlineto{\pgfqpoint{1.976247in}{1.351085in}}%
\pgfpathlineto{\pgfqpoint{1.976247in}{1.338711in}}%
\pgfpathlineto{\pgfqpoint{1.987734in}{1.338711in}}%
\pgfpathlineto{\pgfqpoint{1.987734in}{1.348215in}}%
\pgfpathlineto{\pgfqpoint{1.999220in}{1.348215in}}%
\pgfpathlineto{\pgfqpoint{1.999220in}{1.370094in}}%
\pgfpathlineto{\pgfqpoint{2.010707in}{1.370094in}}%
\pgfpathlineto{\pgfqpoint{2.010707in}{1.350009in}}%
\pgfpathlineto{\pgfqpoint{2.022193in}{1.350009in}}%
\pgfpathlineto{\pgfqpoint{2.022193in}{1.362383in}}%
\pgfpathlineto{\pgfqpoint{2.033680in}{1.362383in}}%
\pgfpathlineto{\pgfqpoint{2.033680in}{1.359155in}}%
\pgfpathlineto{\pgfqpoint{2.045166in}{1.359155in}}%
\pgfpathlineto{\pgfqpoint{2.045166in}{1.347139in}}%
\pgfpathlineto{\pgfqpoint{2.056652in}{1.347139in}}%
\pgfpathlineto{\pgfqpoint{2.056652in}{1.357720in}}%
\pgfpathlineto{\pgfqpoint{2.068139in}{1.357720in}}%
\pgfpathlineto{\pgfqpoint{2.068139in}{1.331537in}}%
\pgfpathlineto{\pgfqpoint{2.079625in}{1.331537in}}%
\pgfpathlineto{\pgfqpoint{2.079625in}{1.339787in}}%
\pgfpathlineto{\pgfqpoint{2.091112in}{1.339787in}}%
\pgfpathlineto{\pgfqpoint{2.091112in}{1.322032in}}%
\pgfpathlineto{\pgfqpoint{2.102598in}{1.322032in}}%
\pgfpathlineto{\pgfqpoint{2.102598in}{1.326695in}}%
\pgfpathlineto{\pgfqpoint{2.114084in}{1.326695in}}%
\pgfpathlineto{\pgfqpoint{2.114084in}{1.292263in}}%
\pgfpathlineto{\pgfqpoint{2.125571in}{1.292263in}}%
\pgfpathlineto{\pgfqpoint{2.125571in}{1.273074in}}%
\pgfpathlineto{\pgfqpoint{2.137057in}{1.273074in}}%
\pgfpathlineto{\pgfqpoint{2.137057in}{1.244739in}}%
\pgfpathlineto{\pgfqpoint{2.148544in}{1.244739in}}%
\pgfpathlineto{\pgfqpoint{2.148544in}{1.234158in}}%
\pgfpathlineto{\pgfqpoint{2.160030in}{1.234158in}}%
\pgfpathlineto{\pgfqpoint{2.160030in}{1.199546in}}%
\pgfpathlineto{\pgfqpoint{2.171516in}{1.199546in}}%
\pgfpathlineto{\pgfqpoint{2.171516in}{1.195601in}}%
\pgfpathlineto{\pgfqpoint{2.183003in}{1.195601in}}%
\pgfpathlineto{\pgfqpoint{2.183003in}{1.158837in}}%
\pgfpathlineto{\pgfqpoint{2.194489in}{1.158837in}}%
\pgfpathlineto{\pgfqpoint{2.194489in}{1.136061in}}%
\pgfpathlineto{\pgfqpoint{2.205976in}{1.136061in}}%
\pgfpathlineto{\pgfqpoint{2.205976in}{1.118845in}}%
\pgfpathlineto{\pgfqpoint{2.217462in}{1.118845in}}%
\pgfpathlineto{\pgfqpoint{2.217462in}{1.101091in}}%
\pgfpathlineto{\pgfqpoint{2.228948in}{1.101091in}}%
\pgfpathlineto{\pgfqpoint{2.228948in}{1.100553in}}%
\pgfpathlineto{\pgfqpoint{2.240435in}{1.100553in}}%
\pgfpathlineto{\pgfqpoint{2.240435in}{1.076522in}}%
\pgfpathlineto{\pgfqpoint{2.251921in}{1.076522in}}%
\pgfpathlineto{\pgfqpoint{2.251921in}{1.051235in}}%
\pgfpathlineto{\pgfqpoint{2.263408in}{1.051235in}}%
\pgfpathlineto{\pgfqpoint{2.263408in}{1.027025in}}%
\pgfpathlineto{\pgfqpoint{2.274894in}{1.027025in}}%
\pgfpathlineto{\pgfqpoint{2.274894in}{1.033840in}}%
\pgfpathlineto{\pgfqpoint{2.286380in}{1.033840in}}%
\pgfpathlineto{\pgfqpoint{2.286380in}{0.996538in}}%
\pgfpathlineto{\pgfqpoint{2.297867in}{0.996538in}}%
\pgfpathlineto{\pgfqpoint{2.297867in}{0.992951in}}%
\pgfpathlineto{\pgfqpoint{2.309353in}{0.992951in}}%
\pgfpathlineto{\pgfqpoint{2.309353in}{0.964078in}}%
\pgfpathlineto{\pgfqpoint{2.320840in}{0.964078in}}%
\pgfpathlineto{\pgfqpoint{2.320840in}{0.965871in}}%
\pgfpathlineto{\pgfqpoint{2.332326in}{0.965871in}}%
\pgfpathlineto{\pgfqpoint{2.332326in}{0.955470in}}%
\pgfpathlineto{\pgfqpoint{2.343812in}{0.955470in}}%
\pgfpathlineto{\pgfqpoint{2.343812in}{0.919423in}}%
\pgfpathlineto{\pgfqpoint{2.355299in}{0.919423in}}%
\pgfpathlineto{\pgfqpoint{2.355299in}{0.937895in}}%
\pgfpathlineto{\pgfqpoint{2.366785in}{0.937895in}}%
\pgfpathlineto{\pgfqpoint{2.366785in}{0.934129in}}%
\pgfpathlineto{\pgfqpoint{2.378272in}{0.934129in}}%
\pgfpathlineto{\pgfqpoint{2.378272in}{0.900773in}}%
\pgfpathlineto{\pgfqpoint{2.389758in}{0.900773in}}%
\pgfpathlineto{\pgfqpoint{2.389758in}{0.897544in}}%
\pgfpathlineto{\pgfqpoint{2.401245in}{0.897544in}}%
\pgfpathlineto{\pgfqpoint{2.401245in}{0.882839in}}%
\pgfpathlineto{\pgfqpoint{2.412731in}{0.882839in}}%
\pgfpathlineto{\pgfqpoint{2.412731in}{0.880149in}}%
\pgfpathlineto{\pgfqpoint{2.424217in}{0.880149in}}%
\pgfpathlineto{\pgfqpoint{2.424217in}{0.871720in}}%
\pgfpathlineto{\pgfqpoint{2.435704in}{0.871720in}}%
\pgfpathlineto{\pgfqpoint{2.435704in}{0.844820in}}%
\pgfpathlineto{\pgfqpoint{2.447190in}{0.844820in}}%
\pgfpathlineto{\pgfqpoint{2.447190in}{0.843744in}}%
\pgfpathlineto{\pgfqpoint{2.458677in}{0.843744in}}%
\pgfpathlineto{\pgfqpoint{2.458677in}{0.828859in}}%
\pgfpathlineto{\pgfqpoint{2.470163in}{0.828859in}}%
\pgfpathlineto{\pgfqpoint{2.470163in}{0.837826in}}%
\pgfpathlineto{\pgfqpoint{2.481649in}{0.837826in}}%
\pgfpathlineto{\pgfqpoint{2.481649in}{0.828859in}}%
\pgfpathlineto{\pgfqpoint{2.493136in}{0.828859in}}%
\pgfpathlineto{\pgfqpoint{2.493136in}{0.815588in}}%
\pgfpathlineto{\pgfqpoint{2.504622in}{0.815588in}}%
\pgfpathlineto{\pgfqpoint{2.504622in}{0.804649in}}%
\pgfpathlineto{\pgfqpoint{2.516109in}{0.804649in}}%
\pgfpathlineto{\pgfqpoint{2.516109in}{0.792633in}}%
\pgfpathlineto{\pgfqpoint{2.527595in}{0.792633in}}%
\pgfpathlineto{\pgfqpoint{2.527595in}{0.784563in}}%
\pgfpathlineto{\pgfqpoint{2.539081in}{0.784563in}}%
\pgfpathlineto{\pgfqpoint{2.539081in}{0.784384in}}%
\pgfpathlineto{\pgfqpoint{2.550568in}{0.784384in}}%
\pgfpathlineto{\pgfqpoint{2.550568in}{0.767705in}}%
\pgfpathlineto{\pgfqpoint{2.562054in}{0.767705in}}%
\pgfpathlineto{\pgfqpoint{2.562054in}{0.759456in}}%
\pgfpathlineto{\pgfqpoint{2.573541in}{0.759456in}}%
\pgfpathlineto{\pgfqpoint{2.573541in}{0.759097in}}%
\pgfpathlineto{\pgfqpoint{2.585027in}{0.759097in}}%
\pgfpathlineto{\pgfqpoint{2.585027in}{0.749413in}}%
\pgfpathlineto{\pgfqpoint{2.596513in}{0.749413in}}%
\pgfpathlineto{\pgfqpoint{2.596513in}{0.755331in}}%
\pgfpathlineto{\pgfqpoint{2.608000in}{0.755331in}}%
\pgfpathlineto{\pgfqpoint{2.608000in}{0.744392in}}%
\pgfpathlineto{\pgfqpoint{2.619486in}{0.744392in}}%
\pgfpathlineto{\pgfqpoint{2.619486in}{0.739729in}}%
\pgfpathlineto{\pgfqpoint{2.630973in}{0.739729in}}%
\pgfpathlineto{\pgfqpoint{2.630973in}{0.738294in}}%
\pgfpathlineto{\pgfqpoint{2.642459in}{0.738294in}}%
\pgfpathlineto{\pgfqpoint{2.642459in}{0.722333in}}%
\pgfpathlineto{\pgfqpoint{2.653945in}{0.722333in}}%
\pgfpathlineto{\pgfqpoint{2.653945in}{0.724306in}}%
\pgfpathlineto{\pgfqpoint{2.665432in}{0.724306in}}%
\pgfpathlineto{\pgfqpoint{2.665432in}{0.719105in}}%
\pgfpathlineto{\pgfqpoint{2.676918in}{0.719105in}}%
\pgfpathlineto{\pgfqpoint{2.676918in}{0.709242in}}%
\pgfpathlineto{\pgfqpoint{2.688405in}{0.709242in}}%
\pgfpathlineto{\pgfqpoint{2.688405in}{0.703503in}}%
\pgfpathlineto{\pgfqpoint{2.699891in}{0.703503in}}%
\pgfpathlineto{\pgfqpoint{2.699891in}{0.713546in}}%
\pgfpathlineto{\pgfqpoint{2.711377in}{0.713546in}}%
\pgfpathlineto{\pgfqpoint{2.711377in}{0.697764in}}%
\pgfpathlineto{\pgfqpoint{2.722864in}{0.697764in}}%
\pgfpathlineto{\pgfqpoint{2.722864in}{0.699378in}}%
\pgfpathlineto{\pgfqpoint{2.734350in}{0.699378in}}%
\pgfpathlineto{\pgfqpoint{2.734350in}{0.680369in}}%
\pgfpathlineto{\pgfqpoint{2.745837in}{0.680369in}}%
\pgfpathlineto{\pgfqpoint{2.745837in}{0.678575in}}%
\pgfpathlineto{\pgfqpoint{2.757323in}{0.678575in}}%
\pgfpathlineto{\pgfqpoint{2.757323in}{0.672478in}}%
\pgfpathlineto{\pgfqpoint{2.768810in}{0.672478in}}%
\pgfpathlineto{\pgfqpoint{2.768810in}{0.678755in}}%
\pgfpathlineto{\pgfqpoint{2.780296in}{0.678755in}}%
\pgfpathlineto{\pgfqpoint{2.780296in}{0.664408in}}%
\pgfpathlineto{\pgfqpoint{2.791782in}{0.664408in}}%
\pgfpathlineto{\pgfqpoint{2.791782in}{0.660642in}}%
\pgfpathlineto{\pgfqpoint{2.803269in}{0.660642in}}%
\pgfpathlineto{\pgfqpoint{2.803269in}{0.665305in}}%
\pgfpathlineto{\pgfqpoint{2.814755in}{0.665305in}}%
\pgfpathlineto{\pgfqpoint{2.814755in}{0.669071in}}%
\pgfpathlineto{\pgfqpoint{2.826242in}{0.669071in}}%
\pgfpathlineto{\pgfqpoint{2.826242in}{0.657593in}}%
\pgfpathlineto{\pgfqpoint{2.837728in}{0.657593in}}%
\pgfpathlineto{\pgfqpoint{2.837728in}{0.655082in}}%
\pgfpathlineto{\pgfqpoint{2.849214in}{0.655082in}}%
\pgfpathlineto{\pgfqpoint{2.849214in}{0.652930in}}%
\pgfpathlineto{\pgfqpoint{2.860701in}{0.652930in}}%
\pgfpathlineto{\pgfqpoint{2.860701in}{0.647012in}}%
\pgfpathlineto{\pgfqpoint{2.872187in}{0.647012in}}%
\pgfpathlineto{\pgfqpoint{2.872187in}{0.640736in}}%
\pgfpathlineto{\pgfqpoint{2.883674in}{0.640736in}}%
\pgfpathlineto{\pgfqpoint{2.883674in}{0.630513in}}%
\pgfpathlineto{\pgfqpoint{2.895160in}{0.630513in}}%
\pgfpathlineto{\pgfqpoint{2.895160in}{0.637866in}}%
\pgfpathlineto{\pgfqpoint{2.906646in}{0.637866in}}%
\pgfpathlineto{\pgfqpoint{2.906646in}{0.634638in}}%
\pgfpathlineto{\pgfqpoint{2.918133in}{0.634638in}}%
\pgfpathlineto{\pgfqpoint{2.918133in}{0.632845in}}%
\pgfpathlineto{\pgfqpoint{2.929619in}{0.632845in}}%
\pgfpathlineto{\pgfqpoint{2.929619in}{0.625851in}}%
\pgfpathlineto{\pgfqpoint{2.941106in}{0.625851in}}%
\pgfpathlineto{\pgfqpoint{2.941106in}{0.635355in}}%
\pgfpathlineto{\pgfqpoint{2.952592in}{0.635355in}}%
\pgfpathlineto{\pgfqpoint{2.952592in}{0.623161in}}%
\pgfpathlineto{\pgfqpoint{2.964078in}{0.623161in}}%
\pgfpathlineto{\pgfqpoint{2.964078in}{0.625671in}}%
\pgfpathlineto{\pgfqpoint{2.975565in}{0.625671in}}%
\pgfpathlineto{\pgfqpoint{2.975565in}{0.617960in}}%
\pgfpathlineto{\pgfqpoint{2.987051in}{0.617960in}}%
\pgfpathlineto{\pgfqpoint{2.987051in}{0.615270in}}%
\pgfpathlineto{\pgfqpoint{2.998538in}{0.615270in}}%
\pgfpathlineto{\pgfqpoint{2.998538in}{0.615091in}}%
\pgfpathlineto{\pgfqpoint{3.010024in}{0.615091in}}%
\pgfpathlineto{\pgfqpoint{3.010024in}{0.612580in}}%
\pgfpathlineto{\pgfqpoint{3.021510in}{0.612580in}}%
\pgfpathlineto{\pgfqpoint{3.021510in}{0.613118in}}%
\pgfpathlineto{\pgfqpoint{3.032997in}{0.613118in}}%
\pgfpathlineto{\pgfqpoint{3.032997in}{0.598233in}}%
\pgfpathlineto{\pgfqpoint{3.044483in}{0.598233in}}%
\pgfpathlineto{\pgfqpoint{3.044483in}{0.605765in}}%
\pgfpathlineto{\pgfqpoint{3.055970in}{0.605765in}}%
\pgfpathlineto{\pgfqpoint{3.055970in}{0.606303in}}%
\pgfpathlineto{\pgfqpoint{3.067456in}{0.606303in}}%
\pgfpathlineto{\pgfqpoint{3.067456in}{0.590342in}}%
\pgfpathlineto{\pgfqpoint{3.078942in}{0.590342in}}%
\pgfpathlineto{\pgfqpoint{3.078942in}{0.594467in}}%
\pgfpathlineto{\pgfqpoint{3.090429in}{0.594467in}}%
\pgfpathlineto{\pgfqpoint{3.090429in}{0.587293in}}%
\pgfpathlineto{\pgfqpoint{3.101915in}{0.587293in}}%
\pgfpathlineto{\pgfqpoint{3.101915in}{0.584245in}}%
\pgfpathlineto{\pgfqpoint{3.113402in}{0.584245in}}%
\pgfpathlineto{\pgfqpoint{3.113402in}{0.600206in}}%
\pgfpathlineto{\pgfqpoint{3.124888in}{0.600206in}}%
\pgfpathlineto{\pgfqpoint{3.124888in}{0.582810in}}%
\pgfpathlineto{\pgfqpoint{3.136374in}{0.582810in}}%
\pgfpathlineto{\pgfqpoint{3.136374in}{0.576354in}}%
\pgfpathlineto{\pgfqpoint{3.147861in}{0.576354in}}%
\pgfpathlineto{\pgfqpoint{3.147861in}{0.581375in}}%
\pgfpathlineto{\pgfqpoint{3.159347in}{0.581375in}}%
\pgfpathlineto{\pgfqpoint{3.159347in}{0.574919in}}%
\pgfpathlineto{\pgfqpoint{3.170834in}{0.574919in}}%
\pgfpathlineto{\pgfqpoint{3.170834in}{0.582093in}}%
\pgfpathlineto{\pgfqpoint{3.182320in}{0.582093in}}%
\pgfpathlineto{\pgfqpoint{3.182320in}{0.578506in}}%
\pgfpathlineto{\pgfqpoint{3.193807in}{0.578506in}}%
\pgfpathlineto{\pgfqpoint{3.193807in}{0.584962in}}%
\pgfpathlineto{\pgfqpoint{3.205293in}{0.584962in}}%
\pgfpathlineto{\pgfqpoint{3.205293in}{0.573305in}}%
\pgfpathlineto{\pgfqpoint{3.216779in}{0.573305in}}%
\pgfpathlineto{\pgfqpoint{3.216779in}{0.575278in}}%
\pgfpathlineto{\pgfqpoint{3.228266in}{0.575278in}}%
\pgfpathlineto{\pgfqpoint{3.228266in}{0.570615in}}%
\pgfpathlineto{\pgfqpoint{3.239752in}{0.570615in}}%
\pgfpathlineto{\pgfqpoint{3.239752in}{0.565594in}}%
\pgfpathlineto{\pgfqpoint{3.251239in}{0.565594in}}%
\pgfpathlineto{\pgfqpoint{3.251239in}{0.574740in}}%
\pgfpathlineto{\pgfqpoint{3.262725in}{0.574740in}}%
\pgfpathlineto{\pgfqpoint{3.262725in}{0.566311in}}%
\pgfpathlineto{\pgfqpoint{3.274211in}{0.566311in}}%
\pgfpathlineto{\pgfqpoint{3.274211in}{0.558420in}}%
\pgfpathlineto{\pgfqpoint{3.285698in}{0.558420in}}%
\pgfpathlineto{\pgfqpoint{3.285698in}{0.565235in}}%
\pgfpathlineto{\pgfqpoint{3.297184in}{0.565235in}}%
\pgfpathlineto{\pgfqpoint{3.297184in}{0.553220in}}%
\pgfpathlineto{\pgfqpoint{3.308671in}{0.553220in}}%
\pgfpathlineto{\pgfqpoint{3.308671in}{0.562724in}}%
\pgfpathlineto{\pgfqpoint{3.320157in}{0.562724in}}%
\pgfpathlineto{\pgfqpoint{3.320157in}{0.559138in}}%
\pgfpathlineto{\pgfqpoint{3.331643in}{0.559138in}}%
\pgfpathlineto{\pgfqpoint{3.331643in}{0.554834in}}%
\pgfpathlineto{\pgfqpoint{3.343130in}{0.554834in}}%
\pgfpathlineto{\pgfqpoint{3.343130in}{0.553758in}}%
\pgfpathlineto{\pgfqpoint{3.354616in}{0.553758in}}%
\pgfpathlineto{\pgfqpoint{3.354616in}{0.548916in}}%
\pgfpathlineto{\pgfqpoint{3.366103in}{0.548916in}}%
\pgfpathlineto{\pgfqpoint{3.366103in}{0.549812in}}%
\pgfpathlineto{\pgfqpoint{3.377589in}{0.549812in}}%
\pgfpathlineto{\pgfqpoint{3.377589in}{0.553937in}}%
\pgfpathlineto{\pgfqpoint{3.389075in}{0.553937in}}%
\pgfpathlineto{\pgfqpoint{3.389075in}{0.545150in}}%
\pgfpathlineto{\pgfqpoint{3.400562in}{0.545150in}}%
\pgfpathlineto{\pgfqpoint{3.400562in}{0.541025in}}%
\pgfpathlineto{\pgfqpoint{3.412048in}{0.541025in}}%
\pgfpathlineto{\pgfqpoint{3.412048in}{0.542280in}}%
\pgfpathlineto{\pgfqpoint{3.423535in}{0.542280in}}%
\pgfpathlineto{\pgfqpoint{3.423535in}{0.536541in}}%
\pgfpathlineto{\pgfqpoint{3.435021in}{0.536541in}}%
\pgfpathlineto{\pgfqpoint{3.435021in}{0.539411in}}%
\pgfpathlineto{\pgfqpoint{3.446507in}{0.539411in}}%
\pgfpathlineto{\pgfqpoint{3.446507in}{0.530444in}}%
\pgfpathlineto{\pgfqpoint{3.457994in}{0.530444in}}%
\pgfpathlineto{\pgfqpoint{3.457994in}{0.536721in}}%
\pgfpathlineto{\pgfqpoint{3.469480in}{0.536721in}}%
\pgfpathlineto{\pgfqpoint{3.469480in}{0.440955in}}%
\pgfpathlineto{\pgfqpoint{3.457994in}{0.440955in}}%
\pgfpathlineto{\pgfqpoint{3.457994in}{0.440955in}}%
\pgfpathlineto{\pgfqpoint{3.446507in}{0.440955in}}%
\pgfpathlineto{\pgfqpoint{3.446507in}{0.440955in}}%
\pgfpathlineto{\pgfqpoint{3.435021in}{0.440955in}}%
\pgfpathlineto{\pgfqpoint{3.435021in}{0.440955in}}%
\pgfpathlineto{\pgfqpoint{3.423535in}{0.440955in}}%
\pgfpathlineto{\pgfqpoint{3.423535in}{0.440955in}}%
\pgfpathlineto{\pgfqpoint{3.412048in}{0.440955in}}%
\pgfpathlineto{\pgfqpoint{3.412048in}{0.440955in}}%
\pgfpathlineto{\pgfqpoint{3.400562in}{0.440955in}}%
\pgfpathlineto{\pgfqpoint{3.400562in}{0.440955in}}%
\pgfpathlineto{\pgfqpoint{3.389075in}{0.440955in}}%
\pgfpathlineto{\pgfqpoint{3.389075in}{0.440955in}}%
\pgfpathlineto{\pgfqpoint{3.377589in}{0.440955in}}%
\pgfpathlineto{\pgfqpoint{3.377589in}{0.440955in}}%
\pgfpathlineto{\pgfqpoint{3.366103in}{0.440955in}}%
\pgfpathlineto{\pgfqpoint{3.366103in}{0.440955in}}%
\pgfpathlineto{\pgfqpoint{3.354616in}{0.440955in}}%
\pgfpathlineto{\pgfqpoint{3.354616in}{0.440955in}}%
\pgfpathlineto{\pgfqpoint{3.343130in}{0.440955in}}%
\pgfpathlineto{\pgfqpoint{3.343130in}{0.440955in}}%
\pgfpathlineto{\pgfqpoint{3.331643in}{0.440955in}}%
\pgfpathlineto{\pgfqpoint{3.331643in}{0.440955in}}%
\pgfpathlineto{\pgfqpoint{3.320157in}{0.440955in}}%
\pgfpathlineto{\pgfqpoint{3.320157in}{0.440955in}}%
\pgfpathlineto{\pgfqpoint{3.308671in}{0.440955in}}%
\pgfpathlineto{\pgfqpoint{3.308671in}{0.440955in}}%
\pgfpathlineto{\pgfqpoint{3.297184in}{0.440955in}}%
\pgfpathlineto{\pgfqpoint{3.297184in}{0.440955in}}%
\pgfpathlineto{\pgfqpoint{3.285698in}{0.440955in}}%
\pgfpathlineto{\pgfqpoint{3.285698in}{0.440955in}}%
\pgfpathlineto{\pgfqpoint{3.274211in}{0.440955in}}%
\pgfpathlineto{\pgfqpoint{3.274211in}{0.440955in}}%
\pgfpathlineto{\pgfqpoint{3.262725in}{0.440955in}}%
\pgfpathlineto{\pgfqpoint{3.262725in}{0.440955in}}%
\pgfpathlineto{\pgfqpoint{3.251239in}{0.440955in}}%
\pgfpathlineto{\pgfqpoint{3.251239in}{0.440955in}}%
\pgfpathlineto{\pgfqpoint{3.239752in}{0.440955in}}%
\pgfpathlineto{\pgfqpoint{3.239752in}{0.440955in}}%
\pgfpathlineto{\pgfqpoint{3.228266in}{0.440955in}}%
\pgfpathlineto{\pgfqpoint{3.228266in}{0.440955in}}%
\pgfpathlineto{\pgfqpoint{3.216779in}{0.440955in}}%
\pgfpathlineto{\pgfqpoint{3.216779in}{0.440955in}}%
\pgfpathlineto{\pgfqpoint{3.205293in}{0.440955in}}%
\pgfpathlineto{\pgfqpoint{3.205293in}{0.440955in}}%
\pgfpathlineto{\pgfqpoint{3.193807in}{0.440955in}}%
\pgfpathlineto{\pgfqpoint{3.193807in}{0.440955in}}%
\pgfpathlineto{\pgfqpoint{3.182320in}{0.440955in}}%
\pgfpathlineto{\pgfqpoint{3.182320in}{0.440955in}}%
\pgfpathlineto{\pgfqpoint{3.170834in}{0.440955in}}%
\pgfpathlineto{\pgfqpoint{3.170834in}{0.440955in}}%
\pgfpathlineto{\pgfqpoint{3.159347in}{0.440955in}}%
\pgfpathlineto{\pgfqpoint{3.159347in}{0.440955in}}%
\pgfpathlineto{\pgfqpoint{3.147861in}{0.440955in}}%
\pgfpathlineto{\pgfqpoint{3.147861in}{0.440955in}}%
\pgfpathlineto{\pgfqpoint{3.136374in}{0.440955in}}%
\pgfpathlineto{\pgfqpoint{3.136374in}{0.440955in}}%
\pgfpathlineto{\pgfqpoint{3.124888in}{0.440955in}}%
\pgfpathlineto{\pgfqpoint{3.124888in}{0.440955in}}%
\pgfpathlineto{\pgfqpoint{3.113402in}{0.440955in}}%
\pgfpathlineto{\pgfqpoint{3.113402in}{0.440955in}}%
\pgfpathlineto{\pgfqpoint{3.101915in}{0.440955in}}%
\pgfpathlineto{\pgfqpoint{3.101915in}{0.440955in}}%
\pgfpathlineto{\pgfqpoint{3.090429in}{0.440955in}}%
\pgfpathlineto{\pgfqpoint{3.090429in}{0.440955in}}%
\pgfpathlineto{\pgfqpoint{3.078942in}{0.440955in}}%
\pgfpathlineto{\pgfqpoint{3.078942in}{0.440955in}}%
\pgfpathlineto{\pgfqpoint{3.067456in}{0.440955in}}%
\pgfpathlineto{\pgfqpoint{3.067456in}{0.440955in}}%
\pgfpathlineto{\pgfqpoint{3.055970in}{0.440955in}}%
\pgfpathlineto{\pgfqpoint{3.055970in}{0.440955in}}%
\pgfpathlineto{\pgfqpoint{3.044483in}{0.440955in}}%
\pgfpathlineto{\pgfqpoint{3.044483in}{0.440955in}}%
\pgfpathlineto{\pgfqpoint{3.032997in}{0.440955in}}%
\pgfpathlineto{\pgfqpoint{3.032997in}{0.440955in}}%
\pgfpathlineto{\pgfqpoint{3.021510in}{0.440955in}}%
\pgfpathlineto{\pgfqpoint{3.021510in}{0.440955in}}%
\pgfpathlineto{\pgfqpoint{3.010024in}{0.440955in}}%
\pgfpathlineto{\pgfqpoint{3.010024in}{0.440955in}}%
\pgfpathlineto{\pgfqpoint{2.998538in}{0.440955in}}%
\pgfpathlineto{\pgfqpoint{2.998538in}{0.440955in}}%
\pgfpathlineto{\pgfqpoint{2.987051in}{0.440955in}}%
\pgfpathlineto{\pgfqpoint{2.987051in}{0.440955in}}%
\pgfpathlineto{\pgfqpoint{2.975565in}{0.440955in}}%
\pgfpathlineto{\pgfqpoint{2.975565in}{0.440955in}}%
\pgfpathlineto{\pgfqpoint{2.964078in}{0.440955in}}%
\pgfpathlineto{\pgfqpoint{2.964078in}{0.440955in}}%
\pgfpathlineto{\pgfqpoint{2.952592in}{0.440955in}}%
\pgfpathlineto{\pgfqpoint{2.952592in}{0.440955in}}%
\pgfpathlineto{\pgfqpoint{2.941106in}{0.440955in}}%
\pgfpathlineto{\pgfqpoint{2.941106in}{0.440955in}}%
\pgfpathlineto{\pgfqpoint{2.929619in}{0.440955in}}%
\pgfpathlineto{\pgfqpoint{2.929619in}{0.440955in}}%
\pgfpathlineto{\pgfqpoint{2.918133in}{0.440955in}}%
\pgfpathlineto{\pgfqpoint{2.918133in}{0.440955in}}%
\pgfpathlineto{\pgfqpoint{2.906646in}{0.440955in}}%
\pgfpathlineto{\pgfqpoint{2.906646in}{0.440955in}}%
\pgfpathlineto{\pgfqpoint{2.895160in}{0.440955in}}%
\pgfpathlineto{\pgfqpoint{2.895160in}{0.440955in}}%
\pgfpathlineto{\pgfqpoint{2.883674in}{0.440955in}}%
\pgfpathlineto{\pgfqpoint{2.883674in}{0.440955in}}%
\pgfpathlineto{\pgfqpoint{2.872187in}{0.440955in}}%
\pgfpathlineto{\pgfqpoint{2.872187in}{0.440955in}}%
\pgfpathlineto{\pgfqpoint{2.860701in}{0.440955in}}%
\pgfpathlineto{\pgfqpoint{2.860701in}{0.440955in}}%
\pgfpathlineto{\pgfqpoint{2.849214in}{0.440955in}}%
\pgfpathlineto{\pgfqpoint{2.849214in}{0.440955in}}%
\pgfpathlineto{\pgfqpoint{2.837728in}{0.440955in}}%
\pgfpathlineto{\pgfqpoint{2.837728in}{0.440955in}}%
\pgfpathlineto{\pgfqpoint{2.826242in}{0.440955in}}%
\pgfpathlineto{\pgfqpoint{2.826242in}{0.440955in}}%
\pgfpathlineto{\pgfqpoint{2.814755in}{0.440955in}}%
\pgfpathlineto{\pgfqpoint{2.814755in}{0.440955in}}%
\pgfpathlineto{\pgfqpoint{2.803269in}{0.440955in}}%
\pgfpathlineto{\pgfqpoint{2.803269in}{0.440955in}}%
\pgfpathlineto{\pgfqpoint{2.791782in}{0.440955in}}%
\pgfpathlineto{\pgfqpoint{2.791782in}{0.440955in}}%
\pgfpathlineto{\pgfqpoint{2.780296in}{0.440955in}}%
\pgfpathlineto{\pgfqpoint{2.780296in}{0.440955in}}%
\pgfpathlineto{\pgfqpoint{2.768810in}{0.440955in}}%
\pgfpathlineto{\pgfqpoint{2.768810in}{0.440955in}}%
\pgfpathlineto{\pgfqpoint{2.757323in}{0.440955in}}%
\pgfpathlineto{\pgfqpoint{2.757323in}{0.440955in}}%
\pgfpathlineto{\pgfqpoint{2.745837in}{0.440955in}}%
\pgfpathlineto{\pgfqpoint{2.745837in}{0.440955in}}%
\pgfpathlineto{\pgfqpoint{2.734350in}{0.440955in}}%
\pgfpathlineto{\pgfqpoint{2.734350in}{0.440955in}}%
\pgfpathlineto{\pgfqpoint{2.722864in}{0.440955in}}%
\pgfpathlineto{\pgfqpoint{2.722864in}{0.440955in}}%
\pgfpathlineto{\pgfqpoint{2.711377in}{0.440955in}}%
\pgfpathlineto{\pgfqpoint{2.711377in}{0.440955in}}%
\pgfpathlineto{\pgfqpoint{2.699891in}{0.440955in}}%
\pgfpathlineto{\pgfqpoint{2.699891in}{0.440955in}}%
\pgfpathlineto{\pgfqpoint{2.688405in}{0.440955in}}%
\pgfpathlineto{\pgfqpoint{2.688405in}{0.440955in}}%
\pgfpathlineto{\pgfqpoint{2.676918in}{0.440955in}}%
\pgfpathlineto{\pgfqpoint{2.676918in}{0.440955in}}%
\pgfpathlineto{\pgfqpoint{2.665432in}{0.440955in}}%
\pgfpathlineto{\pgfqpoint{2.665432in}{0.440955in}}%
\pgfpathlineto{\pgfqpoint{2.653945in}{0.440955in}}%
\pgfpathlineto{\pgfqpoint{2.653945in}{0.440955in}}%
\pgfpathlineto{\pgfqpoint{2.642459in}{0.440955in}}%
\pgfpathlineto{\pgfqpoint{2.642459in}{0.440955in}}%
\pgfpathlineto{\pgfqpoint{2.630973in}{0.440955in}}%
\pgfpathlineto{\pgfqpoint{2.630973in}{0.440955in}}%
\pgfpathlineto{\pgfqpoint{2.619486in}{0.440955in}}%
\pgfpathlineto{\pgfqpoint{2.619486in}{0.440955in}}%
\pgfpathlineto{\pgfqpoint{2.608000in}{0.440955in}}%
\pgfpathlineto{\pgfqpoint{2.608000in}{0.440955in}}%
\pgfpathlineto{\pgfqpoint{2.596513in}{0.440955in}}%
\pgfpathlineto{\pgfqpoint{2.596513in}{0.440955in}}%
\pgfpathlineto{\pgfqpoint{2.585027in}{0.440955in}}%
\pgfpathlineto{\pgfqpoint{2.585027in}{0.440955in}}%
\pgfpathlineto{\pgfqpoint{2.573541in}{0.440955in}}%
\pgfpathlineto{\pgfqpoint{2.573541in}{0.440955in}}%
\pgfpathlineto{\pgfqpoint{2.562054in}{0.440955in}}%
\pgfpathlineto{\pgfqpoint{2.562054in}{0.440955in}}%
\pgfpathlineto{\pgfqpoint{2.550568in}{0.440955in}}%
\pgfpathlineto{\pgfqpoint{2.550568in}{0.440955in}}%
\pgfpathlineto{\pgfqpoint{2.539081in}{0.440955in}}%
\pgfpathlineto{\pgfqpoint{2.539081in}{0.440955in}}%
\pgfpathlineto{\pgfqpoint{2.527595in}{0.440955in}}%
\pgfpathlineto{\pgfqpoint{2.527595in}{0.440955in}}%
\pgfpathlineto{\pgfqpoint{2.516109in}{0.440955in}}%
\pgfpathlineto{\pgfqpoint{2.516109in}{0.440955in}}%
\pgfpathlineto{\pgfqpoint{2.504622in}{0.440955in}}%
\pgfpathlineto{\pgfqpoint{2.504622in}{0.440955in}}%
\pgfpathlineto{\pgfqpoint{2.493136in}{0.440955in}}%
\pgfpathlineto{\pgfqpoint{2.493136in}{0.440955in}}%
\pgfpathlineto{\pgfqpoint{2.481649in}{0.440955in}}%
\pgfpathlineto{\pgfqpoint{2.481649in}{0.440955in}}%
\pgfpathlineto{\pgfqpoint{2.470163in}{0.440955in}}%
\pgfpathlineto{\pgfqpoint{2.470163in}{0.440955in}}%
\pgfpathlineto{\pgfqpoint{2.458677in}{0.440955in}}%
\pgfpathlineto{\pgfqpoint{2.458677in}{0.440955in}}%
\pgfpathlineto{\pgfqpoint{2.447190in}{0.440955in}}%
\pgfpathlineto{\pgfqpoint{2.447190in}{0.440955in}}%
\pgfpathlineto{\pgfqpoint{2.435704in}{0.440955in}}%
\pgfpathlineto{\pgfqpoint{2.435704in}{0.440955in}}%
\pgfpathlineto{\pgfqpoint{2.424217in}{0.440955in}}%
\pgfpathlineto{\pgfqpoint{2.424217in}{0.440955in}}%
\pgfpathlineto{\pgfqpoint{2.412731in}{0.440955in}}%
\pgfpathlineto{\pgfqpoint{2.412731in}{0.440955in}}%
\pgfpathlineto{\pgfqpoint{2.401245in}{0.440955in}}%
\pgfpathlineto{\pgfqpoint{2.401245in}{0.440955in}}%
\pgfpathlineto{\pgfqpoint{2.389758in}{0.440955in}}%
\pgfpathlineto{\pgfqpoint{2.389758in}{0.440955in}}%
\pgfpathlineto{\pgfqpoint{2.378272in}{0.440955in}}%
\pgfpathlineto{\pgfqpoint{2.378272in}{0.440955in}}%
\pgfpathlineto{\pgfqpoint{2.366785in}{0.440955in}}%
\pgfpathlineto{\pgfqpoint{2.366785in}{0.440955in}}%
\pgfpathlineto{\pgfqpoint{2.355299in}{0.440955in}}%
\pgfpathlineto{\pgfqpoint{2.355299in}{0.440955in}}%
\pgfpathlineto{\pgfqpoint{2.343812in}{0.440955in}}%
\pgfpathlineto{\pgfqpoint{2.343812in}{0.440955in}}%
\pgfpathlineto{\pgfqpoint{2.332326in}{0.440955in}}%
\pgfpathlineto{\pgfqpoint{2.332326in}{0.440955in}}%
\pgfpathlineto{\pgfqpoint{2.320840in}{0.440955in}}%
\pgfpathlineto{\pgfqpoint{2.320840in}{0.440955in}}%
\pgfpathlineto{\pgfqpoint{2.309353in}{0.440955in}}%
\pgfpathlineto{\pgfqpoint{2.309353in}{0.440955in}}%
\pgfpathlineto{\pgfqpoint{2.297867in}{0.440955in}}%
\pgfpathlineto{\pgfqpoint{2.297867in}{0.440955in}}%
\pgfpathlineto{\pgfqpoint{2.286380in}{0.440955in}}%
\pgfpathlineto{\pgfqpoint{2.286380in}{0.440955in}}%
\pgfpathlineto{\pgfqpoint{2.274894in}{0.440955in}}%
\pgfpathlineto{\pgfqpoint{2.274894in}{0.440955in}}%
\pgfpathlineto{\pgfqpoint{2.263408in}{0.440955in}}%
\pgfpathlineto{\pgfqpoint{2.263408in}{0.440955in}}%
\pgfpathlineto{\pgfqpoint{2.251921in}{0.440955in}}%
\pgfpathlineto{\pgfqpoint{2.251921in}{0.440955in}}%
\pgfpathlineto{\pgfqpoint{2.240435in}{0.440955in}}%
\pgfpathlineto{\pgfqpoint{2.240435in}{0.440955in}}%
\pgfpathlineto{\pgfqpoint{2.228948in}{0.440955in}}%
\pgfpathlineto{\pgfqpoint{2.228948in}{0.440955in}}%
\pgfpathlineto{\pgfqpoint{2.217462in}{0.440955in}}%
\pgfpathlineto{\pgfqpoint{2.217462in}{0.440955in}}%
\pgfpathlineto{\pgfqpoint{2.205976in}{0.440955in}}%
\pgfpathlineto{\pgfqpoint{2.205976in}{0.440955in}}%
\pgfpathlineto{\pgfqpoint{2.194489in}{0.440955in}}%
\pgfpathlineto{\pgfqpoint{2.194489in}{0.440955in}}%
\pgfpathlineto{\pgfqpoint{2.183003in}{0.440955in}}%
\pgfpathlineto{\pgfqpoint{2.183003in}{0.440955in}}%
\pgfpathlineto{\pgfqpoint{2.171516in}{0.440955in}}%
\pgfpathlineto{\pgfqpoint{2.171516in}{0.440955in}}%
\pgfpathlineto{\pgfqpoint{2.160030in}{0.440955in}}%
\pgfpathlineto{\pgfqpoint{2.160030in}{0.440955in}}%
\pgfpathlineto{\pgfqpoint{2.148544in}{0.440955in}}%
\pgfpathlineto{\pgfqpoint{2.148544in}{0.440955in}}%
\pgfpathlineto{\pgfqpoint{2.137057in}{0.440955in}}%
\pgfpathlineto{\pgfqpoint{2.137057in}{0.440955in}}%
\pgfpathlineto{\pgfqpoint{2.125571in}{0.440955in}}%
\pgfpathlineto{\pgfqpoint{2.125571in}{0.440955in}}%
\pgfpathlineto{\pgfqpoint{2.114084in}{0.440955in}}%
\pgfpathlineto{\pgfqpoint{2.114084in}{0.440955in}}%
\pgfpathlineto{\pgfqpoint{2.102598in}{0.440955in}}%
\pgfpathlineto{\pgfqpoint{2.102598in}{0.440955in}}%
\pgfpathlineto{\pgfqpoint{2.091112in}{0.440955in}}%
\pgfpathlineto{\pgfqpoint{2.091112in}{0.440955in}}%
\pgfpathlineto{\pgfqpoint{2.079625in}{0.440955in}}%
\pgfpathlineto{\pgfqpoint{2.079625in}{0.440955in}}%
\pgfpathlineto{\pgfqpoint{2.068139in}{0.440955in}}%
\pgfpathlineto{\pgfqpoint{2.068139in}{0.440955in}}%
\pgfpathlineto{\pgfqpoint{2.056652in}{0.440955in}}%
\pgfpathlineto{\pgfqpoint{2.056652in}{0.440955in}}%
\pgfpathlineto{\pgfqpoint{2.045166in}{0.440955in}}%
\pgfpathlineto{\pgfqpoint{2.045166in}{0.440955in}}%
\pgfpathlineto{\pgfqpoint{2.033680in}{0.440955in}}%
\pgfpathlineto{\pgfqpoint{2.033680in}{0.440955in}}%
\pgfpathlineto{\pgfqpoint{2.022193in}{0.440955in}}%
\pgfpathlineto{\pgfqpoint{2.022193in}{0.440955in}}%
\pgfpathlineto{\pgfqpoint{2.010707in}{0.440955in}}%
\pgfpathlineto{\pgfqpoint{2.010707in}{0.440955in}}%
\pgfpathlineto{\pgfqpoint{1.999220in}{0.440955in}}%
\pgfpathlineto{\pgfqpoint{1.999220in}{0.440955in}}%
\pgfpathlineto{\pgfqpoint{1.987734in}{0.440955in}}%
\pgfpathlineto{\pgfqpoint{1.987734in}{0.440955in}}%
\pgfpathlineto{\pgfqpoint{1.976247in}{0.440955in}}%
\pgfpathlineto{\pgfqpoint{1.976247in}{0.440955in}}%
\pgfpathlineto{\pgfqpoint{1.964761in}{0.440955in}}%
\pgfpathlineto{\pgfqpoint{1.964761in}{0.440955in}}%
\pgfpathlineto{\pgfqpoint{1.953275in}{0.440955in}}%
\pgfpathlineto{\pgfqpoint{1.953275in}{0.440955in}}%
\pgfpathlineto{\pgfqpoint{1.941788in}{0.440955in}}%
\pgfpathlineto{\pgfqpoint{1.941788in}{0.440955in}}%
\pgfpathlineto{\pgfqpoint{1.930302in}{0.440955in}}%
\pgfpathlineto{\pgfqpoint{1.930302in}{0.440955in}}%
\pgfpathlineto{\pgfqpoint{1.918815in}{0.440955in}}%
\pgfpathlineto{\pgfqpoint{1.918815in}{0.440955in}}%
\pgfpathlineto{\pgfqpoint{1.907329in}{0.440955in}}%
\pgfpathlineto{\pgfqpoint{1.907329in}{0.440955in}}%
\pgfpathlineto{\pgfqpoint{1.895843in}{0.440955in}}%
\pgfpathlineto{\pgfqpoint{1.895843in}{0.440955in}}%
\pgfpathlineto{\pgfqpoint{1.884356in}{0.440955in}}%
\pgfpathlineto{\pgfqpoint{1.884356in}{0.440955in}}%
\pgfpathlineto{\pgfqpoint{1.872870in}{0.440955in}}%
\pgfpathlineto{\pgfqpoint{1.872870in}{0.440955in}}%
\pgfpathlineto{\pgfqpoint{1.861383in}{0.440955in}}%
\pgfpathlineto{\pgfqpoint{1.861383in}{0.440955in}}%
\pgfpathlineto{\pgfqpoint{1.849897in}{0.440955in}}%
\pgfpathlineto{\pgfqpoint{1.849897in}{0.440955in}}%
\pgfpathlineto{\pgfqpoint{1.838411in}{0.440955in}}%
\pgfpathlineto{\pgfqpoint{1.838411in}{0.440955in}}%
\pgfpathlineto{\pgfqpoint{1.826924in}{0.440955in}}%
\pgfpathlineto{\pgfqpoint{1.826924in}{0.440955in}}%
\pgfpathlineto{\pgfqpoint{1.815438in}{0.440955in}}%
\pgfpathlineto{\pgfqpoint{1.815438in}{0.440955in}}%
\pgfpathlineto{\pgfqpoint{1.803951in}{0.440955in}}%
\pgfpathlineto{\pgfqpoint{1.803951in}{0.440955in}}%
\pgfpathlineto{\pgfqpoint{1.792465in}{0.440955in}}%
\pgfpathlineto{\pgfqpoint{1.792465in}{0.440955in}}%
\pgfpathlineto{\pgfqpoint{1.780979in}{0.440955in}}%
\pgfpathlineto{\pgfqpoint{1.780979in}{0.440955in}}%
\pgfpathlineto{\pgfqpoint{1.769492in}{0.440955in}}%
\pgfpathlineto{\pgfqpoint{1.769492in}{0.440955in}}%
\pgfpathlineto{\pgfqpoint{1.758006in}{0.440955in}}%
\pgfpathlineto{\pgfqpoint{1.758006in}{0.440955in}}%
\pgfpathlineto{\pgfqpoint{1.746519in}{0.440955in}}%
\pgfpathlineto{\pgfqpoint{1.746519in}{0.440955in}}%
\pgfpathlineto{\pgfqpoint{1.735033in}{0.440955in}}%
\pgfpathlineto{\pgfqpoint{1.735033in}{0.440955in}}%
\pgfpathlineto{\pgfqpoint{1.723547in}{0.440955in}}%
\pgfpathlineto{\pgfqpoint{1.723547in}{0.440955in}}%
\pgfpathlineto{\pgfqpoint{1.712060in}{0.440955in}}%
\pgfpathlineto{\pgfqpoint{1.712060in}{0.440955in}}%
\pgfpathlineto{\pgfqpoint{1.700574in}{0.440955in}}%
\pgfpathlineto{\pgfqpoint{1.700574in}{0.440955in}}%
\pgfpathlineto{\pgfqpoint{1.689087in}{0.440955in}}%
\pgfpathlineto{\pgfqpoint{1.689087in}{0.440955in}}%
\pgfpathlineto{\pgfqpoint{1.677601in}{0.440955in}}%
\pgfpathlineto{\pgfqpoint{1.677601in}{0.440955in}}%
\pgfpathlineto{\pgfqpoint{1.666115in}{0.440955in}}%
\pgfpathlineto{\pgfqpoint{1.666115in}{0.440955in}}%
\pgfpathlineto{\pgfqpoint{1.654628in}{0.440955in}}%
\pgfpathlineto{\pgfqpoint{1.654628in}{0.440955in}}%
\pgfpathlineto{\pgfqpoint{1.643142in}{0.440955in}}%
\pgfpathlineto{\pgfqpoint{1.643142in}{0.440955in}}%
\pgfpathlineto{\pgfqpoint{1.631655in}{0.440955in}}%
\pgfpathlineto{\pgfqpoint{1.631655in}{0.440955in}}%
\pgfpathlineto{\pgfqpoint{1.620169in}{0.440955in}}%
\pgfpathlineto{\pgfqpoint{1.620169in}{0.440955in}}%
\pgfpathlineto{\pgfqpoint{1.608682in}{0.440955in}}%
\pgfpathlineto{\pgfqpoint{1.608682in}{0.440955in}}%
\pgfpathlineto{\pgfqpoint{1.597196in}{0.440955in}}%
\pgfpathlineto{\pgfqpoint{1.597196in}{0.440955in}}%
\pgfpathlineto{\pgfqpoint{1.585710in}{0.440955in}}%
\pgfpathlineto{\pgfqpoint{1.585710in}{0.440955in}}%
\pgfpathlineto{\pgfqpoint{1.574223in}{0.440955in}}%
\pgfpathlineto{\pgfqpoint{1.574223in}{0.440955in}}%
\pgfpathlineto{\pgfqpoint{1.562737in}{0.440955in}}%
\pgfpathlineto{\pgfqpoint{1.562737in}{0.440955in}}%
\pgfpathlineto{\pgfqpoint{1.551250in}{0.440955in}}%
\pgfpathlineto{\pgfqpoint{1.551250in}{0.440955in}}%
\pgfpathlineto{\pgfqpoint{1.539764in}{0.440955in}}%
\pgfpathlineto{\pgfqpoint{1.539764in}{0.440955in}}%
\pgfpathlineto{\pgfqpoint{1.528278in}{0.440955in}}%
\pgfpathlineto{\pgfqpoint{1.528278in}{0.440955in}}%
\pgfpathlineto{\pgfqpoint{1.516791in}{0.440955in}}%
\pgfpathlineto{\pgfqpoint{1.516791in}{0.440955in}}%
\pgfpathlineto{\pgfqpoint{1.505305in}{0.440955in}}%
\pgfpathlineto{\pgfqpoint{1.505305in}{0.440955in}}%
\pgfpathlineto{\pgfqpoint{1.493818in}{0.440955in}}%
\pgfpathlineto{\pgfqpoint{1.493818in}{0.440955in}}%
\pgfpathlineto{\pgfqpoint{1.482332in}{0.440955in}}%
\pgfpathlineto{\pgfqpoint{1.482332in}{0.440955in}}%
\pgfpathlineto{\pgfqpoint{1.470846in}{0.440955in}}%
\pgfpathlineto{\pgfqpoint{1.470846in}{0.440955in}}%
\pgfpathlineto{\pgfqpoint{1.459359in}{0.440955in}}%
\pgfpathlineto{\pgfqpoint{1.459359in}{0.440955in}}%
\pgfpathlineto{\pgfqpoint{1.447873in}{0.440955in}}%
\pgfpathlineto{\pgfqpoint{1.447873in}{0.440955in}}%
\pgfpathlineto{\pgfqpoint{1.436386in}{0.440955in}}%
\pgfpathlineto{\pgfqpoint{1.436386in}{0.440955in}}%
\pgfpathlineto{\pgfqpoint{1.424900in}{0.440955in}}%
\pgfpathlineto{\pgfqpoint{1.424900in}{0.440955in}}%
\pgfpathlineto{\pgfqpoint{1.413414in}{0.440955in}}%
\pgfpathlineto{\pgfqpoint{1.413414in}{0.440955in}}%
\pgfpathlineto{\pgfqpoint{1.401927in}{0.440955in}}%
\pgfpathlineto{\pgfqpoint{1.401927in}{0.440955in}}%
\pgfpathlineto{\pgfqpoint{1.390441in}{0.440955in}}%
\pgfpathlineto{\pgfqpoint{1.390441in}{0.440955in}}%
\pgfpathlineto{\pgfqpoint{1.378954in}{0.440955in}}%
\pgfpathlineto{\pgfqpoint{1.378954in}{0.440955in}}%
\pgfpathlineto{\pgfqpoint{1.367468in}{0.440955in}}%
\pgfpathlineto{\pgfqpoint{1.367468in}{0.440955in}}%
\pgfpathlineto{\pgfqpoint{1.355982in}{0.440955in}}%
\pgfpathlineto{\pgfqpoint{1.355982in}{0.440955in}}%
\pgfpathlineto{\pgfqpoint{1.344495in}{0.440955in}}%
\pgfpathlineto{\pgfqpoint{1.344495in}{0.440955in}}%
\pgfpathlineto{\pgfqpoint{1.333009in}{0.440955in}}%
\pgfpathlineto{\pgfqpoint{1.333009in}{0.440955in}}%
\pgfpathlineto{\pgfqpoint{1.321522in}{0.440955in}}%
\pgfpathlineto{\pgfqpoint{1.321522in}{0.440955in}}%
\pgfpathlineto{\pgfqpoint{1.310036in}{0.440955in}}%
\pgfpathlineto{\pgfqpoint{1.310036in}{0.440955in}}%
\pgfpathlineto{\pgfqpoint{1.298550in}{0.440955in}}%
\pgfpathlineto{\pgfqpoint{1.298550in}{0.440955in}}%
\pgfpathlineto{\pgfqpoint{1.287063in}{0.440955in}}%
\pgfpathlineto{\pgfqpoint{1.287063in}{0.440955in}}%
\pgfpathlineto{\pgfqpoint{1.275577in}{0.440955in}}%
\pgfpathlineto{\pgfqpoint{1.275577in}{0.440955in}}%
\pgfpathlineto{\pgfqpoint{1.264090in}{0.440955in}}%
\pgfpathlineto{\pgfqpoint{1.264090in}{0.440955in}}%
\pgfpathlineto{\pgfqpoint{1.252604in}{0.440955in}}%
\pgfpathlineto{\pgfqpoint{1.252604in}{0.440955in}}%
\pgfpathlineto{\pgfqpoint{1.241117in}{0.440955in}}%
\pgfpathlineto{\pgfqpoint{1.241117in}{0.440955in}}%
\pgfpathlineto{\pgfqpoint{1.229631in}{0.440955in}}%
\pgfpathlineto{\pgfqpoint{1.229631in}{0.440955in}}%
\pgfpathlineto{\pgfqpoint{1.218145in}{0.440955in}}%
\pgfpathlineto{\pgfqpoint{1.218145in}{0.440955in}}%
\pgfpathlineto{\pgfqpoint{1.206658in}{0.440955in}}%
\pgfpathlineto{\pgfqpoint{1.206658in}{0.440955in}}%
\pgfpathlineto{\pgfqpoint{1.195172in}{0.440955in}}%
\pgfpathlineto{\pgfqpoint{1.195172in}{0.440955in}}%
\pgfpathlineto{\pgfqpoint{1.183685in}{0.440955in}}%
\pgfpathlineto{\pgfqpoint{1.183685in}{0.440955in}}%
\pgfpathlineto{\pgfqpoint{1.172199in}{0.440955in}}%
\pgfpathlineto{\pgfqpoint{1.172199in}{0.440955in}}%
\pgfpathlineto{\pgfqpoint{1.160713in}{0.440955in}}%
\pgfpathlineto{\pgfqpoint{1.160713in}{0.440955in}}%
\pgfpathlineto{\pgfqpoint{1.149226in}{0.440955in}}%
\pgfpathlineto{\pgfqpoint{1.149226in}{0.440955in}}%
\pgfpathlineto{\pgfqpoint{1.137740in}{0.440955in}}%
\pgfpathlineto{\pgfqpoint{1.137740in}{0.440955in}}%
\pgfpathlineto{\pgfqpoint{1.126253in}{0.440955in}}%
\pgfpathlineto{\pgfqpoint{1.126253in}{0.440955in}}%
\pgfpathlineto{\pgfqpoint{1.114767in}{0.440955in}}%
\pgfpathlineto{\pgfqpoint{1.114767in}{0.440955in}}%
\pgfpathlineto{\pgfqpoint{1.103281in}{0.440955in}}%
\pgfpathlineto{\pgfqpoint{1.103281in}{0.440955in}}%
\pgfpathlineto{\pgfqpoint{1.091794in}{0.440955in}}%
\pgfpathlineto{\pgfqpoint{1.091794in}{0.440955in}}%
\pgfpathlineto{\pgfqpoint{1.080308in}{0.440955in}}%
\pgfpathlineto{\pgfqpoint{1.080308in}{0.440955in}}%
\pgfpathlineto{\pgfqpoint{1.068821in}{0.440955in}}%
\pgfpathlineto{\pgfqpoint{1.068821in}{0.440955in}}%
\pgfpathlineto{\pgfqpoint{1.057335in}{0.440955in}}%
\pgfpathlineto{\pgfqpoint{1.057335in}{0.440955in}}%
\pgfpathlineto{\pgfqpoint{1.045849in}{0.440955in}}%
\pgfpathlineto{\pgfqpoint{1.045849in}{0.440955in}}%
\pgfpathlineto{\pgfqpoint{1.034362in}{0.440955in}}%
\pgfpathlineto{\pgfqpoint{1.034362in}{0.440955in}}%
\pgfpathlineto{\pgfqpoint{1.022876in}{0.440955in}}%
\pgfpathlineto{\pgfqpoint{1.022876in}{0.440955in}}%
\pgfpathlineto{\pgfqpoint{1.011389in}{0.440955in}}%
\pgfpathlineto{\pgfqpoint{1.011389in}{0.440955in}}%
\pgfpathlineto{\pgfqpoint{0.999903in}{0.440955in}}%
\pgfpathlineto{\pgfqpoint{0.999903in}{0.440955in}}%
\pgfpathlineto{\pgfqpoint{0.988417in}{0.440955in}}%
\pgfpathlineto{\pgfqpoint{0.988417in}{0.440955in}}%
\pgfpathlineto{\pgfqpoint{0.976930in}{0.440955in}}%
\pgfpathlineto{\pgfqpoint{0.976930in}{0.440955in}}%
\pgfpathlineto{\pgfqpoint{0.965444in}{0.440955in}}%
\pgfpathlineto{\pgfqpoint{0.965444in}{0.440955in}}%
\pgfpathlineto{\pgfqpoint{0.953957in}{0.440955in}}%
\pgfpathlineto{\pgfqpoint{0.953957in}{0.440955in}}%
\pgfpathlineto{\pgfqpoint{0.942471in}{0.440955in}}%
\pgfpathlineto{\pgfqpoint{0.942471in}{0.440955in}}%
\pgfpathlineto{\pgfqpoint{0.930985in}{0.440955in}}%
\pgfpathlineto{\pgfqpoint{0.930985in}{0.440955in}}%
\pgfpathlineto{\pgfqpoint{0.919498in}{0.440955in}}%
\pgfpathlineto{\pgfqpoint{0.919498in}{0.440955in}}%
\pgfpathlineto{\pgfqpoint{0.908012in}{0.440955in}}%
\pgfpathlineto{\pgfqpoint{0.908012in}{0.440955in}}%
\pgfpathlineto{\pgfqpoint{0.896525in}{0.440955in}}%
\pgfpathlineto{\pgfqpoint{0.896525in}{0.440955in}}%
\pgfpathlineto{\pgfqpoint{0.885039in}{0.440955in}}%
\pgfpathlineto{\pgfqpoint{0.885039in}{0.440955in}}%
\pgfpathlineto{\pgfqpoint{0.873552in}{0.440955in}}%
\pgfpathlineto{\pgfqpoint{0.873552in}{0.440955in}}%
\pgfpathlineto{\pgfqpoint{0.862066in}{0.440955in}}%
\pgfpathlineto{\pgfqpoint{0.862066in}{0.440955in}}%
\pgfpathlineto{\pgfqpoint{0.850580in}{0.440955in}}%
\pgfpathlineto{\pgfqpoint{0.850580in}{0.440955in}}%
\pgfpathlineto{\pgfqpoint{0.839093in}{0.440955in}}%
\pgfpathlineto{\pgfqpoint{0.839093in}{0.440955in}}%
\pgfpathlineto{\pgfqpoint{0.827607in}{0.440955in}}%
\pgfpathlineto{\pgfqpoint{0.827607in}{0.440955in}}%
\pgfpathlineto{\pgfqpoint{0.816120in}{0.440955in}}%
\pgfpathlineto{\pgfqpoint{0.816120in}{0.440955in}}%
\pgfpathlineto{\pgfqpoint{0.804634in}{0.440955in}}%
\pgfpathlineto{\pgfqpoint{0.804634in}{0.440955in}}%
\pgfpathlineto{\pgfqpoint{0.793148in}{0.440955in}}%
\pgfpathlineto{\pgfqpoint{0.793148in}{0.440955in}}%
\pgfpathlineto{\pgfqpoint{0.781661in}{0.440955in}}%
\pgfpathlineto{\pgfqpoint{0.781661in}{0.440955in}}%
\pgfpathlineto{\pgfqpoint{0.770175in}{0.440955in}}%
\pgfpathlineto{\pgfqpoint{0.770175in}{0.440955in}}%
\pgfpathlineto{\pgfqpoint{0.758688in}{0.440955in}}%
\pgfpathlineto{\pgfqpoint{0.758688in}{0.440955in}}%
\pgfpathlineto{\pgfqpoint{0.747202in}{0.440955in}}%
\pgfpathlineto{\pgfqpoint{0.747202in}{0.440955in}}%
\pgfpathlineto{\pgfqpoint{0.735716in}{0.440955in}}%
\pgfpathlineto{\pgfqpoint{0.735716in}{0.440955in}}%
\pgfpathlineto{\pgfqpoint{0.724229in}{0.440955in}}%
\pgfpathlineto{\pgfqpoint{0.724229in}{0.440955in}}%
\pgfpathlineto{\pgfqpoint{0.712743in}{0.440955in}}%
\pgfpathlineto{\pgfqpoint{0.712743in}{0.440955in}}%
\pgfpathlineto{\pgfqpoint{0.701256in}{0.440955in}}%
\pgfpathlineto{\pgfqpoint{0.701256in}{0.440955in}}%
\pgfpathlineto{\pgfqpoint{0.689770in}{0.440955in}}%
\pgfpathlineto{\pgfqpoint{0.689770in}{0.440955in}}%
\pgfpathlineto{\pgfqpoint{0.678284in}{0.440955in}}%
\pgfpathlineto{\pgfqpoint{0.678284in}{0.440955in}}%
\pgfpathlineto{\pgfqpoint{0.666797in}{0.440955in}}%
\pgfpathlineto{\pgfqpoint{0.666797in}{0.440955in}}%
\pgfpathlineto{\pgfqpoint{0.655311in}{0.440955in}}%
\pgfpathlineto{\pgfqpoint{0.655311in}{0.440955in}}%
\pgfpathlineto{\pgfqpoint{0.643824in}{0.440955in}}%
\pgfpathlineto{\pgfqpoint{0.643824in}{0.440955in}}%
\pgfpathlineto{\pgfqpoint{0.632338in}{0.440955in}}%
\pgfpathlineto{\pgfqpoint{0.632338in}{0.440955in}}%
\pgfpathlineto{\pgfqpoint{0.620852in}{0.440955in}}%
\pgfpathlineto{\pgfqpoint{0.620852in}{0.440955in}}%
\pgfpathlineto{\pgfqpoint{0.609365in}{0.440955in}}%
\pgfpathlineto{\pgfqpoint{0.609365in}{0.440955in}}%
\pgfpathlineto{\pgfqpoint{0.597879in}{0.440955in}}%
\pgfusepath{fill}%
\end{pgfscope}%
\begin{pgfscope}%
\pgfsetrectcap%
\pgfsetmiterjoin%
\pgfsetlinewidth{1.003750pt}%
\definecolor{currentstroke}{rgb}{0.000000,0.000000,0.000000}%
\pgfsetstrokecolor{currentstroke}%
\pgfsetdash{}{0pt}%
\pgfpathmoveto{\pgfqpoint{0.597879in}{2.054978in}}%
\pgfpathlineto{\pgfqpoint{3.469480in}{2.054978in}}%
\pgfusepath{stroke}%
\end{pgfscope}%
\begin{pgfscope}%
\pgfsetrectcap%
\pgfsetmiterjoin%
\pgfsetlinewidth{1.003750pt}%
\definecolor{currentstroke}{rgb}{0.000000,0.000000,0.000000}%
\pgfsetstrokecolor{currentstroke}%
\pgfsetdash{}{0pt}%
\pgfpathmoveto{\pgfqpoint{3.469480in}{0.440955in}}%
\pgfpathlineto{\pgfqpoint{3.469480in}{2.054978in}}%
\pgfusepath{stroke}%
\end{pgfscope}%
\begin{pgfscope}%
\pgfsetrectcap%
\pgfsetmiterjoin%
\pgfsetlinewidth{1.003750pt}%
\definecolor{currentstroke}{rgb}{0.000000,0.000000,0.000000}%
\pgfsetstrokecolor{currentstroke}%
\pgfsetdash{}{0pt}%
\pgfpathmoveto{\pgfqpoint{0.597879in}{0.440955in}}%
\pgfpathlineto{\pgfqpoint{3.469480in}{0.440955in}}%
\pgfusepath{stroke}%
\end{pgfscope}%
\begin{pgfscope}%
\pgfsetrectcap%
\pgfsetmiterjoin%
\pgfsetlinewidth{1.003750pt}%
\definecolor{currentstroke}{rgb}{0.000000,0.000000,0.000000}%
\pgfsetstrokecolor{currentstroke}%
\pgfsetdash{}{0pt}%
\pgfpathmoveto{\pgfqpoint{0.597879in}{0.440955in}}%
\pgfpathlineto{\pgfqpoint{0.597879in}{2.054978in}}%
\pgfusepath{stroke}%
\end{pgfscope}%
\begin{pgfscope}%
\pgfsetbuttcap%
\pgfsetroundjoin%
\definecolor{currentfill}{rgb}{0.000000,0.000000,0.000000}%
\pgfsetfillcolor{currentfill}%
\pgfsetlinewidth{0.501875pt}%
\definecolor{currentstroke}{rgb}{0.000000,0.000000,0.000000}%
\pgfsetstrokecolor{currentstroke}%
\pgfsetdash{}{0pt}%
\pgfsys@defobject{currentmarker}{\pgfqpoint{0.000000in}{0.000000in}}{\pgfqpoint{0.000000in}{0.069444in}}{%
\pgfpathmoveto{\pgfqpoint{0.000000in}{0.000000in}}%
\pgfpathlineto{\pgfqpoint{0.000000in}{0.069444in}}%
\pgfusepath{stroke,fill}%
}%
\begin{pgfscope}%
\pgfsys@transformshift{1.076479in}{0.440955in}%
\pgfsys@useobject{currentmarker}{}%
\end{pgfscope}%
\end{pgfscope}%
\begin{pgfscope}%
\pgfsetbuttcap%
\pgfsetroundjoin%
\definecolor{currentfill}{rgb}{0.000000,0.000000,0.000000}%
\pgfsetfillcolor{currentfill}%
\pgfsetlinewidth{0.501875pt}%
\definecolor{currentstroke}{rgb}{0.000000,0.000000,0.000000}%
\pgfsetstrokecolor{currentstroke}%
\pgfsetdash{}{0pt}%
\pgfsys@defobject{currentmarker}{\pgfqpoint{0.000000in}{-0.069444in}}{\pgfqpoint{0.000000in}{0.000000in}}{%
\pgfpathmoveto{\pgfqpoint{0.000000in}{0.000000in}}%
\pgfpathlineto{\pgfqpoint{0.000000in}{-0.069444in}}%
\pgfusepath{stroke,fill}%
}%
\begin{pgfscope}%
\pgfsys@transformshift{1.076479in}{2.054978in}%
\pgfsys@useobject{currentmarker}{}%
\end{pgfscope}%
\end{pgfscope}%
\begin{pgfscope}%
\pgftext[x=1.076479in,y=0.371511in,,top]{\rmfamily\fontsize{8.000000}{9.600000}\selectfont 5000}%
\end{pgfscope}%
\begin{pgfscope}%
\pgfsetbuttcap%
\pgfsetroundjoin%
\definecolor{currentfill}{rgb}{0.000000,0.000000,0.000000}%
\pgfsetfillcolor{currentfill}%
\pgfsetlinewidth{0.501875pt}%
\definecolor{currentstroke}{rgb}{0.000000,0.000000,0.000000}%
\pgfsetstrokecolor{currentstroke}%
\pgfsetdash{}{0pt}%
\pgfsys@defobject{currentmarker}{\pgfqpoint{0.000000in}{0.000000in}}{\pgfqpoint{0.000000in}{0.069444in}}{%
\pgfpathmoveto{\pgfqpoint{0.000000in}{0.000000in}}%
\pgfpathlineto{\pgfqpoint{0.000000in}{0.069444in}}%
\pgfusepath{stroke,fill}%
}%
\begin{pgfscope}%
\pgfsys@transformshift{1.674729in}{0.440955in}%
\pgfsys@useobject{currentmarker}{}%
\end{pgfscope}%
\end{pgfscope}%
\begin{pgfscope}%
\pgfsetbuttcap%
\pgfsetroundjoin%
\definecolor{currentfill}{rgb}{0.000000,0.000000,0.000000}%
\pgfsetfillcolor{currentfill}%
\pgfsetlinewidth{0.501875pt}%
\definecolor{currentstroke}{rgb}{0.000000,0.000000,0.000000}%
\pgfsetstrokecolor{currentstroke}%
\pgfsetdash{}{0pt}%
\pgfsys@defobject{currentmarker}{\pgfqpoint{0.000000in}{-0.069444in}}{\pgfqpoint{0.000000in}{0.000000in}}{%
\pgfpathmoveto{\pgfqpoint{0.000000in}{0.000000in}}%
\pgfpathlineto{\pgfqpoint{0.000000in}{-0.069444in}}%
\pgfusepath{stroke,fill}%
}%
\begin{pgfscope}%
\pgfsys@transformshift{1.674729in}{2.054978in}%
\pgfsys@useobject{currentmarker}{}%
\end{pgfscope}%
\end{pgfscope}%
\begin{pgfscope}%
\pgftext[x=1.674729in,y=0.371511in,,top]{\rmfamily\fontsize{8.000000}{9.600000}\selectfont 5500}%
\end{pgfscope}%
\begin{pgfscope}%
\pgfsetbuttcap%
\pgfsetroundjoin%
\definecolor{currentfill}{rgb}{0.000000,0.000000,0.000000}%
\pgfsetfillcolor{currentfill}%
\pgfsetlinewidth{0.501875pt}%
\definecolor{currentstroke}{rgb}{0.000000,0.000000,0.000000}%
\pgfsetstrokecolor{currentstroke}%
\pgfsetdash{}{0pt}%
\pgfsys@defobject{currentmarker}{\pgfqpoint{0.000000in}{0.000000in}}{\pgfqpoint{0.000000in}{0.069444in}}{%
\pgfpathmoveto{\pgfqpoint{0.000000in}{0.000000in}}%
\pgfpathlineto{\pgfqpoint{0.000000in}{0.069444in}}%
\pgfusepath{stroke,fill}%
}%
\begin{pgfscope}%
\pgfsys@transformshift{2.272980in}{0.440955in}%
\pgfsys@useobject{currentmarker}{}%
\end{pgfscope}%
\end{pgfscope}%
\begin{pgfscope}%
\pgfsetbuttcap%
\pgfsetroundjoin%
\definecolor{currentfill}{rgb}{0.000000,0.000000,0.000000}%
\pgfsetfillcolor{currentfill}%
\pgfsetlinewidth{0.501875pt}%
\definecolor{currentstroke}{rgb}{0.000000,0.000000,0.000000}%
\pgfsetstrokecolor{currentstroke}%
\pgfsetdash{}{0pt}%
\pgfsys@defobject{currentmarker}{\pgfqpoint{0.000000in}{-0.069444in}}{\pgfqpoint{0.000000in}{0.000000in}}{%
\pgfpathmoveto{\pgfqpoint{0.000000in}{0.000000in}}%
\pgfpathlineto{\pgfqpoint{0.000000in}{-0.069444in}}%
\pgfusepath{stroke,fill}%
}%
\begin{pgfscope}%
\pgfsys@transformshift{2.272980in}{2.054978in}%
\pgfsys@useobject{currentmarker}{}%
\end{pgfscope}%
\end{pgfscope}%
\begin{pgfscope}%
\pgftext[x=2.272980in,y=0.371511in,,top]{\rmfamily\fontsize{8.000000}{9.600000}\selectfont 6000}%
\end{pgfscope}%
\begin{pgfscope}%
\pgfsetbuttcap%
\pgfsetroundjoin%
\definecolor{currentfill}{rgb}{0.000000,0.000000,0.000000}%
\pgfsetfillcolor{currentfill}%
\pgfsetlinewidth{0.501875pt}%
\definecolor{currentstroke}{rgb}{0.000000,0.000000,0.000000}%
\pgfsetstrokecolor{currentstroke}%
\pgfsetdash{}{0pt}%
\pgfsys@defobject{currentmarker}{\pgfqpoint{0.000000in}{0.000000in}}{\pgfqpoint{0.000000in}{0.069444in}}{%
\pgfpathmoveto{\pgfqpoint{0.000000in}{0.000000in}}%
\pgfpathlineto{\pgfqpoint{0.000000in}{0.069444in}}%
\pgfusepath{stroke,fill}%
}%
\begin{pgfscope}%
\pgfsys@transformshift{2.871230in}{0.440955in}%
\pgfsys@useobject{currentmarker}{}%
\end{pgfscope}%
\end{pgfscope}%
\begin{pgfscope}%
\pgfsetbuttcap%
\pgfsetroundjoin%
\definecolor{currentfill}{rgb}{0.000000,0.000000,0.000000}%
\pgfsetfillcolor{currentfill}%
\pgfsetlinewidth{0.501875pt}%
\definecolor{currentstroke}{rgb}{0.000000,0.000000,0.000000}%
\pgfsetstrokecolor{currentstroke}%
\pgfsetdash{}{0pt}%
\pgfsys@defobject{currentmarker}{\pgfqpoint{0.000000in}{-0.069444in}}{\pgfqpoint{0.000000in}{0.000000in}}{%
\pgfpathmoveto{\pgfqpoint{0.000000in}{0.000000in}}%
\pgfpathlineto{\pgfqpoint{0.000000in}{-0.069444in}}%
\pgfusepath{stroke,fill}%
}%
\begin{pgfscope}%
\pgfsys@transformshift{2.871230in}{2.054978in}%
\pgfsys@useobject{currentmarker}{}%
\end{pgfscope}%
\end{pgfscope}%
\begin{pgfscope}%
\pgftext[x=2.871230in,y=0.371511in,,top]{\rmfamily\fontsize{8.000000}{9.600000}\selectfont 6500}%
\end{pgfscope}%
\begin{pgfscope}%
\pgfsetbuttcap%
\pgfsetroundjoin%
\definecolor{currentfill}{rgb}{0.000000,0.000000,0.000000}%
\pgfsetfillcolor{currentfill}%
\pgfsetlinewidth{0.501875pt}%
\definecolor{currentstroke}{rgb}{0.000000,0.000000,0.000000}%
\pgfsetstrokecolor{currentstroke}%
\pgfsetdash{}{0pt}%
\pgfsys@defobject{currentmarker}{\pgfqpoint{0.000000in}{0.000000in}}{\pgfqpoint{0.000000in}{0.069444in}}{%
\pgfpathmoveto{\pgfqpoint{0.000000in}{0.000000in}}%
\pgfpathlineto{\pgfqpoint{0.000000in}{0.069444in}}%
\pgfusepath{stroke,fill}%
}%
\begin{pgfscope}%
\pgfsys@transformshift{3.469480in}{0.440955in}%
\pgfsys@useobject{currentmarker}{}%
\end{pgfscope}%
\end{pgfscope}%
\begin{pgfscope}%
\pgfsetbuttcap%
\pgfsetroundjoin%
\definecolor{currentfill}{rgb}{0.000000,0.000000,0.000000}%
\pgfsetfillcolor{currentfill}%
\pgfsetlinewidth{0.501875pt}%
\definecolor{currentstroke}{rgb}{0.000000,0.000000,0.000000}%
\pgfsetstrokecolor{currentstroke}%
\pgfsetdash{}{0pt}%
\pgfsys@defobject{currentmarker}{\pgfqpoint{0.000000in}{-0.069444in}}{\pgfqpoint{0.000000in}{0.000000in}}{%
\pgfpathmoveto{\pgfqpoint{0.000000in}{0.000000in}}%
\pgfpathlineto{\pgfqpoint{0.000000in}{-0.069444in}}%
\pgfusepath{stroke,fill}%
}%
\begin{pgfscope}%
\pgfsys@transformshift{3.469480in}{2.054978in}%
\pgfsys@useobject{currentmarker}{}%
\end{pgfscope}%
\end{pgfscope}%
\begin{pgfscope}%
\pgftext[x=3.469480in,y=0.371511in,,top]{\rmfamily\fontsize{8.000000}{9.600000}\selectfont 7000}%
\end{pgfscope}%
\begin{pgfscope}%
\pgftext[x=2.033680in,y=0.194536in,,top]{\rmfamily\fontsize{9.000000}{10.800000}\selectfont \(\displaystyle m(K^+\pi^-\mu^+\mu^-)\ /\ \mathrm{MeV}\)}%
\end{pgfscope}%
\begin{pgfscope}%
\pgfsetbuttcap%
\pgfsetroundjoin%
\definecolor{currentfill}{rgb}{0.000000,0.000000,0.000000}%
\pgfsetfillcolor{currentfill}%
\pgfsetlinewidth{0.501875pt}%
\definecolor{currentstroke}{rgb}{0.000000,0.000000,0.000000}%
\pgfsetstrokecolor{currentstroke}%
\pgfsetdash{}{0pt}%
\pgfsys@defobject{currentmarker}{\pgfqpoint{0.000000in}{0.000000in}}{\pgfqpoint{0.069444in}{0.000000in}}{%
\pgfpathmoveto{\pgfqpoint{0.000000in}{0.000000in}}%
\pgfpathlineto{\pgfqpoint{0.069444in}{0.000000in}}%
\pgfusepath{stroke,fill}%
}%
\begin{pgfscope}%
\pgfsys@transformshift{0.597879in}{0.440955in}%
\pgfsys@useobject{currentmarker}{}%
\end{pgfscope}%
\end{pgfscope}%
\begin{pgfscope}%
\pgfsetbuttcap%
\pgfsetroundjoin%
\definecolor{currentfill}{rgb}{0.000000,0.000000,0.000000}%
\pgfsetfillcolor{currentfill}%
\pgfsetlinewidth{0.501875pt}%
\definecolor{currentstroke}{rgb}{0.000000,0.000000,0.000000}%
\pgfsetstrokecolor{currentstroke}%
\pgfsetdash{}{0pt}%
\pgfsys@defobject{currentmarker}{\pgfqpoint{-0.069444in}{0.000000in}}{\pgfqpoint{0.000000in}{0.000000in}}{%
\pgfpathmoveto{\pgfqpoint{0.000000in}{0.000000in}}%
\pgfpathlineto{\pgfqpoint{-0.069444in}{0.000000in}}%
\pgfusepath{stroke,fill}%
}%
\begin{pgfscope}%
\pgfsys@transformshift{3.469480in}{0.440955in}%
\pgfsys@useobject{currentmarker}{}%
\end{pgfscope}%
\end{pgfscope}%
\begin{pgfscope}%
\pgftext[x=0.528434in,y=0.440955in,right,]{\rmfamily\fontsize{8.000000}{9.600000}\selectfont 0}%
\end{pgfscope}%
\begin{pgfscope}%
\pgfsetbuttcap%
\pgfsetroundjoin%
\definecolor{currentfill}{rgb}{0.000000,0.000000,0.000000}%
\pgfsetfillcolor{currentfill}%
\pgfsetlinewidth{0.501875pt}%
\definecolor{currentstroke}{rgb}{0.000000,0.000000,0.000000}%
\pgfsetstrokecolor{currentstroke}%
\pgfsetdash{}{0pt}%
\pgfsys@defobject{currentmarker}{\pgfqpoint{0.000000in}{0.000000in}}{\pgfqpoint{0.069444in}{0.000000in}}{%
\pgfpathmoveto{\pgfqpoint{0.000000in}{0.000000in}}%
\pgfpathlineto{\pgfqpoint{0.069444in}{0.000000in}}%
\pgfusepath{stroke,fill}%
}%
\begin{pgfscope}%
\pgfsys@transformshift{0.597879in}{0.620291in}%
\pgfsys@useobject{currentmarker}{}%
\end{pgfscope}%
\end{pgfscope}%
\begin{pgfscope}%
\pgfsetbuttcap%
\pgfsetroundjoin%
\definecolor{currentfill}{rgb}{0.000000,0.000000,0.000000}%
\pgfsetfillcolor{currentfill}%
\pgfsetlinewidth{0.501875pt}%
\definecolor{currentstroke}{rgb}{0.000000,0.000000,0.000000}%
\pgfsetstrokecolor{currentstroke}%
\pgfsetdash{}{0pt}%
\pgfsys@defobject{currentmarker}{\pgfqpoint{-0.069444in}{0.000000in}}{\pgfqpoint{0.000000in}{0.000000in}}{%
\pgfpathmoveto{\pgfqpoint{0.000000in}{0.000000in}}%
\pgfpathlineto{\pgfqpoint{-0.069444in}{0.000000in}}%
\pgfusepath{stroke,fill}%
}%
\begin{pgfscope}%
\pgfsys@transformshift{3.469480in}{0.620291in}%
\pgfsys@useobject{currentmarker}{}%
\end{pgfscope}%
\end{pgfscope}%
\begin{pgfscope}%
\pgftext[x=0.528434in,y=0.620291in,right,]{\rmfamily\fontsize{8.000000}{9.600000}\selectfont 1000}%
\end{pgfscope}%
\begin{pgfscope}%
\pgfsetbuttcap%
\pgfsetroundjoin%
\definecolor{currentfill}{rgb}{0.000000,0.000000,0.000000}%
\pgfsetfillcolor{currentfill}%
\pgfsetlinewidth{0.501875pt}%
\definecolor{currentstroke}{rgb}{0.000000,0.000000,0.000000}%
\pgfsetstrokecolor{currentstroke}%
\pgfsetdash{}{0pt}%
\pgfsys@defobject{currentmarker}{\pgfqpoint{0.000000in}{0.000000in}}{\pgfqpoint{0.069444in}{0.000000in}}{%
\pgfpathmoveto{\pgfqpoint{0.000000in}{0.000000in}}%
\pgfpathlineto{\pgfqpoint{0.069444in}{0.000000in}}%
\pgfusepath{stroke,fill}%
}%
\begin{pgfscope}%
\pgfsys@transformshift{0.597879in}{0.799627in}%
\pgfsys@useobject{currentmarker}{}%
\end{pgfscope}%
\end{pgfscope}%
\begin{pgfscope}%
\pgfsetbuttcap%
\pgfsetroundjoin%
\definecolor{currentfill}{rgb}{0.000000,0.000000,0.000000}%
\pgfsetfillcolor{currentfill}%
\pgfsetlinewidth{0.501875pt}%
\definecolor{currentstroke}{rgb}{0.000000,0.000000,0.000000}%
\pgfsetstrokecolor{currentstroke}%
\pgfsetdash{}{0pt}%
\pgfsys@defobject{currentmarker}{\pgfqpoint{-0.069444in}{0.000000in}}{\pgfqpoint{0.000000in}{0.000000in}}{%
\pgfpathmoveto{\pgfqpoint{0.000000in}{0.000000in}}%
\pgfpathlineto{\pgfqpoint{-0.069444in}{0.000000in}}%
\pgfusepath{stroke,fill}%
}%
\begin{pgfscope}%
\pgfsys@transformshift{3.469480in}{0.799627in}%
\pgfsys@useobject{currentmarker}{}%
\end{pgfscope}%
\end{pgfscope}%
\begin{pgfscope}%
\pgftext[x=0.528434in,y=0.799627in,right,]{\rmfamily\fontsize{8.000000}{9.600000}\selectfont 2000}%
\end{pgfscope}%
\begin{pgfscope}%
\pgfsetbuttcap%
\pgfsetroundjoin%
\definecolor{currentfill}{rgb}{0.000000,0.000000,0.000000}%
\pgfsetfillcolor{currentfill}%
\pgfsetlinewidth{0.501875pt}%
\definecolor{currentstroke}{rgb}{0.000000,0.000000,0.000000}%
\pgfsetstrokecolor{currentstroke}%
\pgfsetdash{}{0pt}%
\pgfsys@defobject{currentmarker}{\pgfqpoint{0.000000in}{0.000000in}}{\pgfqpoint{0.069444in}{0.000000in}}{%
\pgfpathmoveto{\pgfqpoint{0.000000in}{0.000000in}}%
\pgfpathlineto{\pgfqpoint{0.069444in}{0.000000in}}%
\pgfusepath{stroke,fill}%
}%
\begin{pgfscope}%
\pgfsys@transformshift{0.597879in}{0.978963in}%
\pgfsys@useobject{currentmarker}{}%
\end{pgfscope}%
\end{pgfscope}%
\begin{pgfscope}%
\pgfsetbuttcap%
\pgfsetroundjoin%
\definecolor{currentfill}{rgb}{0.000000,0.000000,0.000000}%
\pgfsetfillcolor{currentfill}%
\pgfsetlinewidth{0.501875pt}%
\definecolor{currentstroke}{rgb}{0.000000,0.000000,0.000000}%
\pgfsetstrokecolor{currentstroke}%
\pgfsetdash{}{0pt}%
\pgfsys@defobject{currentmarker}{\pgfqpoint{-0.069444in}{0.000000in}}{\pgfqpoint{0.000000in}{0.000000in}}{%
\pgfpathmoveto{\pgfqpoint{0.000000in}{0.000000in}}%
\pgfpathlineto{\pgfqpoint{-0.069444in}{0.000000in}}%
\pgfusepath{stroke,fill}%
}%
\begin{pgfscope}%
\pgfsys@transformshift{3.469480in}{0.978963in}%
\pgfsys@useobject{currentmarker}{}%
\end{pgfscope}%
\end{pgfscope}%
\begin{pgfscope}%
\pgftext[x=0.528434in,y=0.978963in,right,]{\rmfamily\fontsize{8.000000}{9.600000}\selectfont 3000}%
\end{pgfscope}%
\begin{pgfscope}%
\pgfsetbuttcap%
\pgfsetroundjoin%
\definecolor{currentfill}{rgb}{0.000000,0.000000,0.000000}%
\pgfsetfillcolor{currentfill}%
\pgfsetlinewidth{0.501875pt}%
\definecolor{currentstroke}{rgb}{0.000000,0.000000,0.000000}%
\pgfsetstrokecolor{currentstroke}%
\pgfsetdash{}{0pt}%
\pgfsys@defobject{currentmarker}{\pgfqpoint{0.000000in}{0.000000in}}{\pgfqpoint{0.069444in}{0.000000in}}{%
\pgfpathmoveto{\pgfqpoint{0.000000in}{0.000000in}}%
\pgfpathlineto{\pgfqpoint{0.069444in}{0.000000in}}%
\pgfusepath{stroke,fill}%
}%
\begin{pgfscope}%
\pgfsys@transformshift{0.597879in}{1.158299in}%
\pgfsys@useobject{currentmarker}{}%
\end{pgfscope}%
\end{pgfscope}%
\begin{pgfscope}%
\pgfsetbuttcap%
\pgfsetroundjoin%
\definecolor{currentfill}{rgb}{0.000000,0.000000,0.000000}%
\pgfsetfillcolor{currentfill}%
\pgfsetlinewidth{0.501875pt}%
\definecolor{currentstroke}{rgb}{0.000000,0.000000,0.000000}%
\pgfsetstrokecolor{currentstroke}%
\pgfsetdash{}{0pt}%
\pgfsys@defobject{currentmarker}{\pgfqpoint{-0.069444in}{0.000000in}}{\pgfqpoint{0.000000in}{0.000000in}}{%
\pgfpathmoveto{\pgfqpoint{0.000000in}{0.000000in}}%
\pgfpathlineto{\pgfqpoint{-0.069444in}{0.000000in}}%
\pgfusepath{stroke,fill}%
}%
\begin{pgfscope}%
\pgfsys@transformshift{3.469480in}{1.158299in}%
\pgfsys@useobject{currentmarker}{}%
\end{pgfscope}%
\end{pgfscope}%
\begin{pgfscope}%
\pgftext[x=0.528434in,y=1.158299in,right,]{\rmfamily\fontsize{8.000000}{9.600000}\selectfont 4000}%
\end{pgfscope}%
\begin{pgfscope}%
\pgfsetbuttcap%
\pgfsetroundjoin%
\definecolor{currentfill}{rgb}{0.000000,0.000000,0.000000}%
\pgfsetfillcolor{currentfill}%
\pgfsetlinewidth{0.501875pt}%
\definecolor{currentstroke}{rgb}{0.000000,0.000000,0.000000}%
\pgfsetstrokecolor{currentstroke}%
\pgfsetdash{}{0pt}%
\pgfsys@defobject{currentmarker}{\pgfqpoint{0.000000in}{0.000000in}}{\pgfqpoint{0.069444in}{0.000000in}}{%
\pgfpathmoveto{\pgfqpoint{0.000000in}{0.000000in}}%
\pgfpathlineto{\pgfqpoint{0.069444in}{0.000000in}}%
\pgfusepath{stroke,fill}%
}%
\begin{pgfscope}%
\pgfsys@transformshift{0.597879in}{1.337635in}%
\pgfsys@useobject{currentmarker}{}%
\end{pgfscope}%
\end{pgfscope}%
\begin{pgfscope}%
\pgfsetbuttcap%
\pgfsetroundjoin%
\definecolor{currentfill}{rgb}{0.000000,0.000000,0.000000}%
\pgfsetfillcolor{currentfill}%
\pgfsetlinewidth{0.501875pt}%
\definecolor{currentstroke}{rgb}{0.000000,0.000000,0.000000}%
\pgfsetstrokecolor{currentstroke}%
\pgfsetdash{}{0pt}%
\pgfsys@defobject{currentmarker}{\pgfqpoint{-0.069444in}{0.000000in}}{\pgfqpoint{0.000000in}{0.000000in}}{%
\pgfpathmoveto{\pgfqpoint{0.000000in}{0.000000in}}%
\pgfpathlineto{\pgfqpoint{-0.069444in}{0.000000in}}%
\pgfusepath{stroke,fill}%
}%
\begin{pgfscope}%
\pgfsys@transformshift{3.469480in}{1.337635in}%
\pgfsys@useobject{currentmarker}{}%
\end{pgfscope}%
\end{pgfscope}%
\begin{pgfscope}%
\pgftext[x=0.528434in,y=1.337635in,right,]{\rmfamily\fontsize{8.000000}{9.600000}\selectfont 5000}%
\end{pgfscope}%
\begin{pgfscope}%
\pgfsetbuttcap%
\pgfsetroundjoin%
\definecolor{currentfill}{rgb}{0.000000,0.000000,0.000000}%
\pgfsetfillcolor{currentfill}%
\pgfsetlinewidth{0.501875pt}%
\definecolor{currentstroke}{rgb}{0.000000,0.000000,0.000000}%
\pgfsetstrokecolor{currentstroke}%
\pgfsetdash{}{0pt}%
\pgfsys@defobject{currentmarker}{\pgfqpoint{0.000000in}{0.000000in}}{\pgfqpoint{0.069444in}{0.000000in}}{%
\pgfpathmoveto{\pgfqpoint{0.000000in}{0.000000in}}%
\pgfpathlineto{\pgfqpoint{0.069444in}{0.000000in}}%
\pgfusepath{stroke,fill}%
}%
\begin{pgfscope}%
\pgfsys@transformshift{0.597879in}{1.516970in}%
\pgfsys@useobject{currentmarker}{}%
\end{pgfscope}%
\end{pgfscope}%
\begin{pgfscope}%
\pgfsetbuttcap%
\pgfsetroundjoin%
\definecolor{currentfill}{rgb}{0.000000,0.000000,0.000000}%
\pgfsetfillcolor{currentfill}%
\pgfsetlinewidth{0.501875pt}%
\definecolor{currentstroke}{rgb}{0.000000,0.000000,0.000000}%
\pgfsetstrokecolor{currentstroke}%
\pgfsetdash{}{0pt}%
\pgfsys@defobject{currentmarker}{\pgfqpoint{-0.069444in}{0.000000in}}{\pgfqpoint{0.000000in}{0.000000in}}{%
\pgfpathmoveto{\pgfqpoint{0.000000in}{0.000000in}}%
\pgfpathlineto{\pgfqpoint{-0.069444in}{0.000000in}}%
\pgfusepath{stroke,fill}%
}%
\begin{pgfscope}%
\pgfsys@transformshift{3.469480in}{1.516970in}%
\pgfsys@useobject{currentmarker}{}%
\end{pgfscope}%
\end{pgfscope}%
\begin{pgfscope}%
\pgftext[x=0.528434in,y=1.516970in,right,]{\rmfamily\fontsize{8.000000}{9.600000}\selectfont 6000}%
\end{pgfscope}%
\begin{pgfscope}%
\pgfsetbuttcap%
\pgfsetroundjoin%
\definecolor{currentfill}{rgb}{0.000000,0.000000,0.000000}%
\pgfsetfillcolor{currentfill}%
\pgfsetlinewidth{0.501875pt}%
\definecolor{currentstroke}{rgb}{0.000000,0.000000,0.000000}%
\pgfsetstrokecolor{currentstroke}%
\pgfsetdash{}{0pt}%
\pgfsys@defobject{currentmarker}{\pgfqpoint{0.000000in}{0.000000in}}{\pgfqpoint{0.069444in}{0.000000in}}{%
\pgfpathmoveto{\pgfqpoint{0.000000in}{0.000000in}}%
\pgfpathlineto{\pgfqpoint{0.069444in}{0.000000in}}%
\pgfusepath{stroke,fill}%
}%
\begin{pgfscope}%
\pgfsys@transformshift{0.597879in}{1.696306in}%
\pgfsys@useobject{currentmarker}{}%
\end{pgfscope}%
\end{pgfscope}%
\begin{pgfscope}%
\pgfsetbuttcap%
\pgfsetroundjoin%
\definecolor{currentfill}{rgb}{0.000000,0.000000,0.000000}%
\pgfsetfillcolor{currentfill}%
\pgfsetlinewidth{0.501875pt}%
\definecolor{currentstroke}{rgb}{0.000000,0.000000,0.000000}%
\pgfsetstrokecolor{currentstroke}%
\pgfsetdash{}{0pt}%
\pgfsys@defobject{currentmarker}{\pgfqpoint{-0.069444in}{0.000000in}}{\pgfqpoint{0.000000in}{0.000000in}}{%
\pgfpathmoveto{\pgfqpoint{0.000000in}{0.000000in}}%
\pgfpathlineto{\pgfqpoint{-0.069444in}{0.000000in}}%
\pgfusepath{stroke,fill}%
}%
\begin{pgfscope}%
\pgfsys@transformshift{3.469480in}{1.696306in}%
\pgfsys@useobject{currentmarker}{}%
\end{pgfscope}%
\end{pgfscope}%
\begin{pgfscope}%
\pgftext[x=0.528434in,y=1.696306in,right,]{\rmfamily\fontsize{8.000000}{9.600000}\selectfont 7000}%
\end{pgfscope}%
\begin{pgfscope}%
\pgfsetbuttcap%
\pgfsetroundjoin%
\definecolor{currentfill}{rgb}{0.000000,0.000000,0.000000}%
\pgfsetfillcolor{currentfill}%
\pgfsetlinewidth{0.501875pt}%
\definecolor{currentstroke}{rgb}{0.000000,0.000000,0.000000}%
\pgfsetstrokecolor{currentstroke}%
\pgfsetdash{}{0pt}%
\pgfsys@defobject{currentmarker}{\pgfqpoint{0.000000in}{0.000000in}}{\pgfqpoint{0.069444in}{0.000000in}}{%
\pgfpathmoveto{\pgfqpoint{0.000000in}{0.000000in}}%
\pgfpathlineto{\pgfqpoint{0.069444in}{0.000000in}}%
\pgfusepath{stroke,fill}%
}%
\begin{pgfscope}%
\pgfsys@transformshift{0.597879in}{1.875642in}%
\pgfsys@useobject{currentmarker}{}%
\end{pgfscope}%
\end{pgfscope}%
\begin{pgfscope}%
\pgfsetbuttcap%
\pgfsetroundjoin%
\definecolor{currentfill}{rgb}{0.000000,0.000000,0.000000}%
\pgfsetfillcolor{currentfill}%
\pgfsetlinewidth{0.501875pt}%
\definecolor{currentstroke}{rgb}{0.000000,0.000000,0.000000}%
\pgfsetstrokecolor{currentstroke}%
\pgfsetdash{}{0pt}%
\pgfsys@defobject{currentmarker}{\pgfqpoint{-0.069444in}{0.000000in}}{\pgfqpoint{0.000000in}{0.000000in}}{%
\pgfpathmoveto{\pgfqpoint{0.000000in}{0.000000in}}%
\pgfpathlineto{\pgfqpoint{-0.069444in}{0.000000in}}%
\pgfusepath{stroke,fill}%
}%
\begin{pgfscope}%
\pgfsys@transformshift{3.469480in}{1.875642in}%
\pgfsys@useobject{currentmarker}{}%
\end{pgfscope}%
\end{pgfscope}%
\begin{pgfscope}%
\pgftext[x=0.528434in,y=1.875642in,right,]{\rmfamily\fontsize{8.000000}{9.600000}\selectfont 8000}%
\end{pgfscope}%
\begin{pgfscope}%
\pgfsetbuttcap%
\pgfsetroundjoin%
\definecolor{currentfill}{rgb}{0.000000,0.000000,0.000000}%
\pgfsetfillcolor{currentfill}%
\pgfsetlinewidth{0.501875pt}%
\definecolor{currentstroke}{rgb}{0.000000,0.000000,0.000000}%
\pgfsetstrokecolor{currentstroke}%
\pgfsetdash{}{0pt}%
\pgfsys@defobject{currentmarker}{\pgfqpoint{0.000000in}{0.000000in}}{\pgfqpoint{0.069444in}{0.000000in}}{%
\pgfpathmoveto{\pgfqpoint{0.000000in}{0.000000in}}%
\pgfpathlineto{\pgfqpoint{0.069444in}{0.000000in}}%
\pgfusepath{stroke,fill}%
}%
\begin{pgfscope}%
\pgfsys@transformshift{0.597879in}{2.054978in}%
\pgfsys@useobject{currentmarker}{}%
\end{pgfscope}%
\end{pgfscope}%
\begin{pgfscope}%
\pgfsetbuttcap%
\pgfsetroundjoin%
\definecolor{currentfill}{rgb}{0.000000,0.000000,0.000000}%
\pgfsetfillcolor{currentfill}%
\pgfsetlinewidth{0.501875pt}%
\definecolor{currentstroke}{rgb}{0.000000,0.000000,0.000000}%
\pgfsetstrokecolor{currentstroke}%
\pgfsetdash{}{0pt}%
\pgfsys@defobject{currentmarker}{\pgfqpoint{-0.069444in}{0.000000in}}{\pgfqpoint{0.000000in}{0.000000in}}{%
\pgfpathmoveto{\pgfqpoint{0.000000in}{0.000000in}}%
\pgfpathlineto{\pgfqpoint{-0.069444in}{0.000000in}}%
\pgfusepath{stroke,fill}%
}%
\begin{pgfscope}%
\pgfsys@transformshift{3.469480in}{2.054978in}%
\pgfsys@useobject{currentmarker}{}%
\end{pgfscope}%
\end{pgfscope}%
\begin{pgfscope}%
\pgftext[x=0.528434in,y=2.054978in,right,]{\rmfamily\fontsize{8.000000}{9.600000}\selectfont 9000}%
\end{pgfscope}%
\begin{pgfscope}%
\pgftext[x=0.176221in,y=1.247967in,,bottom,rotate=90.000000]{\rmfamily\fontsize{9.000000}{10.800000}\selectfont Candidates \(\displaystyle /\ 9.60\ \mathrm{MeV}\)}%
\end{pgfscope}%
\end{pgfpicture}%
\makeatother%
\endgroup%
}
  \caption{
    Reconstructed \PBzero mass of the signal data sample after trigger and stripping requirements.
    A window around the nominal \PBzero mass has been removed as a means of blinding the measurement.
  }
\end{figure}

\subsection{Trigger selection}

Both of the decays $\PBzero\to\APDzero\APmuon\Pmuon$ and $\PBzero\to\PJpsi(\APmuon\Pmuon)\PKstar$ feature two muons in their final states.
Accordingly, a selection of general and muon-specific trigger algorithms has been chosen for the analysis.
See table \ref{tab:trigger} for a list of the trigger algorithms and the efficiencies associated with each trigger stage, as well as the total trigger efficiency for both decay channels, evaluated using simulated candidates.

The reduced trigger efficiency of $B^0\to\APDzero\APmuon\Pmuon$ compared to the normalization channel can be explained by the fact that it contains a significant fraction of low $q^2$ muon pairs.
These low-momentum muons are not triggered as efficiently in the L0 trigger as those originating from a \PJpsi resonance.

The reconstructed $B^0$ mass distribution after applying stripping and trigger requirements is given in figure \ref{fig:bmass}.

The fact that the trigger efficiencies differ strongly between the signal and normalization channels means that the data-driven TISTOS method \cite{TisTos}, which would require using the normalization channel as a proxy for the signal, is not applicable.

\begin{table}
  \centering
  \caption{
    Trigger algorithms required for both the signal and normalization channels.
    The strategies used consist mainly of muon and topological requirements.
    Events are required to match at least one of the trigger algorithms per stage.
    The trigger efficiencies as determined on simulated candidates are given for both the signal and normalization channel.
  }
  \begin{tabular}{l l S[table-format=2.2,table-figures-uncertainty=1] S[table-format=2.2,table-figures-uncertainty=1]}
    \toprule
    Stage & Trigger algorithms & {Eff. (signal) $/\ \si{\percent}$} & {Eff. (norm.) $/\ \si{\percent}$} \\
    \midrule
    L0   & \texttt{L0Muon} & 75.87 \pm 0.13 & 87.34 \pm 0.04 \\ 
    \midrule
    HLT1 & \texttt{Hlt1TrackAllL0} & 94.99 \pm 0.07 & 94.92 \pm 0.03 \\ 
         & \texttt{Hlt1TrackMuon} & & \\ 
    \midrule
    HLT2 & \texttt{Hlt2SingleMuonDecision} & 96.12 \pm 0.07 & 96.62 \pm 0.02 \\ 
         & \texttt{Hlt2DiMuonDetachedDecision} & & \\ 
         & \texttt{Hlt2Topo\{2,3,4\}BodyBBDT} & & \\ 
         & \texttt{Hlt2TopoMu\{2,3,4\}BodyBBDT} & & \\ 
    \midrule
    Total & & 69.27 \pm 0.14 & 80.10 \pm 0.04 \\
    \bottomrule
  \end{tabular}
  \label{tab:trigger}
\end{table}

\begin{figure}
  \centering
  {%% Creator: Matplotlib, PGF backend
%%
%% To include the figure in your LaTeX document, write
%%   \input{<filename>.pgf}
%%
%% Make sure the required packages are loaded in your preamble
%%   \usepackage{pgf}
%%
%% Figures using additional raster images can only be included by \input if
%% they are in the same directory as the main LaTeX file. For loading figures
%% from other directories you can use the `import` package
%%   \usepackage{import}
%% and then include the figures with
%%   \import{<path to file>}{<filename>.pgf}
%%
%% Matplotlib used the following preamble
%%   \usepackage{fontspec}
%%   \setmainfont{DejaVu Serif}
%%   \setsansfont{DejaVu Sans}
%%   \setmonofont{DejaVu Sans Mono}
%%
\begingroup%
\makeatletter%
\begin{pgfpicture}%
\pgfpathrectangle{\pgfpointorigin}{\pgfqpoint{3.661135in}{2.158743in}}%
\pgfusepath{use as bounding box, clip}%
\begin{pgfscope}%
\pgfsetbuttcap%
\pgfsetmiterjoin%
\definecolor{currentfill}{rgb}{1.000000,1.000000,1.000000}%
\pgfsetfillcolor{currentfill}%
\pgfsetlinewidth{0.000000pt}%
\definecolor{currentstroke}{rgb}{1.000000,1.000000,1.000000}%
\pgfsetstrokecolor{currentstroke}%
\pgfsetdash{}{0pt}%
\pgfpathmoveto{\pgfqpoint{0.000000in}{0.000000in}}%
\pgfpathlineto{\pgfqpoint{3.661135in}{0.000000in}}%
\pgfpathlineto{\pgfqpoint{3.661135in}{2.158743in}}%
\pgfpathlineto{\pgfqpoint{0.000000in}{2.158743in}}%
\pgfpathclose%
\pgfusepath{fill}%
\end{pgfscope}%
\begin{pgfscope}%
\pgfsetbuttcap%
\pgfsetmiterjoin%
\definecolor{currentfill}{rgb}{1.000000,1.000000,1.000000}%
\pgfsetfillcolor{currentfill}%
\pgfsetlinewidth{0.000000pt}%
\definecolor{currentstroke}{rgb}{0.000000,0.000000,0.000000}%
\pgfsetstrokecolor{currentstroke}%
\pgfsetstrokeopacity{0.000000}%
\pgfsetdash{}{0pt}%
\pgfpathmoveto{\pgfqpoint{0.668571in}{0.440955in}}%
\pgfpathlineto{\pgfqpoint{3.469750in}{0.440955in}}%
\pgfpathlineto{\pgfqpoint{3.469750in}{2.054978in}}%
\pgfpathlineto{\pgfqpoint{0.668571in}{2.054978in}}%
\pgfpathclose%
\pgfusepath{fill}%
\end{pgfscope}%
\begin{pgfscope}%
\pgfpathrectangle{\pgfqpoint{0.668571in}{0.440955in}}{\pgfqpoint{2.801179in}{1.614023in}} %
\pgfusepath{clip}%
\pgfsetbuttcap%
\pgfsetmiterjoin%
\definecolor{currentfill}{rgb}{0.215686,0.470588,0.749020}%
\pgfsetfillcolor{currentfill}%
\pgfsetlinewidth{0.000000pt}%
\definecolor{currentstroke}{rgb}{0.000000,0.000000,0.000000}%
\pgfsetstrokecolor{currentstroke}%
\pgfsetdash{}{0pt}%
\pgfpathmoveto{\pgfqpoint{0.668571in}{0.440955in}}%
\pgfpathlineto{\pgfqpoint{0.668571in}{1.202082in}}%
\pgfpathlineto{\pgfqpoint{0.679776in}{1.202082in}}%
\pgfpathlineto{\pgfqpoint{0.679776in}{1.208446in}}%
\pgfpathlineto{\pgfqpoint{0.690980in}{1.208446in}}%
\pgfpathlineto{\pgfqpoint{0.690980in}{1.202912in}}%
\pgfpathlineto{\pgfqpoint{0.702185in}{1.202912in}}%
\pgfpathlineto{\pgfqpoint{0.702185in}{1.191891in}}%
\pgfpathlineto{\pgfqpoint{0.713390in}{1.191891in}}%
\pgfpathlineto{\pgfqpoint{0.713390in}{1.191176in}}%
\pgfpathlineto{\pgfqpoint{0.724595in}{1.191176in}}%
\pgfpathlineto{\pgfqpoint{0.724595in}{1.196110in}}%
\pgfpathlineto{\pgfqpoint{0.735799in}{1.196110in}}%
\pgfpathlineto{\pgfqpoint{0.735799in}{1.189032in}}%
\pgfpathlineto{\pgfqpoint{0.747004in}{1.189032in}}%
\pgfpathlineto{\pgfqpoint{0.747004in}{1.196941in}}%
\pgfpathlineto{\pgfqpoint{0.758209in}{1.196941in}}%
\pgfpathlineto{\pgfqpoint{0.758209in}{1.184905in}}%
\pgfpathlineto{\pgfqpoint{0.769413in}{1.184905in}}%
\pgfpathlineto{\pgfqpoint{0.769413in}{1.183221in}}%
\pgfpathlineto{\pgfqpoint{0.780618in}{1.183221in}}%
\pgfpathlineto{\pgfqpoint{0.780618in}{1.184213in}}%
\pgfpathlineto{\pgfqpoint{0.791823in}{1.184213in}}%
\pgfpathlineto{\pgfqpoint{0.791823in}{1.179601in}}%
\pgfpathlineto{\pgfqpoint{0.803028in}{1.179601in}}%
\pgfpathlineto{\pgfqpoint{0.803028in}{1.176050in}}%
\pgfpathlineto{\pgfqpoint{0.814232in}{1.176050in}}%
\pgfpathlineto{\pgfqpoint{0.814232in}{1.175981in}}%
\pgfpathlineto{\pgfqpoint{0.825437in}{1.175981in}}%
\pgfpathlineto{\pgfqpoint{0.825437in}{1.173284in}}%
\pgfpathlineto{\pgfqpoint{0.836642in}{1.173284in}}%
\pgfpathlineto{\pgfqpoint{0.836642in}{1.166528in}}%
\pgfpathlineto{\pgfqpoint{0.847846in}{1.166528in}}%
\pgfpathlineto{\pgfqpoint{0.847846in}{1.163968in}}%
\pgfpathlineto{\pgfqpoint{0.859051in}{1.163968in}}%
\pgfpathlineto{\pgfqpoint{0.859051in}{1.167150in}}%
\pgfpathlineto{\pgfqpoint{0.870256in}{1.167150in}}%
\pgfpathlineto{\pgfqpoint{0.870256in}{1.169871in}}%
\pgfpathlineto{\pgfqpoint{0.881461in}{1.169871in}}%
\pgfpathlineto{\pgfqpoint{0.881461in}{1.160717in}}%
\pgfpathlineto{\pgfqpoint{0.892665in}{1.160717in}}%
\pgfpathlineto{\pgfqpoint{0.892665in}{1.156498in}}%
\pgfpathlineto{\pgfqpoint{0.903870in}{1.156498in}}%
\pgfpathlineto{\pgfqpoint{0.903870in}{1.158619in}}%
\pgfpathlineto{\pgfqpoint{0.915075in}{1.158619in}}%
\pgfpathlineto{\pgfqpoint{0.915075in}{1.145845in}}%
\pgfpathlineto{\pgfqpoint{0.926279in}{1.145845in}}%
\pgfpathlineto{\pgfqpoint{0.926279in}{1.148935in}}%
\pgfpathlineto{\pgfqpoint{0.937484in}{1.148935in}}%
\pgfpathlineto{\pgfqpoint{0.937484in}{1.143355in}}%
\pgfpathlineto{\pgfqpoint{0.948689in}{1.143355in}}%
\pgfpathlineto{\pgfqpoint{0.948689in}{1.152117in}}%
\pgfpathlineto{\pgfqpoint{0.959894in}{1.152117in}}%
\pgfpathlineto{\pgfqpoint{0.959894in}{1.148266in}}%
\pgfpathlineto{\pgfqpoint{0.971098in}{1.148266in}}%
\pgfpathlineto{\pgfqpoint{0.971098in}{1.142848in}}%
\pgfpathlineto{\pgfqpoint{0.982303in}{1.142848in}}%
\pgfpathlineto{\pgfqpoint{0.982303in}{1.130996in}}%
\pgfpathlineto{\pgfqpoint{0.993508in}{1.130996in}}%
\pgfpathlineto{\pgfqpoint{0.993508in}{1.131895in}}%
\pgfpathlineto{\pgfqpoint{1.004712in}{1.131895in}}%
\pgfpathlineto{\pgfqpoint{1.004712in}{1.133117in}}%
\pgfpathlineto{\pgfqpoint{1.015917in}{1.133117in}}%
\pgfpathlineto{\pgfqpoint{1.015917in}{1.128091in}}%
\pgfpathlineto{\pgfqpoint{1.027122in}{1.128091in}}%
\pgfpathlineto{\pgfqpoint{1.027122in}{1.128552in}}%
\pgfpathlineto{\pgfqpoint{1.038327in}{1.128552in}}%
\pgfpathlineto{\pgfqpoint{1.038327in}{1.119790in}}%
\pgfpathlineto{\pgfqpoint{1.049531in}{1.119790in}}%
\pgfpathlineto{\pgfqpoint{1.049531in}{1.120805in}}%
\pgfpathlineto{\pgfqpoint{1.060736in}{1.120805in}}%
\pgfpathlineto{\pgfqpoint{1.060736in}{1.116447in}}%
\pgfpathlineto{\pgfqpoint{1.071941in}{1.116447in}}%
\pgfpathlineto{\pgfqpoint{1.071941in}{1.110014in}}%
\pgfpathlineto{\pgfqpoint{1.083145in}{1.110014in}}%
\pgfpathlineto{\pgfqpoint{1.083145in}{1.111674in}}%
\pgfpathlineto{\pgfqpoint{1.094350in}{1.111674in}}%
\pgfpathlineto{\pgfqpoint{1.094350in}{1.108077in}}%
\pgfpathlineto{\pgfqpoint{1.105555in}{1.108077in}}%
\pgfpathlineto{\pgfqpoint{1.105555in}{1.099500in}}%
\pgfpathlineto{\pgfqpoint{1.116760in}{1.099500in}}%
\pgfpathlineto{\pgfqpoint{1.116760in}{1.100952in}}%
\pgfpathlineto{\pgfqpoint{1.127964in}{1.100952in}}%
\pgfpathlineto{\pgfqpoint{1.127964in}{1.094104in}}%
\pgfpathlineto{\pgfqpoint{1.139169in}{1.094104in}}%
\pgfpathlineto{\pgfqpoint{1.139169in}{1.088778in}}%
\pgfpathlineto{\pgfqpoint{1.150374in}{1.088778in}}%
\pgfpathlineto{\pgfqpoint{1.150374in}{1.092836in}}%
\pgfpathlineto{\pgfqpoint{1.161578in}{1.092836in}}%
\pgfpathlineto{\pgfqpoint{1.161578in}{1.089516in}}%
\pgfpathlineto{\pgfqpoint{1.172783in}{1.089516in}}%
\pgfpathlineto{\pgfqpoint{1.172783in}{1.091476in}}%
\pgfpathlineto{\pgfqpoint{1.183988in}{1.091476in}}%
\pgfpathlineto{\pgfqpoint{1.183988in}{1.077618in}}%
\pgfpathlineto{\pgfqpoint{1.195193in}{1.077618in}}%
\pgfpathlineto{\pgfqpoint{1.195193in}{1.085112in}}%
\pgfpathlineto{\pgfqpoint{1.206397in}{1.085112in}}%
\pgfpathlineto{\pgfqpoint{1.206397in}{1.072130in}}%
\pgfpathlineto{\pgfqpoint{1.217602in}{1.072130in}}%
\pgfpathlineto{\pgfqpoint{1.217602in}{1.074136in}}%
\pgfpathlineto{\pgfqpoint{1.228807in}{1.074136in}}%
\pgfpathlineto{\pgfqpoint{1.228807in}{1.069018in}}%
\pgfpathlineto{\pgfqpoint{1.240011in}{1.069018in}}%
\pgfpathlineto{\pgfqpoint{1.240011in}{1.068833in}}%
\pgfpathlineto{\pgfqpoint{1.251216in}{1.068833in}}%
\pgfpathlineto{\pgfqpoint{1.251216in}{1.065928in}}%
\pgfpathlineto{\pgfqpoint{1.262421in}{1.065928in}}%
\pgfpathlineto{\pgfqpoint{1.262421in}{1.061639in}}%
\pgfpathlineto{\pgfqpoint{1.273626in}{1.061639in}}%
\pgfpathlineto{\pgfqpoint{1.273626in}{1.058665in}}%
\pgfpathlineto{\pgfqpoint{1.284830in}{1.058665in}}%
\pgfpathlineto{\pgfqpoint{1.284830in}{1.061317in}}%
\pgfpathlineto{\pgfqpoint{1.296035in}{1.061317in}}%
\pgfpathlineto{\pgfqpoint{1.296035in}{1.057189in}}%
\pgfpathlineto{\pgfqpoint{1.307240in}{1.057189in}}%
\pgfpathlineto{\pgfqpoint{1.307240in}{1.055644in}}%
\pgfpathlineto{\pgfqpoint{1.318444in}{1.055644in}}%
\pgfpathlineto{\pgfqpoint{1.318444in}{1.057351in}}%
\pgfpathlineto{\pgfqpoint{1.329649in}{1.057351in}}%
\pgfpathlineto{\pgfqpoint{1.329649in}{1.059610in}}%
\pgfpathlineto{\pgfqpoint{1.340854in}{1.059610in}}%
\pgfpathlineto{\pgfqpoint{1.340854in}{1.050018in}}%
\pgfpathlineto{\pgfqpoint{1.352059in}{1.050018in}}%
\pgfpathlineto{\pgfqpoint{1.352059in}{1.055575in}}%
\pgfpathlineto{\pgfqpoint{1.363263in}{1.055575in}}%
\pgfpathlineto{\pgfqpoint{1.363263in}{1.046560in}}%
\pgfpathlineto{\pgfqpoint{1.374468in}{1.046560in}}%
\pgfpathlineto{\pgfqpoint{1.374468in}{1.051056in}}%
\pgfpathlineto{\pgfqpoint{1.385673in}{1.051056in}}%
\pgfpathlineto{\pgfqpoint{1.385673in}{1.048566in}}%
\pgfpathlineto{\pgfqpoint{1.396878in}{1.048566in}}%
\pgfpathlineto{\pgfqpoint{1.396878in}{1.044600in}}%
\pgfpathlineto{\pgfqpoint{1.408082in}{1.044600in}}%
\pgfpathlineto{\pgfqpoint{1.408082in}{1.053615in}}%
\pgfpathlineto{\pgfqpoint{1.419287in}{1.053615in}}%
\pgfpathlineto{\pgfqpoint{1.419287in}{1.043424in}}%
\pgfpathlineto{\pgfqpoint{1.430492in}{1.043424in}}%
\pgfpathlineto{\pgfqpoint{1.430492in}{1.055460in}}%
\pgfpathlineto{\pgfqpoint{1.441696in}{1.055460in}}%
\pgfpathlineto{\pgfqpoint{1.441696in}{1.054238in}}%
\pgfpathlineto{\pgfqpoint{1.452901in}{1.054238in}}%
\pgfpathlineto{\pgfqpoint{1.452901in}{1.051563in}}%
\pgfpathlineto{\pgfqpoint{1.464106in}{1.051563in}}%
\pgfpathlineto{\pgfqpoint{1.464106in}{1.054238in}}%
\pgfpathlineto{\pgfqpoint{1.475311in}{1.054238in}}%
\pgfpathlineto{\pgfqpoint{1.475311in}{1.064729in}}%
\pgfpathlineto{\pgfqpoint{1.486515in}{1.064729in}}%
\pgfpathlineto{\pgfqpoint{1.486515in}{1.066089in}}%
\pgfpathlineto{\pgfqpoint{1.497720in}{1.066089in}}%
\pgfpathlineto{\pgfqpoint{1.497720in}{1.052947in}}%
\pgfpathlineto{\pgfqpoint{1.508925in}{1.052947in}}%
\pgfpathlineto{\pgfqpoint{1.508925in}{1.062008in}}%
\pgfpathlineto{\pgfqpoint{1.520129in}{1.062008in}}%
\pgfpathlineto{\pgfqpoint{1.520129in}{1.063000in}}%
\pgfpathlineto{\pgfqpoint{1.531334in}{1.063000in}}%
\pgfpathlineto{\pgfqpoint{1.531334in}{1.080523in}}%
\pgfpathlineto{\pgfqpoint{1.542539in}{1.080523in}}%
\pgfpathlineto{\pgfqpoint{1.542539in}{1.074113in}}%
\pgfpathlineto{\pgfqpoint{1.553744in}{1.074113in}}%
\pgfpathlineto{\pgfqpoint{1.553744in}{1.076281in}}%
\pgfpathlineto{\pgfqpoint{1.564948in}{1.076281in}}%
\pgfpathlineto{\pgfqpoint{1.564948in}{1.076696in}}%
\pgfpathlineto{\pgfqpoint{1.576153in}{1.076696in}}%
\pgfpathlineto{\pgfqpoint{1.576153in}{1.082299in}}%
\pgfpathlineto{\pgfqpoint{1.587358in}{1.082299in}}%
\pgfpathlineto{\pgfqpoint{1.587358in}{1.083452in}}%
\pgfpathlineto{\pgfqpoint{1.598562in}{1.083452in}}%
\pgfpathlineto{\pgfqpoint{1.598562in}{1.086703in}}%
\pgfpathlineto{\pgfqpoint{1.609767in}{1.086703in}}%
\pgfpathlineto{\pgfqpoint{1.609767in}{1.088962in}}%
\pgfpathlineto{\pgfqpoint{1.620972in}{1.088962in}}%
\pgfpathlineto{\pgfqpoint{1.620972in}{1.094335in}}%
\pgfpathlineto{\pgfqpoint{1.632177in}{1.094335in}}%
\pgfpathlineto{\pgfqpoint{1.632177in}{1.097724in}}%
\pgfpathlineto{\pgfqpoint{1.643381in}{1.097724in}}%
\pgfpathlineto{\pgfqpoint{1.643381in}{1.100768in}}%
\pgfpathlineto{\pgfqpoint{1.654586in}{1.100768in}}%
\pgfpathlineto{\pgfqpoint{1.654586in}{1.102244in}}%
\pgfpathlineto{\pgfqpoint{1.665791in}{1.102244in}}%
\pgfpathlineto{\pgfqpoint{1.665791in}{1.115063in}}%
\pgfpathlineto{\pgfqpoint{1.676995in}{1.115063in}}%
\pgfpathlineto{\pgfqpoint{1.676995in}{1.121658in}}%
\pgfpathlineto{\pgfqpoint{1.688200in}{1.121658in}}%
\pgfpathlineto{\pgfqpoint{1.688200in}{1.131319in}}%
\pgfpathlineto{\pgfqpoint{1.699405in}{1.131319in}}%
\pgfpathlineto{\pgfqpoint{1.699405in}{1.130166in}}%
\pgfpathlineto{\pgfqpoint{1.710610in}{1.130166in}}%
\pgfpathlineto{\pgfqpoint{1.710610in}{1.147229in}}%
\pgfpathlineto{\pgfqpoint{1.721814in}{1.147229in}}%
\pgfpathlineto{\pgfqpoint{1.721814in}{1.138213in}}%
\pgfpathlineto{\pgfqpoint{1.733019in}{1.138213in}}%
\pgfpathlineto{\pgfqpoint{1.733019in}{1.148382in}}%
\pgfpathlineto{\pgfqpoint{1.744224in}{1.148382in}}%
\pgfpathlineto{\pgfqpoint{1.744224in}{1.158412in}}%
\pgfpathlineto{\pgfqpoint{1.755428in}{1.158412in}}%
\pgfpathlineto{\pgfqpoint{1.755428in}{1.175082in}}%
\pgfpathlineto{\pgfqpoint{1.766633in}{1.175082in}}%
\pgfpathlineto{\pgfqpoint{1.766633in}{1.186818in}}%
\pgfpathlineto{\pgfqpoint{1.777838in}{1.186818in}}%
\pgfpathlineto{\pgfqpoint{1.777838in}{1.199984in}}%
\pgfpathlineto{\pgfqpoint{1.789043in}{1.199984in}}%
\pgfpathlineto{\pgfqpoint{1.789043in}{1.215202in}}%
\pgfpathlineto{\pgfqpoint{1.800247in}{1.215202in}}%
\pgfpathlineto{\pgfqpoint{1.800247in}{1.224863in}}%
\pgfpathlineto{\pgfqpoint{1.811452in}{1.224863in}}%
\pgfpathlineto{\pgfqpoint{1.811452in}{1.241349in}}%
\pgfpathlineto{\pgfqpoint{1.822657in}{1.241349in}}%
\pgfpathlineto{\pgfqpoint{1.822657in}{1.263208in}}%
\pgfpathlineto{\pgfqpoint{1.833861in}{1.263208in}}%
\pgfpathlineto{\pgfqpoint{1.833861in}{1.285435in}}%
\pgfpathlineto{\pgfqpoint{1.845066in}{1.285435in}}%
\pgfpathlineto{\pgfqpoint{1.845066in}{1.333418in}}%
\pgfpathlineto{\pgfqpoint{1.856271in}{1.333418in}}%
\pgfpathlineto{\pgfqpoint{1.856271in}{1.374183in}}%
\pgfpathlineto{\pgfqpoint{1.867476in}{1.374183in}}%
\pgfpathlineto{\pgfqpoint{1.867476in}{1.411882in}}%
\pgfpathlineto{\pgfqpoint{1.878680in}{1.411882in}}%
\pgfpathlineto{\pgfqpoint{1.878680in}{1.469526in}}%
\pgfpathlineto{\pgfqpoint{1.889885in}{1.469526in}}%
\pgfpathlineto{\pgfqpoint{1.889885in}{1.522743in}}%
\pgfpathlineto{\pgfqpoint{1.901090in}{1.522743in}}%
\pgfpathlineto{\pgfqpoint{1.901090in}{1.592076in}}%
\pgfpathlineto{\pgfqpoint{1.912294in}{1.592076in}}%
\pgfpathlineto{\pgfqpoint{1.912294in}{1.657583in}}%
\pgfpathlineto{\pgfqpoint{1.923499in}{1.657583in}}%
\pgfpathlineto{\pgfqpoint{1.923499in}{1.725648in}}%
\pgfpathlineto{\pgfqpoint{1.934704in}{1.725648in}}%
\pgfpathlineto{\pgfqpoint{1.934704in}{1.796043in}}%
\pgfpathlineto{\pgfqpoint{1.945909in}{1.796043in}}%
\pgfpathlineto{\pgfqpoint{1.945909in}{1.864385in}}%
\pgfpathlineto{\pgfqpoint{1.957113in}{1.864385in}}%
\pgfpathlineto{\pgfqpoint{1.957113in}{1.920161in}}%
\pgfpathlineto{\pgfqpoint{1.968318in}{1.920161in}}%
\pgfpathlineto{\pgfqpoint{1.968318in}{1.954655in}}%
\pgfpathlineto{\pgfqpoint{1.979523in}{1.954655in}}%
\pgfpathlineto{\pgfqpoint{1.979523in}{1.989772in}}%
\pgfpathlineto{\pgfqpoint{1.990727in}{1.989772in}}%
\pgfpathlineto{\pgfqpoint{1.990727in}{2.005497in}}%
\pgfpathlineto{\pgfqpoint{2.001932in}{2.005497in}}%
\pgfpathlineto{\pgfqpoint{2.001932in}{1.982370in}}%
\pgfpathlineto{\pgfqpoint{2.013137in}{1.982370in}}%
\pgfpathlineto{\pgfqpoint{2.013137in}{1.952142in}}%
\pgfpathlineto{\pgfqpoint{2.024342in}{1.952142in}}%
\pgfpathlineto{\pgfqpoint{2.024342in}{1.894729in}}%
\pgfpathlineto{\pgfqpoint{2.035546in}{1.894729in}}%
\pgfpathlineto{\pgfqpoint{2.035546in}{1.834064in}}%
\pgfpathlineto{\pgfqpoint{2.046751in}{1.834064in}}%
\pgfpathlineto{\pgfqpoint{2.046751in}{1.754216in}}%
\pgfpathlineto{\pgfqpoint{2.057956in}{1.754216in}}%
\pgfpathlineto{\pgfqpoint{2.057956in}{1.692492in}}%
\pgfpathlineto{\pgfqpoint{2.069160in}{1.692492in}}%
\pgfpathlineto{\pgfqpoint{2.069160in}{1.593667in}}%
\pgfpathlineto{\pgfqpoint{2.080365in}{1.593667in}}%
\pgfpathlineto{\pgfqpoint{2.080365in}{1.511606in}}%
\pgfpathlineto{\pgfqpoint{2.091570in}{1.511606in}}%
\pgfpathlineto{\pgfqpoint{2.091570in}{1.434963in}}%
\pgfpathlineto{\pgfqpoint{2.102775in}{1.434963in}}%
\pgfpathlineto{\pgfqpoint{2.102775in}{1.362793in}}%
\pgfpathlineto{\pgfqpoint{2.113979in}{1.362793in}}%
\pgfpathlineto{\pgfqpoint{2.113979in}{1.302959in}}%
\pgfpathlineto{\pgfqpoint{2.125184in}{1.302959in}}%
\pgfpathlineto{\pgfqpoint{2.125184in}{1.244462in}}%
\pgfpathlineto{\pgfqpoint{2.136389in}{1.244462in}}%
\pgfpathlineto{\pgfqpoint{2.136389in}{1.195949in}}%
\pgfpathlineto{\pgfqpoint{2.147593in}{1.195949in}}%
\pgfpathlineto{\pgfqpoint{2.147593in}{1.150964in}}%
\pgfpathlineto{\pgfqpoint{2.158798in}{1.150964in}}%
\pgfpathlineto{\pgfqpoint{2.158798in}{1.112988in}}%
\pgfpathlineto{\pgfqpoint{2.170003in}{1.112988in}}%
\pgfpathlineto{\pgfqpoint{2.170003in}{1.083913in}}%
\pgfpathlineto{\pgfqpoint{2.181208in}{1.083913in}}%
\pgfpathlineto{\pgfqpoint{2.181208in}{1.057305in}}%
\pgfpathlineto{\pgfqpoint{2.192412in}{1.057305in}}%
\pgfpathlineto{\pgfqpoint{2.192412in}{1.036092in}}%
\pgfpathlineto{\pgfqpoint{2.203617in}{1.036092in}}%
\pgfpathlineto{\pgfqpoint{2.203617in}{1.018360in}}%
\pgfpathlineto{\pgfqpoint{2.214822in}{1.018360in}}%
\pgfpathlineto{\pgfqpoint{2.214822in}{1.009945in}}%
\pgfpathlineto{\pgfqpoint{2.226026in}{1.009945in}}%
\pgfpathlineto{\pgfqpoint{2.226026in}{0.994727in}}%
\pgfpathlineto{\pgfqpoint{2.237231in}{0.994727in}}%
\pgfpathlineto{\pgfqpoint{2.237231in}{0.987625in}}%
\pgfpathlineto{\pgfqpoint{2.248436in}{0.987625in}}%
\pgfpathlineto{\pgfqpoint{2.248436in}{0.975358in}}%
\pgfpathlineto{\pgfqpoint{2.259641in}{0.975358in}}%
\pgfpathlineto{\pgfqpoint{2.259641in}{0.967680in}}%
\pgfpathlineto{\pgfqpoint{2.270845in}{0.967680in}}%
\pgfpathlineto{\pgfqpoint{2.270845in}{0.959795in}}%
\pgfpathlineto{\pgfqpoint{2.282050in}{0.959795in}}%
\pgfpathlineto{\pgfqpoint{2.282050in}{0.947459in}}%
\pgfpathlineto{\pgfqpoint{2.293255in}{0.947459in}}%
\pgfpathlineto{\pgfqpoint{2.293255in}{0.951355in}}%
\pgfpathlineto{\pgfqpoint{2.304460in}{0.951355in}}%
\pgfpathlineto{\pgfqpoint{2.304460in}{0.939435in}}%
\pgfpathlineto{\pgfqpoint{2.315664in}{0.939435in}}%
\pgfpathlineto{\pgfqpoint{2.315664in}{0.936991in}}%
\pgfpathlineto{\pgfqpoint{2.326869in}{0.936991in}}%
\pgfpathlineto{\pgfqpoint{2.326869in}{0.931987in}}%
\pgfpathlineto{\pgfqpoint{2.338074in}{0.931987in}}%
\pgfpathlineto{\pgfqpoint{2.338074in}{0.933878in}}%
\pgfpathlineto{\pgfqpoint{2.349278in}{0.933878in}}%
\pgfpathlineto{\pgfqpoint{2.349278in}{0.928413in}}%
\pgfpathlineto{\pgfqpoint{2.360483in}{0.928413in}}%
\pgfpathlineto{\pgfqpoint{2.360483in}{0.915732in}}%
\pgfpathlineto{\pgfqpoint{2.371688in}{0.915732in}}%
\pgfpathlineto{\pgfqpoint{2.371688in}{0.921611in}}%
\pgfpathlineto{\pgfqpoint{2.382893in}{0.921611in}}%
\pgfpathlineto{\pgfqpoint{2.382893in}{0.910567in}}%
\pgfpathlineto{\pgfqpoint{2.394097in}{0.910567in}}%
\pgfpathlineto{\pgfqpoint{2.394097in}{0.908953in}}%
\pgfpathlineto{\pgfqpoint{2.405302in}{0.908953in}}%
\pgfpathlineto{\pgfqpoint{2.405302in}{0.904111in}}%
\pgfpathlineto{\pgfqpoint{2.416507in}{0.904111in}}%
\pgfpathlineto{\pgfqpoint{2.416507in}{0.900860in}}%
\pgfpathlineto{\pgfqpoint{2.427711in}{0.900860in}}%
\pgfpathlineto{\pgfqpoint{2.427711in}{0.893181in}}%
\pgfpathlineto{\pgfqpoint{2.438916in}{0.893181in}}%
\pgfpathlineto{\pgfqpoint{2.438916in}{0.896917in}}%
\pgfpathlineto{\pgfqpoint{2.450121in}{0.896917in}}%
\pgfpathlineto{\pgfqpoint{2.450121in}{0.893366in}}%
\pgfpathlineto{\pgfqpoint{2.461326in}{0.893366in}}%
\pgfpathlineto{\pgfqpoint{2.461326in}{0.882667in}}%
\pgfpathlineto{\pgfqpoint{2.472530in}{0.882667in}}%
\pgfpathlineto{\pgfqpoint{2.472530in}{0.879693in}}%
\pgfpathlineto{\pgfqpoint{2.483735in}{0.879693in}}%
\pgfpathlineto{\pgfqpoint{2.483735in}{0.878955in}}%
\pgfpathlineto{\pgfqpoint{2.494940in}{0.878955in}}%
\pgfpathlineto{\pgfqpoint{2.494940in}{0.868671in}}%
\pgfpathlineto{\pgfqpoint{2.506144in}{0.868671in}}%
\pgfpathlineto{\pgfqpoint{2.506144in}{0.866320in}}%
\pgfpathlineto{\pgfqpoint{2.517349in}{0.866320in}}%
\pgfpathlineto{\pgfqpoint{2.517349in}{0.868671in}}%
\pgfpathlineto{\pgfqpoint{2.528554in}{0.868671in}}%
\pgfpathlineto{\pgfqpoint{2.528554in}{0.872476in}}%
\pgfpathlineto{\pgfqpoint{2.539759in}{0.872476in}}%
\pgfpathlineto{\pgfqpoint{2.539759in}{0.862561in}}%
\pgfpathlineto{\pgfqpoint{2.550963in}{0.862561in}}%
\pgfpathlineto{\pgfqpoint{2.550963in}{0.862930in}}%
\pgfpathlineto{\pgfqpoint{2.562168in}{0.862930in}}%
\pgfpathlineto{\pgfqpoint{2.562168in}{0.858480in}}%
\pgfpathlineto{\pgfqpoint{2.573373in}{0.858480in}}%
\pgfpathlineto{\pgfqpoint{2.573373in}{0.851009in}}%
\pgfpathlineto{\pgfqpoint{2.584577in}{0.851009in}}%
\pgfpathlineto{\pgfqpoint{2.584577in}{0.850917in}}%
\pgfpathlineto{\pgfqpoint{2.595782in}{0.850917in}}%
\pgfpathlineto{\pgfqpoint{2.595782in}{0.854537in}}%
\pgfpathlineto{\pgfqpoint{2.606987in}{0.854537in}}%
\pgfpathlineto{\pgfqpoint{2.606987in}{0.849211in}}%
\pgfpathlineto{\pgfqpoint{2.618192in}{0.849211in}}%
\pgfpathlineto{\pgfqpoint{2.618192in}{0.850202in}}%
\pgfpathlineto{\pgfqpoint{2.629396in}{0.850202in}}%
\pgfpathlineto{\pgfqpoint{2.629396in}{0.849787in}}%
\pgfpathlineto{\pgfqpoint{2.640601in}{0.849787in}}%
\pgfpathlineto{\pgfqpoint{2.640601in}{0.843008in}}%
\pgfpathlineto{\pgfqpoint{2.651806in}{0.843008in}}%
\pgfpathlineto{\pgfqpoint{2.651806in}{0.842363in}}%
\pgfpathlineto{\pgfqpoint{2.663010in}{0.842363in}}%
\pgfpathlineto{\pgfqpoint{2.663010in}{0.840057in}}%
\pgfpathlineto{\pgfqpoint{2.674215in}{0.840057in}}%
\pgfpathlineto{\pgfqpoint{2.674215in}{0.839020in}}%
\pgfpathlineto{\pgfqpoint{2.685420in}{0.839020in}}%
\pgfpathlineto{\pgfqpoint{2.685420in}{0.837590in}}%
\pgfpathlineto{\pgfqpoint{2.696625in}{0.837590in}}%
\pgfpathlineto{\pgfqpoint{2.696625in}{0.838927in}}%
\pgfpathlineto{\pgfqpoint{2.707829in}{0.838927in}}%
\pgfpathlineto{\pgfqpoint{2.707829in}{0.839158in}}%
\pgfpathlineto{\pgfqpoint{2.719034in}{0.839158in}}%
\pgfpathlineto{\pgfqpoint{2.719034in}{0.834062in}}%
\pgfpathlineto{\pgfqpoint{2.730239in}{0.834062in}}%
\pgfpathlineto{\pgfqpoint{2.730239in}{0.832978in}}%
\pgfpathlineto{\pgfqpoint{2.741443in}{0.832978in}}%
\pgfpathlineto{\pgfqpoint{2.741443in}{0.831341in}}%
\pgfpathlineto{\pgfqpoint{2.752648in}{0.831341in}}%
\pgfpathlineto{\pgfqpoint{2.752648in}{0.828229in}}%
\pgfpathlineto{\pgfqpoint{2.763853in}{0.828229in}}%
\pgfpathlineto{\pgfqpoint{2.763853in}{0.826430in}}%
\pgfpathlineto{\pgfqpoint{2.775058in}{0.826430in}}%
\pgfpathlineto{\pgfqpoint{2.775058in}{0.825439in}}%
\pgfpathlineto{\pgfqpoint{2.786262in}{0.825439in}}%
\pgfpathlineto{\pgfqpoint{2.786262in}{0.825485in}}%
\pgfpathlineto{\pgfqpoint{2.797467in}{0.825485in}}%
\pgfpathlineto{\pgfqpoint{2.797467in}{0.820228in}}%
\pgfpathlineto{\pgfqpoint{2.808672in}{0.820228in}}%
\pgfpathlineto{\pgfqpoint{2.808672in}{0.818498in}}%
\pgfpathlineto{\pgfqpoint{2.819876in}{0.818498in}}%
\pgfpathlineto{\pgfqpoint{2.819876in}{0.818360in}}%
\pgfpathlineto{\pgfqpoint{2.831081in}{0.818360in}}%
\pgfpathlineto{\pgfqpoint{2.831081in}{0.810682in}}%
\pgfpathlineto{\pgfqpoint{2.842286in}{0.810682in}}%
\pgfpathlineto{\pgfqpoint{2.842286in}{0.810982in}}%
\pgfpathlineto{\pgfqpoint{2.853491in}{0.810982in}}%
\pgfpathlineto{\pgfqpoint{2.853491in}{0.811212in}}%
\pgfpathlineto{\pgfqpoint{2.864695in}{0.811212in}}%
\pgfpathlineto{\pgfqpoint{2.864695in}{0.808791in}}%
\pgfpathlineto{\pgfqpoint{2.875900in}{0.808791in}}%
\pgfpathlineto{\pgfqpoint{2.875900in}{0.804526in}}%
\pgfpathlineto{\pgfqpoint{2.887105in}{0.804526in}}%
\pgfpathlineto{\pgfqpoint{2.887105in}{0.805701in}}%
\pgfpathlineto{\pgfqpoint{2.898309in}{0.805701in}}%
\pgfpathlineto{\pgfqpoint{2.898309in}{0.803834in}}%
\pgfpathlineto{\pgfqpoint{2.909514in}{0.803834in}}%
\pgfpathlineto{\pgfqpoint{2.909514in}{0.800283in}}%
\pgfpathlineto{\pgfqpoint{2.920719in}{0.800283in}}%
\pgfpathlineto{\pgfqpoint{2.920719in}{0.803234in}}%
\pgfpathlineto{\pgfqpoint{2.931924in}{0.803234in}}%
\pgfpathlineto{\pgfqpoint{2.931924in}{0.794334in}}%
\pgfpathlineto{\pgfqpoint{2.943128in}{0.794334in}}%
\pgfpathlineto{\pgfqpoint{2.943128in}{0.794057in}}%
\pgfpathlineto{\pgfqpoint{2.954333in}{0.794057in}}%
\pgfpathlineto{\pgfqpoint{2.954333in}{0.787901in}}%
\pgfpathlineto{\pgfqpoint{2.965538in}{0.787901in}}%
\pgfpathlineto{\pgfqpoint{2.965538in}{0.795418in}}%
\pgfpathlineto{\pgfqpoint{2.976742in}{0.795418in}}%
\pgfpathlineto{\pgfqpoint{2.976742in}{0.788893in}}%
\pgfpathlineto{\pgfqpoint{2.987947in}{0.788893in}}%
\pgfpathlineto{\pgfqpoint{2.987947in}{0.782621in}}%
\pgfpathlineto{\pgfqpoint{2.999152in}{0.782621in}}%
\pgfpathlineto{\pgfqpoint{2.999152in}{0.781745in}}%
\pgfpathlineto{\pgfqpoint{3.010357in}{0.781745in}}%
\pgfpathlineto{\pgfqpoint{3.010357in}{0.780799in}}%
\pgfpathlineto{\pgfqpoint{3.021561in}{0.780799in}}%
\pgfpathlineto{\pgfqpoint{3.021561in}{0.778009in}}%
\pgfpathlineto{\pgfqpoint{3.032766in}{0.778009in}}%
\pgfpathlineto{\pgfqpoint{3.032766in}{0.785987in}}%
\pgfpathlineto{\pgfqpoint{3.043971in}{0.785987in}}%
\pgfpathlineto{\pgfqpoint{3.043971in}{0.779301in}}%
\pgfpathlineto{\pgfqpoint{3.055175in}{0.779301in}}%
\pgfpathlineto{\pgfqpoint{3.055175in}{0.774366in}}%
\pgfpathlineto{\pgfqpoint{3.066380in}{0.774366in}}%
\pgfpathlineto{\pgfqpoint{3.066380in}{0.778033in}}%
\pgfpathlineto{\pgfqpoint{3.077585in}{0.778033in}}%
\pgfpathlineto{\pgfqpoint{3.077585in}{0.767495in}}%
\pgfpathlineto{\pgfqpoint{3.088790in}{0.767495in}}%
\pgfpathlineto{\pgfqpoint{3.088790in}{0.768602in}}%
\pgfpathlineto{\pgfqpoint{3.099994in}{0.768602in}}%
\pgfpathlineto{\pgfqpoint{3.099994in}{0.766642in}}%
\pgfpathlineto{\pgfqpoint{3.111199in}{0.766642in}}%
\pgfpathlineto{\pgfqpoint{3.111199in}{0.764774in}}%
\pgfpathlineto{\pgfqpoint{3.122404in}{0.764774in}}%
\pgfpathlineto{\pgfqpoint{3.122404in}{0.763783in}}%
\pgfpathlineto{\pgfqpoint{3.133608in}{0.763783in}}%
\pgfpathlineto{\pgfqpoint{3.133608in}{0.759241in}}%
\pgfpathlineto{\pgfqpoint{3.144813in}{0.759241in}}%
\pgfpathlineto{\pgfqpoint{3.144813in}{0.762446in}}%
\pgfpathlineto{\pgfqpoint{3.156018in}{0.762446in}}%
\pgfpathlineto{\pgfqpoint{3.156018in}{0.756681in}}%
\pgfpathlineto{\pgfqpoint{3.167223in}{0.756681in}}%
\pgfpathlineto{\pgfqpoint{3.167223in}{0.763368in}}%
\pgfpathlineto{\pgfqpoint{3.178427in}{0.763368in}}%
\pgfpathlineto{\pgfqpoint{3.178427in}{0.755782in}}%
\pgfpathlineto{\pgfqpoint{3.189632in}{0.755782in}}%
\pgfpathlineto{\pgfqpoint{3.189632in}{0.755344in}}%
\pgfpathlineto{\pgfqpoint{3.200837in}{0.755344in}}%
\pgfpathlineto{\pgfqpoint{3.200837in}{0.752416in}}%
\pgfpathlineto{\pgfqpoint{3.212041in}{0.752416in}}%
\pgfpathlineto{\pgfqpoint{3.212041in}{0.746813in}}%
\pgfpathlineto{\pgfqpoint{3.223246in}{0.746813in}}%
\pgfpathlineto{\pgfqpoint{3.223246in}{0.747919in}}%
\pgfpathlineto{\pgfqpoint{3.234451in}{0.747919in}}%
\pgfpathlineto{\pgfqpoint{3.234451in}{0.745683in}}%
\pgfpathlineto{\pgfqpoint{3.245656in}{0.745683in}}%
\pgfpathlineto{\pgfqpoint{3.245656in}{0.740034in}}%
\pgfpathlineto{\pgfqpoint{3.256860in}{0.740034in}}%
\pgfpathlineto{\pgfqpoint{3.256860in}{0.746974in}}%
\pgfpathlineto{\pgfqpoint{3.268065in}{0.746974in}}%
\pgfpathlineto{\pgfqpoint{3.268065in}{0.743008in}}%
\pgfpathlineto{\pgfqpoint{3.279270in}{0.743008in}}%
\pgfpathlineto{\pgfqpoint{3.279270in}{0.741463in}}%
\pgfpathlineto{\pgfqpoint{3.290475in}{0.741463in}}%
\pgfpathlineto{\pgfqpoint{3.290475in}{0.741925in}}%
\pgfpathlineto{\pgfqpoint{3.301679in}{0.741925in}}%
\pgfpathlineto{\pgfqpoint{3.301679in}{0.736252in}}%
\pgfpathlineto{\pgfqpoint{3.312884in}{0.736252in}}%
\pgfpathlineto{\pgfqpoint{3.312884in}{0.737497in}}%
\pgfpathlineto{\pgfqpoint{3.324089in}{0.737497in}}%
\pgfpathlineto{\pgfqpoint{3.324089in}{0.730396in}}%
\pgfpathlineto{\pgfqpoint{3.335293in}{0.730396in}}%
\pgfpathlineto{\pgfqpoint{3.335293in}{0.729381in}}%
\pgfpathlineto{\pgfqpoint{3.346498in}{0.729381in}}%
\pgfpathlineto{\pgfqpoint{3.346498in}{0.730165in}}%
\pgfpathlineto{\pgfqpoint{3.357703in}{0.730165in}}%
\pgfpathlineto{\pgfqpoint{3.357703in}{0.730696in}}%
\pgfpathlineto{\pgfqpoint{3.368908in}{0.730696in}}%
\pgfpathlineto{\pgfqpoint{3.368908in}{0.731134in}}%
\pgfpathlineto{\pgfqpoint{3.380112in}{0.731134in}}%
\pgfpathlineto{\pgfqpoint{3.380112in}{0.726430in}}%
\pgfpathlineto{\pgfqpoint{3.391317in}{0.726430in}}%
\pgfpathlineto{\pgfqpoint{3.391317in}{0.727398in}}%
\pgfpathlineto{\pgfqpoint{3.402522in}{0.727398in}}%
\pgfpathlineto{\pgfqpoint{3.402522in}{0.721957in}}%
\pgfpathlineto{\pgfqpoint{3.413726in}{0.721957in}}%
\pgfpathlineto{\pgfqpoint{3.413726in}{0.723087in}}%
\pgfpathlineto{\pgfqpoint{3.424931in}{0.723087in}}%
\pgfpathlineto{\pgfqpoint{3.424931in}{0.719052in}}%
\pgfpathlineto{\pgfqpoint{3.436136in}{0.719052in}}%
\pgfpathlineto{\pgfqpoint{3.436136in}{0.718660in}}%
\pgfpathlineto{\pgfqpoint{3.447341in}{0.718660in}}%
\pgfpathlineto{\pgfqpoint{3.447341in}{0.717207in}}%
\pgfpathlineto{\pgfqpoint{3.458545in}{0.717207in}}%
\pgfpathlineto{\pgfqpoint{3.458545in}{0.715985in}}%
\pgfpathlineto{\pgfqpoint{3.469750in}{0.715985in}}%
\pgfpathlineto{\pgfqpoint{3.469750in}{0.440955in}}%
\pgfpathlineto{\pgfqpoint{3.458545in}{0.440955in}}%
\pgfpathlineto{\pgfqpoint{3.458545in}{0.440955in}}%
\pgfpathlineto{\pgfqpoint{3.447341in}{0.440955in}}%
\pgfpathlineto{\pgfqpoint{3.447341in}{0.440955in}}%
\pgfpathlineto{\pgfqpoint{3.436136in}{0.440955in}}%
\pgfpathlineto{\pgfqpoint{3.436136in}{0.440955in}}%
\pgfpathlineto{\pgfqpoint{3.424931in}{0.440955in}}%
\pgfpathlineto{\pgfqpoint{3.424931in}{0.440955in}}%
\pgfpathlineto{\pgfqpoint{3.413726in}{0.440955in}}%
\pgfpathlineto{\pgfqpoint{3.413726in}{0.440955in}}%
\pgfpathlineto{\pgfqpoint{3.402522in}{0.440955in}}%
\pgfpathlineto{\pgfqpoint{3.402522in}{0.440955in}}%
\pgfpathlineto{\pgfqpoint{3.391317in}{0.440955in}}%
\pgfpathlineto{\pgfqpoint{3.391317in}{0.440955in}}%
\pgfpathlineto{\pgfqpoint{3.380112in}{0.440955in}}%
\pgfpathlineto{\pgfqpoint{3.380112in}{0.440955in}}%
\pgfpathlineto{\pgfqpoint{3.368908in}{0.440955in}}%
\pgfpathlineto{\pgfqpoint{3.368908in}{0.440955in}}%
\pgfpathlineto{\pgfqpoint{3.357703in}{0.440955in}}%
\pgfpathlineto{\pgfqpoint{3.357703in}{0.440955in}}%
\pgfpathlineto{\pgfqpoint{3.346498in}{0.440955in}}%
\pgfpathlineto{\pgfqpoint{3.346498in}{0.440955in}}%
\pgfpathlineto{\pgfqpoint{3.335293in}{0.440955in}}%
\pgfpathlineto{\pgfqpoint{3.335293in}{0.440955in}}%
\pgfpathlineto{\pgfqpoint{3.324089in}{0.440955in}}%
\pgfpathlineto{\pgfqpoint{3.324089in}{0.440955in}}%
\pgfpathlineto{\pgfqpoint{3.312884in}{0.440955in}}%
\pgfpathlineto{\pgfqpoint{3.312884in}{0.440955in}}%
\pgfpathlineto{\pgfqpoint{3.301679in}{0.440955in}}%
\pgfpathlineto{\pgfqpoint{3.301679in}{0.440955in}}%
\pgfpathlineto{\pgfqpoint{3.290475in}{0.440955in}}%
\pgfpathlineto{\pgfqpoint{3.290475in}{0.440955in}}%
\pgfpathlineto{\pgfqpoint{3.279270in}{0.440955in}}%
\pgfpathlineto{\pgfqpoint{3.279270in}{0.440955in}}%
\pgfpathlineto{\pgfqpoint{3.268065in}{0.440955in}}%
\pgfpathlineto{\pgfqpoint{3.268065in}{0.440955in}}%
\pgfpathlineto{\pgfqpoint{3.256860in}{0.440955in}}%
\pgfpathlineto{\pgfqpoint{3.256860in}{0.440955in}}%
\pgfpathlineto{\pgfqpoint{3.245656in}{0.440955in}}%
\pgfpathlineto{\pgfqpoint{3.245656in}{0.440955in}}%
\pgfpathlineto{\pgfqpoint{3.234451in}{0.440955in}}%
\pgfpathlineto{\pgfqpoint{3.234451in}{0.440955in}}%
\pgfpathlineto{\pgfqpoint{3.223246in}{0.440955in}}%
\pgfpathlineto{\pgfqpoint{3.223246in}{0.440955in}}%
\pgfpathlineto{\pgfqpoint{3.212041in}{0.440955in}}%
\pgfpathlineto{\pgfqpoint{3.212041in}{0.440955in}}%
\pgfpathlineto{\pgfqpoint{3.200837in}{0.440955in}}%
\pgfpathlineto{\pgfqpoint{3.200837in}{0.440955in}}%
\pgfpathlineto{\pgfqpoint{3.189632in}{0.440955in}}%
\pgfpathlineto{\pgfqpoint{3.189632in}{0.440955in}}%
\pgfpathlineto{\pgfqpoint{3.178427in}{0.440955in}}%
\pgfpathlineto{\pgfqpoint{3.178427in}{0.440955in}}%
\pgfpathlineto{\pgfqpoint{3.167223in}{0.440955in}}%
\pgfpathlineto{\pgfqpoint{3.167223in}{0.440955in}}%
\pgfpathlineto{\pgfqpoint{3.156018in}{0.440955in}}%
\pgfpathlineto{\pgfqpoint{3.156018in}{0.440955in}}%
\pgfpathlineto{\pgfqpoint{3.144813in}{0.440955in}}%
\pgfpathlineto{\pgfqpoint{3.144813in}{0.440955in}}%
\pgfpathlineto{\pgfqpoint{3.133608in}{0.440955in}}%
\pgfpathlineto{\pgfqpoint{3.133608in}{0.440955in}}%
\pgfpathlineto{\pgfqpoint{3.122404in}{0.440955in}}%
\pgfpathlineto{\pgfqpoint{3.122404in}{0.440955in}}%
\pgfpathlineto{\pgfqpoint{3.111199in}{0.440955in}}%
\pgfpathlineto{\pgfqpoint{3.111199in}{0.440955in}}%
\pgfpathlineto{\pgfqpoint{3.099994in}{0.440955in}}%
\pgfpathlineto{\pgfqpoint{3.099994in}{0.440955in}}%
\pgfpathlineto{\pgfqpoint{3.088790in}{0.440955in}}%
\pgfpathlineto{\pgfqpoint{3.088790in}{0.440955in}}%
\pgfpathlineto{\pgfqpoint{3.077585in}{0.440955in}}%
\pgfpathlineto{\pgfqpoint{3.077585in}{0.440955in}}%
\pgfpathlineto{\pgfqpoint{3.066380in}{0.440955in}}%
\pgfpathlineto{\pgfqpoint{3.066380in}{0.440955in}}%
\pgfpathlineto{\pgfqpoint{3.055175in}{0.440955in}}%
\pgfpathlineto{\pgfqpoint{3.055175in}{0.440955in}}%
\pgfpathlineto{\pgfqpoint{3.043971in}{0.440955in}}%
\pgfpathlineto{\pgfqpoint{3.043971in}{0.440955in}}%
\pgfpathlineto{\pgfqpoint{3.032766in}{0.440955in}}%
\pgfpathlineto{\pgfqpoint{3.032766in}{0.440955in}}%
\pgfpathlineto{\pgfqpoint{3.021561in}{0.440955in}}%
\pgfpathlineto{\pgfqpoint{3.021561in}{0.440955in}}%
\pgfpathlineto{\pgfqpoint{3.010357in}{0.440955in}}%
\pgfpathlineto{\pgfqpoint{3.010357in}{0.440955in}}%
\pgfpathlineto{\pgfqpoint{2.999152in}{0.440955in}}%
\pgfpathlineto{\pgfqpoint{2.999152in}{0.440955in}}%
\pgfpathlineto{\pgfqpoint{2.987947in}{0.440955in}}%
\pgfpathlineto{\pgfqpoint{2.987947in}{0.440955in}}%
\pgfpathlineto{\pgfqpoint{2.976742in}{0.440955in}}%
\pgfpathlineto{\pgfqpoint{2.976742in}{0.440955in}}%
\pgfpathlineto{\pgfqpoint{2.965538in}{0.440955in}}%
\pgfpathlineto{\pgfqpoint{2.965538in}{0.440955in}}%
\pgfpathlineto{\pgfqpoint{2.954333in}{0.440955in}}%
\pgfpathlineto{\pgfqpoint{2.954333in}{0.440955in}}%
\pgfpathlineto{\pgfqpoint{2.943128in}{0.440955in}}%
\pgfpathlineto{\pgfqpoint{2.943128in}{0.440955in}}%
\pgfpathlineto{\pgfqpoint{2.931924in}{0.440955in}}%
\pgfpathlineto{\pgfqpoint{2.931924in}{0.440955in}}%
\pgfpathlineto{\pgfqpoint{2.920719in}{0.440955in}}%
\pgfpathlineto{\pgfqpoint{2.920719in}{0.440955in}}%
\pgfpathlineto{\pgfqpoint{2.909514in}{0.440955in}}%
\pgfpathlineto{\pgfqpoint{2.909514in}{0.440955in}}%
\pgfpathlineto{\pgfqpoint{2.898309in}{0.440955in}}%
\pgfpathlineto{\pgfqpoint{2.898309in}{0.440955in}}%
\pgfpathlineto{\pgfqpoint{2.887105in}{0.440955in}}%
\pgfpathlineto{\pgfqpoint{2.887105in}{0.440955in}}%
\pgfpathlineto{\pgfqpoint{2.875900in}{0.440955in}}%
\pgfpathlineto{\pgfqpoint{2.875900in}{0.440955in}}%
\pgfpathlineto{\pgfqpoint{2.864695in}{0.440955in}}%
\pgfpathlineto{\pgfqpoint{2.864695in}{0.440955in}}%
\pgfpathlineto{\pgfqpoint{2.853491in}{0.440955in}}%
\pgfpathlineto{\pgfqpoint{2.853491in}{0.440955in}}%
\pgfpathlineto{\pgfqpoint{2.842286in}{0.440955in}}%
\pgfpathlineto{\pgfqpoint{2.842286in}{0.440955in}}%
\pgfpathlineto{\pgfqpoint{2.831081in}{0.440955in}}%
\pgfpathlineto{\pgfqpoint{2.831081in}{0.440955in}}%
\pgfpathlineto{\pgfqpoint{2.819876in}{0.440955in}}%
\pgfpathlineto{\pgfqpoint{2.819876in}{0.440955in}}%
\pgfpathlineto{\pgfqpoint{2.808672in}{0.440955in}}%
\pgfpathlineto{\pgfqpoint{2.808672in}{0.440955in}}%
\pgfpathlineto{\pgfqpoint{2.797467in}{0.440955in}}%
\pgfpathlineto{\pgfqpoint{2.797467in}{0.440955in}}%
\pgfpathlineto{\pgfqpoint{2.786262in}{0.440955in}}%
\pgfpathlineto{\pgfqpoint{2.786262in}{0.440955in}}%
\pgfpathlineto{\pgfqpoint{2.775058in}{0.440955in}}%
\pgfpathlineto{\pgfqpoint{2.775058in}{0.440955in}}%
\pgfpathlineto{\pgfqpoint{2.763853in}{0.440955in}}%
\pgfpathlineto{\pgfqpoint{2.763853in}{0.440955in}}%
\pgfpathlineto{\pgfqpoint{2.752648in}{0.440955in}}%
\pgfpathlineto{\pgfqpoint{2.752648in}{0.440955in}}%
\pgfpathlineto{\pgfqpoint{2.741443in}{0.440955in}}%
\pgfpathlineto{\pgfqpoint{2.741443in}{0.440955in}}%
\pgfpathlineto{\pgfqpoint{2.730239in}{0.440955in}}%
\pgfpathlineto{\pgfqpoint{2.730239in}{0.440955in}}%
\pgfpathlineto{\pgfqpoint{2.719034in}{0.440955in}}%
\pgfpathlineto{\pgfqpoint{2.719034in}{0.440955in}}%
\pgfpathlineto{\pgfqpoint{2.707829in}{0.440955in}}%
\pgfpathlineto{\pgfqpoint{2.707829in}{0.440955in}}%
\pgfpathlineto{\pgfqpoint{2.696625in}{0.440955in}}%
\pgfpathlineto{\pgfqpoint{2.696625in}{0.440955in}}%
\pgfpathlineto{\pgfqpoint{2.685420in}{0.440955in}}%
\pgfpathlineto{\pgfqpoint{2.685420in}{0.440955in}}%
\pgfpathlineto{\pgfqpoint{2.674215in}{0.440955in}}%
\pgfpathlineto{\pgfqpoint{2.674215in}{0.440955in}}%
\pgfpathlineto{\pgfqpoint{2.663010in}{0.440955in}}%
\pgfpathlineto{\pgfqpoint{2.663010in}{0.440955in}}%
\pgfpathlineto{\pgfqpoint{2.651806in}{0.440955in}}%
\pgfpathlineto{\pgfqpoint{2.651806in}{0.440955in}}%
\pgfpathlineto{\pgfqpoint{2.640601in}{0.440955in}}%
\pgfpathlineto{\pgfqpoint{2.640601in}{0.440955in}}%
\pgfpathlineto{\pgfqpoint{2.629396in}{0.440955in}}%
\pgfpathlineto{\pgfqpoint{2.629396in}{0.440955in}}%
\pgfpathlineto{\pgfqpoint{2.618192in}{0.440955in}}%
\pgfpathlineto{\pgfqpoint{2.618192in}{0.440955in}}%
\pgfpathlineto{\pgfqpoint{2.606987in}{0.440955in}}%
\pgfpathlineto{\pgfqpoint{2.606987in}{0.440955in}}%
\pgfpathlineto{\pgfqpoint{2.595782in}{0.440955in}}%
\pgfpathlineto{\pgfqpoint{2.595782in}{0.440955in}}%
\pgfpathlineto{\pgfqpoint{2.584577in}{0.440955in}}%
\pgfpathlineto{\pgfqpoint{2.584577in}{0.440955in}}%
\pgfpathlineto{\pgfqpoint{2.573373in}{0.440955in}}%
\pgfpathlineto{\pgfqpoint{2.573373in}{0.440955in}}%
\pgfpathlineto{\pgfqpoint{2.562168in}{0.440955in}}%
\pgfpathlineto{\pgfqpoint{2.562168in}{0.440955in}}%
\pgfpathlineto{\pgfqpoint{2.550963in}{0.440955in}}%
\pgfpathlineto{\pgfqpoint{2.550963in}{0.440955in}}%
\pgfpathlineto{\pgfqpoint{2.539759in}{0.440955in}}%
\pgfpathlineto{\pgfqpoint{2.539759in}{0.440955in}}%
\pgfpathlineto{\pgfqpoint{2.528554in}{0.440955in}}%
\pgfpathlineto{\pgfqpoint{2.528554in}{0.440955in}}%
\pgfpathlineto{\pgfqpoint{2.517349in}{0.440955in}}%
\pgfpathlineto{\pgfqpoint{2.517349in}{0.440955in}}%
\pgfpathlineto{\pgfqpoint{2.506144in}{0.440955in}}%
\pgfpathlineto{\pgfqpoint{2.506144in}{0.440955in}}%
\pgfpathlineto{\pgfqpoint{2.494940in}{0.440955in}}%
\pgfpathlineto{\pgfqpoint{2.494940in}{0.440955in}}%
\pgfpathlineto{\pgfqpoint{2.483735in}{0.440955in}}%
\pgfpathlineto{\pgfqpoint{2.483735in}{0.440955in}}%
\pgfpathlineto{\pgfqpoint{2.472530in}{0.440955in}}%
\pgfpathlineto{\pgfqpoint{2.472530in}{0.440955in}}%
\pgfpathlineto{\pgfqpoint{2.461326in}{0.440955in}}%
\pgfpathlineto{\pgfqpoint{2.461326in}{0.440955in}}%
\pgfpathlineto{\pgfqpoint{2.450121in}{0.440955in}}%
\pgfpathlineto{\pgfqpoint{2.450121in}{0.440955in}}%
\pgfpathlineto{\pgfqpoint{2.438916in}{0.440955in}}%
\pgfpathlineto{\pgfqpoint{2.438916in}{0.440955in}}%
\pgfpathlineto{\pgfqpoint{2.427711in}{0.440955in}}%
\pgfpathlineto{\pgfqpoint{2.427711in}{0.440955in}}%
\pgfpathlineto{\pgfqpoint{2.416507in}{0.440955in}}%
\pgfpathlineto{\pgfqpoint{2.416507in}{0.440955in}}%
\pgfpathlineto{\pgfqpoint{2.405302in}{0.440955in}}%
\pgfpathlineto{\pgfqpoint{2.405302in}{0.440955in}}%
\pgfpathlineto{\pgfqpoint{2.394097in}{0.440955in}}%
\pgfpathlineto{\pgfqpoint{2.394097in}{0.440955in}}%
\pgfpathlineto{\pgfqpoint{2.382893in}{0.440955in}}%
\pgfpathlineto{\pgfqpoint{2.382893in}{0.440955in}}%
\pgfpathlineto{\pgfqpoint{2.371688in}{0.440955in}}%
\pgfpathlineto{\pgfqpoint{2.371688in}{0.440955in}}%
\pgfpathlineto{\pgfqpoint{2.360483in}{0.440955in}}%
\pgfpathlineto{\pgfqpoint{2.360483in}{0.440955in}}%
\pgfpathlineto{\pgfqpoint{2.349278in}{0.440955in}}%
\pgfpathlineto{\pgfqpoint{2.349278in}{0.440955in}}%
\pgfpathlineto{\pgfqpoint{2.338074in}{0.440955in}}%
\pgfpathlineto{\pgfqpoint{2.338074in}{0.440955in}}%
\pgfpathlineto{\pgfqpoint{2.326869in}{0.440955in}}%
\pgfpathlineto{\pgfqpoint{2.326869in}{0.440955in}}%
\pgfpathlineto{\pgfqpoint{2.315664in}{0.440955in}}%
\pgfpathlineto{\pgfqpoint{2.315664in}{0.440955in}}%
\pgfpathlineto{\pgfqpoint{2.304460in}{0.440955in}}%
\pgfpathlineto{\pgfqpoint{2.304460in}{0.440955in}}%
\pgfpathlineto{\pgfqpoint{2.293255in}{0.440955in}}%
\pgfpathlineto{\pgfqpoint{2.293255in}{0.440955in}}%
\pgfpathlineto{\pgfqpoint{2.282050in}{0.440955in}}%
\pgfpathlineto{\pgfqpoint{2.282050in}{0.440955in}}%
\pgfpathlineto{\pgfqpoint{2.270845in}{0.440955in}}%
\pgfpathlineto{\pgfqpoint{2.270845in}{0.440955in}}%
\pgfpathlineto{\pgfqpoint{2.259641in}{0.440955in}}%
\pgfpathlineto{\pgfqpoint{2.259641in}{0.440955in}}%
\pgfpathlineto{\pgfqpoint{2.248436in}{0.440955in}}%
\pgfpathlineto{\pgfqpoint{2.248436in}{0.440955in}}%
\pgfpathlineto{\pgfqpoint{2.237231in}{0.440955in}}%
\pgfpathlineto{\pgfqpoint{2.237231in}{0.440955in}}%
\pgfpathlineto{\pgfqpoint{2.226026in}{0.440955in}}%
\pgfpathlineto{\pgfqpoint{2.226026in}{0.440955in}}%
\pgfpathlineto{\pgfqpoint{2.214822in}{0.440955in}}%
\pgfpathlineto{\pgfqpoint{2.214822in}{0.440955in}}%
\pgfpathlineto{\pgfqpoint{2.203617in}{0.440955in}}%
\pgfpathlineto{\pgfqpoint{2.203617in}{0.440955in}}%
\pgfpathlineto{\pgfqpoint{2.192412in}{0.440955in}}%
\pgfpathlineto{\pgfqpoint{2.192412in}{0.440955in}}%
\pgfpathlineto{\pgfqpoint{2.181208in}{0.440955in}}%
\pgfpathlineto{\pgfqpoint{2.181208in}{0.440955in}}%
\pgfpathlineto{\pgfqpoint{2.170003in}{0.440955in}}%
\pgfpathlineto{\pgfqpoint{2.170003in}{0.440955in}}%
\pgfpathlineto{\pgfqpoint{2.158798in}{0.440955in}}%
\pgfpathlineto{\pgfqpoint{2.158798in}{0.440955in}}%
\pgfpathlineto{\pgfqpoint{2.147593in}{0.440955in}}%
\pgfpathlineto{\pgfqpoint{2.147593in}{0.440955in}}%
\pgfpathlineto{\pgfqpoint{2.136389in}{0.440955in}}%
\pgfpathlineto{\pgfqpoint{2.136389in}{0.440955in}}%
\pgfpathlineto{\pgfqpoint{2.125184in}{0.440955in}}%
\pgfpathlineto{\pgfqpoint{2.125184in}{0.440955in}}%
\pgfpathlineto{\pgfqpoint{2.113979in}{0.440955in}}%
\pgfpathlineto{\pgfqpoint{2.113979in}{0.440955in}}%
\pgfpathlineto{\pgfqpoint{2.102775in}{0.440955in}}%
\pgfpathlineto{\pgfqpoint{2.102775in}{0.440955in}}%
\pgfpathlineto{\pgfqpoint{2.091570in}{0.440955in}}%
\pgfpathlineto{\pgfqpoint{2.091570in}{0.440955in}}%
\pgfpathlineto{\pgfqpoint{2.080365in}{0.440955in}}%
\pgfpathlineto{\pgfqpoint{2.080365in}{0.440955in}}%
\pgfpathlineto{\pgfqpoint{2.069160in}{0.440955in}}%
\pgfpathlineto{\pgfqpoint{2.069160in}{0.440955in}}%
\pgfpathlineto{\pgfqpoint{2.057956in}{0.440955in}}%
\pgfpathlineto{\pgfqpoint{2.057956in}{0.440955in}}%
\pgfpathlineto{\pgfqpoint{2.046751in}{0.440955in}}%
\pgfpathlineto{\pgfqpoint{2.046751in}{0.440955in}}%
\pgfpathlineto{\pgfqpoint{2.035546in}{0.440955in}}%
\pgfpathlineto{\pgfqpoint{2.035546in}{0.440955in}}%
\pgfpathlineto{\pgfqpoint{2.024342in}{0.440955in}}%
\pgfpathlineto{\pgfqpoint{2.024342in}{0.440955in}}%
\pgfpathlineto{\pgfqpoint{2.013137in}{0.440955in}}%
\pgfpathlineto{\pgfqpoint{2.013137in}{0.440955in}}%
\pgfpathlineto{\pgfqpoint{2.001932in}{0.440955in}}%
\pgfpathlineto{\pgfqpoint{2.001932in}{0.440955in}}%
\pgfpathlineto{\pgfqpoint{1.990727in}{0.440955in}}%
\pgfpathlineto{\pgfqpoint{1.990727in}{0.440955in}}%
\pgfpathlineto{\pgfqpoint{1.979523in}{0.440955in}}%
\pgfpathlineto{\pgfqpoint{1.979523in}{0.440955in}}%
\pgfpathlineto{\pgfqpoint{1.968318in}{0.440955in}}%
\pgfpathlineto{\pgfqpoint{1.968318in}{0.440955in}}%
\pgfpathlineto{\pgfqpoint{1.957113in}{0.440955in}}%
\pgfpathlineto{\pgfqpoint{1.957113in}{0.440955in}}%
\pgfpathlineto{\pgfqpoint{1.945909in}{0.440955in}}%
\pgfpathlineto{\pgfqpoint{1.945909in}{0.440955in}}%
\pgfpathlineto{\pgfqpoint{1.934704in}{0.440955in}}%
\pgfpathlineto{\pgfqpoint{1.934704in}{0.440955in}}%
\pgfpathlineto{\pgfqpoint{1.923499in}{0.440955in}}%
\pgfpathlineto{\pgfqpoint{1.923499in}{0.440955in}}%
\pgfpathlineto{\pgfqpoint{1.912294in}{0.440955in}}%
\pgfpathlineto{\pgfqpoint{1.912294in}{0.440955in}}%
\pgfpathlineto{\pgfqpoint{1.901090in}{0.440955in}}%
\pgfpathlineto{\pgfqpoint{1.901090in}{0.440955in}}%
\pgfpathlineto{\pgfqpoint{1.889885in}{0.440955in}}%
\pgfpathlineto{\pgfqpoint{1.889885in}{0.440955in}}%
\pgfpathlineto{\pgfqpoint{1.878680in}{0.440955in}}%
\pgfpathlineto{\pgfqpoint{1.878680in}{0.440955in}}%
\pgfpathlineto{\pgfqpoint{1.867476in}{0.440955in}}%
\pgfpathlineto{\pgfqpoint{1.867476in}{0.440955in}}%
\pgfpathlineto{\pgfqpoint{1.856271in}{0.440955in}}%
\pgfpathlineto{\pgfqpoint{1.856271in}{0.440955in}}%
\pgfpathlineto{\pgfqpoint{1.845066in}{0.440955in}}%
\pgfpathlineto{\pgfqpoint{1.845066in}{0.440955in}}%
\pgfpathlineto{\pgfqpoint{1.833861in}{0.440955in}}%
\pgfpathlineto{\pgfqpoint{1.833861in}{0.440955in}}%
\pgfpathlineto{\pgfqpoint{1.822657in}{0.440955in}}%
\pgfpathlineto{\pgfqpoint{1.822657in}{0.440955in}}%
\pgfpathlineto{\pgfqpoint{1.811452in}{0.440955in}}%
\pgfpathlineto{\pgfqpoint{1.811452in}{0.440955in}}%
\pgfpathlineto{\pgfqpoint{1.800247in}{0.440955in}}%
\pgfpathlineto{\pgfqpoint{1.800247in}{0.440955in}}%
\pgfpathlineto{\pgfqpoint{1.789043in}{0.440955in}}%
\pgfpathlineto{\pgfqpoint{1.789043in}{0.440955in}}%
\pgfpathlineto{\pgfqpoint{1.777838in}{0.440955in}}%
\pgfpathlineto{\pgfqpoint{1.777838in}{0.440955in}}%
\pgfpathlineto{\pgfqpoint{1.766633in}{0.440955in}}%
\pgfpathlineto{\pgfqpoint{1.766633in}{0.440955in}}%
\pgfpathlineto{\pgfqpoint{1.755428in}{0.440955in}}%
\pgfpathlineto{\pgfqpoint{1.755428in}{0.440955in}}%
\pgfpathlineto{\pgfqpoint{1.744224in}{0.440955in}}%
\pgfpathlineto{\pgfqpoint{1.744224in}{0.440955in}}%
\pgfpathlineto{\pgfqpoint{1.733019in}{0.440955in}}%
\pgfpathlineto{\pgfqpoint{1.733019in}{0.440955in}}%
\pgfpathlineto{\pgfqpoint{1.721814in}{0.440955in}}%
\pgfpathlineto{\pgfqpoint{1.721814in}{0.440955in}}%
\pgfpathlineto{\pgfqpoint{1.710610in}{0.440955in}}%
\pgfpathlineto{\pgfqpoint{1.710610in}{0.440955in}}%
\pgfpathlineto{\pgfqpoint{1.699405in}{0.440955in}}%
\pgfpathlineto{\pgfqpoint{1.699405in}{0.440955in}}%
\pgfpathlineto{\pgfqpoint{1.688200in}{0.440955in}}%
\pgfpathlineto{\pgfqpoint{1.688200in}{0.440955in}}%
\pgfpathlineto{\pgfqpoint{1.676995in}{0.440955in}}%
\pgfpathlineto{\pgfqpoint{1.676995in}{0.440955in}}%
\pgfpathlineto{\pgfqpoint{1.665791in}{0.440955in}}%
\pgfpathlineto{\pgfqpoint{1.665791in}{0.440955in}}%
\pgfpathlineto{\pgfqpoint{1.654586in}{0.440955in}}%
\pgfpathlineto{\pgfqpoint{1.654586in}{0.440955in}}%
\pgfpathlineto{\pgfqpoint{1.643381in}{0.440955in}}%
\pgfpathlineto{\pgfqpoint{1.643381in}{0.440955in}}%
\pgfpathlineto{\pgfqpoint{1.632177in}{0.440955in}}%
\pgfpathlineto{\pgfqpoint{1.632177in}{0.440955in}}%
\pgfpathlineto{\pgfqpoint{1.620972in}{0.440955in}}%
\pgfpathlineto{\pgfqpoint{1.620972in}{0.440955in}}%
\pgfpathlineto{\pgfqpoint{1.609767in}{0.440955in}}%
\pgfpathlineto{\pgfqpoint{1.609767in}{0.440955in}}%
\pgfpathlineto{\pgfqpoint{1.598562in}{0.440955in}}%
\pgfpathlineto{\pgfqpoint{1.598562in}{0.440955in}}%
\pgfpathlineto{\pgfqpoint{1.587358in}{0.440955in}}%
\pgfpathlineto{\pgfqpoint{1.587358in}{0.440955in}}%
\pgfpathlineto{\pgfqpoint{1.576153in}{0.440955in}}%
\pgfpathlineto{\pgfqpoint{1.576153in}{0.440955in}}%
\pgfpathlineto{\pgfqpoint{1.564948in}{0.440955in}}%
\pgfpathlineto{\pgfqpoint{1.564948in}{0.440955in}}%
\pgfpathlineto{\pgfqpoint{1.553744in}{0.440955in}}%
\pgfpathlineto{\pgfqpoint{1.553744in}{0.440955in}}%
\pgfpathlineto{\pgfqpoint{1.542539in}{0.440955in}}%
\pgfpathlineto{\pgfqpoint{1.542539in}{0.440955in}}%
\pgfpathlineto{\pgfqpoint{1.531334in}{0.440955in}}%
\pgfpathlineto{\pgfqpoint{1.531334in}{0.440955in}}%
\pgfpathlineto{\pgfqpoint{1.520129in}{0.440955in}}%
\pgfpathlineto{\pgfqpoint{1.520129in}{0.440955in}}%
\pgfpathlineto{\pgfqpoint{1.508925in}{0.440955in}}%
\pgfpathlineto{\pgfqpoint{1.508925in}{0.440955in}}%
\pgfpathlineto{\pgfqpoint{1.497720in}{0.440955in}}%
\pgfpathlineto{\pgfqpoint{1.497720in}{0.440955in}}%
\pgfpathlineto{\pgfqpoint{1.486515in}{0.440955in}}%
\pgfpathlineto{\pgfqpoint{1.486515in}{0.440955in}}%
\pgfpathlineto{\pgfqpoint{1.475311in}{0.440955in}}%
\pgfpathlineto{\pgfqpoint{1.475311in}{0.440955in}}%
\pgfpathlineto{\pgfqpoint{1.464106in}{0.440955in}}%
\pgfpathlineto{\pgfqpoint{1.464106in}{0.440955in}}%
\pgfpathlineto{\pgfqpoint{1.452901in}{0.440955in}}%
\pgfpathlineto{\pgfqpoint{1.452901in}{0.440955in}}%
\pgfpathlineto{\pgfqpoint{1.441696in}{0.440955in}}%
\pgfpathlineto{\pgfqpoint{1.441696in}{0.440955in}}%
\pgfpathlineto{\pgfqpoint{1.430492in}{0.440955in}}%
\pgfpathlineto{\pgfqpoint{1.430492in}{0.440955in}}%
\pgfpathlineto{\pgfqpoint{1.419287in}{0.440955in}}%
\pgfpathlineto{\pgfqpoint{1.419287in}{0.440955in}}%
\pgfpathlineto{\pgfqpoint{1.408082in}{0.440955in}}%
\pgfpathlineto{\pgfqpoint{1.408082in}{0.440955in}}%
\pgfpathlineto{\pgfqpoint{1.396878in}{0.440955in}}%
\pgfpathlineto{\pgfqpoint{1.396878in}{0.440955in}}%
\pgfpathlineto{\pgfqpoint{1.385673in}{0.440955in}}%
\pgfpathlineto{\pgfqpoint{1.385673in}{0.440955in}}%
\pgfpathlineto{\pgfqpoint{1.374468in}{0.440955in}}%
\pgfpathlineto{\pgfqpoint{1.374468in}{0.440955in}}%
\pgfpathlineto{\pgfqpoint{1.363263in}{0.440955in}}%
\pgfpathlineto{\pgfqpoint{1.363263in}{0.440955in}}%
\pgfpathlineto{\pgfqpoint{1.352059in}{0.440955in}}%
\pgfpathlineto{\pgfqpoint{1.352059in}{0.440955in}}%
\pgfpathlineto{\pgfqpoint{1.340854in}{0.440955in}}%
\pgfpathlineto{\pgfqpoint{1.340854in}{0.440955in}}%
\pgfpathlineto{\pgfqpoint{1.329649in}{0.440955in}}%
\pgfpathlineto{\pgfqpoint{1.329649in}{0.440955in}}%
\pgfpathlineto{\pgfqpoint{1.318444in}{0.440955in}}%
\pgfpathlineto{\pgfqpoint{1.318444in}{0.440955in}}%
\pgfpathlineto{\pgfqpoint{1.307240in}{0.440955in}}%
\pgfpathlineto{\pgfqpoint{1.307240in}{0.440955in}}%
\pgfpathlineto{\pgfqpoint{1.296035in}{0.440955in}}%
\pgfpathlineto{\pgfqpoint{1.296035in}{0.440955in}}%
\pgfpathlineto{\pgfqpoint{1.284830in}{0.440955in}}%
\pgfpathlineto{\pgfqpoint{1.284830in}{0.440955in}}%
\pgfpathlineto{\pgfqpoint{1.273626in}{0.440955in}}%
\pgfpathlineto{\pgfqpoint{1.273626in}{0.440955in}}%
\pgfpathlineto{\pgfqpoint{1.262421in}{0.440955in}}%
\pgfpathlineto{\pgfqpoint{1.262421in}{0.440955in}}%
\pgfpathlineto{\pgfqpoint{1.251216in}{0.440955in}}%
\pgfpathlineto{\pgfqpoint{1.251216in}{0.440955in}}%
\pgfpathlineto{\pgfqpoint{1.240011in}{0.440955in}}%
\pgfpathlineto{\pgfqpoint{1.240011in}{0.440955in}}%
\pgfpathlineto{\pgfqpoint{1.228807in}{0.440955in}}%
\pgfpathlineto{\pgfqpoint{1.228807in}{0.440955in}}%
\pgfpathlineto{\pgfqpoint{1.217602in}{0.440955in}}%
\pgfpathlineto{\pgfqpoint{1.217602in}{0.440955in}}%
\pgfpathlineto{\pgfqpoint{1.206397in}{0.440955in}}%
\pgfpathlineto{\pgfqpoint{1.206397in}{0.440955in}}%
\pgfpathlineto{\pgfqpoint{1.195193in}{0.440955in}}%
\pgfpathlineto{\pgfqpoint{1.195193in}{0.440955in}}%
\pgfpathlineto{\pgfqpoint{1.183988in}{0.440955in}}%
\pgfpathlineto{\pgfqpoint{1.183988in}{0.440955in}}%
\pgfpathlineto{\pgfqpoint{1.172783in}{0.440955in}}%
\pgfpathlineto{\pgfqpoint{1.172783in}{0.440955in}}%
\pgfpathlineto{\pgfqpoint{1.161578in}{0.440955in}}%
\pgfpathlineto{\pgfqpoint{1.161578in}{0.440955in}}%
\pgfpathlineto{\pgfqpoint{1.150374in}{0.440955in}}%
\pgfpathlineto{\pgfqpoint{1.150374in}{0.440955in}}%
\pgfpathlineto{\pgfqpoint{1.139169in}{0.440955in}}%
\pgfpathlineto{\pgfqpoint{1.139169in}{0.440955in}}%
\pgfpathlineto{\pgfqpoint{1.127964in}{0.440955in}}%
\pgfpathlineto{\pgfqpoint{1.127964in}{0.440955in}}%
\pgfpathlineto{\pgfqpoint{1.116760in}{0.440955in}}%
\pgfpathlineto{\pgfqpoint{1.116760in}{0.440955in}}%
\pgfpathlineto{\pgfqpoint{1.105555in}{0.440955in}}%
\pgfpathlineto{\pgfqpoint{1.105555in}{0.440955in}}%
\pgfpathlineto{\pgfqpoint{1.094350in}{0.440955in}}%
\pgfpathlineto{\pgfqpoint{1.094350in}{0.440955in}}%
\pgfpathlineto{\pgfqpoint{1.083145in}{0.440955in}}%
\pgfpathlineto{\pgfqpoint{1.083145in}{0.440955in}}%
\pgfpathlineto{\pgfqpoint{1.071941in}{0.440955in}}%
\pgfpathlineto{\pgfqpoint{1.071941in}{0.440955in}}%
\pgfpathlineto{\pgfqpoint{1.060736in}{0.440955in}}%
\pgfpathlineto{\pgfqpoint{1.060736in}{0.440955in}}%
\pgfpathlineto{\pgfqpoint{1.049531in}{0.440955in}}%
\pgfpathlineto{\pgfqpoint{1.049531in}{0.440955in}}%
\pgfpathlineto{\pgfqpoint{1.038327in}{0.440955in}}%
\pgfpathlineto{\pgfqpoint{1.038327in}{0.440955in}}%
\pgfpathlineto{\pgfqpoint{1.027122in}{0.440955in}}%
\pgfpathlineto{\pgfqpoint{1.027122in}{0.440955in}}%
\pgfpathlineto{\pgfqpoint{1.015917in}{0.440955in}}%
\pgfpathlineto{\pgfqpoint{1.015917in}{0.440955in}}%
\pgfpathlineto{\pgfqpoint{1.004712in}{0.440955in}}%
\pgfpathlineto{\pgfqpoint{1.004712in}{0.440955in}}%
\pgfpathlineto{\pgfqpoint{0.993508in}{0.440955in}}%
\pgfpathlineto{\pgfqpoint{0.993508in}{0.440955in}}%
\pgfpathlineto{\pgfqpoint{0.982303in}{0.440955in}}%
\pgfpathlineto{\pgfqpoint{0.982303in}{0.440955in}}%
\pgfpathlineto{\pgfqpoint{0.971098in}{0.440955in}}%
\pgfpathlineto{\pgfqpoint{0.971098in}{0.440955in}}%
\pgfpathlineto{\pgfqpoint{0.959894in}{0.440955in}}%
\pgfpathlineto{\pgfqpoint{0.959894in}{0.440955in}}%
\pgfpathlineto{\pgfqpoint{0.948689in}{0.440955in}}%
\pgfpathlineto{\pgfqpoint{0.948689in}{0.440955in}}%
\pgfpathlineto{\pgfqpoint{0.937484in}{0.440955in}}%
\pgfpathlineto{\pgfqpoint{0.937484in}{0.440955in}}%
\pgfpathlineto{\pgfqpoint{0.926279in}{0.440955in}}%
\pgfpathlineto{\pgfqpoint{0.926279in}{0.440955in}}%
\pgfpathlineto{\pgfqpoint{0.915075in}{0.440955in}}%
\pgfpathlineto{\pgfqpoint{0.915075in}{0.440955in}}%
\pgfpathlineto{\pgfqpoint{0.903870in}{0.440955in}}%
\pgfpathlineto{\pgfqpoint{0.903870in}{0.440955in}}%
\pgfpathlineto{\pgfqpoint{0.892665in}{0.440955in}}%
\pgfpathlineto{\pgfqpoint{0.892665in}{0.440955in}}%
\pgfpathlineto{\pgfqpoint{0.881461in}{0.440955in}}%
\pgfpathlineto{\pgfqpoint{0.881461in}{0.440955in}}%
\pgfpathlineto{\pgfqpoint{0.870256in}{0.440955in}}%
\pgfpathlineto{\pgfqpoint{0.870256in}{0.440955in}}%
\pgfpathlineto{\pgfqpoint{0.859051in}{0.440955in}}%
\pgfpathlineto{\pgfqpoint{0.859051in}{0.440955in}}%
\pgfpathlineto{\pgfqpoint{0.847846in}{0.440955in}}%
\pgfpathlineto{\pgfqpoint{0.847846in}{0.440955in}}%
\pgfpathlineto{\pgfqpoint{0.836642in}{0.440955in}}%
\pgfpathlineto{\pgfqpoint{0.836642in}{0.440955in}}%
\pgfpathlineto{\pgfqpoint{0.825437in}{0.440955in}}%
\pgfpathlineto{\pgfqpoint{0.825437in}{0.440955in}}%
\pgfpathlineto{\pgfqpoint{0.814232in}{0.440955in}}%
\pgfpathlineto{\pgfqpoint{0.814232in}{0.440955in}}%
\pgfpathlineto{\pgfqpoint{0.803028in}{0.440955in}}%
\pgfpathlineto{\pgfqpoint{0.803028in}{0.440955in}}%
\pgfpathlineto{\pgfqpoint{0.791823in}{0.440955in}}%
\pgfpathlineto{\pgfqpoint{0.791823in}{0.440955in}}%
\pgfpathlineto{\pgfqpoint{0.780618in}{0.440955in}}%
\pgfpathlineto{\pgfqpoint{0.780618in}{0.440955in}}%
\pgfpathlineto{\pgfqpoint{0.769413in}{0.440955in}}%
\pgfpathlineto{\pgfqpoint{0.769413in}{0.440955in}}%
\pgfpathlineto{\pgfqpoint{0.758209in}{0.440955in}}%
\pgfpathlineto{\pgfqpoint{0.758209in}{0.440955in}}%
\pgfpathlineto{\pgfqpoint{0.747004in}{0.440955in}}%
\pgfpathlineto{\pgfqpoint{0.747004in}{0.440955in}}%
\pgfpathlineto{\pgfqpoint{0.735799in}{0.440955in}}%
\pgfpathlineto{\pgfqpoint{0.735799in}{0.440955in}}%
\pgfpathlineto{\pgfqpoint{0.724595in}{0.440955in}}%
\pgfpathlineto{\pgfqpoint{0.724595in}{0.440955in}}%
\pgfpathlineto{\pgfqpoint{0.713390in}{0.440955in}}%
\pgfpathlineto{\pgfqpoint{0.713390in}{0.440955in}}%
\pgfpathlineto{\pgfqpoint{0.702185in}{0.440955in}}%
\pgfpathlineto{\pgfqpoint{0.702185in}{0.440955in}}%
\pgfpathlineto{\pgfqpoint{0.690980in}{0.440955in}}%
\pgfpathlineto{\pgfqpoint{0.690980in}{0.440955in}}%
\pgfpathlineto{\pgfqpoint{0.679776in}{0.440955in}}%
\pgfpathlineto{\pgfqpoint{0.679776in}{0.440955in}}%
\pgfpathlineto{\pgfqpoint{0.668571in}{0.440955in}}%
\pgfusepath{fill}%
\end{pgfscope}%
\begin{pgfscope}%
\pgfsetrectcap%
\pgfsetmiterjoin%
\pgfsetlinewidth{1.003750pt}%
\definecolor{currentstroke}{rgb}{0.000000,0.000000,0.000000}%
\pgfsetstrokecolor{currentstroke}%
\pgfsetdash{}{0pt}%
\pgfpathmoveto{\pgfqpoint{0.668571in}{2.054978in}}%
\pgfpathlineto{\pgfqpoint{3.469750in}{2.054978in}}%
\pgfusepath{stroke}%
\end{pgfscope}%
\begin{pgfscope}%
\pgfsetrectcap%
\pgfsetmiterjoin%
\pgfsetlinewidth{1.003750pt}%
\definecolor{currentstroke}{rgb}{0.000000,0.000000,0.000000}%
\pgfsetstrokecolor{currentstroke}%
\pgfsetdash{}{0pt}%
\pgfpathmoveto{\pgfqpoint{3.469750in}{0.440955in}}%
\pgfpathlineto{\pgfqpoint{3.469750in}{2.054978in}}%
\pgfusepath{stroke}%
\end{pgfscope}%
\begin{pgfscope}%
\pgfsetrectcap%
\pgfsetmiterjoin%
\pgfsetlinewidth{1.003750pt}%
\definecolor{currentstroke}{rgb}{0.000000,0.000000,0.000000}%
\pgfsetstrokecolor{currentstroke}%
\pgfsetdash{}{0pt}%
\pgfpathmoveto{\pgfqpoint{0.668571in}{0.440955in}}%
\pgfpathlineto{\pgfqpoint{3.469750in}{0.440955in}}%
\pgfusepath{stroke}%
\end{pgfscope}%
\begin{pgfscope}%
\pgfsetrectcap%
\pgfsetmiterjoin%
\pgfsetlinewidth{1.003750pt}%
\definecolor{currentstroke}{rgb}{0.000000,0.000000,0.000000}%
\pgfsetstrokecolor{currentstroke}%
\pgfsetdash{}{0pt}%
\pgfpathmoveto{\pgfqpoint{0.668571in}{0.440955in}}%
\pgfpathlineto{\pgfqpoint{0.668571in}{2.054978in}}%
\pgfusepath{stroke}%
\end{pgfscope}%
\begin{pgfscope}%
\pgfsetbuttcap%
\pgfsetroundjoin%
\definecolor{currentfill}{rgb}{0.000000,0.000000,0.000000}%
\pgfsetfillcolor{currentfill}%
\pgfsetlinewidth{0.501875pt}%
\definecolor{currentstroke}{rgb}{0.000000,0.000000,0.000000}%
\pgfsetstrokecolor{currentstroke}%
\pgfsetdash{}{0pt}%
\pgfsys@defobject{currentmarker}{\pgfqpoint{0.000000in}{0.000000in}}{\pgfqpoint{0.000000in}{0.069444in}}{%
\pgfpathmoveto{\pgfqpoint{0.000000in}{0.000000in}}%
\pgfpathlineto{\pgfqpoint{0.000000in}{0.069444in}}%
\pgfusepath{stroke,fill}%
}%
\begin{pgfscope}%
\pgfsys@transformshift{0.668571in}{0.440955in}%
\pgfsys@useobject{currentmarker}{}%
\end{pgfscope}%
\end{pgfscope}%
\begin{pgfscope}%
\pgfsetbuttcap%
\pgfsetroundjoin%
\definecolor{currentfill}{rgb}{0.000000,0.000000,0.000000}%
\pgfsetfillcolor{currentfill}%
\pgfsetlinewidth{0.501875pt}%
\definecolor{currentstroke}{rgb}{0.000000,0.000000,0.000000}%
\pgfsetstrokecolor{currentstroke}%
\pgfsetdash{}{0pt}%
\pgfsys@defobject{currentmarker}{\pgfqpoint{0.000000in}{-0.069444in}}{\pgfqpoint{0.000000in}{0.000000in}}{%
\pgfpathmoveto{\pgfqpoint{0.000000in}{0.000000in}}%
\pgfpathlineto{\pgfqpoint{0.000000in}{-0.069444in}}%
\pgfusepath{stroke,fill}%
}%
\begin{pgfscope}%
\pgfsys@transformshift{0.668571in}{2.054978in}%
\pgfsys@useobject{currentmarker}{}%
\end{pgfscope}%
\end{pgfscope}%
\begin{pgfscope}%
\pgftext[x=0.668571in,y=0.371511in,,top]{\rmfamily\fontsize{8.000000}{9.600000}\selectfont 5000}%
\end{pgfscope}%
\begin{pgfscope}%
\pgfsetbuttcap%
\pgfsetroundjoin%
\definecolor{currentfill}{rgb}{0.000000,0.000000,0.000000}%
\pgfsetfillcolor{currentfill}%
\pgfsetlinewidth{0.501875pt}%
\definecolor{currentstroke}{rgb}{0.000000,0.000000,0.000000}%
\pgfsetstrokecolor{currentstroke}%
\pgfsetdash{}{0pt}%
\pgfsys@defobject{currentmarker}{\pgfqpoint{0.000000in}{0.000000in}}{\pgfqpoint{0.000000in}{0.069444in}}{%
\pgfpathmoveto{\pgfqpoint{0.000000in}{0.000000in}}%
\pgfpathlineto{\pgfqpoint{0.000000in}{0.069444in}}%
\pgfusepath{stroke,fill}%
}%
\begin{pgfscope}%
\pgfsys@transformshift{1.135434in}{0.440955in}%
\pgfsys@useobject{currentmarker}{}%
\end{pgfscope}%
\end{pgfscope}%
\begin{pgfscope}%
\pgfsetbuttcap%
\pgfsetroundjoin%
\definecolor{currentfill}{rgb}{0.000000,0.000000,0.000000}%
\pgfsetfillcolor{currentfill}%
\pgfsetlinewidth{0.501875pt}%
\definecolor{currentstroke}{rgb}{0.000000,0.000000,0.000000}%
\pgfsetstrokecolor{currentstroke}%
\pgfsetdash{}{0pt}%
\pgfsys@defobject{currentmarker}{\pgfqpoint{0.000000in}{-0.069444in}}{\pgfqpoint{0.000000in}{0.000000in}}{%
\pgfpathmoveto{\pgfqpoint{0.000000in}{0.000000in}}%
\pgfpathlineto{\pgfqpoint{0.000000in}{-0.069444in}}%
\pgfusepath{stroke,fill}%
}%
\begin{pgfscope}%
\pgfsys@transformshift{1.135434in}{2.054978in}%
\pgfsys@useobject{currentmarker}{}%
\end{pgfscope}%
\end{pgfscope}%
\begin{pgfscope}%
\pgftext[x=1.135434in,y=0.371511in,,top]{\rmfamily\fontsize{8.000000}{9.600000}\selectfont 5100}%
\end{pgfscope}%
\begin{pgfscope}%
\pgfsetbuttcap%
\pgfsetroundjoin%
\definecolor{currentfill}{rgb}{0.000000,0.000000,0.000000}%
\pgfsetfillcolor{currentfill}%
\pgfsetlinewidth{0.501875pt}%
\definecolor{currentstroke}{rgb}{0.000000,0.000000,0.000000}%
\pgfsetstrokecolor{currentstroke}%
\pgfsetdash{}{0pt}%
\pgfsys@defobject{currentmarker}{\pgfqpoint{0.000000in}{0.000000in}}{\pgfqpoint{0.000000in}{0.069444in}}{%
\pgfpathmoveto{\pgfqpoint{0.000000in}{0.000000in}}%
\pgfpathlineto{\pgfqpoint{0.000000in}{0.069444in}}%
\pgfusepath{stroke,fill}%
}%
\begin{pgfscope}%
\pgfsys@transformshift{1.602297in}{0.440955in}%
\pgfsys@useobject{currentmarker}{}%
\end{pgfscope}%
\end{pgfscope}%
\begin{pgfscope}%
\pgfsetbuttcap%
\pgfsetroundjoin%
\definecolor{currentfill}{rgb}{0.000000,0.000000,0.000000}%
\pgfsetfillcolor{currentfill}%
\pgfsetlinewidth{0.501875pt}%
\definecolor{currentstroke}{rgb}{0.000000,0.000000,0.000000}%
\pgfsetstrokecolor{currentstroke}%
\pgfsetdash{}{0pt}%
\pgfsys@defobject{currentmarker}{\pgfqpoint{0.000000in}{-0.069444in}}{\pgfqpoint{0.000000in}{0.000000in}}{%
\pgfpathmoveto{\pgfqpoint{0.000000in}{0.000000in}}%
\pgfpathlineto{\pgfqpoint{0.000000in}{-0.069444in}}%
\pgfusepath{stroke,fill}%
}%
\begin{pgfscope}%
\pgfsys@transformshift{1.602297in}{2.054978in}%
\pgfsys@useobject{currentmarker}{}%
\end{pgfscope}%
\end{pgfscope}%
\begin{pgfscope}%
\pgftext[x=1.602297in,y=0.371511in,,top]{\rmfamily\fontsize{8.000000}{9.600000}\selectfont 5200}%
\end{pgfscope}%
\begin{pgfscope}%
\pgfsetbuttcap%
\pgfsetroundjoin%
\definecolor{currentfill}{rgb}{0.000000,0.000000,0.000000}%
\pgfsetfillcolor{currentfill}%
\pgfsetlinewidth{0.501875pt}%
\definecolor{currentstroke}{rgb}{0.000000,0.000000,0.000000}%
\pgfsetstrokecolor{currentstroke}%
\pgfsetdash{}{0pt}%
\pgfsys@defobject{currentmarker}{\pgfqpoint{0.000000in}{0.000000in}}{\pgfqpoint{0.000000in}{0.069444in}}{%
\pgfpathmoveto{\pgfqpoint{0.000000in}{0.000000in}}%
\pgfpathlineto{\pgfqpoint{0.000000in}{0.069444in}}%
\pgfusepath{stroke,fill}%
}%
\begin{pgfscope}%
\pgfsys@transformshift{2.069160in}{0.440955in}%
\pgfsys@useobject{currentmarker}{}%
\end{pgfscope}%
\end{pgfscope}%
\begin{pgfscope}%
\pgfsetbuttcap%
\pgfsetroundjoin%
\definecolor{currentfill}{rgb}{0.000000,0.000000,0.000000}%
\pgfsetfillcolor{currentfill}%
\pgfsetlinewidth{0.501875pt}%
\definecolor{currentstroke}{rgb}{0.000000,0.000000,0.000000}%
\pgfsetstrokecolor{currentstroke}%
\pgfsetdash{}{0pt}%
\pgfsys@defobject{currentmarker}{\pgfqpoint{0.000000in}{-0.069444in}}{\pgfqpoint{0.000000in}{0.000000in}}{%
\pgfpathmoveto{\pgfqpoint{0.000000in}{0.000000in}}%
\pgfpathlineto{\pgfqpoint{0.000000in}{-0.069444in}}%
\pgfusepath{stroke,fill}%
}%
\begin{pgfscope}%
\pgfsys@transformshift{2.069160in}{2.054978in}%
\pgfsys@useobject{currentmarker}{}%
\end{pgfscope}%
\end{pgfscope}%
\begin{pgfscope}%
\pgftext[x=2.069160in,y=0.371511in,,top]{\rmfamily\fontsize{8.000000}{9.600000}\selectfont 5300}%
\end{pgfscope}%
\begin{pgfscope}%
\pgfsetbuttcap%
\pgfsetroundjoin%
\definecolor{currentfill}{rgb}{0.000000,0.000000,0.000000}%
\pgfsetfillcolor{currentfill}%
\pgfsetlinewidth{0.501875pt}%
\definecolor{currentstroke}{rgb}{0.000000,0.000000,0.000000}%
\pgfsetstrokecolor{currentstroke}%
\pgfsetdash{}{0pt}%
\pgfsys@defobject{currentmarker}{\pgfqpoint{0.000000in}{0.000000in}}{\pgfqpoint{0.000000in}{0.069444in}}{%
\pgfpathmoveto{\pgfqpoint{0.000000in}{0.000000in}}%
\pgfpathlineto{\pgfqpoint{0.000000in}{0.069444in}}%
\pgfusepath{stroke,fill}%
}%
\begin{pgfscope}%
\pgfsys@transformshift{2.536024in}{0.440955in}%
\pgfsys@useobject{currentmarker}{}%
\end{pgfscope}%
\end{pgfscope}%
\begin{pgfscope}%
\pgfsetbuttcap%
\pgfsetroundjoin%
\definecolor{currentfill}{rgb}{0.000000,0.000000,0.000000}%
\pgfsetfillcolor{currentfill}%
\pgfsetlinewidth{0.501875pt}%
\definecolor{currentstroke}{rgb}{0.000000,0.000000,0.000000}%
\pgfsetstrokecolor{currentstroke}%
\pgfsetdash{}{0pt}%
\pgfsys@defobject{currentmarker}{\pgfqpoint{0.000000in}{-0.069444in}}{\pgfqpoint{0.000000in}{0.000000in}}{%
\pgfpathmoveto{\pgfqpoint{0.000000in}{0.000000in}}%
\pgfpathlineto{\pgfqpoint{0.000000in}{-0.069444in}}%
\pgfusepath{stroke,fill}%
}%
\begin{pgfscope}%
\pgfsys@transformshift{2.536024in}{2.054978in}%
\pgfsys@useobject{currentmarker}{}%
\end{pgfscope}%
\end{pgfscope}%
\begin{pgfscope}%
\pgftext[x=2.536024in,y=0.371511in,,top]{\rmfamily\fontsize{8.000000}{9.600000}\selectfont 5400}%
\end{pgfscope}%
\begin{pgfscope}%
\pgfsetbuttcap%
\pgfsetroundjoin%
\definecolor{currentfill}{rgb}{0.000000,0.000000,0.000000}%
\pgfsetfillcolor{currentfill}%
\pgfsetlinewidth{0.501875pt}%
\definecolor{currentstroke}{rgb}{0.000000,0.000000,0.000000}%
\pgfsetstrokecolor{currentstroke}%
\pgfsetdash{}{0pt}%
\pgfsys@defobject{currentmarker}{\pgfqpoint{0.000000in}{0.000000in}}{\pgfqpoint{0.000000in}{0.069444in}}{%
\pgfpathmoveto{\pgfqpoint{0.000000in}{0.000000in}}%
\pgfpathlineto{\pgfqpoint{0.000000in}{0.069444in}}%
\pgfusepath{stroke,fill}%
}%
\begin{pgfscope}%
\pgfsys@transformshift{3.002887in}{0.440955in}%
\pgfsys@useobject{currentmarker}{}%
\end{pgfscope}%
\end{pgfscope}%
\begin{pgfscope}%
\pgfsetbuttcap%
\pgfsetroundjoin%
\definecolor{currentfill}{rgb}{0.000000,0.000000,0.000000}%
\pgfsetfillcolor{currentfill}%
\pgfsetlinewidth{0.501875pt}%
\definecolor{currentstroke}{rgb}{0.000000,0.000000,0.000000}%
\pgfsetstrokecolor{currentstroke}%
\pgfsetdash{}{0pt}%
\pgfsys@defobject{currentmarker}{\pgfqpoint{0.000000in}{-0.069444in}}{\pgfqpoint{0.000000in}{0.000000in}}{%
\pgfpathmoveto{\pgfqpoint{0.000000in}{0.000000in}}%
\pgfpathlineto{\pgfqpoint{0.000000in}{-0.069444in}}%
\pgfusepath{stroke,fill}%
}%
\begin{pgfscope}%
\pgfsys@transformshift{3.002887in}{2.054978in}%
\pgfsys@useobject{currentmarker}{}%
\end{pgfscope}%
\end{pgfscope}%
\begin{pgfscope}%
\pgftext[x=3.002887in,y=0.371511in,,top]{\rmfamily\fontsize{8.000000}{9.600000}\selectfont 5500}%
\end{pgfscope}%
\begin{pgfscope}%
\pgfsetbuttcap%
\pgfsetroundjoin%
\definecolor{currentfill}{rgb}{0.000000,0.000000,0.000000}%
\pgfsetfillcolor{currentfill}%
\pgfsetlinewidth{0.501875pt}%
\definecolor{currentstroke}{rgb}{0.000000,0.000000,0.000000}%
\pgfsetstrokecolor{currentstroke}%
\pgfsetdash{}{0pt}%
\pgfsys@defobject{currentmarker}{\pgfqpoint{0.000000in}{0.000000in}}{\pgfqpoint{0.000000in}{0.069444in}}{%
\pgfpathmoveto{\pgfqpoint{0.000000in}{0.000000in}}%
\pgfpathlineto{\pgfqpoint{0.000000in}{0.069444in}}%
\pgfusepath{stroke,fill}%
}%
\begin{pgfscope}%
\pgfsys@transformshift{3.469750in}{0.440955in}%
\pgfsys@useobject{currentmarker}{}%
\end{pgfscope}%
\end{pgfscope}%
\begin{pgfscope}%
\pgfsetbuttcap%
\pgfsetroundjoin%
\definecolor{currentfill}{rgb}{0.000000,0.000000,0.000000}%
\pgfsetfillcolor{currentfill}%
\pgfsetlinewidth{0.501875pt}%
\definecolor{currentstroke}{rgb}{0.000000,0.000000,0.000000}%
\pgfsetstrokecolor{currentstroke}%
\pgfsetdash{}{0pt}%
\pgfsys@defobject{currentmarker}{\pgfqpoint{0.000000in}{-0.069444in}}{\pgfqpoint{0.000000in}{0.000000in}}{%
\pgfpathmoveto{\pgfqpoint{0.000000in}{0.000000in}}%
\pgfpathlineto{\pgfqpoint{0.000000in}{-0.069444in}}%
\pgfusepath{stroke,fill}%
}%
\begin{pgfscope}%
\pgfsys@transformshift{3.469750in}{2.054978in}%
\pgfsys@useobject{currentmarker}{}%
\end{pgfscope}%
\end{pgfscope}%
\begin{pgfscope}%
\pgftext[x=3.469750in,y=0.371511in,,top]{\rmfamily\fontsize{8.000000}{9.600000}\selectfont 5600}%
\end{pgfscope}%
\begin{pgfscope}%
\pgftext[x=2.069160in,y=0.194536in,,top]{\rmfamily\fontsize{9.000000}{10.800000}\selectfont \(\displaystyle m(K^+\pi^-\mu^+\mu^-)\ /\ \mathrm{MeV}\)}%
\end{pgfscope}%
\begin{pgfscope}%
\pgfsetbuttcap%
\pgfsetroundjoin%
\definecolor{currentfill}{rgb}{0.000000,0.000000,0.000000}%
\pgfsetfillcolor{currentfill}%
\pgfsetlinewidth{0.501875pt}%
\definecolor{currentstroke}{rgb}{0.000000,0.000000,0.000000}%
\pgfsetstrokecolor{currentstroke}%
\pgfsetdash{}{0pt}%
\pgfsys@defobject{currentmarker}{\pgfqpoint{0.000000in}{0.000000in}}{\pgfqpoint{0.069444in}{0.000000in}}{%
\pgfpathmoveto{\pgfqpoint{0.000000in}{0.000000in}}%
\pgfpathlineto{\pgfqpoint{0.069444in}{0.000000in}}%
\pgfusepath{stroke,fill}%
}%
\begin{pgfscope}%
\pgfsys@transformshift{0.668571in}{0.440955in}%
\pgfsys@useobject{currentmarker}{}%
\end{pgfscope}%
\end{pgfscope}%
\begin{pgfscope}%
\pgfsetbuttcap%
\pgfsetroundjoin%
\definecolor{currentfill}{rgb}{0.000000,0.000000,0.000000}%
\pgfsetfillcolor{currentfill}%
\pgfsetlinewidth{0.501875pt}%
\definecolor{currentstroke}{rgb}{0.000000,0.000000,0.000000}%
\pgfsetstrokecolor{currentstroke}%
\pgfsetdash{}{0pt}%
\pgfsys@defobject{currentmarker}{\pgfqpoint{-0.069444in}{0.000000in}}{\pgfqpoint{0.000000in}{0.000000in}}{%
\pgfpathmoveto{\pgfqpoint{0.000000in}{0.000000in}}%
\pgfpathlineto{\pgfqpoint{-0.069444in}{0.000000in}}%
\pgfusepath{stroke,fill}%
}%
\begin{pgfscope}%
\pgfsys@transformshift{3.469750in}{0.440955in}%
\pgfsys@useobject{currentmarker}{}%
\end{pgfscope}%
\end{pgfscope}%
\begin{pgfscope}%
\pgftext[x=0.599127in,y=0.440955in,right,]{\rmfamily\fontsize{8.000000}{9.600000}\selectfont 0}%
\end{pgfscope}%
\begin{pgfscope}%
\pgfsetbuttcap%
\pgfsetroundjoin%
\definecolor{currentfill}{rgb}{0.000000,0.000000,0.000000}%
\pgfsetfillcolor{currentfill}%
\pgfsetlinewidth{0.501875pt}%
\definecolor{currentstroke}{rgb}{0.000000,0.000000,0.000000}%
\pgfsetstrokecolor{currentstroke}%
\pgfsetdash{}{0pt}%
\pgfsys@defobject{currentmarker}{\pgfqpoint{0.000000in}{0.000000in}}{\pgfqpoint{0.069444in}{0.000000in}}{%
\pgfpathmoveto{\pgfqpoint{0.000000in}{0.000000in}}%
\pgfpathlineto{\pgfqpoint{0.069444in}{0.000000in}}%
\pgfusepath{stroke,fill}%
}%
\begin{pgfscope}%
\pgfsys@transformshift{0.668571in}{0.671530in}%
\pgfsys@useobject{currentmarker}{}%
\end{pgfscope}%
\end{pgfscope}%
\begin{pgfscope}%
\pgfsetbuttcap%
\pgfsetroundjoin%
\definecolor{currentfill}{rgb}{0.000000,0.000000,0.000000}%
\pgfsetfillcolor{currentfill}%
\pgfsetlinewidth{0.501875pt}%
\definecolor{currentstroke}{rgb}{0.000000,0.000000,0.000000}%
\pgfsetstrokecolor{currentstroke}%
\pgfsetdash{}{0pt}%
\pgfsys@defobject{currentmarker}{\pgfqpoint{-0.069444in}{0.000000in}}{\pgfqpoint{0.000000in}{0.000000in}}{%
\pgfpathmoveto{\pgfqpoint{0.000000in}{0.000000in}}%
\pgfpathlineto{\pgfqpoint{-0.069444in}{0.000000in}}%
\pgfusepath{stroke,fill}%
}%
\begin{pgfscope}%
\pgfsys@transformshift{3.469750in}{0.671530in}%
\pgfsys@useobject{currentmarker}{}%
\end{pgfscope}%
\end{pgfscope}%
\begin{pgfscope}%
\pgftext[x=0.599127in,y=0.671530in,right,]{\rmfamily\fontsize{8.000000}{9.600000}\selectfont 10000}%
\end{pgfscope}%
\begin{pgfscope}%
\pgfsetbuttcap%
\pgfsetroundjoin%
\definecolor{currentfill}{rgb}{0.000000,0.000000,0.000000}%
\pgfsetfillcolor{currentfill}%
\pgfsetlinewidth{0.501875pt}%
\definecolor{currentstroke}{rgb}{0.000000,0.000000,0.000000}%
\pgfsetstrokecolor{currentstroke}%
\pgfsetdash{}{0pt}%
\pgfsys@defobject{currentmarker}{\pgfqpoint{0.000000in}{0.000000in}}{\pgfqpoint{0.069444in}{0.000000in}}{%
\pgfpathmoveto{\pgfqpoint{0.000000in}{0.000000in}}%
\pgfpathlineto{\pgfqpoint{0.069444in}{0.000000in}}%
\pgfusepath{stroke,fill}%
}%
\begin{pgfscope}%
\pgfsys@transformshift{0.668571in}{0.902105in}%
\pgfsys@useobject{currentmarker}{}%
\end{pgfscope}%
\end{pgfscope}%
\begin{pgfscope}%
\pgfsetbuttcap%
\pgfsetroundjoin%
\definecolor{currentfill}{rgb}{0.000000,0.000000,0.000000}%
\pgfsetfillcolor{currentfill}%
\pgfsetlinewidth{0.501875pt}%
\definecolor{currentstroke}{rgb}{0.000000,0.000000,0.000000}%
\pgfsetstrokecolor{currentstroke}%
\pgfsetdash{}{0pt}%
\pgfsys@defobject{currentmarker}{\pgfqpoint{-0.069444in}{0.000000in}}{\pgfqpoint{0.000000in}{0.000000in}}{%
\pgfpathmoveto{\pgfqpoint{0.000000in}{0.000000in}}%
\pgfpathlineto{\pgfqpoint{-0.069444in}{0.000000in}}%
\pgfusepath{stroke,fill}%
}%
\begin{pgfscope}%
\pgfsys@transformshift{3.469750in}{0.902105in}%
\pgfsys@useobject{currentmarker}{}%
\end{pgfscope}%
\end{pgfscope}%
\begin{pgfscope}%
\pgftext[x=0.599127in,y=0.902105in,right,]{\rmfamily\fontsize{8.000000}{9.600000}\selectfont 20000}%
\end{pgfscope}%
\begin{pgfscope}%
\pgfsetbuttcap%
\pgfsetroundjoin%
\definecolor{currentfill}{rgb}{0.000000,0.000000,0.000000}%
\pgfsetfillcolor{currentfill}%
\pgfsetlinewidth{0.501875pt}%
\definecolor{currentstroke}{rgb}{0.000000,0.000000,0.000000}%
\pgfsetstrokecolor{currentstroke}%
\pgfsetdash{}{0pt}%
\pgfsys@defobject{currentmarker}{\pgfqpoint{0.000000in}{0.000000in}}{\pgfqpoint{0.069444in}{0.000000in}}{%
\pgfpathmoveto{\pgfqpoint{0.000000in}{0.000000in}}%
\pgfpathlineto{\pgfqpoint{0.069444in}{0.000000in}}%
\pgfusepath{stroke,fill}%
}%
\begin{pgfscope}%
\pgfsys@transformshift{0.668571in}{1.132679in}%
\pgfsys@useobject{currentmarker}{}%
\end{pgfscope}%
\end{pgfscope}%
\begin{pgfscope}%
\pgfsetbuttcap%
\pgfsetroundjoin%
\definecolor{currentfill}{rgb}{0.000000,0.000000,0.000000}%
\pgfsetfillcolor{currentfill}%
\pgfsetlinewidth{0.501875pt}%
\definecolor{currentstroke}{rgb}{0.000000,0.000000,0.000000}%
\pgfsetstrokecolor{currentstroke}%
\pgfsetdash{}{0pt}%
\pgfsys@defobject{currentmarker}{\pgfqpoint{-0.069444in}{0.000000in}}{\pgfqpoint{0.000000in}{0.000000in}}{%
\pgfpathmoveto{\pgfqpoint{0.000000in}{0.000000in}}%
\pgfpathlineto{\pgfqpoint{-0.069444in}{0.000000in}}%
\pgfusepath{stroke,fill}%
}%
\begin{pgfscope}%
\pgfsys@transformshift{3.469750in}{1.132679in}%
\pgfsys@useobject{currentmarker}{}%
\end{pgfscope}%
\end{pgfscope}%
\begin{pgfscope}%
\pgftext[x=0.599127in,y=1.132679in,right,]{\rmfamily\fontsize{8.000000}{9.600000}\selectfont 30000}%
\end{pgfscope}%
\begin{pgfscope}%
\pgfsetbuttcap%
\pgfsetroundjoin%
\definecolor{currentfill}{rgb}{0.000000,0.000000,0.000000}%
\pgfsetfillcolor{currentfill}%
\pgfsetlinewidth{0.501875pt}%
\definecolor{currentstroke}{rgb}{0.000000,0.000000,0.000000}%
\pgfsetstrokecolor{currentstroke}%
\pgfsetdash{}{0pt}%
\pgfsys@defobject{currentmarker}{\pgfqpoint{0.000000in}{0.000000in}}{\pgfqpoint{0.069444in}{0.000000in}}{%
\pgfpathmoveto{\pgfqpoint{0.000000in}{0.000000in}}%
\pgfpathlineto{\pgfqpoint{0.069444in}{0.000000in}}%
\pgfusepath{stroke,fill}%
}%
\begin{pgfscope}%
\pgfsys@transformshift{0.668571in}{1.363254in}%
\pgfsys@useobject{currentmarker}{}%
\end{pgfscope}%
\end{pgfscope}%
\begin{pgfscope}%
\pgfsetbuttcap%
\pgfsetroundjoin%
\definecolor{currentfill}{rgb}{0.000000,0.000000,0.000000}%
\pgfsetfillcolor{currentfill}%
\pgfsetlinewidth{0.501875pt}%
\definecolor{currentstroke}{rgb}{0.000000,0.000000,0.000000}%
\pgfsetstrokecolor{currentstroke}%
\pgfsetdash{}{0pt}%
\pgfsys@defobject{currentmarker}{\pgfqpoint{-0.069444in}{0.000000in}}{\pgfqpoint{0.000000in}{0.000000in}}{%
\pgfpathmoveto{\pgfqpoint{0.000000in}{0.000000in}}%
\pgfpathlineto{\pgfqpoint{-0.069444in}{0.000000in}}%
\pgfusepath{stroke,fill}%
}%
\begin{pgfscope}%
\pgfsys@transformshift{3.469750in}{1.363254in}%
\pgfsys@useobject{currentmarker}{}%
\end{pgfscope}%
\end{pgfscope}%
\begin{pgfscope}%
\pgftext[x=0.599127in,y=1.363254in,right,]{\rmfamily\fontsize{8.000000}{9.600000}\selectfont 40000}%
\end{pgfscope}%
\begin{pgfscope}%
\pgfsetbuttcap%
\pgfsetroundjoin%
\definecolor{currentfill}{rgb}{0.000000,0.000000,0.000000}%
\pgfsetfillcolor{currentfill}%
\pgfsetlinewidth{0.501875pt}%
\definecolor{currentstroke}{rgb}{0.000000,0.000000,0.000000}%
\pgfsetstrokecolor{currentstroke}%
\pgfsetdash{}{0pt}%
\pgfsys@defobject{currentmarker}{\pgfqpoint{0.000000in}{0.000000in}}{\pgfqpoint{0.069444in}{0.000000in}}{%
\pgfpathmoveto{\pgfqpoint{0.000000in}{0.000000in}}%
\pgfpathlineto{\pgfqpoint{0.069444in}{0.000000in}}%
\pgfusepath{stroke,fill}%
}%
\begin{pgfscope}%
\pgfsys@transformshift{0.668571in}{1.593829in}%
\pgfsys@useobject{currentmarker}{}%
\end{pgfscope}%
\end{pgfscope}%
\begin{pgfscope}%
\pgfsetbuttcap%
\pgfsetroundjoin%
\definecolor{currentfill}{rgb}{0.000000,0.000000,0.000000}%
\pgfsetfillcolor{currentfill}%
\pgfsetlinewidth{0.501875pt}%
\definecolor{currentstroke}{rgb}{0.000000,0.000000,0.000000}%
\pgfsetstrokecolor{currentstroke}%
\pgfsetdash{}{0pt}%
\pgfsys@defobject{currentmarker}{\pgfqpoint{-0.069444in}{0.000000in}}{\pgfqpoint{0.000000in}{0.000000in}}{%
\pgfpathmoveto{\pgfqpoint{0.000000in}{0.000000in}}%
\pgfpathlineto{\pgfqpoint{-0.069444in}{0.000000in}}%
\pgfusepath{stroke,fill}%
}%
\begin{pgfscope}%
\pgfsys@transformshift{3.469750in}{1.593829in}%
\pgfsys@useobject{currentmarker}{}%
\end{pgfscope}%
\end{pgfscope}%
\begin{pgfscope}%
\pgftext[x=0.599127in,y=1.593829in,right,]{\rmfamily\fontsize{8.000000}{9.600000}\selectfont 50000}%
\end{pgfscope}%
\begin{pgfscope}%
\pgfsetbuttcap%
\pgfsetroundjoin%
\definecolor{currentfill}{rgb}{0.000000,0.000000,0.000000}%
\pgfsetfillcolor{currentfill}%
\pgfsetlinewidth{0.501875pt}%
\definecolor{currentstroke}{rgb}{0.000000,0.000000,0.000000}%
\pgfsetstrokecolor{currentstroke}%
\pgfsetdash{}{0pt}%
\pgfsys@defobject{currentmarker}{\pgfqpoint{0.000000in}{0.000000in}}{\pgfqpoint{0.069444in}{0.000000in}}{%
\pgfpathmoveto{\pgfqpoint{0.000000in}{0.000000in}}%
\pgfpathlineto{\pgfqpoint{0.069444in}{0.000000in}}%
\pgfusepath{stroke,fill}%
}%
\begin{pgfscope}%
\pgfsys@transformshift{0.668571in}{1.824403in}%
\pgfsys@useobject{currentmarker}{}%
\end{pgfscope}%
\end{pgfscope}%
\begin{pgfscope}%
\pgfsetbuttcap%
\pgfsetroundjoin%
\definecolor{currentfill}{rgb}{0.000000,0.000000,0.000000}%
\pgfsetfillcolor{currentfill}%
\pgfsetlinewidth{0.501875pt}%
\definecolor{currentstroke}{rgb}{0.000000,0.000000,0.000000}%
\pgfsetstrokecolor{currentstroke}%
\pgfsetdash{}{0pt}%
\pgfsys@defobject{currentmarker}{\pgfqpoint{-0.069444in}{0.000000in}}{\pgfqpoint{0.000000in}{0.000000in}}{%
\pgfpathmoveto{\pgfqpoint{0.000000in}{0.000000in}}%
\pgfpathlineto{\pgfqpoint{-0.069444in}{0.000000in}}%
\pgfusepath{stroke,fill}%
}%
\begin{pgfscope}%
\pgfsys@transformshift{3.469750in}{1.824403in}%
\pgfsys@useobject{currentmarker}{}%
\end{pgfscope}%
\end{pgfscope}%
\begin{pgfscope}%
\pgftext[x=0.599127in,y=1.824403in,right,]{\rmfamily\fontsize{8.000000}{9.600000}\selectfont 60000}%
\end{pgfscope}%
\begin{pgfscope}%
\pgfsetbuttcap%
\pgfsetroundjoin%
\definecolor{currentfill}{rgb}{0.000000,0.000000,0.000000}%
\pgfsetfillcolor{currentfill}%
\pgfsetlinewidth{0.501875pt}%
\definecolor{currentstroke}{rgb}{0.000000,0.000000,0.000000}%
\pgfsetstrokecolor{currentstroke}%
\pgfsetdash{}{0pt}%
\pgfsys@defobject{currentmarker}{\pgfqpoint{0.000000in}{0.000000in}}{\pgfqpoint{0.069444in}{0.000000in}}{%
\pgfpathmoveto{\pgfqpoint{0.000000in}{0.000000in}}%
\pgfpathlineto{\pgfqpoint{0.069444in}{0.000000in}}%
\pgfusepath{stroke,fill}%
}%
\begin{pgfscope}%
\pgfsys@transformshift{0.668571in}{2.054978in}%
\pgfsys@useobject{currentmarker}{}%
\end{pgfscope}%
\end{pgfscope}%
\begin{pgfscope}%
\pgfsetbuttcap%
\pgfsetroundjoin%
\definecolor{currentfill}{rgb}{0.000000,0.000000,0.000000}%
\pgfsetfillcolor{currentfill}%
\pgfsetlinewidth{0.501875pt}%
\definecolor{currentstroke}{rgb}{0.000000,0.000000,0.000000}%
\pgfsetstrokecolor{currentstroke}%
\pgfsetdash{}{0pt}%
\pgfsys@defobject{currentmarker}{\pgfqpoint{-0.069444in}{0.000000in}}{\pgfqpoint{0.000000in}{0.000000in}}{%
\pgfpathmoveto{\pgfqpoint{0.000000in}{0.000000in}}%
\pgfpathlineto{\pgfqpoint{-0.069444in}{0.000000in}}%
\pgfusepath{stroke,fill}%
}%
\begin{pgfscope}%
\pgfsys@transformshift{3.469750in}{2.054978in}%
\pgfsys@useobject{currentmarker}{}%
\end{pgfscope}%
\end{pgfscope}%
\begin{pgfscope}%
\pgftext[x=0.599127in,y=2.054978in,right,]{\rmfamily\fontsize{8.000000}{9.600000}\selectfont 70000}%
\end{pgfscope}%
\begin{pgfscope}%
\pgftext[x=0.176221in,y=1.247967in,,bottom,rotate=90.000000]{\rmfamily\fontsize{9.000000}{10.800000}\selectfont Candidates \(\displaystyle /\ 2.40\ \mathrm{MeV}\)}%
\end{pgfscope}%
\end{pgfpicture}%
\makeatother%
\endgroup%
}
  \caption{
    Reconstructed \PBzero mass of the normalization data sample after trigger and stripping requirements.
  }
  \label{fig:bmass}
\end{figure}

\subsection{Vetoes to reject specific physical backgrounds}

Before proceeding with the multivariate selection, a number of vetoes is applied to the two datasets to reject specific physical backgrounds.
These are summarized in table \ref{tab:signalcuts}.

\begin{table}
  \centering
  \caption{
    Summary of all preselection cuts applied to the signal ($\PBzero\to\APDzero\APmuon\Pmuon$) dataset.
    Each efficiency is calculated based on the output of the previous selection cut.
  }
  \begin{tabular}{l l S[table-format=2.3,table-figures-uncertainty=1]}
    \toprule
    Background & Veto & {Efficiency $/\ \si{\percent}$} \\
    \midrule
    $-$ & $m(\APmuon\Pmuon) > \SI{3500}{MeV}$ & 100 \\
    $\PBzero\to\PJpsi X$ & $\SI{2900}{MeV} < m(\APmuon\Pmuon) < \SI{3200}{MeV}$ & 87.00 \pm 0.12 \\
    $\PBzero\to\PJpsi\PKstar$& $\SI{2900}{MeV} < m(\APmuon\pi^-_\mu) < \SI{3200}{MeV}$ & 93.22 \pm 0.10 \\
    $\PBzero\to\PpsiTwoS\PKstar$&$\SI{3500}{MeV} < m(\APmuon\pi^-_\mu) < \SI{3800}{MeV}$ & 96.72 \pm 0.07 \\
    $\PBzero\to\PDstar^-\APmuon\Pneutrino$ & $\SI{1990}{MeV} < m(\PKplus\Ppiminus\mu^-_\pi) < \SI{2030}{MeV}$ & 99.957 \pm 0.008 \\
    \midrule
    Total & & 78.40 \pm 0.15 \\
    \bottomrule
  \end{tabular}
  \label{tab:signalcuts}
\end{table}


\subsubsection{Dimuon mass vetoes}

% TODO what about a Kaon swap?

Two regions in the invariant mass distribution $m(\APmuon\Pmuon)$ of the two muons (see figure \ref{fig:B_vs_Jpsi}) are excluded:
The first ranges from \SI{2900}{MeV} to \SI{3200}{MeV} and is designed to exclude muon pairs originating from resonant \PJpsi decays.
The second excludes all values of $m(\APmuon\Pmuon)$ upwards of \SI{3500}{MeV}.
This region can be safely excluded because it lies outside of the phase space of the $\PBzero\to\APDzero\APmuon\Pmuon$ decay (see figure \ref{fig:mumu_mc}).

\begin{figure}
  \centering
  \input{store/vetoes/B_vs_Jpsi.pgf}
  \caption{
    A two-dimensional histogram of the invariant mass of all $B^0$ decay products and the $q^2$ of the two muons.
    Regions excluded as part of the $m(\APmuon\Pmuon)$ vetoes are marked in red.
  }
  \label{fig:B_vs_Jpsi}
\end{figure}

\begin{figure}
  \centering
  %% Creator: Matplotlib, PGF backend
%%
%% To include the figure in your LaTeX document, write
%%   \input{<filename>.pgf}
%%
%% Make sure the required packages are loaded in your preamble
%%   \usepackage{pgf}
%%
%% Figures using additional raster images can only be included by \input if
%% they are in the same directory as the main LaTeX file. For loading figures
%% from other directories you can use the `import` package
%%   \usepackage{import}
%% and then include the figures with
%%   \import{<path to file>}{<filename>.pgf}
%%
%% Matplotlib used the following preamble
%%   \usepackage{fontspec}
%%   \setmainfont{DejaVu Serif}
%%   \setsansfont{DejaVu Sans}
%%   \setmonofont{DejaVu Sans Mono}
%%
\begingroup%
\makeatletter%
\begin{pgfpicture}%
\pgfpathrectangle{\pgfpointorigin}{\pgfqpoint{3.935370in}{2.600755in}}%
\pgfusepath{use as bounding box, clip}%
\begin{pgfscope}%
\pgfsetbuttcap%
\pgfsetmiterjoin%
\definecolor{currentfill}{rgb}{1.000000,1.000000,1.000000}%
\pgfsetfillcolor{currentfill}%
\pgfsetlinewidth{0.000000pt}%
\definecolor{currentstroke}{rgb}{1.000000,1.000000,1.000000}%
\pgfsetstrokecolor{currentstroke}%
\pgfsetdash{}{0pt}%
\pgfpathmoveto{\pgfqpoint{0.000000in}{0.000000in}}%
\pgfpathlineto{\pgfqpoint{3.935370in}{0.000000in}}%
\pgfpathlineto{\pgfqpoint{3.935370in}{2.600755in}}%
\pgfpathlineto{\pgfqpoint{0.000000in}{2.600755in}}%
\pgfpathclose%
\pgfusepath{fill}%
\end{pgfscope}%
\begin{pgfscope}%
\pgfsetbuttcap%
\pgfsetmiterjoin%
\definecolor{currentfill}{rgb}{1.000000,1.000000,1.000000}%
\pgfsetfillcolor{currentfill}%
\pgfsetlinewidth{0.000000pt}%
\definecolor{currentstroke}{rgb}{0.000000,0.000000,0.000000}%
\pgfsetstrokecolor{currentstroke}%
\pgfsetstrokeopacity{0.000000}%
\pgfsetdash{}{0pt}%
\pgfpathmoveto{\pgfqpoint{0.592630in}{0.441418in}}%
\pgfpathlineto{\pgfqpoint{3.814677in}{0.441418in}}%
\pgfpathlineto{\pgfqpoint{3.814677in}{2.496990in}}%
\pgfpathlineto{\pgfqpoint{0.592630in}{2.496990in}}%
\pgfpathclose%
\pgfusepath{fill}%
\end{pgfscope}%
\begin{pgfscope}%
\pgfpathrectangle{\pgfqpoint{0.592630in}{0.441418in}}{\pgfqpoint{3.222048in}{2.055572in}} %
\pgfusepath{clip}%
\pgfsetbuttcap%
\pgfsetmiterjoin%
\definecolor{currentfill}{rgb}{0.215686,0.470588,0.749020}%
\pgfsetfillcolor{currentfill}%
\pgfsetlinewidth{1.003750pt}%
\definecolor{currentstroke}{rgb}{0.000000,0.000000,0.000000}%
\pgfsetstrokecolor{currentstroke}%
\pgfsetdash{}{0pt}%
\pgfpathmoveto{\pgfqpoint{0.602983in}{0.441418in}}%
\pgfpathlineto{\pgfqpoint{0.602983in}{2.028320in}}%
\pgfpathlineto{\pgfqpoint{0.630261in}{2.028320in}}%
\pgfpathlineto{\pgfqpoint{0.630261in}{2.398323in}}%
\pgfpathlineto{\pgfqpoint{0.657539in}{2.398323in}}%
\pgfpathlineto{\pgfqpoint{0.657539in}{2.482601in}}%
\pgfpathlineto{\pgfqpoint{0.684817in}{2.482601in}}%
\pgfpathlineto{\pgfqpoint{0.684817in}{2.301711in}}%
\pgfpathlineto{\pgfqpoint{0.712095in}{2.301711in}}%
\pgfpathlineto{\pgfqpoint{0.712095in}{2.464101in}}%
\pgfpathlineto{\pgfqpoint{0.739373in}{2.464101in}}%
\pgfpathlineto{\pgfqpoint{0.739373in}{2.264710in}}%
\pgfpathlineto{\pgfqpoint{0.766651in}{2.264710in}}%
\pgfpathlineto{\pgfqpoint{0.766651in}{2.221543in}}%
\pgfpathlineto{\pgfqpoint{0.793929in}{2.221543in}}%
\pgfpathlineto{\pgfqpoint{0.793929in}{2.410656in}}%
\pgfpathlineto{\pgfqpoint{0.821207in}{2.410656in}}%
\pgfpathlineto{\pgfqpoint{0.821207in}{2.303766in}}%
\pgfpathlineto{\pgfqpoint{0.848485in}{2.303766in}}%
\pgfpathlineto{\pgfqpoint{0.848485in}{2.281155in}}%
\pgfpathlineto{\pgfqpoint{0.875763in}{2.281155in}}%
\pgfpathlineto{\pgfqpoint{0.875763in}{2.311989in}}%
\pgfpathlineto{\pgfqpoint{0.903041in}{2.311989in}}%
\pgfpathlineto{\pgfqpoint{0.903041in}{2.250321in}}%
\pgfpathlineto{\pgfqpoint{0.930319in}{2.250321in}}%
\pgfpathlineto{\pgfqpoint{0.930319in}{2.361322in}}%
\pgfpathlineto{\pgfqpoint{0.957597in}{2.361322in}}%
\pgfpathlineto{\pgfqpoint{0.957597in}{2.429156in}}%
\pgfpathlineto{\pgfqpoint{0.984875in}{2.429156in}}%
\pgfpathlineto{\pgfqpoint{0.984875in}{2.381878in}}%
\pgfpathlineto{\pgfqpoint{1.012153in}{2.381878in}}%
\pgfpathlineto{\pgfqpoint{1.012153in}{2.355156in}}%
\pgfpathlineto{\pgfqpoint{1.039431in}{2.355156in}}%
\pgfpathlineto{\pgfqpoint{1.039431in}{2.402434in}}%
\pgfpathlineto{\pgfqpoint{1.066709in}{2.402434in}}%
\pgfpathlineto{\pgfqpoint{1.066709in}{2.351044in}}%
\pgfpathlineto{\pgfqpoint{1.093987in}{2.351044in}}%
\pgfpathlineto{\pgfqpoint{1.093987in}{2.322266in}}%
\pgfpathlineto{\pgfqpoint{1.121265in}{2.322266in}}%
\pgfpathlineto{\pgfqpoint{1.121265in}{2.272933in}}%
\pgfpathlineto{\pgfqpoint{1.148543in}{2.272933in}}%
\pgfpathlineto{\pgfqpoint{1.148543in}{2.398323in}}%
\pgfpathlineto{\pgfqpoint{1.175821in}{2.398323in}}%
\pgfpathlineto{\pgfqpoint{1.175821in}{2.429156in}}%
\pgfpathlineto{\pgfqpoint{1.203099in}{2.429156in}}%
\pgfpathlineto{\pgfqpoint{1.203099in}{2.338711in}}%
\pgfpathlineto{\pgfqpoint{1.230377in}{2.338711in}}%
\pgfpathlineto{\pgfqpoint{1.230377in}{2.285266in}}%
\pgfpathlineto{\pgfqpoint{1.257655in}{2.285266in}}%
\pgfpathlineto{\pgfqpoint{1.257655in}{2.256488in}}%
\pgfpathlineto{\pgfqpoint{1.284933in}{2.256488in}}%
\pgfpathlineto{\pgfqpoint{1.284933in}{2.277044in}}%
\pgfpathlineto{\pgfqpoint{1.312211in}{2.277044in}}%
\pgfpathlineto{\pgfqpoint{1.312211in}{2.420934in}}%
\pgfpathlineto{\pgfqpoint{1.339489in}{2.420934in}}%
\pgfpathlineto{\pgfqpoint{1.339489in}{2.291433in}}%
\pgfpathlineto{\pgfqpoint{1.366767in}{2.291433in}}%
\pgfpathlineto{\pgfqpoint{1.366767in}{2.398323in}}%
\pgfpathlineto{\pgfqpoint{1.394045in}{2.398323in}}%
\pgfpathlineto{\pgfqpoint{1.394045in}{2.248266in}}%
\pgfpathlineto{\pgfqpoint{1.421323in}{2.248266in}}%
\pgfpathlineto{\pgfqpoint{1.421323in}{2.379822in}}%
\pgfpathlineto{\pgfqpoint{1.448601in}{2.379822in}}%
\pgfpathlineto{\pgfqpoint{1.448601in}{2.309933in}}%
\pgfpathlineto{\pgfqpoint{1.475879in}{2.309933in}}%
\pgfpathlineto{\pgfqpoint{1.475879in}{2.291433in}}%
\pgfpathlineto{\pgfqpoint{1.503157in}{2.291433in}}%
\pgfpathlineto{\pgfqpoint{1.503157in}{2.400378in}}%
\pgfpathlineto{\pgfqpoint{1.530435in}{2.400378in}}%
\pgfpathlineto{\pgfqpoint{1.530435in}{2.338711in}}%
\pgfpathlineto{\pgfqpoint{1.557713in}{2.338711in}}%
\pgfpathlineto{\pgfqpoint{1.557713in}{2.410656in}}%
\pgfpathlineto{\pgfqpoint{1.584991in}{2.410656in}}%
\pgfpathlineto{\pgfqpoint{1.584991in}{2.447656in}}%
\pgfpathlineto{\pgfqpoint{1.612269in}{2.447656in}}%
\pgfpathlineto{\pgfqpoint{1.612269in}{2.264710in}}%
\pgfpathlineto{\pgfqpoint{1.639547in}{2.264710in}}%
\pgfpathlineto{\pgfqpoint{1.639547in}{2.309933in}}%
\pgfpathlineto{\pgfqpoint{1.666825in}{2.309933in}}%
\pgfpathlineto{\pgfqpoint{1.666825in}{2.254433in}}%
\pgfpathlineto{\pgfqpoint{1.694103in}{2.254433in}}%
\pgfpathlineto{\pgfqpoint{1.694103in}{2.392156in}}%
\pgfpathlineto{\pgfqpoint{1.721381in}{2.392156in}}%
\pgfpathlineto{\pgfqpoint{1.721381in}{2.289377in}}%
\pgfpathlineto{\pgfqpoint{1.748659in}{2.289377in}}%
\pgfpathlineto{\pgfqpoint{1.748659in}{2.348989in}}%
\pgfpathlineto{\pgfqpoint{1.775937in}{2.348989in}}%
\pgfpathlineto{\pgfqpoint{1.775937in}{2.299655in}}%
\pgfpathlineto{\pgfqpoint{1.803215in}{2.299655in}}%
\pgfpathlineto{\pgfqpoint{1.803215in}{2.385989in}}%
\pgfpathlineto{\pgfqpoint{1.830493in}{2.385989in}}%
\pgfpathlineto{\pgfqpoint{1.830493in}{2.330489in}}%
\pgfpathlineto{\pgfqpoint{1.857771in}{2.330489in}}%
\pgfpathlineto{\pgfqpoint{1.857771in}{2.334600in}}%
\pgfpathlineto{\pgfqpoint{1.885049in}{2.334600in}}%
\pgfpathlineto{\pgfqpoint{1.885049in}{2.250321in}}%
\pgfpathlineto{\pgfqpoint{1.912327in}{2.250321in}}%
\pgfpathlineto{\pgfqpoint{1.912327in}{2.272933in}}%
\pgfpathlineto{\pgfqpoint{1.939605in}{2.272933in}}%
\pgfpathlineto{\pgfqpoint{1.939605in}{2.318155in}}%
\pgfpathlineto{\pgfqpoint{1.966883in}{2.318155in}}%
\pgfpathlineto{\pgfqpoint{1.966883in}{2.200988in}}%
\pgfpathlineto{\pgfqpoint{1.994161in}{2.200988in}}%
\pgfpathlineto{\pgfqpoint{1.994161in}{2.213321in}}%
\pgfpathlineto{\pgfqpoint{2.021439in}{2.213321in}}%
\pgfpathlineto{\pgfqpoint{2.021439in}{2.351044in}}%
\pgfpathlineto{\pgfqpoint{2.048717in}{2.351044in}}%
\pgfpathlineto{\pgfqpoint{2.048717in}{2.301711in}}%
\pgfpathlineto{\pgfqpoint{2.075995in}{2.301711in}}%
\pgfpathlineto{\pgfqpoint{2.075995in}{2.215377in}}%
\pgfpathlineto{\pgfqpoint{2.103273in}{2.215377in}}%
\pgfpathlineto{\pgfqpoint{2.103273in}{2.161932in}}%
\pgfpathlineto{\pgfqpoint{2.130551in}{2.161932in}}%
\pgfpathlineto{\pgfqpoint{2.130551in}{2.215377in}}%
\pgfpathlineto{\pgfqpoint{2.157829in}{2.215377in}}%
\pgfpathlineto{\pgfqpoint{2.157829in}{2.184543in}}%
\pgfpathlineto{\pgfqpoint{2.185107in}{2.184543in}}%
\pgfpathlineto{\pgfqpoint{2.185107in}{2.151654in}}%
\pgfpathlineto{\pgfqpoint{2.212385in}{2.151654in}}%
\pgfpathlineto{\pgfqpoint{2.212385in}{2.233877in}}%
\pgfpathlineto{\pgfqpoint{2.239663in}{2.233877in}}%
\pgfpathlineto{\pgfqpoint{2.239663in}{2.215377in}}%
\pgfpathlineto{\pgfqpoint{2.266941in}{2.215377in}}%
\pgfpathlineto{\pgfqpoint{2.266941in}{2.207154in}}%
\pgfpathlineto{\pgfqpoint{2.294219in}{2.207154in}}%
\pgfpathlineto{\pgfqpoint{2.294219in}{2.198932in}}%
\pgfpathlineto{\pgfqpoint{2.321497in}{2.198932in}}%
\pgfpathlineto{\pgfqpoint{2.321497in}{2.135209in}}%
\pgfpathlineto{\pgfqpoint{2.348775in}{2.135209in}}%
\pgfpathlineto{\pgfqpoint{2.348775in}{2.059153in}}%
\pgfpathlineto{\pgfqpoint{2.376053in}{2.059153in}}%
\pgfpathlineto{\pgfqpoint{2.376053in}{1.962541in}}%
\pgfpathlineto{\pgfqpoint{2.403331in}{1.962541in}}%
\pgfpathlineto{\pgfqpoint{2.403331in}{2.046820in}}%
\pgfpathlineto{\pgfqpoint{2.430609in}{2.046820in}}%
\pgfpathlineto{\pgfqpoint{2.430609in}{2.063264in}}%
\pgfpathlineto{\pgfqpoint{2.457887in}{2.063264in}}%
\pgfpathlineto{\pgfqpoint{2.457887in}{1.987208in}}%
\pgfpathlineto{\pgfqpoint{2.485165in}{1.987208in}}%
\pgfpathlineto{\pgfqpoint{2.485165in}{1.987208in}}%
\pgfpathlineto{\pgfqpoint{2.512443in}{1.987208in}}%
\pgfpathlineto{\pgfqpoint{2.512443in}{2.020097in}}%
\pgfpathlineto{\pgfqpoint{2.539721in}{2.020097in}}%
\pgfpathlineto{\pgfqpoint{2.539721in}{1.985153in}}%
\pgfpathlineto{\pgfqpoint{2.566999in}{1.985153in}}%
\pgfpathlineto{\pgfqpoint{2.566999in}{1.894708in}}%
\pgfpathlineto{\pgfqpoint{2.594277in}{1.894708in}}%
\pgfpathlineto{\pgfqpoint{2.594277in}{1.890596in}}%
\pgfpathlineto{\pgfqpoint{2.621555in}{1.890596in}}%
\pgfpathlineto{\pgfqpoint{2.621555in}{1.867985in}}%
\pgfpathlineto{\pgfqpoint{2.648833in}{1.867985in}}%
\pgfpathlineto{\pgfqpoint{2.648833in}{1.882374in}}%
\pgfpathlineto{\pgfqpoint{2.676111in}{1.882374in}}%
\pgfpathlineto{\pgfqpoint{2.676111in}{1.880319in}}%
\pgfpathlineto{\pgfqpoint{2.703389in}{1.880319in}}%
\pgfpathlineto{\pgfqpoint{2.703389in}{1.812485in}}%
\pgfpathlineto{\pgfqpoint{2.730667in}{1.812485in}}%
\pgfpathlineto{\pgfqpoint{2.730667in}{1.841263in}}%
\pgfpathlineto{\pgfqpoint{2.757945in}{1.841263in}}%
\pgfpathlineto{\pgfqpoint{2.757945in}{1.824818in}}%
\pgfpathlineto{\pgfqpoint{2.785223in}{1.824818in}}%
\pgfpathlineto{\pgfqpoint{2.785223in}{1.703539in}}%
\pgfpathlineto{\pgfqpoint{2.812501in}{1.703539in}}%
\pgfpathlineto{\pgfqpoint{2.812501in}{1.738484in}}%
\pgfpathlineto{\pgfqpoint{2.839779in}{1.738484in}}%
\pgfpathlineto{\pgfqpoint{2.839779in}{1.615150in}}%
\pgfpathlineto{\pgfqpoint{2.867057in}{1.615150in}}%
\pgfpathlineto{\pgfqpoint{2.867057in}{1.662428in}}%
\pgfpathlineto{\pgfqpoint{2.894335in}{1.662428in}}%
\pgfpathlineto{\pgfqpoint{2.894335in}{1.526760in}}%
\pgfpathlineto{\pgfqpoint{2.921613in}{1.526760in}}%
\pgfpathlineto{\pgfqpoint{2.921613in}{1.518538in}}%
\pgfpathlineto{\pgfqpoint{2.948891in}{1.518538in}}%
\pgfpathlineto{\pgfqpoint{2.948891in}{1.522649in}}%
\pgfpathlineto{\pgfqpoint{2.976169in}{1.522649in}}%
\pgfpathlineto{\pgfqpoint{2.976169in}{1.514427in}}%
\pgfpathlineto{\pgfqpoint{3.003447in}{1.514427in}}%
\pgfpathlineto{\pgfqpoint{3.003447in}{1.436315in}}%
\pgfpathlineto{\pgfqpoint{3.030725in}{1.436315in}}%
\pgfpathlineto{\pgfqpoint{3.030725in}{1.319148in}}%
\pgfpathlineto{\pgfqpoint{3.058003in}{1.319148in}}%
\pgfpathlineto{\pgfqpoint{3.058003in}{1.321203in}}%
\pgfpathlineto{\pgfqpoint{3.085281in}{1.321203in}}%
\pgfpathlineto{\pgfqpoint{3.085281in}{1.259536in}}%
\pgfpathlineto{\pgfqpoint{3.112559in}{1.259536in}}%
\pgfpathlineto{\pgfqpoint{3.112559in}{1.204036in}}%
\pgfpathlineto{\pgfqpoint{3.139837in}{1.204036in}}%
\pgfpathlineto{\pgfqpoint{3.139837in}{1.136202in}}%
\pgfpathlineto{\pgfqpoint{3.167115in}{1.136202in}}%
\pgfpathlineto{\pgfqpoint{3.167115in}{1.074535in}}%
\pgfpathlineto{\pgfqpoint{3.194393in}{1.074535in}}%
\pgfpathlineto{\pgfqpoint{3.194393in}{0.885422in}}%
\pgfpathlineto{\pgfqpoint{3.221671in}{0.885422in}}%
\pgfpathlineto{\pgfqpoint{3.221671in}{0.815533in}}%
\pgfpathlineto{\pgfqpoint{3.248949in}{0.815533in}}%
\pgfpathlineto{\pgfqpoint{3.248949in}{0.659309in}}%
\pgfpathlineto{\pgfqpoint{3.276227in}{0.659309in}}%
\pgfpathlineto{\pgfqpoint{3.276227in}{0.503086in}}%
\pgfpathlineto{\pgfqpoint{3.303505in}{0.503086in}}%
\pgfpathlineto{\pgfqpoint{3.303505in}{0.449641in}}%
\pgfpathlineto{\pgfqpoint{3.330783in}{0.449641in}}%
\pgfpathlineto{\pgfqpoint{3.330783in}{0.441418in}}%
\pgfpathlineto{\pgfqpoint{3.303505in}{0.441418in}}%
\pgfpathlineto{\pgfqpoint{3.303505in}{0.441418in}}%
\pgfpathlineto{\pgfqpoint{3.276227in}{0.441418in}}%
\pgfpathlineto{\pgfqpoint{3.276227in}{0.441418in}}%
\pgfpathlineto{\pgfqpoint{3.248949in}{0.441418in}}%
\pgfpathlineto{\pgfqpoint{3.248949in}{0.441418in}}%
\pgfpathlineto{\pgfqpoint{3.221671in}{0.441418in}}%
\pgfpathlineto{\pgfqpoint{3.221671in}{0.441418in}}%
\pgfpathlineto{\pgfqpoint{3.194393in}{0.441418in}}%
\pgfpathlineto{\pgfqpoint{3.194393in}{0.441418in}}%
\pgfpathlineto{\pgfqpoint{3.167115in}{0.441418in}}%
\pgfpathlineto{\pgfqpoint{3.167115in}{0.441418in}}%
\pgfpathlineto{\pgfqpoint{3.139837in}{0.441418in}}%
\pgfpathlineto{\pgfqpoint{3.139837in}{0.441418in}}%
\pgfpathlineto{\pgfqpoint{3.112559in}{0.441418in}}%
\pgfpathlineto{\pgfqpoint{3.112559in}{0.441418in}}%
\pgfpathlineto{\pgfqpoint{3.085281in}{0.441418in}}%
\pgfpathlineto{\pgfqpoint{3.085281in}{0.441418in}}%
\pgfpathlineto{\pgfqpoint{3.058003in}{0.441418in}}%
\pgfpathlineto{\pgfqpoint{3.058003in}{0.441418in}}%
\pgfpathlineto{\pgfqpoint{3.030725in}{0.441418in}}%
\pgfpathlineto{\pgfqpoint{3.030725in}{0.441418in}}%
\pgfpathlineto{\pgfqpoint{3.003447in}{0.441418in}}%
\pgfpathlineto{\pgfqpoint{3.003447in}{0.441418in}}%
\pgfpathlineto{\pgfqpoint{2.976169in}{0.441418in}}%
\pgfpathlineto{\pgfqpoint{2.976169in}{0.441418in}}%
\pgfpathlineto{\pgfqpoint{2.948891in}{0.441418in}}%
\pgfpathlineto{\pgfqpoint{2.948891in}{0.441418in}}%
\pgfpathlineto{\pgfqpoint{2.921613in}{0.441418in}}%
\pgfpathlineto{\pgfqpoint{2.921613in}{0.441418in}}%
\pgfpathlineto{\pgfqpoint{2.894335in}{0.441418in}}%
\pgfpathlineto{\pgfqpoint{2.894335in}{0.441418in}}%
\pgfpathlineto{\pgfqpoint{2.867057in}{0.441418in}}%
\pgfpathlineto{\pgfqpoint{2.867057in}{0.441418in}}%
\pgfpathlineto{\pgfqpoint{2.839779in}{0.441418in}}%
\pgfpathlineto{\pgfqpoint{2.839779in}{0.441418in}}%
\pgfpathlineto{\pgfqpoint{2.812501in}{0.441418in}}%
\pgfpathlineto{\pgfqpoint{2.812501in}{0.441418in}}%
\pgfpathlineto{\pgfqpoint{2.785223in}{0.441418in}}%
\pgfpathlineto{\pgfqpoint{2.785223in}{0.441418in}}%
\pgfpathlineto{\pgfqpoint{2.757945in}{0.441418in}}%
\pgfpathlineto{\pgfqpoint{2.757945in}{0.441418in}}%
\pgfpathlineto{\pgfqpoint{2.730667in}{0.441418in}}%
\pgfpathlineto{\pgfqpoint{2.730667in}{0.441418in}}%
\pgfpathlineto{\pgfqpoint{2.703389in}{0.441418in}}%
\pgfpathlineto{\pgfqpoint{2.703389in}{0.441418in}}%
\pgfpathlineto{\pgfqpoint{2.676111in}{0.441418in}}%
\pgfpathlineto{\pgfqpoint{2.676111in}{0.441418in}}%
\pgfpathlineto{\pgfqpoint{2.648833in}{0.441418in}}%
\pgfpathlineto{\pgfqpoint{2.648833in}{0.441418in}}%
\pgfpathlineto{\pgfqpoint{2.621555in}{0.441418in}}%
\pgfpathlineto{\pgfqpoint{2.621555in}{0.441418in}}%
\pgfpathlineto{\pgfqpoint{2.594277in}{0.441418in}}%
\pgfpathlineto{\pgfqpoint{2.594277in}{0.441418in}}%
\pgfpathlineto{\pgfqpoint{2.566999in}{0.441418in}}%
\pgfpathlineto{\pgfqpoint{2.566999in}{0.441418in}}%
\pgfpathlineto{\pgfqpoint{2.539721in}{0.441418in}}%
\pgfpathlineto{\pgfqpoint{2.539721in}{0.441418in}}%
\pgfpathlineto{\pgfqpoint{2.512443in}{0.441418in}}%
\pgfpathlineto{\pgfqpoint{2.512443in}{0.441418in}}%
\pgfpathlineto{\pgfqpoint{2.485165in}{0.441418in}}%
\pgfpathlineto{\pgfqpoint{2.485165in}{0.441418in}}%
\pgfpathlineto{\pgfqpoint{2.457887in}{0.441418in}}%
\pgfpathlineto{\pgfqpoint{2.457887in}{0.441418in}}%
\pgfpathlineto{\pgfqpoint{2.430609in}{0.441418in}}%
\pgfpathlineto{\pgfqpoint{2.430609in}{0.441418in}}%
\pgfpathlineto{\pgfqpoint{2.403331in}{0.441418in}}%
\pgfpathlineto{\pgfqpoint{2.403331in}{0.441418in}}%
\pgfpathlineto{\pgfqpoint{2.376053in}{0.441418in}}%
\pgfpathlineto{\pgfqpoint{2.376053in}{0.441418in}}%
\pgfpathlineto{\pgfqpoint{2.348775in}{0.441418in}}%
\pgfpathlineto{\pgfqpoint{2.348775in}{0.441418in}}%
\pgfpathlineto{\pgfqpoint{2.321497in}{0.441418in}}%
\pgfpathlineto{\pgfqpoint{2.321497in}{0.441418in}}%
\pgfpathlineto{\pgfqpoint{2.294219in}{0.441418in}}%
\pgfpathlineto{\pgfqpoint{2.294219in}{0.441418in}}%
\pgfpathlineto{\pgfqpoint{2.266941in}{0.441418in}}%
\pgfpathlineto{\pgfqpoint{2.266941in}{0.441418in}}%
\pgfpathlineto{\pgfqpoint{2.239663in}{0.441418in}}%
\pgfpathlineto{\pgfqpoint{2.239663in}{0.441418in}}%
\pgfpathlineto{\pgfqpoint{2.212385in}{0.441418in}}%
\pgfpathlineto{\pgfqpoint{2.212385in}{0.441418in}}%
\pgfpathlineto{\pgfqpoint{2.185107in}{0.441418in}}%
\pgfpathlineto{\pgfqpoint{2.185107in}{0.441418in}}%
\pgfpathlineto{\pgfqpoint{2.157829in}{0.441418in}}%
\pgfpathlineto{\pgfqpoint{2.157829in}{0.441418in}}%
\pgfpathlineto{\pgfqpoint{2.130551in}{0.441418in}}%
\pgfpathlineto{\pgfqpoint{2.130551in}{0.441418in}}%
\pgfpathlineto{\pgfqpoint{2.103273in}{0.441418in}}%
\pgfpathlineto{\pgfqpoint{2.103273in}{0.441418in}}%
\pgfpathlineto{\pgfqpoint{2.075995in}{0.441418in}}%
\pgfpathlineto{\pgfqpoint{2.075995in}{0.441418in}}%
\pgfpathlineto{\pgfqpoint{2.048717in}{0.441418in}}%
\pgfpathlineto{\pgfqpoint{2.048717in}{0.441418in}}%
\pgfpathlineto{\pgfqpoint{2.021439in}{0.441418in}}%
\pgfpathlineto{\pgfqpoint{2.021439in}{0.441418in}}%
\pgfpathlineto{\pgfqpoint{1.994161in}{0.441418in}}%
\pgfpathlineto{\pgfqpoint{1.994161in}{0.441418in}}%
\pgfpathlineto{\pgfqpoint{1.966883in}{0.441418in}}%
\pgfpathlineto{\pgfqpoint{1.966883in}{0.441418in}}%
\pgfpathlineto{\pgfqpoint{1.939605in}{0.441418in}}%
\pgfpathlineto{\pgfqpoint{1.939605in}{0.441418in}}%
\pgfpathlineto{\pgfqpoint{1.912327in}{0.441418in}}%
\pgfpathlineto{\pgfqpoint{1.912327in}{0.441418in}}%
\pgfpathlineto{\pgfqpoint{1.885049in}{0.441418in}}%
\pgfpathlineto{\pgfqpoint{1.885049in}{0.441418in}}%
\pgfpathlineto{\pgfqpoint{1.857771in}{0.441418in}}%
\pgfpathlineto{\pgfqpoint{1.857771in}{0.441418in}}%
\pgfpathlineto{\pgfqpoint{1.830493in}{0.441418in}}%
\pgfpathlineto{\pgfqpoint{1.830493in}{0.441418in}}%
\pgfpathlineto{\pgfqpoint{1.803215in}{0.441418in}}%
\pgfpathlineto{\pgfqpoint{1.803215in}{0.441418in}}%
\pgfpathlineto{\pgfqpoint{1.775937in}{0.441418in}}%
\pgfpathlineto{\pgfqpoint{1.775937in}{0.441418in}}%
\pgfpathlineto{\pgfqpoint{1.748659in}{0.441418in}}%
\pgfpathlineto{\pgfqpoint{1.748659in}{0.441418in}}%
\pgfpathlineto{\pgfqpoint{1.721381in}{0.441418in}}%
\pgfpathlineto{\pgfqpoint{1.721381in}{0.441418in}}%
\pgfpathlineto{\pgfqpoint{1.694103in}{0.441418in}}%
\pgfpathlineto{\pgfqpoint{1.694103in}{0.441418in}}%
\pgfpathlineto{\pgfqpoint{1.666825in}{0.441418in}}%
\pgfpathlineto{\pgfqpoint{1.666825in}{0.441418in}}%
\pgfpathlineto{\pgfqpoint{1.639547in}{0.441418in}}%
\pgfpathlineto{\pgfqpoint{1.639547in}{0.441418in}}%
\pgfpathlineto{\pgfqpoint{1.612269in}{0.441418in}}%
\pgfpathlineto{\pgfqpoint{1.612269in}{0.441418in}}%
\pgfpathlineto{\pgfqpoint{1.584991in}{0.441418in}}%
\pgfpathlineto{\pgfqpoint{1.584991in}{0.441418in}}%
\pgfpathlineto{\pgfqpoint{1.557713in}{0.441418in}}%
\pgfpathlineto{\pgfqpoint{1.557713in}{0.441418in}}%
\pgfpathlineto{\pgfqpoint{1.530435in}{0.441418in}}%
\pgfpathlineto{\pgfqpoint{1.530435in}{0.441418in}}%
\pgfpathlineto{\pgfqpoint{1.503157in}{0.441418in}}%
\pgfpathlineto{\pgfqpoint{1.503157in}{0.441418in}}%
\pgfpathlineto{\pgfqpoint{1.475879in}{0.441418in}}%
\pgfpathlineto{\pgfqpoint{1.475879in}{0.441418in}}%
\pgfpathlineto{\pgfqpoint{1.448601in}{0.441418in}}%
\pgfpathlineto{\pgfqpoint{1.448601in}{0.441418in}}%
\pgfpathlineto{\pgfqpoint{1.421323in}{0.441418in}}%
\pgfpathlineto{\pgfqpoint{1.421323in}{0.441418in}}%
\pgfpathlineto{\pgfqpoint{1.394045in}{0.441418in}}%
\pgfpathlineto{\pgfqpoint{1.394045in}{0.441418in}}%
\pgfpathlineto{\pgfqpoint{1.366767in}{0.441418in}}%
\pgfpathlineto{\pgfqpoint{1.366767in}{0.441418in}}%
\pgfpathlineto{\pgfqpoint{1.339489in}{0.441418in}}%
\pgfpathlineto{\pgfqpoint{1.339489in}{0.441418in}}%
\pgfpathlineto{\pgfqpoint{1.312211in}{0.441418in}}%
\pgfpathlineto{\pgfqpoint{1.312211in}{0.441418in}}%
\pgfpathlineto{\pgfqpoint{1.284933in}{0.441418in}}%
\pgfpathlineto{\pgfqpoint{1.284933in}{0.441418in}}%
\pgfpathlineto{\pgfqpoint{1.257655in}{0.441418in}}%
\pgfpathlineto{\pgfqpoint{1.257655in}{0.441418in}}%
\pgfpathlineto{\pgfqpoint{1.230377in}{0.441418in}}%
\pgfpathlineto{\pgfqpoint{1.230377in}{0.441418in}}%
\pgfpathlineto{\pgfqpoint{1.203099in}{0.441418in}}%
\pgfpathlineto{\pgfqpoint{1.203099in}{0.441418in}}%
\pgfpathlineto{\pgfqpoint{1.175821in}{0.441418in}}%
\pgfpathlineto{\pgfqpoint{1.175821in}{0.441418in}}%
\pgfpathlineto{\pgfqpoint{1.148543in}{0.441418in}}%
\pgfpathlineto{\pgfqpoint{1.148543in}{0.441418in}}%
\pgfpathlineto{\pgfqpoint{1.121265in}{0.441418in}}%
\pgfpathlineto{\pgfqpoint{1.121265in}{0.441418in}}%
\pgfpathlineto{\pgfqpoint{1.093987in}{0.441418in}}%
\pgfpathlineto{\pgfqpoint{1.093987in}{0.441418in}}%
\pgfpathlineto{\pgfqpoint{1.066709in}{0.441418in}}%
\pgfpathlineto{\pgfqpoint{1.066709in}{0.441418in}}%
\pgfpathlineto{\pgfqpoint{1.039431in}{0.441418in}}%
\pgfpathlineto{\pgfqpoint{1.039431in}{0.441418in}}%
\pgfpathlineto{\pgfqpoint{1.012153in}{0.441418in}}%
\pgfpathlineto{\pgfqpoint{1.012153in}{0.441418in}}%
\pgfpathlineto{\pgfqpoint{0.984875in}{0.441418in}}%
\pgfpathlineto{\pgfqpoint{0.984875in}{0.441418in}}%
\pgfpathlineto{\pgfqpoint{0.957597in}{0.441418in}}%
\pgfpathlineto{\pgfqpoint{0.957597in}{0.441418in}}%
\pgfpathlineto{\pgfqpoint{0.930319in}{0.441418in}}%
\pgfpathlineto{\pgfqpoint{0.930319in}{0.441418in}}%
\pgfpathlineto{\pgfqpoint{0.903041in}{0.441418in}}%
\pgfpathlineto{\pgfqpoint{0.903041in}{0.441418in}}%
\pgfpathlineto{\pgfqpoint{0.875763in}{0.441418in}}%
\pgfpathlineto{\pgfqpoint{0.875763in}{0.441418in}}%
\pgfpathlineto{\pgfqpoint{0.848485in}{0.441418in}}%
\pgfpathlineto{\pgfqpoint{0.848485in}{0.441418in}}%
\pgfpathlineto{\pgfqpoint{0.821207in}{0.441418in}}%
\pgfpathlineto{\pgfqpoint{0.821207in}{0.441418in}}%
\pgfpathlineto{\pgfqpoint{0.793929in}{0.441418in}}%
\pgfpathlineto{\pgfqpoint{0.793929in}{0.441418in}}%
\pgfpathlineto{\pgfqpoint{0.766651in}{0.441418in}}%
\pgfpathlineto{\pgfqpoint{0.766651in}{0.441418in}}%
\pgfpathlineto{\pgfqpoint{0.739373in}{0.441418in}}%
\pgfpathlineto{\pgfqpoint{0.739373in}{0.441418in}}%
\pgfpathlineto{\pgfqpoint{0.712095in}{0.441418in}}%
\pgfpathlineto{\pgfqpoint{0.712095in}{0.441418in}}%
\pgfpathlineto{\pgfqpoint{0.684817in}{0.441418in}}%
\pgfpathlineto{\pgfqpoint{0.684817in}{0.441418in}}%
\pgfpathlineto{\pgfqpoint{0.657539in}{0.441418in}}%
\pgfpathlineto{\pgfqpoint{0.657539in}{0.441418in}}%
\pgfpathlineto{\pgfqpoint{0.630261in}{0.441418in}}%
\pgfpathlineto{\pgfqpoint{0.630261in}{0.441418in}}%
\pgfpathlineto{\pgfqpoint{0.602983in}{0.441418in}}%
\pgfusepath{stroke,fill}%
\end{pgfscope}%
\begin{pgfscope}%
\pgfpathrectangle{\pgfqpoint{0.592630in}{0.441418in}}{\pgfqpoint{3.222048in}{2.055572in}} %
\pgfusepath{clip}%
\pgfsetbuttcap%
\pgfsetmiterjoin%
\definecolor{currentfill}{rgb}{1.000000,0.000000,0.000000}%
\pgfsetfillcolor{currentfill}%
\pgfsetfillopacity{0.100000}%
\pgfsetlinewidth{1.003750pt}%
\definecolor{currentstroke}{rgb}{1.000000,0.000000,0.000000}%
\pgfsetstrokecolor{currentstroke}%
\pgfsetstrokeopacity{0.100000}%
\pgfsetdash{}{0pt}%
\pgfpathmoveto{\pgfqpoint{2.528160in}{0.441418in}}%
\pgfpathlineto{\pgfqpoint{2.528160in}{2.496990in}}%
\pgfpathlineto{\pgfqpoint{2.949327in}{2.496990in}}%
\pgfpathlineto{\pgfqpoint{2.949327in}{0.441418in}}%
\pgfpathlineto{\pgfqpoint{2.528160in}{0.441418in}}%
\pgfusepath{stroke,fill}%
\end{pgfscope}%
\begin{pgfscope}%
\pgfpathrectangle{\pgfqpoint{0.592630in}{0.441418in}}{\pgfqpoint{3.222048in}{2.055572in}} %
\pgfusepath{clip}%
\pgfsetbuttcap%
\pgfsetmiterjoin%
\definecolor{currentfill}{rgb}{1.000000,0.000000,0.000000}%
\pgfsetfillcolor{currentfill}%
\pgfsetfillopacity{0.100000}%
\pgfsetlinewidth{1.003750pt}%
\definecolor{currentstroke}{rgb}{1.000000,0.000000,0.000000}%
\pgfsetstrokecolor{currentstroke}%
\pgfsetstrokeopacity{0.100000}%
\pgfsetdash{}{0pt}%
\pgfpathmoveto{\pgfqpoint{3.411921in}{0.441418in}}%
\pgfpathlineto{\pgfqpoint{3.411921in}{2.496990in}}%
\pgfpathlineto{\pgfqpoint{3.814677in}{2.496990in}}%
\pgfpathlineto{\pgfqpoint{3.814677in}{0.441418in}}%
\pgfpathlineto{\pgfqpoint{3.411921in}{0.441418in}}%
\pgfusepath{stroke,fill}%
\end{pgfscope}%
\begin{pgfscope}%
\pgfpathrectangle{\pgfqpoint{0.592630in}{0.441418in}}{\pgfqpoint{3.222048in}{2.055572in}} %
\pgfusepath{clip}%
\pgfsetrectcap%
\pgfsetroundjoin%
\pgfsetlinewidth{1.003750pt}%
\definecolor{currentstroke}{rgb}{1.000000,0.000000,0.000000}%
\pgfsetstrokecolor{currentstroke}%
\pgfsetdash{}{0pt}%
\pgfpathmoveto{\pgfqpoint{2.528160in}{0.441418in}}%
\pgfpathlineto{\pgfqpoint{2.528160in}{2.496990in}}%
\pgfusepath{stroke}%
\end{pgfscope}%
\begin{pgfscope}%
\pgfpathrectangle{\pgfqpoint{0.592630in}{0.441418in}}{\pgfqpoint{3.222048in}{2.055572in}} %
\pgfusepath{clip}%
\pgfsetrectcap%
\pgfsetroundjoin%
\pgfsetlinewidth{1.003750pt}%
\definecolor{currentstroke}{rgb}{1.000000,0.000000,0.000000}%
\pgfsetstrokecolor{currentstroke}%
\pgfsetdash{}{0pt}%
\pgfpathmoveto{\pgfqpoint{2.949327in}{0.441418in}}%
\pgfpathlineto{\pgfqpoint{2.949327in}{2.496990in}}%
\pgfusepath{stroke}%
\end{pgfscope}%
\begin{pgfscope}%
\pgfpathrectangle{\pgfqpoint{0.592630in}{0.441418in}}{\pgfqpoint{3.222048in}{2.055572in}} %
\pgfusepath{clip}%
\pgfsetrectcap%
\pgfsetroundjoin%
\pgfsetlinewidth{1.003750pt}%
\definecolor{currentstroke}{rgb}{1.000000,0.000000,0.000000}%
\pgfsetstrokecolor{currentstroke}%
\pgfsetdash{}{0pt}%
\pgfpathmoveto{\pgfqpoint{3.411921in}{0.441418in}}%
\pgfpathlineto{\pgfqpoint{3.411921in}{2.496990in}}%
\pgfusepath{stroke}%
\end{pgfscope}%
\begin{pgfscope}%
\pgfsetrectcap%
\pgfsetmiterjoin%
\pgfsetlinewidth{1.003750pt}%
\definecolor{currentstroke}{rgb}{0.000000,0.000000,0.000000}%
\pgfsetstrokecolor{currentstroke}%
\pgfsetdash{}{0pt}%
\pgfpathmoveto{\pgfqpoint{0.592630in}{2.496990in}}%
\pgfpathlineto{\pgfqpoint{3.814677in}{2.496990in}}%
\pgfusepath{stroke}%
\end{pgfscope}%
\begin{pgfscope}%
\pgfsetrectcap%
\pgfsetmiterjoin%
\pgfsetlinewidth{1.003750pt}%
\definecolor{currentstroke}{rgb}{0.000000,0.000000,0.000000}%
\pgfsetstrokecolor{currentstroke}%
\pgfsetdash{}{0pt}%
\pgfpathmoveto{\pgfqpoint{3.814677in}{0.441418in}}%
\pgfpathlineto{\pgfqpoint{3.814677in}{2.496990in}}%
\pgfusepath{stroke}%
\end{pgfscope}%
\begin{pgfscope}%
\pgfsetrectcap%
\pgfsetmiterjoin%
\pgfsetlinewidth{1.003750pt}%
\definecolor{currentstroke}{rgb}{0.000000,0.000000,0.000000}%
\pgfsetstrokecolor{currentstroke}%
\pgfsetdash{}{0pt}%
\pgfpathmoveto{\pgfqpoint{0.592630in}{0.441418in}}%
\pgfpathlineto{\pgfqpoint{3.814677in}{0.441418in}}%
\pgfusepath{stroke}%
\end{pgfscope}%
\begin{pgfscope}%
\pgfsetrectcap%
\pgfsetmiterjoin%
\pgfsetlinewidth{1.003750pt}%
\definecolor{currentstroke}{rgb}{0.000000,0.000000,0.000000}%
\pgfsetstrokecolor{currentstroke}%
\pgfsetdash{}{0pt}%
\pgfpathmoveto{\pgfqpoint{0.592630in}{0.441418in}}%
\pgfpathlineto{\pgfqpoint{0.592630in}{2.496990in}}%
\pgfusepath{stroke}%
\end{pgfscope}%
\begin{pgfscope}%
\pgfsetbuttcap%
\pgfsetroundjoin%
\definecolor{currentfill}{rgb}{0.000000,0.000000,0.000000}%
\pgfsetfillcolor{currentfill}%
\pgfsetlinewidth{0.501875pt}%
\definecolor{currentstroke}{rgb}{0.000000,0.000000,0.000000}%
\pgfsetstrokecolor{currentstroke}%
\pgfsetdash{}{0pt}%
\pgfsys@defobject{currentmarker}{\pgfqpoint{0.000000in}{0.000000in}}{\pgfqpoint{0.000000in}{0.069444in}}{%
\pgfpathmoveto{\pgfqpoint{0.000000in}{0.000000in}}%
\pgfpathlineto{\pgfqpoint{0.000000in}{0.069444in}}%
\pgfusepath{stroke,fill}%
}%
\begin{pgfscope}%
\pgfsys@transformshift{0.592630in}{0.441418in}%
\pgfsys@useobject{currentmarker}{}%
\end{pgfscope}%
\end{pgfscope}%
\begin{pgfscope}%
\pgfsetbuttcap%
\pgfsetroundjoin%
\definecolor{currentfill}{rgb}{0.000000,0.000000,0.000000}%
\pgfsetfillcolor{currentfill}%
\pgfsetlinewidth{0.501875pt}%
\definecolor{currentstroke}{rgb}{0.000000,0.000000,0.000000}%
\pgfsetstrokecolor{currentstroke}%
\pgfsetdash{}{0pt}%
\pgfsys@defobject{currentmarker}{\pgfqpoint{0.000000in}{-0.069444in}}{\pgfqpoint{0.000000in}{0.000000in}}{%
\pgfpathmoveto{\pgfqpoint{0.000000in}{0.000000in}}%
\pgfpathlineto{\pgfqpoint{0.000000in}{-0.069444in}}%
\pgfusepath{stroke,fill}%
}%
\begin{pgfscope}%
\pgfsys@transformshift{0.592630in}{2.496990in}%
\pgfsys@useobject{currentmarker}{}%
\end{pgfscope}%
\end{pgfscope}%
\begin{pgfscope}%
\pgftext[x=0.592630in,y=0.371974in,,top]{\rmfamily\fontsize{8.000000}{9.600000}\selectfont 0}%
\end{pgfscope}%
\begin{pgfscope}%
\pgfsetbuttcap%
\pgfsetroundjoin%
\definecolor{currentfill}{rgb}{0.000000,0.000000,0.000000}%
\pgfsetfillcolor{currentfill}%
\pgfsetlinewidth{0.501875pt}%
\definecolor{currentstroke}{rgb}{0.000000,0.000000,0.000000}%
\pgfsetstrokecolor{currentstroke}%
\pgfsetdash{}{0pt}%
\pgfsys@defobject{currentmarker}{\pgfqpoint{0.000000in}{0.000000in}}{\pgfqpoint{0.000000in}{0.069444in}}{%
\pgfpathmoveto{\pgfqpoint{0.000000in}{0.000000in}}%
\pgfpathlineto{\pgfqpoint{0.000000in}{0.069444in}}%
\pgfusepath{stroke,fill}%
}%
\begin{pgfscope}%
\pgfsys@transformshift{1.052922in}{0.441418in}%
\pgfsys@useobject{currentmarker}{}%
\end{pgfscope}%
\end{pgfscope}%
\begin{pgfscope}%
\pgfsetbuttcap%
\pgfsetroundjoin%
\definecolor{currentfill}{rgb}{0.000000,0.000000,0.000000}%
\pgfsetfillcolor{currentfill}%
\pgfsetlinewidth{0.501875pt}%
\definecolor{currentstroke}{rgb}{0.000000,0.000000,0.000000}%
\pgfsetstrokecolor{currentstroke}%
\pgfsetdash{}{0pt}%
\pgfsys@defobject{currentmarker}{\pgfqpoint{0.000000in}{-0.069444in}}{\pgfqpoint{0.000000in}{0.000000in}}{%
\pgfpathmoveto{\pgfqpoint{0.000000in}{0.000000in}}%
\pgfpathlineto{\pgfqpoint{0.000000in}{-0.069444in}}%
\pgfusepath{stroke,fill}%
}%
\begin{pgfscope}%
\pgfsys@transformshift{1.052922in}{2.496990in}%
\pgfsys@useobject{currentmarker}{}%
\end{pgfscope}%
\end{pgfscope}%
\begin{pgfscope}%
\pgftext[x=1.052922in,y=0.371974in,,top]{\rmfamily\fontsize{8.000000}{9.600000}\selectfont 2}%
\end{pgfscope}%
\begin{pgfscope}%
\pgfsetbuttcap%
\pgfsetroundjoin%
\definecolor{currentfill}{rgb}{0.000000,0.000000,0.000000}%
\pgfsetfillcolor{currentfill}%
\pgfsetlinewidth{0.501875pt}%
\definecolor{currentstroke}{rgb}{0.000000,0.000000,0.000000}%
\pgfsetstrokecolor{currentstroke}%
\pgfsetdash{}{0pt}%
\pgfsys@defobject{currentmarker}{\pgfqpoint{0.000000in}{0.000000in}}{\pgfqpoint{0.000000in}{0.069444in}}{%
\pgfpathmoveto{\pgfqpoint{0.000000in}{0.000000in}}%
\pgfpathlineto{\pgfqpoint{0.000000in}{0.069444in}}%
\pgfusepath{stroke,fill}%
}%
\begin{pgfscope}%
\pgfsys@transformshift{1.513215in}{0.441418in}%
\pgfsys@useobject{currentmarker}{}%
\end{pgfscope}%
\end{pgfscope}%
\begin{pgfscope}%
\pgfsetbuttcap%
\pgfsetroundjoin%
\definecolor{currentfill}{rgb}{0.000000,0.000000,0.000000}%
\pgfsetfillcolor{currentfill}%
\pgfsetlinewidth{0.501875pt}%
\definecolor{currentstroke}{rgb}{0.000000,0.000000,0.000000}%
\pgfsetstrokecolor{currentstroke}%
\pgfsetdash{}{0pt}%
\pgfsys@defobject{currentmarker}{\pgfqpoint{0.000000in}{-0.069444in}}{\pgfqpoint{0.000000in}{0.000000in}}{%
\pgfpathmoveto{\pgfqpoint{0.000000in}{0.000000in}}%
\pgfpathlineto{\pgfqpoint{0.000000in}{-0.069444in}}%
\pgfusepath{stroke,fill}%
}%
\begin{pgfscope}%
\pgfsys@transformshift{1.513215in}{2.496990in}%
\pgfsys@useobject{currentmarker}{}%
\end{pgfscope}%
\end{pgfscope}%
\begin{pgfscope}%
\pgftext[x=1.513215in,y=0.371974in,,top]{\rmfamily\fontsize{8.000000}{9.600000}\selectfont 4}%
\end{pgfscope}%
\begin{pgfscope}%
\pgfsetbuttcap%
\pgfsetroundjoin%
\definecolor{currentfill}{rgb}{0.000000,0.000000,0.000000}%
\pgfsetfillcolor{currentfill}%
\pgfsetlinewidth{0.501875pt}%
\definecolor{currentstroke}{rgb}{0.000000,0.000000,0.000000}%
\pgfsetstrokecolor{currentstroke}%
\pgfsetdash{}{0pt}%
\pgfsys@defobject{currentmarker}{\pgfqpoint{0.000000in}{0.000000in}}{\pgfqpoint{0.000000in}{0.069444in}}{%
\pgfpathmoveto{\pgfqpoint{0.000000in}{0.000000in}}%
\pgfpathlineto{\pgfqpoint{0.000000in}{0.069444in}}%
\pgfusepath{stroke,fill}%
}%
\begin{pgfscope}%
\pgfsys@transformshift{1.973507in}{0.441418in}%
\pgfsys@useobject{currentmarker}{}%
\end{pgfscope}%
\end{pgfscope}%
\begin{pgfscope}%
\pgfsetbuttcap%
\pgfsetroundjoin%
\definecolor{currentfill}{rgb}{0.000000,0.000000,0.000000}%
\pgfsetfillcolor{currentfill}%
\pgfsetlinewidth{0.501875pt}%
\definecolor{currentstroke}{rgb}{0.000000,0.000000,0.000000}%
\pgfsetstrokecolor{currentstroke}%
\pgfsetdash{}{0pt}%
\pgfsys@defobject{currentmarker}{\pgfqpoint{0.000000in}{-0.069444in}}{\pgfqpoint{0.000000in}{0.000000in}}{%
\pgfpathmoveto{\pgfqpoint{0.000000in}{0.000000in}}%
\pgfpathlineto{\pgfqpoint{0.000000in}{-0.069444in}}%
\pgfusepath{stroke,fill}%
}%
\begin{pgfscope}%
\pgfsys@transformshift{1.973507in}{2.496990in}%
\pgfsys@useobject{currentmarker}{}%
\end{pgfscope}%
\end{pgfscope}%
\begin{pgfscope}%
\pgftext[x=1.973507in,y=0.371974in,,top]{\rmfamily\fontsize{8.000000}{9.600000}\selectfont 6}%
\end{pgfscope}%
\begin{pgfscope}%
\pgfsetbuttcap%
\pgfsetroundjoin%
\definecolor{currentfill}{rgb}{0.000000,0.000000,0.000000}%
\pgfsetfillcolor{currentfill}%
\pgfsetlinewidth{0.501875pt}%
\definecolor{currentstroke}{rgb}{0.000000,0.000000,0.000000}%
\pgfsetstrokecolor{currentstroke}%
\pgfsetdash{}{0pt}%
\pgfsys@defobject{currentmarker}{\pgfqpoint{0.000000in}{0.000000in}}{\pgfqpoint{0.000000in}{0.069444in}}{%
\pgfpathmoveto{\pgfqpoint{0.000000in}{0.000000in}}%
\pgfpathlineto{\pgfqpoint{0.000000in}{0.069444in}}%
\pgfusepath{stroke,fill}%
}%
\begin{pgfscope}%
\pgfsys@transformshift{2.433800in}{0.441418in}%
\pgfsys@useobject{currentmarker}{}%
\end{pgfscope}%
\end{pgfscope}%
\begin{pgfscope}%
\pgfsetbuttcap%
\pgfsetroundjoin%
\definecolor{currentfill}{rgb}{0.000000,0.000000,0.000000}%
\pgfsetfillcolor{currentfill}%
\pgfsetlinewidth{0.501875pt}%
\definecolor{currentstroke}{rgb}{0.000000,0.000000,0.000000}%
\pgfsetstrokecolor{currentstroke}%
\pgfsetdash{}{0pt}%
\pgfsys@defobject{currentmarker}{\pgfqpoint{0.000000in}{-0.069444in}}{\pgfqpoint{0.000000in}{0.000000in}}{%
\pgfpathmoveto{\pgfqpoint{0.000000in}{0.000000in}}%
\pgfpathlineto{\pgfqpoint{0.000000in}{-0.069444in}}%
\pgfusepath{stroke,fill}%
}%
\begin{pgfscope}%
\pgfsys@transformshift{2.433800in}{2.496990in}%
\pgfsys@useobject{currentmarker}{}%
\end{pgfscope}%
\end{pgfscope}%
\begin{pgfscope}%
\pgftext[x=2.433800in,y=0.371974in,,top]{\rmfamily\fontsize{8.000000}{9.600000}\selectfont 8}%
\end{pgfscope}%
\begin{pgfscope}%
\pgfsetbuttcap%
\pgfsetroundjoin%
\definecolor{currentfill}{rgb}{0.000000,0.000000,0.000000}%
\pgfsetfillcolor{currentfill}%
\pgfsetlinewidth{0.501875pt}%
\definecolor{currentstroke}{rgb}{0.000000,0.000000,0.000000}%
\pgfsetstrokecolor{currentstroke}%
\pgfsetdash{}{0pt}%
\pgfsys@defobject{currentmarker}{\pgfqpoint{0.000000in}{0.000000in}}{\pgfqpoint{0.000000in}{0.069444in}}{%
\pgfpathmoveto{\pgfqpoint{0.000000in}{0.000000in}}%
\pgfpathlineto{\pgfqpoint{0.000000in}{0.069444in}}%
\pgfusepath{stroke,fill}%
}%
\begin{pgfscope}%
\pgfsys@transformshift{2.894092in}{0.441418in}%
\pgfsys@useobject{currentmarker}{}%
\end{pgfscope}%
\end{pgfscope}%
\begin{pgfscope}%
\pgfsetbuttcap%
\pgfsetroundjoin%
\definecolor{currentfill}{rgb}{0.000000,0.000000,0.000000}%
\pgfsetfillcolor{currentfill}%
\pgfsetlinewidth{0.501875pt}%
\definecolor{currentstroke}{rgb}{0.000000,0.000000,0.000000}%
\pgfsetstrokecolor{currentstroke}%
\pgfsetdash{}{0pt}%
\pgfsys@defobject{currentmarker}{\pgfqpoint{0.000000in}{-0.069444in}}{\pgfqpoint{0.000000in}{0.000000in}}{%
\pgfpathmoveto{\pgfqpoint{0.000000in}{0.000000in}}%
\pgfpathlineto{\pgfqpoint{0.000000in}{-0.069444in}}%
\pgfusepath{stroke,fill}%
}%
\begin{pgfscope}%
\pgfsys@transformshift{2.894092in}{2.496990in}%
\pgfsys@useobject{currentmarker}{}%
\end{pgfscope}%
\end{pgfscope}%
\begin{pgfscope}%
\pgftext[x=2.894092in,y=0.371974in,,top]{\rmfamily\fontsize{8.000000}{9.600000}\selectfont 10}%
\end{pgfscope}%
\begin{pgfscope}%
\pgfsetbuttcap%
\pgfsetroundjoin%
\definecolor{currentfill}{rgb}{0.000000,0.000000,0.000000}%
\pgfsetfillcolor{currentfill}%
\pgfsetlinewidth{0.501875pt}%
\definecolor{currentstroke}{rgb}{0.000000,0.000000,0.000000}%
\pgfsetstrokecolor{currentstroke}%
\pgfsetdash{}{0pt}%
\pgfsys@defobject{currentmarker}{\pgfqpoint{0.000000in}{0.000000in}}{\pgfqpoint{0.000000in}{0.069444in}}{%
\pgfpathmoveto{\pgfqpoint{0.000000in}{0.000000in}}%
\pgfpathlineto{\pgfqpoint{0.000000in}{0.069444in}}%
\pgfusepath{stroke,fill}%
}%
\begin{pgfscope}%
\pgfsys@transformshift{3.354385in}{0.441418in}%
\pgfsys@useobject{currentmarker}{}%
\end{pgfscope}%
\end{pgfscope}%
\begin{pgfscope}%
\pgfsetbuttcap%
\pgfsetroundjoin%
\definecolor{currentfill}{rgb}{0.000000,0.000000,0.000000}%
\pgfsetfillcolor{currentfill}%
\pgfsetlinewidth{0.501875pt}%
\definecolor{currentstroke}{rgb}{0.000000,0.000000,0.000000}%
\pgfsetstrokecolor{currentstroke}%
\pgfsetdash{}{0pt}%
\pgfsys@defobject{currentmarker}{\pgfqpoint{0.000000in}{-0.069444in}}{\pgfqpoint{0.000000in}{0.000000in}}{%
\pgfpathmoveto{\pgfqpoint{0.000000in}{0.000000in}}%
\pgfpathlineto{\pgfqpoint{0.000000in}{-0.069444in}}%
\pgfusepath{stroke,fill}%
}%
\begin{pgfscope}%
\pgfsys@transformshift{3.354385in}{2.496990in}%
\pgfsys@useobject{currentmarker}{}%
\end{pgfscope}%
\end{pgfscope}%
\begin{pgfscope}%
\pgftext[x=3.354385in,y=0.371974in,,top]{\rmfamily\fontsize{8.000000}{9.600000}\selectfont 12}%
\end{pgfscope}%
\begin{pgfscope}%
\pgfsetbuttcap%
\pgfsetroundjoin%
\definecolor{currentfill}{rgb}{0.000000,0.000000,0.000000}%
\pgfsetfillcolor{currentfill}%
\pgfsetlinewidth{0.501875pt}%
\definecolor{currentstroke}{rgb}{0.000000,0.000000,0.000000}%
\pgfsetstrokecolor{currentstroke}%
\pgfsetdash{}{0pt}%
\pgfsys@defobject{currentmarker}{\pgfqpoint{0.000000in}{0.000000in}}{\pgfqpoint{0.000000in}{0.069444in}}{%
\pgfpathmoveto{\pgfqpoint{0.000000in}{0.000000in}}%
\pgfpathlineto{\pgfqpoint{0.000000in}{0.069444in}}%
\pgfusepath{stroke,fill}%
}%
\begin{pgfscope}%
\pgfsys@transformshift{3.814677in}{0.441418in}%
\pgfsys@useobject{currentmarker}{}%
\end{pgfscope}%
\end{pgfscope}%
\begin{pgfscope}%
\pgfsetbuttcap%
\pgfsetroundjoin%
\definecolor{currentfill}{rgb}{0.000000,0.000000,0.000000}%
\pgfsetfillcolor{currentfill}%
\pgfsetlinewidth{0.501875pt}%
\definecolor{currentstroke}{rgb}{0.000000,0.000000,0.000000}%
\pgfsetstrokecolor{currentstroke}%
\pgfsetdash{}{0pt}%
\pgfsys@defobject{currentmarker}{\pgfqpoint{0.000000in}{-0.069444in}}{\pgfqpoint{0.000000in}{0.000000in}}{%
\pgfpathmoveto{\pgfqpoint{0.000000in}{0.000000in}}%
\pgfpathlineto{\pgfqpoint{0.000000in}{-0.069444in}}%
\pgfusepath{stroke,fill}%
}%
\begin{pgfscope}%
\pgfsys@transformshift{3.814677in}{2.496990in}%
\pgfsys@useobject{currentmarker}{}%
\end{pgfscope}%
\end{pgfscope}%
\begin{pgfscope}%
\pgftext[x=3.814677in,y=0.371974in,,top]{\rmfamily\fontsize{8.000000}{9.600000}\selectfont 14}%
\end{pgfscope}%
\begin{pgfscope}%
\pgftext[x=2.203654in,y=0.194999in,,top]{\rmfamily\fontsize{9.000000}{10.800000}\selectfont \(\displaystyle q^2(\mu^+\mu^-)\ /\ \mathrm{GeV}^2\)}%
\end{pgfscope}%
\begin{pgfscope}%
\pgfsetbuttcap%
\pgfsetroundjoin%
\definecolor{currentfill}{rgb}{0.000000,0.000000,0.000000}%
\pgfsetfillcolor{currentfill}%
\pgfsetlinewidth{0.501875pt}%
\definecolor{currentstroke}{rgb}{0.000000,0.000000,0.000000}%
\pgfsetstrokecolor{currentstroke}%
\pgfsetdash{}{0pt}%
\pgfsys@defobject{currentmarker}{\pgfqpoint{0.000000in}{0.000000in}}{\pgfqpoint{0.069444in}{0.000000in}}{%
\pgfpathmoveto{\pgfqpoint{0.000000in}{0.000000in}}%
\pgfpathlineto{\pgfqpoint{0.069444in}{0.000000in}}%
\pgfusepath{stroke,fill}%
}%
\begin{pgfscope}%
\pgfsys@transformshift{0.592630in}{0.441418in}%
\pgfsys@useobject{currentmarker}{}%
\end{pgfscope}%
\end{pgfscope}%
\begin{pgfscope}%
\pgfsetbuttcap%
\pgfsetroundjoin%
\definecolor{currentfill}{rgb}{0.000000,0.000000,0.000000}%
\pgfsetfillcolor{currentfill}%
\pgfsetlinewidth{0.501875pt}%
\definecolor{currentstroke}{rgb}{0.000000,0.000000,0.000000}%
\pgfsetstrokecolor{currentstroke}%
\pgfsetdash{}{0pt}%
\pgfsys@defobject{currentmarker}{\pgfqpoint{-0.069444in}{0.000000in}}{\pgfqpoint{0.000000in}{0.000000in}}{%
\pgfpathmoveto{\pgfqpoint{0.000000in}{0.000000in}}%
\pgfpathlineto{\pgfqpoint{-0.069444in}{0.000000in}}%
\pgfusepath{stroke,fill}%
}%
\begin{pgfscope}%
\pgfsys@transformshift{3.814677in}{0.441418in}%
\pgfsys@useobject{currentmarker}{}%
\end{pgfscope}%
\end{pgfscope}%
\begin{pgfscope}%
\pgftext[x=0.523185in,y=0.441418in,right,]{\rmfamily\fontsize{8.000000}{9.600000}\selectfont 0}%
\end{pgfscope}%
\begin{pgfscope}%
\pgfsetbuttcap%
\pgfsetroundjoin%
\definecolor{currentfill}{rgb}{0.000000,0.000000,0.000000}%
\pgfsetfillcolor{currentfill}%
\pgfsetlinewidth{0.501875pt}%
\definecolor{currentstroke}{rgb}{0.000000,0.000000,0.000000}%
\pgfsetstrokecolor{currentstroke}%
\pgfsetdash{}{0pt}%
\pgfsys@defobject{currentmarker}{\pgfqpoint{0.000000in}{0.000000in}}{\pgfqpoint{0.069444in}{0.000000in}}{%
\pgfpathmoveto{\pgfqpoint{0.000000in}{0.000000in}}%
\pgfpathlineto{\pgfqpoint{0.069444in}{0.000000in}}%
\pgfusepath{stroke,fill}%
}%
\begin{pgfscope}%
\pgfsys@transformshift{0.592630in}{0.852533in}%
\pgfsys@useobject{currentmarker}{}%
\end{pgfscope}%
\end{pgfscope}%
\begin{pgfscope}%
\pgfsetbuttcap%
\pgfsetroundjoin%
\definecolor{currentfill}{rgb}{0.000000,0.000000,0.000000}%
\pgfsetfillcolor{currentfill}%
\pgfsetlinewidth{0.501875pt}%
\definecolor{currentstroke}{rgb}{0.000000,0.000000,0.000000}%
\pgfsetstrokecolor{currentstroke}%
\pgfsetdash{}{0pt}%
\pgfsys@defobject{currentmarker}{\pgfqpoint{-0.069444in}{0.000000in}}{\pgfqpoint{0.000000in}{0.000000in}}{%
\pgfpathmoveto{\pgfqpoint{0.000000in}{0.000000in}}%
\pgfpathlineto{\pgfqpoint{-0.069444in}{0.000000in}}%
\pgfusepath{stroke,fill}%
}%
\begin{pgfscope}%
\pgfsys@transformshift{3.814677in}{0.852533in}%
\pgfsys@useobject{currentmarker}{}%
\end{pgfscope}%
\end{pgfscope}%
\begin{pgfscope}%
\pgftext[x=0.523185in,y=0.852533in,right,]{\rmfamily\fontsize{8.000000}{9.600000}\selectfont 200}%
\end{pgfscope}%
\begin{pgfscope}%
\pgfsetbuttcap%
\pgfsetroundjoin%
\definecolor{currentfill}{rgb}{0.000000,0.000000,0.000000}%
\pgfsetfillcolor{currentfill}%
\pgfsetlinewidth{0.501875pt}%
\definecolor{currentstroke}{rgb}{0.000000,0.000000,0.000000}%
\pgfsetstrokecolor{currentstroke}%
\pgfsetdash{}{0pt}%
\pgfsys@defobject{currentmarker}{\pgfqpoint{0.000000in}{0.000000in}}{\pgfqpoint{0.069444in}{0.000000in}}{%
\pgfpathmoveto{\pgfqpoint{0.000000in}{0.000000in}}%
\pgfpathlineto{\pgfqpoint{0.069444in}{0.000000in}}%
\pgfusepath{stroke,fill}%
}%
\begin{pgfscope}%
\pgfsys@transformshift{0.592630in}{1.263647in}%
\pgfsys@useobject{currentmarker}{}%
\end{pgfscope}%
\end{pgfscope}%
\begin{pgfscope}%
\pgfsetbuttcap%
\pgfsetroundjoin%
\definecolor{currentfill}{rgb}{0.000000,0.000000,0.000000}%
\pgfsetfillcolor{currentfill}%
\pgfsetlinewidth{0.501875pt}%
\definecolor{currentstroke}{rgb}{0.000000,0.000000,0.000000}%
\pgfsetstrokecolor{currentstroke}%
\pgfsetdash{}{0pt}%
\pgfsys@defobject{currentmarker}{\pgfqpoint{-0.069444in}{0.000000in}}{\pgfqpoint{0.000000in}{0.000000in}}{%
\pgfpathmoveto{\pgfqpoint{0.000000in}{0.000000in}}%
\pgfpathlineto{\pgfqpoint{-0.069444in}{0.000000in}}%
\pgfusepath{stroke,fill}%
}%
\begin{pgfscope}%
\pgfsys@transformshift{3.814677in}{1.263647in}%
\pgfsys@useobject{currentmarker}{}%
\end{pgfscope}%
\end{pgfscope}%
\begin{pgfscope}%
\pgftext[x=0.523185in,y=1.263647in,right,]{\rmfamily\fontsize{8.000000}{9.600000}\selectfont 400}%
\end{pgfscope}%
\begin{pgfscope}%
\pgfsetbuttcap%
\pgfsetroundjoin%
\definecolor{currentfill}{rgb}{0.000000,0.000000,0.000000}%
\pgfsetfillcolor{currentfill}%
\pgfsetlinewidth{0.501875pt}%
\definecolor{currentstroke}{rgb}{0.000000,0.000000,0.000000}%
\pgfsetstrokecolor{currentstroke}%
\pgfsetdash{}{0pt}%
\pgfsys@defobject{currentmarker}{\pgfqpoint{0.000000in}{0.000000in}}{\pgfqpoint{0.069444in}{0.000000in}}{%
\pgfpathmoveto{\pgfqpoint{0.000000in}{0.000000in}}%
\pgfpathlineto{\pgfqpoint{0.069444in}{0.000000in}}%
\pgfusepath{stroke,fill}%
}%
\begin{pgfscope}%
\pgfsys@transformshift{0.592630in}{1.674761in}%
\pgfsys@useobject{currentmarker}{}%
\end{pgfscope}%
\end{pgfscope}%
\begin{pgfscope}%
\pgfsetbuttcap%
\pgfsetroundjoin%
\definecolor{currentfill}{rgb}{0.000000,0.000000,0.000000}%
\pgfsetfillcolor{currentfill}%
\pgfsetlinewidth{0.501875pt}%
\definecolor{currentstroke}{rgb}{0.000000,0.000000,0.000000}%
\pgfsetstrokecolor{currentstroke}%
\pgfsetdash{}{0pt}%
\pgfsys@defobject{currentmarker}{\pgfqpoint{-0.069444in}{0.000000in}}{\pgfqpoint{0.000000in}{0.000000in}}{%
\pgfpathmoveto{\pgfqpoint{0.000000in}{0.000000in}}%
\pgfpathlineto{\pgfqpoint{-0.069444in}{0.000000in}}%
\pgfusepath{stroke,fill}%
}%
\begin{pgfscope}%
\pgfsys@transformshift{3.814677in}{1.674761in}%
\pgfsys@useobject{currentmarker}{}%
\end{pgfscope}%
\end{pgfscope}%
\begin{pgfscope}%
\pgftext[x=0.523185in,y=1.674761in,right,]{\rmfamily\fontsize{8.000000}{9.600000}\selectfont 600}%
\end{pgfscope}%
\begin{pgfscope}%
\pgfsetbuttcap%
\pgfsetroundjoin%
\definecolor{currentfill}{rgb}{0.000000,0.000000,0.000000}%
\pgfsetfillcolor{currentfill}%
\pgfsetlinewidth{0.501875pt}%
\definecolor{currentstroke}{rgb}{0.000000,0.000000,0.000000}%
\pgfsetstrokecolor{currentstroke}%
\pgfsetdash{}{0pt}%
\pgfsys@defobject{currentmarker}{\pgfqpoint{0.000000in}{0.000000in}}{\pgfqpoint{0.069444in}{0.000000in}}{%
\pgfpathmoveto{\pgfqpoint{0.000000in}{0.000000in}}%
\pgfpathlineto{\pgfqpoint{0.069444in}{0.000000in}}%
\pgfusepath{stroke,fill}%
}%
\begin{pgfscope}%
\pgfsys@transformshift{0.592630in}{2.085876in}%
\pgfsys@useobject{currentmarker}{}%
\end{pgfscope}%
\end{pgfscope}%
\begin{pgfscope}%
\pgfsetbuttcap%
\pgfsetroundjoin%
\definecolor{currentfill}{rgb}{0.000000,0.000000,0.000000}%
\pgfsetfillcolor{currentfill}%
\pgfsetlinewidth{0.501875pt}%
\definecolor{currentstroke}{rgb}{0.000000,0.000000,0.000000}%
\pgfsetstrokecolor{currentstroke}%
\pgfsetdash{}{0pt}%
\pgfsys@defobject{currentmarker}{\pgfqpoint{-0.069444in}{0.000000in}}{\pgfqpoint{0.000000in}{0.000000in}}{%
\pgfpathmoveto{\pgfqpoint{0.000000in}{0.000000in}}%
\pgfpathlineto{\pgfqpoint{-0.069444in}{0.000000in}}%
\pgfusepath{stroke,fill}%
}%
\begin{pgfscope}%
\pgfsys@transformshift{3.814677in}{2.085876in}%
\pgfsys@useobject{currentmarker}{}%
\end{pgfscope}%
\end{pgfscope}%
\begin{pgfscope}%
\pgftext[x=0.523185in,y=2.085876in,right,]{\rmfamily\fontsize{8.000000}{9.600000}\selectfont 800}%
\end{pgfscope}%
\begin{pgfscope}%
\pgfsetbuttcap%
\pgfsetroundjoin%
\definecolor{currentfill}{rgb}{0.000000,0.000000,0.000000}%
\pgfsetfillcolor{currentfill}%
\pgfsetlinewidth{0.501875pt}%
\definecolor{currentstroke}{rgb}{0.000000,0.000000,0.000000}%
\pgfsetstrokecolor{currentstroke}%
\pgfsetdash{}{0pt}%
\pgfsys@defobject{currentmarker}{\pgfqpoint{0.000000in}{0.000000in}}{\pgfqpoint{0.069444in}{0.000000in}}{%
\pgfpathmoveto{\pgfqpoint{0.000000in}{0.000000in}}%
\pgfpathlineto{\pgfqpoint{0.069444in}{0.000000in}}%
\pgfusepath{stroke,fill}%
}%
\begin{pgfscope}%
\pgfsys@transformshift{0.592630in}{2.496990in}%
\pgfsys@useobject{currentmarker}{}%
\end{pgfscope}%
\end{pgfscope}%
\begin{pgfscope}%
\pgfsetbuttcap%
\pgfsetroundjoin%
\definecolor{currentfill}{rgb}{0.000000,0.000000,0.000000}%
\pgfsetfillcolor{currentfill}%
\pgfsetlinewidth{0.501875pt}%
\definecolor{currentstroke}{rgb}{0.000000,0.000000,0.000000}%
\pgfsetstrokecolor{currentstroke}%
\pgfsetdash{}{0pt}%
\pgfsys@defobject{currentmarker}{\pgfqpoint{-0.069444in}{0.000000in}}{\pgfqpoint{0.000000in}{0.000000in}}{%
\pgfpathmoveto{\pgfqpoint{0.000000in}{0.000000in}}%
\pgfpathlineto{\pgfqpoint{-0.069444in}{0.000000in}}%
\pgfusepath{stroke,fill}%
}%
\begin{pgfscope}%
\pgfsys@transformshift{3.814677in}{2.496990in}%
\pgfsys@useobject{currentmarker}{}%
\end{pgfscope}%
\end{pgfscope}%
\begin{pgfscope}%
\pgftext[x=0.523185in,y=2.496990in,right,]{\rmfamily\fontsize{8.000000}{9.600000}\selectfont 1000}%
\end{pgfscope}%
\begin{pgfscope}%
\pgftext[x=0.170972in,y=1.469204in,,bottom,rotate=90.000000]{\rmfamily\fontsize{9.000000}{10.800000}\selectfont Simulated candidates}%
\end{pgfscope}%
\end{pgfpicture}%
\makeatother%
\endgroup%

  \caption{
    Histogram of $q^2$ of the two muons from simulated $\PBzero\to\APDzero\APmuon\Pmuon$ candidates after initial preselection.
    The left region marked in red corresponds to the selection cut to remove resonant $\PJpsi$ candidates in the data.
    The right region marked in red can be removed without loss of signal, as the phase space of $\PBzero\to\APDzero\APmuon\Pmuon$ ends roughly at $q^2 = \SI{12}{GeV^2}$.
  }
  \label{fig:mumu_mc}
\end{figure}

\subsubsection{Vetoes on \texorpdfstring{$\PBzero\to\PJpsi\PKstar$}{B0->JpsiK*} and \texorpdfstring{$\PBzero\to\PpsiTwoS\PKstar$}{B0->psi(2S)K*} with \texorpdfstring{$\pi/\mu$}{pion-muon} swap}

A possible background that peaks directly in the signal region originates from misreconstructed $\PBzero\to\PJpsi(\APmuon\Pmuon)\PKstar(\to\PKplus\Ppiminus)$ decays, where \Ppiminus and \Pmuon have been swapped.
Because this background appears in the blinded signal region, the contribution is studied in an indirect way.
A $\PBzero\to\PJpsi\PKstar$ data sample from a later stage of the analysis is used.
The $m(\PKplus\mu^-_\pi)$ invariant mass is reconstructed using these candidates under a $\Ppiminus$ mass hypothesis for the $\Pmuon$.
As can be seen from figure \ref{fig:doubleswap}, a significant number of candidates pass the $\PDzero$ reconstructed mass requirement.
In order to reject this background contribution, the invariant mass $m(\APmuon\pi^-_\mu)$ is reconstructed for the signal sample under a \Pmuon hypothesis for the \Ppiminus.
This corresponds to the \PJpsi invariant mass for swapped $\PBzero\to\PJpsi\PKstar$ candidates.
Two cuts are placed on this invariant mass: $\SI{2900}{MeV} < m(\APmuon\pi^-_\mu) < \SI{3200}{MeV}$ and $\SI{3500}{MeV} < m(\APmuon\pi^0_\mu) < \SI{3800}{MeV}$.
The first is designed to reject swapped $\PBzero\to\PJpsi\PKstar$ candidates, while the second correspondingly rejects swapped $\PBzero\to\PpsiTwoS\PKstar$.

\begin{figure}
  \centering
  %% Creator: Matplotlib, PGF backend
%%
%% To include the figure in your LaTeX document, write
%%   \input{<filename>.pgf}
%%
%% Make sure the required packages are loaded in your preamble
%%   \usepackage{pgf}
%%
%% Figures using additional raster images can only be included by \input if
%% they are in the same directory as the main LaTeX file. For loading figures
%% from other directories you can use the `import` package
%%   \usepackage{import}
%% and then include the figures with
%%   \import{<path to file>}{<filename>.pgf}
%%
%% Matplotlib used the following preamble
%%   \usepackage{fontspec}
%%   \setmainfont{DejaVu Serif}
%%   \setsansfont{DejaVu Sans}
%%   \setmonofont{DejaVu Sans Mono}
%%
\begingroup%
\makeatletter%
\begin{pgfpicture}%
\pgfpathrectangle{\pgfpointorigin}{\pgfqpoint{4.006062in}{2.600292in}}%
\pgfusepath{use as bounding box, clip}%
\begin{pgfscope}%
\pgfsetbuttcap%
\pgfsetmiterjoin%
\definecolor{currentfill}{rgb}{1.000000,1.000000,1.000000}%
\pgfsetfillcolor{currentfill}%
\pgfsetlinewidth{0.000000pt}%
\definecolor{currentstroke}{rgb}{1.000000,1.000000,1.000000}%
\pgfsetstrokecolor{currentstroke}%
\pgfsetdash{}{0pt}%
\pgfpathmoveto{\pgfqpoint{0.000000in}{0.000000in}}%
\pgfpathlineto{\pgfqpoint{4.006062in}{0.000000in}}%
\pgfpathlineto{\pgfqpoint{4.006062in}{2.600292in}}%
\pgfpathlineto{\pgfqpoint{0.000000in}{2.600292in}}%
\pgfpathclose%
\pgfusepath{fill}%
\end{pgfscope}%
\begin{pgfscope}%
\pgfsetbuttcap%
\pgfsetmiterjoin%
\definecolor{currentfill}{rgb}{1.000000,1.000000,1.000000}%
\pgfsetfillcolor{currentfill}%
\pgfsetlinewidth{0.000000pt}%
\definecolor{currentstroke}{rgb}{0.000000,0.000000,0.000000}%
\pgfsetstrokecolor{currentstroke}%
\pgfsetstrokeopacity{0.000000}%
\pgfsetdash{}{0pt}%
\pgfpathmoveto{\pgfqpoint{0.592630in}{0.440955in}}%
\pgfpathlineto{\pgfqpoint{3.814677in}{0.440955in}}%
\pgfpathlineto{\pgfqpoint{3.814677in}{2.496527in}}%
\pgfpathlineto{\pgfqpoint{0.592630in}{2.496527in}}%
\pgfpathclose%
\pgfusepath{fill}%
\end{pgfscope}%
\begin{pgfscope}%
\pgfpathrectangle{\pgfqpoint{0.592630in}{0.440955in}}{\pgfqpoint{3.222048in}{2.055572in}} %
\pgfusepath{clip}%
\pgfsetbuttcap%
\pgfsetmiterjoin%
\definecolor{currentfill}{rgb}{0.215686,0.470588,0.749020}%
\pgfsetfillcolor{currentfill}%
\pgfsetlinewidth{1.003750pt}%
\definecolor{currentstroke}{rgb}{0.000000,0.000000,0.000000}%
\pgfsetstrokecolor{currentstroke}%
\pgfsetdash{}{0pt}%
\pgfpathmoveto{\pgfqpoint{1.068985in}{0.440955in}}%
\pgfpathlineto{\pgfqpoint{1.068985in}{0.497459in}}%
\pgfpathlineto{\pgfqpoint{1.093742in}{0.497459in}}%
\pgfpathlineto{\pgfqpoint{1.093742in}{0.630334in}}%
\pgfpathlineto{\pgfqpoint{1.118499in}{0.630334in}}%
\pgfpathlineto{\pgfqpoint{1.118499in}{0.747417in}}%
\pgfpathlineto{\pgfqpoint{1.143255in}{0.747417in}}%
\pgfpathlineto{\pgfqpoint{1.143255in}{0.833319in}}%
\pgfpathlineto{\pgfqpoint{1.168012in}{0.833319in}}%
\pgfpathlineto{\pgfqpoint{1.168012in}{0.934960in}}%
\pgfpathlineto{\pgfqpoint{1.192769in}{0.934960in}}%
\pgfpathlineto{\pgfqpoint{1.192769in}{0.974249in}}%
\pgfpathlineto{\pgfqpoint{1.217526in}{0.974249in}}%
\pgfpathlineto{\pgfqpoint{1.217526in}{0.993673in}}%
\pgfpathlineto{\pgfqpoint{1.242282in}{0.993673in}}%
\pgfpathlineto{\pgfqpoint{1.242282in}{1.062503in}}%
\pgfpathlineto{\pgfqpoint{1.267039in}{1.062503in}}%
\pgfpathlineto{\pgfqpoint{1.267039in}{1.098098in}}%
\pgfpathlineto{\pgfqpoint{1.291796in}{1.098098in}}%
\pgfpathlineto{\pgfqpoint{1.291796in}{1.122645in}}%
\pgfpathlineto{\pgfqpoint{1.316553in}{1.122645in}}%
\pgfpathlineto{\pgfqpoint{1.316553in}{1.140584in}}%
\pgfpathlineto{\pgfqpoint{1.341310in}{1.140584in}}%
\pgfpathlineto{\pgfqpoint{1.341310in}{1.230175in}}%
\pgfpathlineto{\pgfqpoint{1.366066in}{1.230175in}}%
\pgfpathlineto{\pgfqpoint{1.366066in}{1.269861in}}%
\pgfpathlineto{\pgfqpoint{1.390823in}{1.269861in}}%
\pgfpathlineto{\pgfqpoint{1.390823in}{1.307672in}}%
\pgfpathlineto{\pgfqpoint{1.415580in}{1.307672in}}%
\pgfpathlineto{\pgfqpoint{1.415580in}{1.328968in}}%
\pgfpathlineto{\pgfqpoint{1.440337in}{1.328968in}}%
\pgfpathlineto{\pgfqpoint{1.440337in}{1.379750in}}%
\pgfpathlineto{\pgfqpoint{1.465094in}{1.379750in}}%
\pgfpathlineto{\pgfqpoint{1.465094in}{1.436016in}}%
\pgfpathlineto{\pgfqpoint{1.489850in}{1.436016in}}%
\pgfpathlineto{\pgfqpoint{1.489850in}{1.471199in}}%
\pgfpathlineto{\pgfqpoint{1.514607in}{1.471199in}}%
\pgfpathlineto{\pgfqpoint{1.514607in}{1.518027in}}%
\pgfpathlineto{\pgfqpoint{1.539364in}{1.518027in}}%
\pgfpathlineto{\pgfqpoint{1.539364in}{1.561865in}}%
\pgfpathlineto{\pgfqpoint{1.564121in}{1.561865in}}%
\pgfpathlineto{\pgfqpoint{1.564121in}{1.590225in}}%
\pgfpathlineto{\pgfqpoint{1.588877in}{1.590225in}}%
\pgfpathlineto{\pgfqpoint{1.588877in}{1.596140in}}%
\pgfpathlineto{\pgfqpoint{1.613634in}{1.596140in}}%
\pgfpathlineto{\pgfqpoint{1.613634in}{1.674307in}}%
\pgfpathlineto{\pgfqpoint{1.638391in}{1.674307in}}%
\pgfpathlineto{\pgfqpoint{1.638391in}{1.737135in}}%
\pgfpathlineto{\pgfqpoint{1.663148in}{1.737135in}}%
\pgfpathlineto{\pgfqpoint{1.663148in}{1.744678in}}%
\pgfpathlineto{\pgfqpoint{1.687905in}{1.744678in}}%
\pgfpathlineto{\pgfqpoint{1.687905in}{1.839095in}}%
\pgfpathlineto{\pgfqpoint{1.712661in}{1.839095in}}%
\pgfpathlineto{\pgfqpoint{1.712661in}{1.851406in}}%
\pgfpathlineto{\pgfqpoint{1.737418in}{1.851406in}}%
\pgfpathlineto{\pgfqpoint{1.737418in}{1.911115in}}%
\pgfpathlineto{\pgfqpoint{1.762175in}{1.911115in}}%
\pgfpathlineto{\pgfqpoint{1.762175in}{1.949161in}}%
\pgfpathlineto{\pgfqpoint{1.786932in}{1.949161in}}%
\pgfpathlineto{\pgfqpoint{1.786932in}{2.004303in}}%
\pgfpathlineto{\pgfqpoint{1.811689in}{2.004303in}}%
\pgfpathlineto{\pgfqpoint{1.811689in}{2.046809in}}%
\pgfpathlineto{\pgfqpoint{1.836445in}{2.046809in}}%
\pgfpathlineto{\pgfqpoint{1.836445in}{2.139491in}}%
\pgfpathlineto{\pgfqpoint{1.861202in}{2.139491in}}%
\pgfpathlineto{\pgfqpoint{1.861202in}{2.201119in}}%
\pgfpathlineto{\pgfqpoint{1.885959in}{2.201119in}}%
\pgfpathlineto{\pgfqpoint{1.885959in}{2.151965in}}%
\pgfpathlineto{\pgfqpoint{1.910716in}{2.151965in}}%
\pgfpathlineto{\pgfqpoint{1.910716in}{2.204289in}}%
\pgfpathlineto{\pgfqpoint{1.935473in}{2.204289in}}%
\pgfpathlineto{\pgfqpoint{1.935473in}{2.255244in}}%
\pgfpathlineto{\pgfqpoint{1.960229in}{2.255244in}}%
\pgfpathlineto{\pgfqpoint{1.960229in}{2.257921in}}%
\pgfpathlineto{\pgfqpoint{1.984986in}{2.257921in}}%
\pgfpathlineto{\pgfqpoint{1.984986in}{2.353596in}}%
\pgfpathlineto{\pgfqpoint{2.009743in}{2.353596in}}%
\pgfpathlineto{\pgfqpoint{2.009743in}{2.386178in}}%
\pgfpathlineto{\pgfqpoint{2.034500in}{2.386178in}}%
\pgfpathlineto{\pgfqpoint{2.034500in}{2.357427in}}%
\pgfpathlineto{\pgfqpoint{2.059256in}{2.357427in}}%
\pgfpathlineto{\pgfqpoint{2.059256in}{2.375478in}}%
\pgfpathlineto{\pgfqpoint{2.084013in}{2.375478in}}%
\pgfpathlineto{\pgfqpoint{2.084013in}{2.360712in}}%
\pgfpathlineto{\pgfqpoint{2.108770in}{2.360712in}}%
\pgfpathlineto{\pgfqpoint{2.108770in}{2.320105in}}%
\pgfpathlineto{\pgfqpoint{2.133527in}{2.320105in}}%
\pgfpathlineto{\pgfqpoint{2.133527in}{2.390470in}}%
\pgfpathlineto{\pgfqpoint{2.158284in}{2.390470in}}%
\pgfpathlineto{\pgfqpoint{2.158284in}{2.373047in}}%
\pgfpathlineto{\pgfqpoint{2.183040in}{2.373047in}}%
\pgfpathlineto{\pgfqpoint{2.183040in}{2.364399in}}%
\pgfpathlineto{\pgfqpoint{2.207797in}{2.364399in}}%
\pgfpathlineto{\pgfqpoint{2.207797in}{2.342008in}}%
\pgfpathlineto{\pgfqpoint{2.232554in}{2.342008in}}%
\pgfpathlineto{\pgfqpoint{2.232554in}{2.290654in}}%
\pgfpathlineto{\pgfqpoint{2.257311in}{2.290654in}}%
\pgfpathlineto{\pgfqpoint{2.257311in}{2.235472in}}%
\pgfpathlineto{\pgfqpoint{2.282068in}{2.235472in}}%
\pgfpathlineto{\pgfqpoint{2.282068in}{2.227352in}}%
\pgfpathlineto{\pgfqpoint{2.306824in}{2.227352in}}%
\pgfpathlineto{\pgfqpoint{2.306824in}{2.208329in}}%
\pgfpathlineto{\pgfqpoint{2.331581in}{2.208329in}}%
\pgfpathlineto{\pgfqpoint{2.331581in}{2.159269in}}%
\pgfpathlineto{\pgfqpoint{2.356338in}{2.159269in}}%
\pgfpathlineto{\pgfqpoint{2.356338in}{2.101894in}}%
\pgfpathlineto{\pgfqpoint{2.381095in}{2.101894in}}%
\pgfpathlineto{\pgfqpoint{2.381095in}{2.054032in}}%
\pgfpathlineto{\pgfqpoint{2.405851in}{2.054032in}}%
\pgfpathlineto{\pgfqpoint{2.405851in}{1.983819in}}%
\pgfpathlineto{\pgfqpoint{2.430608in}{1.983819in}}%
\pgfpathlineto{\pgfqpoint{2.430608in}{2.009151in}}%
\pgfpathlineto{\pgfqpoint{2.455365in}{2.009151in}}%
\pgfpathlineto{\pgfqpoint{2.455365in}{1.949991in}}%
\pgfpathlineto{\pgfqpoint{2.480122in}{1.949991in}}%
\pgfpathlineto{\pgfqpoint{2.480122in}{1.913950in}}%
\pgfpathlineto{\pgfqpoint{2.504879in}{1.913950in}}%
\pgfpathlineto{\pgfqpoint{2.504879in}{1.927711in}}%
\pgfpathlineto{\pgfqpoint{2.529635in}{1.927711in}}%
\pgfpathlineto{\pgfqpoint{2.529635in}{1.917135in}}%
\pgfpathlineto{\pgfqpoint{2.554392in}{1.917135in}}%
\pgfpathlineto{\pgfqpoint{2.554392in}{1.858469in}}%
\pgfpathlineto{\pgfqpoint{2.579149in}{1.858469in}}%
\pgfpathlineto{\pgfqpoint{2.579149in}{1.882594in}}%
\pgfpathlineto{\pgfqpoint{2.603906in}{1.882594in}}%
\pgfpathlineto{\pgfqpoint{2.603906in}{1.841428in}}%
\pgfpathlineto{\pgfqpoint{2.628663in}{1.841428in}}%
\pgfpathlineto{\pgfqpoint{2.628663in}{1.882394in}}%
\pgfpathlineto{\pgfqpoint{2.653419in}{1.882394in}}%
\pgfpathlineto{\pgfqpoint{2.653419in}{1.824040in}}%
\pgfpathlineto{\pgfqpoint{2.678176in}{1.824040in}}%
\pgfpathlineto{\pgfqpoint{2.678176in}{1.816766in}}%
\pgfpathlineto{\pgfqpoint{2.702933in}{1.816766in}}%
\pgfpathlineto{\pgfqpoint{2.702933in}{1.846879in}}%
\pgfpathlineto{\pgfqpoint{2.727690in}{1.846879in}}%
\pgfpathlineto{\pgfqpoint{2.727690in}{1.790040in}}%
\pgfpathlineto{\pgfqpoint{2.752446in}{1.790040in}}%
\pgfpathlineto{\pgfqpoint{2.752446in}{1.773711in}}%
\pgfpathlineto{\pgfqpoint{2.777203in}{1.773711in}}%
\pgfpathlineto{\pgfqpoint{2.777203in}{1.700602in}}%
\pgfpathlineto{\pgfqpoint{2.801960in}{1.700602in}}%
\pgfpathlineto{\pgfqpoint{2.801960in}{1.729827in}}%
\pgfpathlineto{\pgfqpoint{2.826717in}{1.729827in}}%
\pgfpathlineto{\pgfqpoint{2.826717in}{1.706527in}}%
\pgfpathlineto{\pgfqpoint{2.851474in}{1.706527in}}%
\pgfpathlineto{\pgfqpoint{2.851474in}{1.671945in}}%
\pgfpathlineto{\pgfqpoint{2.876230in}{1.671945in}}%
\pgfpathlineto{\pgfqpoint{2.876230in}{1.609350in}}%
\pgfpathlineto{\pgfqpoint{2.900987in}{1.609350in}}%
\pgfpathlineto{\pgfqpoint{2.900987in}{1.663696in}}%
\pgfpathlineto{\pgfqpoint{2.925744in}{1.663696in}}%
\pgfpathlineto{\pgfqpoint{2.925744in}{1.589516in}}%
\pgfpathlineto{\pgfqpoint{2.950501in}{1.589516in}}%
\pgfpathlineto{\pgfqpoint{2.950501in}{1.550027in}}%
\pgfpathlineto{\pgfqpoint{2.975258in}{1.550027in}}%
\pgfpathlineto{\pgfqpoint{2.975258in}{1.517077in}}%
\pgfpathlineto{\pgfqpoint{3.000014in}{1.517077in}}%
\pgfpathlineto{\pgfqpoint{3.000014in}{1.454976in}}%
\pgfpathlineto{\pgfqpoint{3.024771in}{1.454976in}}%
\pgfpathlineto{\pgfqpoint{3.024771in}{1.449450in}}%
\pgfpathlineto{\pgfqpoint{3.049528in}{1.449450in}}%
\pgfpathlineto{\pgfqpoint{3.049528in}{1.363750in}}%
\pgfpathlineto{\pgfqpoint{3.074285in}{1.363750in}}%
\pgfpathlineto{\pgfqpoint{3.074285in}{1.307157in}}%
\pgfpathlineto{\pgfqpoint{3.099041in}{1.307157in}}%
\pgfpathlineto{\pgfqpoint{3.099041in}{1.266983in}}%
\pgfpathlineto{\pgfqpoint{3.123798in}{1.266983in}}%
\pgfpathlineto{\pgfqpoint{3.123798in}{1.180689in}}%
\pgfpathlineto{\pgfqpoint{3.148555in}{1.180689in}}%
\pgfpathlineto{\pgfqpoint{3.148555in}{1.102401in}}%
\pgfpathlineto{\pgfqpoint{3.173312in}{1.102401in}}%
\pgfpathlineto{\pgfqpoint{3.173312in}{1.070629in}}%
\pgfpathlineto{\pgfqpoint{3.198069in}{1.070629in}}%
\pgfpathlineto{\pgfqpoint{3.198069in}{0.952158in}}%
\pgfpathlineto{\pgfqpoint{3.222825in}{0.952158in}}%
\pgfpathlineto{\pgfqpoint{3.222825in}{0.878414in}}%
\pgfpathlineto{\pgfqpoint{3.247582in}{0.878414in}}%
\pgfpathlineto{\pgfqpoint{3.247582in}{0.835093in}}%
\pgfpathlineto{\pgfqpoint{3.272339in}{0.835093in}}%
\pgfpathlineto{\pgfqpoint{3.272339in}{0.733941in}}%
\pgfpathlineto{\pgfqpoint{3.297096in}{0.733941in}}%
\pgfpathlineto{\pgfqpoint{3.297096in}{0.694972in}}%
\pgfpathlineto{\pgfqpoint{3.321853in}{0.694972in}}%
\pgfpathlineto{\pgfqpoint{3.321853in}{0.622562in}}%
\pgfpathlineto{\pgfqpoint{3.346609in}{0.622562in}}%
\pgfpathlineto{\pgfqpoint{3.346609in}{0.576436in}}%
\pgfpathlineto{\pgfqpoint{3.371366in}{0.576436in}}%
\pgfpathlineto{\pgfqpoint{3.371366in}{0.547818in}}%
\pgfpathlineto{\pgfqpoint{3.396123in}{0.547818in}}%
\pgfpathlineto{\pgfqpoint{3.396123in}{0.499874in}}%
\pgfpathlineto{\pgfqpoint{3.420880in}{0.499874in}}%
\pgfpathlineto{\pgfqpoint{3.420880in}{0.475274in}}%
\pgfpathlineto{\pgfqpoint{3.445636in}{0.475274in}}%
\pgfpathlineto{\pgfqpoint{3.445636in}{0.459474in}}%
\pgfpathlineto{\pgfqpoint{3.470393in}{0.459474in}}%
\pgfpathlineto{\pgfqpoint{3.470393in}{0.444861in}}%
\pgfpathlineto{\pgfqpoint{3.495150in}{0.444861in}}%
\pgfpathlineto{\pgfqpoint{3.495150in}{0.441238in}}%
\pgfpathlineto{\pgfqpoint{3.519907in}{0.441238in}}%
\pgfpathlineto{\pgfqpoint{3.519907in}{0.440730in}}%
\pgfpathlineto{\pgfqpoint{3.544664in}{0.440730in}}%
\pgfpathlineto{\pgfqpoint{3.544664in}{0.440955in}}%
\pgfpathlineto{\pgfqpoint{3.519907in}{0.440955in}}%
\pgfpathlineto{\pgfqpoint{3.519907in}{0.440955in}}%
\pgfpathlineto{\pgfqpoint{3.495150in}{0.440955in}}%
\pgfpathlineto{\pgfqpoint{3.495150in}{0.440955in}}%
\pgfpathlineto{\pgfqpoint{3.470393in}{0.440955in}}%
\pgfpathlineto{\pgfqpoint{3.470393in}{0.440955in}}%
\pgfpathlineto{\pgfqpoint{3.445636in}{0.440955in}}%
\pgfpathlineto{\pgfqpoint{3.445636in}{0.440955in}}%
\pgfpathlineto{\pgfqpoint{3.420880in}{0.440955in}}%
\pgfpathlineto{\pgfqpoint{3.420880in}{0.440955in}}%
\pgfpathlineto{\pgfqpoint{3.396123in}{0.440955in}}%
\pgfpathlineto{\pgfqpoint{3.396123in}{0.440955in}}%
\pgfpathlineto{\pgfqpoint{3.371366in}{0.440955in}}%
\pgfpathlineto{\pgfqpoint{3.371366in}{0.440955in}}%
\pgfpathlineto{\pgfqpoint{3.346609in}{0.440955in}}%
\pgfpathlineto{\pgfqpoint{3.346609in}{0.440955in}}%
\pgfpathlineto{\pgfqpoint{3.321853in}{0.440955in}}%
\pgfpathlineto{\pgfqpoint{3.321853in}{0.440955in}}%
\pgfpathlineto{\pgfqpoint{3.297096in}{0.440955in}}%
\pgfpathlineto{\pgfqpoint{3.297096in}{0.440955in}}%
\pgfpathlineto{\pgfqpoint{3.272339in}{0.440955in}}%
\pgfpathlineto{\pgfqpoint{3.272339in}{0.440955in}}%
\pgfpathlineto{\pgfqpoint{3.247582in}{0.440955in}}%
\pgfpathlineto{\pgfqpoint{3.247582in}{0.440955in}}%
\pgfpathlineto{\pgfqpoint{3.222825in}{0.440955in}}%
\pgfpathlineto{\pgfqpoint{3.222825in}{0.440955in}}%
\pgfpathlineto{\pgfqpoint{3.198069in}{0.440955in}}%
\pgfpathlineto{\pgfqpoint{3.198069in}{0.440955in}}%
\pgfpathlineto{\pgfqpoint{3.173312in}{0.440955in}}%
\pgfpathlineto{\pgfqpoint{3.173312in}{0.440955in}}%
\pgfpathlineto{\pgfqpoint{3.148555in}{0.440955in}}%
\pgfpathlineto{\pgfqpoint{3.148555in}{0.440955in}}%
\pgfpathlineto{\pgfqpoint{3.123798in}{0.440955in}}%
\pgfpathlineto{\pgfqpoint{3.123798in}{0.440955in}}%
\pgfpathlineto{\pgfqpoint{3.099041in}{0.440955in}}%
\pgfpathlineto{\pgfqpoint{3.099041in}{0.440955in}}%
\pgfpathlineto{\pgfqpoint{3.074285in}{0.440955in}}%
\pgfpathlineto{\pgfqpoint{3.074285in}{0.440955in}}%
\pgfpathlineto{\pgfqpoint{3.049528in}{0.440955in}}%
\pgfpathlineto{\pgfqpoint{3.049528in}{0.440955in}}%
\pgfpathlineto{\pgfqpoint{3.024771in}{0.440955in}}%
\pgfpathlineto{\pgfqpoint{3.024771in}{0.440955in}}%
\pgfpathlineto{\pgfqpoint{3.000014in}{0.440955in}}%
\pgfpathlineto{\pgfqpoint{3.000014in}{0.440955in}}%
\pgfpathlineto{\pgfqpoint{2.975258in}{0.440955in}}%
\pgfpathlineto{\pgfqpoint{2.975258in}{0.440955in}}%
\pgfpathlineto{\pgfqpoint{2.950501in}{0.440955in}}%
\pgfpathlineto{\pgfqpoint{2.950501in}{0.440955in}}%
\pgfpathlineto{\pgfqpoint{2.925744in}{0.440955in}}%
\pgfpathlineto{\pgfqpoint{2.925744in}{0.440955in}}%
\pgfpathlineto{\pgfqpoint{2.900987in}{0.440955in}}%
\pgfpathlineto{\pgfqpoint{2.900987in}{0.440955in}}%
\pgfpathlineto{\pgfqpoint{2.876230in}{0.440955in}}%
\pgfpathlineto{\pgfqpoint{2.876230in}{0.440955in}}%
\pgfpathlineto{\pgfqpoint{2.851474in}{0.440955in}}%
\pgfpathlineto{\pgfqpoint{2.851474in}{0.440955in}}%
\pgfpathlineto{\pgfqpoint{2.826717in}{0.440955in}}%
\pgfpathlineto{\pgfqpoint{2.826717in}{0.440955in}}%
\pgfpathlineto{\pgfqpoint{2.801960in}{0.440955in}}%
\pgfpathlineto{\pgfqpoint{2.801960in}{0.440955in}}%
\pgfpathlineto{\pgfqpoint{2.777203in}{0.440955in}}%
\pgfpathlineto{\pgfqpoint{2.777203in}{0.440955in}}%
\pgfpathlineto{\pgfqpoint{2.752446in}{0.440955in}}%
\pgfpathlineto{\pgfqpoint{2.752446in}{0.440955in}}%
\pgfpathlineto{\pgfqpoint{2.727690in}{0.440955in}}%
\pgfpathlineto{\pgfqpoint{2.727690in}{0.440955in}}%
\pgfpathlineto{\pgfqpoint{2.702933in}{0.440955in}}%
\pgfpathlineto{\pgfqpoint{2.702933in}{0.440955in}}%
\pgfpathlineto{\pgfqpoint{2.678176in}{0.440955in}}%
\pgfpathlineto{\pgfqpoint{2.678176in}{0.440955in}}%
\pgfpathlineto{\pgfqpoint{2.653419in}{0.440955in}}%
\pgfpathlineto{\pgfqpoint{2.653419in}{0.440955in}}%
\pgfpathlineto{\pgfqpoint{2.628663in}{0.440955in}}%
\pgfpathlineto{\pgfqpoint{2.628663in}{0.440955in}}%
\pgfpathlineto{\pgfqpoint{2.603906in}{0.440955in}}%
\pgfpathlineto{\pgfqpoint{2.603906in}{0.440955in}}%
\pgfpathlineto{\pgfqpoint{2.579149in}{0.440955in}}%
\pgfpathlineto{\pgfqpoint{2.579149in}{0.440955in}}%
\pgfpathlineto{\pgfqpoint{2.554392in}{0.440955in}}%
\pgfpathlineto{\pgfqpoint{2.554392in}{0.440955in}}%
\pgfpathlineto{\pgfqpoint{2.529635in}{0.440955in}}%
\pgfpathlineto{\pgfqpoint{2.529635in}{0.440955in}}%
\pgfpathlineto{\pgfqpoint{2.504879in}{0.440955in}}%
\pgfpathlineto{\pgfqpoint{2.504879in}{0.440955in}}%
\pgfpathlineto{\pgfqpoint{2.480122in}{0.440955in}}%
\pgfpathlineto{\pgfqpoint{2.480122in}{0.440955in}}%
\pgfpathlineto{\pgfqpoint{2.455365in}{0.440955in}}%
\pgfpathlineto{\pgfqpoint{2.455365in}{0.440955in}}%
\pgfpathlineto{\pgfqpoint{2.430608in}{0.440955in}}%
\pgfpathlineto{\pgfqpoint{2.430608in}{0.440955in}}%
\pgfpathlineto{\pgfqpoint{2.405851in}{0.440955in}}%
\pgfpathlineto{\pgfqpoint{2.405851in}{0.440955in}}%
\pgfpathlineto{\pgfqpoint{2.381095in}{0.440955in}}%
\pgfpathlineto{\pgfqpoint{2.381095in}{0.440955in}}%
\pgfpathlineto{\pgfqpoint{2.356338in}{0.440955in}}%
\pgfpathlineto{\pgfqpoint{2.356338in}{0.440955in}}%
\pgfpathlineto{\pgfqpoint{2.331581in}{0.440955in}}%
\pgfpathlineto{\pgfqpoint{2.331581in}{0.440955in}}%
\pgfpathlineto{\pgfqpoint{2.306824in}{0.440955in}}%
\pgfpathlineto{\pgfqpoint{2.306824in}{0.440955in}}%
\pgfpathlineto{\pgfqpoint{2.282068in}{0.440955in}}%
\pgfpathlineto{\pgfqpoint{2.282068in}{0.440955in}}%
\pgfpathlineto{\pgfqpoint{2.257311in}{0.440955in}}%
\pgfpathlineto{\pgfqpoint{2.257311in}{0.440955in}}%
\pgfpathlineto{\pgfqpoint{2.232554in}{0.440955in}}%
\pgfpathlineto{\pgfqpoint{2.232554in}{0.440955in}}%
\pgfpathlineto{\pgfqpoint{2.207797in}{0.440955in}}%
\pgfpathlineto{\pgfqpoint{2.207797in}{0.440955in}}%
\pgfpathlineto{\pgfqpoint{2.183040in}{0.440955in}}%
\pgfpathlineto{\pgfqpoint{2.183040in}{0.440955in}}%
\pgfpathlineto{\pgfqpoint{2.158284in}{0.440955in}}%
\pgfpathlineto{\pgfqpoint{2.158284in}{0.440955in}}%
\pgfpathlineto{\pgfqpoint{2.133527in}{0.440955in}}%
\pgfpathlineto{\pgfqpoint{2.133527in}{0.440955in}}%
\pgfpathlineto{\pgfqpoint{2.108770in}{0.440955in}}%
\pgfpathlineto{\pgfqpoint{2.108770in}{0.440955in}}%
\pgfpathlineto{\pgfqpoint{2.084013in}{0.440955in}}%
\pgfpathlineto{\pgfqpoint{2.084013in}{0.440955in}}%
\pgfpathlineto{\pgfqpoint{2.059256in}{0.440955in}}%
\pgfpathlineto{\pgfqpoint{2.059256in}{0.440955in}}%
\pgfpathlineto{\pgfqpoint{2.034500in}{0.440955in}}%
\pgfpathlineto{\pgfqpoint{2.034500in}{0.440955in}}%
\pgfpathlineto{\pgfqpoint{2.009743in}{0.440955in}}%
\pgfpathlineto{\pgfqpoint{2.009743in}{0.440955in}}%
\pgfpathlineto{\pgfqpoint{1.984986in}{0.440955in}}%
\pgfpathlineto{\pgfqpoint{1.984986in}{0.440955in}}%
\pgfpathlineto{\pgfqpoint{1.960229in}{0.440955in}}%
\pgfpathlineto{\pgfqpoint{1.960229in}{0.440955in}}%
\pgfpathlineto{\pgfqpoint{1.935473in}{0.440955in}}%
\pgfpathlineto{\pgfqpoint{1.935473in}{0.440955in}}%
\pgfpathlineto{\pgfqpoint{1.910716in}{0.440955in}}%
\pgfpathlineto{\pgfqpoint{1.910716in}{0.440955in}}%
\pgfpathlineto{\pgfqpoint{1.885959in}{0.440955in}}%
\pgfpathlineto{\pgfqpoint{1.885959in}{0.440955in}}%
\pgfpathlineto{\pgfqpoint{1.861202in}{0.440955in}}%
\pgfpathlineto{\pgfqpoint{1.861202in}{0.440955in}}%
\pgfpathlineto{\pgfqpoint{1.836445in}{0.440955in}}%
\pgfpathlineto{\pgfqpoint{1.836445in}{0.440955in}}%
\pgfpathlineto{\pgfqpoint{1.811689in}{0.440955in}}%
\pgfpathlineto{\pgfqpoint{1.811689in}{0.440955in}}%
\pgfpathlineto{\pgfqpoint{1.786932in}{0.440955in}}%
\pgfpathlineto{\pgfqpoint{1.786932in}{0.440955in}}%
\pgfpathlineto{\pgfqpoint{1.762175in}{0.440955in}}%
\pgfpathlineto{\pgfqpoint{1.762175in}{0.440955in}}%
\pgfpathlineto{\pgfqpoint{1.737418in}{0.440955in}}%
\pgfpathlineto{\pgfqpoint{1.737418in}{0.440955in}}%
\pgfpathlineto{\pgfqpoint{1.712661in}{0.440955in}}%
\pgfpathlineto{\pgfqpoint{1.712661in}{0.440955in}}%
\pgfpathlineto{\pgfqpoint{1.687905in}{0.440955in}}%
\pgfpathlineto{\pgfqpoint{1.687905in}{0.440955in}}%
\pgfpathlineto{\pgfqpoint{1.663148in}{0.440955in}}%
\pgfpathlineto{\pgfqpoint{1.663148in}{0.440955in}}%
\pgfpathlineto{\pgfqpoint{1.638391in}{0.440955in}}%
\pgfpathlineto{\pgfqpoint{1.638391in}{0.440955in}}%
\pgfpathlineto{\pgfqpoint{1.613634in}{0.440955in}}%
\pgfpathlineto{\pgfqpoint{1.613634in}{0.440955in}}%
\pgfpathlineto{\pgfqpoint{1.588877in}{0.440955in}}%
\pgfpathlineto{\pgfqpoint{1.588877in}{0.440955in}}%
\pgfpathlineto{\pgfqpoint{1.564121in}{0.440955in}}%
\pgfpathlineto{\pgfqpoint{1.564121in}{0.440955in}}%
\pgfpathlineto{\pgfqpoint{1.539364in}{0.440955in}}%
\pgfpathlineto{\pgfqpoint{1.539364in}{0.440955in}}%
\pgfpathlineto{\pgfqpoint{1.514607in}{0.440955in}}%
\pgfpathlineto{\pgfqpoint{1.514607in}{0.440955in}}%
\pgfpathlineto{\pgfqpoint{1.489850in}{0.440955in}}%
\pgfpathlineto{\pgfqpoint{1.489850in}{0.440955in}}%
\pgfpathlineto{\pgfqpoint{1.465094in}{0.440955in}}%
\pgfpathlineto{\pgfqpoint{1.465094in}{0.440955in}}%
\pgfpathlineto{\pgfqpoint{1.440337in}{0.440955in}}%
\pgfpathlineto{\pgfqpoint{1.440337in}{0.440955in}}%
\pgfpathlineto{\pgfqpoint{1.415580in}{0.440955in}}%
\pgfpathlineto{\pgfqpoint{1.415580in}{0.440955in}}%
\pgfpathlineto{\pgfqpoint{1.390823in}{0.440955in}}%
\pgfpathlineto{\pgfqpoint{1.390823in}{0.440955in}}%
\pgfpathlineto{\pgfqpoint{1.366066in}{0.440955in}}%
\pgfpathlineto{\pgfqpoint{1.366066in}{0.440955in}}%
\pgfpathlineto{\pgfqpoint{1.341310in}{0.440955in}}%
\pgfpathlineto{\pgfqpoint{1.341310in}{0.440955in}}%
\pgfpathlineto{\pgfqpoint{1.316553in}{0.440955in}}%
\pgfpathlineto{\pgfqpoint{1.316553in}{0.440955in}}%
\pgfpathlineto{\pgfqpoint{1.291796in}{0.440955in}}%
\pgfpathlineto{\pgfqpoint{1.291796in}{0.440955in}}%
\pgfpathlineto{\pgfqpoint{1.267039in}{0.440955in}}%
\pgfpathlineto{\pgfqpoint{1.267039in}{0.440955in}}%
\pgfpathlineto{\pgfqpoint{1.242282in}{0.440955in}}%
\pgfpathlineto{\pgfqpoint{1.242282in}{0.440955in}}%
\pgfpathlineto{\pgfqpoint{1.217526in}{0.440955in}}%
\pgfpathlineto{\pgfqpoint{1.217526in}{0.440955in}}%
\pgfpathlineto{\pgfqpoint{1.192769in}{0.440955in}}%
\pgfpathlineto{\pgfqpoint{1.192769in}{0.440955in}}%
\pgfpathlineto{\pgfqpoint{1.168012in}{0.440955in}}%
\pgfpathlineto{\pgfqpoint{1.168012in}{0.440955in}}%
\pgfpathlineto{\pgfqpoint{1.143255in}{0.440955in}}%
\pgfpathlineto{\pgfqpoint{1.143255in}{0.440955in}}%
\pgfpathlineto{\pgfqpoint{1.118499in}{0.440955in}}%
\pgfpathlineto{\pgfqpoint{1.118499in}{0.440955in}}%
\pgfpathlineto{\pgfqpoint{1.093742in}{0.440955in}}%
\pgfpathlineto{\pgfqpoint{1.093742in}{0.440955in}}%
\pgfpathlineto{\pgfqpoint{1.068985in}{0.440955in}}%
\pgfusepath{stroke,fill}%
\end{pgfscope}%
\begin{pgfscope}%
\pgfpathrectangle{\pgfqpoint{0.592630in}{0.440955in}}{\pgfqpoint{3.222048in}{2.055572in}} %
\pgfusepath{clip}%
\pgfsetbuttcap%
\pgfsetmiterjoin%
\definecolor{currentfill}{rgb}{1.000000,0.000000,0.000000}%
\pgfsetfillcolor{currentfill}%
\pgfsetfillopacity{0.100000}%
\pgfsetlinewidth{1.003750pt}%
\definecolor{currentstroke}{rgb}{1.000000,0.000000,0.000000}%
\pgfsetstrokecolor{currentstroke}%
\pgfsetstrokeopacity{0.100000}%
\pgfsetdash{}{0pt}%
\pgfpathmoveto{\pgfqpoint{0.950635in}{0.440955in}}%
\pgfpathlineto{\pgfqpoint{0.950635in}{2.496527in}}%
\pgfpathlineto{\pgfqpoint{1.856274in}{2.496527in}}%
\pgfpathlineto{\pgfqpoint{1.856274in}{0.440955in}}%
\pgfpathlineto{\pgfqpoint{0.950635in}{0.440955in}}%
\pgfusepath{stroke,fill}%
\end{pgfscope}%
\begin{pgfscope}%
\pgfpathrectangle{\pgfqpoint{0.592630in}{0.440955in}}{\pgfqpoint{3.222048in}{2.055572in}} %
\pgfusepath{clip}%
\pgfsetbuttcap%
\pgfsetmiterjoin%
\definecolor{currentfill}{rgb}{1.000000,0.000000,0.000000}%
\pgfsetfillcolor{currentfill}%
\pgfsetfillopacity{0.100000}%
\pgfsetlinewidth{1.003750pt}%
\definecolor{currentstroke}{rgb}{1.000000,0.000000,0.000000}%
\pgfsetstrokecolor{currentstroke}%
\pgfsetstrokeopacity{0.100000}%
\pgfsetdash{}{0pt}%
\pgfpathmoveto{\pgfqpoint{1.999476in}{0.440955in}}%
\pgfpathlineto{\pgfqpoint{1.999476in}{2.496527in}}%
\pgfpathlineto{\pgfqpoint{3.814677in}{2.496527in}}%
\pgfpathlineto{\pgfqpoint{3.814677in}{0.440955in}}%
\pgfpathlineto{\pgfqpoint{1.999476in}{0.440955in}}%
\pgfusepath{stroke,fill}%
\end{pgfscope}%
\begin{pgfscope}%
\pgfpathrectangle{\pgfqpoint{0.592630in}{0.440955in}}{\pgfqpoint{3.222048in}{2.055572in}} %
\pgfusepath{clip}%
\pgfsetrectcap%
\pgfsetroundjoin%
\pgfsetlinewidth{1.003750pt}%
\definecolor{currentstroke}{rgb}{1.000000,0.000000,0.000000}%
\pgfsetstrokecolor{currentstroke}%
\pgfsetdash{}{0pt}%
\pgfpathmoveto{\pgfqpoint{1.856274in}{0.440955in}}%
\pgfpathlineto{\pgfqpoint{1.856274in}{2.496527in}}%
\pgfusepath{stroke}%
\end{pgfscope}%
\begin{pgfscope}%
\pgfpathrectangle{\pgfqpoint{0.592630in}{0.440955in}}{\pgfqpoint{3.222048in}{2.055572in}} %
\pgfusepath{clip}%
\pgfsetrectcap%
\pgfsetroundjoin%
\pgfsetlinewidth{1.003750pt}%
\definecolor{currentstroke}{rgb}{1.000000,0.000000,0.000000}%
\pgfsetstrokecolor{currentstroke}%
\pgfsetdash{}{0pt}%
\pgfpathmoveto{\pgfqpoint{1.999476in}{0.440955in}}%
\pgfpathlineto{\pgfqpoint{1.999476in}{2.496527in}}%
\pgfusepath{stroke}%
\end{pgfscope}%
\begin{pgfscope}%
\pgfsetrectcap%
\pgfsetmiterjoin%
\pgfsetlinewidth{1.003750pt}%
\definecolor{currentstroke}{rgb}{0.000000,0.000000,0.000000}%
\pgfsetstrokecolor{currentstroke}%
\pgfsetdash{}{0pt}%
\pgfpathmoveto{\pgfqpoint{0.592630in}{2.496527in}}%
\pgfpathlineto{\pgfqpoint{3.814677in}{2.496527in}}%
\pgfusepath{stroke}%
\end{pgfscope}%
\begin{pgfscope}%
\pgfsetrectcap%
\pgfsetmiterjoin%
\pgfsetlinewidth{1.003750pt}%
\definecolor{currentstroke}{rgb}{0.000000,0.000000,0.000000}%
\pgfsetstrokecolor{currentstroke}%
\pgfsetdash{}{0pt}%
\pgfpathmoveto{\pgfqpoint{3.814677in}{0.440955in}}%
\pgfpathlineto{\pgfqpoint{3.814677in}{2.496527in}}%
\pgfusepath{stroke}%
\end{pgfscope}%
\begin{pgfscope}%
\pgfsetrectcap%
\pgfsetmiterjoin%
\pgfsetlinewidth{1.003750pt}%
\definecolor{currentstroke}{rgb}{0.000000,0.000000,0.000000}%
\pgfsetstrokecolor{currentstroke}%
\pgfsetdash{}{0pt}%
\pgfpathmoveto{\pgfqpoint{0.592630in}{0.440955in}}%
\pgfpathlineto{\pgfqpoint{3.814677in}{0.440955in}}%
\pgfusepath{stroke}%
\end{pgfscope}%
\begin{pgfscope}%
\pgfsetrectcap%
\pgfsetmiterjoin%
\pgfsetlinewidth{1.003750pt}%
\definecolor{currentstroke}{rgb}{0.000000,0.000000,0.000000}%
\pgfsetstrokecolor{currentstroke}%
\pgfsetdash{}{0pt}%
\pgfpathmoveto{\pgfqpoint{0.592630in}{0.440955in}}%
\pgfpathlineto{\pgfqpoint{0.592630in}{2.496527in}}%
\pgfusepath{stroke}%
\end{pgfscope}%
\begin{pgfscope}%
\pgfsetbuttcap%
\pgfsetroundjoin%
\definecolor{currentfill}{rgb}{0.000000,0.000000,0.000000}%
\pgfsetfillcolor{currentfill}%
\pgfsetlinewidth{0.501875pt}%
\definecolor{currentstroke}{rgb}{0.000000,0.000000,0.000000}%
\pgfsetstrokecolor{currentstroke}%
\pgfsetdash{}{0pt}%
\pgfsys@defobject{currentmarker}{\pgfqpoint{0.000000in}{0.000000in}}{\pgfqpoint{0.000000in}{0.069444in}}{%
\pgfpathmoveto{\pgfqpoint{0.000000in}{0.000000in}}%
\pgfpathlineto{\pgfqpoint{0.000000in}{0.069444in}}%
\pgfusepath{stroke,fill}%
}%
\begin{pgfscope}%
\pgfsys@transformshift{0.592630in}{0.440955in}%
\pgfsys@useobject{currentmarker}{}%
\end{pgfscope}%
\end{pgfscope}%
\begin{pgfscope}%
\pgfsetbuttcap%
\pgfsetroundjoin%
\definecolor{currentfill}{rgb}{0.000000,0.000000,0.000000}%
\pgfsetfillcolor{currentfill}%
\pgfsetlinewidth{0.501875pt}%
\definecolor{currentstroke}{rgb}{0.000000,0.000000,0.000000}%
\pgfsetstrokecolor{currentstroke}%
\pgfsetdash{}{0pt}%
\pgfsys@defobject{currentmarker}{\pgfqpoint{0.000000in}{-0.069444in}}{\pgfqpoint{0.000000in}{0.000000in}}{%
\pgfpathmoveto{\pgfqpoint{0.000000in}{0.000000in}}%
\pgfpathlineto{\pgfqpoint{0.000000in}{-0.069444in}}%
\pgfusepath{stroke,fill}%
}%
\begin{pgfscope}%
\pgfsys@transformshift{0.592630in}{2.496527in}%
\pgfsys@useobject{currentmarker}{}%
\end{pgfscope}%
\end{pgfscope}%
\begin{pgfscope}%
\pgftext[x=0.592630in,y=0.371511in,,top]{\rmfamily\fontsize{8.000000}{9.600000}\selectfont 0}%
\end{pgfscope}%
\begin{pgfscope}%
\pgfsetbuttcap%
\pgfsetroundjoin%
\definecolor{currentfill}{rgb}{0.000000,0.000000,0.000000}%
\pgfsetfillcolor{currentfill}%
\pgfsetlinewidth{0.501875pt}%
\definecolor{currentstroke}{rgb}{0.000000,0.000000,0.000000}%
\pgfsetstrokecolor{currentstroke}%
\pgfsetdash{}{0pt}%
\pgfsys@defobject{currentmarker}{\pgfqpoint{0.000000in}{0.000000in}}{\pgfqpoint{0.000000in}{0.069444in}}{%
\pgfpathmoveto{\pgfqpoint{0.000000in}{0.000000in}}%
\pgfpathlineto{\pgfqpoint{0.000000in}{0.069444in}}%
\pgfusepath{stroke,fill}%
}%
\begin{pgfscope}%
\pgfsys@transformshift{0.950635in}{0.440955in}%
\pgfsys@useobject{currentmarker}{}%
\end{pgfscope}%
\end{pgfscope}%
\begin{pgfscope}%
\pgfsetbuttcap%
\pgfsetroundjoin%
\definecolor{currentfill}{rgb}{0.000000,0.000000,0.000000}%
\pgfsetfillcolor{currentfill}%
\pgfsetlinewidth{0.501875pt}%
\definecolor{currentstroke}{rgb}{0.000000,0.000000,0.000000}%
\pgfsetstrokecolor{currentstroke}%
\pgfsetdash{}{0pt}%
\pgfsys@defobject{currentmarker}{\pgfqpoint{0.000000in}{-0.069444in}}{\pgfqpoint{0.000000in}{0.000000in}}{%
\pgfpathmoveto{\pgfqpoint{0.000000in}{0.000000in}}%
\pgfpathlineto{\pgfqpoint{0.000000in}{-0.069444in}}%
\pgfusepath{stroke,fill}%
}%
\begin{pgfscope}%
\pgfsys@transformshift{0.950635in}{2.496527in}%
\pgfsys@useobject{currentmarker}{}%
\end{pgfscope}%
\end{pgfscope}%
\begin{pgfscope}%
\pgftext[x=0.950635in,y=0.371511in,,top]{\rmfamily\fontsize{8.000000}{9.600000}\selectfont 500}%
\end{pgfscope}%
\begin{pgfscope}%
\pgfsetbuttcap%
\pgfsetroundjoin%
\definecolor{currentfill}{rgb}{0.000000,0.000000,0.000000}%
\pgfsetfillcolor{currentfill}%
\pgfsetlinewidth{0.501875pt}%
\definecolor{currentstroke}{rgb}{0.000000,0.000000,0.000000}%
\pgfsetstrokecolor{currentstroke}%
\pgfsetdash{}{0pt}%
\pgfsys@defobject{currentmarker}{\pgfqpoint{0.000000in}{0.000000in}}{\pgfqpoint{0.000000in}{0.069444in}}{%
\pgfpathmoveto{\pgfqpoint{0.000000in}{0.000000in}}%
\pgfpathlineto{\pgfqpoint{0.000000in}{0.069444in}}%
\pgfusepath{stroke,fill}%
}%
\begin{pgfscope}%
\pgfsys@transformshift{1.308640in}{0.440955in}%
\pgfsys@useobject{currentmarker}{}%
\end{pgfscope}%
\end{pgfscope}%
\begin{pgfscope}%
\pgfsetbuttcap%
\pgfsetroundjoin%
\definecolor{currentfill}{rgb}{0.000000,0.000000,0.000000}%
\pgfsetfillcolor{currentfill}%
\pgfsetlinewidth{0.501875pt}%
\definecolor{currentstroke}{rgb}{0.000000,0.000000,0.000000}%
\pgfsetstrokecolor{currentstroke}%
\pgfsetdash{}{0pt}%
\pgfsys@defobject{currentmarker}{\pgfqpoint{0.000000in}{-0.069444in}}{\pgfqpoint{0.000000in}{0.000000in}}{%
\pgfpathmoveto{\pgfqpoint{0.000000in}{0.000000in}}%
\pgfpathlineto{\pgfqpoint{0.000000in}{-0.069444in}}%
\pgfusepath{stroke,fill}%
}%
\begin{pgfscope}%
\pgfsys@transformshift{1.308640in}{2.496527in}%
\pgfsys@useobject{currentmarker}{}%
\end{pgfscope}%
\end{pgfscope}%
\begin{pgfscope}%
\pgftext[x=1.308640in,y=0.371511in,,top]{\rmfamily\fontsize{8.000000}{9.600000}\selectfont 1000}%
\end{pgfscope}%
\begin{pgfscope}%
\pgfsetbuttcap%
\pgfsetroundjoin%
\definecolor{currentfill}{rgb}{0.000000,0.000000,0.000000}%
\pgfsetfillcolor{currentfill}%
\pgfsetlinewidth{0.501875pt}%
\definecolor{currentstroke}{rgb}{0.000000,0.000000,0.000000}%
\pgfsetstrokecolor{currentstroke}%
\pgfsetdash{}{0pt}%
\pgfsys@defobject{currentmarker}{\pgfqpoint{0.000000in}{0.000000in}}{\pgfqpoint{0.000000in}{0.069444in}}{%
\pgfpathmoveto{\pgfqpoint{0.000000in}{0.000000in}}%
\pgfpathlineto{\pgfqpoint{0.000000in}{0.069444in}}%
\pgfusepath{stroke,fill}%
}%
\begin{pgfscope}%
\pgfsys@transformshift{1.666646in}{0.440955in}%
\pgfsys@useobject{currentmarker}{}%
\end{pgfscope}%
\end{pgfscope}%
\begin{pgfscope}%
\pgfsetbuttcap%
\pgfsetroundjoin%
\definecolor{currentfill}{rgb}{0.000000,0.000000,0.000000}%
\pgfsetfillcolor{currentfill}%
\pgfsetlinewidth{0.501875pt}%
\definecolor{currentstroke}{rgb}{0.000000,0.000000,0.000000}%
\pgfsetstrokecolor{currentstroke}%
\pgfsetdash{}{0pt}%
\pgfsys@defobject{currentmarker}{\pgfqpoint{0.000000in}{-0.069444in}}{\pgfqpoint{0.000000in}{0.000000in}}{%
\pgfpathmoveto{\pgfqpoint{0.000000in}{0.000000in}}%
\pgfpathlineto{\pgfqpoint{0.000000in}{-0.069444in}}%
\pgfusepath{stroke,fill}%
}%
\begin{pgfscope}%
\pgfsys@transformshift{1.666646in}{2.496527in}%
\pgfsys@useobject{currentmarker}{}%
\end{pgfscope}%
\end{pgfscope}%
\begin{pgfscope}%
\pgftext[x=1.666646in,y=0.371511in,,top]{\rmfamily\fontsize{8.000000}{9.600000}\selectfont 1500}%
\end{pgfscope}%
\begin{pgfscope}%
\pgfsetbuttcap%
\pgfsetroundjoin%
\definecolor{currentfill}{rgb}{0.000000,0.000000,0.000000}%
\pgfsetfillcolor{currentfill}%
\pgfsetlinewidth{0.501875pt}%
\definecolor{currentstroke}{rgb}{0.000000,0.000000,0.000000}%
\pgfsetstrokecolor{currentstroke}%
\pgfsetdash{}{0pt}%
\pgfsys@defobject{currentmarker}{\pgfqpoint{0.000000in}{0.000000in}}{\pgfqpoint{0.000000in}{0.069444in}}{%
\pgfpathmoveto{\pgfqpoint{0.000000in}{0.000000in}}%
\pgfpathlineto{\pgfqpoint{0.000000in}{0.069444in}}%
\pgfusepath{stroke,fill}%
}%
\begin{pgfscope}%
\pgfsys@transformshift{2.024651in}{0.440955in}%
\pgfsys@useobject{currentmarker}{}%
\end{pgfscope}%
\end{pgfscope}%
\begin{pgfscope}%
\pgfsetbuttcap%
\pgfsetroundjoin%
\definecolor{currentfill}{rgb}{0.000000,0.000000,0.000000}%
\pgfsetfillcolor{currentfill}%
\pgfsetlinewidth{0.501875pt}%
\definecolor{currentstroke}{rgb}{0.000000,0.000000,0.000000}%
\pgfsetstrokecolor{currentstroke}%
\pgfsetdash{}{0pt}%
\pgfsys@defobject{currentmarker}{\pgfqpoint{0.000000in}{-0.069444in}}{\pgfqpoint{0.000000in}{0.000000in}}{%
\pgfpathmoveto{\pgfqpoint{0.000000in}{0.000000in}}%
\pgfpathlineto{\pgfqpoint{0.000000in}{-0.069444in}}%
\pgfusepath{stroke,fill}%
}%
\begin{pgfscope}%
\pgfsys@transformshift{2.024651in}{2.496527in}%
\pgfsys@useobject{currentmarker}{}%
\end{pgfscope}%
\end{pgfscope}%
\begin{pgfscope}%
\pgftext[x=2.024651in,y=0.371511in,,top]{\rmfamily\fontsize{8.000000}{9.600000}\selectfont 2000}%
\end{pgfscope}%
\begin{pgfscope}%
\pgfsetbuttcap%
\pgfsetroundjoin%
\definecolor{currentfill}{rgb}{0.000000,0.000000,0.000000}%
\pgfsetfillcolor{currentfill}%
\pgfsetlinewidth{0.501875pt}%
\definecolor{currentstroke}{rgb}{0.000000,0.000000,0.000000}%
\pgfsetstrokecolor{currentstroke}%
\pgfsetdash{}{0pt}%
\pgfsys@defobject{currentmarker}{\pgfqpoint{0.000000in}{0.000000in}}{\pgfqpoint{0.000000in}{0.069444in}}{%
\pgfpathmoveto{\pgfqpoint{0.000000in}{0.000000in}}%
\pgfpathlineto{\pgfqpoint{0.000000in}{0.069444in}}%
\pgfusepath{stroke,fill}%
}%
\begin{pgfscope}%
\pgfsys@transformshift{2.382656in}{0.440955in}%
\pgfsys@useobject{currentmarker}{}%
\end{pgfscope}%
\end{pgfscope}%
\begin{pgfscope}%
\pgfsetbuttcap%
\pgfsetroundjoin%
\definecolor{currentfill}{rgb}{0.000000,0.000000,0.000000}%
\pgfsetfillcolor{currentfill}%
\pgfsetlinewidth{0.501875pt}%
\definecolor{currentstroke}{rgb}{0.000000,0.000000,0.000000}%
\pgfsetstrokecolor{currentstroke}%
\pgfsetdash{}{0pt}%
\pgfsys@defobject{currentmarker}{\pgfqpoint{0.000000in}{-0.069444in}}{\pgfqpoint{0.000000in}{0.000000in}}{%
\pgfpathmoveto{\pgfqpoint{0.000000in}{0.000000in}}%
\pgfpathlineto{\pgfqpoint{0.000000in}{-0.069444in}}%
\pgfusepath{stroke,fill}%
}%
\begin{pgfscope}%
\pgfsys@transformshift{2.382656in}{2.496527in}%
\pgfsys@useobject{currentmarker}{}%
\end{pgfscope}%
\end{pgfscope}%
\begin{pgfscope}%
\pgftext[x=2.382656in,y=0.371511in,,top]{\rmfamily\fontsize{8.000000}{9.600000}\selectfont 2500}%
\end{pgfscope}%
\begin{pgfscope}%
\pgfsetbuttcap%
\pgfsetroundjoin%
\definecolor{currentfill}{rgb}{0.000000,0.000000,0.000000}%
\pgfsetfillcolor{currentfill}%
\pgfsetlinewidth{0.501875pt}%
\definecolor{currentstroke}{rgb}{0.000000,0.000000,0.000000}%
\pgfsetstrokecolor{currentstroke}%
\pgfsetdash{}{0pt}%
\pgfsys@defobject{currentmarker}{\pgfqpoint{0.000000in}{0.000000in}}{\pgfqpoint{0.000000in}{0.069444in}}{%
\pgfpathmoveto{\pgfqpoint{0.000000in}{0.000000in}}%
\pgfpathlineto{\pgfqpoint{0.000000in}{0.069444in}}%
\pgfusepath{stroke,fill}%
}%
\begin{pgfscope}%
\pgfsys@transformshift{2.740662in}{0.440955in}%
\pgfsys@useobject{currentmarker}{}%
\end{pgfscope}%
\end{pgfscope}%
\begin{pgfscope}%
\pgfsetbuttcap%
\pgfsetroundjoin%
\definecolor{currentfill}{rgb}{0.000000,0.000000,0.000000}%
\pgfsetfillcolor{currentfill}%
\pgfsetlinewidth{0.501875pt}%
\definecolor{currentstroke}{rgb}{0.000000,0.000000,0.000000}%
\pgfsetstrokecolor{currentstroke}%
\pgfsetdash{}{0pt}%
\pgfsys@defobject{currentmarker}{\pgfqpoint{0.000000in}{-0.069444in}}{\pgfqpoint{0.000000in}{0.000000in}}{%
\pgfpathmoveto{\pgfqpoint{0.000000in}{0.000000in}}%
\pgfpathlineto{\pgfqpoint{0.000000in}{-0.069444in}}%
\pgfusepath{stroke,fill}%
}%
\begin{pgfscope}%
\pgfsys@transformshift{2.740662in}{2.496527in}%
\pgfsys@useobject{currentmarker}{}%
\end{pgfscope}%
\end{pgfscope}%
\begin{pgfscope}%
\pgftext[x=2.740662in,y=0.371511in,,top]{\rmfamily\fontsize{8.000000}{9.600000}\selectfont 3000}%
\end{pgfscope}%
\begin{pgfscope}%
\pgfsetbuttcap%
\pgfsetroundjoin%
\definecolor{currentfill}{rgb}{0.000000,0.000000,0.000000}%
\pgfsetfillcolor{currentfill}%
\pgfsetlinewidth{0.501875pt}%
\definecolor{currentstroke}{rgb}{0.000000,0.000000,0.000000}%
\pgfsetstrokecolor{currentstroke}%
\pgfsetdash{}{0pt}%
\pgfsys@defobject{currentmarker}{\pgfqpoint{0.000000in}{0.000000in}}{\pgfqpoint{0.000000in}{0.069444in}}{%
\pgfpathmoveto{\pgfqpoint{0.000000in}{0.000000in}}%
\pgfpathlineto{\pgfqpoint{0.000000in}{0.069444in}}%
\pgfusepath{stroke,fill}%
}%
\begin{pgfscope}%
\pgfsys@transformshift{3.098667in}{0.440955in}%
\pgfsys@useobject{currentmarker}{}%
\end{pgfscope}%
\end{pgfscope}%
\begin{pgfscope}%
\pgfsetbuttcap%
\pgfsetroundjoin%
\definecolor{currentfill}{rgb}{0.000000,0.000000,0.000000}%
\pgfsetfillcolor{currentfill}%
\pgfsetlinewidth{0.501875pt}%
\definecolor{currentstroke}{rgb}{0.000000,0.000000,0.000000}%
\pgfsetstrokecolor{currentstroke}%
\pgfsetdash{}{0pt}%
\pgfsys@defobject{currentmarker}{\pgfqpoint{0.000000in}{-0.069444in}}{\pgfqpoint{0.000000in}{0.000000in}}{%
\pgfpathmoveto{\pgfqpoint{0.000000in}{0.000000in}}%
\pgfpathlineto{\pgfqpoint{0.000000in}{-0.069444in}}%
\pgfusepath{stroke,fill}%
}%
\begin{pgfscope}%
\pgfsys@transformshift{3.098667in}{2.496527in}%
\pgfsys@useobject{currentmarker}{}%
\end{pgfscope}%
\end{pgfscope}%
\begin{pgfscope}%
\pgftext[x=3.098667in,y=0.371511in,,top]{\rmfamily\fontsize{8.000000}{9.600000}\selectfont 3500}%
\end{pgfscope}%
\begin{pgfscope}%
\pgfsetbuttcap%
\pgfsetroundjoin%
\definecolor{currentfill}{rgb}{0.000000,0.000000,0.000000}%
\pgfsetfillcolor{currentfill}%
\pgfsetlinewidth{0.501875pt}%
\definecolor{currentstroke}{rgb}{0.000000,0.000000,0.000000}%
\pgfsetstrokecolor{currentstroke}%
\pgfsetdash{}{0pt}%
\pgfsys@defobject{currentmarker}{\pgfqpoint{0.000000in}{0.000000in}}{\pgfqpoint{0.000000in}{0.069444in}}{%
\pgfpathmoveto{\pgfqpoint{0.000000in}{0.000000in}}%
\pgfpathlineto{\pgfqpoint{0.000000in}{0.069444in}}%
\pgfusepath{stroke,fill}%
}%
\begin{pgfscope}%
\pgfsys@transformshift{3.456672in}{0.440955in}%
\pgfsys@useobject{currentmarker}{}%
\end{pgfscope}%
\end{pgfscope}%
\begin{pgfscope}%
\pgfsetbuttcap%
\pgfsetroundjoin%
\definecolor{currentfill}{rgb}{0.000000,0.000000,0.000000}%
\pgfsetfillcolor{currentfill}%
\pgfsetlinewidth{0.501875pt}%
\definecolor{currentstroke}{rgb}{0.000000,0.000000,0.000000}%
\pgfsetstrokecolor{currentstroke}%
\pgfsetdash{}{0pt}%
\pgfsys@defobject{currentmarker}{\pgfqpoint{0.000000in}{-0.069444in}}{\pgfqpoint{0.000000in}{0.000000in}}{%
\pgfpathmoveto{\pgfqpoint{0.000000in}{0.000000in}}%
\pgfpathlineto{\pgfqpoint{0.000000in}{-0.069444in}}%
\pgfusepath{stroke,fill}%
}%
\begin{pgfscope}%
\pgfsys@transformshift{3.456672in}{2.496527in}%
\pgfsys@useobject{currentmarker}{}%
\end{pgfscope}%
\end{pgfscope}%
\begin{pgfscope}%
\pgftext[x=3.456672in,y=0.371511in,,top]{\rmfamily\fontsize{8.000000}{9.600000}\selectfont 4000}%
\end{pgfscope}%
\begin{pgfscope}%
\pgfsetbuttcap%
\pgfsetroundjoin%
\definecolor{currentfill}{rgb}{0.000000,0.000000,0.000000}%
\pgfsetfillcolor{currentfill}%
\pgfsetlinewidth{0.501875pt}%
\definecolor{currentstroke}{rgb}{0.000000,0.000000,0.000000}%
\pgfsetstrokecolor{currentstroke}%
\pgfsetdash{}{0pt}%
\pgfsys@defobject{currentmarker}{\pgfqpoint{0.000000in}{0.000000in}}{\pgfqpoint{0.000000in}{0.069444in}}{%
\pgfpathmoveto{\pgfqpoint{0.000000in}{0.000000in}}%
\pgfpathlineto{\pgfqpoint{0.000000in}{0.069444in}}%
\pgfusepath{stroke,fill}%
}%
\begin{pgfscope}%
\pgfsys@transformshift{3.814677in}{0.440955in}%
\pgfsys@useobject{currentmarker}{}%
\end{pgfscope}%
\end{pgfscope}%
\begin{pgfscope}%
\pgfsetbuttcap%
\pgfsetroundjoin%
\definecolor{currentfill}{rgb}{0.000000,0.000000,0.000000}%
\pgfsetfillcolor{currentfill}%
\pgfsetlinewidth{0.501875pt}%
\definecolor{currentstroke}{rgb}{0.000000,0.000000,0.000000}%
\pgfsetstrokecolor{currentstroke}%
\pgfsetdash{}{0pt}%
\pgfsys@defobject{currentmarker}{\pgfqpoint{0.000000in}{-0.069444in}}{\pgfqpoint{0.000000in}{0.000000in}}{%
\pgfpathmoveto{\pgfqpoint{0.000000in}{0.000000in}}%
\pgfpathlineto{\pgfqpoint{0.000000in}{-0.069444in}}%
\pgfusepath{stroke,fill}%
}%
\begin{pgfscope}%
\pgfsys@transformshift{3.814677in}{2.496527in}%
\pgfsys@useobject{currentmarker}{}%
\end{pgfscope}%
\end{pgfscope}%
\begin{pgfscope}%
\pgftext[x=3.814677in,y=0.371511in,,top]{\rmfamily\fontsize{8.000000}{9.600000}\selectfont 4500}%
\end{pgfscope}%
\begin{pgfscope}%
\pgftext[x=2.203654in,y=0.194536in,,top]{\rmfamily\fontsize{9.000000}{10.800000}\selectfont \(\displaystyle m(K^+(\mu^-\to\pi^-))\ /\ \mathrm{MeV}\)}%
\end{pgfscope}%
\begin{pgfscope}%
\pgfsetbuttcap%
\pgfsetroundjoin%
\definecolor{currentfill}{rgb}{0.000000,0.000000,0.000000}%
\pgfsetfillcolor{currentfill}%
\pgfsetlinewidth{0.501875pt}%
\definecolor{currentstroke}{rgb}{0.000000,0.000000,0.000000}%
\pgfsetstrokecolor{currentstroke}%
\pgfsetdash{}{0pt}%
\pgfsys@defobject{currentmarker}{\pgfqpoint{0.000000in}{0.000000in}}{\pgfqpoint{0.069444in}{0.000000in}}{%
\pgfpathmoveto{\pgfqpoint{0.000000in}{0.000000in}}%
\pgfpathlineto{\pgfqpoint{0.069444in}{0.000000in}}%
\pgfusepath{stroke,fill}%
}%
\begin{pgfscope}%
\pgfsys@transformshift{0.592630in}{0.440955in}%
\pgfsys@useobject{currentmarker}{}%
\end{pgfscope}%
\end{pgfscope}%
\begin{pgfscope}%
\pgfsetbuttcap%
\pgfsetroundjoin%
\definecolor{currentfill}{rgb}{0.000000,0.000000,0.000000}%
\pgfsetfillcolor{currentfill}%
\pgfsetlinewidth{0.501875pt}%
\definecolor{currentstroke}{rgb}{0.000000,0.000000,0.000000}%
\pgfsetstrokecolor{currentstroke}%
\pgfsetdash{}{0pt}%
\pgfsys@defobject{currentmarker}{\pgfqpoint{-0.069444in}{0.000000in}}{\pgfqpoint{0.000000in}{0.000000in}}{%
\pgfpathmoveto{\pgfqpoint{0.000000in}{0.000000in}}%
\pgfpathlineto{\pgfqpoint{-0.069444in}{0.000000in}}%
\pgfusepath{stroke,fill}%
}%
\begin{pgfscope}%
\pgfsys@transformshift{3.814677in}{0.440955in}%
\pgfsys@useobject{currentmarker}{}%
\end{pgfscope}%
\end{pgfscope}%
\begin{pgfscope}%
\pgftext[x=0.523185in,y=0.440955in,right,]{\rmfamily\fontsize{8.000000}{9.600000}\selectfont 0}%
\end{pgfscope}%
\begin{pgfscope}%
\pgfsetbuttcap%
\pgfsetroundjoin%
\definecolor{currentfill}{rgb}{0.000000,0.000000,0.000000}%
\pgfsetfillcolor{currentfill}%
\pgfsetlinewidth{0.501875pt}%
\definecolor{currentstroke}{rgb}{0.000000,0.000000,0.000000}%
\pgfsetstrokecolor{currentstroke}%
\pgfsetdash{}{0pt}%
\pgfsys@defobject{currentmarker}{\pgfqpoint{0.000000in}{0.000000in}}{\pgfqpoint{0.069444in}{0.000000in}}{%
\pgfpathmoveto{\pgfqpoint{0.000000in}{0.000000in}}%
\pgfpathlineto{\pgfqpoint{0.069444in}{0.000000in}}%
\pgfusepath{stroke,fill}%
}%
\begin{pgfscope}%
\pgfsys@transformshift{0.592630in}{0.852070in}%
\pgfsys@useobject{currentmarker}{}%
\end{pgfscope}%
\end{pgfscope}%
\begin{pgfscope}%
\pgfsetbuttcap%
\pgfsetroundjoin%
\definecolor{currentfill}{rgb}{0.000000,0.000000,0.000000}%
\pgfsetfillcolor{currentfill}%
\pgfsetlinewidth{0.501875pt}%
\definecolor{currentstroke}{rgb}{0.000000,0.000000,0.000000}%
\pgfsetstrokecolor{currentstroke}%
\pgfsetdash{}{0pt}%
\pgfsys@defobject{currentmarker}{\pgfqpoint{-0.069444in}{0.000000in}}{\pgfqpoint{0.000000in}{0.000000in}}{%
\pgfpathmoveto{\pgfqpoint{0.000000in}{0.000000in}}%
\pgfpathlineto{\pgfqpoint{-0.069444in}{0.000000in}}%
\pgfusepath{stroke,fill}%
}%
\begin{pgfscope}%
\pgfsys@transformshift{3.814677in}{0.852070in}%
\pgfsys@useobject{currentmarker}{}%
\end{pgfscope}%
\end{pgfscope}%
\begin{pgfscope}%
\pgftext[x=0.523185in,y=0.852070in,right,]{\rmfamily\fontsize{8.000000}{9.600000}\selectfont 1000}%
\end{pgfscope}%
\begin{pgfscope}%
\pgfsetbuttcap%
\pgfsetroundjoin%
\definecolor{currentfill}{rgb}{0.000000,0.000000,0.000000}%
\pgfsetfillcolor{currentfill}%
\pgfsetlinewidth{0.501875pt}%
\definecolor{currentstroke}{rgb}{0.000000,0.000000,0.000000}%
\pgfsetstrokecolor{currentstroke}%
\pgfsetdash{}{0pt}%
\pgfsys@defobject{currentmarker}{\pgfqpoint{0.000000in}{0.000000in}}{\pgfqpoint{0.069444in}{0.000000in}}{%
\pgfpathmoveto{\pgfqpoint{0.000000in}{0.000000in}}%
\pgfpathlineto{\pgfqpoint{0.069444in}{0.000000in}}%
\pgfusepath{stroke,fill}%
}%
\begin{pgfscope}%
\pgfsys@transformshift{0.592630in}{1.263184in}%
\pgfsys@useobject{currentmarker}{}%
\end{pgfscope}%
\end{pgfscope}%
\begin{pgfscope}%
\pgfsetbuttcap%
\pgfsetroundjoin%
\definecolor{currentfill}{rgb}{0.000000,0.000000,0.000000}%
\pgfsetfillcolor{currentfill}%
\pgfsetlinewidth{0.501875pt}%
\definecolor{currentstroke}{rgb}{0.000000,0.000000,0.000000}%
\pgfsetstrokecolor{currentstroke}%
\pgfsetdash{}{0pt}%
\pgfsys@defobject{currentmarker}{\pgfqpoint{-0.069444in}{0.000000in}}{\pgfqpoint{0.000000in}{0.000000in}}{%
\pgfpathmoveto{\pgfqpoint{0.000000in}{0.000000in}}%
\pgfpathlineto{\pgfqpoint{-0.069444in}{0.000000in}}%
\pgfusepath{stroke,fill}%
}%
\begin{pgfscope}%
\pgfsys@transformshift{3.814677in}{1.263184in}%
\pgfsys@useobject{currentmarker}{}%
\end{pgfscope}%
\end{pgfscope}%
\begin{pgfscope}%
\pgftext[x=0.523185in,y=1.263184in,right,]{\rmfamily\fontsize{8.000000}{9.600000}\selectfont 2000}%
\end{pgfscope}%
\begin{pgfscope}%
\pgfsetbuttcap%
\pgfsetroundjoin%
\definecolor{currentfill}{rgb}{0.000000,0.000000,0.000000}%
\pgfsetfillcolor{currentfill}%
\pgfsetlinewidth{0.501875pt}%
\definecolor{currentstroke}{rgb}{0.000000,0.000000,0.000000}%
\pgfsetstrokecolor{currentstroke}%
\pgfsetdash{}{0pt}%
\pgfsys@defobject{currentmarker}{\pgfqpoint{0.000000in}{0.000000in}}{\pgfqpoint{0.069444in}{0.000000in}}{%
\pgfpathmoveto{\pgfqpoint{0.000000in}{0.000000in}}%
\pgfpathlineto{\pgfqpoint{0.069444in}{0.000000in}}%
\pgfusepath{stroke,fill}%
}%
\begin{pgfscope}%
\pgfsys@transformshift{0.592630in}{1.674298in}%
\pgfsys@useobject{currentmarker}{}%
\end{pgfscope}%
\end{pgfscope}%
\begin{pgfscope}%
\pgfsetbuttcap%
\pgfsetroundjoin%
\definecolor{currentfill}{rgb}{0.000000,0.000000,0.000000}%
\pgfsetfillcolor{currentfill}%
\pgfsetlinewidth{0.501875pt}%
\definecolor{currentstroke}{rgb}{0.000000,0.000000,0.000000}%
\pgfsetstrokecolor{currentstroke}%
\pgfsetdash{}{0pt}%
\pgfsys@defobject{currentmarker}{\pgfqpoint{-0.069444in}{0.000000in}}{\pgfqpoint{0.000000in}{0.000000in}}{%
\pgfpathmoveto{\pgfqpoint{0.000000in}{0.000000in}}%
\pgfpathlineto{\pgfqpoint{-0.069444in}{0.000000in}}%
\pgfusepath{stroke,fill}%
}%
\begin{pgfscope}%
\pgfsys@transformshift{3.814677in}{1.674298in}%
\pgfsys@useobject{currentmarker}{}%
\end{pgfscope}%
\end{pgfscope}%
\begin{pgfscope}%
\pgftext[x=0.523185in,y=1.674298in,right,]{\rmfamily\fontsize{8.000000}{9.600000}\selectfont 3000}%
\end{pgfscope}%
\begin{pgfscope}%
\pgfsetbuttcap%
\pgfsetroundjoin%
\definecolor{currentfill}{rgb}{0.000000,0.000000,0.000000}%
\pgfsetfillcolor{currentfill}%
\pgfsetlinewidth{0.501875pt}%
\definecolor{currentstroke}{rgb}{0.000000,0.000000,0.000000}%
\pgfsetstrokecolor{currentstroke}%
\pgfsetdash{}{0pt}%
\pgfsys@defobject{currentmarker}{\pgfqpoint{0.000000in}{0.000000in}}{\pgfqpoint{0.069444in}{0.000000in}}{%
\pgfpathmoveto{\pgfqpoint{0.000000in}{0.000000in}}%
\pgfpathlineto{\pgfqpoint{0.069444in}{0.000000in}}%
\pgfusepath{stroke,fill}%
}%
\begin{pgfscope}%
\pgfsys@transformshift{0.592630in}{2.085413in}%
\pgfsys@useobject{currentmarker}{}%
\end{pgfscope}%
\end{pgfscope}%
\begin{pgfscope}%
\pgfsetbuttcap%
\pgfsetroundjoin%
\definecolor{currentfill}{rgb}{0.000000,0.000000,0.000000}%
\pgfsetfillcolor{currentfill}%
\pgfsetlinewidth{0.501875pt}%
\definecolor{currentstroke}{rgb}{0.000000,0.000000,0.000000}%
\pgfsetstrokecolor{currentstroke}%
\pgfsetdash{}{0pt}%
\pgfsys@defobject{currentmarker}{\pgfqpoint{-0.069444in}{0.000000in}}{\pgfqpoint{0.000000in}{0.000000in}}{%
\pgfpathmoveto{\pgfqpoint{0.000000in}{0.000000in}}%
\pgfpathlineto{\pgfqpoint{-0.069444in}{0.000000in}}%
\pgfusepath{stroke,fill}%
}%
\begin{pgfscope}%
\pgfsys@transformshift{3.814677in}{2.085413in}%
\pgfsys@useobject{currentmarker}{}%
\end{pgfscope}%
\end{pgfscope}%
\begin{pgfscope}%
\pgftext[x=0.523185in,y=2.085413in,right,]{\rmfamily\fontsize{8.000000}{9.600000}\selectfont 4000}%
\end{pgfscope}%
\begin{pgfscope}%
\pgfsetbuttcap%
\pgfsetroundjoin%
\definecolor{currentfill}{rgb}{0.000000,0.000000,0.000000}%
\pgfsetfillcolor{currentfill}%
\pgfsetlinewidth{0.501875pt}%
\definecolor{currentstroke}{rgb}{0.000000,0.000000,0.000000}%
\pgfsetstrokecolor{currentstroke}%
\pgfsetdash{}{0pt}%
\pgfsys@defobject{currentmarker}{\pgfqpoint{0.000000in}{0.000000in}}{\pgfqpoint{0.069444in}{0.000000in}}{%
\pgfpathmoveto{\pgfqpoint{0.000000in}{0.000000in}}%
\pgfpathlineto{\pgfqpoint{0.069444in}{0.000000in}}%
\pgfusepath{stroke,fill}%
}%
\begin{pgfscope}%
\pgfsys@transformshift{0.592630in}{2.496527in}%
\pgfsys@useobject{currentmarker}{}%
\end{pgfscope}%
\end{pgfscope}%
\begin{pgfscope}%
\pgfsetbuttcap%
\pgfsetroundjoin%
\definecolor{currentfill}{rgb}{0.000000,0.000000,0.000000}%
\pgfsetfillcolor{currentfill}%
\pgfsetlinewidth{0.501875pt}%
\definecolor{currentstroke}{rgb}{0.000000,0.000000,0.000000}%
\pgfsetstrokecolor{currentstroke}%
\pgfsetdash{}{0pt}%
\pgfsys@defobject{currentmarker}{\pgfqpoint{-0.069444in}{0.000000in}}{\pgfqpoint{0.000000in}{0.000000in}}{%
\pgfpathmoveto{\pgfqpoint{0.000000in}{0.000000in}}%
\pgfpathlineto{\pgfqpoint{-0.069444in}{0.000000in}}%
\pgfusepath{stroke,fill}%
}%
\begin{pgfscope}%
\pgfsys@transformshift{3.814677in}{2.496527in}%
\pgfsys@useobject{currentmarker}{}%
\end{pgfscope}%
\end{pgfscope}%
\begin{pgfscope}%
\pgftext[x=0.523185in,y=2.496527in,right,]{\rmfamily\fontsize{8.000000}{9.600000}\selectfont 5000}%
\end{pgfscope}%
\begin{pgfscope}%
\pgftext[x=0.170972in,y=1.468741in,,bottom,rotate=90.000000]{\rmfamily\fontsize{9.000000}{10.800000}\selectfont Candidates}%
\end{pgfscope}%
\end{pgfpicture}%
\makeatother%
\endgroup%

  \caption{
    Invariant mass $m(\PKplus\Pmuon)$ of $\PBzero\to\PJpsi\PKstar$ candidates from data, reconstructed under a \Ppiminus hypothesis for the \Pmuon.
    The exclusion region of the $|m(\PKplus\Ppiminus) - m(\PDzero)| < \SI{100}{MeV}$ stripping cut is marked in red.
    This demonstrates that misreconstructed $\PBzero\to\PJpsi\PKstar$ candidates consistute a significant background contribution to the analysis.
  }
  \label{fig:doubleswap}
\end{figure}

\subsubsection{Veto on partially reconstructed \texorpdfstring{$\PBzero\to\PDstar^-\APmuon\Pneutrino$}{B->D*munu}}

Decays of \PBzero to $\PDstar^-\APmuon\Pneutrino$, where the $\PDstar^-$ decays to $\APDzero\Ppiminus$ and the \Ppiminus is identified as a \Pmuon, while the \Pneutrino is not reconstructed, constitute another background.
Because the neutrino momentum is not accounted for, this background appears in the lower sideband of the reconstructed \PBzero mass, ranging up to the nominal \PBzero mass.

It is found that this background contribution is not sufficiently removed by the later multivariate selection step.
The invariant mass of the $\PDstar^-$, corresponding to $m(\PKplus\Ppiminus\mu^-_\pi)$ with a \Ppiminus mass hypothesis for the \Pmuon is reconstructed (see figure \ref{fig:dstar}) and the background contribution is removed by excluding candidates with $\SI{1990}{GeV} < m(\PKplus\Ppiminus\mu^-_\pi) < \SI{2030}{GeV}$.

\begin{figure}
  \centering
  %% Creator: Matplotlib, PGF backend
%%
%% To include the figure in your LaTeX document, write
%%   \input{<filename>.pgf}
%%
%% Make sure the required packages are loaded in your preamble
%%   \usepackage{pgf}
%%
%% Figures using additional raster images can only be included by \input if
%% they are in the same directory as the main LaTeX file. For loading figures
%% from other directories you can use the `import` package
%%   \usepackage{import}
%% and then include the figures with
%%   \import{<path to file>}{<filename>.pgf}
%%
%% Matplotlib used the following preamble
%%   \usepackage{fontspec}
%%   \setmainfont{DejaVu Serif}
%%   \setsansfont{DejaVu Sans}
%%   \setmonofont{DejaVu Sans Mono}
%%
\begingroup%
\makeatletter%
\begin{pgfpicture}%
\pgfpathrectangle{\pgfpointorigin}{\pgfqpoint{3.976875in}{2.319335in}}%
\pgfusepath{use as bounding box, clip}%
\begin{pgfscope}%
\pgfsetbuttcap%
\pgfsetmiterjoin%
\definecolor{currentfill}{rgb}{1.000000,1.000000,1.000000}%
\pgfsetfillcolor{currentfill}%
\pgfsetlinewidth{0.000000pt}%
\definecolor{currentstroke}{rgb}{1.000000,1.000000,1.000000}%
\pgfsetstrokecolor{currentstroke}%
\pgfsetdash{}{0pt}%
\pgfpathmoveto{\pgfqpoint{0.000000in}{0.000000in}}%
\pgfpathlineto{\pgfqpoint{3.976875in}{0.000000in}}%
\pgfpathlineto{\pgfqpoint{3.976875in}{2.319335in}}%
\pgfpathlineto{\pgfqpoint{0.000000in}{2.319335in}}%
\pgfpathclose%
\pgfusepath{fill}%
\end{pgfscope}%
\begin{pgfscope}%
\pgfsetbuttcap%
\pgfsetmiterjoin%
\definecolor{currentfill}{rgb}{1.000000,1.000000,1.000000}%
\pgfsetfillcolor{currentfill}%
\pgfsetlinewidth{0.000000pt}%
\definecolor{currentstroke}{rgb}{0.000000,0.000000,0.000000}%
\pgfsetstrokecolor{currentstroke}%
\pgfsetstrokeopacity{0.000000}%
\pgfsetdash{}{0pt}%
\pgfpathmoveto{\pgfqpoint{0.331521in}{0.440955in}}%
\pgfpathlineto{\pgfqpoint{3.785491in}{0.440955in}}%
\pgfpathlineto{\pgfqpoint{3.785491in}{2.215570in}}%
\pgfpathlineto{\pgfqpoint{0.331521in}{2.215570in}}%
\pgfpathclose%
\pgfusepath{fill}%
\end{pgfscope}%
\begin{pgfscope}%
\pgfpathrectangle{\pgfqpoint{0.331521in}{0.440955in}}{\pgfqpoint{3.453970in}{1.774614in}} %
\pgfusepath{clip}%
\pgfsetbuttcap%
\pgfsetmiterjoin%
\definecolor{currentfill}{rgb}{0.215686,0.470588,0.749020}%
\pgfsetfillcolor{currentfill}%
\pgfsetlinewidth{1.003750pt}%
\definecolor{currentstroke}{rgb}{0.000000,0.000000,0.000000}%
\pgfsetstrokecolor{currentstroke}%
\pgfsetdash{}{0pt}%
\pgfpathmoveto{\pgfqpoint{0.331521in}{0.440955in}}%
\pgfpathlineto{\pgfqpoint{0.331521in}{0.440955in}}%
\pgfpathlineto{\pgfqpoint{0.366061in}{0.440955in}}%
\pgfpathlineto{\pgfqpoint{0.366061in}{0.440955in}}%
\pgfpathlineto{\pgfqpoint{0.400601in}{0.440955in}}%
\pgfpathlineto{\pgfqpoint{0.400601in}{0.440955in}}%
\pgfpathlineto{\pgfqpoint{0.435140in}{0.440955in}}%
\pgfpathlineto{\pgfqpoint{0.435140in}{0.440955in}}%
\pgfpathlineto{\pgfqpoint{0.469680in}{0.440955in}}%
\pgfpathlineto{\pgfqpoint{0.469680in}{0.449828in}}%
\pgfpathlineto{\pgfqpoint{0.504220in}{0.449828in}}%
\pgfpathlineto{\pgfqpoint{0.504220in}{0.449828in}}%
\pgfpathlineto{\pgfqpoint{0.538759in}{0.449828in}}%
\pgfpathlineto{\pgfqpoint{0.538759in}{0.485321in}}%
\pgfpathlineto{\pgfqpoint{0.573299in}{0.485321in}}%
\pgfpathlineto{\pgfqpoint{0.573299in}{0.476448in}}%
\pgfpathlineto{\pgfqpoint{0.607839in}{0.476448in}}%
\pgfpathlineto{\pgfqpoint{0.607839in}{0.463138in}}%
\pgfpathlineto{\pgfqpoint{0.642379in}{0.463138in}}%
\pgfpathlineto{\pgfqpoint{0.642379in}{0.476448in}}%
\pgfpathlineto{\pgfqpoint{0.676918in}{0.476448in}}%
\pgfpathlineto{\pgfqpoint{0.676918in}{0.498630in}}%
\pgfpathlineto{\pgfqpoint{0.711458in}{0.498630in}}%
\pgfpathlineto{\pgfqpoint{0.711458in}{0.498630in}}%
\pgfpathlineto{\pgfqpoint{0.745998in}{0.498630in}}%
\pgfpathlineto{\pgfqpoint{0.745998in}{0.529686in}}%
\pgfpathlineto{\pgfqpoint{0.780537in}{0.529686in}}%
\pgfpathlineto{\pgfqpoint{0.780537in}{0.467575in}}%
\pgfpathlineto{\pgfqpoint{0.815077in}{0.467575in}}%
\pgfpathlineto{\pgfqpoint{0.815077in}{0.520813in}}%
\pgfpathlineto{\pgfqpoint{0.849617in}{0.520813in}}%
\pgfpathlineto{\pgfqpoint{0.849617in}{0.516377in}}%
\pgfpathlineto{\pgfqpoint{0.884156in}{0.516377in}}%
\pgfpathlineto{\pgfqpoint{0.884156in}{0.534123in}}%
\pgfpathlineto{\pgfqpoint{0.918696in}{0.534123in}}%
\pgfpathlineto{\pgfqpoint{0.918696in}{0.520813in}}%
\pgfpathlineto{\pgfqpoint{0.953236in}{0.520813in}}%
\pgfpathlineto{\pgfqpoint{0.953236in}{0.529686in}}%
\pgfpathlineto{\pgfqpoint{0.987775in}{0.529686in}}%
\pgfpathlineto{\pgfqpoint{0.987775in}{0.569615in}}%
\pgfpathlineto{\pgfqpoint{1.022315in}{0.569615in}}%
\pgfpathlineto{\pgfqpoint{1.022315in}{0.538559in}}%
\pgfpathlineto{\pgfqpoint{1.056855in}{0.538559in}}%
\pgfpathlineto{\pgfqpoint{1.056855in}{0.560742in}}%
\pgfpathlineto{\pgfqpoint{1.091395in}{0.560742in}}%
\pgfpathlineto{\pgfqpoint{1.091395in}{0.591798in}}%
\pgfpathlineto{\pgfqpoint{1.125934in}{0.591798in}}%
\pgfpathlineto{\pgfqpoint{1.125934in}{0.591798in}}%
\pgfpathlineto{\pgfqpoint{1.160474in}{0.591798in}}%
\pgfpathlineto{\pgfqpoint{1.160474in}{0.569615in}}%
\pgfpathlineto{\pgfqpoint{1.195014in}{0.569615in}}%
\pgfpathlineto{\pgfqpoint{1.195014in}{0.534123in}}%
\pgfpathlineto{\pgfqpoint{1.229553in}{0.534123in}}%
\pgfpathlineto{\pgfqpoint{1.229553in}{0.551869in}}%
\pgfpathlineto{\pgfqpoint{1.264093in}{0.551869in}}%
\pgfpathlineto{\pgfqpoint{1.264093in}{0.600671in}}%
\pgfpathlineto{\pgfqpoint{1.298633in}{0.600671in}}%
\pgfpathlineto{\pgfqpoint{1.298633in}{0.569615in}}%
\pgfpathlineto{\pgfqpoint{1.333172in}{0.569615in}}%
\pgfpathlineto{\pgfqpoint{1.333172in}{0.649473in}}%
\pgfpathlineto{\pgfqpoint{1.367712in}{0.649473in}}%
\pgfpathlineto{\pgfqpoint{1.367712in}{0.600671in}}%
\pgfpathlineto{\pgfqpoint{1.402252in}{0.600671in}}%
\pgfpathlineto{\pgfqpoint{1.402252in}{0.640600in}}%
\pgfpathlineto{\pgfqpoint{1.436792in}{0.640600in}}%
\pgfpathlineto{\pgfqpoint{1.436792in}{0.600671in}}%
\pgfpathlineto{\pgfqpoint{1.471331in}{0.600671in}}%
\pgfpathlineto{\pgfqpoint{1.471331in}{0.631726in}}%
\pgfpathlineto{\pgfqpoint{1.505871in}{0.631726in}}%
\pgfpathlineto{\pgfqpoint{1.505871in}{0.649473in}}%
\pgfpathlineto{\pgfqpoint{1.540411in}{0.649473in}}%
\pgfpathlineto{\pgfqpoint{1.540411in}{0.605107in}}%
\pgfpathlineto{\pgfqpoint{1.574950in}{0.605107in}}%
\pgfpathlineto{\pgfqpoint{1.574950in}{0.596234in}}%
\pgfpathlineto{\pgfqpoint{1.609490in}{0.596234in}}%
\pgfpathlineto{\pgfqpoint{1.609490in}{0.645036in}}%
\pgfpathlineto{\pgfqpoint{1.644030in}{0.645036in}}%
\pgfpathlineto{\pgfqpoint{1.644030in}{0.724894in}}%
\pgfpathlineto{\pgfqpoint{1.678569in}{0.724894in}}%
\pgfpathlineto{\pgfqpoint{1.678569in}{0.702711in}}%
\pgfpathlineto{\pgfqpoint{1.713109in}{0.702711in}}%
\pgfpathlineto{\pgfqpoint{1.713109in}{0.707148in}}%
\pgfpathlineto{\pgfqpoint{1.747649in}{0.707148in}}%
\pgfpathlineto{\pgfqpoint{1.747649in}{0.720457in}}%
\pgfpathlineto{\pgfqpoint{1.782188in}{0.720457in}}%
\pgfpathlineto{\pgfqpoint{1.782188in}{0.769259in}}%
\pgfpathlineto{\pgfqpoint{1.816728in}{0.769259in}}%
\pgfpathlineto{\pgfqpoint{1.816728in}{0.760386in}}%
\pgfpathlineto{\pgfqpoint{1.851268in}{0.760386in}}%
\pgfpathlineto{\pgfqpoint{1.851268in}{0.787005in}}%
\pgfpathlineto{\pgfqpoint{1.885808in}{0.787005in}}%
\pgfpathlineto{\pgfqpoint{1.885808in}{0.804751in}}%
\pgfpathlineto{\pgfqpoint{1.920347in}{0.804751in}}%
\pgfpathlineto{\pgfqpoint{1.920347in}{0.818061in}}%
\pgfpathlineto{\pgfqpoint{1.954887in}{0.818061in}}%
\pgfpathlineto{\pgfqpoint{1.954887in}{0.960030in}}%
\pgfpathlineto{\pgfqpoint{1.989427in}{0.960030in}}%
\pgfpathlineto{\pgfqpoint{1.989427in}{1.017705in}}%
\pgfpathlineto{\pgfqpoint{2.023966in}{1.017705in}}%
\pgfpathlineto{\pgfqpoint{2.023966in}{1.168547in}}%
\pgfpathlineto{\pgfqpoint{2.058506in}{1.168547in}}%
\pgfpathlineto{\pgfqpoint{2.058506in}{1.243968in}}%
\pgfpathlineto{\pgfqpoint{2.093046in}{1.243968in}}%
\pgfpathlineto{\pgfqpoint{2.093046in}{1.452485in}}%
\pgfpathlineto{\pgfqpoint{2.127585in}{1.452485in}}%
\pgfpathlineto{\pgfqpoint{2.127585in}{1.532343in}}%
\pgfpathlineto{\pgfqpoint{2.162125in}{1.532343in}}%
\pgfpathlineto{\pgfqpoint{2.162125in}{1.616637in}}%
\pgfpathlineto{\pgfqpoint{2.196665in}{1.616637in}}%
\pgfpathlineto{\pgfqpoint{2.196665in}{1.953814in}}%
\pgfpathlineto{\pgfqpoint{2.231204in}{1.953814in}}%
\pgfpathlineto{\pgfqpoint{2.231204in}{1.856210in}}%
\pgfpathlineto{\pgfqpoint{2.265744in}{1.856210in}}%
\pgfpathlineto{\pgfqpoint{2.265744in}{1.967124in}}%
\pgfpathlineto{\pgfqpoint{2.300284in}{1.967124in}}%
\pgfpathlineto{\pgfqpoint{2.300284in}{2.029235in}}%
\pgfpathlineto{\pgfqpoint{2.334824in}{2.029235in}}%
\pgfpathlineto{\pgfqpoint{2.334824in}{1.878393in}}%
\pgfpathlineto{\pgfqpoint{2.369363in}{1.878393in}}%
\pgfpathlineto{\pgfqpoint{2.369363in}{1.674312in}}%
\pgfpathlineto{\pgfqpoint{2.403903in}{1.674312in}}%
\pgfpathlineto{\pgfqpoint{2.403903in}{1.700931in}}%
\pgfpathlineto{\pgfqpoint{2.438443in}{1.700931in}}%
\pgfpathlineto{\pgfqpoint{2.438443in}{1.412557in}}%
\pgfpathlineto{\pgfqpoint{2.472982in}{1.412557in}}%
\pgfpathlineto{\pgfqpoint{2.472982in}{1.354882in}}%
\pgfpathlineto{\pgfqpoint{2.507522in}{1.354882in}}%
\pgfpathlineto{\pgfqpoint{2.507522in}{1.314953in}}%
\pgfpathlineto{\pgfqpoint{2.542062in}{1.314953in}}%
\pgfpathlineto{\pgfqpoint{2.542062in}{1.212913in}}%
\pgfpathlineto{\pgfqpoint{2.576601in}{1.212913in}}%
\pgfpathlineto{\pgfqpoint{2.576601in}{1.146365in}}%
\pgfpathlineto{\pgfqpoint{2.611141in}{1.146365in}}%
\pgfpathlineto{\pgfqpoint{2.611141in}{1.235095in}}%
\pgfpathlineto{\pgfqpoint{2.645681in}{1.235095in}}%
\pgfpathlineto{\pgfqpoint{2.645681in}{1.128618in}}%
\pgfpathlineto{\pgfqpoint{2.680221in}{1.128618in}}%
\pgfpathlineto{\pgfqpoint{2.680221in}{1.172984in}}%
\pgfpathlineto{\pgfqpoint{2.714760in}{1.172984in}}%
\pgfpathlineto{\pgfqpoint{2.714760in}{1.195166in}}%
\pgfpathlineto{\pgfqpoint{2.749300in}{1.195166in}}%
\pgfpathlineto{\pgfqpoint{2.749300in}{1.199603in}}%
\pgfpathlineto{\pgfqpoint{2.783840in}{1.199603in}}%
\pgfpathlineto{\pgfqpoint{2.783840in}{1.159674in}}%
\pgfpathlineto{\pgfqpoint{2.818379in}{1.159674in}}%
\pgfpathlineto{\pgfqpoint{2.818379in}{1.146365in}}%
\pgfpathlineto{\pgfqpoint{2.852919in}{1.146365in}}%
\pgfpathlineto{\pgfqpoint{2.852919in}{1.199603in}}%
\pgfpathlineto{\pgfqpoint{2.887459in}{1.199603in}}%
\pgfpathlineto{\pgfqpoint{2.887459in}{1.199603in}}%
\pgfpathlineto{\pgfqpoint{2.921998in}{1.199603in}}%
\pgfpathlineto{\pgfqpoint{2.921998in}{1.181857in}}%
\pgfpathlineto{\pgfqpoint{2.956538in}{1.181857in}}%
\pgfpathlineto{\pgfqpoint{2.956538in}{1.204040in}}%
\pgfpathlineto{\pgfqpoint{2.991078in}{1.204040in}}%
\pgfpathlineto{\pgfqpoint{2.991078in}{1.323826in}}%
\pgfpathlineto{\pgfqpoint{3.025617in}{1.323826in}}%
\pgfpathlineto{\pgfqpoint{3.025617in}{1.390374in}}%
\pgfpathlineto{\pgfqpoint{3.060157in}{1.390374in}}%
\pgfpathlineto{\pgfqpoint{3.060157in}{1.275024in}}%
\pgfpathlineto{\pgfqpoint{3.094697in}{1.275024in}}%
\pgfpathlineto{\pgfqpoint{3.094697in}{1.288334in}}%
\pgfpathlineto{\pgfqpoint{3.129237in}{1.288334in}}%
\pgfpathlineto{\pgfqpoint{3.129237in}{1.350445in}}%
\pgfpathlineto{\pgfqpoint{3.163776in}{1.350445in}}%
\pgfpathlineto{\pgfqpoint{3.163776in}{1.306080in}}%
\pgfpathlineto{\pgfqpoint{3.198316in}{1.306080in}}%
\pgfpathlineto{\pgfqpoint{3.198316in}{1.252841in}}%
\pgfpathlineto{\pgfqpoint{3.232856in}{1.252841in}}%
\pgfpathlineto{\pgfqpoint{3.232856in}{1.275024in}}%
\pgfpathlineto{\pgfqpoint{3.267395in}{1.275024in}}%
\pgfpathlineto{\pgfqpoint{3.267395in}{1.319389in}}%
\pgfpathlineto{\pgfqpoint{3.301935in}{1.319389in}}%
\pgfpathlineto{\pgfqpoint{3.301935in}{1.297207in}}%
\pgfpathlineto{\pgfqpoint{3.336475in}{1.297207in}}%
\pgfpathlineto{\pgfqpoint{3.336475in}{1.354882in}}%
\pgfpathlineto{\pgfqpoint{3.371014in}{1.354882in}}%
\pgfpathlineto{\pgfqpoint{3.371014in}{1.319389in}}%
\pgfpathlineto{\pgfqpoint{3.405554in}{1.319389in}}%
\pgfpathlineto{\pgfqpoint{3.405554in}{1.443612in}}%
\pgfpathlineto{\pgfqpoint{3.440094in}{1.443612in}}%
\pgfpathlineto{\pgfqpoint{3.440094in}{1.421430in}}%
\pgfpathlineto{\pgfqpoint{3.474634in}{1.421430in}}%
\pgfpathlineto{\pgfqpoint{3.474634in}{1.381501in}}%
\pgfpathlineto{\pgfqpoint{3.509173in}{1.381501in}}%
\pgfpathlineto{\pgfqpoint{3.509173in}{1.385937in}}%
\pgfpathlineto{\pgfqpoint{3.543713in}{1.385937in}}%
\pgfpathlineto{\pgfqpoint{3.543713in}{1.496851in}}%
\pgfpathlineto{\pgfqpoint{3.578253in}{1.496851in}}%
\pgfpathlineto{\pgfqpoint{3.578253in}{1.487978in}}%
\pgfpathlineto{\pgfqpoint{3.612792in}{1.487978in}}%
\pgfpathlineto{\pgfqpoint{3.612792in}{1.328263in}}%
\pgfpathlineto{\pgfqpoint{3.647332in}{1.328263in}}%
\pgfpathlineto{\pgfqpoint{3.647332in}{1.541216in}}%
\pgfpathlineto{\pgfqpoint{3.681872in}{1.541216in}}%
\pgfpathlineto{\pgfqpoint{3.681872in}{1.558962in}}%
\pgfpathlineto{\pgfqpoint{3.716411in}{1.558962in}}%
\pgfpathlineto{\pgfqpoint{3.716411in}{1.492414in}}%
\pgfpathlineto{\pgfqpoint{3.750951in}{1.492414in}}%
\pgfpathlineto{\pgfqpoint{3.750951in}{1.465795in}}%
\pgfpathlineto{\pgfqpoint{3.785491in}{1.465795in}}%
\pgfpathlineto{\pgfqpoint{3.785491in}{0.440955in}}%
\pgfpathlineto{\pgfqpoint{3.750951in}{0.440955in}}%
\pgfpathlineto{\pgfqpoint{3.750951in}{0.440955in}}%
\pgfpathlineto{\pgfqpoint{3.716411in}{0.440955in}}%
\pgfpathlineto{\pgfqpoint{3.716411in}{0.440955in}}%
\pgfpathlineto{\pgfqpoint{3.681872in}{0.440955in}}%
\pgfpathlineto{\pgfqpoint{3.681872in}{0.440955in}}%
\pgfpathlineto{\pgfqpoint{3.647332in}{0.440955in}}%
\pgfpathlineto{\pgfqpoint{3.647332in}{0.440955in}}%
\pgfpathlineto{\pgfqpoint{3.612792in}{0.440955in}}%
\pgfpathlineto{\pgfqpoint{3.612792in}{0.440955in}}%
\pgfpathlineto{\pgfqpoint{3.578253in}{0.440955in}}%
\pgfpathlineto{\pgfqpoint{3.578253in}{0.440955in}}%
\pgfpathlineto{\pgfqpoint{3.543713in}{0.440955in}}%
\pgfpathlineto{\pgfqpoint{3.543713in}{0.440955in}}%
\pgfpathlineto{\pgfqpoint{3.509173in}{0.440955in}}%
\pgfpathlineto{\pgfqpoint{3.509173in}{0.440955in}}%
\pgfpathlineto{\pgfqpoint{3.474634in}{0.440955in}}%
\pgfpathlineto{\pgfqpoint{3.474634in}{0.440955in}}%
\pgfpathlineto{\pgfqpoint{3.440094in}{0.440955in}}%
\pgfpathlineto{\pgfqpoint{3.440094in}{0.440955in}}%
\pgfpathlineto{\pgfqpoint{3.405554in}{0.440955in}}%
\pgfpathlineto{\pgfqpoint{3.405554in}{0.440955in}}%
\pgfpathlineto{\pgfqpoint{3.371014in}{0.440955in}}%
\pgfpathlineto{\pgfqpoint{3.371014in}{0.440955in}}%
\pgfpathlineto{\pgfqpoint{3.336475in}{0.440955in}}%
\pgfpathlineto{\pgfqpoint{3.336475in}{0.440955in}}%
\pgfpathlineto{\pgfqpoint{3.301935in}{0.440955in}}%
\pgfpathlineto{\pgfqpoint{3.301935in}{0.440955in}}%
\pgfpathlineto{\pgfqpoint{3.267395in}{0.440955in}}%
\pgfpathlineto{\pgfqpoint{3.267395in}{0.440955in}}%
\pgfpathlineto{\pgfqpoint{3.232856in}{0.440955in}}%
\pgfpathlineto{\pgfqpoint{3.232856in}{0.440955in}}%
\pgfpathlineto{\pgfqpoint{3.198316in}{0.440955in}}%
\pgfpathlineto{\pgfqpoint{3.198316in}{0.440955in}}%
\pgfpathlineto{\pgfqpoint{3.163776in}{0.440955in}}%
\pgfpathlineto{\pgfqpoint{3.163776in}{0.440955in}}%
\pgfpathlineto{\pgfqpoint{3.129237in}{0.440955in}}%
\pgfpathlineto{\pgfqpoint{3.129237in}{0.440955in}}%
\pgfpathlineto{\pgfqpoint{3.094697in}{0.440955in}}%
\pgfpathlineto{\pgfqpoint{3.094697in}{0.440955in}}%
\pgfpathlineto{\pgfqpoint{3.060157in}{0.440955in}}%
\pgfpathlineto{\pgfqpoint{3.060157in}{0.440955in}}%
\pgfpathlineto{\pgfqpoint{3.025617in}{0.440955in}}%
\pgfpathlineto{\pgfqpoint{3.025617in}{0.440955in}}%
\pgfpathlineto{\pgfqpoint{2.991078in}{0.440955in}}%
\pgfpathlineto{\pgfqpoint{2.991078in}{0.440955in}}%
\pgfpathlineto{\pgfqpoint{2.956538in}{0.440955in}}%
\pgfpathlineto{\pgfqpoint{2.956538in}{0.440955in}}%
\pgfpathlineto{\pgfqpoint{2.921998in}{0.440955in}}%
\pgfpathlineto{\pgfqpoint{2.921998in}{0.440955in}}%
\pgfpathlineto{\pgfqpoint{2.887459in}{0.440955in}}%
\pgfpathlineto{\pgfqpoint{2.887459in}{0.440955in}}%
\pgfpathlineto{\pgfqpoint{2.852919in}{0.440955in}}%
\pgfpathlineto{\pgfqpoint{2.852919in}{0.440955in}}%
\pgfpathlineto{\pgfqpoint{2.818379in}{0.440955in}}%
\pgfpathlineto{\pgfqpoint{2.818379in}{0.440955in}}%
\pgfpathlineto{\pgfqpoint{2.783840in}{0.440955in}}%
\pgfpathlineto{\pgfqpoint{2.783840in}{0.440955in}}%
\pgfpathlineto{\pgfqpoint{2.749300in}{0.440955in}}%
\pgfpathlineto{\pgfqpoint{2.749300in}{0.440955in}}%
\pgfpathlineto{\pgfqpoint{2.714760in}{0.440955in}}%
\pgfpathlineto{\pgfqpoint{2.714760in}{0.440955in}}%
\pgfpathlineto{\pgfqpoint{2.680221in}{0.440955in}}%
\pgfpathlineto{\pgfqpoint{2.680221in}{0.440955in}}%
\pgfpathlineto{\pgfqpoint{2.645681in}{0.440955in}}%
\pgfpathlineto{\pgfqpoint{2.645681in}{0.440955in}}%
\pgfpathlineto{\pgfqpoint{2.611141in}{0.440955in}}%
\pgfpathlineto{\pgfqpoint{2.611141in}{0.440955in}}%
\pgfpathlineto{\pgfqpoint{2.576601in}{0.440955in}}%
\pgfpathlineto{\pgfqpoint{2.576601in}{0.440955in}}%
\pgfpathlineto{\pgfqpoint{2.542062in}{0.440955in}}%
\pgfpathlineto{\pgfqpoint{2.542062in}{0.440955in}}%
\pgfpathlineto{\pgfqpoint{2.507522in}{0.440955in}}%
\pgfpathlineto{\pgfqpoint{2.507522in}{0.440955in}}%
\pgfpathlineto{\pgfqpoint{2.472982in}{0.440955in}}%
\pgfpathlineto{\pgfqpoint{2.472982in}{0.440955in}}%
\pgfpathlineto{\pgfqpoint{2.438443in}{0.440955in}}%
\pgfpathlineto{\pgfqpoint{2.438443in}{0.440955in}}%
\pgfpathlineto{\pgfqpoint{2.403903in}{0.440955in}}%
\pgfpathlineto{\pgfqpoint{2.403903in}{0.440955in}}%
\pgfpathlineto{\pgfqpoint{2.369363in}{0.440955in}}%
\pgfpathlineto{\pgfqpoint{2.369363in}{0.440955in}}%
\pgfpathlineto{\pgfqpoint{2.334824in}{0.440955in}}%
\pgfpathlineto{\pgfqpoint{2.334824in}{0.440955in}}%
\pgfpathlineto{\pgfqpoint{2.300284in}{0.440955in}}%
\pgfpathlineto{\pgfqpoint{2.300284in}{0.440955in}}%
\pgfpathlineto{\pgfqpoint{2.265744in}{0.440955in}}%
\pgfpathlineto{\pgfqpoint{2.265744in}{0.440955in}}%
\pgfpathlineto{\pgfqpoint{2.231204in}{0.440955in}}%
\pgfpathlineto{\pgfqpoint{2.231204in}{0.440955in}}%
\pgfpathlineto{\pgfqpoint{2.196665in}{0.440955in}}%
\pgfpathlineto{\pgfqpoint{2.196665in}{0.440955in}}%
\pgfpathlineto{\pgfqpoint{2.162125in}{0.440955in}}%
\pgfpathlineto{\pgfqpoint{2.162125in}{0.440955in}}%
\pgfpathlineto{\pgfqpoint{2.127585in}{0.440955in}}%
\pgfpathlineto{\pgfqpoint{2.127585in}{0.440955in}}%
\pgfpathlineto{\pgfqpoint{2.093046in}{0.440955in}}%
\pgfpathlineto{\pgfqpoint{2.093046in}{0.440955in}}%
\pgfpathlineto{\pgfqpoint{2.058506in}{0.440955in}}%
\pgfpathlineto{\pgfqpoint{2.058506in}{0.440955in}}%
\pgfpathlineto{\pgfqpoint{2.023966in}{0.440955in}}%
\pgfpathlineto{\pgfqpoint{2.023966in}{0.440955in}}%
\pgfpathlineto{\pgfqpoint{1.989427in}{0.440955in}}%
\pgfpathlineto{\pgfqpoint{1.989427in}{0.440955in}}%
\pgfpathlineto{\pgfqpoint{1.954887in}{0.440955in}}%
\pgfpathlineto{\pgfqpoint{1.954887in}{0.440955in}}%
\pgfpathlineto{\pgfqpoint{1.920347in}{0.440955in}}%
\pgfpathlineto{\pgfqpoint{1.920347in}{0.440955in}}%
\pgfpathlineto{\pgfqpoint{1.885808in}{0.440955in}}%
\pgfpathlineto{\pgfqpoint{1.885808in}{0.440955in}}%
\pgfpathlineto{\pgfqpoint{1.851268in}{0.440955in}}%
\pgfpathlineto{\pgfqpoint{1.851268in}{0.440955in}}%
\pgfpathlineto{\pgfqpoint{1.816728in}{0.440955in}}%
\pgfpathlineto{\pgfqpoint{1.816728in}{0.440955in}}%
\pgfpathlineto{\pgfqpoint{1.782188in}{0.440955in}}%
\pgfpathlineto{\pgfqpoint{1.782188in}{0.440955in}}%
\pgfpathlineto{\pgfqpoint{1.747649in}{0.440955in}}%
\pgfpathlineto{\pgfqpoint{1.747649in}{0.440955in}}%
\pgfpathlineto{\pgfqpoint{1.713109in}{0.440955in}}%
\pgfpathlineto{\pgfqpoint{1.713109in}{0.440955in}}%
\pgfpathlineto{\pgfqpoint{1.678569in}{0.440955in}}%
\pgfpathlineto{\pgfqpoint{1.678569in}{0.440955in}}%
\pgfpathlineto{\pgfqpoint{1.644030in}{0.440955in}}%
\pgfpathlineto{\pgfqpoint{1.644030in}{0.440955in}}%
\pgfpathlineto{\pgfqpoint{1.609490in}{0.440955in}}%
\pgfpathlineto{\pgfqpoint{1.609490in}{0.440955in}}%
\pgfpathlineto{\pgfqpoint{1.574950in}{0.440955in}}%
\pgfpathlineto{\pgfqpoint{1.574950in}{0.440955in}}%
\pgfpathlineto{\pgfqpoint{1.540411in}{0.440955in}}%
\pgfpathlineto{\pgfqpoint{1.540411in}{0.440955in}}%
\pgfpathlineto{\pgfqpoint{1.505871in}{0.440955in}}%
\pgfpathlineto{\pgfqpoint{1.505871in}{0.440955in}}%
\pgfpathlineto{\pgfqpoint{1.471331in}{0.440955in}}%
\pgfpathlineto{\pgfqpoint{1.471331in}{0.440955in}}%
\pgfpathlineto{\pgfqpoint{1.436792in}{0.440955in}}%
\pgfpathlineto{\pgfqpoint{1.436792in}{0.440955in}}%
\pgfpathlineto{\pgfqpoint{1.402252in}{0.440955in}}%
\pgfpathlineto{\pgfqpoint{1.402252in}{0.440955in}}%
\pgfpathlineto{\pgfqpoint{1.367712in}{0.440955in}}%
\pgfpathlineto{\pgfqpoint{1.367712in}{0.440955in}}%
\pgfpathlineto{\pgfqpoint{1.333172in}{0.440955in}}%
\pgfpathlineto{\pgfqpoint{1.333172in}{0.440955in}}%
\pgfpathlineto{\pgfqpoint{1.298633in}{0.440955in}}%
\pgfpathlineto{\pgfqpoint{1.298633in}{0.440955in}}%
\pgfpathlineto{\pgfqpoint{1.264093in}{0.440955in}}%
\pgfpathlineto{\pgfqpoint{1.264093in}{0.440955in}}%
\pgfpathlineto{\pgfqpoint{1.229553in}{0.440955in}}%
\pgfpathlineto{\pgfqpoint{1.229553in}{0.440955in}}%
\pgfpathlineto{\pgfqpoint{1.195014in}{0.440955in}}%
\pgfpathlineto{\pgfqpoint{1.195014in}{0.440955in}}%
\pgfpathlineto{\pgfqpoint{1.160474in}{0.440955in}}%
\pgfpathlineto{\pgfqpoint{1.160474in}{0.440955in}}%
\pgfpathlineto{\pgfqpoint{1.125934in}{0.440955in}}%
\pgfpathlineto{\pgfqpoint{1.125934in}{0.440955in}}%
\pgfpathlineto{\pgfqpoint{1.091395in}{0.440955in}}%
\pgfpathlineto{\pgfqpoint{1.091395in}{0.440955in}}%
\pgfpathlineto{\pgfqpoint{1.056855in}{0.440955in}}%
\pgfpathlineto{\pgfqpoint{1.056855in}{0.440955in}}%
\pgfpathlineto{\pgfqpoint{1.022315in}{0.440955in}}%
\pgfpathlineto{\pgfqpoint{1.022315in}{0.440955in}}%
\pgfpathlineto{\pgfqpoint{0.987775in}{0.440955in}}%
\pgfpathlineto{\pgfqpoint{0.987775in}{0.440955in}}%
\pgfpathlineto{\pgfqpoint{0.953236in}{0.440955in}}%
\pgfpathlineto{\pgfqpoint{0.953236in}{0.440955in}}%
\pgfpathlineto{\pgfqpoint{0.918696in}{0.440955in}}%
\pgfpathlineto{\pgfqpoint{0.918696in}{0.440955in}}%
\pgfpathlineto{\pgfqpoint{0.884156in}{0.440955in}}%
\pgfpathlineto{\pgfqpoint{0.884156in}{0.440955in}}%
\pgfpathlineto{\pgfqpoint{0.849617in}{0.440955in}}%
\pgfpathlineto{\pgfqpoint{0.849617in}{0.440955in}}%
\pgfpathlineto{\pgfqpoint{0.815077in}{0.440955in}}%
\pgfpathlineto{\pgfqpoint{0.815077in}{0.440955in}}%
\pgfpathlineto{\pgfqpoint{0.780537in}{0.440955in}}%
\pgfpathlineto{\pgfqpoint{0.780537in}{0.440955in}}%
\pgfpathlineto{\pgfqpoint{0.745998in}{0.440955in}}%
\pgfpathlineto{\pgfqpoint{0.745998in}{0.440955in}}%
\pgfpathlineto{\pgfqpoint{0.711458in}{0.440955in}}%
\pgfpathlineto{\pgfqpoint{0.711458in}{0.440955in}}%
\pgfpathlineto{\pgfqpoint{0.676918in}{0.440955in}}%
\pgfpathlineto{\pgfqpoint{0.676918in}{0.440955in}}%
\pgfpathlineto{\pgfqpoint{0.642379in}{0.440955in}}%
\pgfpathlineto{\pgfqpoint{0.642379in}{0.440955in}}%
\pgfpathlineto{\pgfqpoint{0.607839in}{0.440955in}}%
\pgfpathlineto{\pgfqpoint{0.607839in}{0.440955in}}%
\pgfpathlineto{\pgfqpoint{0.573299in}{0.440955in}}%
\pgfpathlineto{\pgfqpoint{0.573299in}{0.440955in}}%
\pgfpathlineto{\pgfqpoint{0.538759in}{0.440955in}}%
\pgfpathlineto{\pgfqpoint{0.538759in}{0.440955in}}%
\pgfpathlineto{\pgfqpoint{0.504220in}{0.440955in}}%
\pgfpathlineto{\pgfqpoint{0.504220in}{0.440955in}}%
\pgfpathlineto{\pgfqpoint{0.469680in}{0.440955in}}%
\pgfpathlineto{\pgfqpoint{0.469680in}{0.440955in}}%
\pgfpathlineto{\pgfqpoint{0.435140in}{0.440955in}}%
\pgfpathlineto{\pgfqpoint{0.435140in}{0.440955in}}%
\pgfpathlineto{\pgfqpoint{0.400601in}{0.440955in}}%
\pgfpathlineto{\pgfqpoint{0.400601in}{0.440955in}}%
\pgfpathlineto{\pgfqpoint{0.366061in}{0.440955in}}%
\pgfpathlineto{\pgfqpoint{0.366061in}{0.440955in}}%
\pgfpathlineto{\pgfqpoint{0.331521in}{0.440955in}}%
\pgfusepath{stroke,fill}%
\end{pgfscope}%
\begin{pgfscope}%
\pgfpathrectangle{\pgfqpoint{0.331521in}{0.440955in}}{\pgfqpoint{3.453970in}{1.774614in}} %
\pgfusepath{clip}%
\pgfsetbuttcap%
\pgfsetmiterjoin%
\definecolor{currentfill}{rgb}{1.000000,0.000000,0.000000}%
\pgfsetfillcolor{currentfill}%
\pgfsetfillopacity{0.200000}%
\pgfsetlinewidth{1.003750pt}%
\definecolor{currentstroke}{rgb}{1.000000,0.000000,0.000000}%
\pgfsetstrokecolor{currentstroke}%
\pgfsetstrokeopacity{0.200000}%
\pgfsetdash{}{0pt}%
\pgfpathmoveto{\pgfqpoint{1.885808in}{0.440955in}}%
\pgfpathlineto{\pgfqpoint{1.885808in}{2.215570in}}%
\pgfpathlineto{\pgfqpoint{2.576601in}{2.215570in}}%
\pgfpathlineto{\pgfqpoint{2.576601in}{0.440955in}}%
\pgfpathlineto{\pgfqpoint{1.885808in}{0.440955in}}%
\pgfusepath{stroke,fill}%
\end{pgfscope}%
\begin{pgfscope}%
\pgfpathrectangle{\pgfqpoint{0.331521in}{0.440955in}}{\pgfqpoint{3.453970in}{1.774614in}} %
\pgfusepath{clip}%
\pgfsetrectcap%
\pgfsetroundjoin%
\pgfsetlinewidth{1.003750pt}%
\definecolor{currentstroke}{rgb}{1.000000,0.000000,0.000000}%
\pgfsetstrokecolor{currentstroke}%
\pgfsetstrokeopacity{0.800000}%
\pgfsetdash{}{0pt}%
\pgfpathmoveto{\pgfqpoint{1.885808in}{0.440955in}}%
\pgfpathlineto{\pgfqpoint{1.885808in}{2.215570in}}%
\pgfusepath{stroke}%
\end{pgfscope}%
\begin{pgfscope}%
\pgfpathrectangle{\pgfqpoint{0.331521in}{0.440955in}}{\pgfqpoint{3.453970in}{1.774614in}} %
\pgfusepath{clip}%
\pgfsetrectcap%
\pgfsetroundjoin%
\pgfsetlinewidth{1.003750pt}%
\definecolor{currentstroke}{rgb}{1.000000,0.000000,0.000000}%
\pgfsetstrokecolor{currentstroke}%
\pgfsetstrokeopacity{0.800000}%
\pgfsetdash{}{0pt}%
\pgfpathmoveto{\pgfqpoint{2.576601in}{0.440955in}}%
\pgfpathlineto{\pgfqpoint{2.576601in}{2.215570in}}%
\pgfusepath{stroke}%
\end{pgfscope}%
\begin{pgfscope}%
\pgfsetrectcap%
\pgfsetmiterjoin%
\pgfsetlinewidth{1.003750pt}%
\definecolor{currentstroke}{rgb}{0.000000,0.000000,0.000000}%
\pgfsetstrokecolor{currentstroke}%
\pgfsetdash{}{0pt}%
\pgfpathmoveto{\pgfqpoint{0.331521in}{2.215570in}}%
\pgfpathlineto{\pgfqpoint{3.785491in}{2.215570in}}%
\pgfusepath{stroke}%
\end{pgfscope}%
\begin{pgfscope}%
\pgfsetrectcap%
\pgfsetmiterjoin%
\pgfsetlinewidth{1.003750pt}%
\definecolor{currentstroke}{rgb}{0.000000,0.000000,0.000000}%
\pgfsetstrokecolor{currentstroke}%
\pgfsetdash{}{0pt}%
\pgfpathmoveto{\pgfqpoint{3.785491in}{0.440955in}}%
\pgfpathlineto{\pgfqpoint{3.785491in}{2.215570in}}%
\pgfusepath{stroke}%
\end{pgfscope}%
\begin{pgfscope}%
\pgfsetrectcap%
\pgfsetmiterjoin%
\pgfsetlinewidth{1.003750pt}%
\definecolor{currentstroke}{rgb}{0.000000,0.000000,0.000000}%
\pgfsetstrokecolor{currentstroke}%
\pgfsetdash{}{0pt}%
\pgfpathmoveto{\pgfqpoint{0.331521in}{0.440955in}}%
\pgfpathlineto{\pgfqpoint{3.785491in}{0.440955in}}%
\pgfusepath{stroke}%
\end{pgfscope}%
\begin{pgfscope}%
\pgfsetrectcap%
\pgfsetmiterjoin%
\pgfsetlinewidth{1.003750pt}%
\definecolor{currentstroke}{rgb}{0.000000,0.000000,0.000000}%
\pgfsetstrokecolor{currentstroke}%
\pgfsetdash{}{0pt}%
\pgfpathmoveto{\pgfqpoint{0.331521in}{0.440955in}}%
\pgfpathlineto{\pgfqpoint{0.331521in}{2.215570in}}%
\pgfusepath{stroke}%
\end{pgfscope}%
\begin{pgfscope}%
\pgfsetbuttcap%
\pgfsetroundjoin%
\definecolor{currentfill}{rgb}{0.000000,0.000000,0.000000}%
\pgfsetfillcolor{currentfill}%
\pgfsetlinewidth{0.501875pt}%
\definecolor{currentstroke}{rgb}{0.000000,0.000000,0.000000}%
\pgfsetstrokecolor{currentstroke}%
\pgfsetdash{}{0pt}%
\pgfsys@defobject{currentmarker}{\pgfqpoint{0.000000in}{0.000000in}}{\pgfqpoint{0.000000in}{0.069444in}}{%
\pgfpathmoveto{\pgfqpoint{0.000000in}{0.000000in}}%
\pgfpathlineto{\pgfqpoint{0.000000in}{0.069444in}}%
\pgfusepath{stroke,fill}%
}%
\begin{pgfscope}%
\pgfsys@transformshift{0.331521in}{0.440955in}%
\pgfsys@useobject{currentmarker}{}%
\end{pgfscope}%
\end{pgfscope}%
\begin{pgfscope}%
\pgfsetbuttcap%
\pgfsetroundjoin%
\definecolor{currentfill}{rgb}{0.000000,0.000000,0.000000}%
\pgfsetfillcolor{currentfill}%
\pgfsetlinewidth{0.501875pt}%
\definecolor{currentstroke}{rgb}{0.000000,0.000000,0.000000}%
\pgfsetstrokecolor{currentstroke}%
\pgfsetdash{}{0pt}%
\pgfsys@defobject{currentmarker}{\pgfqpoint{0.000000in}{-0.069444in}}{\pgfqpoint{0.000000in}{0.000000in}}{%
\pgfpathmoveto{\pgfqpoint{0.000000in}{0.000000in}}%
\pgfpathlineto{\pgfqpoint{0.000000in}{-0.069444in}}%
\pgfusepath{stroke,fill}%
}%
\begin{pgfscope}%
\pgfsys@transformshift{0.331521in}{2.215570in}%
\pgfsys@useobject{currentmarker}{}%
\end{pgfscope}%
\end{pgfscope}%
\begin{pgfscope}%
\pgftext[x=0.331521in,y=0.371511in,,top]{\rmfamily\fontsize{8.000000}{9.600000}\selectfont 1900}%
\end{pgfscope}%
\begin{pgfscope}%
\pgfsetbuttcap%
\pgfsetroundjoin%
\definecolor{currentfill}{rgb}{0.000000,0.000000,0.000000}%
\pgfsetfillcolor{currentfill}%
\pgfsetlinewidth{0.501875pt}%
\definecolor{currentstroke}{rgb}{0.000000,0.000000,0.000000}%
\pgfsetstrokecolor{currentstroke}%
\pgfsetdash{}{0pt}%
\pgfsys@defobject{currentmarker}{\pgfqpoint{0.000000in}{0.000000in}}{\pgfqpoint{0.000000in}{0.069444in}}{%
\pgfpathmoveto{\pgfqpoint{0.000000in}{0.000000in}}%
\pgfpathlineto{\pgfqpoint{0.000000in}{0.069444in}}%
\pgfusepath{stroke,fill}%
}%
\begin{pgfscope}%
\pgfsys@transformshift{1.195014in}{0.440955in}%
\pgfsys@useobject{currentmarker}{}%
\end{pgfscope}%
\end{pgfscope}%
\begin{pgfscope}%
\pgfsetbuttcap%
\pgfsetroundjoin%
\definecolor{currentfill}{rgb}{0.000000,0.000000,0.000000}%
\pgfsetfillcolor{currentfill}%
\pgfsetlinewidth{0.501875pt}%
\definecolor{currentstroke}{rgb}{0.000000,0.000000,0.000000}%
\pgfsetstrokecolor{currentstroke}%
\pgfsetdash{}{0pt}%
\pgfsys@defobject{currentmarker}{\pgfqpoint{0.000000in}{-0.069444in}}{\pgfqpoint{0.000000in}{0.000000in}}{%
\pgfpathmoveto{\pgfqpoint{0.000000in}{0.000000in}}%
\pgfpathlineto{\pgfqpoint{0.000000in}{-0.069444in}}%
\pgfusepath{stroke,fill}%
}%
\begin{pgfscope}%
\pgfsys@transformshift{1.195014in}{2.215570in}%
\pgfsys@useobject{currentmarker}{}%
\end{pgfscope}%
\end{pgfscope}%
\begin{pgfscope}%
\pgftext[x=1.195014in,y=0.371511in,,top]{\rmfamily\fontsize{8.000000}{9.600000}\selectfont 1950}%
\end{pgfscope}%
\begin{pgfscope}%
\pgfsetbuttcap%
\pgfsetroundjoin%
\definecolor{currentfill}{rgb}{0.000000,0.000000,0.000000}%
\pgfsetfillcolor{currentfill}%
\pgfsetlinewidth{0.501875pt}%
\definecolor{currentstroke}{rgb}{0.000000,0.000000,0.000000}%
\pgfsetstrokecolor{currentstroke}%
\pgfsetdash{}{0pt}%
\pgfsys@defobject{currentmarker}{\pgfqpoint{0.000000in}{0.000000in}}{\pgfqpoint{0.000000in}{0.069444in}}{%
\pgfpathmoveto{\pgfqpoint{0.000000in}{0.000000in}}%
\pgfpathlineto{\pgfqpoint{0.000000in}{0.069444in}}%
\pgfusepath{stroke,fill}%
}%
\begin{pgfscope}%
\pgfsys@transformshift{2.058506in}{0.440955in}%
\pgfsys@useobject{currentmarker}{}%
\end{pgfscope}%
\end{pgfscope}%
\begin{pgfscope}%
\pgfsetbuttcap%
\pgfsetroundjoin%
\definecolor{currentfill}{rgb}{0.000000,0.000000,0.000000}%
\pgfsetfillcolor{currentfill}%
\pgfsetlinewidth{0.501875pt}%
\definecolor{currentstroke}{rgb}{0.000000,0.000000,0.000000}%
\pgfsetstrokecolor{currentstroke}%
\pgfsetdash{}{0pt}%
\pgfsys@defobject{currentmarker}{\pgfqpoint{0.000000in}{-0.069444in}}{\pgfqpoint{0.000000in}{0.000000in}}{%
\pgfpathmoveto{\pgfqpoint{0.000000in}{0.000000in}}%
\pgfpathlineto{\pgfqpoint{0.000000in}{-0.069444in}}%
\pgfusepath{stroke,fill}%
}%
\begin{pgfscope}%
\pgfsys@transformshift{2.058506in}{2.215570in}%
\pgfsys@useobject{currentmarker}{}%
\end{pgfscope}%
\end{pgfscope}%
\begin{pgfscope}%
\pgftext[x=2.058506in,y=0.371511in,,top]{\rmfamily\fontsize{8.000000}{9.600000}\selectfont 2000}%
\end{pgfscope}%
\begin{pgfscope}%
\pgfsetbuttcap%
\pgfsetroundjoin%
\definecolor{currentfill}{rgb}{0.000000,0.000000,0.000000}%
\pgfsetfillcolor{currentfill}%
\pgfsetlinewidth{0.501875pt}%
\definecolor{currentstroke}{rgb}{0.000000,0.000000,0.000000}%
\pgfsetstrokecolor{currentstroke}%
\pgfsetdash{}{0pt}%
\pgfsys@defobject{currentmarker}{\pgfqpoint{0.000000in}{0.000000in}}{\pgfqpoint{0.000000in}{0.069444in}}{%
\pgfpathmoveto{\pgfqpoint{0.000000in}{0.000000in}}%
\pgfpathlineto{\pgfqpoint{0.000000in}{0.069444in}}%
\pgfusepath{stroke,fill}%
}%
\begin{pgfscope}%
\pgfsys@transformshift{2.921998in}{0.440955in}%
\pgfsys@useobject{currentmarker}{}%
\end{pgfscope}%
\end{pgfscope}%
\begin{pgfscope}%
\pgfsetbuttcap%
\pgfsetroundjoin%
\definecolor{currentfill}{rgb}{0.000000,0.000000,0.000000}%
\pgfsetfillcolor{currentfill}%
\pgfsetlinewidth{0.501875pt}%
\definecolor{currentstroke}{rgb}{0.000000,0.000000,0.000000}%
\pgfsetstrokecolor{currentstroke}%
\pgfsetdash{}{0pt}%
\pgfsys@defobject{currentmarker}{\pgfqpoint{0.000000in}{-0.069444in}}{\pgfqpoint{0.000000in}{0.000000in}}{%
\pgfpathmoveto{\pgfqpoint{0.000000in}{0.000000in}}%
\pgfpathlineto{\pgfqpoint{0.000000in}{-0.069444in}}%
\pgfusepath{stroke,fill}%
}%
\begin{pgfscope}%
\pgfsys@transformshift{2.921998in}{2.215570in}%
\pgfsys@useobject{currentmarker}{}%
\end{pgfscope}%
\end{pgfscope}%
\begin{pgfscope}%
\pgftext[x=2.921998in,y=0.371511in,,top]{\rmfamily\fontsize{8.000000}{9.600000}\selectfont 2050}%
\end{pgfscope}%
\begin{pgfscope}%
\pgfsetbuttcap%
\pgfsetroundjoin%
\definecolor{currentfill}{rgb}{0.000000,0.000000,0.000000}%
\pgfsetfillcolor{currentfill}%
\pgfsetlinewidth{0.501875pt}%
\definecolor{currentstroke}{rgb}{0.000000,0.000000,0.000000}%
\pgfsetstrokecolor{currentstroke}%
\pgfsetdash{}{0pt}%
\pgfsys@defobject{currentmarker}{\pgfqpoint{0.000000in}{0.000000in}}{\pgfqpoint{0.000000in}{0.069444in}}{%
\pgfpathmoveto{\pgfqpoint{0.000000in}{0.000000in}}%
\pgfpathlineto{\pgfqpoint{0.000000in}{0.069444in}}%
\pgfusepath{stroke,fill}%
}%
\begin{pgfscope}%
\pgfsys@transformshift{3.785491in}{0.440955in}%
\pgfsys@useobject{currentmarker}{}%
\end{pgfscope}%
\end{pgfscope}%
\begin{pgfscope}%
\pgfsetbuttcap%
\pgfsetroundjoin%
\definecolor{currentfill}{rgb}{0.000000,0.000000,0.000000}%
\pgfsetfillcolor{currentfill}%
\pgfsetlinewidth{0.501875pt}%
\definecolor{currentstroke}{rgb}{0.000000,0.000000,0.000000}%
\pgfsetstrokecolor{currentstroke}%
\pgfsetdash{}{0pt}%
\pgfsys@defobject{currentmarker}{\pgfqpoint{0.000000in}{-0.069444in}}{\pgfqpoint{0.000000in}{0.000000in}}{%
\pgfpathmoveto{\pgfqpoint{0.000000in}{0.000000in}}%
\pgfpathlineto{\pgfqpoint{0.000000in}{-0.069444in}}%
\pgfusepath{stroke,fill}%
}%
\begin{pgfscope}%
\pgfsys@transformshift{3.785491in}{2.215570in}%
\pgfsys@useobject{currentmarker}{}%
\end{pgfscope}%
\end{pgfscope}%
\begin{pgfscope}%
\pgftext[x=3.785491in,y=0.371511in,,top]{\rmfamily\fontsize{8.000000}{9.600000}\selectfont 2100}%
\end{pgfscope}%
\begin{pgfscope}%
\pgftext[x=2.058506in,y=0.194536in,,top]{\rmfamily\fontsize{9.000000}{10.800000}\selectfont \(\displaystyle m(K^+\pi^-(\mu^- \to \pi^-))\ /\ \mathrm{MeV}\)}%
\end{pgfscope}%
\begin{pgfscope}%
\pgfsetbuttcap%
\pgfsetroundjoin%
\definecolor{currentfill}{rgb}{0.000000,0.000000,0.000000}%
\pgfsetfillcolor{currentfill}%
\pgfsetlinewidth{0.501875pt}%
\definecolor{currentstroke}{rgb}{0.000000,0.000000,0.000000}%
\pgfsetstrokecolor{currentstroke}%
\pgfsetdash{}{0pt}%
\pgfsys@defobject{currentmarker}{\pgfqpoint{0.000000in}{0.000000in}}{\pgfqpoint{0.069444in}{0.000000in}}{%
\pgfpathmoveto{\pgfqpoint{0.000000in}{0.000000in}}%
\pgfpathlineto{\pgfqpoint{0.069444in}{0.000000in}}%
\pgfusepath{stroke,fill}%
}%
\begin{pgfscope}%
\pgfsys@transformshift{0.331521in}{0.440955in}%
\pgfsys@useobject{currentmarker}{}%
\end{pgfscope}%
\end{pgfscope}%
\begin{pgfscope}%
\pgfsetbuttcap%
\pgfsetroundjoin%
\definecolor{currentfill}{rgb}{0.000000,0.000000,0.000000}%
\pgfsetfillcolor{currentfill}%
\pgfsetlinewidth{0.501875pt}%
\definecolor{currentstroke}{rgb}{0.000000,0.000000,0.000000}%
\pgfsetstrokecolor{currentstroke}%
\pgfsetdash{}{0pt}%
\pgfsys@defobject{currentmarker}{\pgfqpoint{-0.069444in}{0.000000in}}{\pgfqpoint{0.000000in}{0.000000in}}{%
\pgfpathmoveto{\pgfqpoint{0.000000in}{0.000000in}}%
\pgfpathlineto{\pgfqpoint{-0.069444in}{0.000000in}}%
\pgfusepath{stroke,fill}%
}%
\begin{pgfscope}%
\pgfsys@transformshift{3.785491in}{0.440955in}%
\pgfsys@useobject{currentmarker}{}%
\end{pgfscope}%
\end{pgfscope}%
\begin{pgfscope}%
\pgftext[x=0.262077in,y=0.440955in,right,]{\rmfamily\fontsize{8.000000}{9.600000}\selectfont 0}%
\end{pgfscope}%
\begin{pgfscope}%
\pgfsetbuttcap%
\pgfsetroundjoin%
\definecolor{currentfill}{rgb}{0.000000,0.000000,0.000000}%
\pgfsetfillcolor{currentfill}%
\pgfsetlinewidth{0.501875pt}%
\definecolor{currentstroke}{rgb}{0.000000,0.000000,0.000000}%
\pgfsetstrokecolor{currentstroke}%
\pgfsetdash{}{0pt}%
\pgfsys@defobject{currentmarker}{\pgfqpoint{0.000000in}{0.000000in}}{\pgfqpoint{0.069444in}{0.000000in}}{%
\pgfpathmoveto{\pgfqpoint{0.000000in}{0.000000in}}%
\pgfpathlineto{\pgfqpoint{0.069444in}{0.000000in}}%
\pgfusepath{stroke,fill}%
}%
\begin{pgfscope}%
\pgfsys@transformshift{0.331521in}{0.662782in}%
\pgfsys@useobject{currentmarker}{}%
\end{pgfscope}%
\end{pgfscope}%
\begin{pgfscope}%
\pgfsetbuttcap%
\pgfsetroundjoin%
\definecolor{currentfill}{rgb}{0.000000,0.000000,0.000000}%
\pgfsetfillcolor{currentfill}%
\pgfsetlinewidth{0.501875pt}%
\definecolor{currentstroke}{rgb}{0.000000,0.000000,0.000000}%
\pgfsetstrokecolor{currentstroke}%
\pgfsetdash{}{0pt}%
\pgfsys@defobject{currentmarker}{\pgfqpoint{-0.069444in}{0.000000in}}{\pgfqpoint{0.000000in}{0.000000in}}{%
\pgfpathmoveto{\pgfqpoint{0.000000in}{0.000000in}}%
\pgfpathlineto{\pgfqpoint{-0.069444in}{0.000000in}}%
\pgfusepath{stroke,fill}%
}%
\begin{pgfscope}%
\pgfsys@transformshift{3.785491in}{0.662782in}%
\pgfsys@useobject{currentmarker}{}%
\end{pgfscope}%
\end{pgfscope}%
\begin{pgfscope}%
\pgftext[x=0.262077in,y=0.662782in,right,]{\rmfamily\fontsize{8.000000}{9.600000}\selectfont 50}%
\end{pgfscope}%
\begin{pgfscope}%
\pgfsetbuttcap%
\pgfsetroundjoin%
\definecolor{currentfill}{rgb}{0.000000,0.000000,0.000000}%
\pgfsetfillcolor{currentfill}%
\pgfsetlinewidth{0.501875pt}%
\definecolor{currentstroke}{rgb}{0.000000,0.000000,0.000000}%
\pgfsetstrokecolor{currentstroke}%
\pgfsetdash{}{0pt}%
\pgfsys@defobject{currentmarker}{\pgfqpoint{0.000000in}{0.000000in}}{\pgfqpoint{0.069444in}{0.000000in}}{%
\pgfpathmoveto{\pgfqpoint{0.000000in}{0.000000in}}%
\pgfpathlineto{\pgfqpoint{0.069444in}{0.000000in}}%
\pgfusepath{stroke,fill}%
}%
\begin{pgfscope}%
\pgfsys@transformshift{0.331521in}{0.884609in}%
\pgfsys@useobject{currentmarker}{}%
\end{pgfscope}%
\end{pgfscope}%
\begin{pgfscope}%
\pgfsetbuttcap%
\pgfsetroundjoin%
\definecolor{currentfill}{rgb}{0.000000,0.000000,0.000000}%
\pgfsetfillcolor{currentfill}%
\pgfsetlinewidth{0.501875pt}%
\definecolor{currentstroke}{rgb}{0.000000,0.000000,0.000000}%
\pgfsetstrokecolor{currentstroke}%
\pgfsetdash{}{0pt}%
\pgfsys@defobject{currentmarker}{\pgfqpoint{-0.069444in}{0.000000in}}{\pgfqpoint{0.000000in}{0.000000in}}{%
\pgfpathmoveto{\pgfqpoint{0.000000in}{0.000000in}}%
\pgfpathlineto{\pgfqpoint{-0.069444in}{0.000000in}}%
\pgfusepath{stroke,fill}%
}%
\begin{pgfscope}%
\pgfsys@transformshift{3.785491in}{0.884609in}%
\pgfsys@useobject{currentmarker}{}%
\end{pgfscope}%
\end{pgfscope}%
\begin{pgfscope}%
\pgftext[x=0.262077in,y=0.884609in,right,]{\rmfamily\fontsize{8.000000}{9.600000}\selectfont 100}%
\end{pgfscope}%
\begin{pgfscope}%
\pgfsetbuttcap%
\pgfsetroundjoin%
\definecolor{currentfill}{rgb}{0.000000,0.000000,0.000000}%
\pgfsetfillcolor{currentfill}%
\pgfsetlinewidth{0.501875pt}%
\definecolor{currentstroke}{rgb}{0.000000,0.000000,0.000000}%
\pgfsetstrokecolor{currentstroke}%
\pgfsetdash{}{0pt}%
\pgfsys@defobject{currentmarker}{\pgfqpoint{0.000000in}{0.000000in}}{\pgfqpoint{0.069444in}{0.000000in}}{%
\pgfpathmoveto{\pgfqpoint{0.000000in}{0.000000in}}%
\pgfpathlineto{\pgfqpoint{0.069444in}{0.000000in}}%
\pgfusepath{stroke,fill}%
}%
\begin{pgfscope}%
\pgfsys@transformshift{0.331521in}{1.106436in}%
\pgfsys@useobject{currentmarker}{}%
\end{pgfscope}%
\end{pgfscope}%
\begin{pgfscope}%
\pgfsetbuttcap%
\pgfsetroundjoin%
\definecolor{currentfill}{rgb}{0.000000,0.000000,0.000000}%
\pgfsetfillcolor{currentfill}%
\pgfsetlinewidth{0.501875pt}%
\definecolor{currentstroke}{rgb}{0.000000,0.000000,0.000000}%
\pgfsetstrokecolor{currentstroke}%
\pgfsetdash{}{0pt}%
\pgfsys@defobject{currentmarker}{\pgfqpoint{-0.069444in}{0.000000in}}{\pgfqpoint{0.000000in}{0.000000in}}{%
\pgfpathmoveto{\pgfqpoint{0.000000in}{0.000000in}}%
\pgfpathlineto{\pgfqpoint{-0.069444in}{0.000000in}}%
\pgfusepath{stroke,fill}%
}%
\begin{pgfscope}%
\pgfsys@transformshift{3.785491in}{1.106436in}%
\pgfsys@useobject{currentmarker}{}%
\end{pgfscope}%
\end{pgfscope}%
\begin{pgfscope}%
\pgftext[x=0.262077in,y=1.106436in,right,]{\rmfamily\fontsize{8.000000}{9.600000}\selectfont 150}%
\end{pgfscope}%
\begin{pgfscope}%
\pgfsetbuttcap%
\pgfsetroundjoin%
\definecolor{currentfill}{rgb}{0.000000,0.000000,0.000000}%
\pgfsetfillcolor{currentfill}%
\pgfsetlinewidth{0.501875pt}%
\definecolor{currentstroke}{rgb}{0.000000,0.000000,0.000000}%
\pgfsetstrokecolor{currentstroke}%
\pgfsetdash{}{0pt}%
\pgfsys@defobject{currentmarker}{\pgfqpoint{0.000000in}{0.000000in}}{\pgfqpoint{0.069444in}{0.000000in}}{%
\pgfpathmoveto{\pgfqpoint{0.000000in}{0.000000in}}%
\pgfpathlineto{\pgfqpoint{0.069444in}{0.000000in}}%
\pgfusepath{stroke,fill}%
}%
\begin{pgfscope}%
\pgfsys@transformshift{0.331521in}{1.328263in}%
\pgfsys@useobject{currentmarker}{}%
\end{pgfscope}%
\end{pgfscope}%
\begin{pgfscope}%
\pgfsetbuttcap%
\pgfsetroundjoin%
\definecolor{currentfill}{rgb}{0.000000,0.000000,0.000000}%
\pgfsetfillcolor{currentfill}%
\pgfsetlinewidth{0.501875pt}%
\definecolor{currentstroke}{rgb}{0.000000,0.000000,0.000000}%
\pgfsetstrokecolor{currentstroke}%
\pgfsetdash{}{0pt}%
\pgfsys@defobject{currentmarker}{\pgfqpoint{-0.069444in}{0.000000in}}{\pgfqpoint{0.000000in}{0.000000in}}{%
\pgfpathmoveto{\pgfqpoint{0.000000in}{0.000000in}}%
\pgfpathlineto{\pgfqpoint{-0.069444in}{0.000000in}}%
\pgfusepath{stroke,fill}%
}%
\begin{pgfscope}%
\pgfsys@transformshift{3.785491in}{1.328263in}%
\pgfsys@useobject{currentmarker}{}%
\end{pgfscope}%
\end{pgfscope}%
\begin{pgfscope}%
\pgftext[x=0.262077in,y=1.328263in,right,]{\rmfamily\fontsize{8.000000}{9.600000}\selectfont 200}%
\end{pgfscope}%
\begin{pgfscope}%
\pgfsetbuttcap%
\pgfsetroundjoin%
\definecolor{currentfill}{rgb}{0.000000,0.000000,0.000000}%
\pgfsetfillcolor{currentfill}%
\pgfsetlinewidth{0.501875pt}%
\definecolor{currentstroke}{rgb}{0.000000,0.000000,0.000000}%
\pgfsetstrokecolor{currentstroke}%
\pgfsetdash{}{0pt}%
\pgfsys@defobject{currentmarker}{\pgfqpoint{0.000000in}{0.000000in}}{\pgfqpoint{0.069444in}{0.000000in}}{%
\pgfpathmoveto{\pgfqpoint{0.000000in}{0.000000in}}%
\pgfpathlineto{\pgfqpoint{0.069444in}{0.000000in}}%
\pgfusepath{stroke,fill}%
}%
\begin{pgfscope}%
\pgfsys@transformshift{0.331521in}{1.550089in}%
\pgfsys@useobject{currentmarker}{}%
\end{pgfscope}%
\end{pgfscope}%
\begin{pgfscope}%
\pgfsetbuttcap%
\pgfsetroundjoin%
\definecolor{currentfill}{rgb}{0.000000,0.000000,0.000000}%
\pgfsetfillcolor{currentfill}%
\pgfsetlinewidth{0.501875pt}%
\definecolor{currentstroke}{rgb}{0.000000,0.000000,0.000000}%
\pgfsetstrokecolor{currentstroke}%
\pgfsetdash{}{0pt}%
\pgfsys@defobject{currentmarker}{\pgfqpoint{-0.069444in}{0.000000in}}{\pgfqpoint{0.000000in}{0.000000in}}{%
\pgfpathmoveto{\pgfqpoint{0.000000in}{0.000000in}}%
\pgfpathlineto{\pgfqpoint{-0.069444in}{0.000000in}}%
\pgfusepath{stroke,fill}%
}%
\begin{pgfscope}%
\pgfsys@transformshift{3.785491in}{1.550089in}%
\pgfsys@useobject{currentmarker}{}%
\end{pgfscope}%
\end{pgfscope}%
\begin{pgfscope}%
\pgftext[x=0.262077in,y=1.550089in,right,]{\rmfamily\fontsize{8.000000}{9.600000}\selectfont 250}%
\end{pgfscope}%
\begin{pgfscope}%
\pgfsetbuttcap%
\pgfsetroundjoin%
\definecolor{currentfill}{rgb}{0.000000,0.000000,0.000000}%
\pgfsetfillcolor{currentfill}%
\pgfsetlinewidth{0.501875pt}%
\definecolor{currentstroke}{rgb}{0.000000,0.000000,0.000000}%
\pgfsetstrokecolor{currentstroke}%
\pgfsetdash{}{0pt}%
\pgfsys@defobject{currentmarker}{\pgfqpoint{0.000000in}{0.000000in}}{\pgfqpoint{0.069444in}{0.000000in}}{%
\pgfpathmoveto{\pgfqpoint{0.000000in}{0.000000in}}%
\pgfpathlineto{\pgfqpoint{0.069444in}{0.000000in}}%
\pgfusepath{stroke,fill}%
}%
\begin{pgfscope}%
\pgfsys@transformshift{0.331521in}{1.771916in}%
\pgfsys@useobject{currentmarker}{}%
\end{pgfscope}%
\end{pgfscope}%
\begin{pgfscope}%
\pgfsetbuttcap%
\pgfsetroundjoin%
\definecolor{currentfill}{rgb}{0.000000,0.000000,0.000000}%
\pgfsetfillcolor{currentfill}%
\pgfsetlinewidth{0.501875pt}%
\definecolor{currentstroke}{rgb}{0.000000,0.000000,0.000000}%
\pgfsetstrokecolor{currentstroke}%
\pgfsetdash{}{0pt}%
\pgfsys@defobject{currentmarker}{\pgfqpoint{-0.069444in}{0.000000in}}{\pgfqpoint{0.000000in}{0.000000in}}{%
\pgfpathmoveto{\pgfqpoint{0.000000in}{0.000000in}}%
\pgfpathlineto{\pgfqpoint{-0.069444in}{0.000000in}}%
\pgfusepath{stroke,fill}%
}%
\begin{pgfscope}%
\pgfsys@transformshift{3.785491in}{1.771916in}%
\pgfsys@useobject{currentmarker}{}%
\end{pgfscope}%
\end{pgfscope}%
\begin{pgfscope}%
\pgftext[x=0.262077in,y=1.771916in,right,]{\rmfamily\fontsize{8.000000}{9.600000}\selectfont 300}%
\end{pgfscope}%
\begin{pgfscope}%
\pgfsetbuttcap%
\pgfsetroundjoin%
\definecolor{currentfill}{rgb}{0.000000,0.000000,0.000000}%
\pgfsetfillcolor{currentfill}%
\pgfsetlinewidth{0.501875pt}%
\definecolor{currentstroke}{rgb}{0.000000,0.000000,0.000000}%
\pgfsetstrokecolor{currentstroke}%
\pgfsetdash{}{0pt}%
\pgfsys@defobject{currentmarker}{\pgfqpoint{0.000000in}{0.000000in}}{\pgfqpoint{0.069444in}{0.000000in}}{%
\pgfpathmoveto{\pgfqpoint{0.000000in}{0.000000in}}%
\pgfpathlineto{\pgfqpoint{0.069444in}{0.000000in}}%
\pgfusepath{stroke,fill}%
}%
\begin{pgfscope}%
\pgfsys@transformshift{0.331521in}{1.993743in}%
\pgfsys@useobject{currentmarker}{}%
\end{pgfscope}%
\end{pgfscope}%
\begin{pgfscope}%
\pgfsetbuttcap%
\pgfsetroundjoin%
\definecolor{currentfill}{rgb}{0.000000,0.000000,0.000000}%
\pgfsetfillcolor{currentfill}%
\pgfsetlinewidth{0.501875pt}%
\definecolor{currentstroke}{rgb}{0.000000,0.000000,0.000000}%
\pgfsetstrokecolor{currentstroke}%
\pgfsetdash{}{0pt}%
\pgfsys@defobject{currentmarker}{\pgfqpoint{-0.069444in}{0.000000in}}{\pgfqpoint{0.000000in}{0.000000in}}{%
\pgfpathmoveto{\pgfqpoint{0.000000in}{0.000000in}}%
\pgfpathlineto{\pgfqpoint{-0.069444in}{0.000000in}}%
\pgfusepath{stroke,fill}%
}%
\begin{pgfscope}%
\pgfsys@transformshift{3.785491in}{1.993743in}%
\pgfsys@useobject{currentmarker}{}%
\end{pgfscope}%
\end{pgfscope}%
\begin{pgfscope}%
\pgftext[x=0.262077in,y=1.993743in,right,]{\rmfamily\fontsize{8.000000}{9.600000}\selectfont 350}%
\end{pgfscope}%
\begin{pgfscope}%
\pgfsetbuttcap%
\pgfsetroundjoin%
\definecolor{currentfill}{rgb}{0.000000,0.000000,0.000000}%
\pgfsetfillcolor{currentfill}%
\pgfsetlinewidth{0.501875pt}%
\definecolor{currentstroke}{rgb}{0.000000,0.000000,0.000000}%
\pgfsetstrokecolor{currentstroke}%
\pgfsetdash{}{0pt}%
\pgfsys@defobject{currentmarker}{\pgfqpoint{0.000000in}{0.000000in}}{\pgfqpoint{0.069444in}{0.000000in}}{%
\pgfpathmoveto{\pgfqpoint{0.000000in}{0.000000in}}%
\pgfpathlineto{\pgfqpoint{0.069444in}{0.000000in}}%
\pgfusepath{stroke,fill}%
}%
\begin{pgfscope}%
\pgfsys@transformshift{0.331521in}{2.215570in}%
\pgfsys@useobject{currentmarker}{}%
\end{pgfscope}%
\end{pgfscope}%
\begin{pgfscope}%
\pgfsetbuttcap%
\pgfsetroundjoin%
\definecolor{currentfill}{rgb}{0.000000,0.000000,0.000000}%
\pgfsetfillcolor{currentfill}%
\pgfsetlinewidth{0.501875pt}%
\definecolor{currentstroke}{rgb}{0.000000,0.000000,0.000000}%
\pgfsetstrokecolor{currentstroke}%
\pgfsetdash{}{0pt}%
\pgfsys@defobject{currentmarker}{\pgfqpoint{-0.069444in}{0.000000in}}{\pgfqpoint{0.000000in}{0.000000in}}{%
\pgfpathmoveto{\pgfqpoint{0.000000in}{0.000000in}}%
\pgfpathlineto{\pgfqpoint{-0.069444in}{0.000000in}}%
\pgfusepath{stroke,fill}%
}%
\begin{pgfscope}%
\pgfsys@transformshift{3.785491in}{2.215570in}%
\pgfsys@useobject{currentmarker}{}%
\end{pgfscope}%
\end{pgfscope}%
\begin{pgfscope}%
\pgftext[x=0.262077in,y=2.215570in,right,]{\rmfamily\fontsize{8.000000}{9.600000}\selectfont 400}%
\end{pgfscope}%
\end{pgfpicture}%
\makeatother%
\endgroup%

  \caption{
    Invariant mass $m(\PKplus\Ppiminus\mu^-_\pi)$ constructed with a \Ppiminus mass hypothesis for the \Pmuon.
    The decay $B^0\to D^{*-} \mu^+ \APneutrino$ is reponsible for the peak around \SI{2010}{GeV}.
    The region marked in red is removed as part of the preselection.
  }
  \label{fig:dstar}
\end{figure}

\subsubsection{Vetoes used for the normalization channel}

The data sample used for the normalization channel also contains various physical backgrounds that need to be excluded before a reliable estimate of the normalization channel yield can be performed.

As $B^0\to\PJpsi\PKstar$ has previously been analyzed at LHCb, background rejection cuts have been adapted from a previous analysis.
The selection procedure of $\PBzero\to\PKstar\APmuon\Pmuon$ using the full LHCb Run I dataset \cite{Citation needed} has been reproduced for this purpose with a few modifications:
Where the $B^0\to K^{*0}\APmuon\Pmuon$ analysis used combined mass and PID cuts, only the corresponding mass cuts are used for this analysis.
This is done to avoid introducing systematic errors resulting from discrepancies between real data and simulation.
While the backgrounds are still fully rejected using this method, the achieved normalization channel efficiency is lower.
By using a data-driven approach to measure the normalization efficiencies or by using one of the methods for eliminating data-simulation differences discussed in section \ref{mva}, combined mass and PID cuts could be introduced to significantly increase the normalization channel efficiency.

Cuts on $m(\APmuon\Pmuon)$ (to select \PJpsi candidates) and $m(\PKplus\Ppiminus)$ (to select \PKstar candidates) are applied.
Decays of $\PBplus$ to $\PJpsi\PKplus$, where an additional \Ppiminus has been added to the candidate, are excluded by constructing $m(\PKplus\APmuon\Pmuon)$ and rejecting a window around the nominal $B^0$ mass.
The background $\PBzero\to\PJpsi\Pphi(\PKplus\PKminus)$, where \PKminus has been identified as \Ppiminus is excluded by constructing $m(\PKplus\pi^-_K)$ and rejecting a window around the nominal \Pphi mass.
Finally, $\PBzero\to\PJpsi\PKstar$ decays where a \PKminus has been identified as \Ppiminus and \Ppiplus as \PKplus are rejected by constructing $m(K^+_\pi \pi^-_K)$ and rejecting a window around the \PKstar mass.
This cut in particular has a low efficieny for the normalization channel without a complementary PID cut.

All selection cuts applied to the dataset are summarized in table \ref{tab:normcuts}.

\begin{table}
  \centering
  \caption{
    Summary of all preselection cuts applied to the normalization ($\PBzero\to\PJpsi\PKstar$) dataset.
    Each efficiency is calculated based on the output of the previous selection cut.
  }
  \begin{tabular}{l l S[table-format=2.4,table-figures-uncertainty=1]}
    \toprule
    Background & Veto & {Efficiency $/\ \si{\percent}$} \\
    \midrule
    $-$ & $\mathrm{DLL}_{K/\pi} < -5$ & 99.49 \pm 0.009 \\
    $-$ & $m(\PKplus\Ppiminus)\ \text{outside}\ (\SI{800}{MeV},\SI{1000}{MeV})$ & 87.54 \pm 0.04 \\
    $-$ & $m(\APmuon\Pmuon)\ \text{outside}\ (\SI{2970}{MeV},\SI{3170}{MeV})$ & 99.9928 \pm 0.0011 \\
    $\PBzero\to\PJpsi\PKplus$ & $\SI{5220}{MeV} < m(\PKplus\APmuon\Pmuon) < \SI{5340}{MeV}$ & 99.993 \pm 0.001 \\
    $\PBzero\to\PJpsi\Pphi$ & $\SI{1005}{MeV} < m(\PKplus\pi^-_K) < \SI{1035}{MeV}$ & 99.335 \pm 0.011 \\
    \PKplus\Ppiminus swap & $\SI{795}{MeV} < m(K^+_\pi \pi^-_K) < \SI{995}{MeV}$ & 67.27 \pm 0.06 \\
    \midrule
    Total & & 58.13 \pm 0.06 \\
    \bottomrule
  \end{tabular}
  \label{tab:normcuts}
\end{table}

\section{Multivariate classification}
\label{mva}

As the analysis requires an identification of an excess of only a few ($\approx 10^1$) events in a dataset of roughly $10^6$ (mostly combinatorial) background events, an efficient way to improve the ratio between combinatorial background and possible signal events is needed.
For this purpose, a multivariate classification algorithm is employed.

% TODO queue explanation of MVA
% TODO why is it useful?
% TODO which classifier has been used?

In order to train the classifier, pure signal and background samples are needed.
Because no pure signal data sample is available, the simulated signal dataset is used as a proxy.
To reduce the systematic uncertainty resulting from different behaviour of simulated and real events, the procedure described in section \ref{resampling} is applied.

The upper mass sideband of the $\PBzero\to\APDzero\APmuon\Pmuon$ dataset (up to \SI{5600}{MeV}) is used as a background proxy.
The upper mass sideband is used, because training on the lower sideband could partially remove (and thus obscure) the presence of partially reconstructed backgrounds, which would lead to an error in the later parametrization of the combinatorial background.

Ten variables with high separation power (see table \ref{tab:mvavariables}) have been selected from the dataset to serve as input to the classifier.
Figure \ref{fig:features} compares the signal and background distributions of each variable.

% TODO how many signal/background events were used?

\begin{table}
  \centering
  \caption{Variables used for the multivariate classification}
  \begin{tabular}{l}
    \toprule
    Variable \\
    \midrule
    $\mathrm{cos}(\text{DIRA angle})$ \\
    $B^0\ \text{vertex}\ \chi^2/\text{ndf}$ \\
    $B^0\ \text{muon isolation BDT response}$ \\
    $t_{B^0}$ \\
    $K^+ \mathrm{DLL}_{K\pi}$ \\
    $K^+ \mathrm{DLL}_{\mu\pi}$ \\
    $\pi^- \mathrm{DLL}_{K\pi}$ \\
    $\pi^- \mathrm{DLL}_{\mu\pi}$ \\
    $\mu^+ \mathrm{DLL}_{\mu\pi}$ \\
    $\mu^- \mathrm{DLL}_{\mu\pi}$ \\
    \bottomrule
  \end{tabular}
  \label{tab:mvavariables}
  % TODO explanations
\end{table}

In order to estimate the efficiency of the classification procedure and to optimize its hyperparameters, the training dataset itself is classified.
A possible danger when doing this naively by using the entire signal and background samples for both training and evaluation is that a classifier's performance in general differs between its training inputs and previously unseen data.
In order to avoid this problem, a $k$-fold cross-validation procedure\cite{Elements} is used.
This consists of using a $\frac{k - 1}{k}$ fraction of the data to train the classifier and using the remaining $\frac{1}{k}$ of the data to validate it.
This is done iteratively $k$ times until each fraction of the dataset has been used once for validation.
$k = 5$ has been chosen for this analysis.

The hyperparameters of the classification algorithm are optimized using the area under the ROC curve as a guideline.
The resulting parameters, which are used for the rest of the analysis, are given in table \ref{tab:mvaparams}.

\begin{table}
  \centering
  \caption{Parameters used for the XGBoost gradient boosting classifier}
  \begin{tabular}{l S}
    \toprule
    Parameter & {Value} \\
    \midrule
    $N_\text{trees}$ & 150 \\
    $\gamma$ & 12 \\
    max. tree depth & 10 \\
    $\eta$ & 0.3 \\
    \bottomrule
  \end{tabular}
  \label{tab:mvaparams}
\end{table}

The classifier returns a posterior probability $p$ that a given candidate belongs to the signal class.
The distribution of probabilities (ranging from $0$ to $1$) can be mapped to a potential range of $-\infty$ to $+\infty$ using the logit function
\begin{equation}
  \mathrm{logit}(p) = \mathrm{log}\left(\frac{p}{1 - p}\right)\:.
\end{equation}
The calculated value is referred to as the \emph{classifier response} in the following analysis steps.
The classifier response for the signal and background training samples is given in figure \ref{fig:response}.

The classifier is used to calculate a response for the entire (blinded) $B^0\to\APDzero\APmuon\Pmuon$ data sample.
Care is taken to conduct the classification of the upper mass sideband (used for training) in a $k$-fold manner.
When classifying the rest of the dataset, one of the $k$ classifiers is chosen at random.

In order to use the calculated classifier response to reject background candidates, a cut is applied to the classifier response distribution of the $B^0\to\APDzero\APmuon\Pmuon$ dataset.

\begin{figure}
  \centering
  %% Creator: Matplotlib, PGF backend
%%
%% To include the figure in your LaTeX document, write
%%   \input{<filename>.pgf}
%%
%% Make sure the required packages are loaded in your preamble
%%   \usepackage{pgf}
%%
%% Figures using additional raster images can only be included by \input if
%% they are in the same directory as the main LaTeX file. For loading figures
%% from other directories you can use the `import` package
%%   \usepackage{import}
%% and then include the figures with
%%   \import{<path to file>}{<filename>.pgf}
%%
%% Matplotlib used the following preamble
%%   \usepackage{fontspec}
%%   \setmainfont{DejaVu Serif}
%%   \setsansfont{DejaVu Sans}
%%   \setmonofont{DejaVu Sans Mono}
%%
\begingroup%
\makeatletter%
\begin{pgfpicture}%
\pgfpathrectangle{\pgfpointorigin}{\pgfqpoint{3.979659in}{2.696729in}}%
\pgfusepath{use as bounding box, clip}%
\begin{pgfscope}%
\pgfsetbuttcap%
\pgfsetmiterjoin%
\definecolor{currentfill}{rgb}{1.000000,1.000000,1.000000}%
\pgfsetfillcolor{currentfill}%
\pgfsetlinewidth{0.000000pt}%
\definecolor{currentstroke}{rgb}{1.000000,1.000000,1.000000}%
\pgfsetstrokecolor{currentstroke}%
\pgfsetdash{}{0pt}%
\pgfpathmoveto{\pgfqpoint{0.000000in}{0.000000in}}%
\pgfpathlineto{\pgfqpoint{3.979659in}{0.000000in}}%
\pgfpathlineto{\pgfqpoint{3.979659in}{2.696729in}}%
\pgfpathlineto{\pgfqpoint{0.000000in}{2.696729in}}%
\pgfpathclose%
\pgfusepath{fill}%
\end{pgfscope}%
\begin{pgfscope}%
\pgfsetbuttcap%
\pgfsetmiterjoin%
\definecolor{currentfill}{rgb}{1.000000,1.000000,1.000000}%
\pgfsetfillcolor{currentfill}%
\pgfsetlinewidth{0.000000pt}%
\definecolor{currentstroke}{rgb}{0.000000,0.000000,0.000000}%
\pgfsetstrokecolor{currentstroke}%
\pgfsetstrokeopacity{0.000000}%
\pgfsetdash{}{0pt}%
\pgfpathmoveto{\pgfqpoint{0.366840in}{0.417391in}}%
\pgfpathlineto{\pgfqpoint{3.894313in}{0.417391in}}%
\pgfpathlineto{\pgfqpoint{3.894313in}{2.592964in}}%
\pgfpathlineto{\pgfqpoint{0.366840in}{2.592964in}}%
\pgfpathclose%
\pgfusepath{fill}%
\end{pgfscope}%
\begin{pgfscope}%
\pgfpathrectangle{\pgfqpoint{0.366840in}{0.417391in}}{\pgfqpoint{3.527473in}{2.175573in}} %
\pgfusepath{clip}%
\pgfsetbuttcap%
\pgfsetmiterjoin%
\definecolor{currentfill}{rgb}{0.215686,0.470588,0.749020}%
\pgfsetfillcolor{currentfill}%
\pgfsetlinewidth{0.000000pt}%
\definecolor{currentstroke}{rgb}{0.000000,0.000000,0.000000}%
\pgfsetstrokecolor{currentstroke}%
\pgfsetdash{}{0pt}%
\pgfpathmoveto{\pgfqpoint{0.526127in}{0.417391in}}%
\pgfpathlineto{\pgfqpoint{0.526127in}{0.417554in}}%
\pgfpathlineto{\pgfqpoint{0.589073in}{0.417554in}}%
\pgfpathlineto{\pgfqpoint{0.589073in}{0.417391in}}%
\pgfpathlineto{\pgfqpoint{0.652019in}{0.417391in}}%
\pgfpathlineto{\pgfqpoint{0.652019in}{0.418449in}}%
\pgfpathlineto{\pgfqpoint{0.714965in}{0.418449in}}%
\pgfpathlineto{\pgfqpoint{0.714965in}{0.418062in}}%
\pgfpathlineto{\pgfqpoint{0.777912in}{0.418062in}}%
\pgfpathlineto{\pgfqpoint{0.777912in}{0.419203in}}%
\pgfpathlineto{\pgfqpoint{0.840858in}{0.419203in}}%
\pgfpathlineto{\pgfqpoint{0.840858in}{0.419972in}}%
\pgfpathlineto{\pgfqpoint{0.903804in}{0.419972in}}%
\pgfpathlineto{\pgfqpoint{0.903804in}{0.422091in}}%
\pgfpathlineto{\pgfqpoint{0.966750in}{0.422091in}}%
\pgfpathlineto{\pgfqpoint{0.966750in}{0.424966in}}%
\pgfpathlineto{\pgfqpoint{1.029696in}{0.424966in}}%
\pgfpathlineto{\pgfqpoint{1.029696in}{0.430275in}}%
\pgfpathlineto{\pgfqpoint{1.092642in}{0.430275in}}%
\pgfpathlineto{\pgfqpoint{1.092642in}{0.428979in}}%
\pgfpathlineto{\pgfqpoint{1.155588in}{0.428979in}}%
\pgfpathlineto{\pgfqpoint{1.155588in}{0.440523in}}%
\pgfpathlineto{\pgfqpoint{1.218534in}{0.440523in}}%
\pgfpathlineto{\pgfqpoint{1.218534in}{0.447717in}}%
\pgfpathlineto{\pgfqpoint{1.281480in}{0.447717in}}%
\pgfpathlineto{\pgfqpoint{1.281480in}{0.460682in}}%
\pgfpathlineto{\pgfqpoint{1.344426in}{0.460682in}}%
\pgfpathlineto{\pgfqpoint{1.344426in}{0.461292in}}%
\pgfpathlineto{\pgfqpoint{1.407373in}{0.461292in}}%
\pgfpathlineto{\pgfqpoint{1.407373in}{0.480147in}}%
\pgfpathlineto{\pgfqpoint{1.470319in}{0.480147in}}%
\pgfpathlineto{\pgfqpoint{1.470319in}{0.486709in}}%
\pgfpathlineto{\pgfqpoint{1.533265in}{0.486709in}}%
\pgfpathlineto{\pgfqpoint{1.533265in}{0.497098in}}%
\pgfpathlineto{\pgfqpoint{1.596211in}{0.497098in}}%
\pgfpathlineto{\pgfqpoint{1.596211in}{0.522318in}}%
\pgfpathlineto{\pgfqpoint{1.659157in}{0.522318in}}%
\pgfpathlineto{\pgfqpoint{1.659157in}{0.506232in}}%
\pgfpathlineto{\pgfqpoint{1.722103in}{0.506232in}}%
\pgfpathlineto{\pgfqpoint{1.722103in}{0.560076in}}%
\pgfpathlineto{\pgfqpoint{1.785049in}{0.560076in}}%
\pgfpathlineto{\pgfqpoint{1.785049in}{0.578633in}}%
\pgfpathlineto{\pgfqpoint{1.847995in}{0.578633in}}%
\pgfpathlineto{\pgfqpoint{1.847995in}{0.612709in}}%
\pgfpathlineto{\pgfqpoint{1.910941in}{0.612709in}}%
\pgfpathlineto{\pgfqpoint{1.910941in}{0.666883in}}%
\pgfpathlineto{\pgfqpoint{1.973887in}{0.666883in}}%
\pgfpathlineto{\pgfqpoint{1.973887in}{0.683530in}}%
\pgfpathlineto{\pgfqpoint{2.036834in}{0.683530in}}%
\pgfpathlineto{\pgfqpoint{2.036834in}{0.696476in}}%
\pgfpathlineto{\pgfqpoint{2.099780in}{0.696476in}}%
\pgfpathlineto{\pgfqpoint{2.099780in}{0.726784in}}%
\pgfpathlineto{\pgfqpoint{2.162726in}{0.726784in}}%
\pgfpathlineto{\pgfqpoint{2.162726in}{0.811975in}}%
\pgfpathlineto{\pgfqpoint{2.225672in}{0.811975in}}%
\pgfpathlineto{\pgfqpoint{2.225672in}{0.814424in}}%
\pgfpathlineto{\pgfqpoint{2.288618in}{0.814424in}}%
\pgfpathlineto{\pgfqpoint{2.288618in}{0.902178in}}%
\pgfpathlineto{\pgfqpoint{2.351564in}{0.902178in}}%
\pgfpathlineto{\pgfqpoint{2.351564in}{0.976390in}}%
\pgfpathlineto{\pgfqpoint{2.414510in}{0.976390in}}%
\pgfpathlineto{\pgfqpoint{2.414510in}{0.964811in}}%
\pgfpathlineto{\pgfqpoint{2.477456in}{0.964811in}}%
\pgfpathlineto{\pgfqpoint{2.477456in}{1.132585in}}%
\pgfpathlineto{\pgfqpoint{2.540402in}{1.132585in}}%
\pgfpathlineto{\pgfqpoint{2.540402in}{1.122462in}}%
\pgfpathlineto{\pgfqpoint{2.603348in}{1.122462in}}%
\pgfpathlineto{\pgfqpoint{2.603348in}{1.226233in}}%
\pgfpathlineto{\pgfqpoint{2.666295in}{1.226233in}}%
\pgfpathlineto{\pgfqpoint{2.666295in}{1.299598in}}%
\pgfpathlineto{\pgfqpoint{2.729241in}{1.299598in}}%
\pgfpathlineto{\pgfqpoint{2.729241in}{1.440721in}}%
\pgfpathlineto{\pgfqpoint{2.792187in}{1.440721in}}%
\pgfpathlineto{\pgfqpoint{2.792187in}{1.609939in}}%
\pgfpathlineto{\pgfqpoint{2.855133in}{1.609939in}}%
\pgfpathlineto{\pgfqpoint{2.855133in}{1.683839in}}%
\pgfpathlineto{\pgfqpoint{2.918079in}{1.683839in}}%
\pgfpathlineto{\pgfqpoint{2.918079in}{1.989370in}}%
\pgfpathlineto{\pgfqpoint{2.981025in}{1.989370in}}%
\pgfpathlineto{\pgfqpoint{2.981025in}{2.006824in}}%
\pgfpathlineto{\pgfqpoint{3.043971in}{2.006824in}}%
\pgfpathlineto{\pgfqpoint{3.043971in}{2.206956in}}%
\pgfpathlineto{\pgfqpoint{3.106917in}{2.206956in}}%
\pgfpathlineto{\pgfqpoint{3.106917in}{2.262014in}}%
\pgfpathlineto{\pgfqpoint{3.169863in}{2.262014in}}%
\pgfpathlineto{\pgfqpoint{3.169863in}{2.329454in}}%
\pgfpathlineto{\pgfqpoint{3.232810in}{2.329454in}}%
\pgfpathlineto{\pgfqpoint{3.232810in}{2.282498in}}%
\pgfpathlineto{\pgfqpoint{3.295756in}{2.282498in}}%
\pgfpathlineto{\pgfqpoint{3.295756in}{2.357161in}}%
\pgfpathlineto{\pgfqpoint{3.358702in}{2.357161in}}%
\pgfpathlineto{\pgfqpoint{3.358702in}{2.214878in}}%
\pgfpathlineto{\pgfqpoint{3.421648in}{2.214878in}}%
\pgfpathlineto{\pgfqpoint{3.421648in}{1.826442in}}%
\pgfpathlineto{\pgfqpoint{3.484594in}{1.826442in}}%
\pgfpathlineto{\pgfqpoint{3.484594in}{1.581701in}}%
\pgfpathlineto{\pgfqpoint{3.547540in}{1.581701in}}%
\pgfpathlineto{\pgfqpoint{3.547540in}{1.140431in}}%
\pgfpathlineto{\pgfqpoint{3.610486in}{1.140431in}}%
\pgfpathlineto{\pgfqpoint{3.610486in}{0.670055in}}%
\pgfpathlineto{\pgfqpoint{3.673432in}{0.670055in}}%
\pgfpathlineto{\pgfqpoint{3.673432in}{0.417391in}}%
\pgfpathlineto{\pgfqpoint{3.610486in}{0.417391in}}%
\pgfpathlineto{\pgfqpoint{3.610486in}{0.417391in}}%
\pgfpathlineto{\pgfqpoint{3.547540in}{0.417391in}}%
\pgfpathlineto{\pgfqpoint{3.547540in}{0.417391in}}%
\pgfpathlineto{\pgfqpoint{3.484594in}{0.417391in}}%
\pgfpathlineto{\pgfqpoint{3.484594in}{0.417391in}}%
\pgfpathlineto{\pgfqpoint{3.421648in}{0.417391in}}%
\pgfpathlineto{\pgfqpoint{3.421648in}{0.417391in}}%
\pgfpathlineto{\pgfqpoint{3.358702in}{0.417391in}}%
\pgfpathlineto{\pgfqpoint{3.358702in}{0.417391in}}%
\pgfpathlineto{\pgfqpoint{3.295756in}{0.417391in}}%
\pgfpathlineto{\pgfqpoint{3.295756in}{0.417391in}}%
\pgfpathlineto{\pgfqpoint{3.232810in}{0.417391in}}%
\pgfpathlineto{\pgfqpoint{3.232810in}{0.417391in}}%
\pgfpathlineto{\pgfqpoint{3.169863in}{0.417391in}}%
\pgfpathlineto{\pgfqpoint{3.169863in}{0.417391in}}%
\pgfpathlineto{\pgfqpoint{3.106917in}{0.417391in}}%
\pgfpathlineto{\pgfqpoint{3.106917in}{0.417391in}}%
\pgfpathlineto{\pgfqpoint{3.043971in}{0.417391in}}%
\pgfpathlineto{\pgfqpoint{3.043971in}{0.417391in}}%
\pgfpathlineto{\pgfqpoint{2.981025in}{0.417391in}}%
\pgfpathlineto{\pgfqpoint{2.981025in}{0.417391in}}%
\pgfpathlineto{\pgfqpoint{2.918079in}{0.417391in}}%
\pgfpathlineto{\pgfqpoint{2.918079in}{0.417391in}}%
\pgfpathlineto{\pgfqpoint{2.855133in}{0.417391in}}%
\pgfpathlineto{\pgfqpoint{2.855133in}{0.417391in}}%
\pgfpathlineto{\pgfqpoint{2.792187in}{0.417391in}}%
\pgfpathlineto{\pgfqpoint{2.792187in}{0.417391in}}%
\pgfpathlineto{\pgfqpoint{2.729241in}{0.417391in}}%
\pgfpathlineto{\pgfqpoint{2.729241in}{0.417391in}}%
\pgfpathlineto{\pgfqpoint{2.666295in}{0.417391in}}%
\pgfpathlineto{\pgfqpoint{2.666295in}{0.417391in}}%
\pgfpathlineto{\pgfqpoint{2.603348in}{0.417391in}}%
\pgfpathlineto{\pgfqpoint{2.603348in}{0.417391in}}%
\pgfpathlineto{\pgfqpoint{2.540402in}{0.417391in}}%
\pgfpathlineto{\pgfqpoint{2.540402in}{0.417391in}}%
\pgfpathlineto{\pgfqpoint{2.477456in}{0.417391in}}%
\pgfpathlineto{\pgfqpoint{2.477456in}{0.417391in}}%
\pgfpathlineto{\pgfqpoint{2.414510in}{0.417391in}}%
\pgfpathlineto{\pgfqpoint{2.414510in}{0.417391in}}%
\pgfpathlineto{\pgfqpoint{2.351564in}{0.417391in}}%
\pgfpathlineto{\pgfqpoint{2.351564in}{0.417391in}}%
\pgfpathlineto{\pgfqpoint{2.288618in}{0.417391in}}%
\pgfpathlineto{\pgfqpoint{2.288618in}{0.417391in}}%
\pgfpathlineto{\pgfqpoint{2.225672in}{0.417391in}}%
\pgfpathlineto{\pgfqpoint{2.225672in}{0.417391in}}%
\pgfpathlineto{\pgfqpoint{2.162726in}{0.417391in}}%
\pgfpathlineto{\pgfqpoint{2.162726in}{0.417391in}}%
\pgfpathlineto{\pgfqpoint{2.099780in}{0.417391in}}%
\pgfpathlineto{\pgfqpoint{2.099780in}{0.417391in}}%
\pgfpathlineto{\pgfqpoint{2.036834in}{0.417391in}}%
\pgfpathlineto{\pgfqpoint{2.036834in}{0.417391in}}%
\pgfpathlineto{\pgfqpoint{1.973887in}{0.417391in}}%
\pgfpathlineto{\pgfqpoint{1.973887in}{0.417391in}}%
\pgfpathlineto{\pgfqpoint{1.910941in}{0.417391in}}%
\pgfpathlineto{\pgfqpoint{1.910941in}{0.417391in}}%
\pgfpathlineto{\pgfqpoint{1.847995in}{0.417391in}}%
\pgfpathlineto{\pgfqpoint{1.847995in}{0.417391in}}%
\pgfpathlineto{\pgfqpoint{1.785049in}{0.417391in}}%
\pgfpathlineto{\pgfqpoint{1.785049in}{0.417391in}}%
\pgfpathlineto{\pgfqpoint{1.722103in}{0.417391in}}%
\pgfpathlineto{\pgfqpoint{1.722103in}{0.417391in}}%
\pgfpathlineto{\pgfqpoint{1.659157in}{0.417391in}}%
\pgfpathlineto{\pgfqpoint{1.659157in}{0.417391in}}%
\pgfpathlineto{\pgfqpoint{1.596211in}{0.417391in}}%
\pgfpathlineto{\pgfqpoint{1.596211in}{0.417391in}}%
\pgfpathlineto{\pgfqpoint{1.533265in}{0.417391in}}%
\pgfpathlineto{\pgfqpoint{1.533265in}{0.417391in}}%
\pgfpathlineto{\pgfqpoint{1.470319in}{0.417391in}}%
\pgfpathlineto{\pgfqpoint{1.470319in}{0.417391in}}%
\pgfpathlineto{\pgfqpoint{1.407373in}{0.417391in}}%
\pgfpathlineto{\pgfqpoint{1.407373in}{0.417391in}}%
\pgfpathlineto{\pgfqpoint{1.344426in}{0.417391in}}%
\pgfpathlineto{\pgfqpoint{1.344426in}{0.417391in}}%
\pgfpathlineto{\pgfqpoint{1.281480in}{0.417391in}}%
\pgfpathlineto{\pgfqpoint{1.281480in}{0.417391in}}%
\pgfpathlineto{\pgfqpoint{1.218534in}{0.417391in}}%
\pgfpathlineto{\pgfqpoint{1.218534in}{0.417391in}}%
\pgfpathlineto{\pgfqpoint{1.155588in}{0.417391in}}%
\pgfpathlineto{\pgfqpoint{1.155588in}{0.417391in}}%
\pgfpathlineto{\pgfqpoint{1.092642in}{0.417391in}}%
\pgfpathlineto{\pgfqpoint{1.092642in}{0.417391in}}%
\pgfpathlineto{\pgfqpoint{1.029696in}{0.417391in}}%
\pgfpathlineto{\pgfqpoint{1.029696in}{0.417391in}}%
\pgfpathlineto{\pgfqpoint{0.966750in}{0.417391in}}%
\pgfpathlineto{\pgfqpoint{0.966750in}{0.417391in}}%
\pgfpathlineto{\pgfqpoint{0.903804in}{0.417391in}}%
\pgfpathlineto{\pgfqpoint{0.903804in}{0.417391in}}%
\pgfpathlineto{\pgfqpoint{0.840858in}{0.417391in}}%
\pgfpathlineto{\pgfqpoint{0.840858in}{0.417391in}}%
\pgfpathlineto{\pgfqpoint{0.777912in}{0.417391in}}%
\pgfpathlineto{\pgfqpoint{0.777912in}{0.417391in}}%
\pgfpathlineto{\pgfqpoint{0.714965in}{0.417391in}}%
\pgfpathlineto{\pgfqpoint{0.714965in}{0.417391in}}%
\pgfpathlineto{\pgfqpoint{0.652019in}{0.417391in}}%
\pgfpathlineto{\pgfqpoint{0.652019in}{0.417391in}}%
\pgfpathlineto{\pgfqpoint{0.589073in}{0.417391in}}%
\pgfpathlineto{\pgfqpoint{0.589073in}{0.417391in}}%
\pgfpathlineto{\pgfqpoint{0.526127in}{0.417391in}}%
\pgfusepath{fill}%
\end{pgfscope}%
\begin{pgfscope}%
\pgfpathrectangle{\pgfqpoint{0.366840in}{0.417391in}}{\pgfqpoint{3.527473in}{2.175573in}} %
\pgfusepath{clip}%
\pgfsetbuttcap%
\pgfsetmiterjoin%
\pgfsetlinewidth{0.501875pt}%
\definecolor{currentstroke}{rgb}{1.000000,0.000000,0.000000}%
\pgfsetstrokecolor{currentstroke}%
\pgfsetdash{}{0pt}%
\pgfpathmoveto{\pgfqpoint{0.526127in}{0.417391in}}%
\pgfpathlineto{\pgfqpoint{0.526127in}{1.273037in}}%
\pgfpathlineto{\pgfqpoint{0.589073in}{1.273037in}}%
\pgfpathlineto{\pgfqpoint{0.589073in}{1.569132in}}%
\pgfpathlineto{\pgfqpoint{0.652019in}{1.569132in}}%
\pgfpathlineto{\pgfqpoint{0.652019in}{1.899980in}}%
\pgfpathlineto{\pgfqpoint{0.714965in}{1.899980in}}%
\pgfpathlineto{\pgfqpoint{0.714965in}{2.093578in}}%
\pgfpathlineto{\pgfqpoint{0.777912in}{2.093578in}}%
\pgfpathlineto{\pgfqpoint{0.777912in}{2.239551in}}%
\pgfpathlineto{\pgfqpoint{0.840858in}{2.239551in}}%
\pgfpathlineto{\pgfqpoint{0.840858in}{2.225552in}}%
\pgfpathlineto{\pgfqpoint{0.903804in}{2.225552in}}%
\pgfpathlineto{\pgfqpoint{0.903804in}{2.197834in}}%
\pgfpathlineto{\pgfqpoint{0.966750in}{2.197834in}}%
\pgfpathlineto{\pgfqpoint{0.966750in}{2.115668in}}%
\pgfpathlineto{\pgfqpoint{1.029696in}{2.115668in}}%
\pgfpathlineto{\pgfqpoint{1.029696in}{2.009090in}}%
\pgfpathlineto{\pgfqpoint{1.092642in}{2.009090in}}%
\pgfpathlineto{\pgfqpoint{1.092642in}{1.865861in}}%
\pgfpathlineto{\pgfqpoint{1.155588in}{1.865861in}}%
\pgfpathlineto{\pgfqpoint{1.155588in}{1.721507in}}%
\pgfpathlineto{\pgfqpoint{1.218534in}{1.721507in}}%
\pgfpathlineto{\pgfqpoint{1.218534in}{1.595513in}}%
\pgfpathlineto{\pgfqpoint{1.281480in}{1.595513in}}%
\pgfpathlineto{\pgfqpoint{1.281480in}{1.492523in}}%
\pgfpathlineto{\pgfqpoint{1.344426in}{1.492523in}}%
\pgfpathlineto{\pgfqpoint{1.344426in}{1.388900in}}%
\pgfpathlineto{\pgfqpoint{1.407373in}{1.388900in}}%
\pgfpathlineto{\pgfqpoint{1.407373in}{1.306804in}}%
\pgfpathlineto{\pgfqpoint{1.470319in}{1.306804in}}%
\pgfpathlineto{\pgfqpoint{1.470319in}{1.222386in}}%
\pgfpathlineto{\pgfqpoint{1.533265in}{1.222386in}}%
\pgfpathlineto{\pgfqpoint{1.533265in}{1.136350in}}%
\pgfpathlineto{\pgfqpoint{1.596211in}{1.136350in}}%
\pgfpathlineto{\pgfqpoint{1.596211in}{1.064454in}}%
\pgfpathlineto{\pgfqpoint{1.659157in}{1.064454in}}%
\pgfpathlineto{\pgfqpoint{1.659157in}{1.008035in}}%
\pgfpathlineto{\pgfqpoint{1.722103in}{1.008035in}}%
\pgfpathlineto{\pgfqpoint{1.722103in}{0.955414in}}%
\pgfpathlineto{\pgfqpoint{1.785049in}{0.955414in}}%
\pgfpathlineto{\pgfqpoint{1.785049in}{0.913135in}}%
\pgfpathlineto{\pgfqpoint{1.847995in}{0.913135in}}%
\pgfpathlineto{\pgfqpoint{1.847995in}{0.866424in}}%
\pgfpathlineto{\pgfqpoint{1.910941in}{0.866424in}}%
\pgfpathlineto{\pgfqpoint{1.910941in}{0.830546in}}%
\pgfpathlineto{\pgfqpoint{1.973887in}{0.830546in}}%
\pgfpathlineto{\pgfqpoint{1.973887in}{0.800226in}}%
\pgfpathlineto{\pgfqpoint{2.036834in}{0.800226in}}%
\pgfpathlineto{\pgfqpoint{2.036834in}{0.766529in}}%
\pgfpathlineto{\pgfqpoint{2.099780in}{0.766529in}}%
\pgfpathlineto{\pgfqpoint{2.099780in}{0.741837in}}%
\pgfpathlineto{\pgfqpoint{2.162726in}{0.741837in}}%
\pgfpathlineto{\pgfqpoint{2.162726in}{0.702372in}}%
\pgfpathlineto{\pgfqpoint{2.225672in}{0.702372in}}%
\pgfpathlineto{\pgfqpoint{2.225672in}{0.682463in}}%
\pgfpathlineto{\pgfqpoint{2.288618in}{0.682463in}}%
\pgfpathlineto{\pgfqpoint{2.288618in}{0.657912in}}%
\pgfpathlineto{\pgfqpoint{2.351564in}{0.657912in}}%
\pgfpathlineto{\pgfqpoint{2.351564in}{0.640817in}}%
\pgfpathlineto{\pgfqpoint{2.414510in}{0.640817in}}%
\pgfpathlineto{\pgfqpoint{2.414510in}{0.616969in}}%
\pgfpathlineto{\pgfqpoint{2.477456in}{0.616969in}}%
\pgfpathlineto{\pgfqpoint{2.477456in}{0.603673in}}%
\pgfpathlineto{\pgfqpoint{2.540402in}{0.603673in}}%
\pgfpathlineto{\pgfqpoint{2.540402in}{0.582217in}}%
\pgfpathlineto{\pgfqpoint{2.603348in}{0.582217in}}%
\pgfpathlineto{\pgfqpoint{2.603348in}{0.575886in}}%
\pgfpathlineto{\pgfqpoint{2.666295in}{0.575886in}}%
\pgfpathlineto{\pgfqpoint{2.666295in}{0.559916in}}%
\pgfpathlineto{\pgfqpoint{2.729241in}{0.559916in}}%
\pgfpathlineto{\pgfqpoint{2.729241in}{0.546199in}}%
\pgfpathlineto{\pgfqpoint{2.792187in}{0.546199in}}%
\pgfpathlineto{\pgfqpoint{2.792187in}{0.530652in}}%
\pgfpathlineto{\pgfqpoint{2.855133in}{0.530652in}}%
\pgfpathlineto{\pgfqpoint{2.855133in}{0.519396in}}%
\pgfpathlineto{\pgfqpoint{2.918079in}{0.519396in}}%
\pgfpathlineto{\pgfqpoint{2.918079in}{0.502090in}}%
\pgfpathlineto{\pgfqpoint{2.981025in}{0.502090in}}%
\pgfpathlineto{\pgfqpoint{2.981025in}{0.491608in}}%
\pgfpathlineto{\pgfqpoint{3.043971in}{0.491608in}}%
\pgfpathlineto{\pgfqpoint{3.043971in}{0.479860in}}%
\pgfpathlineto{\pgfqpoint{3.106917in}{0.479860in}}%
\pgfpathlineto{\pgfqpoint{3.106917in}{0.467057in}}%
\pgfpathlineto{\pgfqpoint{3.169863in}{0.467057in}}%
\pgfpathlineto{\pgfqpoint{3.169863in}{0.451580in}}%
\pgfpathlineto{\pgfqpoint{3.232810in}{0.451580in}}%
\pgfpathlineto{\pgfqpoint{3.232810in}{0.441943in}}%
\pgfpathlineto{\pgfqpoint{3.295756in}{0.441943in}}%
\pgfpathlineto{\pgfqpoint{3.295756in}{0.435470in}}%
\pgfpathlineto{\pgfqpoint{3.358702in}{0.435470in}}%
\pgfpathlineto{\pgfqpoint{3.358702in}{0.428365in}}%
\pgfpathlineto{\pgfqpoint{3.421648in}{0.428365in}}%
\pgfpathlineto{\pgfqpoint{3.421648in}{0.424496in}}%
\pgfpathlineto{\pgfqpoint{3.484594in}{0.424496in}}%
\pgfpathlineto{\pgfqpoint{3.484594in}{0.420346in}}%
\pgfpathlineto{\pgfqpoint{3.547540in}{0.420346in}}%
\pgfpathlineto{\pgfqpoint{3.547540in}{0.419009in}}%
\pgfpathlineto{\pgfqpoint{3.610486in}{0.419009in}}%
\pgfpathlineto{\pgfqpoint{3.610486in}{0.417532in}}%
\pgfpathlineto{\pgfqpoint{3.673432in}{0.417532in}}%
\pgfpathlineto{\pgfqpoint{3.673432in}{0.417391in}}%
\pgfusepath{stroke}%
\end{pgfscope}%
\begin{pgfscope}%
\pgfsetrectcap%
\pgfsetmiterjoin%
\pgfsetlinewidth{1.003750pt}%
\definecolor{currentstroke}{rgb}{0.000000,0.000000,0.000000}%
\pgfsetstrokecolor{currentstroke}%
\pgfsetdash{}{0pt}%
\pgfpathmoveto{\pgfqpoint{0.366840in}{2.592964in}}%
\pgfpathlineto{\pgfqpoint{3.894313in}{2.592964in}}%
\pgfusepath{stroke}%
\end{pgfscope}%
\begin{pgfscope}%
\pgfsetrectcap%
\pgfsetmiterjoin%
\pgfsetlinewidth{1.003750pt}%
\definecolor{currentstroke}{rgb}{0.000000,0.000000,0.000000}%
\pgfsetstrokecolor{currentstroke}%
\pgfsetdash{}{0pt}%
\pgfpathmoveto{\pgfqpoint{3.894313in}{0.417391in}}%
\pgfpathlineto{\pgfqpoint{3.894313in}{2.592964in}}%
\pgfusepath{stroke}%
\end{pgfscope}%
\begin{pgfscope}%
\pgfsetrectcap%
\pgfsetmiterjoin%
\pgfsetlinewidth{1.003750pt}%
\definecolor{currentstroke}{rgb}{0.000000,0.000000,0.000000}%
\pgfsetstrokecolor{currentstroke}%
\pgfsetdash{}{0pt}%
\pgfpathmoveto{\pgfqpoint{0.366840in}{0.417391in}}%
\pgfpathlineto{\pgfqpoint{3.894313in}{0.417391in}}%
\pgfusepath{stroke}%
\end{pgfscope}%
\begin{pgfscope}%
\pgfsetrectcap%
\pgfsetmiterjoin%
\pgfsetlinewidth{1.003750pt}%
\definecolor{currentstroke}{rgb}{0.000000,0.000000,0.000000}%
\pgfsetstrokecolor{currentstroke}%
\pgfsetdash{}{0pt}%
\pgfpathmoveto{\pgfqpoint{0.366840in}{0.417391in}}%
\pgfpathlineto{\pgfqpoint{0.366840in}{2.592964in}}%
\pgfusepath{stroke}%
\end{pgfscope}%
\begin{pgfscope}%
\pgfsetbuttcap%
\pgfsetroundjoin%
\definecolor{currentfill}{rgb}{0.000000,0.000000,0.000000}%
\pgfsetfillcolor{currentfill}%
\pgfsetlinewidth{0.501875pt}%
\definecolor{currentstroke}{rgb}{0.000000,0.000000,0.000000}%
\pgfsetstrokecolor{currentstroke}%
\pgfsetdash{}{0pt}%
\pgfsys@defobject{currentmarker}{\pgfqpoint{0.000000in}{0.000000in}}{\pgfqpoint{0.000000in}{0.069444in}}{%
\pgfpathmoveto{\pgfqpoint{0.000000in}{0.000000in}}%
\pgfpathlineto{\pgfqpoint{0.000000in}{0.069444in}}%
\pgfusepath{stroke,fill}%
}%
\begin{pgfscope}%
\pgfsys@transformshift{0.366840in}{0.417391in}%
\pgfsys@useobject{currentmarker}{}%
\end{pgfscope}%
\end{pgfscope}%
\begin{pgfscope}%
\pgfsetbuttcap%
\pgfsetroundjoin%
\definecolor{currentfill}{rgb}{0.000000,0.000000,0.000000}%
\pgfsetfillcolor{currentfill}%
\pgfsetlinewidth{0.501875pt}%
\definecolor{currentstroke}{rgb}{0.000000,0.000000,0.000000}%
\pgfsetstrokecolor{currentstroke}%
\pgfsetdash{}{0pt}%
\pgfsys@defobject{currentmarker}{\pgfqpoint{0.000000in}{-0.069444in}}{\pgfqpoint{0.000000in}{0.000000in}}{%
\pgfpathmoveto{\pgfqpoint{0.000000in}{0.000000in}}%
\pgfpathlineto{\pgfqpoint{0.000000in}{-0.069444in}}%
\pgfusepath{stroke,fill}%
}%
\begin{pgfscope}%
\pgfsys@transformshift{0.366840in}{2.592964in}%
\pgfsys@useobject{currentmarker}{}%
\end{pgfscope}%
\end{pgfscope}%
\begin{pgfscope}%
\pgftext[x=0.366840in,y=0.347947in,,top]{\rmfamily\fontsize{8.000000}{9.600000}\selectfont −6}%
\end{pgfscope}%
\begin{pgfscope}%
\pgfsetbuttcap%
\pgfsetroundjoin%
\definecolor{currentfill}{rgb}{0.000000,0.000000,0.000000}%
\pgfsetfillcolor{currentfill}%
\pgfsetlinewidth{0.501875pt}%
\definecolor{currentstroke}{rgb}{0.000000,0.000000,0.000000}%
\pgfsetstrokecolor{currentstroke}%
\pgfsetdash{}{0pt}%
\pgfsys@defobject{currentmarker}{\pgfqpoint{0.000000in}{0.000000in}}{\pgfqpoint{0.000000in}{0.069444in}}{%
\pgfpathmoveto{\pgfqpoint{0.000000in}{0.000000in}}%
\pgfpathlineto{\pgfqpoint{0.000000in}{0.069444in}}%
\pgfusepath{stroke,fill}%
}%
\begin{pgfscope}%
\pgfsys@transformshift{0.954752in}{0.417391in}%
\pgfsys@useobject{currentmarker}{}%
\end{pgfscope}%
\end{pgfscope}%
\begin{pgfscope}%
\pgfsetbuttcap%
\pgfsetroundjoin%
\definecolor{currentfill}{rgb}{0.000000,0.000000,0.000000}%
\pgfsetfillcolor{currentfill}%
\pgfsetlinewidth{0.501875pt}%
\definecolor{currentstroke}{rgb}{0.000000,0.000000,0.000000}%
\pgfsetstrokecolor{currentstroke}%
\pgfsetdash{}{0pt}%
\pgfsys@defobject{currentmarker}{\pgfqpoint{0.000000in}{-0.069444in}}{\pgfqpoint{0.000000in}{0.000000in}}{%
\pgfpathmoveto{\pgfqpoint{0.000000in}{0.000000in}}%
\pgfpathlineto{\pgfqpoint{0.000000in}{-0.069444in}}%
\pgfusepath{stroke,fill}%
}%
\begin{pgfscope}%
\pgfsys@transformshift{0.954752in}{2.592964in}%
\pgfsys@useobject{currentmarker}{}%
\end{pgfscope}%
\end{pgfscope}%
\begin{pgfscope}%
\pgftext[x=0.954752in,y=0.347947in,,top]{\rmfamily\fontsize{8.000000}{9.600000}\selectfont −4}%
\end{pgfscope}%
\begin{pgfscope}%
\pgfsetbuttcap%
\pgfsetroundjoin%
\definecolor{currentfill}{rgb}{0.000000,0.000000,0.000000}%
\pgfsetfillcolor{currentfill}%
\pgfsetlinewidth{0.501875pt}%
\definecolor{currentstroke}{rgb}{0.000000,0.000000,0.000000}%
\pgfsetstrokecolor{currentstroke}%
\pgfsetdash{}{0pt}%
\pgfsys@defobject{currentmarker}{\pgfqpoint{0.000000in}{0.000000in}}{\pgfqpoint{0.000000in}{0.069444in}}{%
\pgfpathmoveto{\pgfqpoint{0.000000in}{0.000000in}}%
\pgfpathlineto{\pgfqpoint{0.000000in}{0.069444in}}%
\pgfusepath{stroke,fill}%
}%
\begin{pgfscope}%
\pgfsys@transformshift{1.542665in}{0.417391in}%
\pgfsys@useobject{currentmarker}{}%
\end{pgfscope}%
\end{pgfscope}%
\begin{pgfscope}%
\pgfsetbuttcap%
\pgfsetroundjoin%
\definecolor{currentfill}{rgb}{0.000000,0.000000,0.000000}%
\pgfsetfillcolor{currentfill}%
\pgfsetlinewidth{0.501875pt}%
\definecolor{currentstroke}{rgb}{0.000000,0.000000,0.000000}%
\pgfsetstrokecolor{currentstroke}%
\pgfsetdash{}{0pt}%
\pgfsys@defobject{currentmarker}{\pgfqpoint{0.000000in}{-0.069444in}}{\pgfqpoint{0.000000in}{0.000000in}}{%
\pgfpathmoveto{\pgfqpoint{0.000000in}{0.000000in}}%
\pgfpathlineto{\pgfqpoint{0.000000in}{-0.069444in}}%
\pgfusepath{stroke,fill}%
}%
\begin{pgfscope}%
\pgfsys@transformshift{1.542665in}{2.592964in}%
\pgfsys@useobject{currentmarker}{}%
\end{pgfscope}%
\end{pgfscope}%
\begin{pgfscope}%
\pgftext[x=1.542665in,y=0.347947in,,top]{\rmfamily\fontsize{8.000000}{9.600000}\selectfont −2}%
\end{pgfscope}%
\begin{pgfscope}%
\pgfsetbuttcap%
\pgfsetroundjoin%
\definecolor{currentfill}{rgb}{0.000000,0.000000,0.000000}%
\pgfsetfillcolor{currentfill}%
\pgfsetlinewidth{0.501875pt}%
\definecolor{currentstroke}{rgb}{0.000000,0.000000,0.000000}%
\pgfsetstrokecolor{currentstroke}%
\pgfsetdash{}{0pt}%
\pgfsys@defobject{currentmarker}{\pgfqpoint{0.000000in}{0.000000in}}{\pgfqpoint{0.000000in}{0.069444in}}{%
\pgfpathmoveto{\pgfqpoint{0.000000in}{0.000000in}}%
\pgfpathlineto{\pgfqpoint{0.000000in}{0.069444in}}%
\pgfusepath{stroke,fill}%
}%
\begin{pgfscope}%
\pgfsys@transformshift{2.130577in}{0.417391in}%
\pgfsys@useobject{currentmarker}{}%
\end{pgfscope}%
\end{pgfscope}%
\begin{pgfscope}%
\pgfsetbuttcap%
\pgfsetroundjoin%
\definecolor{currentfill}{rgb}{0.000000,0.000000,0.000000}%
\pgfsetfillcolor{currentfill}%
\pgfsetlinewidth{0.501875pt}%
\definecolor{currentstroke}{rgb}{0.000000,0.000000,0.000000}%
\pgfsetstrokecolor{currentstroke}%
\pgfsetdash{}{0pt}%
\pgfsys@defobject{currentmarker}{\pgfqpoint{0.000000in}{-0.069444in}}{\pgfqpoint{0.000000in}{0.000000in}}{%
\pgfpathmoveto{\pgfqpoint{0.000000in}{0.000000in}}%
\pgfpathlineto{\pgfqpoint{0.000000in}{-0.069444in}}%
\pgfusepath{stroke,fill}%
}%
\begin{pgfscope}%
\pgfsys@transformshift{2.130577in}{2.592964in}%
\pgfsys@useobject{currentmarker}{}%
\end{pgfscope}%
\end{pgfscope}%
\begin{pgfscope}%
\pgftext[x=2.130577in,y=0.347947in,,top]{\rmfamily\fontsize{8.000000}{9.600000}\selectfont 0}%
\end{pgfscope}%
\begin{pgfscope}%
\pgfsetbuttcap%
\pgfsetroundjoin%
\definecolor{currentfill}{rgb}{0.000000,0.000000,0.000000}%
\pgfsetfillcolor{currentfill}%
\pgfsetlinewidth{0.501875pt}%
\definecolor{currentstroke}{rgb}{0.000000,0.000000,0.000000}%
\pgfsetstrokecolor{currentstroke}%
\pgfsetdash{}{0pt}%
\pgfsys@defobject{currentmarker}{\pgfqpoint{0.000000in}{0.000000in}}{\pgfqpoint{0.000000in}{0.069444in}}{%
\pgfpathmoveto{\pgfqpoint{0.000000in}{0.000000in}}%
\pgfpathlineto{\pgfqpoint{0.000000in}{0.069444in}}%
\pgfusepath{stroke,fill}%
}%
\begin{pgfscope}%
\pgfsys@transformshift{2.718489in}{0.417391in}%
\pgfsys@useobject{currentmarker}{}%
\end{pgfscope}%
\end{pgfscope}%
\begin{pgfscope}%
\pgfsetbuttcap%
\pgfsetroundjoin%
\definecolor{currentfill}{rgb}{0.000000,0.000000,0.000000}%
\pgfsetfillcolor{currentfill}%
\pgfsetlinewidth{0.501875pt}%
\definecolor{currentstroke}{rgb}{0.000000,0.000000,0.000000}%
\pgfsetstrokecolor{currentstroke}%
\pgfsetdash{}{0pt}%
\pgfsys@defobject{currentmarker}{\pgfqpoint{0.000000in}{-0.069444in}}{\pgfqpoint{0.000000in}{0.000000in}}{%
\pgfpathmoveto{\pgfqpoint{0.000000in}{0.000000in}}%
\pgfpathlineto{\pgfqpoint{0.000000in}{-0.069444in}}%
\pgfusepath{stroke,fill}%
}%
\begin{pgfscope}%
\pgfsys@transformshift{2.718489in}{2.592964in}%
\pgfsys@useobject{currentmarker}{}%
\end{pgfscope}%
\end{pgfscope}%
\begin{pgfscope}%
\pgftext[x=2.718489in,y=0.347947in,,top]{\rmfamily\fontsize{8.000000}{9.600000}\selectfont 2}%
\end{pgfscope}%
\begin{pgfscope}%
\pgfsetbuttcap%
\pgfsetroundjoin%
\definecolor{currentfill}{rgb}{0.000000,0.000000,0.000000}%
\pgfsetfillcolor{currentfill}%
\pgfsetlinewidth{0.501875pt}%
\definecolor{currentstroke}{rgb}{0.000000,0.000000,0.000000}%
\pgfsetstrokecolor{currentstroke}%
\pgfsetdash{}{0pt}%
\pgfsys@defobject{currentmarker}{\pgfqpoint{0.000000in}{0.000000in}}{\pgfqpoint{0.000000in}{0.069444in}}{%
\pgfpathmoveto{\pgfqpoint{0.000000in}{0.000000in}}%
\pgfpathlineto{\pgfqpoint{0.000000in}{0.069444in}}%
\pgfusepath{stroke,fill}%
}%
\begin{pgfscope}%
\pgfsys@transformshift{3.306401in}{0.417391in}%
\pgfsys@useobject{currentmarker}{}%
\end{pgfscope}%
\end{pgfscope}%
\begin{pgfscope}%
\pgfsetbuttcap%
\pgfsetroundjoin%
\definecolor{currentfill}{rgb}{0.000000,0.000000,0.000000}%
\pgfsetfillcolor{currentfill}%
\pgfsetlinewidth{0.501875pt}%
\definecolor{currentstroke}{rgb}{0.000000,0.000000,0.000000}%
\pgfsetstrokecolor{currentstroke}%
\pgfsetdash{}{0pt}%
\pgfsys@defobject{currentmarker}{\pgfqpoint{0.000000in}{-0.069444in}}{\pgfqpoint{0.000000in}{0.000000in}}{%
\pgfpathmoveto{\pgfqpoint{0.000000in}{0.000000in}}%
\pgfpathlineto{\pgfqpoint{0.000000in}{-0.069444in}}%
\pgfusepath{stroke,fill}%
}%
\begin{pgfscope}%
\pgfsys@transformshift{3.306401in}{2.592964in}%
\pgfsys@useobject{currentmarker}{}%
\end{pgfscope}%
\end{pgfscope}%
\begin{pgfscope}%
\pgftext[x=3.306401in,y=0.347947in,,top]{\rmfamily\fontsize{8.000000}{9.600000}\selectfont 4}%
\end{pgfscope}%
\begin{pgfscope}%
\pgfsetbuttcap%
\pgfsetroundjoin%
\definecolor{currentfill}{rgb}{0.000000,0.000000,0.000000}%
\pgfsetfillcolor{currentfill}%
\pgfsetlinewidth{0.501875pt}%
\definecolor{currentstroke}{rgb}{0.000000,0.000000,0.000000}%
\pgfsetstrokecolor{currentstroke}%
\pgfsetdash{}{0pt}%
\pgfsys@defobject{currentmarker}{\pgfqpoint{0.000000in}{0.000000in}}{\pgfqpoint{0.000000in}{0.069444in}}{%
\pgfpathmoveto{\pgfqpoint{0.000000in}{0.000000in}}%
\pgfpathlineto{\pgfqpoint{0.000000in}{0.069444in}}%
\pgfusepath{stroke,fill}%
}%
\begin{pgfscope}%
\pgfsys@transformshift{3.894313in}{0.417391in}%
\pgfsys@useobject{currentmarker}{}%
\end{pgfscope}%
\end{pgfscope}%
\begin{pgfscope}%
\pgfsetbuttcap%
\pgfsetroundjoin%
\definecolor{currentfill}{rgb}{0.000000,0.000000,0.000000}%
\pgfsetfillcolor{currentfill}%
\pgfsetlinewidth{0.501875pt}%
\definecolor{currentstroke}{rgb}{0.000000,0.000000,0.000000}%
\pgfsetstrokecolor{currentstroke}%
\pgfsetdash{}{0pt}%
\pgfsys@defobject{currentmarker}{\pgfqpoint{0.000000in}{-0.069444in}}{\pgfqpoint{0.000000in}{0.000000in}}{%
\pgfpathmoveto{\pgfqpoint{0.000000in}{0.000000in}}%
\pgfpathlineto{\pgfqpoint{0.000000in}{-0.069444in}}%
\pgfusepath{stroke,fill}%
}%
\begin{pgfscope}%
\pgfsys@transformshift{3.894313in}{2.592964in}%
\pgfsys@useobject{currentmarker}{}%
\end{pgfscope}%
\end{pgfscope}%
\begin{pgfscope}%
\pgftext[x=3.894313in,y=0.347947in,,top]{\rmfamily\fontsize{8.000000}{9.600000}\selectfont 6}%
\end{pgfscope}%
\begin{pgfscope}%
\pgftext[x=2.130577in,y=0.170972in,,top]{\rmfamily\fontsize{9.000000}{10.800000}\selectfont classifier response}%
\end{pgfscope}%
\begin{pgfscope}%
\pgfsetbuttcap%
\pgfsetroundjoin%
\definecolor{currentfill}{rgb}{0.000000,0.000000,0.000000}%
\pgfsetfillcolor{currentfill}%
\pgfsetlinewidth{0.501875pt}%
\definecolor{currentstroke}{rgb}{0.000000,0.000000,0.000000}%
\pgfsetstrokecolor{currentstroke}%
\pgfsetdash{}{0pt}%
\pgfsys@defobject{currentmarker}{\pgfqpoint{0.000000in}{0.000000in}}{\pgfqpoint{0.069444in}{0.000000in}}{%
\pgfpathmoveto{\pgfqpoint{0.000000in}{0.000000in}}%
\pgfpathlineto{\pgfqpoint{0.069444in}{0.000000in}}%
\pgfusepath{stroke,fill}%
}%
\begin{pgfscope}%
\pgfsys@transformshift{0.366840in}{0.417391in}%
\pgfsys@useobject{currentmarker}{}%
\end{pgfscope}%
\end{pgfscope}%
\begin{pgfscope}%
\pgfsetbuttcap%
\pgfsetroundjoin%
\definecolor{currentfill}{rgb}{0.000000,0.000000,0.000000}%
\pgfsetfillcolor{currentfill}%
\pgfsetlinewidth{0.501875pt}%
\definecolor{currentstroke}{rgb}{0.000000,0.000000,0.000000}%
\pgfsetstrokecolor{currentstroke}%
\pgfsetdash{}{0pt}%
\pgfsys@defobject{currentmarker}{\pgfqpoint{-0.069444in}{0.000000in}}{\pgfqpoint{0.000000in}{0.000000in}}{%
\pgfpathmoveto{\pgfqpoint{0.000000in}{0.000000in}}%
\pgfpathlineto{\pgfqpoint{-0.069444in}{0.000000in}}%
\pgfusepath{stroke,fill}%
}%
\begin{pgfscope}%
\pgfsys@transformshift{3.894313in}{0.417391in}%
\pgfsys@useobject{currentmarker}{}%
\end{pgfscope}%
\end{pgfscope}%
\begin{pgfscope}%
\pgftext[x=0.297396in,y=0.417391in,right,]{\rmfamily\fontsize{8.000000}{9.600000}\selectfont 0.00}%
\end{pgfscope}%
\begin{pgfscope}%
\pgfsetbuttcap%
\pgfsetroundjoin%
\definecolor{currentfill}{rgb}{0.000000,0.000000,0.000000}%
\pgfsetfillcolor{currentfill}%
\pgfsetlinewidth{0.501875pt}%
\definecolor{currentstroke}{rgb}{0.000000,0.000000,0.000000}%
\pgfsetstrokecolor{currentstroke}%
\pgfsetdash{}{0pt}%
\pgfsys@defobject{currentmarker}{\pgfqpoint{0.000000in}{0.000000in}}{\pgfqpoint{0.069444in}{0.000000in}}{%
\pgfpathmoveto{\pgfqpoint{0.000000in}{0.000000in}}%
\pgfpathlineto{\pgfqpoint{0.069444in}{0.000000in}}%
\pgfusepath{stroke,fill}%
}%
\begin{pgfscope}%
\pgfsys@transformshift{0.366840in}{0.728187in}%
\pgfsys@useobject{currentmarker}{}%
\end{pgfscope}%
\end{pgfscope}%
\begin{pgfscope}%
\pgfsetbuttcap%
\pgfsetroundjoin%
\definecolor{currentfill}{rgb}{0.000000,0.000000,0.000000}%
\pgfsetfillcolor{currentfill}%
\pgfsetlinewidth{0.501875pt}%
\definecolor{currentstroke}{rgb}{0.000000,0.000000,0.000000}%
\pgfsetstrokecolor{currentstroke}%
\pgfsetdash{}{0pt}%
\pgfsys@defobject{currentmarker}{\pgfqpoint{-0.069444in}{0.000000in}}{\pgfqpoint{0.000000in}{0.000000in}}{%
\pgfpathmoveto{\pgfqpoint{0.000000in}{0.000000in}}%
\pgfpathlineto{\pgfqpoint{-0.069444in}{0.000000in}}%
\pgfusepath{stroke,fill}%
}%
\begin{pgfscope}%
\pgfsys@transformshift{3.894313in}{0.728187in}%
\pgfsys@useobject{currentmarker}{}%
\end{pgfscope}%
\end{pgfscope}%
\begin{pgfscope}%
\pgftext[x=0.297396in,y=0.728187in,right,]{\rmfamily\fontsize{8.000000}{9.600000}\selectfont 0.05}%
\end{pgfscope}%
\begin{pgfscope}%
\pgfsetbuttcap%
\pgfsetroundjoin%
\definecolor{currentfill}{rgb}{0.000000,0.000000,0.000000}%
\pgfsetfillcolor{currentfill}%
\pgfsetlinewidth{0.501875pt}%
\definecolor{currentstroke}{rgb}{0.000000,0.000000,0.000000}%
\pgfsetstrokecolor{currentstroke}%
\pgfsetdash{}{0pt}%
\pgfsys@defobject{currentmarker}{\pgfqpoint{0.000000in}{0.000000in}}{\pgfqpoint{0.069444in}{0.000000in}}{%
\pgfpathmoveto{\pgfqpoint{0.000000in}{0.000000in}}%
\pgfpathlineto{\pgfqpoint{0.069444in}{0.000000in}}%
\pgfusepath{stroke,fill}%
}%
\begin{pgfscope}%
\pgfsys@transformshift{0.366840in}{1.038983in}%
\pgfsys@useobject{currentmarker}{}%
\end{pgfscope}%
\end{pgfscope}%
\begin{pgfscope}%
\pgfsetbuttcap%
\pgfsetroundjoin%
\definecolor{currentfill}{rgb}{0.000000,0.000000,0.000000}%
\pgfsetfillcolor{currentfill}%
\pgfsetlinewidth{0.501875pt}%
\definecolor{currentstroke}{rgb}{0.000000,0.000000,0.000000}%
\pgfsetstrokecolor{currentstroke}%
\pgfsetdash{}{0pt}%
\pgfsys@defobject{currentmarker}{\pgfqpoint{-0.069444in}{0.000000in}}{\pgfqpoint{0.000000in}{0.000000in}}{%
\pgfpathmoveto{\pgfqpoint{0.000000in}{0.000000in}}%
\pgfpathlineto{\pgfqpoint{-0.069444in}{0.000000in}}%
\pgfusepath{stroke,fill}%
}%
\begin{pgfscope}%
\pgfsys@transformshift{3.894313in}{1.038983in}%
\pgfsys@useobject{currentmarker}{}%
\end{pgfscope}%
\end{pgfscope}%
\begin{pgfscope}%
\pgftext[x=0.297396in,y=1.038983in,right,]{\rmfamily\fontsize{8.000000}{9.600000}\selectfont 0.10}%
\end{pgfscope}%
\begin{pgfscope}%
\pgfsetbuttcap%
\pgfsetroundjoin%
\definecolor{currentfill}{rgb}{0.000000,0.000000,0.000000}%
\pgfsetfillcolor{currentfill}%
\pgfsetlinewidth{0.501875pt}%
\definecolor{currentstroke}{rgb}{0.000000,0.000000,0.000000}%
\pgfsetstrokecolor{currentstroke}%
\pgfsetdash{}{0pt}%
\pgfsys@defobject{currentmarker}{\pgfqpoint{0.000000in}{0.000000in}}{\pgfqpoint{0.069444in}{0.000000in}}{%
\pgfpathmoveto{\pgfqpoint{0.000000in}{0.000000in}}%
\pgfpathlineto{\pgfqpoint{0.069444in}{0.000000in}}%
\pgfusepath{stroke,fill}%
}%
\begin{pgfscope}%
\pgfsys@transformshift{0.366840in}{1.349779in}%
\pgfsys@useobject{currentmarker}{}%
\end{pgfscope}%
\end{pgfscope}%
\begin{pgfscope}%
\pgfsetbuttcap%
\pgfsetroundjoin%
\definecolor{currentfill}{rgb}{0.000000,0.000000,0.000000}%
\pgfsetfillcolor{currentfill}%
\pgfsetlinewidth{0.501875pt}%
\definecolor{currentstroke}{rgb}{0.000000,0.000000,0.000000}%
\pgfsetstrokecolor{currentstroke}%
\pgfsetdash{}{0pt}%
\pgfsys@defobject{currentmarker}{\pgfqpoint{-0.069444in}{0.000000in}}{\pgfqpoint{0.000000in}{0.000000in}}{%
\pgfpathmoveto{\pgfqpoint{0.000000in}{0.000000in}}%
\pgfpathlineto{\pgfqpoint{-0.069444in}{0.000000in}}%
\pgfusepath{stroke,fill}%
}%
\begin{pgfscope}%
\pgfsys@transformshift{3.894313in}{1.349779in}%
\pgfsys@useobject{currentmarker}{}%
\end{pgfscope}%
\end{pgfscope}%
\begin{pgfscope}%
\pgftext[x=0.297396in,y=1.349779in,right,]{\rmfamily\fontsize{8.000000}{9.600000}\selectfont 0.15}%
\end{pgfscope}%
\begin{pgfscope}%
\pgfsetbuttcap%
\pgfsetroundjoin%
\definecolor{currentfill}{rgb}{0.000000,0.000000,0.000000}%
\pgfsetfillcolor{currentfill}%
\pgfsetlinewidth{0.501875pt}%
\definecolor{currentstroke}{rgb}{0.000000,0.000000,0.000000}%
\pgfsetstrokecolor{currentstroke}%
\pgfsetdash{}{0pt}%
\pgfsys@defobject{currentmarker}{\pgfqpoint{0.000000in}{0.000000in}}{\pgfqpoint{0.069444in}{0.000000in}}{%
\pgfpathmoveto{\pgfqpoint{0.000000in}{0.000000in}}%
\pgfpathlineto{\pgfqpoint{0.069444in}{0.000000in}}%
\pgfusepath{stroke,fill}%
}%
\begin{pgfscope}%
\pgfsys@transformshift{0.366840in}{1.660576in}%
\pgfsys@useobject{currentmarker}{}%
\end{pgfscope}%
\end{pgfscope}%
\begin{pgfscope}%
\pgfsetbuttcap%
\pgfsetroundjoin%
\definecolor{currentfill}{rgb}{0.000000,0.000000,0.000000}%
\pgfsetfillcolor{currentfill}%
\pgfsetlinewidth{0.501875pt}%
\definecolor{currentstroke}{rgb}{0.000000,0.000000,0.000000}%
\pgfsetstrokecolor{currentstroke}%
\pgfsetdash{}{0pt}%
\pgfsys@defobject{currentmarker}{\pgfqpoint{-0.069444in}{0.000000in}}{\pgfqpoint{0.000000in}{0.000000in}}{%
\pgfpathmoveto{\pgfqpoint{0.000000in}{0.000000in}}%
\pgfpathlineto{\pgfqpoint{-0.069444in}{0.000000in}}%
\pgfusepath{stroke,fill}%
}%
\begin{pgfscope}%
\pgfsys@transformshift{3.894313in}{1.660576in}%
\pgfsys@useobject{currentmarker}{}%
\end{pgfscope}%
\end{pgfscope}%
\begin{pgfscope}%
\pgftext[x=0.297396in,y=1.660576in,right,]{\rmfamily\fontsize{8.000000}{9.600000}\selectfont 0.20}%
\end{pgfscope}%
\begin{pgfscope}%
\pgfsetbuttcap%
\pgfsetroundjoin%
\definecolor{currentfill}{rgb}{0.000000,0.000000,0.000000}%
\pgfsetfillcolor{currentfill}%
\pgfsetlinewidth{0.501875pt}%
\definecolor{currentstroke}{rgb}{0.000000,0.000000,0.000000}%
\pgfsetstrokecolor{currentstroke}%
\pgfsetdash{}{0pt}%
\pgfsys@defobject{currentmarker}{\pgfqpoint{0.000000in}{0.000000in}}{\pgfqpoint{0.069444in}{0.000000in}}{%
\pgfpathmoveto{\pgfqpoint{0.000000in}{0.000000in}}%
\pgfpathlineto{\pgfqpoint{0.069444in}{0.000000in}}%
\pgfusepath{stroke,fill}%
}%
\begin{pgfscope}%
\pgfsys@transformshift{0.366840in}{1.971372in}%
\pgfsys@useobject{currentmarker}{}%
\end{pgfscope}%
\end{pgfscope}%
\begin{pgfscope}%
\pgfsetbuttcap%
\pgfsetroundjoin%
\definecolor{currentfill}{rgb}{0.000000,0.000000,0.000000}%
\pgfsetfillcolor{currentfill}%
\pgfsetlinewidth{0.501875pt}%
\definecolor{currentstroke}{rgb}{0.000000,0.000000,0.000000}%
\pgfsetstrokecolor{currentstroke}%
\pgfsetdash{}{0pt}%
\pgfsys@defobject{currentmarker}{\pgfqpoint{-0.069444in}{0.000000in}}{\pgfqpoint{0.000000in}{0.000000in}}{%
\pgfpathmoveto{\pgfqpoint{0.000000in}{0.000000in}}%
\pgfpathlineto{\pgfqpoint{-0.069444in}{0.000000in}}%
\pgfusepath{stroke,fill}%
}%
\begin{pgfscope}%
\pgfsys@transformshift{3.894313in}{1.971372in}%
\pgfsys@useobject{currentmarker}{}%
\end{pgfscope}%
\end{pgfscope}%
\begin{pgfscope}%
\pgftext[x=0.297396in,y=1.971372in,right,]{\rmfamily\fontsize{8.000000}{9.600000}\selectfont 0.25}%
\end{pgfscope}%
\begin{pgfscope}%
\pgfsetbuttcap%
\pgfsetroundjoin%
\definecolor{currentfill}{rgb}{0.000000,0.000000,0.000000}%
\pgfsetfillcolor{currentfill}%
\pgfsetlinewidth{0.501875pt}%
\definecolor{currentstroke}{rgb}{0.000000,0.000000,0.000000}%
\pgfsetstrokecolor{currentstroke}%
\pgfsetdash{}{0pt}%
\pgfsys@defobject{currentmarker}{\pgfqpoint{0.000000in}{0.000000in}}{\pgfqpoint{0.069444in}{0.000000in}}{%
\pgfpathmoveto{\pgfqpoint{0.000000in}{0.000000in}}%
\pgfpathlineto{\pgfqpoint{0.069444in}{0.000000in}}%
\pgfusepath{stroke,fill}%
}%
\begin{pgfscope}%
\pgfsys@transformshift{0.366840in}{2.282168in}%
\pgfsys@useobject{currentmarker}{}%
\end{pgfscope}%
\end{pgfscope}%
\begin{pgfscope}%
\pgfsetbuttcap%
\pgfsetroundjoin%
\definecolor{currentfill}{rgb}{0.000000,0.000000,0.000000}%
\pgfsetfillcolor{currentfill}%
\pgfsetlinewidth{0.501875pt}%
\definecolor{currentstroke}{rgb}{0.000000,0.000000,0.000000}%
\pgfsetstrokecolor{currentstroke}%
\pgfsetdash{}{0pt}%
\pgfsys@defobject{currentmarker}{\pgfqpoint{-0.069444in}{0.000000in}}{\pgfqpoint{0.000000in}{0.000000in}}{%
\pgfpathmoveto{\pgfqpoint{0.000000in}{0.000000in}}%
\pgfpathlineto{\pgfqpoint{-0.069444in}{0.000000in}}%
\pgfusepath{stroke,fill}%
}%
\begin{pgfscope}%
\pgfsys@transformshift{3.894313in}{2.282168in}%
\pgfsys@useobject{currentmarker}{}%
\end{pgfscope}%
\end{pgfscope}%
\begin{pgfscope}%
\pgftext[x=0.297396in,y=2.282168in,right,]{\rmfamily\fontsize{8.000000}{9.600000}\selectfont 0.30}%
\end{pgfscope}%
\begin{pgfscope}%
\pgfsetbuttcap%
\pgfsetroundjoin%
\definecolor{currentfill}{rgb}{0.000000,0.000000,0.000000}%
\pgfsetfillcolor{currentfill}%
\pgfsetlinewidth{0.501875pt}%
\definecolor{currentstroke}{rgb}{0.000000,0.000000,0.000000}%
\pgfsetstrokecolor{currentstroke}%
\pgfsetdash{}{0pt}%
\pgfsys@defobject{currentmarker}{\pgfqpoint{0.000000in}{0.000000in}}{\pgfqpoint{0.069444in}{0.000000in}}{%
\pgfpathmoveto{\pgfqpoint{0.000000in}{0.000000in}}%
\pgfpathlineto{\pgfqpoint{0.069444in}{0.000000in}}%
\pgfusepath{stroke,fill}%
}%
\begin{pgfscope}%
\pgfsys@transformshift{0.366840in}{2.592964in}%
\pgfsys@useobject{currentmarker}{}%
\end{pgfscope}%
\end{pgfscope}%
\begin{pgfscope}%
\pgfsetbuttcap%
\pgfsetroundjoin%
\definecolor{currentfill}{rgb}{0.000000,0.000000,0.000000}%
\pgfsetfillcolor{currentfill}%
\pgfsetlinewidth{0.501875pt}%
\definecolor{currentstroke}{rgb}{0.000000,0.000000,0.000000}%
\pgfsetstrokecolor{currentstroke}%
\pgfsetdash{}{0pt}%
\pgfsys@defobject{currentmarker}{\pgfqpoint{-0.069444in}{0.000000in}}{\pgfqpoint{0.000000in}{0.000000in}}{%
\pgfpathmoveto{\pgfqpoint{0.000000in}{0.000000in}}%
\pgfpathlineto{\pgfqpoint{-0.069444in}{0.000000in}}%
\pgfusepath{stroke,fill}%
}%
\begin{pgfscope}%
\pgfsys@transformshift{3.894313in}{2.592964in}%
\pgfsys@useobject{currentmarker}{}%
\end{pgfscope}%
\end{pgfscope}%
\begin{pgfscope}%
\pgftext[x=0.297396in,y=2.592964in,right,]{\rmfamily\fontsize{8.000000}{9.600000}\selectfont 0.35}%
\end{pgfscope}%
\end{pgfpicture}%
\makeatother%
\endgroup%

  \caption{
    Response of the classifier on the signal (blue) and background (red) subsamples of the training dataset.
    The two distributions are normalized to $1$.
  }
  \label{fig:response}
\end{figure}

An objective criterium to select an optimal threshold on the classifier response is through the optimization of a \emph{figure of merit} (FOM).
A FOM is a function of the expected number of signal and background decays in a signal window, both of which depend on the chosen threshold.
The optimal threshold is determined by finding the value that maximizes a given FOM.

For this analysis, the \emph{Punzi} figure of merit \cite{Punzi}
\begin{equation}
  \mathrm{FOM}(s, b) = \frac{s}{\sqrt{b} + \frac{a}{2}}
\end{equation}
with $a=3$ has been chosen.
Here, $s$ and $b$ are the expected numbers of signal and background decays, given a threshold on the classifier response.
The initial number of expected signal decays does not affect the maximum of the FOM and can be factorized out of the function.
The FOM then depends on the estimated signal efficiency only.
The expected number of background decays has been estimated by measuring the initial number (before the multivariate selection) through a fit to the $\PBzero\to\APDzero\APmuon\Pmuon$ dataset (see figure \ref{fig:bkginitial}).
The fit model consists of a single exponential function.
The estimate is calculated by integrating this function in a $\pm 1\sigma$ window around the nominal $B^0$ mass, where $\sigma$ is the standard deviation of the $B^0$ mass distribution as determined using simulated candidates (see section \ref{signalmodel}).
This yields a value of roughly $N_\text{bkg} = 6622$.
$N_\text{bkg}$ is multiplied with the background efficiency of the multivariate selection as determined on the training sample.

The dependence of the FOM on the chosen threshold is shown in figure \ref{fig:fom}.
The maximum is found for a threshold at $3.905$.
Applying the threshold on the simulated signal sample yields an efficiency of
\begin{equation}
  \varepsilon_\text{classifier}(\PBzero\to\APDzero\APmuon\Pmuon) = \SI{27.53 \pm 0.18}{\percent}
\end{equation}
with a background rejection of \SI{99.842 \pm 0.006}{\percent} on the entire blinded $\PBzero\to\APDzero\APmuon\Pmuon$ dataset.

\begin{figure}
  \centering
  %% Creator: Matplotlib, PGF backend
%%
%% To include the figure in your LaTeX document, write
%%   \input{<filename>.pgf}
%%
%% Make sure the required packages are loaded in your preamble
%%   \usepackage{pgf}
%%
%% Figures using additional raster images can only be included by \input if
%% they are in the same directory as the main LaTeX file. For loading figures
%% from other directories you can use the `import` package
%%   \usepackage{import}
%% and then include the figures with
%%   \import{<path to file>}{<filename>.pgf}
%%
%% Matplotlib used the following preamble
%%   \usepackage{fontspec}
%%   \setmainfont{DejaVu Serif}
%%   \setsansfont{DejaVu Sans}
%%   \setmonofont{DejaVu Sans Mono}
%%
\begingroup%
\makeatletter%
\begin{pgfpicture}%
\pgfpathrectangle{\pgfpointorigin}{\pgfqpoint{3.848410in}{3.039209in}}%
\pgfusepath{use as bounding box, clip}%
\begin{pgfscope}%
\pgfsetbuttcap%
\pgfsetmiterjoin%
\definecolor{currentfill}{rgb}{1.000000,1.000000,1.000000}%
\pgfsetfillcolor{currentfill}%
\pgfsetlinewidth{0.000000pt}%
\definecolor{currentstroke}{rgb}{1.000000,1.000000,1.000000}%
\pgfsetstrokecolor{currentstroke}%
\pgfsetdash{}{0pt}%
\pgfpathmoveto{\pgfqpoint{0.000000in}{0.000000in}}%
\pgfpathlineto{\pgfqpoint{3.848410in}{0.000000in}}%
\pgfpathlineto{\pgfqpoint{3.848410in}{3.039209in}}%
\pgfpathlineto{\pgfqpoint{0.000000in}{3.039209in}}%
\pgfpathclose%
\pgfusepath{fill}%
\end{pgfscope}%
\begin{pgfscope}%
\pgfsetbuttcap%
\pgfsetmiterjoin%
\definecolor{currentfill}{rgb}{1.000000,1.000000,1.000000}%
\pgfsetfillcolor{currentfill}%
\pgfsetlinewidth{0.000000pt}%
\definecolor{currentstroke}{rgb}{0.000000,0.000000,0.000000}%
\pgfsetstrokecolor{currentstroke}%
\pgfsetstrokeopacity{0.000000}%
\pgfsetdash{}{0pt}%
\pgfpathmoveto{\pgfqpoint{0.636356in}{0.440955in}}%
\pgfpathlineto{\pgfqpoint{3.657026in}{0.440955in}}%
\pgfpathlineto{\pgfqpoint{3.657026in}{0.731012in}}%
\pgfpathlineto{\pgfqpoint{0.636356in}{0.731012in}}%
\pgfpathclose%
\pgfusepath{fill}%
\end{pgfscope}%
\begin{pgfscope}%
\pgfpathrectangle{\pgfqpoint{0.636356in}{0.440955in}}{\pgfqpoint{3.020670in}{0.290057in}} %
\pgfusepath{clip}%
\pgfsetbuttcap%
\pgfsetroundjoin%
\definecolor{currentfill}{rgb}{0.733333,0.733333,0.733333}%
\pgfsetfillcolor{currentfill}%
\pgfsetlinewidth{0.000000pt}%
\definecolor{currentstroke}{rgb}{0.733333,0.733333,0.733333}%
\pgfsetstrokecolor{currentstroke}%
\pgfsetdash{}{0pt}%
\pgfpathmoveto{\pgfqpoint{0.636356in}{0.634327in}}%
\pgfpathlineto{\pgfqpoint{0.636356in}{0.682669in}}%
\pgfpathlineto{\pgfqpoint{3.657026in}{0.682669in}}%
\pgfpathlineto{\pgfqpoint{3.657026in}{0.634327in}}%
\pgfpathlineto{\pgfqpoint{3.657026in}{0.634327in}}%
\pgfpathlineto{\pgfqpoint{0.636356in}{0.634327in}}%
\pgfpathlineto{\pgfqpoint{0.636356in}{0.634327in}}%
\pgfusepath{fill}%
\end{pgfscope}%
\begin{pgfscope}%
\pgfpathrectangle{\pgfqpoint{0.636356in}{0.440955in}}{\pgfqpoint{3.020670in}{0.290057in}} %
\pgfusepath{clip}%
\pgfsetbuttcap%
\pgfsetroundjoin%
\definecolor{currentfill}{rgb}{0.733333,0.733333,0.733333}%
\pgfsetfillcolor{currentfill}%
\pgfsetlinewidth{0.000000pt}%
\definecolor{currentstroke}{rgb}{0.733333,0.733333,0.733333}%
\pgfsetstrokecolor{currentstroke}%
\pgfsetdash{}{0pt}%
\pgfpathmoveto{\pgfqpoint{0.636356in}{0.537641in}}%
\pgfpathlineto{\pgfqpoint{0.636356in}{0.489298in}}%
\pgfpathlineto{\pgfqpoint{3.657026in}{0.489298in}}%
\pgfpathlineto{\pgfqpoint{3.657026in}{0.537641in}}%
\pgfpathlineto{\pgfqpoint{3.657026in}{0.537641in}}%
\pgfpathlineto{\pgfqpoint{0.636356in}{0.537641in}}%
\pgfpathlineto{\pgfqpoint{0.636356in}{0.537641in}}%
\pgfusepath{fill}%
\end{pgfscope}%
\begin{pgfscope}%
\pgfpathrectangle{\pgfqpoint{0.636356in}{0.440955in}}{\pgfqpoint{3.020670in}{0.290057in}} %
\pgfusepath{clip}%
\pgfsetbuttcap%
\pgfsetmiterjoin%
\definecolor{currentfill}{rgb}{0.800000,0.266667,0.266667}%
\pgfsetfillcolor{currentfill}%
\pgfsetlinewidth{0.501875pt}%
\definecolor{currentstroke}{rgb}{0.000000,0.000000,0.000000}%
\pgfsetstrokecolor{currentstroke}%
\pgfsetdash{}{0pt}%
\pgfpathmoveto{\pgfqpoint{0.636356in}{0.436938in}}%
\pgfpathlineto{\pgfqpoint{0.666563in}{0.436938in}}%
\pgfpathlineto{\pgfqpoint{0.666563in}{0.585984in}}%
\pgfpathlineto{\pgfqpoint{0.636356in}{0.585984in}}%
\pgfpathlineto{\pgfqpoint{0.636356in}{0.436938in}}%
\pgfusepath{stroke,fill}%
\end{pgfscope}%
\begin{pgfscope}%
\pgfpathrectangle{\pgfqpoint{0.636356in}{0.440955in}}{\pgfqpoint{3.020670in}{0.290057in}} %
\pgfusepath{clip}%
\pgfsetbuttcap%
\pgfsetmiterjoin%
\definecolor{currentfill}{rgb}{0.333333,0.333333,0.333333}%
\pgfsetfillcolor{currentfill}%
\pgfsetlinewidth{0.501875pt}%
\definecolor{currentstroke}{rgb}{0.000000,0.000000,0.000000}%
\pgfsetstrokecolor{currentstroke}%
\pgfsetdash{}{0pt}%
\pgfpathmoveto{\pgfqpoint{0.666563in}{0.585498in}}%
\pgfpathlineto{\pgfqpoint{0.696769in}{0.585498in}}%
\pgfpathlineto{\pgfqpoint{0.696769in}{0.585984in}}%
\pgfpathlineto{\pgfqpoint{0.666563in}{0.585984in}}%
\pgfpathlineto{\pgfqpoint{0.666563in}{0.585498in}}%
\pgfusepath{stroke,fill}%
\end{pgfscope}%
\begin{pgfscope}%
\pgfpathrectangle{\pgfqpoint{0.636356in}{0.440955in}}{\pgfqpoint{3.020670in}{0.290057in}} %
\pgfusepath{clip}%
\pgfsetbuttcap%
\pgfsetmiterjoin%
\definecolor{currentfill}{rgb}{0.333333,0.333333,0.333333}%
\pgfsetfillcolor{currentfill}%
\pgfsetlinewidth{0.501875pt}%
\definecolor{currentstroke}{rgb}{0.000000,0.000000,0.000000}%
\pgfsetstrokecolor{currentstroke}%
\pgfsetdash{}{0pt}%
\pgfpathmoveto{\pgfqpoint{0.696769in}{0.499790in}}%
\pgfpathlineto{\pgfqpoint{0.726976in}{0.499790in}}%
\pgfpathlineto{\pgfqpoint{0.726976in}{0.585984in}}%
\pgfpathlineto{\pgfqpoint{0.696769in}{0.585984in}}%
\pgfpathlineto{\pgfqpoint{0.696769in}{0.499790in}}%
\pgfusepath{stroke,fill}%
\end{pgfscope}%
\begin{pgfscope}%
\pgfpathrectangle{\pgfqpoint{0.636356in}{0.440955in}}{\pgfqpoint{3.020670in}{0.290057in}} %
\pgfusepath{clip}%
\pgfsetbuttcap%
\pgfsetmiterjoin%
\definecolor{currentfill}{rgb}{0.333333,0.333333,0.333333}%
\pgfsetfillcolor{currentfill}%
\pgfsetlinewidth{0.501875pt}%
\definecolor{currentstroke}{rgb}{0.000000,0.000000,0.000000}%
\pgfsetstrokecolor{currentstroke}%
\pgfsetdash{}{0pt}%
\pgfpathmoveto{\pgfqpoint{0.726976in}{0.507693in}}%
\pgfpathlineto{\pgfqpoint{0.757183in}{0.507693in}}%
\pgfpathlineto{\pgfqpoint{0.757183in}{0.585984in}}%
\pgfpathlineto{\pgfqpoint{0.726976in}{0.585984in}}%
\pgfpathlineto{\pgfqpoint{0.726976in}{0.507693in}}%
\pgfusepath{stroke,fill}%
\end{pgfscope}%
\begin{pgfscope}%
\pgfpathrectangle{\pgfqpoint{0.636356in}{0.440955in}}{\pgfqpoint{3.020670in}{0.290057in}} %
\pgfusepath{clip}%
\pgfsetbuttcap%
\pgfsetmiterjoin%
\definecolor{currentfill}{rgb}{0.333333,0.333333,0.333333}%
\pgfsetfillcolor{currentfill}%
\pgfsetlinewidth{0.501875pt}%
\definecolor{currentstroke}{rgb}{0.000000,0.000000,0.000000}%
\pgfsetstrokecolor{currentstroke}%
\pgfsetdash{}{0pt}%
\pgfpathmoveto{\pgfqpoint{0.757183in}{0.585984in}}%
\pgfpathlineto{\pgfqpoint{0.787389in}{0.585984in}}%
\pgfpathlineto{\pgfqpoint{0.787389in}{0.606607in}}%
\pgfpathlineto{\pgfqpoint{0.757183in}{0.606607in}}%
\pgfpathlineto{\pgfqpoint{0.757183in}{0.585984in}}%
\pgfusepath{stroke,fill}%
\end{pgfscope}%
\begin{pgfscope}%
\pgfpathrectangle{\pgfqpoint{0.636356in}{0.440955in}}{\pgfqpoint{3.020670in}{0.290057in}} %
\pgfusepath{clip}%
\pgfsetbuttcap%
\pgfsetmiterjoin%
\definecolor{currentfill}{rgb}{0.333333,0.333333,0.333333}%
\pgfsetfillcolor{currentfill}%
\pgfsetlinewidth{0.501875pt}%
\definecolor{currentstroke}{rgb}{0.000000,0.000000,0.000000}%
\pgfsetstrokecolor{currentstroke}%
\pgfsetdash{}{0pt}%
\pgfpathmoveto{\pgfqpoint{0.787389in}{0.585984in}}%
\pgfpathlineto{\pgfqpoint{0.817596in}{0.585984in}}%
\pgfpathlineto{\pgfqpoint{0.817596in}{0.592733in}}%
\pgfpathlineto{\pgfqpoint{0.787389in}{0.592733in}}%
\pgfpathlineto{\pgfqpoint{0.787389in}{0.585984in}}%
\pgfusepath{stroke,fill}%
\end{pgfscope}%
\begin{pgfscope}%
\pgfpathrectangle{\pgfqpoint{0.636356in}{0.440955in}}{\pgfqpoint{3.020670in}{0.290057in}} %
\pgfusepath{clip}%
\pgfsetbuttcap%
\pgfsetmiterjoin%
\definecolor{currentfill}{rgb}{0.333333,0.333333,0.333333}%
\pgfsetfillcolor{currentfill}%
\pgfsetlinewidth{0.501875pt}%
\definecolor{currentstroke}{rgb}{0.000000,0.000000,0.000000}%
\pgfsetstrokecolor{currentstroke}%
\pgfsetdash{}{0pt}%
\pgfpathmoveto{\pgfqpoint{0.817596in}{0.585984in}}%
\pgfpathlineto{\pgfqpoint{0.847803in}{0.585984in}}%
\pgfpathlineto{\pgfqpoint{0.847803in}{0.641355in}}%
\pgfpathlineto{\pgfqpoint{0.817596in}{0.641355in}}%
\pgfpathlineto{\pgfqpoint{0.817596in}{0.585984in}}%
\pgfusepath{stroke,fill}%
\end{pgfscope}%
\begin{pgfscope}%
\pgfpathrectangle{\pgfqpoint{0.636356in}{0.440955in}}{\pgfqpoint{3.020670in}{0.290057in}} %
\pgfusepath{clip}%
\pgfsetbuttcap%
\pgfsetmiterjoin%
\definecolor{currentfill}{rgb}{0.333333,0.333333,0.333333}%
\pgfsetfillcolor{currentfill}%
\pgfsetlinewidth{0.501875pt}%
\definecolor{currentstroke}{rgb}{0.000000,0.000000,0.000000}%
\pgfsetstrokecolor{currentstroke}%
\pgfsetdash{}{0pt}%
\pgfpathmoveto{\pgfqpoint{0.847803in}{0.585984in}}%
\pgfpathlineto{\pgfqpoint{0.878009in}{0.585984in}}%
\pgfpathlineto{\pgfqpoint{0.878009in}{0.595888in}}%
\pgfpathlineto{\pgfqpoint{0.847803in}{0.595888in}}%
\pgfpathlineto{\pgfqpoint{0.847803in}{0.585984in}}%
\pgfusepath{stroke,fill}%
\end{pgfscope}%
\begin{pgfscope}%
\pgfpathrectangle{\pgfqpoint{0.636356in}{0.440955in}}{\pgfqpoint{3.020670in}{0.290057in}} %
\pgfusepath{clip}%
\pgfsetbuttcap%
\pgfsetmiterjoin%
\definecolor{currentfill}{rgb}{0.333333,0.333333,0.333333}%
\pgfsetfillcolor{currentfill}%
\pgfsetlinewidth{0.501875pt}%
\definecolor{currentstroke}{rgb}{0.000000,0.000000,0.000000}%
\pgfsetstrokecolor{currentstroke}%
\pgfsetdash{}{0pt}%
\pgfpathmoveto{\pgfqpoint{0.878009in}{0.575008in}}%
\pgfpathlineto{\pgfqpoint{0.908216in}{0.575008in}}%
\pgfpathlineto{\pgfqpoint{0.908216in}{0.585984in}}%
\pgfpathlineto{\pgfqpoint{0.878009in}{0.585984in}}%
\pgfpathlineto{\pgfqpoint{0.878009in}{0.575008in}}%
\pgfusepath{stroke,fill}%
\end{pgfscope}%
\begin{pgfscope}%
\pgfpathrectangle{\pgfqpoint{0.636356in}{0.440955in}}{\pgfqpoint{3.020670in}{0.290057in}} %
\pgfusepath{clip}%
\pgfsetbuttcap%
\pgfsetmiterjoin%
\definecolor{currentfill}{rgb}{0.333333,0.333333,0.333333}%
\pgfsetfillcolor{currentfill}%
\pgfsetlinewidth{0.501875pt}%
\definecolor{currentstroke}{rgb}{0.000000,0.000000,0.000000}%
\pgfsetstrokecolor{currentstroke}%
\pgfsetdash{}{0pt}%
\pgfpathmoveto{\pgfqpoint{0.908216in}{0.561369in}}%
\pgfpathlineto{\pgfqpoint{0.938423in}{0.561369in}}%
\pgfpathlineto{\pgfqpoint{0.938423in}{0.585984in}}%
\pgfpathlineto{\pgfqpoint{0.908216in}{0.585984in}}%
\pgfpathlineto{\pgfqpoint{0.908216in}{0.561369in}}%
\pgfusepath{stroke,fill}%
\end{pgfscope}%
\begin{pgfscope}%
\pgfpathrectangle{\pgfqpoint{0.636356in}{0.440955in}}{\pgfqpoint{3.020670in}{0.290057in}} %
\pgfusepath{clip}%
\pgfsetbuttcap%
\pgfsetmiterjoin%
\definecolor{currentfill}{rgb}{0.333333,0.333333,0.333333}%
\pgfsetfillcolor{currentfill}%
\pgfsetlinewidth{0.501875pt}%
\definecolor{currentstroke}{rgb}{0.000000,0.000000,0.000000}%
\pgfsetstrokecolor{currentstroke}%
\pgfsetdash{}{0pt}%
\pgfpathmoveto{\pgfqpoint{0.938423in}{0.510314in}}%
\pgfpathlineto{\pgfqpoint{0.968629in}{0.510314in}}%
\pgfpathlineto{\pgfqpoint{0.968629in}{0.585984in}}%
\pgfpathlineto{\pgfqpoint{0.938423in}{0.585984in}}%
\pgfpathlineto{\pgfqpoint{0.938423in}{0.510314in}}%
\pgfusepath{stroke,fill}%
\end{pgfscope}%
\begin{pgfscope}%
\pgfpathrectangle{\pgfqpoint{0.636356in}{0.440955in}}{\pgfqpoint{3.020670in}{0.290057in}} %
\pgfusepath{clip}%
\pgfsetbuttcap%
\pgfsetmiterjoin%
\definecolor{currentfill}{rgb}{0.333333,0.333333,0.333333}%
\pgfsetfillcolor{currentfill}%
\pgfsetlinewidth{0.501875pt}%
\definecolor{currentstroke}{rgb}{0.000000,0.000000,0.000000}%
\pgfsetstrokecolor{currentstroke}%
\pgfsetdash{}{0pt}%
\pgfpathmoveto{\pgfqpoint{0.968629in}{0.585984in}}%
\pgfpathlineto{\pgfqpoint{0.998836in}{0.585984in}}%
\pgfpathlineto{\pgfqpoint{0.998836in}{0.626519in}}%
\pgfpathlineto{\pgfqpoint{0.968629in}{0.626519in}}%
\pgfpathlineto{\pgfqpoint{0.968629in}{0.585984in}}%
\pgfusepath{stroke,fill}%
\end{pgfscope}%
\begin{pgfscope}%
\pgfpathrectangle{\pgfqpoint{0.636356in}{0.440955in}}{\pgfqpoint{3.020670in}{0.290057in}} %
\pgfusepath{clip}%
\pgfsetbuttcap%
\pgfsetmiterjoin%
\definecolor{currentfill}{rgb}{0.333333,0.333333,0.333333}%
\pgfsetfillcolor{currentfill}%
\pgfsetlinewidth{0.501875pt}%
\definecolor{currentstroke}{rgb}{0.000000,0.000000,0.000000}%
\pgfsetstrokecolor{currentstroke}%
\pgfsetdash{}{0pt}%
\pgfpathmoveto{\pgfqpoint{0.998836in}{0.585984in}}%
\pgfpathlineto{\pgfqpoint{1.029043in}{0.585984in}}%
\pgfpathlineto{\pgfqpoint{1.029043in}{0.637530in}}%
\pgfpathlineto{\pgfqpoint{0.998836in}{0.637530in}}%
\pgfpathlineto{\pgfqpoint{0.998836in}{0.585984in}}%
\pgfusepath{stroke,fill}%
\end{pgfscope}%
\begin{pgfscope}%
\pgfpathrectangle{\pgfqpoint{0.636356in}{0.440955in}}{\pgfqpoint{3.020670in}{0.290057in}} %
\pgfusepath{clip}%
\pgfsetbuttcap%
\pgfsetmiterjoin%
\definecolor{currentfill}{rgb}{0.333333,0.333333,0.333333}%
\pgfsetfillcolor{currentfill}%
\pgfsetlinewidth{0.501875pt}%
\definecolor{currentstroke}{rgb}{0.000000,0.000000,0.000000}%
\pgfsetstrokecolor{currentstroke}%
\pgfsetdash{}{0pt}%
\pgfpathmoveto{\pgfqpoint{1.029043in}{0.585984in}}%
\pgfpathlineto{\pgfqpoint{1.059250in}{0.585984in}}%
\pgfpathlineto{\pgfqpoint{1.059250in}{0.596526in}}%
\pgfpathlineto{\pgfqpoint{1.029043in}{0.596526in}}%
\pgfpathlineto{\pgfqpoint{1.029043in}{0.585984in}}%
\pgfusepath{stroke,fill}%
\end{pgfscope}%
\begin{pgfscope}%
\pgfpathrectangle{\pgfqpoint{0.636356in}{0.440955in}}{\pgfqpoint{3.020670in}{0.290057in}} %
\pgfusepath{clip}%
\pgfsetbuttcap%
\pgfsetmiterjoin%
\definecolor{currentfill}{rgb}{0.333333,0.333333,0.333333}%
\pgfsetfillcolor{currentfill}%
\pgfsetlinewidth{0.501875pt}%
\definecolor{currentstroke}{rgb}{0.000000,0.000000,0.000000}%
\pgfsetstrokecolor{currentstroke}%
\pgfsetdash{}{0pt}%
\pgfpathmoveto{\pgfqpoint{1.059250in}{0.585984in}}%
\pgfpathlineto{\pgfqpoint{1.089456in}{0.585984in}}%
\pgfpathlineto{\pgfqpoint{1.089456in}{0.682836in}}%
\pgfpathlineto{\pgfqpoint{1.059250in}{0.682836in}}%
\pgfpathlineto{\pgfqpoint{1.059250in}{0.585984in}}%
\pgfusepath{stroke,fill}%
\end{pgfscope}%
\begin{pgfscope}%
\pgfpathrectangle{\pgfqpoint{0.636356in}{0.440955in}}{\pgfqpoint{3.020670in}{0.290057in}} %
\pgfusepath{clip}%
\pgfsetbuttcap%
\pgfsetmiterjoin%
\definecolor{currentfill}{rgb}{0.800000,0.266667,0.266667}%
\pgfsetfillcolor{currentfill}%
\pgfsetlinewidth{0.501875pt}%
\definecolor{currentstroke}{rgb}{0.000000,0.000000,0.000000}%
\pgfsetstrokecolor{currentstroke}%
\pgfsetdash{}{0pt}%
\pgfpathmoveto{\pgfqpoint{1.089456in}{-0.611368in}}%
\pgfpathmoveto{\pgfqpoint{1.119663in}{0.430955in}}%
\pgfpathlineto{\pgfqpoint{1.119663in}{0.585984in}}%
\pgfpathlineto{\pgfqpoint{1.089456in}{0.585984in}}%
\pgfpathlineto{\pgfqpoint{1.089456in}{0.430955in}}%
\pgfusepath{stroke,fill}%
\end{pgfscope}%
\begin{pgfscope}%
\pgfpathrectangle{\pgfqpoint{0.636356in}{0.440955in}}{\pgfqpoint{3.020670in}{0.290057in}} %
\pgfusepath{clip}%
\pgfsetbuttcap%
\pgfsetmiterjoin%
\definecolor{currentfill}{rgb}{0.333333,0.333333,0.333333}%
\pgfsetfillcolor{currentfill}%
\pgfsetlinewidth{0.501875pt}%
\definecolor{currentstroke}{rgb}{0.000000,0.000000,0.000000}%
\pgfsetstrokecolor{currentstroke}%
\pgfsetdash{}{0pt}%
\pgfusepath{stroke,fill}%
\end{pgfscope}%
\begin{pgfscope}%
\pgfpathrectangle{\pgfqpoint{0.636356in}{0.440955in}}{\pgfqpoint{3.020670in}{0.290057in}} %
\pgfusepath{clip}%
\pgfsetbuttcap%
\pgfsetmiterjoin%
\definecolor{currentfill}{rgb}{0.333333,0.333333,0.333333}%
\pgfsetfillcolor{currentfill}%
\pgfsetlinewidth{0.501875pt}%
\definecolor{currentstroke}{rgb}{0.000000,0.000000,0.000000}%
\pgfsetstrokecolor{currentstroke}%
\pgfsetdash{}{0pt}%
\pgfusepath{stroke,fill}%
\end{pgfscope}%
\begin{pgfscope}%
\pgfpathrectangle{\pgfqpoint{0.636356in}{0.440955in}}{\pgfqpoint{3.020670in}{0.290057in}} %
\pgfusepath{clip}%
\pgfsetbuttcap%
\pgfsetmiterjoin%
\definecolor{currentfill}{rgb}{0.333333,0.333333,0.333333}%
\pgfsetfillcolor{currentfill}%
\pgfsetlinewidth{0.501875pt}%
\definecolor{currentstroke}{rgb}{0.000000,0.000000,0.000000}%
\pgfsetstrokecolor{currentstroke}%
\pgfsetdash{}{0pt}%
\pgfusepath{stroke,fill}%
\end{pgfscope}%
\begin{pgfscope}%
\pgfpathrectangle{\pgfqpoint{0.636356in}{0.440955in}}{\pgfqpoint{3.020670in}{0.290057in}} %
\pgfusepath{clip}%
\pgfsetbuttcap%
\pgfsetmiterjoin%
\definecolor{currentfill}{rgb}{0.333333,0.333333,0.333333}%
\pgfsetfillcolor{currentfill}%
\pgfsetlinewidth{0.501875pt}%
\definecolor{currentstroke}{rgb}{0.000000,0.000000,0.000000}%
\pgfsetstrokecolor{currentstroke}%
\pgfsetdash{}{0pt}%
\pgfusepath{stroke,fill}%
\end{pgfscope}%
\begin{pgfscope}%
\pgfpathrectangle{\pgfqpoint{0.636356in}{0.440955in}}{\pgfqpoint{3.020670in}{0.290057in}} %
\pgfusepath{clip}%
\pgfsetbuttcap%
\pgfsetmiterjoin%
\definecolor{currentfill}{rgb}{0.800000,0.266667,0.266667}%
\pgfsetfillcolor{currentfill}%
\pgfsetlinewidth{0.501875pt}%
\definecolor{currentstroke}{rgb}{0.000000,0.000000,0.000000}%
\pgfsetstrokecolor{currentstroke}%
\pgfsetdash{}{0pt}%
\pgfpathmoveto{\pgfqpoint{1.240490in}{-0.873138in}}%
\pgfpathmoveto{\pgfqpoint{1.270696in}{0.430955in}}%
\pgfpathlineto{\pgfqpoint{1.270696in}{0.585984in}}%
\pgfpathlineto{\pgfqpoint{1.240490in}{0.585984in}}%
\pgfpathlineto{\pgfqpoint{1.240490in}{0.430955in}}%
\pgfusepath{stroke,fill}%
\end{pgfscope}%
\begin{pgfscope}%
\pgfpathrectangle{\pgfqpoint{0.636356in}{0.440955in}}{\pgfqpoint{3.020670in}{0.290057in}} %
\pgfusepath{clip}%
\pgfsetbuttcap%
\pgfsetmiterjoin%
\definecolor{currentfill}{rgb}{0.333333,0.333333,0.333333}%
\pgfsetfillcolor{currentfill}%
\pgfsetlinewidth{0.501875pt}%
\definecolor{currentstroke}{rgb}{0.000000,0.000000,0.000000}%
\pgfsetstrokecolor{currentstroke}%
\pgfsetdash{}{0pt}%
\pgfpathmoveto{\pgfqpoint{1.270696in}{0.585984in}}%
\pgfpathlineto{\pgfqpoint{1.300903in}{0.585984in}}%
\pgfpathlineto{\pgfqpoint{1.300903in}{0.670158in}}%
\pgfpathlineto{\pgfqpoint{1.270696in}{0.670158in}}%
\pgfpathlineto{\pgfqpoint{1.270696in}{0.585984in}}%
\pgfusepath{stroke,fill}%
\end{pgfscope}%
\begin{pgfscope}%
\pgfpathrectangle{\pgfqpoint{0.636356in}{0.440955in}}{\pgfqpoint{3.020670in}{0.290057in}} %
\pgfusepath{clip}%
\pgfsetbuttcap%
\pgfsetmiterjoin%
\definecolor{currentfill}{rgb}{0.333333,0.333333,0.333333}%
\pgfsetfillcolor{currentfill}%
\pgfsetlinewidth{0.501875pt}%
\definecolor{currentstroke}{rgb}{0.000000,0.000000,0.000000}%
\pgfsetstrokecolor{currentstroke}%
\pgfsetdash{}{0pt}%
\pgfpathmoveto{\pgfqpoint{1.300903in}{0.546744in}}%
\pgfpathlineto{\pgfqpoint{1.331110in}{0.546744in}}%
\pgfpathlineto{\pgfqpoint{1.331110in}{0.585984in}}%
\pgfpathlineto{\pgfqpoint{1.300903in}{0.585984in}}%
\pgfpathlineto{\pgfqpoint{1.300903in}{0.546744in}}%
\pgfusepath{stroke,fill}%
\end{pgfscope}%
\begin{pgfscope}%
\pgfpathrectangle{\pgfqpoint{0.636356in}{0.440955in}}{\pgfqpoint{3.020670in}{0.290057in}} %
\pgfusepath{clip}%
\pgfsetbuttcap%
\pgfsetmiterjoin%
\definecolor{currentfill}{rgb}{0.333333,0.333333,0.333333}%
\pgfsetfillcolor{currentfill}%
\pgfsetlinewidth{0.501875pt}%
\definecolor{currentstroke}{rgb}{0.000000,0.000000,0.000000}%
\pgfsetstrokecolor{currentstroke}%
\pgfsetdash{}{0pt}%
\pgfpathmoveto{\pgfqpoint{1.331110in}{0.585984in}}%
\pgfpathlineto{\pgfqpoint{1.361317in}{0.585984in}}%
\pgfpathlineto{\pgfqpoint{1.361317in}{0.636362in}}%
\pgfpathlineto{\pgfqpoint{1.331110in}{0.636362in}}%
\pgfpathlineto{\pgfqpoint{1.331110in}{0.585984in}}%
\pgfusepath{stroke,fill}%
\end{pgfscope}%
\begin{pgfscope}%
\pgfpathrectangle{\pgfqpoint{0.636356in}{0.440955in}}{\pgfqpoint{3.020670in}{0.290057in}} %
\pgfusepath{clip}%
\pgfsetbuttcap%
\pgfsetmiterjoin%
\definecolor{currentfill}{rgb}{0.333333,0.333333,0.333333}%
\pgfsetfillcolor{currentfill}%
\pgfsetlinewidth{0.501875pt}%
\definecolor{currentstroke}{rgb}{0.000000,0.000000,0.000000}%
\pgfsetstrokecolor{currentstroke}%
\pgfsetdash{}{0pt}%
\pgfpathmoveto{\pgfqpoint{1.361317in}{0.585984in}}%
\pgfpathlineto{\pgfqpoint{1.391523in}{0.585984in}}%
\pgfpathlineto{\pgfqpoint{1.391523in}{0.658945in}}%
\pgfpathlineto{\pgfqpoint{1.361317in}{0.658945in}}%
\pgfpathlineto{\pgfqpoint{1.361317in}{0.585984in}}%
\pgfusepath{stroke,fill}%
\end{pgfscope}%
\begin{pgfscope}%
\pgfpathrectangle{\pgfqpoint{0.636356in}{0.440955in}}{\pgfqpoint{3.020670in}{0.290057in}} %
\pgfusepath{clip}%
\pgfsetbuttcap%
\pgfsetmiterjoin%
\definecolor{currentfill}{rgb}{0.333333,0.333333,0.333333}%
\pgfsetfillcolor{currentfill}%
\pgfsetlinewidth{0.501875pt}%
\definecolor{currentstroke}{rgb}{0.000000,0.000000,0.000000}%
\pgfsetstrokecolor{currentstroke}%
\pgfsetdash{}{0pt}%
\pgfpathmoveto{\pgfqpoint{1.391523in}{0.585984in}}%
\pgfpathlineto{\pgfqpoint{1.421730in}{0.585984in}}%
\pgfpathlineto{\pgfqpoint{1.421730in}{0.666641in}}%
\pgfpathlineto{\pgfqpoint{1.391523in}{0.666641in}}%
\pgfpathlineto{\pgfqpoint{1.391523in}{0.585984in}}%
\pgfusepath{stroke,fill}%
\end{pgfscope}%
\begin{pgfscope}%
\pgfpathrectangle{\pgfqpoint{0.636356in}{0.440955in}}{\pgfqpoint{3.020670in}{0.290057in}} %
\pgfusepath{clip}%
\pgfsetbuttcap%
\pgfsetmiterjoin%
\definecolor{currentfill}{rgb}{0.333333,0.333333,0.333333}%
\pgfsetfillcolor{currentfill}%
\pgfsetlinewidth{0.501875pt}%
\definecolor{currentstroke}{rgb}{0.000000,0.000000,0.000000}%
\pgfsetstrokecolor{currentstroke}%
\pgfsetdash{}{0pt}%
\pgfpathmoveto{\pgfqpoint{1.421730in}{0.585984in}}%
\pgfpathlineto{\pgfqpoint{1.451937in}{0.585984in}}%
\pgfpathlineto{\pgfqpoint{1.451937in}{0.628792in}}%
\pgfpathlineto{\pgfqpoint{1.421730in}{0.628792in}}%
\pgfpathlineto{\pgfqpoint{1.421730in}{0.585984in}}%
\pgfusepath{stroke,fill}%
\end{pgfscope}%
\begin{pgfscope}%
\pgfpathrectangle{\pgfqpoint{0.636356in}{0.440955in}}{\pgfqpoint{3.020670in}{0.290057in}} %
\pgfusepath{clip}%
\pgfsetbuttcap%
\pgfsetmiterjoin%
\definecolor{currentfill}{rgb}{0.333333,0.333333,0.333333}%
\pgfsetfillcolor{currentfill}%
\pgfsetlinewidth{0.501875pt}%
\definecolor{currentstroke}{rgb}{0.000000,0.000000,0.000000}%
\pgfsetstrokecolor{currentstroke}%
\pgfsetdash{}{0pt}%
\pgfpathmoveto{\pgfqpoint{1.451937in}{0.585984in}}%
\pgfpathlineto{\pgfqpoint{1.482143in}{0.585984in}}%
\pgfpathlineto{\pgfqpoint{1.482143in}{0.667535in}}%
\pgfpathlineto{\pgfqpoint{1.451937in}{0.667535in}}%
\pgfpathlineto{\pgfqpoint{1.451937in}{0.585984in}}%
\pgfusepath{stroke,fill}%
\end{pgfscope}%
\begin{pgfscope}%
\pgfpathrectangle{\pgfqpoint{0.636356in}{0.440955in}}{\pgfqpoint{3.020670in}{0.290057in}} %
\pgfusepath{clip}%
\pgfsetbuttcap%
\pgfsetmiterjoin%
\definecolor{currentfill}{rgb}{0.333333,0.333333,0.333333}%
\pgfsetfillcolor{currentfill}%
\pgfsetlinewidth{0.501875pt}%
\definecolor{currentstroke}{rgb}{0.000000,0.000000,0.000000}%
\pgfsetstrokecolor{currentstroke}%
\pgfsetdash{}{0pt}%
\pgfpathmoveto{\pgfqpoint{1.482143in}{0.585984in}}%
\pgfpathlineto{\pgfqpoint{1.512350in}{0.585984in}}%
\pgfpathlineto{\pgfqpoint{1.512350in}{0.711110in}}%
\pgfpathlineto{\pgfqpoint{1.482143in}{0.711110in}}%
\pgfpathlineto{\pgfqpoint{1.482143in}{0.585984in}}%
\pgfusepath{stroke,fill}%
\end{pgfscope}%
\begin{pgfscope}%
\pgfpathrectangle{\pgfqpoint{0.636356in}{0.440955in}}{\pgfqpoint{3.020670in}{0.290057in}} %
\pgfusepath{clip}%
\pgfsetbuttcap%
\pgfsetmiterjoin%
\definecolor{currentfill}{rgb}{0.333333,0.333333,0.333333}%
\pgfsetfillcolor{currentfill}%
\pgfsetlinewidth{0.501875pt}%
\definecolor{currentstroke}{rgb}{0.000000,0.000000,0.000000}%
\pgfsetstrokecolor{currentstroke}%
\pgfsetdash{}{0pt}%
\pgfpathmoveto{\pgfqpoint{1.512350in}{0.585984in}}%
\pgfpathlineto{\pgfqpoint{1.542557in}{0.585984in}}%
\pgfpathlineto{\pgfqpoint{1.542557in}{0.703320in}}%
\pgfpathlineto{\pgfqpoint{1.512350in}{0.703320in}}%
\pgfpathlineto{\pgfqpoint{1.512350in}{0.585984in}}%
\pgfusepath{stroke,fill}%
\end{pgfscope}%
\begin{pgfscope}%
\pgfpathrectangle{\pgfqpoint{0.636356in}{0.440955in}}{\pgfqpoint{3.020670in}{0.290057in}} %
\pgfusepath{clip}%
\pgfsetbuttcap%
\pgfsetmiterjoin%
\definecolor{currentfill}{rgb}{0.333333,0.333333,0.333333}%
\pgfsetfillcolor{currentfill}%
\pgfsetlinewidth{0.501875pt}%
\definecolor{currentstroke}{rgb}{0.000000,0.000000,0.000000}%
\pgfsetstrokecolor{currentstroke}%
\pgfsetdash{}{0pt}%
\pgfpathmoveto{\pgfqpoint{1.542557in}{0.585984in}}%
\pgfpathlineto{\pgfqpoint{1.572763in}{0.585984in}}%
\pgfpathlineto{\pgfqpoint{1.572763in}{0.634613in}}%
\pgfpathlineto{\pgfqpoint{1.542557in}{0.634613in}}%
\pgfpathlineto{\pgfqpoint{1.542557in}{0.585984in}}%
\pgfusepath{stroke,fill}%
\end{pgfscope}%
\begin{pgfscope}%
\pgfpathrectangle{\pgfqpoint{0.636356in}{0.440955in}}{\pgfqpoint{3.020670in}{0.290057in}} %
\pgfusepath{clip}%
\pgfsetbuttcap%
\pgfsetmiterjoin%
\definecolor{currentfill}{rgb}{0.333333,0.333333,0.333333}%
\pgfsetfillcolor{currentfill}%
\pgfsetlinewidth{0.501875pt}%
\definecolor{currentstroke}{rgb}{0.000000,0.000000,0.000000}%
\pgfsetstrokecolor{currentstroke}%
\pgfsetdash{}{0pt}%
\pgfpathmoveto{\pgfqpoint{1.572763in}{0.585984in}}%
\pgfpathlineto{\pgfqpoint{1.602970in}{0.585984in}}%
\pgfpathlineto{\pgfqpoint{1.602970in}{0.657366in}}%
\pgfpathlineto{\pgfqpoint{1.572763in}{0.657366in}}%
\pgfpathlineto{\pgfqpoint{1.572763in}{0.585984in}}%
\pgfusepath{stroke,fill}%
\end{pgfscope}%
\begin{pgfscope}%
\pgfpathrectangle{\pgfqpoint{0.636356in}{0.440955in}}{\pgfqpoint{3.020670in}{0.290057in}} %
\pgfusepath{clip}%
\pgfsetbuttcap%
\pgfsetmiterjoin%
\definecolor{currentfill}{rgb}{0.333333,0.333333,0.333333}%
\pgfsetfillcolor{currentfill}%
\pgfsetlinewidth{0.501875pt}%
\definecolor{currentstroke}{rgb}{0.000000,0.000000,0.000000}%
\pgfsetstrokecolor{currentstroke}%
\pgfsetdash{}{0pt}%
\pgfpathmoveto{\pgfqpoint{1.602970in}{0.585984in}}%
\pgfpathlineto{\pgfqpoint{1.633177in}{0.585984in}}%
\pgfpathlineto{\pgfqpoint{1.633177in}{0.593303in}}%
\pgfpathlineto{\pgfqpoint{1.602970in}{0.593303in}}%
\pgfpathlineto{\pgfqpoint{1.602970in}{0.585984in}}%
\pgfusepath{stroke,fill}%
\end{pgfscope}%
\begin{pgfscope}%
\pgfpathrectangle{\pgfqpoint{0.636356in}{0.440955in}}{\pgfqpoint{3.020670in}{0.290057in}} %
\pgfusepath{clip}%
\pgfsetbuttcap%
\pgfsetmiterjoin%
\definecolor{currentfill}{rgb}{0.333333,0.333333,0.333333}%
\pgfsetfillcolor{currentfill}%
\pgfsetlinewidth{0.501875pt}%
\definecolor{currentstroke}{rgb}{0.000000,0.000000,0.000000}%
\pgfsetstrokecolor{currentstroke}%
\pgfsetdash{}{0pt}%
\pgfpathmoveto{\pgfqpoint{1.633177in}{0.585984in}}%
\pgfpathlineto{\pgfqpoint{1.663384in}{0.585984in}}%
\pgfpathlineto{\pgfqpoint{1.663384in}{0.686955in}}%
\pgfpathlineto{\pgfqpoint{1.633177in}{0.686955in}}%
\pgfpathlineto{\pgfqpoint{1.633177in}{0.585984in}}%
\pgfusepath{stroke,fill}%
\end{pgfscope}%
\begin{pgfscope}%
\pgfpathrectangle{\pgfqpoint{0.636356in}{0.440955in}}{\pgfqpoint{3.020670in}{0.290057in}} %
\pgfusepath{clip}%
\pgfsetbuttcap%
\pgfsetmiterjoin%
\definecolor{currentfill}{rgb}{0.333333,0.333333,0.333333}%
\pgfsetfillcolor{currentfill}%
\pgfsetlinewidth{0.501875pt}%
\definecolor{currentstroke}{rgb}{0.000000,0.000000,0.000000}%
\pgfsetstrokecolor{currentstroke}%
\pgfsetdash{}{0pt}%
\pgfpathmoveto{\pgfqpoint{1.663384in}{0.513174in}}%
\pgfpathlineto{\pgfqpoint{1.693590in}{0.513174in}}%
\pgfpathlineto{\pgfqpoint{1.693590in}{0.585984in}}%
\pgfpathlineto{\pgfqpoint{1.663384in}{0.585984in}}%
\pgfpathlineto{\pgfqpoint{1.663384in}{0.513174in}}%
\pgfusepath{stroke,fill}%
\end{pgfscope}%
\begin{pgfscope}%
\pgfpathrectangle{\pgfqpoint{0.636356in}{0.440955in}}{\pgfqpoint{3.020670in}{0.290057in}} %
\pgfusepath{clip}%
\pgfsetbuttcap%
\pgfsetmiterjoin%
\definecolor{currentfill}{rgb}{0.333333,0.333333,0.333333}%
\pgfsetfillcolor{currentfill}%
\pgfsetlinewidth{0.501875pt}%
\definecolor{currentstroke}{rgb}{0.000000,0.000000,0.000000}%
\pgfsetstrokecolor{currentstroke}%
\pgfsetdash{}{0pt}%
\pgfpathmoveto{\pgfqpoint{1.693590in}{0.524645in}}%
\pgfpathlineto{\pgfqpoint{1.723797in}{0.524645in}}%
\pgfpathlineto{\pgfqpoint{1.723797in}{0.585984in}}%
\pgfpathlineto{\pgfqpoint{1.693590in}{0.585984in}}%
\pgfpathlineto{\pgfqpoint{1.693590in}{0.524645in}}%
\pgfusepath{stroke,fill}%
\end{pgfscope}%
\begin{pgfscope}%
\pgfpathrectangle{\pgfqpoint{0.636356in}{0.440955in}}{\pgfqpoint{3.020670in}{0.290057in}} %
\pgfusepath{clip}%
\pgfsetbuttcap%
\pgfsetmiterjoin%
\definecolor{currentfill}{rgb}{0.333333,0.333333,0.333333}%
\pgfsetfillcolor{currentfill}%
\pgfsetlinewidth{0.501875pt}%
\definecolor{currentstroke}{rgb}{0.000000,0.000000,0.000000}%
\pgfsetstrokecolor{currentstroke}%
\pgfsetdash{}{0pt}%
\pgfpathmoveto{\pgfqpoint{1.723797in}{0.582044in}}%
\pgfpathlineto{\pgfqpoint{1.754004in}{0.582044in}}%
\pgfpathlineto{\pgfqpoint{1.754004in}{0.585984in}}%
\pgfpathlineto{\pgfqpoint{1.723797in}{0.585984in}}%
\pgfpathlineto{\pgfqpoint{1.723797in}{0.582044in}}%
\pgfusepath{stroke,fill}%
\end{pgfscope}%
\begin{pgfscope}%
\pgfpathrectangle{\pgfqpoint{0.636356in}{0.440955in}}{\pgfqpoint{3.020670in}{0.290057in}} %
\pgfusepath{clip}%
\pgfsetbuttcap%
\pgfsetmiterjoin%
\definecolor{currentfill}{rgb}{0.333333,0.333333,0.333333}%
\pgfsetfillcolor{currentfill}%
\pgfsetlinewidth{0.501875pt}%
\definecolor{currentstroke}{rgb}{0.000000,0.000000,0.000000}%
\pgfsetstrokecolor{currentstroke}%
\pgfsetdash{}{0pt}%
\pgfpathmoveto{\pgfqpoint{1.754004in}{0.507937in}}%
\pgfpathlineto{\pgfqpoint{1.784210in}{0.507937in}}%
\pgfpathlineto{\pgfqpoint{1.784210in}{0.585984in}}%
\pgfpathlineto{\pgfqpoint{1.754004in}{0.585984in}}%
\pgfpathlineto{\pgfqpoint{1.754004in}{0.507937in}}%
\pgfusepath{stroke,fill}%
\end{pgfscope}%
\begin{pgfscope}%
\pgfpathrectangle{\pgfqpoint{0.636356in}{0.440955in}}{\pgfqpoint{3.020670in}{0.290057in}} %
\pgfusepath{clip}%
\pgfsetbuttcap%
\pgfsetmiterjoin%
\definecolor{currentfill}{rgb}{0.333333,0.333333,0.333333}%
\pgfsetfillcolor{currentfill}%
\pgfsetlinewidth{0.501875pt}%
\definecolor{currentstroke}{rgb}{0.000000,0.000000,0.000000}%
\pgfsetstrokecolor{currentstroke}%
\pgfsetdash{}{0pt}%
\pgfpathmoveto{\pgfqpoint{1.784210in}{0.508408in}}%
\pgfpathlineto{\pgfqpoint{1.814417in}{0.508408in}}%
\pgfpathlineto{\pgfqpoint{1.814417in}{0.585984in}}%
\pgfpathlineto{\pgfqpoint{1.784210in}{0.585984in}}%
\pgfpathlineto{\pgfqpoint{1.784210in}{0.508408in}}%
\pgfusepath{stroke,fill}%
\end{pgfscope}%
\begin{pgfscope}%
\pgfpathrectangle{\pgfqpoint{0.636356in}{0.440955in}}{\pgfqpoint{3.020670in}{0.290057in}} %
\pgfusepath{clip}%
\pgfsetbuttcap%
\pgfsetmiterjoin%
\definecolor{currentfill}{rgb}{0.333333,0.333333,0.333333}%
\pgfsetfillcolor{currentfill}%
\pgfsetlinewidth{0.501875pt}%
\definecolor{currentstroke}{rgb}{0.000000,0.000000,0.000000}%
\pgfsetstrokecolor{currentstroke}%
\pgfsetdash{}{0pt}%
\pgfpathmoveto{\pgfqpoint{1.814417in}{0.514571in}}%
\pgfpathlineto{\pgfqpoint{1.844624in}{0.514571in}}%
\pgfpathlineto{\pgfqpoint{1.844624in}{0.585984in}}%
\pgfpathlineto{\pgfqpoint{1.814417in}{0.585984in}}%
\pgfpathlineto{\pgfqpoint{1.814417in}{0.514571in}}%
\pgfusepath{stroke,fill}%
\end{pgfscope}%
\begin{pgfscope}%
\pgfpathrectangle{\pgfqpoint{0.636356in}{0.440955in}}{\pgfqpoint{3.020670in}{0.290057in}} %
\pgfusepath{clip}%
\pgfsetbuttcap%
\pgfsetmiterjoin%
\definecolor{currentfill}{rgb}{0.333333,0.333333,0.333333}%
\pgfsetfillcolor{currentfill}%
\pgfsetlinewidth{0.501875pt}%
\definecolor{currentstroke}{rgb}{0.000000,0.000000,0.000000}%
\pgfsetstrokecolor{currentstroke}%
\pgfsetdash{}{0pt}%
\pgfpathmoveto{\pgfqpoint{1.844624in}{0.585984in}}%
\pgfpathlineto{\pgfqpoint{1.874830in}{0.585984in}}%
\pgfpathlineto{\pgfqpoint{1.874830in}{0.596873in}}%
\pgfpathlineto{\pgfqpoint{1.844624in}{0.596873in}}%
\pgfpathlineto{\pgfqpoint{1.844624in}{0.585984in}}%
\pgfusepath{stroke,fill}%
\end{pgfscope}%
\begin{pgfscope}%
\pgfpathrectangle{\pgfqpoint{0.636356in}{0.440955in}}{\pgfqpoint{3.020670in}{0.290057in}} %
\pgfusepath{clip}%
\pgfsetbuttcap%
\pgfsetmiterjoin%
\definecolor{currentfill}{rgb}{0.333333,0.333333,0.333333}%
\pgfsetfillcolor{currentfill}%
\pgfsetlinewidth{0.501875pt}%
\definecolor{currentstroke}{rgb}{0.000000,0.000000,0.000000}%
\pgfsetstrokecolor{currentstroke}%
\pgfsetdash{}{0pt}%
\pgfpathmoveto{\pgfqpoint{1.874830in}{0.531469in}}%
\pgfpathlineto{\pgfqpoint{1.905037in}{0.531469in}}%
\pgfpathlineto{\pgfqpoint{1.905037in}{0.585984in}}%
\pgfpathlineto{\pgfqpoint{1.874830in}{0.585984in}}%
\pgfpathlineto{\pgfqpoint{1.874830in}{0.531469in}}%
\pgfusepath{stroke,fill}%
\end{pgfscope}%
\begin{pgfscope}%
\pgfpathrectangle{\pgfqpoint{0.636356in}{0.440955in}}{\pgfqpoint{3.020670in}{0.290057in}} %
\pgfusepath{clip}%
\pgfsetbuttcap%
\pgfsetmiterjoin%
\definecolor{currentfill}{rgb}{0.333333,0.333333,0.333333}%
\pgfsetfillcolor{currentfill}%
\pgfsetlinewidth{0.501875pt}%
\definecolor{currentstroke}{rgb}{0.000000,0.000000,0.000000}%
\pgfsetstrokecolor{currentstroke}%
\pgfsetdash{}{0pt}%
\pgfpathmoveto{\pgfqpoint{1.905037in}{0.550039in}}%
\pgfpathlineto{\pgfqpoint{1.935244in}{0.550039in}}%
\pgfpathlineto{\pgfqpoint{1.935244in}{0.585984in}}%
\pgfpathlineto{\pgfqpoint{1.905037in}{0.585984in}}%
\pgfpathlineto{\pgfqpoint{1.905037in}{0.550039in}}%
\pgfusepath{stroke,fill}%
\end{pgfscope}%
\begin{pgfscope}%
\pgfpathrectangle{\pgfqpoint{0.636356in}{0.440955in}}{\pgfqpoint{3.020670in}{0.290057in}} %
\pgfusepath{clip}%
\pgfsetbuttcap%
\pgfsetmiterjoin%
\definecolor{currentfill}{rgb}{0.333333,0.333333,0.333333}%
\pgfsetfillcolor{currentfill}%
\pgfsetlinewidth{0.501875pt}%
\definecolor{currentstroke}{rgb}{0.000000,0.000000,0.000000}%
\pgfsetstrokecolor{currentstroke}%
\pgfsetdash{}{0pt}%
\pgfpathmoveto{\pgfqpoint{1.935244in}{0.576489in}}%
\pgfpathlineto{\pgfqpoint{1.965450in}{0.576489in}}%
\pgfpathlineto{\pgfqpoint{1.965450in}{0.585984in}}%
\pgfpathlineto{\pgfqpoint{1.935244in}{0.585984in}}%
\pgfpathlineto{\pgfqpoint{1.935244in}{0.576489in}}%
\pgfusepath{stroke,fill}%
\end{pgfscope}%
\begin{pgfscope}%
\pgfpathrectangle{\pgfqpoint{0.636356in}{0.440955in}}{\pgfqpoint{3.020670in}{0.290057in}} %
\pgfusepath{clip}%
\pgfsetbuttcap%
\pgfsetmiterjoin%
\definecolor{currentfill}{rgb}{0.333333,0.333333,0.333333}%
\pgfsetfillcolor{currentfill}%
\pgfsetlinewidth{0.501875pt}%
\definecolor{currentstroke}{rgb}{0.000000,0.000000,0.000000}%
\pgfsetstrokecolor{currentstroke}%
\pgfsetdash{}{0pt}%
\pgfpathmoveto{\pgfqpoint{1.965450in}{0.527133in}}%
\pgfpathlineto{\pgfqpoint{1.995657in}{0.527133in}}%
\pgfpathlineto{\pgfqpoint{1.995657in}{0.585984in}}%
\pgfpathlineto{\pgfqpoint{1.965450in}{0.585984in}}%
\pgfpathlineto{\pgfqpoint{1.965450in}{0.527133in}}%
\pgfusepath{stroke,fill}%
\end{pgfscope}%
\begin{pgfscope}%
\pgfpathrectangle{\pgfqpoint{0.636356in}{0.440955in}}{\pgfqpoint{3.020670in}{0.290057in}} %
\pgfusepath{clip}%
\pgfsetbuttcap%
\pgfsetmiterjoin%
\definecolor{currentfill}{rgb}{0.333333,0.333333,0.333333}%
\pgfsetfillcolor{currentfill}%
\pgfsetlinewidth{0.501875pt}%
\definecolor{currentstroke}{rgb}{0.000000,0.000000,0.000000}%
\pgfsetstrokecolor{currentstroke}%
\pgfsetdash{}{0pt}%
\pgfpathmoveto{\pgfqpoint{1.995657in}{0.472382in}}%
\pgfpathlineto{\pgfqpoint{2.025864in}{0.472382in}}%
\pgfpathlineto{\pgfqpoint{2.025864in}{0.585984in}}%
\pgfpathlineto{\pgfqpoint{1.995657in}{0.585984in}}%
\pgfpathlineto{\pgfqpoint{1.995657in}{0.472382in}}%
\pgfusepath{stroke,fill}%
\end{pgfscope}%
\begin{pgfscope}%
\pgfpathrectangle{\pgfqpoint{0.636356in}{0.440955in}}{\pgfqpoint{3.020670in}{0.290057in}} %
\pgfusepath{clip}%
\pgfsetbuttcap%
\pgfsetmiterjoin%
\definecolor{currentfill}{rgb}{0.333333,0.333333,0.333333}%
\pgfsetfillcolor{currentfill}%
\pgfsetlinewidth{0.501875pt}%
\definecolor{currentstroke}{rgb}{0.000000,0.000000,0.000000}%
\pgfsetstrokecolor{currentstroke}%
\pgfsetdash{}{0pt}%
\pgfpathmoveto{\pgfqpoint{2.025864in}{0.585984in}}%
\pgfpathlineto{\pgfqpoint{2.056071in}{0.585984in}}%
\pgfpathlineto{\pgfqpoint{2.056071in}{0.589616in}}%
\pgfpathlineto{\pgfqpoint{2.025864in}{0.589616in}}%
\pgfpathlineto{\pgfqpoint{2.025864in}{0.585984in}}%
\pgfusepath{stroke,fill}%
\end{pgfscope}%
\begin{pgfscope}%
\pgfpathrectangle{\pgfqpoint{0.636356in}{0.440955in}}{\pgfqpoint{3.020670in}{0.290057in}} %
\pgfusepath{clip}%
\pgfsetbuttcap%
\pgfsetmiterjoin%
\definecolor{currentfill}{rgb}{0.333333,0.333333,0.333333}%
\pgfsetfillcolor{currentfill}%
\pgfsetlinewidth{0.501875pt}%
\definecolor{currentstroke}{rgb}{0.000000,0.000000,0.000000}%
\pgfsetstrokecolor{currentstroke}%
\pgfsetdash{}{0pt}%
\pgfpathmoveto{\pgfqpoint{2.056071in}{0.585984in}}%
\pgfpathlineto{\pgfqpoint{2.086277in}{0.585984in}}%
\pgfpathlineto{\pgfqpoint{2.086277in}{0.586078in}}%
\pgfpathlineto{\pgfqpoint{2.056071in}{0.586078in}}%
\pgfpathlineto{\pgfqpoint{2.056071in}{0.585984in}}%
\pgfusepath{stroke,fill}%
\end{pgfscope}%
\begin{pgfscope}%
\pgfpathrectangle{\pgfqpoint{0.636356in}{0.440955in}}{\pgfqpoint{3.020670in}{0.290057in}} %
\pgfusepath{clip}%
\pgfsetbuttcap%
\pgfsetmiterjoin%
\definecolor{currentfill}{rgb}{0.333333,0.333333,0.333333}%
\pgfsetfillcolor{currentfill}%
\pgfsetlinewidth{0.501875pt}%
\definecolor{currentstroke}{rgb}{0.000000,0.000000,0.000000}%
\pgfsetstrokecolor{currentstroke}%
\pgfsetdash{}{0pt}%
\pgfpathmoveto{\pgfqpoint{2.086277in}{0.525816in}}%
\pgfpathlineto{\pgfqpoint{2.116484in}{0.525816in}}%
\pgfpathlineto{\pgfqpoint{2.116484in}{0.585984in}}%
\pgfpathlineto{\pgfqpoint{2.086277in}{0.585984in}}%
\pgfpathlineto{\pgfqpoint{2.086277in}{0.525816in}}%
\pgfusepath{stroke,fill}%
\end{pgfscope}%
\begin{pgfscope}%
\pgfpathrectangle{\pgfqpoint{0.636356in}{0.440955in}}{\pgfqpoint{3.020670in}{0.290057in}} %
\pgfusepath{clip}%
\pgfsetbuttcap%
\pgfsetmiterjoin%
\definecolor{currentfill}{rgb}{0.333333,0.333333,0.333333}%
\pgfsetfillcolor{currentfill}%
\pgfsetlinewidth{0.501875pt}%
\definecolor{currentstroke}{rgb}{0.000000,0.000000,0.000000}%
\pgfsetstrokecolor{currentstroke}%
\pgfsetdash{}{0pt}%
\pgfpathmoveto{\pgfqpoint{2.116484in}{0.492893in}}%
\pgfpathlineto{\pgfqpoint{2.146691in}{0.492893in}}%
\pgfpathlineto{\pgfqpoint{2.146691in}{0.585984in}}%
\pgfpathlineto{\pgfqpoint{2.116484in}{0.585984in}}%
\pgfpathlineto{\pgfqpoint{2.116484in}{0.492893in}}%
\pgfusepath{stroke,fill}%
\end{pgfscope}%
\begin{pgfscope}%
\pgfpathrectangle{\pgfqpoint{0.636356in}{0.440955in}}{\pgfqpoint{3.020670in}{0.290057in}} %
\pgfusepath{clip}%
\pgfsetbuttcap%
\pgfsetmiterjoin%
\definecolor{currentfill}{rgb}{0.333333,0.333333,0.333333}%
\pgfsetfillcolor{currentfill}%
\pgfsetlinewidth{0.501875pt}%
\definecolor{currentstroke}{rgb}{0.000000,0.000000,0.000000}%
\pgfsetstrokecolor{currentstroke}%
\pgfsetdash{}{0pt}%
\pgfpathmoveto{\pgfqpoint{2.146691in}{0.562966in}}%
\pgfpathlineto{\pgfqpoint{2.176897in}{0.562966in}}%
\pgfpathlineto{\pgfqpoint{2.176897in}{0.585984in}}%
\pgfpathlineto{\pgfqpoint{2.146691in}{0.585984in}}%
\pgfpathlineto{\pgfqpoint{2.146691in}{0.562966in}}%
\pgfusepath{stroke,fill}%
\end{pgfscope}%
\begin{pgfscope}%
\pgfpathrectangle{\pgfqpoint{0.636356in}{0.440955in}}{\pgfqpoint{3.020670in}{0.290057in}} %
\pgfusepath{clip}%
\pgfsetbuttcap%
\pgfsetmiterjoin%
\definecolor{currentfill}{rgb}{0.333333,0.333333,0.333333}%
\pgfsetfillcolor{currentfill}%
\pgfsetlinewidth{0.501875pt}%
\definecolor{currentstroke}{rgb}{0.000000,0.000000,0.000000}%
\pgfsetstrokecolor{currentstroke}%
\pgfsetdash{}{0pt}%
\pgfpathmoveto{\pgfqpoint{2.176897in}{0.452253in}}%
\pgfpathlineto{\pgfqpoint{2.207104in}{0.452253in}}%
\pgfpathlineto{\pgfqpoint{2.207104in}{0.585984in}}%
\pgfpathlineto{\pgfqpoint{2.176897in}{0.585984in}}%
\pgfpathlineto{\pgfqpoint{2.176897in}{0.452253in}}%
\pgfusepath{stroke,fill}%
\end{pgfscope}%
\begin{pgfscope}%
\pgfpathrectangle{\pgfqpoint{0.636356in}{0.440955in}}{\pgfqpoint{3.020670in}{0.290057in}} %
\pgfusepath{clip}%
\pgfsetbuttcap%
\pgfsetmiterjoin%
\definecolor{currentfill}{rgb}{0.333333,0.333333,0.333333}%
\pgfsetfillcolor{currentfill}%
\pgfsetlinewidth{0.501875pt}%
\definecolor{currentstroke}{rgb}{0.000000,0.000000,0.000000}%
\pgfsetstrokecolor{currentstroke}%
\pgfsetdash{}{0pt}%
\pgfpathmoveto{\pgfqpoint{2.207104in}{0.530220in}}%
\pgfpathlineto{\pgfqpoint{2.237311in}{0.530220in}}%
\pgfpathlineto{\pgfqpoint{2.237311in}{0.585984in}}%
\pgfpathlineto{\pgfqpoint{2.207104in}{0.585984in}}%
\pgfpathlineto{\pgfqpoint{2.207104in}{0.530220in}}%
\pgfusepath{stroke,fill}%
\end{pgfscope}%
\begin{pgfscope}%
\pgfpathrectangle{\pgfqpoint{0.636356in}{0.440955in}}{\pgfqpoint{3.020670in}{0.290057in}} %
\pgfusepath{clip}%
\pgfsetbuttcap%
\pgfsetmiterjoin%
\definecolor{currentfill}{rgb}{0.333333,0.333333,0.333333}%
\pgfsetfillcolor{currentfill}%
\pgfsetlinewidth{0.501875pt}%
\definecolor{currentstroke}{rgb}{0.000000,0.000000,0.000000}%
\pgfsetstrokecolor{currentstroke}%
\pgfsetdash{}{0pt}%
\pgfpathmoveto{\pgfqpoint{2.237311in}{0.508919in}}%
\pgfpathlineto{\pgfqpoint{2.267517in}{0.508919in}}%
\pgfpathlineto{\pgfqpoint{2.267517in}{0.585984in}}%
\pgfpathlineto{\pgfqpoint{2.237311in}{0.585984in}}%
\pgfpathlineto{\pgfqpoint{2.237311in}{0.508919in}}%
\pgfusepath{stroke,fill}%
\end{pgfscope}%
\begin{pgfscope}%
\pgfpathrectangle{\pgfqpoint{0.636356in}{0.440955in}}{\pgfqpoint{3.020670in}{0.290057in}} %
\pgfusepath{clip}%
\pgfsetbuttcap%
\pgfsetmiterjoin%
\definecolor{currentfill}{rgb}{0.333333,0.333333,0.333333}%
\pgfsetfillcolor{currentfill}%
\pgfsetlinewidth{0.501875pt}%
\definecolor{currentstroke}{rgb}{0.000000,0.000000,0.000000}%
\pgfsetstrokecolor{currentstroke}%
\pgfsetdash{}{0pt}%
\pgfpathmoveto{\pgfqpoint{2.267517in}{0.543347in}}%
\pgfpathlineto{\pgfqpoint{2.297724in}{0.543347in}}%
\pgfpathlineto{\pgfqpoint{2.297724in}{0.585984in}}%
\pgfpathlineto{\pgfqpoint{2.267517in}{0.585984in}}%
\pgfpathlineto{\pgfqpoint{2.267517in}{0.543347in}}%
\pgfusepath{stroke,fill}%
\end{pgfscope}%
\begin{pgfscope}%
\pgfpathrectangle{\pgfqpoint{0.636356in}{0.440955in}}{\pgfqpoint{3.020670in}{0.290057in}} %
\pgfusepath{clip}%
\pgfsetbuttcap%
\pgfsetmiterjoin%
\definecolor{currentfill}{rgb}{0.800000,0.266667,0.266667}%
\pgfsetfillcolor{currentfill}%
\pgfsetlinewidth{0.501875pt}%
\definecolor{currentstroke}{rgb}{0.000000,0.000000,0.000000}%
\pgfsetstrokecolor{currentstroke}%
\pgfsetdash{}{0pt}%
\pgfpathmoveto{\pgfqpoint{2.297724in}{0.420062in}}%
\pgfpathmoveto{\pgfqpoint{2.327931in}{0.430955in}}%
\pgfpathlineto{\pgfqpoint{2.327931in}{0.585984in}}%
\pgfpathlineto{\pgfqpoint{2.297724in}{0.585984in}}%
\pgfpathlineto{\pgfqpoint{2.297724in}{0.430955in}}%
\pgfusepath{stroke,fill}%
\end{pgfscope}%
\begin{pgfscope}%
\pgfpathrectangle{\pgfqpoint{0.636356in}{0.440955in}}{\pgfqpoint{3.020670in}{0.290057in}} %
\pgfusepath{clip}%
\pgfsetbuttcap%
\pgfsetmiterjoin%
\definecolor{currentfill}{rgb}{0.333333,0.333333,0.333333}%
\pgfsetfillcolor{currentfill}%
\pgfsetlinewidth{0.501875pt}%
\definecolor{currentstroke}{rgb}{0.000000,0.000000,0.000000}%
\pgfsetstrokecolor{currentstroke}%
\pgfsetdash{}{0pt}%
\pgfpathmoveto{\pgfqpoint{2.327931in}{0.585984in}}%
\pgfpathlineto{\pgfqpoint{2.358138in}{0.585984in}}%
\pgfpathlineto{\pgfqpoint{2.358138in}{0.605469in}}%
\pgfpathlineto{\pgfqpoint{2.327931in}{0.605469in}}%
\pgfpathlineto{\pgfqpoint{2.327931in}{0.585984in}}%
\pgfusepath{stroke,fill}%
\end{pgfscope}%
\begin{pgfscope}%
\pgfpathrectangle{\pgfqpoint{0.636356in}{0.440955in}}{\pgfqpoint{3.020670in}{0.290057in}} %
\pgfusepath{clip}%
\pgfsetbuttcap%
\pgfsetmiterjoin%
\definecolor{currentfill}{rgb}{0.333333,0.333333,0.333333}%
\pgfsetfillcolor{currentfill}%
\pgfsetlinewidth{0.501875pt}%
\definecolor{currentstroke}{rgb}{0.000000,0.000000,0.000000}%
\pgfsetstrokecolor{currentstroke}%
\pgfsetdash{}{0pt}%
\pgfpathmoveto{\pgfqpoint{2.358138in}{0.569221in}}%
\pgfpathlineto{\pgfqpoint{2.388344in}{0.569221in}}%
\pgfpathlineto{\pgfqpoint{2.388344in}{0.585984in}}%
\pgfpathlineto{\pgfqpoint{2.358138in}{0.585984in}}%
\pgfpathlineto{\pgfqpoint{2.358138in}{0.569221in}}%
\pgfusepath{stroke,fill}%
\end{pgfscope}%
\begin{pgfscope}%
\pgfpathrectangle{\pgfqpoint{0.636356in}{0.440955in}}{\pgfqpoint{3.020670in}{0.290057in}} %
\pgfusepath{clip}%
\pgfsetbuttcap%
\pgfsetmiterjoin%
\definecolor{currentfill}{rgb}{0.333333,0.333333,0.333333}%
\pgfsetfillcolor{currentfill}%
\pgfsetlinewidth{0.501875pt}%
\definecolor{currentstroke}{rgb}{0.000000,0.000000,0.000000}%
\pgfsetstrokecolor{currentstroke}%
\pgfsetdash{}{0pt}%
\pgfpathmoveto{\pgfqpoint{2.388344in}{0.585984in}}%
\pgfpathlineto{\pgfqpoint{2.418551in}{0.585984in}}%
\pgfpathlineto{\pgfqpoint{2.418551in}{0.623778in}}%
\pgfpathlineto{\pgfqpoint{2.388344in}{0.623778in}}%
\pgfpathlineto{\pgfqpoint{2.388344in}{0.585984in}}%
\pgfusepath{stroke,fill}%
\end{pgfscope}%
\begin{pgfscope}%
\pgfpathrectangle{\pgfqpoint{0.636356in}{0.440955in}}{\pgfqpoint{3.020670in}{0.290057in}} %
\pgfusepath{clip}%
\pgfsetbuttcap%
\pgfsetmiterjoin%
\definecolor{currentfill}{rgb}{0.333333,0.333333,0.333333}%
\pgfsetfillcolor{currentfill}%
\pgfsetlinewidth{0.501875pt}%
\definecolor{currentstroke}{rgb}{0.000000,0.000000,0.000000}%
\pgfsetstrokecolor{currentstroke}%
\pgfsetdash{}{0pt}%
\pgfpathmoveto{\pgfqpoint{2.418551in}{0.585984in}}%
\pgfpathlineto{\pgfqpoint{2.448758in}{0.585984in}}%
\pgfpathlineto{\pgfqpoint{2.448758in}{0.603749in}}%
\pgfpathlineto{\pgfqpoint{2.418551in}{0.603749in}}%
\pgfpathlineto{\pgfqpoint{2.418551in}{0.585984in}}%
\pgfusepath{stroke,fill}%
\end{pgfscope}%
\begin{pgfscope}%
\pgfpathrectangle{\pgfqpoint{0.636356in}{0.440955in}}{\pgfqpoint{3.020670in}{0.290057in}} %
\pgfusepath{clip}%
\pgfsetbuttcap%
\pgfsetmiterjoin%
\definecolor{currentfill}{rgb}{0.333333,0.333333,0.333333}%
\pgfsetfillcolor{currentfill}%
\pgfsetlinewidth{0.501875pt}%
\definecolor{currentstroke}{rgb}{0.000000,0.000000,0.000000}%
\pgfsetstrokecolor{currentstroke}%
\pgfsetdash{}{0pt}%
\pgfpathmoveto{\pgfqpoint{2.448758in}{0.503973in}}%
\pgfpathlineto{\pgfqpoint{2.478964in}{0.503973in}}%
\pgfpathlineto{\pgfqpoint{2.478964in}{0.585984in}}%
\pgfpathlineto{\pgfqpoint{2.448758in}{0.585984in}}%
\pgfpathlineto{\pgfqpoint{2.448758in}{0.503973in}}%
\pgfusepath{stroke,fill}%
\end{pgfscope}%
\begin{pgfscope}%
\pgfpathrectangle{\pgfqpoint{0.636356in}{0.440955in}}{\pgfqpoint{3.020670in}{0.290057in}} %
\pgfusepath{clip}%
\pgfsetbuttcap%
\pgfsetmiterjoin%
\definecolor{currentfill}{rgb}{0.333333,0.333333,0.333333}%
\pgfsetfillcolor{currentfill}%
\pgfsetlinewidth{0.501875pt}%
\definecolor{currentstroke}{rgb}{0.000000,0.000000,0.000000}%
\pgfsetstrokecolor{currentstroke}%
\pgfsetdash{}{0pt}%
\pgfpathmoveto{\pgfqpoint{2.478964in}{0.547837in}}%
\pgfpathlineto{\pgfqpoint{2.509171in}{0.547837in}}%
\pgfpathlineto{\pgfqpoint{2.509171in}{0.585984in}}%
\pgfpathlineto{\pgfqpoint{2.478964in}{0.585984in}}%
\pgfpathlineto{\pgfqpoint{2.478964in}{0.547837in}}%
\pgfusepath{stroke,fill}%
\end{pgfscope}%
\begin{pgfscope}%
\pgfpathrectangle{\pgfqpoint{0.636356in}{0.440955in}}{\pgfqpoint{3.020670in}{0.290057in}} %
\pgfusepath{clip}%
\pgfsetbuttcap%
\pgfsetmiterjoin%
\definecolor{currentfill}{rgb}{0.333333,0.333333,0.333333}%
\pgfsetfillcolor{currentfill}%
\pgfsetlinewidth{0.501875pt}%
\definecolor{currentstroke}{rgb}{0.000000,0.000000,0.000000}%
\pgfsetstrokecolor{currentstroke}%
\pgfsetdash{}{0pt}%
\pgfpathmoveto{\pgfqpoint{2.509171in}{0.516186in}}%
\pgfpathlineto{\pgfqpoint{2.539378in}{0.516186in}}%
\pgfpathlineto{\pgfqpoint{2.539378in}{0.585984in}}%
\pgfpathlineto{\pgfqpoint{2.509171in}{0.585984in}}%
\pgfpathlineto{\pgfqpoint{2.509171in}{0.516186in}}%
\pgfusepath{stroke,fill}%
\end{pgfscope}%
\begin{pgfscope}%
\pgfpathrectangle{\pgfqpoint{0.636356in}{0.440955in}}{\pgfqpoint{3.020670in}{0.290057in}} %
\pgfusepath{clip}%
\pgfsetbuttcap%
\pgfsetmiterjoin%
\definecolor{currentfill}{rgb}{0.333333,0.333333,0.333333}%
\pgfsetfillcolor{currentfill}%
\pgfsetlinewidth{0.501875pt}%
\definecolor{currentstroke}{rgb}{0.000000,0.000000,0.000000}%
\pgfsetstrokecolor{currentstroke}%
\pgfsetdash{}{0pt}%
\pgfpathmoveto{\pgfqpoint{2.539378in}{0.561493in}}%
\pgfpathlineto{\pgfqpoint{2.569584in}{0.561493in}}%
\pgfpathlineto{\pgfqpoint{2.569584in}{0.585984in}}%
\pgfpathlineto{\pgfqpoint{2.539378in}{0.585984in}}%
\pgfpathlineto{\pgfqpoint{2.539378in}{0.561493in}}%
\pgfusepath{stroke,fill}%
\end{pgfscope}%
\begin{pgfscope}%
\pgfpathrectangle{\pgfqpoint{0.636356in}{0.440955in}}{\pgfqpoint{3.020670in}{0.290057in}} %
\pgfusepath{clip}%
\pgfsetbuttcap%
\pgfsetmiterjoin%
\definecolor{currentfill}{rgb}{0.333333,0.333333,0.333333}%
\pgfsetfillcolor{currentfill}%
\pgfsetlinewidth{0.501875pt}%
\definecolor{currentstroke}{rgb}{0.000000,0.000000,0.000000}%
\pgfsetstrokecolor{currentstroke}%
\pgfsetdash{}{0pt}%
\pgfpathmoveto{\pgfqpoint{2.569584in}{0.475039in}}%
\pgfpathlineto{\pgfqpoint{2.599791in}{0.475039in}}%
\pgfpathlineto{\pgfqpoint{2.599791in}{0.585984in}}%
\pgfpathlineto{\pgfqpoint{2.569584in}{0.585984in}}%
\pgfpathlineto{\pgfqpoint{2.569584in}{0.475039in}}%
\pgfusepath{stroke,fill}%
\end{pgfscope}%
\begin{pgfscope}%
\pgfpathrectangle{\pgfqpoint{0.636356in}{0.440955in}}{\pgfqpoint{3.020670in}{0.290057in}} %
\pgfusepath{clip}%
\pgfsetbuttcap%
\pgfsetmiterjoin%
\definecolor{currentfill}{rgb}{0.333333,0.333333,0.333333}%
\pgfsetfillcolor{currentfill}%
\pgfsetlinewidth{0.501875pt}%
\definecolor{currentstroke}{rgb}{0.000000,0.000000,0.000000}%
\pgfsetstrokecolor{currentstroke}%
\pgfsetdash{}{0pt}%
\pgfpathmoveto{\pgfqpoint{2.599791in}{0.576312in}}%
\pgfpathlineto{\pgfqpoint{2.629998in}{0.576312in}}%
\pgfpathlineto{\pgfqpoint{2.629998in}{0.585984in}}%
\pgfpathlineto{\pgfqpoint{2.599791in}{0.585984in}}%
\pgfpathlineto{\pgfqpoint{2.599791in}{0.576312in}}%
\pgfusepath{stroke,fill}%
\end{pgfscope}%
\begin{pgfscope}%
\pgfpathrectangle{\pgfqpoint{0.636356in}{0.440955in}}{\pgfqpoint{3.020670in}{0.290057in}} %
\pgfusepath{clip}%
\pgfsetbuttcap%
\pgfsetmiterjoin%
\definecolor{currentfill}{rgb}{0.333333,0.333333,0.333333}%
\pgfsetfillcolor{currentfill}%
\pgfsetlinewidth{0.501875pt}%
\definecolor{currentstroke}{rgb}{0.000000,0.000000,0.000000}%
\pgfsetstrokecolor{currentstroke}%
\pgfsetdash{}{0pt}%
\pgfpathmoveto{\pgfqpoint{2.629998in}{0.585984in}}%
\pgfpathlineto{\pgfqpoint{2.660205in}{0.585984in}}%
\pgfpathlineto{\pgfqpoint{2.660205in}{0.621315in}}%
\pgfpathlineto{\pgfqpoint{2.629998in}{0.621315in}}%
\pgfpathlineto{\pgfqpoint{2.629998in}{0.585984in}}%
\pgfusepath{stroke,fill}%
\end{pgfscope}%
\begin{pgfscope}%
\pgfpathrectangle{\pgfqpoint{0.636356in}{0.440955in}}{\pgfqpoint{3.020670in}{0.290057in}} %
\pgfusepath{clip}%
\pgfsetbuttcap%
\pgfsetmiterjoin%
\definecolor{currentfill}{rgb}{0.333333,0.333333,0.333333}%
\pgfsetfillcolor{currentfill}%
\pgfsetlinewidth{0.501875pt}%
\definecolor{currentstroke}{rgb}{0.000000,0.000000,0.000000}%
\pgfsetstrokecolor{currentstroke}%
\pgfsetdash{}{0pt}%
\pgfpathmoveto{\pgfqpoint{2.660205in}{0.575090in}}%
\pgfpathlineto{\pgfqpoint{2.690411in}{0.575090in}}%
\pgfpathlineto{\pgfqpoint{2.690411in}{0.585984in}}%
\pgfpathlineto{\pgfqpoint{2.660205in}{0.585984in}}%
\pgfpathlineto{\pgfqpoint{2.660205in}{0.575090in}}%
\pgfusepath{stroke,fill}%
\end{pgfscope}%
\begin{pgfscope}%
\pgfpathrectangle{\pgfqpoint{0.636356in}{0.440955in}}{\pgfqpoint{3.020670in}{0.290057in}} %
\pgfusepath{clip}%
\pgfsetbuttcap%
\pgfsetmiterjoin%
\definecolor{currentfill}{rgb}{0.333333,0.333333,0.333333}%
\pgfsetfillcolor{currentfill}%
\pgfsetlinewidth{0.501875pt}%
\definecolor{currentstroke}{rgb}{0.000000,0.000000,0.000000}%
\pgfsetstrokecolor{currentstroke}%
\pgfsetdash{}{0pt}%
\pgfpathmoveto{\pgfqpoint{2.690411in}{0.585984in}}%
\pgfpathlineto{\pgfqpoint{2.720618in}{0.585984in}}%
\pgfpathlineto{\pgfqpoint{2.720618in}{0.616601in}}%
\pgfpathlineto{\pgfqpoint{2.690411in}{0.616601in}}%
\pgfpathlineto{\pgfqpoint{2.690411in}{0.585984in}}%
\pgfusepath{stroke,fill}%
\end{pgfscope}%
\begin{pgfscope}%
\pgfpathrectangle{\pgfqpoint{0.636356in}{0.440955in}}{\pgfqpoint{3.020670in}{0.290057in}} %
\pgfusepath{clip}%
\pgfsetbuttcap%
\pgfsetmiterjoin%
\definecolor{currentfill}{rgb}{0.333333,0.333333,0.333333}%
\pgfsetfillcolor{currentfill}%
\pgfsetlinewidth{0.501875pt}%
\definecolor{currentstroke}{rgb}{0.000000,0.000000,0.000000}%
\pgfsetstrokecolor{currentstroke}%
\pgfsetdash{}{0pt}%
\pgfpathmoveto{\pgfqpoint{2.720618in}{0.585984in}}%
\pgfpathlineto{\pgfqpoint{2.750825in}{0.585984in}}%
\pgfpathlineto{\pgfqpoint{2.750825in}{0.591978in}}%
\pgfpathlineto{\pgfqpoint{2.720618in}{0.591978in}}%
\pgfpathlineto{\pgfqpoint{2.720618in}{0.585984in}}%
\pgfusepath{stroke,fill}%
\end{pgfscope}%
\begin{pgfscope}%
\pgfpathrectangle{\pgfqpoint{0.636356in}{0.440955in}}{\pgfqpoint{3.020670in}{0.290057in}} %
\pgfusepath{clip}%
\pgfsetbuttcap%
\pgfsetmiterjoin%
\definecolor{currentfill}{rgb}{0.333333,0.333333,0.333333}%
\pgfsetfillcolor{currentfill}%
\pgfsetlinewidth{0.501875pt}%
\definecolor{currentstroke}{rgb}{0.000000,0.000000,0.000000}%
\pgfsetstrokecolor{currentstroke}%
\pgfsetdash{}{0pt}%
\pgfpathmoveto{\pgfqpoint{2.750825in}{0.582436in}}%
\pgfpathlineto{\pgfqpoint{2.781031in}{0.582436in}}%
\pgfpathlineto{\pgfqpoint{2.781031in}{0.585984in}}%
\pgfpathlineto{\pgfqpoint{2.750825in}{0.585984in}}%
\pgfpathlineto{\pgfqpoint{2.750825in}{0.582436in}}%
\pgfusepath{stroke,fill}%
\end{pgfscope}%
\begin{pgfscope}%
\pgfpathrectangle{\pgfqpoint{0.636356in}{0.440955in}}{\pgfqpoint{3.020670in}{0.290057in}} %
\pgfusepath{clip}%
\pgfsetbuttcap%
\pgfsetmiterjoin%
\definecolor{currentfill}{rgb}{0.333333,0.333333,0.333333}%
\pgfsetfillcolor{currentfill}%
\pgfsetlinewidth{0.501875pt}%
\definecolor{currentstroke}{rgb}{0.000000,0.000000,0.000000}%
\pgfsetstrokecolor{currentstroke}%
\pgfsetdash{}{0pt}%
\pgfpathmoveto{\pgfqpoint{2.781031in}{0.459715in}}%
\pgfpathlineto{\pgfqpoint{2.811238in}{0.459715in}}%
\pgfpathlineto{\pgfqpoint{2.811238in}{0.585984in}}%
\pgfpathlineto{\pgfqpoint{2.781031in}{0.585984in}}%
\pgfpathlineto{\pgfqpoint{2.781031in}{0.459715in}}%
\pgfusepath{stroke,fill}%
\end{pgfscope}%
\begin{pgfscope}%
\pgfpathrectangle{\pgfqpoint{0.636356in}{0.440955in}}{\pgfqpoint{3.020670in}{0.290057in}} %
\pgfusepath{clip}%
\pgfsetbuttcap%
\pgfsetmiterjoin%
\definecolor{currentfill}{rgb}{0.333333,0.333333,0.333333}%
\pgfsetfillcolor{currentfill}%
\pgfsetlinewidth{0.501875pt}%
\definecolor{currentstroke}{rgb}{0.000000,0.000000,0.000000}%
\pgfsetstrokecolor{currentstroke}%
\pgfsetdash{}{0pt}%
\pgfpathmoveto{\pgfqpoint{2.811238in}{0.585984in}}%
\pgfpathlineto{\pgfqpoint{2.841445in}{0.585984in}}%
\pgfpathlineto{\pgfqpoint{2.841445in}{0.587598in}}%
\pgfpathlineto{\pgfqpoint{2.811238in}{0.587598in}}%
\pgfpathlineto{\pgfqpoint{2.811238in}{0.585984in}}%
\pgfusepath{stroke,fill}%
\end{pgfscope}%
\begin{pgfscope}%
\pgfpathrectangle{\pgfqpoint{0.636356in}{0.440955in}}{\pgfqpoint{3.020670in}{0.290057in}} %
\pgfusepath{clip}%
\pgfsetbuttcap%
\pgfsetmiterjoin%
\definecolor{currentfill}{rgb}{0.333333,0.333333,0.333333}%
\pgfsetfillcolor{currentfill}%
\pgfsetlinewidth{0.501875pt}%
\definecolor{currentstroke}{rgb}{0.000000,0.000000,0.000000}%
\pgfsetstrokecolor{currentstroke}%
\pgfsetdash{}{0pt}%
\pgfpathmoveto{\pgfqpoint{2.841445in}{0.477561in}}%
\pgfpathlineto{\pgfqpoint{2.871651in}{0.477561in}}%
\pgfpathlineto{\pgfqpoint{2.871651in}{0.585984in}}%
\pgfpathlineto{\pgfqpoint{2.841445in}{0.585984in}}%
\pgfpathlineto{\pgfqpoint{2.841445in}{0.477561in}}%
\pgfusepath{stroke,fill}%
\end{pgfscope}%
\begin{pgfscope}%
\pgfpathrectangle{\pgfqpoint{0.636356in}{0.440955in}}{\pgfqpoint{3.020670in}{0.290057in}} %
\pgfusepath{clip}%
\pgfsetbuttcap%
\pgfsetmiterjoin%
\definecolor{currentfill}{rgb}{0.333333,0.333333,0.333333}%
\pgfsetfillcolor{currentfill}%
\pgfsetlinewidth{0.501875pt}%
\definecolor{currentstroke}{rgb}{0.000000,0.000000,0.000000}%
\pgfsetstrokecolor{currentstroke}%
\pgfsetdash{}{0pt}%
\pgfpathmoveto{\pgfqpoint{2.871651in}{0.555064in}}%
\pgfpathlineto{\pgfqpoint{2.901858in}{0.555064in}}%
\pgfpathlineto{\pgfqpoint{2.901858in}{0.585984in}}%
\pgfpathlineto{\pgfqpoint{2.871651in}{0.585984in}}%
\pgfpathlineto{\pgfqpoint{2.871651in}{0.555064in}}%
\pgfusepath{stroke,fill}%
\end{pgfscope}%
\begin{pgfscope}%
\pgfpathrectangle{\pgfqpoint{0.636356in}{0.440955in}}{\pgfqpoint{3.020670in}{0.290057in}} %
\pgfusepath{clip}%
\pgfsetbuttcap%
\pgfsetmiterjoin%
\definecolor{currentfill}{rgb}{0.333333,0.333333,0.333333}%
\pgfsetfillcolor{currentfill}%
\pgfsetlinewidth{0.501875pt}%
\definecolor{currentstroke}{rgb}{0.000000,0.000000,0.000000}%
\pgfsetstrokecolor{currentstroke}%
\pgfsetdash{}{0pt}%
\pgfpathmoveto{\pgfqpoint{2.901858in}{0.574288in}}%
\pgfpathlineto{\pgfqpoint{2.932065in}{0.574288in}}%
\pgfpathlineto{\pgfqpoint{2.932065in}{0.585984in}}%
\pgfpathlineto{\pgfqpoint{2.901858in}{0.585984in}}%
\pgfpathlineto{\pgfqpoint{2.901858in}{0.574288in}}%
\pgfusepath{stroke,fill}%
\end{pgfscope}%
\begin{pgfscope}%
\pgfpathrectangle{\pgfqpoint{0.636356in}{0.440955in}}{\pgfqpoint{3.020670in}{0.290057in}} %
\pgfusepath{clip}%
\pgfsetbuttcap%
\pgfsetmiterjoin%
\definecolor{currentfill}{rgb}{0.333333,0.333333,0.333333}%
\pgfsetfillcolor{currentfill}%
\pgfsetlinewidth{0.501875pt}%
\definecolor{currentstroke}{rgb}{0.000000,0.000000,0.000000}%
\pgfsetstrokecolor{currentstroke}%
\pgfsetdash{}{0pt}%
\pgfpathmoveto{\pgfqpoint{2.932065in}{0.522251in}}%
\pgfpathlineto{\pgfqpoint{2.962271in}{0.522251in}}%
\pgfpathlineto{\pgfqpoint{2.962271in}{0.585984in}}%
\pgfpathlineto{\pgfqpoint{2.932065in}{0.585984in}}%
\pgfpathlineto{\pgfqpoint{2.932065in}{0.522251in}}%
\pgfusepath{stroke,fill}%
\end{pgfscope}%
\begin{pgfscope}%
\pgfpathrectangle{\pgfqpoint{0.636356in}{0.440955in}}{\pgfqpoint{3.020670in}{0.290057in}} %
\pgfusepath{clip}%
\pgfsetbuttcap%
\pgfsetmiterjoin%
\definecolor{currentfill}{rgb}{0.333333,0.333333,0.333333}%
\pgfsetfillcolor{currentfill}%
\pgfsetlinewidth{0.501875pt}%
\definecolor{currentstroke}{rgb}{0.000000,0.000000,0.000000}%
\pgfsetstrokecolor{currentstroke}%
\pgfsetdash{}{0pt}%
\pgfpathmoveto{\pgfqpoint{2.962271in}{0.585984in}}%
\pgfpathlineto{\pgfqpoint{2.992478in}{0.585984in}}%
\pgfpathlineto{\pgfqpoint{2.992478in}{0.628164in}}%
\pgfpathlineto{\pgfqpoint{2.962271in}{0.628164in}}%
\pgfpathlineto{\pgfqpoint{2.962271in}{0.585984in}}%
\pgfusepath{stroke,fill}%
\end{pgfscope}%
\begin{pgfscope}%
\pgfpathrectangle{\pgfqpoint{0.636356in}{0.440955in}}{\pgfqpoint{3.020670in}{0.290057in}} %
\pgfusepath{clip}%
\pgfsetbuttcap%
\pgfsetmiterjoin%
\definecolor{currentfill}{rgb}{0.333333,0.333333,0.333333}%
\pgfsetfillcolor{currentfill}%
\pgfsetlinewidth{0.501875pt}%
\definecolor{currentstroke}{rgb}{0.000000,0.000000,0.000000}%
\pgfsetstrokecolor{currentstroke}%
\pgfsetdash{}{0pt}%
\pgfpathmoveto{\pgfqpoint{2.992478in}{0.585984in}}%
\pgfpathlineto{\pgfqpoint{3.022685in}{0.585984in}}%
\pgfpathlineto{\pgfqpoint{3.022685in}{0.624003in}}%
\pgfpathlineto{\pgfqpoint{2.992478in}{0.624003in}}%
\pgfpathlineto{\pgfqpoint{2.992478in}{0.585984in}}%
\pgfusepath{stroke,fill}%
\end{pgfscope}%
\begin{pgfscope}%
\pgfpathrectangle{\pgfqpoint{0.636356in}{0.440955in}}{\pgfqpoint{3.020670in}{0.290057in}} %
\pgfusepath{clip}%
\pgfsetbuttcap%
\pgfsetmiterjoin%
\definecolor{currentfill}{rgb}{0.333333,0.333333,0.333333}%
\pgfsetfillcolor{currentfill}%
\pgfsetlinewidth{0.501875pt}%
\definecolor{currentstroke}{rgb}{0.000000,0.000000,0.000000}%
\pgfsetstrokecolor{currentstroke}%
\pgfsetdash{}{0pt}%
\pgfpathmoveto{\pgfqpoint{3.022685in}{0.585984in}}%
\pgfpathlineto{\pgfqpoint{3.052892in}{0.585984in}}%
\pgfpathlineto{\pgfqpoint{3.052892in}{0.648876in}}%
\pgfpathlineto{\pgfqpoint{3.022685in}{0.648876in}}%
\pgfpathlineto{\pgfqpoint{3.022685in}{0.585984in}}%
\pgfusepath{stroke,fill}%
\end{pgfscope}%
\begin{pgfscope}%
\pgfpathrectangle{\pgfqpoint{0.636356in}{0.440955in}}{\pgfqpoint{3.020670in}{0.290057in}} %
\pgfusepath{clip}%
\pgfsetbuttcap%
\pgfsetmiterjoin%
\definecolor{currentfill}{rgb}{0.333333,0.333333,0.333333}%
\pgfsetfillcolor{currentfill}%
\pgfsetlinewidth{0.501875pt}%
\definecolor{currentstroke}{rgb}{0.000000,0.000000,0.000000}%
\pgfsetstrokecolor{currentstroke}%
\pgfsetdash{}{0pt}%
\pgfpathmoveto{\pgfqpoint{3.052892in}{0.585984in}}%
\pgfpathlineto{\pgfqpoint{3.083098in}{0.585984in}}%
\pgfpathlineto{\pgfqpoint{3.083098in}{0.646853in}}%
\pgfpathlineto{\pgfqpoint{3.052892in}{0.646853in}}%
\pgfpathlineto{\pgfqpoint{3.052892in}{0.585984in}}%
\pgfusepath{stroke,fill}%
\end{pgfscope}%
\begin{pgfscope}%
\pgfpathrectangle{\pgfqpoint{0.636356in}{0.440955in}}{\pgfqpoint{3.020670in}{0.290057in}} %
\pgfusepath{clip}%
\pgfsetbuttcap%
\pgfsetmiterjoin%
\definecolor{currentfill}{rgb}{0.333333,0.333333,0.333333}%
\pgfsetfillcolor{currentfill}%
\pgfsetlinewidth{0.501875pt}%
\definecolor{currentstroke}{rgb}{0.000000,0.000000,0.000000}%
\pgfsetstrokecolor{currentstroke}%
\pgfsetdash{}{0pt}%
\pgfpathmoveto{\pgfqpoint{3.083098in}{0.565631in}}%
\pgfpathlineto{\pgfqpoint{3.113305in}{0.565631in}}%
\pgfpathlineto{\pgfqpoint{3.113305in}{0.585984in}}%
\pgfpathlineto{\pgfqpoint{3.083098in}{0.585984in}}%
\pgfpathlineto{\pgfqpoint{3.083098in}{0.565631in}}%
\pgfusepath{stroke,fill}%
\end{pgfscope}%
\begin{pgfscope}%
\pgfpathrectangle{\pgfqpoint{0.636356in}{0.440955in}}{\pgfqpoint{3.020670in}{0.290057in}} %
\pgfusepath{clip}%
\pgfsetbuttcap%
\pgfsetmiterjoin%
\definecolor{currentfill}{rgb}{0.333333,0.333333,0.333333}%
\pgfsetfillcolor{currentfill}%
\pgfsetlinewidth{0.501875pt}%
\definecolor{currentstroke}{rgb}{0.000000,0.000000,0.000000}%
\pgfsetstrokecolor{currentstroke}%
\pgfsetdash{}{0pt}%
\pgfpathmoveto{\pgfqpoint{3.113305in}{0.585984in}}%
\pgfpathlineto{\pgfqpoint{3.143512in}{0.585984in}}%
\pgfpathlineto{\pgfqpoint{3.143512in}{0.666182in}}%
\pgfpathlineto{\pgfqpoint{3.113305in}{0.666182in}}%
\pgfpathlineto{\pgfqpoint{3.113305in}{0.585984in}}%
\pgfusepath{stroke,fill}%
\end{pgfscope}%
\begin{pgfscope}%
\pgfpathrectangle{\pgfqpoint{0.636356in}{0.440955in}}{\pgfqpoint{3.020670in}{0.290057in}} %
\pgfusepath{clip}%
\pgfsetbuttcap%
\pgfsetmiterjoin%
\definecolor{currentfill}{rgb}{0.333333,0.333333,0.333333}%
\pgfsetfillcolor{currentfill}%
\pgfsetlinewidth{0.501875pt}%
\definecolor{currentstroke}{rgb}{0.000000,0.000000,0.000000}%
\pgfsetstrokecolor{currentstroke}%
\pgfsetdash{}{0pt}%
\pgfpathmoveto{\pgfqpoint{3.143512in}{0.585984in}}%
\pgfpathlineto{\pgfqpoint{3.173718in}{0.585984in}}%
\pgfpathlineto{\pgfqpoint{3.173718in}{0.614362in}}%
\pgfpathlineto{\pgfqpoint{3.143512in}{0.614362in}}%
\pgfpathlineto{\pgfqpoint{3.143512in}{0.585984in}}%
\pgfusepath{stroke,fill}%
\end{pgfscope}%
\begin{pgfscope}%
\pgfpathrectangle{\pgfqpoint{0.636356in}{0.440955in}}{\pgfqpoint{3.020670in}{0.290057in}} %
\pgfusepath{clip}%
\pgfsetbuttcap%
\pgfsetmiterjoin%
\definecolor{currentfill}{rgb}{0.333333,0.333333,0.333333}%
\pgfsetfillcolor{currentfill}%
\pgfsetlinewidth{0.501875pt}%
\definecolor{currentstroke}{rgb}{0.000000,0.000000,0.000000}%
\pgfsetstrokecolor{currentstroke}%
\pgfsetdash{}{0pt}%
\pgfpathmoveto{\pgfqpoint{3.173718in}{0.572559in}}%
\pgfpathlineto{\pgfqpoint{3.203925in}{0.572559in}}%
\pgfpathlineto{\pgfqpoint{3.203925in}{0.585984in}}%
\pgfpathlineto{\pgfqpoint{3.173718in}{0.585984in}}%
\pgfpathlineto{\pgfqpoint{3.173718in}{0.572559in}}%
\pgfusepath{stroke,fill}%
\end{pgfscope}%
\begin{pgfscope}%
\pgfpathrectangle{\pgfqpoint{0.636356in}{0.440955in}}{\pgfqpoint{3.020670in}{0.290057in}} %
\pgfusepath{clip}%
\pgfsetbuttcap%
\pgfsetmiterjoin%
\definecolor{currentfill}{rgb}{0.333333,0.333333,0.333333}%
\pgfsetfillcolor{currentfill}%
\pgfsetlinewidth{0.501875pt}%
\definecolor{currentstroke}{rgb}{0.000000,0.000000,0.000000}%
\pgfsetstrokecolor{currentstroke}%
\pgfsetdash{}{0pt}%
\pgfpathmoveto{\pgfqpoint{3.203925in}{0.585984in}}%
\pgfpathlineto{\pgfqpoint{3.234132in}{0.585984in}}%
\pgfpathlineto{\pgfqpoint{3.234132in}{0.658140in}}%
\pgfpathlineto{\pgfqpoint{3.203925in}{0.658140in}}%
\pgfpathlineto{\pgfqpoint{3.203925in}{0.585984in}}%
\pgfusepath{stroke,fill}%
\end{pgfscope}%
\begin{pgfscope}%
\pgfpathrectangle{\pgfqpoint{0.636356in}{0.440955in}}{\pgfqpoint{3.020670in}{0.290057in}} %
\pgfusepath{clip}%
\pgfsetbuttcap%
\pgfsetmiterjoin%
\definecolor{currentfill}{rgb}{0.333333,0.333333,0.333333}%
\pgfsetfillcolor{currentfill}%
\pgfsetlinewidth{0.501875pt}%
\definecolor{currentstroke}{rgb}{0.000000,0.000000,0.000000}%
\pgfsetstrokecolor{currentstroke}%
\pgfsetdash{}{0pt}%
\pgfpathmoveto{\pgfqpoint{3.234132in}{0.585984in}}%
\pgfpathlineto{\pgfqpoint{3.264338in}{0.585984in}}%
\pgfpathlineto{\pgfqpoint{3.264338in}{0.644282in}}%
\pgfpathlineto{\pgfqpoint{3.234132in}{0.644282in}}%
\pgfpathlineto{\pgfqpoint{3.234132in}{0.585984in}}%
\pgfusepath{stroke,fill}%
\end{pgfscope}%
\begin{pgfscope}%
\pgfpathrectangle{\pgfqpoint{0.636356in}{0.440955in}}{\pgfqpoint{3.020670in}{0.290057in}} %
\pgfusepath{clip}%
\pgfsetbuttcap%
\pgfsetmiterjoin%
\definecolor{currentfill}{rgb}{0.333333,0.333333,0.333333}%
\pgfsetfillcolor{currentfill}%
\pgfsetlinewidth{0.501875pt}%
\definecolor{currentstroke}{rgb}{0.000000,0.000000,0.000000}%
\pgfsetstrokecolor{currentstroke}%
\pgfsetdash{}{0pt}%
\pgfpathmoveto{\pgfqpoint{3.264338in}{0.585984in}}%
\pgfpathlineto{\pgfqpoint{3.294545in}{0.585984in}}%
\pgfpathlineto{\pgfqpoint{3.294545in}{0.631058in}}%
\pgfpathlineto{\pgfqpoint{3.264338in}{0.631058in}}%
\pgfpathlineto{\pgfqpoint{3.264338in}{0.585984in}}%
\pgfusepath{stroke,fill}%
\end{pgfscope}%
\begin{pgfscope}%
\pgfpathrectangle{\pgfqpoint{0.636356in}{0.440955in}}{\pgfqpoint{3.020670in}{0.290057in}} %
\pgfusepath{clip}%
\pgfsetbuttcap%
\pgfsetmiterjoin%
\definecolor{currentfill}{rgb}{0.333333,0.333333,0.333333}%
\pgfsetfillcolor{currentfill}%
\pgfsetlinewidth{0.501875pt}%
\definecolor{currentstroke}{rgb}{0.000000,0.000000,0.000000}%
\pgfsetstrokecolor{currentstroke}%
\pgfsetdash{}{0pt}%
\pgfpathmoveto{\pgfqpoint{3.294545in}{0.585984in}}%
\pgfpathlineto{\pgfqpoint{3.324752in}{0.585984in}}%
\pgfpathlineto{\pgfqpoint{3.324752in}{0.597889in}}%
\pgfpathlineto{\pgfqpoint{3.294545in}{0.597889in}}%
\pgfpathlineto{\pgfqpoint{3.294545in}{0.585984in}}%
\pgfusepath{stroke,fill}%
\end{pgfscope}%
\begin{pgfscope}%
\pgfpathrectangle{\pgfqpoint{0.636356in}{0.440955in}}{\pgfqpoint{3.020670in}{0.290057in}} %
\pgfusepath{clip}%
\pgfsetbuttcap%
\pgfsetmiterjoin%
\definecolor{currentfill}{rgb}{0.800000,0.266667,0.266667}%
\pgfsetfillcolor{currentfill}%
\pgfsetlinewidth{0.501875pt}%
\definecolor{currentstroke}{rgb}{0.000000,0.000000,0.000000}%
\pgfsetstrokecolor{currentstroke}%
\pgfsetdash{}{0pt}%
\pgfpathmoveto{\pgfqpoint{3.324752in}{0.585984in}}%
\pgfpathlineto{\pgfqpoint{3.354959in}{0.585984in}}%
\pgfpathlineto{\pgfqpoint{3.354959in}{0.740422in}}%
\pgfpathlineto{\pgfqpoint{3.324752in}{0.740422in}}%
\pgfpathlineto{\pgfqpoint{3.324752in}{0.585984in}}%
\pgfusepath{stroke,fill}%
\end{pgfscope}%
\begin{pgfscope}%
\pgfpathrectangle{\pgfqpoint{0.636356in}{0.440955in}}{\pgfqpoint{3.020670in}{0.290057in}} %
\pgfusepath{clip}%
\pgfsetbuttcap%
\pgfsetmiterjoin%
\definecolor{currentfill}{rgb}{0.333333,0.333333,0.333333}%
\pgfsetfillcolor{currentfill}%
\pgfsetlinewidth{0.501875pt}%
\definecolor{currentstroke}{rgb}{0.000000,0.000000,0.000000}%
\pgfsetstrokecolor{currentstroke}%
\pgfsetdash{}{0pt}%
\pgfpathmoveto{\pgfqpoint{3.354959in}{0.585984in}}%
\pgfpathlineto{\pgfqpoint{3.385165in}{0.585984in}}%
\pgfpathlineto{\pgfqpoint{3.385165in}{0.645130in}}%
\pgfpathlineto{\pgfqpoint{3.354959in}{0.645130in}}%
\pgfpathlineto{\pgfqpoint{3.354959in}{0.585984in}}%
\pgfusepath{stroke,fill}%
\end{pgfscope}%
\begin{pgfscope}%
\pgfpathrectangle{\pgfqpoint{0.636356in}{0.440955in}}{\pgfqpoint{3.020670in}{0.290057in}} %
\pgfusepath{clip}%
\pgfsetbuttcap%
\pgfsetmiterjoin%
\definecolor{currentfill}{rgb}{0.333333,0.333333,0.333333}%
\pgfsetfillcolor{currentfill}%
\pgfsetlinewidth{0.501875pt}%
\definecolor{currentstroke}{rgb}{0.000000,0.000000,0.000000}%
\pgfsetstrokecolor{currentstroke}%
\pgfsetdash{}{0pt}%
\pgfpathmoveto{\pgfqpoint{3.385165in}{0.585984in}}%
\pgfpathlineto{\pgfqpoint{3.415372in}{0.585984in}}%
\pgfpathlineto{\pgfqpoint{3.415372in}{0.701341in}}%
\pgfpathlineto{\pgfqpoint{3.385165in}{0.701341in}}%
\pgfpathlineto{\pgfqpoint{3.385165in}{0.585984in}}%
\pgfusepath{stroke,fill}%
\end{pgfscope}%
\begin{pgfscope}%
\pgfpathrectangle{\pgfqpoint{0.636356in}{0.440955in}}{\pgfqpoint{3.020670in}{0.290057in}} %
\pgfusepath{clip}%
\pgfsetbuttcap%
\pgfsetmiterjoin%
\definecolor{currentfill}{rgb}{0.333333,0.333333,0.333333}%
\pgfsetfillcolor{currentfill}%
\pgfsetlinewidth{0.501875pt}%
\definecolor{currentstroke}{rgb}{0.000000,0.000000,0.000000}%
\pgfsetstrokecolor{currentstroke}%
\pgfsetdash{}{0pt}%
\pgfpathmoveto{\pgfqpoint{3.415372in}{0.585984in}}%
\pgfpathlineto{\pgfqpoint{3.445579in}{0.585984in}}%
\pgfpathlineto{\pgfqpoint{3.445579in}{0.620297in}}%
\pgfpathlineto{\pgfqpoint{3.415372in}{0.620297in}}%
\pgfpathlineto{\pgfqpoint{3.415372in}{0.585984in}}%
\pgfusepath{stroke,fill}%
\end{pgfscope}%
\begin{pgfscope}%
\pgfpathrectangle{\pgfqpoint{0.636356in}{0.440955in}}{\pgfqpoint{3.020670in}{0.290057in}} %
\pgfusepath{clip}%
\pgfsetbuttcap%
\pgfsetmiterjoin%
\definecolor{currentfill}{rgb}{0.333333,0.333333,0.333333}%
\pgfsetfillcolor{currentfill}%
\pgfsetlinewidth{0.501875pt}%
\definecolor{currentstroke}{rgb}{0.000000,0.000000,0.000000}%
\pgfsetstrokecolor{currentstroke}%
\pgfsetdash{}{0pt}%
\pgfpathmoveto{\pgfqpoint{3.445579in}{0.585984in}}%
\pgfpathlineto{\pgfqpoint{3.475785in}{0.585984in}}%
\pgfpathlineto{\pgfqpoint{3.475785in}{0.645787in}}%
\pgfpathlineto{\pgfqpoint{3.445579in}{0.645787in}}%
\pgfpathlineto{\pgfqpoint{3.445579in}{0.585984in}}%
\pgfusepath{stroke,fill}%
\end{pgfscope}%
\begin{pgfscope}%
\pgfpathrectangle{\pgfqpoint{0.636356in}{0.440955in}}{\pgfqpoint{3.020670in}{0.290057in}} %
\pgfusepath{clip}%
\pgfsetbuttcap%
\pgfsetmiterjoin%
\definecolor{currentfill}{rgb}{0.333333,0.333333,0.333333}%
\pgfsetfillcolor{currentfill}%
\pgfsetlinewidth{0.501875pt}%
\definecolor{currentstroke}{rgb}{0.000000,0.000000,0.000000}%
\pgfsetstrokecolor{currentstroke}%
\pgfsetdash{}{0pt}%
\pgfpathmoveto{\pgfqpoint{3.475785in}{0.585984in}}%
\pgfpathlineto{\pgfqpoint{3.505992in}{0.585984in}}%
\pgfpathlineto{\pgfqpoint{3.505992in}{0.626438in}}%
\pgfpathlineto{\pgfqpoint{3.475785in}{0.626438in}}%
\pgfpathlineto{\pgfqpoint{3.475785in}{0.585984in}}%
\pgfusepath{stroke,fill}%
\end{pgfscope}%
\begin{pgfscope}%
\pgfpathrectangle{\pgfqpoint{0.636356in}{0.440955in}}{\pgfqpoint{3.020670in}{0.290057in}} %
\pgfusepath{clip}%
\pgfsetbuttcap%
\pgfsetmiterjoin%
\definecolor{currentfill}{rgb}{0.333333,0.333333,0.333333}%
\pgfsetfillcolor{currentfill}%
\pgfsetlinewidth{0.501875pt}%
\definecolor{currentstroke}{rgb}{0.000000,0.000000,0.000000}%
\pgfsetstrokecolor{currentstroke}%
\pgfsetdash{}{0pt}%
\pgfpathmoveto{\pgfqpoint{3.505992in}{0.585984in}}%
\pgfpathlineto{\pgfqpoint{3.536199in}{0.585984in}}%
\pgfpathlineto{\pgfqpoint{3.536199in}{0.712416in}}%
\pgfpathlineto{\pgfqpoint{3.505992in}{0.712416in}}%
\pgfpathlineto{\pgfqpoint{3.505992in}{0.585984in}}%
\pgfusepath{stroke,fill}%
\end{pgfscope}%
\begin{pgfscope}%
\pgfpathrectangle{\pgfqpoint{0.636356in}{0.440955in}}{\pgfqpoint{3.020670in}{0.290057in}} %
\pgfusepath{clip}%
\pgfsetbuttcap%
\pgfsetmiterjoin%
\definecolor{currentfill}{rgb}{0.333333,0.333333,0.333333}%
\pgfsetfillcolor{currentfill}%
\pgfsetlinewidth{0.501875pt}%
\definecolor{currentstroke}{rgb}{0.000000,0.000000,0.000000}%
\pgfsetstrokecolor{currentstroke}%
\pgfsetdash{}{0pt}%
\pgfpathmoveto{\pgfqpoint{3.536199in}{0.585984in}}%
\pgfpathlineto{\pgfqpoint{3.566405in}{0.585984in}}%
\pgfpathlineto{\pgfqpoint{3.566405in}{0.722288in}}%
\pgfpathlineto{\pgfqpoint{3.536199in}{0.722288in}}%
\pgfpathlineto{\pgfqpoint{3.536199in}{0.585984in}}%
\pgfusepath{stroke,fill}%
\end{pgfscope}%
\begin{pgfscope}%
\pgfpathrectangle{\pgfqpoint{0.636356in}{0.440955in}}{\pgfqpoint{3.020670in}{0.290057in}} %
\pgfusepath{clip}%
\pgfsetbuttcap%
\pgfsetmiterjoin%
\definecolor{currentfill}{rgb}{0.333333,0.333333,0.333333}%
\pgfsetfillcolor{currentfill}%
\pgfsetlinewidth{0.501875pt}%
\definecolor{currentstroke}{rgb}{0.000000,0.000000,0.000000}%
\pgfsetstrokecolor{currentstroke}%
\pgfsetdash{}{0pt}%
\pgfpathmoveto{\pgfqpoint{3.566405in}{0.585984in}}%
\pgfpathlineto{\pgfqpoint{3.596612in}{0.585984in}}%
\pgfpathlineto{\pgfqpoint{3.596612in}{0.658113in}}%
\pgfpathlineto{\pgfqpoint{3.566405in}{0.658113in}}%
\pgfpathlineto{\pgfqpoint{3.566405in}{0.585984in}}%
\pgfusepath{stroke,fill}%
\end{pgfscope}%
\begin{pgfscope}%
\pgfpathrectangle{\pgfqpoint{0.636356in}{0.440955in}}{\pgfqpoint{3.020670in}{0.290057in}} %
\pgfusepath{clip}%
\pgfsetbuttcap%
\pgfsetmiterjoin%
\definecolor{currentfill}{rgb}{0.333333,0.333333,0.333333}%
\pgfsetfillcolor{currentfill}%
\pgfsetlinewidth{0.501875pt}%
\definecolor{currentstroke}{rgb}{0.000000,0.000000,0.000000}%
\pgfsetstrokecolor{currentstroke}%
\pgfsetdash{}{0pt}%
\pgfpathmoveto{\pgfqpoint{3.596612in}{0.554486in}}%
\pgfpathlineto{\pgfqpoint{3.626819in}{0.554486in}}%
\pgfpathlineto{\pgfqpoint{3.626819in}{0.585984in}}%
\pgfpathlineto{\pgfqpoint{3.596612in}{0.585984in}}%
\pgfpathlineto{\pgfqpoint{3.596612in}{0.554486in}}%
\pgfusepath{stroke,fill}%
\end{pgfscope}%
\begin{pgfscope}%
\pgfpathrectangle{\pgfqpoint{0.636356in}{0.440955in}}{\pgfqpoint{3.020670in}{0.290057in}} %
\pgfusepath{clip}%
\pgfsetbuttcap%
\pgfsetmiterjoin%
\definecolor{currentfill}{rgb}{0.333333,0.333333,0.333333}%
\pgfsetfillcolor{currentfill}%
\pgfsetlinewidth{0.501875pt}%
\definecolor{currentstroke}{rgb}{0.000000,0.000000,0.000000}%
\pgfsetstrokecolor{currentstroke}%
\pgfsetdash{}{0pt}%
\pgfpathmoveto{\pgfqpoint{3.626819in}{0.585984in}}%
\pgfpathlineto{\pgfqpoint{3.657026in}{0.585984in}}%
\pgfpathlineto{\pgfqpoint{3.657026in}{0.614041in}}%
\pgfpathlineto{\pgfqpoint{3.626819in}{0.614041in}}%
\pgfpathlineto{\pgfqpoint{3.626819in}{0.585984in}}%
\pgfusepath{stroke,fill}%
\end{pgfscope}%
\begin{pgfscope}%
\pgfsetrectcap%
\pgfsetmiterjoin%
\pgfsetlinewidth{1.003750pt}%
\definecolor{currentstroke}{rgb}{0.000000,0.000000,0.000000}%
\pgfsetstrokecolor{currentstroke}%
\pgfsetdash{}{0pt}%
\pgfpathmoveto{\pgfqpoint{0.636356in}{0.731012in}}%
\pgfpathlineto{\pgfqpoint{3.657026in}{0.731012in}}%
\pgfusepath{stroke}%
\end{pgfscope}%
\begin{pgfscope}%
\pgfsetrectcap%
\pgfsetmiterjoin%
\pgfsetlinewidth{1.003750pt}%
\definecolor{currentstroke}{rgb}{0.000000,0.000000,0.000000}%
\pgfsetstrokecolor{currentstroke}%
\pgfsetdash{}{0pt}%
\pgfpathmoveto{\pgfqpoint{3.657026in}{0.440955in}}%
\pgfpathlineto{\pgfqpoint{3.657026in}{0.731012in}}%
\pgfusepath{stroke}%
\end{pgfscope}%
\begin{pgfscope}%
\pgfsetrectcap%
\pgfsetmiterjoin%
\pgfsetlinewidth{1.003750pt}%
\definecolor{currentstroke}{rgb}{0.000000,0.000000,0.000000}%
\pgfsetstrokecolor{currentstroke}%
\pgfsetdash{}{0pt}%
\pgfpathmoveto{\pgfqpoint{0.636356in}{0.440955in}}%
\pgfpathlineto{\pgfqpoint{3.657026in}{0.440955in}}%
\pgfusepath{stroke}%
\end{pgfscope}%
\begin{pgfscope}%
\pgfsetrectcap%
\pgfsetmiterjoin%
\pgfsetlinewidth{1.003750pt}%
\definecolor{currentstroke}{rgb}{0.000000,0.000000,0.000000}%
\pgfsetstrokecolor{currentstroke}%
\pgfsetdash{}{0pt}%
\pgfpathmoveto{\pgfqpoint{0.636356in}{0.440955in}}%
\pgfpathlineto{\pgfqpoint{0.636356in}{0.731012in}}%
\pgfusepath{stroke}%
\end{pgfscope}%
\begin{pgfscope}%
\pgfsetbuttcap%
\pgfsetroundjoin%
\definecolor{currentfill}{rgb}{0.000000,0.000000,0.000000}%
\pgfsetfillcolor{currentfill}%
\pgfsetlinewidth{0.501875pt}%
\definecolor{currentstroke}{rgb}{0.000000,0.000000,0.000000}%
\pgfsetstrokecolor{currentstroke}%
\pgfsetdash{}{0pt}%
\pgfsys@defobject{currentmarker}{\pgfqpoint{0.000000in}{0.000000in}}{\pgfqpoint{0.000000in}{0.069444in}}{%
\pgfpathmoveto{\pgfqpoint{0.000000in}{0.000000in}}%
\pgfpathlineto{\pgfqpoint{0.000000in}{0.069444in}}%
\pgfusepath{stroke,fill}%
}%
\begin{pgfscope}%
\pgfsys@transformshift{0.780197in}{0.440955in}%
\pgfsys@useobject{currentmarker}{}%
\end{pgfscope}%
\end{pgfscope}%
\begin{pgfscope}%
\pgfsetbuttcap%
\pgfsetroundjoin%
\definecolor{currentfill}{rgb}{0.000000,0.000000,0.000000}%
\pgfsetfillcolor{currentfill}%
\pgfsetlinewidth{0.501875pt}%
\definecolor{currentstroke}{rgb}{0.000000,0.000000,0.000000}%
\pgfsetstrokecolor{currentstroke}%
\pgfsetdash{}{0pt}%
\pgfsys@defobject{currentmarker}{\pgfqpoint{0.000000in}{-0.069444in}}{\pgfqpoint{0.000000in}{0.000000in}}{%
\pgfpathmoveto{\pgfqpoint{0.000000in}{0.000000in}}%
\pgfpathlineto{\pgfqpoint{0.000000in}{-0.069444in}}%
\pgfusepath{stroke,fill}%
}%
\begin{pgfscope}%
\pgfsys@transformshift{0.780197in}{0.731012in}%
\pgfsys@useobject{currentmarker}{}%
\end{pgfscope}%
\end{pgfscope}%
\begin{pgfscope}%
\pgftext[x=0.780197in,y=0.371511in,,top]{\rmfamily\fontsize{8.000000}{9.600000}\selectfont 5000}%
\end{pgfscope}%
\begin{pgfscope}%
\pgfsetbuttcap%
\pgfsetroundjoin%
\definecolor{currentfill}{rgb}{0.000000,0.000000,0.000000}%
\pgfsetfillcolor{currentfill}%
\pgfsetlinewidth{0.501875pt}%
\definecolor{currentstroke}{rgb}{0.000000,0.000000,0.000000}%
\pgfsetstrokecolor{currentstroke}%
\pgfsetdash{}{0pt}%
\pgfsys@defobject{currentmarker}{\pgfqpoint{0.000000in}{0.000000in}}{\pgfqpoint{0.000000in}{0.069444in}}{%
\pgfpathmoveto{\pgfqpoint{0.000000in}{0.000000in}}%
\pgfpathlineto{\pgfqpoint{0.000000in}{0.069444in}}%
\pgfusepath{stroke,fill}%
}%
\begin{pgfscope}%
\pgfsys@transformshift{1.499404in}{0.440955in}%
\pgfsys@useobject{currentmarker}{}%
\end{pgfscope}%
\end{pgfscope}%
\begin{pgfscope}%
\pgfsetbuttcap%
\pgfsetroundjoin%
\definecolor{currentfill}{rgb}{0.000000,0.000000,0.000000}%
\pgfsetfillcolor{currentfill}%
\pgfsetlinewidth{0.501875pt}%
\definecolor{currentstroke}{rgb}{0.000000,0.000000,0.000000}%
\pgfsetstrokecolor{currentstroke}%
\pgfsetdash{}{0pt}%
\pgfsys@defobject{currentmarker}{\pgfqpoint{0.000000in}{-0.069444in}}{\pgfqpoint{0.000000in}{0.000000in}}{%
\pgfpathmoveto{\pgfqpoint{0.000000in}{0.000000in}}%
\pgfpathlineto{\pgfqpoint{0.000000in}{-0.069444in}}%
\pgfusepath{stroke,fill}%
}%
\begin{pgfscope}%
\pgfsys@transformshift{1.499404in}{0.731012in}%
\pgfsys@useobject{currentmarker}{}%
\end{pgfscope}%
\end{pgfscope}%
\begin{pgfscope}%
\pgftext[x=1.499404in,y=0.371511in,,top]{\rmfamily\fontsize{8.000000}{9.600000}\selectfont 5500}%
\end{pgfscope}%
\begin{pgfscope}%
\pgfsetbuttcap%
\pgfsetroundjoin%
\definecolor{currentfill}{rgb}{0.000000,0.000000,0.000000}%
\pgfsetfillcolor{currentfill}%
\pgfsetlinewidth{0.501875pt}%
\definecolor{currentstroke}{rgb}{0.000000,0.000000,0.000000}%
\pgfsetstrokecolor{currentstroke}%
\pgfsetdash{}{0pt}%
\pgfsys@defobject{currentmarker}{\pgfqpoint{0.000000in}{0.000000in}}{\pgfqpoint{0.000000in}{0.069444in}}{%
\pgfpathmoveto{\pgfqpoint{0.000000in}{0.000000in}}%
\pgfpathlineto{\pgfqpoint{0.000000in}{0.069444in}}%
\pgfusepath{stroke,fill}%
}%
\begin{pgfscope}%
\pgfsys@transformshift{2.218611in}{0.440955in}%
\pgfsys@useobject{currentmarker}{}%
\end{pgfscope}%
\end{pgfscope}%
\begin{pgfscope}%
\pgfsetbuttcap%
\pgfsetroundjoin%
\definecolor{currentfill}{rgb}{0.000000,0.000000,0.000000}%
\pgfsetfillcolor{currentfill}%
\pgfsetlinewidth{0.501875pt}%
\definecolor{currentstroke}{rgb}{0.000000,0.000000,0.000000}%
\pgfsetstrokecolor{currentstroke}%
\pgfsetdash{}{0pt}%
\pgfsys@defobject{currentmarker}{\pgfqpoint{0.000000in}{-0.069444in}}{\pgfqpoint{0.000000in}{0.000000in}}{%
\pgfpathmoveto{\pgfqpoint{0.000000in}{0.000000in}}%
\pgfpathlineto{\pgfqpoint{0.000000in}{-0.069444in}}%
\pgfusepath{stroke,fill}%
}%
\begin{pgfscope}%
\pgfsys@transformshift{2.218611in}{0.731012in}%
\pgfsys@useobject{currentmarker}{}%
\end{pgfscope}%
\end{pgfscope}%
\begin{pgfscope}%
\pgftext[x=2.218611in,y=0.371511in,,top]{\rmfamily\fontsize{8.000000}{9.600000}\selectfont 6000}%
\end{pgfscope}%
\begin{pgfscope}%
\pgfsetbuttcap%
\pgfsetroundjoin%
\definecolor{currentfill}{rgb}{0.000000,0.000000,0.000000}%
\pgfsetfillcolor{currentfill}%
\pgfsetlinewidth{0.501875pt}%
\definecolor{currentstroke}{rgb}{0.000000,0.000000,0.000000}%
\pgfsetstrokecolor{currentstroke}%
\pgfsetdash{}{0pt}%
\pgfsys@defobject{currentmarker}{\pgfqpoint{0.000000in}{0.000000in}}{\pgfqpoint{0.000000in}{0.069444in}}{%
\pgfpathmoveto{\pgfqpoint{0.000000in}{0.000000in}}%
\pgfpathlineto{\pgfqpoint{0.000000in}{0.069444in}}%
\pgfusepath{stroke,fill}%
}%
\begin{pgfscope}%
\pgfsys@transformshift{2.937818in}{0.440955in}%
\pgfsys@useobject{currentmarker}{}%
\end{pgfscope}%
\end{pgfscope}%
\begin{pgfscope}%
\pgfsetbuttcap%
\pgfsetroundjoin%
\definecolor{currentfill}{rgb}{0.000000,0.000000,0.000000}%
\pgfsetfillcolor{currentfill}%
\pgfsetlinewidth{0.501875pt}%
\definecolor{currentstroke}{rgb}{0.000000,0.000000,0.000000}%
\pgfsetstrokecolor{currentstroke}%
\pgfsetdash{}{0pt}%
\pgfsys@defobject{currentmarker}{\pgfqpoint{0.000000in}{-0.069444in}}{\pgfqpoint{0.000000in}{0.000000in}}{%
\pgfpathmoveto{\pgfqpoint{0.000000in}{0.000000in}}%
\pgfpathlineto{\pgfqpoint{0.000000in}{-0.069444in}}%
\pgfusepath{stroke,fill}%
}%
\begin{pgfscope}%
\pgfsys@transformshift{2.937818in}{0.731012in}%
\pgfsys@useobject{currentmarker}{}%
\end{pgfscope}%
\end{pgfscope}%
\begin{pgfscope}%
\pgftext[x=2.937818in,y=0.371511in,,top]{\rmfamily\fontsize{8.000000}{9.600000}\selectfont 6500}%
\end{pgfscope}%
\begin{pgfscope}%
\pgfsetbuttcap%
\pgfsetroundjoin%
\definecolor{currentfill}{rgb}{0.000000,0.000000,0.000000}%
\pgfsetfillcolor{currentfill}%
\pgfsetlinewidth{0.501875pt}%
\definecolor{currentstroke}{rgb}{0.000000,0.000000,0.000000}%
\pgfsetstrokecolor{currentstroke}%
\pgfsetdash{}{0pt}%
\pgfsys@defobject{currentmarker}{\pgfqpoint{0.000000in}{0.000000in}}{\pgfqpoint{0.000000in}{0.069444in}}{%
\pgfpathmoveto{\pgfqpoint{0.000000in}{0.000000in}}%
\pgfpathlineto{\pgfqpoint{0.000000in}{0.069444in}}%
\pgfusepath{stroke,fill}%
}%
\begin{pgfscope}%
\pgfsys@transformshift{3.657026in}{0.440955in}%
\pgfsys@useobject{currentmarker}{}%
\end{pgfscope}%
\end{pgfscope}%
\begin{pgfscope}%
\pgfsetbuttcap%
\pgfsetroundjoin%
\definecolor{currentfill}{rgb}{0.000000,0.000000,0.000000}%
\pgfsetfillcolor{currentfill}%
\pgfsetlinewidth{0.501875pt}%
\definecolor{currentstroke}{rgb}{0.000000,0.000000,0.000000}%
\pgfsetstrokecolor{currentstroke}%
\pgfsetdash{}{0pt}%
\pgfsys@defobject{currentmarker}{\pgfqpoint{0.000000in}{-0.069444in}}{\pgfqpoint{0.000000in}{0.000000in}}{%
\pgfpathmoveto{\pgfqpoint{0.000000in}{0.000000in}}%
\pgfpathlineto{\pgfqpoint{0.000000in}{-0.069444in}}%
\pgfusepath{stroke,fill}%
}%
\begin{pgfscope}%
\pgfsys@transformshift{3.657026in}{0.731012in}%
\pgfsys@useobject{currentmarker}{}%
\end{pgfscope}%
\end{pgfscope}%
\begin{pgfscope}%
\pgftext[x=3.657026in,y=0.371511in,,top]{\rmfamily\fontsize{8.000000}{9.600000}\selectfont 7000}%
\end{pgfscope}%
\begin{pgfscope}%
\pgftext[x=2.146691in,y=0.194536in,,top]{\rmfamily\fontsize{9.000000}{10.800000}\selectfont \(\displaystyle m(K^+\!\pi^-\!\mu^+\!\mu^-)\)}%
\end{pgfscope}%
\begin{pgfscope}%
\pgfsetbuttcap%
\pgfsetroundjoin%
\definecolor{currentfill}{rgb}{0.000000,0.000000,0.000000}%
\pgfsetfillcolor{currentfill}%
\pgfsetlinewidth{0.501875pt}%
\definecolor{currentstroke}{rgb}{0.000000,0.000000,0.000000}%
\pgfsetstrokecolor{currentstroke}%
\pgfsetdash{}{0pt}%
\pgfsys@defobject{currentmarker}{\pgfqpoint{0.000000in}{0.000000in}}{\pgfqpoint{0.069444in}{0.000000in}}{%
\pgfpathmoveto{\pgfqpoint{0.000000in}{0.000000in}}%
\pgfpathlineto{\pgfqpoint{0.069444in}{0.000000in}}%
\pgfusepath{stroke,fill}%
}%
\begin{pgfscope}%
\pgfsys@transformshift{0.636356in}{0.440955in}%
\pgfsys@useobject{currentmarker}{}%
\end{pgfscope}%
\end{pgfscope}%
\begin{pgfscope}%
\pgfsetbuttcap%
\pgfsetroundjoin%
\definecolor{currentfill}{rgb}{0.000000,0.000000,0.000000}%
\pgfsetfillcolor{currentfill}%
\pgfsetlinewidth{0.501875pt}%
\definecolor{currentstroke}{rgb}{0.000000,0.000000,0.000000}%
\pgfsetstrokecolor{currentstroke}%
\pgfsetdash{}{0pt}%
\pgfsys@defobject{currentmarker}{\pgfqpoint{-0.069444in}{0.000000in}}{\pgfqpoint{0.000000in}{0.000000in}}{%
\pgfpathmoveto{\pgfqpoint{0.000000in}{0.000000in}}%
\pgfpathlineto{\pgfqpoint{-0.069444in}{0.000000in}}%
\pgfusepath{stroke,fill}%
}%
\begin{pgfscope}%
\pgfsys@transformshift{3.657026in}{0.440955in}%
\pgfsys@useobject{currentmarker}{}%
\end{pgfscope}%
\end{pgfscope}%
\begin{pgfscope}%
\pgftext[x=0.566911in,y=0.440955in,right,]{\rmfamily\fontsize{8.000000}{9.600000}\selectfont −3}%
\end{pgfscope}%
\begin{pgfscope}%
\pgfsetbuttcap%
\pgfsetroundjoin%
\definecolor{currentfill}{rgb}{0.000000,0.000000,0.000000}%
\pgfsetfillcolor{currentfill}%
\pgfsetlinewidth{0.501875pt}%
\definecolor{currentstroke}{rgb}{0.000000,0.000000,0.000000}%
\pgfsetstrokecolor{currentstroke}%
\pgfsetdash{}{0pt}%
\pgfsys@defobject{currentmarker}{\pgfqpoint{0.000000in}{0.000000in}}{\pgfqpoint{0.069444in}{0.000000in}}{%
\pgfpathmoveto{\pgfqpoint{0.000000in}{0.000000in}}%
\pgfpathlineto{\pgfqpoint{0.069444in}{0.000000in}}%
\pgfusepath{stroke,fill}%
}%
\begin{pgfscope}%
\pgfsys@transformshift{0.636356in}{0.585984in}%
\pgfsys@useobject{currentmarker}{}%
\end{pgfscope}%
\end{pgfscope}%
\begin{pgfscope}%
\pgfsetbuttcap%
\pgfsetroundjoin%
\definecolor{currentfill}{rgb}{0.000000,0.000000,0.000000}%
\pgfsetfillcolor{currentfill}%
\pgfsetlinewidth{0.501875pt}%
\definecolor{currentstroke}{rgb}{0.000000,0.000000,0.000000}%
\pgfsetstrokecolor{currentstroke}%
\pgfsetdash{}{0pt}%
\pgfsys@defobject{currentmarker}{\pgfqpoint{-0.069444in}{0.000000in}}{\pgfqpoint{0.000000in}{0.000000in}}{%
\pgfpathmoveto{\pgfqpoint{0.000000in}{0.000000in}}%
\pgfpathlineto{\pgfqpoint{-0.069444in}{0.000000in}}%
\pgfusepath{stroke,fill}%
}%
\begin{pgfscope}%
\pgfsys@transformshift{3.657026in}{0.585984in}%
\pgfsys@useobject{currentmarker}{}%
\end{pgfscope}%
\end{pgfscope}%
\begin{pgfscope}%
\pgftext[x=0.566911in,y=0.585984in,right,]{\rmfamily\fontsize{8.000000}{9.600000}\selectfont 0}%
\end{pgfscope}%
\begin{pgfscope}%
\pgfsetbuttcap%
\pgfsetroundjoin%
\definecolor{currentfill}{rgb}{0.000000,0.000000,0.000000}%
\pgfsetfillcolor{currentfill}%
\pgfsetlinewidth{0.501875pt}%
\definecolor{currentstroke}{rgb}{0.000000,0.000000,0.000000}%
\pgfsetstrokecolor{currentstroke}%
\pgfsetdash{}{0pt}%
\pgfsys@defobject{currentmarker}{\pgfqpoint{0.000000in}{0.000000in}}{\pgfqpoint{0.069444in}{0.000000in}}{%
\pgfpathmoveto{\pgfqpoint{0.000000in}{0.000000in}}%
\pgfpathlineto{\pgfqpoint{0.069444in}{0.000000in}}%
\pgfusepath{stroke,fill}%
}%
\begin{pgfscope}%
\pgfsys@transformshift{0.636356in}{0.731012in}%
\pgfsys@useobject{currentmarker}{}%
\end{pgfscope}%
\end{pgfscope}%
\begin{pgfscope}%
\pgfsetbuttcap%
\pgfsetroundjoin%
\definecolor{currentfill}{rgb}{0.000000,0.000000,0.000000}%
\pgfsetfillcolor{currentfill}%
\pgfsetlinewidth{0.501875pt}%
\definecolor{currentstroke}{rgb}{0.000000,0.000000,0.000000}%
\pgfsetstrokecolor{currentstroke}%
\pgfsetdash{}{0pt}%
\pgfsys@defobject{currentmarker}{\pgfqpoint{-0.069444in}{0.000000in}}{\pgfqpoint{0.000000in}{0.000000in}}{%
\pgfpathmoveto{\pgfqpoint{0.000000in}{0.000000in}}%
\pgfpathlineto{\pgfqpoint{-0.069444in}{0.000000in}}%
\pgfusepath{stroke,fill}%
}%
\begin{pgfscope}%
\pgfsys@transformshift{3.657026in}{0.731012in}%
\pgfsys@useobject{currentmarker}{}%
\end{pgfscope}%
\end{pgfscope}%
\begin{pgfscope}%
\pgftext[x=0.566911in,y=0.731012in,right,]{\rmfamily\fontsize{8.000000}{9.600000}\selectfont 3}%
\end{pgfscope}%
\begin{pgfscope}%
\pgftext[x=0.333676in,y=0.585984in,,bottom,rotate=90.000000]{\rmfamily\fontsize{9.000000}{10.800000}\selectfont \(\displaystyle \frac{\hat{n}_i -  n_i}{\sigma(n_i)}\)}%
\end{pgfscope}%
\begin{pgfscope}%
\pgfsetbuttcap%
\pgfsetmiterjoin%
\definecolor{currentfill}{rgb}{1.000000,1.000000,1.000000}%
\pgfsetfillcolor{currentfill}%
\pgfsetlinewidth{0.000000pt}%
\definecolor{currentstroke}{rgb}{0.000000,0.000000,0.000000}%
\pgfsetstrokecolor{currentstroke}%
\pgfsetstrokeopacity{0.000000}%
\pgfsetdash{}{0pt}%
\pgfpathmoveto{\pgfqpoint{0.636356in}{0.905046in}}%
\pgfpathlineto{\pgfqpoint{3.657026in}{0.905046in}}%
\pgfpathlineto{\pgfqpoint{3.657026in}{2.935444in}}%
\pgfpathlineto{\pgfqpoint{0.636356in}{2.935444in}}%
\pgfpathclose%
\pgfusepath{fill}%
\end{pgfscope}%
\begin{pgfscope}%
\pgfsetrectcap%
\pgfsetmiterjoin%
\pgfsetlinewidth{1.003750pt}%
\definecolor{currentstroke}{rgb}{0.000000,0.000000,0.000000}%
\pgfsetstrokecolor{currentstroke}%
\pgfsetdash{}{0pt}%
\pgfpathmoveto{\pgfqpoint{0.636356in}{2.935444in}}%
\pgfpathlineto{\pgfqpoint{3.657026in}{2.935444in}}%
\pgfusepath{stroke}%
\end{pgfscope}%
\begin{pgfscope}%
\pgfsetrectcap%
\pgfsetmiterjoin%
\pgfsetlinewidth{1.003750pt}%
\definecolor{currentstroke}{rgb}{0.000000,0.000000,0.000000}%
\pgfsetstrokecolor{currentstroke}%
\pgfsetdash{}{0pt}%
\pgfpathmoveto{\pgfqpoint{3.657026in}{0.905046in}}%
\pgfpathlineto{\pgfqpoint{3.657026in}{2.935444in}}%
\pgfusepath{stroke}%
\end{pgfscope}%
\begin{pgfscope}%
\pgfsetrectcap%
\pgfsetmiterjoin%
\pgfsetlinewidth{1.003750pt}%
\definecolor{currentstroke}{rgb}{0.000000,0.000000,0.000000}%
\pgfsetstrokecolor{currentstroke}%
\pgfsetdash{}{0pt}%
\pgfpathmoveto{\pgfqpoint{0.636356in}{0.905046in}}%
\pgfpathlineto{\pgfqpoint{3.657026in}{0.905046in}}%
\pgfusepath{stroke}%
\end{pgfscope}%
\begin{pgfscope}%
\pgfsetrectcap%
\pgfsetmiterjoin%
\pgfsetlinewidth{1.003750pt}%
\definecolor{currentstroke}{rgb}{0.000000,0.000000,0.000000}%
\pgfsetstrokecolor{currentstroke}%
\pgfsetdash{}{0pt}%
\pgfpathmoveto{\pgfqpoint{0.636356in}{0.905046in}}%
\pgfpathlineto{\pgfqpoint{0.636356in}{2.935444in}}%
\pgfusepath{stroke}%
\end{pgfscope}%
\begin{pgfscope}%
\pgfsetbuttcap%
\pgfsetroundjoin%
\definecolor{currentfill}{rgb}{0.000000,0.000000,0.000000}%
\pgfsetfillcolor{currentfill}%
\pgfsetlinewidth{0.501875pt}%
\definecolor{currentstroke}{rgb}{0.000000,0.000000,0.000000}%
\pgfsetstrokecolor{currentstroke}%
\pgfsetdash{}{0pt}%
\pgfsys@defobject{currentmarker}{\pgfqpoint{0.000000in}{0.000000in}}{\pgfqpoint{0.000000in}{0.069444in}}{%
\pgfpathmoveto{\pgfqpoint{0.000000in}{0.000000in}}%
\pgfpathlineto{\pgfqpoint{0.000000in}{0.069444in}}%
\pgfusepath{stroke,fill}%
}%
\begin{pgfscope}%
\pgfsys@transformshift{0.780197in}{0.905046in}%
\pgfsys@useobject{currentmarker}{}%
\end{pgfscope}%
\end{pgfscope}%
\begin{pgfscope}%
\pgfsetbuttcap%
\pgfsetroundjoin%
\definecolor{currentfill}{rgb}{0.000000,0.000000,0.000000}%
\pgfsetfillcolor{currentfill}%
\pgfsetlinewidth{0.501875pt}%
\definecolor{currentstroke}{rgb}{0.000000,0.000000,0.000000}%
\pgfsetstrokecolor{currentstroke}%
\pgfsetdash{}{0pt}%
\pgfsys@defobject{currentmarker}{\pgfqpoint{0.000000in}{-0.069444in}}{\pgfqpoint{0.000000in}{0.000000in}}{%
\pgfpathmoveto{\pgfqpoint{0.000000in}{0.000000in}}%
\pgfpathlineto{\pgfqpoint{0.000000in}{-0.069444in}}%
\pgfusepath{stroke,fill}%
}%
\begin{pgfscope}%
\pgfsys@transformshift{0.780197in}{2.935444in}%
\pgfsys@useobject{currentmarker}{}%
\end{pgfscope}%
\end{pgfscope}%
\begin{pgfscope}%
\pgfsetbuttcap%
\pgfsetroundjoin%
\definecolor{currentfill}{rgb}{0.000000,0.000000,0.000000}%
\pgfsetfillcolor{currentfill}%
\pgfsetlinewidth{0.501875pt}%
\definecolor{currentstroke}{rgb}{0.000000,0.000000,0.000000}%
\pgfsetstrokecolor{currentstroke}%
\pgfsetdash{}{0pt}%
\pgfsys@defobject{currentmarker}{\pgfqpoint{0.000000in}{0.000000in}}{\pgfqpoint{0.000000in}{0.069444in}}{%
\pgfpathmoveto{\pgfqpoint{0.000000in}{0.000000in}}%
\pgfpathlineto{\pgfqpoint{0.000000in}{0.069444in}}%
\pgfusepath{stroke,fill}%
}%
\begin{pgfscope}%
\pgfsys@transformshift{1.499404in}{0.905046in}%
\pgfsys@useobject{currentmarker}{}%
\end{pgfscope}%
\end{pgfscope}%
\begin{pgfscope}%
\pgfsetbuttcap%
\pgfsetroundjoin%
\definecolor{currentfill}{rgb}{0.000000,0.000000,0.000000}%
\pgfsetfillcolor{currentfill}%
\pgfsetlinewidth{0.501875pt}%
\definecolor{currentstroke}{rgb}{0.000000,0.000000,0.000000}%
\pgfsetstrokecolor{currentstroke}%
\pgfsetdash{}{0pt}%
\pgfsys@defobject{currentmarker}{\pgfqpoint{0.000000in}{-0.069444in}}{\pgfqpoint{0.000000in}{0.000000in}}{%
\pgfpathmoveto{\pgfqpoint{0.000000in}{0.000000in}}%
\pgfpathlineto{\pgfqpoint{0.000000in}{-0.069444in}}%
\pgfusepath{stroke,fill}%
}%
\begin{pgfscope}%
\pgfsys@transformshift{1.499404in}{2.935444in}%
\pgfsys@useobject{currentmarker}{}%
\end{pgfscope}%
\end{pgfscope}%
\begin{pgfscope}%
\pgfsetbuttcap%
\pgfsetroundjoin%
\definecolor{currentfill}{rgb}{0.000000,0.000000,0.000000}%
\pgfsetfillcolor{currentfill}%
\pgfsetlinewidth{0.501875pt}%
\definecolor{currentstroke}{rgb}{0.000000,0.000000,0.000000}%
\pgfsetstrokecolor{currentstroke}%
\pgfsetdash{}{0pt}%
\pgfsys@defobject{currentmarker}{\pgfqpoint{0.000000in}{0.000000in}}{\pgfqpoint{0.000000in}{0.069444in}}{%
\pgfpathmoveto{\pgfqpoint{0.000000in}{0.000000in}}%
\pgfpathlineto{\pgfqpoint{0.000000in}{0.069444in}}%
\pgfusepath{stroke,fill}%
}%
\begin{pgfscope}%
\pgfsys@transformshift{2.218611in}{0.905046in}%
\pgfsys@useobject{currentmarker}{}%
\end{pgfscope}%
\end{pgfscope}%
\begin{pgfscope}%
\pgfsetbuttcap%
\pgfsetroundjoin%
\definecolor{currentfill}{rgb}{0.000000,0.000000,0.000000}%
\pgfsetfillcolor{currentfill}%
\pgfsetlinewidth{0.501875pt}%
\definecolor{currentstroke}{rgb}{0.000000,0.000000,0.000000}%
\pgfsetstrokecolor{currentstroke}%
\pgfsetdash{}{0pt}%
\pgfsys@defobject{currentmarker}{\pgfqpoint{0.000000in}{-0.069444in}}{\pgfqpoint{0.000000in}{0.000000in}}{%
\pgfpathmoveto{\pgfqpoint{0.000000in}{0.000000in}}%
\pgfpathlineto{\pgfqpoint{0.000000in}{-0.069444in}}%
\pgfusepath{stroke,fill}%
}%
\begin{pgfscope}%
\pgfsys@transformshift{2.218611in}{2.935444in}%
\pgfsys@useobject{currentmarker}{}%
\end{pgfscope}%
\end{pgfscope}%
\begin{pgfscope}%
\pgfsetbuttcap%
\pgfsetroundjoin%
\definecolor{currentfill}{rgb}{0.000000,0.000000,0.000000}%
\pgfsetfillcolor{currentfill}%
\pgfsetlinewidth{0.501875pt}%
\definecolor{currentstroke}{rgb}{0.000000,0.000000,0.000000}%
\pgfsetstrokecolor{currentstroke}%
\pgfsetdash{}{0pt}%
\pgfsys@defobject{currentmarker}{\pgfqpoint{0.000000in}{0.000000in}}{\pgfqpoint{0.000000in}{0.069444in}}{%
\pgfpathmoveto{\pgfqpoint{0.000000in}{0.000000in}}%
\pgfpathlineto{\pgfqpoint{0.000000in}{0.069444in}}%
\pgfusepath{stroke,fill}%
}%
\begin{pgfscope}%
\pgfsys@transformshift{2.937818in}{0.905046in}%
\pgfsys@useobject{currentmarker}{}%
\end{pgfscope}%
\end{pgfscope}%
\begin{pgfscope}%
\pgfsetbuttcap%
\pgfsetroundjoin%
\definecolor{currentfill}{rgb}{0.000000,0.000000,0.000000}%
\pgfsetfillcolor{currentfill}%
\pgfsetlinewidth{0.501875pt}%
\definecolor{currentstroke}{rgb}{0.000000,0.000000,0.000000}%
\pgfsetstrokecolor{currentstroke}%
\pgfsetdash{}{0pt}%
\pgfsys@defobject{currentmarker}{\pgfqpoint{0.000000in}{-0.069444in}}{\pgfqpoint{0.000000in}{0.000000in}}{%
\pgfpathmoveto{\pgfqpoint{0.000000in}{0.000000in}}%
\pgfpathlineto{\pgfqpoint{0.000000in}{-0.069444in}}%
\pgfusepath{stroke,fill}%
}%
\begin{pgfscope}%
\pgfsys@transformshift{2.937818in}{2.935444in}%
\pgfsys@useobject{currentmarker}{}%
\end{pgfscope}%
\end{pgfscope}%
\begin{pgfscope}%
\pgfsetbuttcap%
\pgfsetroundjoin%
\definecolor{currentfill}{rgb}{0.000000,0.000000,0.000000}%
\pgfsetfillcolor{currentfill}%
\pgfsetlinewidth{0.501875pt}%
\definecolor{currentstroke}{rgb}{0.000000,0.000000,0.000000}%
\pgfsetstrokecolor{currentstroke}%
\pgfsetdash{}{0pt}%
\pgfsys@defobject{currentmarker}{\pgfqpoint{0.000000in}{0.000000in}}{\pgfqpoint{0.000000in}{0.069444in}}{%
\pgfpathmoveto{\pgfqpoint{0.000000in}{0.000000in}}%
\pgfpathlineto{\pgfqpoint{0.000000in}{0.069444in}}%
\pgfusepath{stroke,fill}%
}%
\begin{pgfscope}%
\pgfsys@transformshift{3.657026in}{0.905046in}%
\pgfsys@useobject{currentmarker}{}%
\end{pgfscope}%
\end{pgfscope}%
\begin{pgfscope}%
\pgfsetbuttcap%
\pgfsetroundjoin%
\definecolor{currentfill}{rgb}{0.000000,0.000000,0.000000}%
\pgfsetfillcolor{currentfill}%
\pgfsetlinewidth{0.501875pt}%
\definecolor{currentstroke}{rgb}{0.000000,0.000000,0.000000}%
\pgfsetstrokecolor{currentstroke}%
\pgfsetdash{}{0pt}%
\pgfsys@defobject{currentmarker}{\pgfqpoint{0.000000in}{-0.069444in}}{\pgfqpoint{0.000000in}{0.000000in}}{%
\pgfpathmoveto{\pgfqpoint{0.000000in}{0.000000in}}%
\pgfpathlineto{\pgfqpoint{0.000000in}{-0.069444in}}%
\pgfusepath{stroke,fill}%
}%
\begin{pgfscope}%
\pgfsys@transformshift{3.657026in}{2.935444in}%
\pgfsys@useobject{currentmarker}{}%
\end{pgfscope}%
\end{pgfscope}%
\begin{pgfscope}%
\pgfsetbuttcap%
\pgfsetroundjoin%
\definecolor{currentfill}{rgb}{0.000000,0.000000,0.000000}%
\pgfsetfillcolor{currentfill}%
\pgfsetlinewidth{0.501875pt}%
\definecolor{currentstroke}{rgb}{0.000000,0.000000,0.000000}%
\pgfsetstrokecolor{currentstroke}%
\pgfsetdash{}{0pt}%
\pgfsys@defobject{currentmarker}{\pgfqpoint{0.000000in}{0.000000in}}{\pgfqpoint{0.069444in}{0.000000in}}{%
\pgfpathmoveto{\pgfqpoint{0.000000in}{0.000000in}}%
\pgfpathlineto{\pgfqpoint{0.069444in}{0.000000in}}%
\pgfusepath{stroke,fill}%
}%
\begin{pgfscope}%
\pgfsys@transformshift{0.636356in}{0.905046in}%
\pgfsys@useobject{currentmarker}{}%
\end{pgfscope}%
\end{pgfscope}%
\begin{pgfscope}%
\pgfsetbuttcap%
\pgfsetroundjoin%
\definecolor{currentfill}{rgb}{0.000000,0.000000,0.000000}%
\pgfsetfillcolor{currentfill}%
\pgfsetlinewidth{0.501875pt}%
\definecolor{currentstroke}{rgb}{0.000000,0.000000,0.000000}%
\pgfsetstrokecolor{currentstroke}%
\pgfsetdash{}{0pt}%
\pgfsys@defobject{currentmarker}{\pgfqpoint{-0.069444in}{0.000000in}}{\pgfqpoint{0.000000in}{0.000000in}}{%
\pgfpathmoveto{\pgfqpoint{0.000000in}{0.000000in}}%
\pgfpathlineto{\pgfqpoint{-0.069444in}{0.000000in}}%
\pgfusepath{stroke,fill}%
}%
\begin{pgfscope}%
\pgfsys@transformshift{3.657026in}{0.905046in}%
\pgfsys@useobject{currentmarker}{}%
\end{pgfscope}%
\end{pgfscope}%
\begin{pgfscope}%
\pgftext[x=0.566911in,y=0.905046in,right,]{\rmfamily\fontsize{8.000000}{9.600000}\selectfont 0}%
\end{pgfscope}%
\begin{pgfscope}%
\pgfsetbuttcap%
\pgfsetroundjoin%
\definecolor{currentfill}{rgb}{0.000000,0.000000,0.000000}%
\pgfsetfillcolor{currentfill}%
\pgfsetlinewidth{0.501875pt}%
\definecolor{currentstroke}{rgb}{0.000000,0.000000,0.000000}%
\pgfsetstrokecolor{currentstroke}%
\pgfsetdash{}{0pt}%
\pgfsys@defobject{currentmarker}{\pgfqpoint{0.000000in}{0.000000in}}{\pgfqpoint{0.069444in}{0.000000in}}{%
\pgfpathmoveto{\pgfqpoint{0.000000in}{0.000000in}}%
\pgfpathlineto{\pgfqpoint{0.069444in}{0.000000in}}%
\pgfusepath{stroke,fill}%
}%
\begin{pgfscope}%
\pgfsys@transformshift{0.636356in}{1.195103in}%
\pgfsys@useobject{currentmarker}{}%
\end{pgfscope}%
\end{pgfscope}%
\begin{pgfscope}%
\pgfsetbuttcap%
\pgfsetroundjoin%
\definecolor{currentfill}{rgb}{0.000000,0.000000,0.000000}%
\pgfsetfillcolor{currentfill}%
\pgfsetlinewidth{0.501875pt}%
\definecolor{currentstroke}{rgb}{0.000000,0.000000,0.000000}%
\pgfsetstrokecolor{currentstroke}%
\pgfsetdash{}{0pt}%
\pgfsys@defobject{currentmarker}{\pgfqpoint{-0.069444in}{0.000000in}}{\pgfqpoint{0.000000in}{0.000000in}}{%
\pgfpathmoveto{\pgfqpoint{0.000000in}{0.000000in}}%
\pgfpathlineto{\pgfqpoint{-0.069444in}{0.000000in}}%
\pgfusepath{stroke,fill}%
}%
\begin{pgfscope}%
\pgfsys@transformshift{3.657026in}{1.195103in}%
\pgfsys@useobject{currentmarker}{}%
\end{pgfscope}%
\end{pgfscope}%
\begin{pgfscope}%
\pgftext[x=0.566911in,y=1.195103in,right,]{\rmfamily\fontsize{8.000000}{9.600000}\selectfont 1000}%
\end{pgfscope}%
\begin{pgfscope}%
\pgfsetbuttcap%
\pgfsetroundjoin%
\definecolor{currentfill}{rgb}{0.000000,0.000000,0.000000}%
\pgfsetfillcolor{currentfill}%
\pgfsetlinewidth{0.501875pt}%
\definecolor{currentstroke}{rgb}{0.000000,0.000000,0.000000}%
\pgfsetstrokecolor{currentstroke}%
\pgfsetdash{}{0pt}%
\pgfsys@defobject{currentmarker}{\pgfqpoint{0.000000in}{0.000000in}}{\pgfqpoint{0.069444in}{0.000000in}}{%
\pgfpathmoveto{\pgfqpoint{0.000000in}{0.000000in}}%
\pgfpathlineto{\pgfqpoint{0.069444in}{0.000000in}}%
\pgfusepath{stroke,fill}%
}%
\begin{pgfscope}%
\pgfsys@transformshift{0.636356in}{1.485160in}%
\pgfsys@useobject{currentmarker}{}%
\end{pgfscope}%
\end{pgfscope}%
\begin{pgfscope}%
\pgfsetbuttcap%
\pgfsetroundjoin%
\definecolor{currentfill}{rgb}{0.000000,0.000000,0.000000}%
\pgfsetfillcolor{currentfill}%
\pgfsetlinewidth{0.501875pt}%
\definecolor{currentstroke}{rgb}{0.000000,0.000000,0.000000}%
\pgfsetstrokecolor{currentstroke}%
\pgfsetdash{}{0pt}%
\pgfsys@defobject{currentmarker}{\pgfqpoint{-0.069444in}{0.000000in}}{\pgfqpoint{0.000000in}{0.000000in}}{%
\pgfpathmoveto{\pgfqpoint{0.000000in}{0.000000in}}%
\pgfpathlineto{\pgfqpoint{-0.069444in}{0.000000in}}%
\pgfusepath{stroke,fill}%
}%
\begin{pgfscope}%
\pgfsys@transformshift{3.657026in}{1.485160in}%
\pgfsys@useobject{currentmarker}{}%
\end{pgfscope}%
\end{pgfscope}%
\begin{pgfscope}%
\pgftext[x=0.566911in,y=1.485160in,right,]{\rmfamily\fontsize{8.000000}{9.600000}\selectfont 2000}%
\end{pgfscope}%
\begin{pgfscope}%
\pgfsetbuttcap%
\pgfsetroundjoin%
\definecolor{currentfill}{rgb}{0.000000,0.000000,0.000000}%
\pgfsetfillcolor{currentfill}%
\pgfsetlinewidth{0.501875pt}%
\definecolor{currentstroke}{rgb}{0.000000,0.000000,0.000000}%
\pgfsetstrokecolor{currentstroke}%
\pgfsetdash{}{0pt}%
\pgfsys@defobject{currentmarker}{\pgfqpoint{0.000000in}{0.000000in}}{\pgfqpoint{0.069444in}{0.000000in}}{%
\pgfpathmoveto{\pgfqpoint{0.000000in}{0.000000in}}%
\pgfpathlineto{\pgfqpoint{0.069444in}{0.000000in}}%
\pgfusepath{stroke,fill}%
}%
\begin{pgfscope}%
\pgfsys@transformshift{0.636356in}{1.775217in}%
\pgfsys@useobject{currentmarker}{}%
\end{pgfscope}%
\end{pgfscope}%
\begin{pgfscope}%
\pgfsetbuttcap%
\pgfsetroundjoin%
\definecolor{currentfill}{rgb}{0.000000,0.000000,0.000000}%
\pgfsetfillcolor{currentfill}%
\pgfsetlinewidth{0.501875pt}%
\definecolor{currentstroke}{rgb}{0.000000,0.000000,0.000000}%
\pgfsetstrokecolor{currentstroke}%
\pgfsetdash{}{0pt}%
\pgfsys@defobject{currentmarker}{\pgfqpoint{-0.069444in}{0.000000in}}{\pgfqpoint{0.000000in}{0.000000in}}{%
\pgfpathmoveto{\pgfqpoint{0.000000in}{0.000000in}}%
\pgfpathlineto{\pgfqpoint{-0.069444in}{0.000000in}}%
\pgfusepath{stroke,fill}%
}%
\begin{pgfscope}%
\pgfsys@transformshift{3.657026in}{1.775217in}%
\pgfsys@useobject{currentmarker}{}%
\end{pgfscope}%
\end{pgfscope}%
\begin{pgfscope}%
\pgftext[x=0.566911in,y=1.775217in,right,]{\rmfamily\fontsize{8.000000}{9.600000}\selectfont 3000}%
\end{pgfscope}%
\begin{pgfscope}%
\pgfsetbuttcap%
\pgfsetroundjoin%
\definecolor{currentfill}{rgb}{0.000000,0.000000,0.000000}%
\pgfsetfillcolor{currentfill}%
\pgfsetlinewidth{0.501875pt}%
\definecolor{currentstroke}{rgb}{0.000000,0.000000,0.000000}%
\pgfsetstrokecolor{currentstroke}%
\pgfsetdash{}{0pt}%
\pgfsys@defobject{currentmarker}{\pgfqpoint{0.000000in}{0.000000in}}{\pgfqpoint{0.069444in}{0.000000in}}{%
\pgfpathmoveto{\pgfqpoint{0.000000in}{0.000000in}}%
\pgfpathlineto{\pgfqpoint{0.069444in}{0.000000in}}%
\pgfusepath{stroke,fill}%
}%
\begin{pgfscope}%
\pgfsys@transformshift{0.636356in}{2.065274in}%
\pgfsys@useobject{currentmarker}{}%
\end{pgfscope}%
\end{pgfscope}%
\begin{pgfscope}%
\pgfsetbuttcap%
\pgfsetroundjoin%
\definecolor{currentfill}{rgb}{0.000000,0.000000,0.000000}%
\pgfsetfillcolor{currentfill}%
\pgfsetlinewidth{0.501875pt}%
\definecolor{currentstroke}{rgb}{0.000000,0.000000,0.000000}%
\pgfsetstrokecolor{currentstroke}%
\pgfsetdash{}{0pt}%
\pgfsys@defobject{currentmarker}{\pgfqpoint{-0.069444in}{0.000000in}}{\pgfqpoint{0.000000in}{0.000000in}}{%
\pgfpathmoveto{\pgfqpoint{0.000000in}{0.000000in}}%
\pgfpathlineto{\pgfqpoint{-0.069444in}{0.000000in}}%
\pgfusepath{stroke,fill}%
}%
\begin{pgfscope}%
\pgfsys@transformshift{3.657026in}{2.065274in}%
\pgfsys@useobject{currentmarker}{}%
\end{pgfscope}%
\end{pgfscope}%
\begin{pgfscope}%
\pgftext[x=0.566911in,y=2.065274in,right,]{\rmfamily\fontsize{8.000000}{9.600000}\selectfont 4000}%
\end{pgfscope}%
\begin{pgfscope}%
\pgfsetbuttcap%
\pgfsetroundjoin%
\definecolor{currentfill}{rgb}{0.000000,0.000000,0.000000}%
\pgfsetfillcolor{currentfill}%
\pgfsetlinewidth{0.501875pt}%
\definecolor{currentstroke}{rgb}{0.000000,0.000000,0.000000}%
\pgfsetstrokecolor{currentstroke}%
\pgfsetdash{}{0pt}%
\pgfsys@defobject{currentmarker}{\pgfqpoint{0.000000in}{0.000000in}}{\pgfqpoint{0.069444in}{0.000000in}}{%
\pgfpathmoveto{\pgfqpoint{0.000000in}{0.000000in}}%
\pgfpathlineto{\pgfqpoint{0.069444in}{0.000000in}}%
\pgfusepath{stroke,fill}%
}%
\begin{pgfscope}%
\pgfsys@transformshift{0.636356in}{2.355330in}%
\pgfsys@useobject{currentmarker}{}%
\end{pgfscope}%
\end{pgfscope}%
\begin{pgfscope}%
\pgfsetbuttcap%
\pgfsetroundjoin%
\definecolor{currentfill}{rgb}{0.000000,0.000000,0.000000}%
\pgfsetfillcolor{currentfill}%
\pgfsetlinewidth{0.501875pt}%
\definecolor{currentstroke}{rgb}{0.000000,0.000000,0.000000}%
\pgfsetstrokecolor{currentstroke}%
\pgfsetdash{}{0pt}%
\pgfsys@defobject{currentmarker}{\pgfqpoint{-0.069444in}{0.000000in}}{\pgfqpoint{0.000000in}{0.000000in}}{%
\pgfpathmoveto{\pgfqpoint{0.000000in}{0.000000in}}%
\pgfpathlineto{\pgfqpoint{-0.069444in}{0.000000in}}%
\pgfusepath{stroke,fill}%
}%
\begin{pgfscope}%
\pgfsys@transformshift{3.657026in}{2.355330in}%
\pgfsys@useobject{currentmarker}{}%
\end{pgfscope}%
\end{pgfscope}%
\begin{pgfscope}%
\pgftext[x=0.566911in,y=2.355330in,right,]{\rmfamily\fontsize{8.000000}{9.600000}\selectfont 5000}%
\end{pgfscope}%
\begin{pgfscope}%
\pgfsetbuttcap%
\pgfsetroundjoin%
\definecolor{currentfill}{rgb}{0.000000,0.000000,0.000000}%
\pgfsetfillcolor{currentfill}%
\pgfsetlinewidth{0.501875pt}%
\definecolor{currentstroke}{rgb}{0.000000,0.000000,0.000000}%
\pgfsetstrokecolor{currentstroke}%
\pgfsetdash{}{0pt}%
\pgfsys@defobject{currentmarker}{\pgfqpoint{0.000000in}{0.000000in}}{\pgfqpoint{0.069444in}{0.000000in}}{%
\pgfpathmoveto{\pgfqpoint{0.000000in}{0.000000in}}%
\pgfpathlineto{\pgfqpoint{0.069444in}{0.000000in}}%
\pgfusepath{stroke,fill}%
}%
\begin{pgfscope}%
\pgfsys@transformshift{0.636356in}{2.645387in}%
\pgfsys@useobject{currentmarker}{}%
\end{pgfscope}%
\end{pgfscope}%
\begin{pgfscope}%
\pgfsetbuttcap%
\pgfsetroundjoin%
\definecolor{currentfill}{rgb}{0.000000,0.000000,0.000000}%
\pgfsetfillcolor{currentfill}%
\pgfsetlinewidth{0.501875pt}%
\definecolor{currentstroke}{rgb}{0.000000,0.000000,0.000000}%
\pgfsetstrokecolor{currentstroke}%
\pgfsetdash{}{0pt}%
\pgfsys@defobject{currentmarker}{\pgfqpoint{-0.069444in}{0.000000in}}{\pgfqpoint{0.000000in}{0.000000in}}{%
\pgfpathmoveto{\pgfqpoint{0.000000in}{0.000000in}}%
\pgfpathlineto{\pgfqpoint{-0.069444in}{0.000000in}}%
\pgfusepath{stroke,fill}%
}%
\begin{pgfscope}%
\pgfsys@transformshift{3.657026in}{2.645387in}%
\pgfsys@useobject{currentmarker}{}%
\end{pgfscope}%
\end{pgfscope}%
\begin{pgfscope}%
\pgftext[x=0.566911in,y=2.645387in,right,]{\rmfamily\fontsize{8.000000}{9.600000}\selectfont 6000}%
\end{pgfscope}%
\begin{pgfscope}%
\pgfsetbuttcap%
\pgfsetroundjoin%
\definecolor{currentfill}{rgb}{0.000000,0.000000,0.000000}%
\pgfsetfillcolor{currentfill}%
\pgfsetlinewidth{0.501875pt}%
\definecolor{currentstroke}{rgb}{0.000000,0.000000,0.000000}%
\pgfsetstrokecolor{currentstroke}%
\pgfsetdash{}{0pt}%
\pgfsys@defobject{currentmarker}{\pgfqpoint{0.000000in}{0.000000in}}{\pgfqpoint{0.069444in}{0.000000in}}{%
\pgfpathmoveto{\pgfqpoint{0.000000in}{0.000000in}}%
\pgfpathlineto{\pgfqpoint{0.069444in}{0.000000in}}%
\pgfusepath{stroke,fill}%
}%
\begin{pgfscope}%
\pgfsys@transformshift{0.636356in}{2.935444in}%
\pgfsys@useobject{currentmarker}{}%
\end{pgfscope}%
\end{pgfscope}%
\begin{pgfscope}%
\pgfsetbuttcap%
\pgfsetroundjoin%
\definecolor{currentfill}{rgb}{0.000000,0.000000,0.000000}%
\pgfsetfillcolor{currentfill}%
\pgfsetlinewidth{0.501875pt}%
\definecolor{currentstroke}{rgb}{0.000000,0.000000,0.000000}%
\pgfsetstrokecolor{currentstroke}%
\pgfsetdash{}{0pt}%
\pgfsys@defobject{currentmarker}{\pgfqpoint{-0.069444in}{0.000000in}}{\pgfqpoint{0.000000in}{0.000000in}}{%
\pgfpathmoveto{\pgfqpoint{0.000000in}{0.000000in}}%
\pgfpathlineto{\pgfqpoint{-0.069444in}{0.000000in}}%
\pgfusepath{stroke,fill}%
}%
\begin{pgfscope}%
\pgfsys@transformshift{3.657026in}{2.935444in}%
\pgfsys@useobject{currentmarker}{}%
\end{pgfscope}%
\end{pgfscope}%
\begin{pgfscope}%
\pgftext[x=0.566911in,y=2.935444in,right,]{\rmfamily\fontsize{8.000000}{9.600000}\selectfont 7000}%
\end{pgfscope}%
\begin{pgfscope}%
\pgftext[x=0.214698in,y=1.920245in,,bottom,rotate=90.000000]{\rmfamily\fontsize{9.000000}{10.800000}\selectfont Candidates}%
\end{pgfscope}%
\begin{pgfscope}%
\pgfsetrectcap%
\pgfsetroundjoin%
\pgfsetlinewidth{1.003750pt}%
\definecolor{currentstroke}{rgb}{1.000000,0.000000,0.000000}%
\pgfsetstrokecolor{currentstroke}%
\pgfsetdash{}{0pt}%
\pgfpathmoveto{\pgfqpoint{0.651535in}{2.919537in}}%
\pgfpathlineto{\pgfqpoint{0.697073in}{2.817706in}}%
\pgfpathlineto{\pgfqpoint{0.742611in}{2.721062in}}%
\pgfpathlineto{\pgfqpoint{0.788148in}{2.629327in}}%
\pgfpathlineto{\pgfqpoint{0.833686in}{2.542264in}}%
\pgfpathlineto{\pgfqpoint{0.879224in}{2.459624in}}%
\pgfpathlineto{\pgfqpoint{0.924761in}{2.381179in}}%
\pgfpathlineto{\pgfqpoint{0.970299in}{2.306725in}}%
\pgfpathlineto{\pgfqpoint{1.015837in}{2.236046in}}%
\pgfpathlineto{\pgfqpoint{1.061375in}{2.168955in}}%
\pgfpathlineto{\pgfqpoint{1.106912in}{2.105269in}}%
\pgfpathmoveto{\pgfqpoint{1.258705in}{1.915450in}}%
\pgfpathlineto{\pgfqpoint{1.304243in}{1.864584in}}%
\pgfpathlineto{\pgfqpoint{1.364960in}{1.800821in}}%
\pgfpathlineto{\pgfqpoint{1.425677in}{1.741251in}}%
\pgfpathlineto{\pgfqpoint{1.486394in}{1.685714in}}%
\pgfpathlineto{\pgfqpoint{1.547111in}{1.633854in}}%
\pgfpathlineto{\pgfqpoint{1.607827in}{1.585475in}}%
\pgfpathlineto{\pgfqpoint{1.668544in}{1.540322in}}%
\pgfpathlineto{\pgfqpoint{1.729261in}{1.498173in}}%
\pgfpathlineto{\pgfqpoint{1.789978in}{1.458853in}}%
\pgfpathlineto{\pgfqpoint{1.850695in}{1.422127in}}%
\pgfpathlineto{\pgfqpoint{1.911412in}{1.387883in}}%
\pgfpathlineto{\pgfqpoint{1.972129in}{1.355877in}}%
\pgfpathlineto{\pgfqpoint{2.048026in}{1.318874in}}%
\pgfpathlineto{\pgfqpoint{2.123922in}{1.284944in}}%
\pgfpathlineto{\pgfqpoint{2.199818in}{1.253803in}}%
\pgfpathlineto{\pgfqpoint{2.275714in}{1.225221in}}%
\pgfpathlineto{\pgfqpoint{2.351610in}{1.198987in}}%
\pgfpathlineto{\pgfqpoint{2.427507in}{1.174908in}}%
\pgfpathlineto{\pgfqpoint{2.518582in}{1.148631in}}%
\pgfpathlineto{\pgfqpoint{2.609658in}{1.124917in}}%
\pgfpathlineto{\pgfqpoint{2.700733in}{1.103515in}}%
\pgfpathlineto{\pgfqpoint{2.791809in}{1.084199in}}%
\pgfpathlineto{\pgfqpoint{2.898063in}{1.064062in}}%
\pgfpathlineto{\pgfqpoint{3.004318in}{1.046178in}}%
\pgfpathlineto{\pgfqpoint{3.125752in}{1.028202in}}%
\pgfpathlineto{\pgfqpoint{3.247186in}{1.012524in}}%
\pgfpathlineto{\pgfqpoint{3.383799in}{0.997275in}}%
\pgfpathlineto{\pgfqpoint{3.520412in}{0.984191in}}%
\pgfpathlineto{\pgfqpoint{3.641846in}{0.974133in}}%
\pgfpathlineto{\pgfqpoint{3.641846in}{0.974133in}}%
\pgfusepath{stroke}%
\end{pgfscope}%
\begin{pgfscope}%
\pgfpathrectangle{\pgfqpoint{0.636356in}{0.905046in}}{\pgfqpoint{3.020670in}{2.030398in}} %
\pgfusepath{clip}%
\pgfsetbuttcap%
\pgfsetroundjoin%
\pgfsetlinewidth{1.003750pt}%
\definecolor{currentstroke}{rgb}{0.000000,0.000000,0.000000}%
\pgfsetstrokecolor{currentstroke}%
\pgfsetdash{}{0pt}%
\pgfpathmoveto{\pgfqpoint{0.651459in}{2.822523in}}%
\pgfpathlineto{\pgfqpoint{0.651459in}{2.869980in}}%
\pgfusepath{stroke}%
\end{pgfscope}%
\begin{pgfscope}%
\pgfpathrectangle{\pgfqpoint{0.636356in}{0.905046in}}{\pgfqpoint{3.020670in}{2.030398in}} %
\pgfusepath{clip}%
\pgfsetbuttcap%
\pgfsetroundjoin%
\pgfsetlinewidth{1.003750pt}%
\definecolor{currentstroke}{rgb}{0.000000,0.000000,0.000000}%
\pgfsetstrokecolor{currentstroke}%
\pgfsetdash{}{0pt}%
\pgfpathmoveto{\pgfqpoint{0.681666in}{2.827712in}}%
\pgfpathlineto{\pgfqpoint{0.681666in}{2.875233in}}%
\pgfusepath{stroke}%
\end{pgfscope}%
\begin{pgfscope}%
\pgfpathrectangle{\pgfqpoint{0.636356in}{0.905046in}}{\pgfqpoint{3.020670in}{2.030398in}} %
\pgfusepath{clip}%
\pgfsetbuttcap%
\pgfsetroundjoin%
\pgfsetlinewidth{1.003750pt}%
\definecolor{currentstroke}{rgb}{0.000000,0.000000,0.000000}%
\pgfsetstrokecolor{currentstroke}%
\pgfsetdash{}{0pt}%
\pgfpathmoveto{\pgfqpoint{0.711873in}{2.721344in}}%
\pgfpathlineto{\pgfqpoint{0.711873in}{2.767539in}}%
\pgfusepath{stroke}%
\end{pgfscope}%
\begin{pgfscope}%
\pgfpathrectangle{\pgfqpoint{0.636356in}{0.905046in}}{\pgfqpoint{3.020670in}{2.030398in}} %
\pgfusepath{clip}%
\pgfsetbuttcap%
\pgfsetroundjoin%
\pgfsetlinewidth{1.003750pt}%
\definecolor{currentstroke}{rgb}{0.000000,0.000000,0.000000}%
\pgfsetstrokecolor{currentstroke}%
\pgfsetdash{}{0pt}%
\pgfpathmoveto{\pgfqpoint{0.742079in}{2.662547in}}%
\pgfpathlineto{\pgfqpoint{0.742079in}{2.707993in}}%
\pgfusepath{stroke}%
\end{pgfscope}%
\begin{pgfscope}%
\pgfpathrectangle{\pgfqpoint{0.636356in}{0.905046in}}{\pgfqpoint{3.020670in}{2.030398in}} %
\pgfusepath{clip}%
\pgfsetbuttcap%
\pgfsetroundjoin%
\pgfsetlinewidth{1.003750pt}%
\definecolor{currentstroke}{rgb}{0.000000,0.000000,0.000000}%
\pgfsetstrokecolor{currentstroke}%
\pgfsetdash{}{0pt}%
\pgfpathmoveto{\pgfqpoint{0.772286in}{2.647848in}}%
\pgfpathlineto{\pgfqpoint{0.772286in}{2.693106in}}%
\pgfusepath{stroke}%
\end{pgfscope}%
\begin{pgfscope}%
\pgfpathrectangle{\pgfqpoint{0.636356in}{0.905046in}}{\pgfqpoint{3.020670in}{2.030398in}} %
\pgfusepath{clip}%
\pgfsetbuttcap%
\pgfsetroundjoin%
\pgfsetlinewidth{1.003750pt}%
\definecolor{currentstroke}{rgb}{0.000000,0.000000,0.000000}%
\pgfsetstrokecolor{currentstroke}%
\pgfsetdash{}{0pt}%
\pgfpathmoveto{\pgfqpoint{0.802493in}{2.582432in}}%
\pgfpathlineto{\pgfqpoint{0.802493in}{2.626837in}}%
\pgfusepath{stroke}%
\end{pgfscope}%
\begin{pgfscope}%
\pgfpathrectangle{\pgfqpoint{0.636356in}{0.905046in}}{\pgfqpoint{3.020670in}{2.030398in}} %
\pgfusepath{clip}%
\pgfsetbuttcap%
\pgfsetroundjoin%
\pgfsetlinewidth{1.003750pt}%
\definecolor{currentstroke}{rgb}{0.000000,0.000000,0.000000}%
\pgfsetstrokecolor{currentstroke}%
\pgfsetdash{}{0pt}%
\pgfpathmoveto{\pgfqpoint{0.832699in}{2.547277in}}%
\pgfpathlineto{\pgfqpoint{0.832699in}{2.591217in}}%
\pgfusepath{stroke}%
\end{pgfscope}%
\begin{pgfscope}%
\pgfpathrectangle{\pgfqpoint{0.636356in}{0.905046in}}{\pgfqpoint{3.020670in}{2.030398in}} %
\pgfusepath{clip}%
\pgfsetbuttcap%
\pgfsetroundjoin%
\pgfsetlinewidth{1.003750pt}%
\definecolor{currentstroke}{rgb}{0.000000,0.000000,0.000000}%
\pgfsetstrokecolor{currentstroke}%
\pgfsetdash{}{0pt}%
\pgfpathmoveto{\pgfqpoint{0.862906in}{2.471790in}}%
\pgfpathlineto{\pgfqpoint{0.862906in}{2.514715in}}%
\pgfusepath{stroke}%
\end{pgfscope}%
\begin{pgfscope}%
\pgfpathrectangle{\pgfqpoint{0.636356in}{0.905046in}}{\pgfqpoint{3.020670in}{2.030398in}} %
\pgfusepath{clip}%
\pgfsetbuttcap%
\pgfsetroundjoin%
\pgfsetlinewidth{1.003750pt}%
\definecolor{currentstroke}{rgb}{0.000000,0.000000,0.000000}%
\pgfsetstrokecolor{currentstroke}%
\pgfsetdash{}{0pt}%
\pgfpathmoveto{\pgfqpoint{0.893113in}{2.409565in}}%
\pgfpathlineto{\pgfqpoint{0.893113in}{2.451635in}}%
\pgfusepath{stroke}%
\end{pgfscope}%
\begin{pgfscope}%
\pgfpathrectangle{\pgfqpoint{0.636356in}{0.905046in}}{\pgfqpoint{3.020670in}{2.030398in}} %
\pgfusepath{clip}%
\pgfsetbuttcap%
\pgfsetroundjoin%
\pgfsetlinewidth{1.003750pt}%
\definecolor{currentstroke}{rgb}{0.000000,0.000000,0.000000}%
\pgfsetstrokecolor{currentstroke}%
\pgfsetdash{}{0pt}%
\pgfpathmoveto{\pgfqpoint{0.923319in}{2.352533in}}%
\pgfpathlineto{\pgfqpoint{0.923319in}{2.393804in}}%
\pgfusepath{stroke}%
\end{pgfscope}%
\begin{pgfscope}%
\pgfpathrectangle{\pgfqpoint{0.636356in}{0.905046in}}{\pgfqpoint{3.020670in}{2.030398in}} %
\pgfusepath{clip}%
\pgfsetbuttcap%
\pgfsetroundjoin%
\pgfsetlinewidth{1.003750pt}%
\definecolor{currentstroke}{rgb}{0.000000,0.000000,0.000000}%
\pgfsetstrokecolor{currentstroke}%
\pgfsetdash{}{0pt}%
\pgfpathmoveto{\pgfqpoint{0.953526in}{2.281975in}}%
\pgfpathlineto{\pgfqpoint{0.953526in}{2.322235in}}%
\pgfusepath{stroke}%
\end{pgfscope}%
\begin{pgfscope}%
\pgfpathrectangle{\pgfqpoint{0.636356in}{0.905046in}}{\pgfqpoint{3.020670in}{2.030398in}} %
\pgfusepath{clip}%
\pgfsetbuttcap%
\pgfsetroundjoin%
\pgfsetlinewidth{1.003750pt}%
\definecolor{currentstroke}{rgb}{0.000000,0.000000,0.000000}%
\pgfsetstrokecolor{currentstroke}%
\pgfsetdash{}{0pt}%
\pgfpathmoveto{\pgfqpoint{0.983733in}{2.282263in}}%
\pgfpathlineto{\pgfqpoint{0.983733in}{2.322527in}}%
\pgfusepath{stroke}%
\end{pgfscope}%
\begin{pgfscope}%
\pgfpathrectangle{\pgfqpoint{0.636356in}{0.905046in}}{\pgfqpoint{3.020670in}{2.030398in}} %
\pgfusepath{clip}%
\pgfsetbuttcap%
\pgfsetroundjoin%
\pgfsetlinewidth{1.003750pt}%
\definecolor{currentstroke}{rgb}{0.000000,0.000000,0.000000}%
\pgfsetstrokecolor{currentstroke}%
\pgfsetdash{}{0pt}%
\pgfpathmoveto{\pgfqpoint{1.013940in}{2.240222in}}%
\pgfpathlineto{\pgfqpoint{1.013940in}{2.279871in}}%
\pgfusepath{stroke}%
\end{pgfscope}%
\begin{pgfscope}%
\pgfpathrectangle{\pgfqpoint{0.636356in}{0.905046in}}{\pgfqpoint{3.020670in}{2.030398in}} %
\pgfusepath{clip}%
\pgfsetbuttcap%
\pgfsetroundjoin%
\pgfsetlinewidth{1.003750pt}%
\definecolor{currentstroke}{rgb}{0.000000,0.000000,0.000000}%
\pgfsetstrokecolor{currentstroke}%
\pgfsetdash{}{0pt}%
\pgfpathmoveto{\pgfqpoint{1.044146in}{2.178897in}}%
\pgfpathlineto{\pgfqpoint{1.044146in}{2.217632in}}%
\pgfusepath{stroke}%
\end{pgfscope}%
\begin{pgfscope}%
\pgfpathrectangle{\pgfqpoint{0.636356in}{0.905046in}}{\pgfqpoint{3.020670in}{2.030398in}} %
\pgfusepath{clip}%
\pgfsetbuttcap%
\pgfsetroundjoin%
\pgfsetlinewidth{1.003750pt}%
\definecolor{currentstroke}{rgb}{0.000000,0.000000,0.000000}%
\pgfsetstrokecolor{currentstroke}%
\pgfsetdash{}{0pt}%
\pgfpathmoveto{\pgfqpoint{1.074353in}{2.169685in}}%
\pgfpathlineto{\pgfqpoint{1.074353in}{2.208280in}}%
\pgfusepath{stroke}%
\end{pgfscope}%
\begin{pgfscope}%
\pgfpathrectangle{\pgfqpoint{0.636356in}{0.905046in}}{\pgfqpoint{3.020670in}{2.030398in}} %
\pgfusepath{clip}%
\pgfsetbuttcap%
\pgfsetroundjoin%
\pgfsetlinewidth{1.003750pt}%
\definecolor{currentstroke}{rgb}{0.000000,0.000000,0.000000}%
\pgfsetstrokecolor{currentstroke}%
\pgfsetdash{}{0pt}%
\pgfpathmoveto{\pgfqpoint{1.104560in}{1.708035in}}%
\pgfpathlineto{\pgfqpoint{1.104560in}{1.738848in}}%
\pgfusepath{stroke}%
\end{pgfscope}%
\begin{pgfscope}%
\pgfpathrectangle{\pgfqpoint{0.636356in}{0.905046in}}{\pgfqpoint{3.020670in}{2.030398in}} %
\pgfusepath{clip}%
\pgfsetbuttcap%
\pgfsetroundjoin%
\pgfsetlinewidth{1.003750pt}%
\definecolor{currentstroke}{rgb}{0.000000,0.000000,0.000000}%
\pgfsetstrokecolor{currentstroke}%
\pgfsetdash{}{0pt}%
\pgfpathmoveto{\pgfqpoint{1.134766in}{0.905046in}}%
\pgfpathlineto{\pgfqpoint{1.134766in}{0.905379in}}%
\pgfusepath{stroke}%
\end{pgfscope}%
\begin{pgfscope}%
\pgfpathrectangle{\pgfqpoint{0.636356in}{0.905046in}}{\pgfqpoint{3.020670in}{2.030398in}} %
\pgfusepath{clip}%
\pgfsetbuttcap%
\pgfsetroundjoin%
\pgfsetlinewidth{1.003750pt}%
\definecolor{currentstroke}{rgb}{0.000000,0.000000,0.000000}%
\pgfsetstrokecolor{currentstroke}%
\pgfsetdash{}{0pt}%
\pgfpathmoveto{\pgfqpoint{1.164973in}{0.905046in}}%
\pgfpathlineto{\pgfqpoint{1.164973in}{0.905379in}}%
\pgfusepath{stroke}%
\end{pgfscope}%
\begin{pgfscope}%
\pgfpathrectangle{\pgfqpoint{0.636356in}{0.905046in}}{\pgfqpoint{3.020670in}{2.030398in}} %
\pgfusepath{clip}%
\pgfsetbuttcap%
\pgfsetroundjoin%
\pgfsetlinewidth{1.003750pt}%
\definecolor{currentstroke}{rgb}{0.000000,0.000000,0.000000}%
\pgfsetstrokecolor{currentstroke}%
\pgfsetdash{}{0pt}%
\pgfpathmoveto{\pgfqpoint{1.195180in}{0.905046in}}%
\pgfpathlineto{\pgfqpoint{1.195180in}{0.905379in}}%
\pgfusepath{stroke}%
\end{pgfscope}%
\begin{pgfscope}%
\pgfpathrectangle{\pgfqpoint{0.636356in}{0.905046in}}{\pgfqpoint{3.020670in}{2.030398in}} %
\pgfusepath{clip}%
\pgfsetbuttcap%
\pgfsetroundjoin%
\pgfsetlinewidth{1.003750pt}%
\definecolor{currentstroke}{rgb}{0.000000,0.000000,0.000000}%
\pgfsetstrokecolor{currentstroke}%
\pgfsetdash{}{0pt}%
\pgfpathmoveto{\pgfqpoint{1.225386in}{0.905046in}}%
\pgfpathlineto{\pgfqpoint{1.225386in}{0.905379in}}%
\pgfusepath{stroke}%
\end{pgfscope}%
\begin{pgfscope}%
\pgfpathrectangle{\pgfqpoint{0.636356in}{0.905046in}}{\pgfqpoint{3.020670in}{2.030398in}} %
\pgfusepath{clip}%
\pgfsetbuttcap%
\pgfsetroundjoin%
\pgfsetlinewidth{1.003750pt}%
\definecolor{currentstroke}{rgb}{0.000000,0.000000,0.000000}%
\pgfsetstrokecolor{currentstroke}%
\pgfsetdash{}{0pt}%
\pgfpathmoveto{\pgfqpoint{1.255593in}{1.500444in}}%
\pgfpathlineto{\pgfqpoint{1.255593in}{1.527017in}}%
\pgfusepath{stroke}%
\end{pgfscope}%
\begin{pgfscope}%
\pgfpathrectangle{\pgfqpoint{0.636356in}{0.905046in}}{\pgfqpoint{3.020670in}{2.030398in}} %
\pgfusepath{clip}%
\pgfsetbuttcap%
\pgfsetroundjoin%
\pgfsetlinewidth{1.003750pt}%
\definecolor{currentstroke}{rgb}{0.000000,0.000000,0.000000}%
\pgfsetstrokecolor{currentstroke}%
\pgfsetdash{}{0pt}%
\pgfpathmoveto{\pgfqpoint{1.285800in}{1.897478in}}%
\pgfpathlineto{\pgfqpoint{1.285800in}{1.931701in}}%
\pgfusepath{stroke}%
\end{pgfscope}%
\begin{pgfscope}%
\pgfpathrectangle{\pgfqpoint{0.636356in}{0.905046in}}{\pgfqpoint{3.020670in}{2.030398in}} %
\pgfusepath{clip}%
\pgfsetbuttcap%
\pgfsetroundjoin%
\pgfsetlinewidth{1.003750pt}%
\definecolor{currentstroke}{rgb}{0.000000,0.000000,0.000000}%
\pgfsetstrokecolor{currentstroke}%
\pgfsetdash{}{0pt}%
\pgfpathmoveto{\pgfqpoint{1.316007in}{1.822139in}}%
\pgfpathlineto{\pgfqpoint{1.316007in}{1.855049in}}%
\pgfusepath{stroke}%
\end{pgfscope}%
\begin{pgfscope}%
\pgfpathrectangle{\pgfqpoint{0.636356in}{0.905046in}}{\pgfqpoint{3.020670in}{2.030398in}} %
\pgfusepath{clip}%
\pgfsetbuttcap%
\pgfsetroundjoin%
\pgfsetlinewidth{1.003750pt}%
\definecolor{currentstroke}{rgb}{0.000000,0.000000,0.000000}%
\pgfsetstrokecolor{currentstroke}%
\pgfsetdash{}{0pt}%
\pgfpathmoveto{\pgfqpoint{1.346213in}{1.820702in}}%
\pgfpathlineto{\pgfqpoint{1.346213in}{1.853586in}}%
\pgfusepath{stroke}%
\end{pgfscope}%
\begin{pgfscope}%
\pgfpathrectangle{\pgfqpoint{0.636356in}{0.905046in}}{\pgfqpoint{3.020670in}{2.030398in}} %
\pgfusepath{clip}%
\pgfsetbuttcap%
\pgfsetroundjoin%
\pgfsetlinewidth{1.003750pt}%
\definecolor{currentstroke}{rgb}{0.000000,0.000000,0.000000}%
\pgfsetstrokecolor{currentstroke}%
\pgfsetdash{}{0pt}%
\pgfpathmoveto{\pgfqpoint{1.376420in}{1.797416in}}%
\pgfpathlineto{\pgfqpoint{1.376420in}{1.829883in}}%
\pgfusepath{stroke}%
\end{pgfscope}%
\begin{pgfscope}%
\pgfpathrectangle{\pgfqpoint{0.636356in}{0.905046in}}{\pgfqpoint{3.020670in}{2.030398in}} %
\pgfusepath{clip}%
\pgfsetbuttcap%
\pgfsetroundjoin%
\pgfsetlinewidth{1.003750pt}%
\definecolor{currentstroke}{rgb}{0.000000,0.000000,0.000000}%
\pgfsetstrokecolor{currentstroke}%
\pgfsetdash{}{0pt}%
\pgfpathmoveto{\pgfqpoint{1.406627in}{1.770108in}}%
\pgfpathlineto{\pgfqpoint{1.406627in}{1.802079in}}%
\pgfusepath{stroke}%
\end{pgfscope}%
\begin{pgfscope}%
\pgfpathrectangle{\pgfqpoint{0.636356in}{0.905046in}}{\pgfqpoint{3.020670in}{2.030398in}} %
\pgfusepath{clip}%
\pgfsetbuttcap%
\pgfsetroundjoin%
\pgfsetlinewidth{1.003750pt}%
\definecolor{currentstroke}{rgb}{0.000000,0.000000,0.000000}%
\pgfsetstrokecolor{currentstroke}%
\pgfsetdash{}{0pt}%
\pgfpathmoveto{\pgfqpoint{1.436833in}{1.729011in}}%
\pgfpathlineto{\pgfqpoint{1.436833in}{1.760220in}}%
\pgfusepath{stroke}%
\end{pgfscope}%
\begin{pgfscope}%
\pgfpathrectangle{\pgfqpoint{0.636356in}{0.905046in}}{\pgfqpoint{3.020670in}{2.030398in}} %
\pgfusepath{clip}%
\pgfsetbuttcap%
\pgfsetroundjoin%
\pgfsetlinewidth{1.003750pt}%
\definecolor{currentstroke}{rgb}{0.000000,0.000000,0.000000}%
\pgfsetstrokecolor{currentstroke}%
\pgfsetdash{}{0pt}%
\pgfpathmoveto{\pgfqpoint{1.467040in}{1.713494in}}%
\pgfpathlineto{\pgfqpoint{1.467040in}{1.744411in}}%
\pgfusepath{stroke}%
\end{pgfscope}%
\begin{pgfscope}%
\pgfpathrectangle{\pgfqpoint{0.636356in}{0.905046in}}{\pgfqpoint{3.020670in}{2.030398in}} %
\pgfusepath{clip}%
\pgfsetbuttcap%
\pgfsetroundjoin%
\pgfsetlinewidth{1.003750pt}%
\definecolor{currentstroke}{rgb}{0.000000,0.000000,0.000000}%
\pgfsetstrokecolor{currentstroke}%
\pgfsetdash{}{0pt}%
\pgfpathmoveto{\pgfqpoint{1.497247in}{1.700277in}}%
\pgfpathlineto{\pgfqpoint{1.497247in}{1.730943in}}%
\pgfusepath{stroke}%
\end{pgfscope}%
\begin{pgfscope}%
\pgfpathrectangle{\pgfqpoint{0.636356in}{0.905046in}}{\pgfqpoint{3.020670in}{2.030398in}} %
\pgfusepath{clip}%
\pgfsetbuttcap%
\pgfsetroundjoin%
\pgfsetlinewidth{1.003750pt}%
\definecolor{currentstroke}{rgb}{0.000000,0.000000,0.000000}%
\pgfsetstrokecolor{currentstroke}%
\pgfsetdash{}{0pt}%
\pgfpathmoveto{\pgfqpoint{1.527453in}{1.671549in}}%
\pgfpathlineto{\pgfqpoint{1.527453in}{1.701660in}}%
\pgfusepath{stroke}%
\end{pgfscope}%
\begin{pgfscope}%
\pgfpathrectangle{\pgfqpoint{0.636356in}{0.905046in}}{\pgfqpoint{3.020670in}{2.030398in}} %
\pgfusepath{clip}%
\pgfsetbuttcap%
\pgfsetroundjoin%
\pgfsetlinewidth{1.003750pt}%
\definecolor{currentstroke}{rgb}{0.000000,0.000000,0.000000}%
\pgfsetstrokecolor{currentstroke}%
\pgfsetdash{}{0pt}%
\pgfpathmoveto{\pgfqpoint{1.557660in}{1.625306in}}%
\pgfpathlineto{\pgfqpoint{1.557660in}{1.654504in}}%
\pgfusepath{stroke}%
\end{pgfscope}%
\begin{pgfscope}%
\pgfpathrectangle{\pgfqpoint{0.636356in}{0.905046in}}{\pgfqpoint{3.020670in}{2.030398in}} %
\pgfusepath{clip}%
\pgfsetbuttcap%
\pgfsetroundjoin%
\pgfsetlinewidth{1.003750pt}%
\definecolor{currentstroke}{rgb}{0.000000,0.000000,0.000000}%
\pgfsetstrokecolor{currentstroke}%
\pgfsetdash{}{0pt}%
\pgfpathmoveto{\pgfqpoint{1.587867in}{1.607790in}}%
\pgfpathlineto{\pgfqpoint{1.587867in}{1.636634in}}%
\pgfusepath{stroke}%
\end{pgfscope}%
\begin{pgfscope}%
\pgfpathrectangle{\pgfqpoint{0.636356in}{0.905046in}}{\pgfqpoint{3.020670in}{2.030398in}} %
\pgfusepath{clip}%
\pgfsetbuttcap%
\pgfsetroundjoin%
\pgfsetlinewidth{1.003750pt}%
\definecolor{currentstroke}{rgb}{0.000000,0.000000,0.000000}%
\pgfsetstrokecolor{currentstroke}%
\pgfsetdash{}{0pt}%
\pgfpathmoveto{\pgfqpoint{1.618073in}{1.565874in}}%
\pgfpathlineto{\pgfqpoint{1.618073in}{1.593853in}}%
\pgfusepath{stroke}%
\end{pgfscope}%
\begin{pgfscope}%
\pgfpathrectangle{\pgfqpoint{0.636356in}{0.905046in}}{\pgfqpoint{3.020670in}{2.030398in}} %
\pgfusepath{clip}%
\pgfsetbuttcap%
\pgfsetroundjoin%
\pgfsetlinewidth{1.003750pt}%
\definecolor{currentstroke}{rgb}{0.000000,0.000000,0.000000}%
\pgfsetstrokecolor{currentstroke}%
\pgfsetdash{}{0pt}%
\pgfpathmoveto{\pgfqpoint{1.648280in}{1.570179in}}%
\pgfpathlineto{\pgfqpoint{1.648280in}{1.598249in}}%
\pgfusepath{stroke}%
\end{pgfscope}%
\begin{pgfscope}%
\pgfpathrectangle{\pgfqpoint{0.636356in}{0.905046in}}{\pgfqpoint{3.020670in}{2.030398in}} %
\pgfusepath{clip}%
\pgfsetbuttcap%
\pgfsetroundjoin%
\pgfsetlinewidth{1.003750pt}%
\definecolor{currentstroke}{rgb}{0.000000,0.000000,0.000000}%
\pgfsetstrokecolor{currentstroke}%
\pgfsetdash{}{0pt}%
\pgfpathmoveto{\pgfqpoint{1.678487in}{1.499870in}}%
\pgfpathlineto{\pgfqpoint{1.678487in}{1.526431in}}%
\pgfusepath{stroke}%
\end{pgfscope}%
\begin{pgfscope}%
\pgfpathrectangle{\pgfqpoint{0.636356in}{0.905046in}}{\pgfqpoint{3.020670in}{2.030398in}} %
\pgfusepath{clip}%
\pgfsetbuttcap%
\pgfsetroundjoin%
\pgfsetlinewidth{1.003750pt}%
\definecolor{currentstroke}{rgb}{0.000000,0.000000,0.000000}%
\pgfsetstrokecolor{currentstroke}%
\pgfsetdash{}{0pt}%
\pgfpathmoveto{\pgfqpoint{1.708694in}{1.482371in}}%
\pgfpathlineto{\pgfqpoint{1.708694in}{1.508542in}}%
\pgfusepath{stroke}%
\end{pgfscope}%
\begin{pgfscope}%
\pgfpathrectangle{\pgfqpoint{0.636356in}{0.905046in}}{\pgfqpoint{3.020670in}{2.030398in}} %
\pgfusepath{clip}%
\pgfsetbuttcap%
\pgfsetroundjoin%
\pgfsetlinewidth{1.003750pt}%
\definecolor{currentstroke}{rgb}{0.000000,0.000000,0.000000}%
\pgfsetstrokecolor{currentstroke}%
\pgfsetdash{}{0pt}%
\pgfpathmoveto{\pgfqpoint{1.738900in}{1.477782in}}%
\pgfpathlineto{\pgfqpoint{1.738900in}{1.503850in}}%
\pgfusepath{stroke}%
\end{pgfscope}%
\begin{pgfscope}%
\pgfpathrectangle{\pgfqpoint{0.636356in}{0.905046in}}{\pgfqpoint{3.020670in}{2.030398in}} %
\pgfusepath{clip}%
\pgfsetbuttcap%
\pgfsetroundjoin%
\pgfsetlinewidth{1.003750pt}%
\definecolor{currentstroke}{rgb}{0.000000,0.000000,0.000000}%
\pgfsetstrokecolor{currentstroke}%
\pgfsetdash{}{0pt}%
\pgfpathmoveto{\pgfqpoint{1.769107in}{1.439068in}}%
\pgfpathlineto{\pgfqpoint{1.769107in}{1.464249in}}%
\pgfusepath{stroke}%
\end{pgfscope}%
\begin{pgfscope}%
\pgfpathrectangle{\pgfqpoint{0.636356in}{0.905046in}}{\pgfqpoint{3.020670in}{2.030398in}} %
\pgfusepath{clip}%
\pgfsetbuttcap%
\pgfsetroundjoin%
\pgfsetlinewidth{1.003750pt}%
\definecolor{currentstroke}{rgb}{0.000000,0.000000,0.000000}%
\pgfsetstrokecolor{currentstroke}%
\pgfsetdash{}{0pt}%
\pgfpathmoveto{\pgfqpoint{1.799314in}{1.420720in}}%
\pgfpathlineto{\pgfqpoint{1.799314in}{1.445470in}}%
\pgfusepath{stroke}%
\end{pgfscope}%
\begin{pgfscope}%
\pgfpathrectangle{\pgfqpoint{0.636356in}{0.905046in}}{\pgfqpoint{3.020670in}{2.030398in}} %
\pgfusepath{clip}%
\pgfsetbuttcap%
\pgfsetroundjoin%
\pgfsetlinewidth{1.003750pt}%
\definecolor{currentstroke}{rgb}{0.000000,0.000000,0.000000}%
\pgfsetstrokecolor{currentstroke}%
\pgfsetdash{}{0pt}%
\pgfpathmoveto{\pgfqpoint{1.829520in}{1.404382in}}%
\pgfpathlineto{\pgfqpoint{1.829520in}{1.428741in}}%
\pgfusepath{stroke}%
\end{pgfscope}%
\begin{pgfscope}%
\pgfpathrectangle{\pgfqpoint{0.636356in}{0.905046in}}{\pgfqpoint{3.020670in}{2.030398in}} %
\pgfusepath{clip}%
\pgfsetbuttcap%
\pgfsetroundjoin%
\pgfsetlinewidth{1.003750pt}%
\definecolor{currentstroke}{rgb}{0.000000,0.000000,0.000000}%
\pgfsetstrokecolor{currentstroke}%
\pgfsetdash{}{0pt}%
\pgfpathmoveto{\pgfqpoint{1.859727in}{1.407534in}}%
\pgfpathlineto{\pgfqpoint{1.859727in}{1.431970in}}%
\pgfusepath{stroke}%
\end{pgfscope}%
\begin{pgfscope}%
\pgfpathrectangle{\pgfqpoint{0.636356in}{0.905046in}}{\pgfqpoint{3.020670in}{2.030398in}} %
\pgfusepath{clip}%
\pgfsetbuttcap%
\pgfsetroundjoin%
\pgfsetlinewidth{1.003750pt}%
\definecolor{currentstroke}{rgb}{0.000000,0.000000,0.000000}%
\pgfsetstrokecolor{currentstroke}%
\pgfsetdash{}{0pt}%
\pgfpathmoveto{\pgfqpoint{1.889934in}{1.374580in}}%
\pgfpathlineto{\pgfqpoint{1.889934in}{1.398211in}}%
\pgfusepath{stroke}%
\end{pgfscope}%
\begin{pgfscope}%
\pgfpathrectangle{\pgfqpoint{0.636356in}{0.905046in}}{\pgfqpoint{3.020670in}{2.030398in}} %
\pgfusepath{clip}%
\pgfsetbuttcap%
\pgfsetroundjoin%
\pgfsetlinewidth{1.003750pt}%
\definecolor{currentstroke}{rgb}{0.000000,0.000000,0.000000}%
\pgfsetstrokecolor{currentstroke}%
\pgfsetdash{}{0pt}%
\pgfpathmoveto{\pgfqpoint{1.920140in}{1.362835in}}%
\pgfpathlineto{\pgfqpoint{1.920140in}{1.386171in}}%
\pgfusepath{stroke}%
\end{pgfscope}%
\begin{pgfscope}%
\pgfpathrectangle{\pgfqpoint{0.636356in}{0.905046in}}{\pgfqpoint{3.020670in}{2.030398in}} %
\pgfusepath{clip}%
\pgfsetbuttcap%
\pgfsetroundjoin%
\pgfsetlinewidth{1.003750pt}%
\definecolor{currentstroke}{rgb}{0.000000,0.000000,0.000000}%
\pgfsetstrokecolor{currentstroke}%
\pgfsetdash{}{0pt}%
\pgfpathmoveto{\pgfqpoint{1.950347in}{1.353383in}}%
\pgfpathlineto{\pgfqpoint{1.950347in}{1.376480in}}%
\pgfusepath{stroke}%
\end{pgfscope}%
\begin{pgfscope}%
\pgfpathrectangle{\pgfqpoint{0.636356in}{0.905046in}}{\pgfqpoint{3.020670in}{2.030398in}} %
\pgfusepath{clip}%
\pgfsetbuttcap%
\pgfsetroundjoin%
\pgfsetlinewidth{1.003750pt}%
\definecolor{currentstroke}{rgb}{0.000000,0.000000,0.000000}%
\pgfsetstrokecolor{currentstroke}%
\pgfsetdash{}{0pt}%
\pgfpathmoveto{\pgfqpoint{1.980554in}{1.326751in}}%
\pgfpathlineto{\pgfqpoint{1.980554in}{1.349161in}}%
\pgfusepath{stroke}%
\end{pgfscope}%
\begin{pgfscope}%
\pgfpathrectangle{\pgfqpoint{0.636356in}{0.905046in}}{\pgfqpoint{3.020670in}{2.030398in}} %
\pgfusepath{clip}%
\pgfsetbuttcap%
\pgfsetroundjoin%
\pgfsetlinewidth{1.003750pt}%
\definecolor{currentstroke}{rgb}{0.000000,0.000000,0.000000}%
\pgfsetstrokecolor{currentstroke}%
\pgfsetdash{}{0pt}%
\pgfpathmoveto{\pgfqpoint{2.010761in}{1.300131in}}%
\pgfpathlineto{\pgfqpoint{2.010761in}{1.321831in}}%
\pgfusepath{stroke}%
\end{pgfscope}%
\begin{pgfscope}%
\pgfpathrectangle{\pgfqpoint{0.636356in}{0.905046in}}{\pgfqpoint{3.020670in}{2.030398in}} %
\pgfusepath{clip}%
\pgfsetbuttcap%
\pgfsetroundjoin%
\pgfsetlinewidth{1.003750pt}%
\definecolor{currentstroke}{rgb}{0.000000,0.000000,0.000000}%
\pgfsetstrokecolor{currentstroke}%
\pgfsetdash{}{0pt}%
\pgfpathmoveto{\pgfqpoint{2.040967in}{1.312152in}}%
\pgfpathlineto{\pgfqpoint{2.040967in}{1.334175in}}%
\pgfusepath{stroke}%
\end{pgfscope}%
\begin{pgfscope}%
\pgfpathrectangle{\pgfqpoint{0.636356in}{0.905046in}}{\pgfqpoint{3.020670in}{2.030398in}} %
\pgfusepath{clip}%
\pgfsetbuttcap%
\pgfsetroundjoin%
\pgfsetlinewidth{1.003750pt}%
\definecolor{currentstroke}{rgb}{0.000000,0.000000,0.000000}%
\pgfsetstrokecolor{currentstroke}%
\pgfsetdash{}{0pt}%
\pgfpathmoveto{\pgfqpoint{2.071174in}{1.297555in}}%
\pgfpathlineto{\pgfqpoint{2.071174in}{1.319185in}}%
\pgfusepath{stroke}%
\end{pgfscope}%
\begin{pgfscope}%
\pgfpathrectangle{\pgfqpoint{0.636356in}{0.905046in}}{\pgfqpoint{3.020670in}{2.030398in}} %
\pgfusepath{clip}%
\pgfsetbuttcap%
\pgfsetroundjoin%
\pgfsetlinewidth{1.003750pt}%
\definecolor{currentstroke}{rgb}{0.000000,0.000000,0.000000}%
\pgfsetstrokecolor{currentstroke}%
\pgfsetdash{}{0pt}%
\pgfpathmoveto{\pgfqpoint{2.101381in}{1.271234in}}%
\pgfpathlineto{\pgfqpoint{2.101381in}{1.292136in}}%
\pgfusepath{stroke}%
\end{pgfscope}%
\begin{pgfscope}%
\pgfpathrectangle{\pgfqpoint{0.636356in}{0.905046in}}{\pgfqpoint{3.020670in}{2.030398in}} %
\pgfusepath{clip}%
\pgfsetbuttcap%
\pgfsetroundjoin%
\pgfsetlinewidth{1.003750pt}%
\definecolor{currentstroke}{rgb}{0.000000,0.000000,0.000000}%
\pgfsetstrokecolor{currentstroke}%
\pgfsetdash{}{0pt}%
\pgfpathmoveto{\pgfqpoint{2.131587in}{1.251787in}}%
\pgfpathlineto{\pgfqpoint{2.131587in}{1.272135in}}%
\pgfusepath{stroke}%
\end{pgfscope}%
\begin{pgfscope}%
\pgfpathrectangle{\pgfqpoint{0.636356in}{0.905046in}}{\pgfqpoint{3.020670in}{2.030398in}} %
\pgfusepath{clip}%
\pgfsetbuttcap%
\pgfsetroundjoin%
\pgfsetlinewidth{1.003750pt}%
\definecolor{currentstroke}{rgb}{0.000000,0.000000,0.000000}%
\pgfsetstrokecolor{currentstroke}%
\pgfsetdash{}{0pt}%
\pgfpathmoveto{\pgfqpoint{2.161794in}{1.254075in}}%
\pgfpathlineto{\pgfqpoint{2.161794in}{1.274488in}}%
\pgfusepath{stroke}%
\end{pgfscope}%
\begin{pgfscope}%
\pgfpathrectangle{\pgfqpoint{0.636356in}{0.905046in}}{\pgfqpoint{3.020670in}{2.030398in}} %
\pgfusepath{clip}%
\pgfsetbuttcap%
\pgfsetroundjoin%
\pgfsetlinewidth{1.003750pt}%
\definecolor{currentstroke}{rgb}{0.000000,0.000000,0.000000}%
\pgfsetstrokecolor{currentstroke}%
\pgfsetdash{}{0pt}%
\pgfpathmoveto{\pgfqpoint{2.192001in}{1.220061in}}%
\pgfpathlineto{\pgfqpoint{2.192001in}{1.239469in}}%
\pgfusepath{stroke}%
\end{pgfscope}%
\begin{pgfscope}%
\pgfpathrectangle{\pgfqpoint{0.636356in}{0.905046in}}{\pgfqpoint{3.020670in}{2.030398in}} %
\pgfusepath{clip}%
\pgfsetbuttcap%
\pgfsetroundjoin%
\pgfsetlinewidth{1.003750pt}%
\definecolor{currentstroke}{rgb}{0.000000,0.000000,0.000000}%
\pgfsetstrokecolor{currentstroke}%
\pgfsetdash{}{0pt}%
\pgfpathmoveto{\pgfqpoint{2.222207in}{1.224061in}}%
\pgfpathlineto{\pgfqpoint{2.222207in}{1.243590in}}%
\pgfusepath{stroke}%
\end{pgfscope}%
\begin{pgfscope}%
\pgfpathrectangle{\pgfqpoint{0.636356in}{0.905046in}}{\pgfqpoint{3.020670in}{2.030398in}} %
\pgfusepath{clip}%
\pgfsetbuttcap%
\pgfsetroundjoin%
\pgfsetlinewidth{1.003750pt}%
\definecolor{currentstroke}{rgb}{0.000000,0.000000,0.000000}%
\pgfsetstrokecolor{currentstroke}%
\pgfsetdash{}{0pt}%
\pgfpathmoveto{\pgfqpoint{2.252414in}{1.208919in}}%
\pgfpathlineto{\pgfqpoint{2.252414in}{1.227986in}}%
\pgfusepath{stroke}%
\end{pgfscope}%
\begin{pgfscope}%
\pgfpathrectangle{\pgfqpoint{0.636356in}{0.905046in}}{\pgfqpoint{3.020670in}{2.030398in}} %
\pgfusepath{clip}%
\pgfsetbuttcap%
\pgfsetroundjoin%
\pgfsetlinewidth{1.003750pt}%
\definecolor{currentstroke}{rgb}{0.000000,0.000000,0.000000}%
\pgfsetstrokecolor{currentstroke}%
\pgfsetdash{}{0pt}%
\pgfpathmoveto{\pgfqpoint{2.282621in}{1.204921in}}%
\pgfpathlineto{\pgfqpoint{2.282621in}{1.223863in}}%
\pgfusepath{stroke}%
\end{pgfscope}%
\begin{pgfscope}%
\pgfpathrectangle{\pgfqpoint{0.636356in}{0.905046in}}{\pgfqpoint{3.020670in}{2.030398in}} %
\pgfusepath{clip}%
\pgfsetbuttcap%
\pgfsetroundjoin%
\pgfsetlinewidth{1.003750pt}%
\definecolor{currentstroke}{rgb}{0.000000,0.000000,0.000000}%
\pgfsetstrokecolor{currentstroke}%
\pgfsetdash{}{0pt}%
\pgfpathmoveto{\pgfqpoint{2.312828in}{1.172089in}}%
\pgfpathlineto{\pgfqpoint{2.312828in}{1.189981in}}%
\pgfusepath{stroke}%
\end{pgfscope}%
\begin{pgfscope}%
\pgfpathrectangle{\pgfqpoint{0.636356in}{0.905046in}}{\pgfqpoint{3.020670in}{2.030398in}} %
\pgfusepath{clip}%
\pgfsetbuttcap%
\pgfsetroundjoin%
\pgfsetlinewidth{1.003750pt}%
\definecolor{currentstroke}{rgb}{0.000000,0.000000,0.000000}%
\pgfsetstrokecolor{currentstroke}%
\pgfsetdash{}{0pt}%
\pgfpathmoveto{\pgfqpoint{2.343034in}{1.196353in}}%
\pgfpathlineto{\pgfqpoint{2.343034in}{1.215027in}}%
\pgfusepath{stroke}%
\end{pgfscope}%
\begin{pgfscope}%
\pgfpathrectangle{\pgfqpoint{0.636356in}{0.905046in}}{\pgfqpoint{3.020670in}{2.030398in}} %
\pgfusepath{clip}%
\pgfsetbuttcap%
\pgfsetroundjoin%
\pgfsetlinewidth{1.003750pt}%
\definecolor{currentstroke}{rgb}{0.000000,0.000000,0.000000}%
\pgfsetstrokecolor{currentstroke}%
\pgfsetdash{}{0pt}%
\pgfpathmoveto{\pgfqpoint{2.373241in}{1.179795in}}%
\pgfpathlineto{\pgfqpoint{2.373241in}{1.197939in}}%
\pgfusepath{stroke}%
\end{pgfscope}%
\begin{pgfscope}%
\pgfpathrectangle{\pgfqpoint{0.636356in}{0.905046in}}{\pgfqpoint{3.020670in}{2.030398in}} %
\pgfusepath{clip}%
\pgfsetbuttcap%
\pgfsetroundjoin%
\pgfsetlinewidth{1.003750pt}%
\definecolor{currentstroke}{rgb}{0.000000,0.000000,0.000000}%
\pgfsetstrokecolor{currentstroke}%
\pgfsetdash{}{0pt}%
\pgfpathmoveto{\pgfqpoint{2.403448in}{1.180366in}}%
\pgfpathlineto{\pgfqpoint{2.403448in}{1.198528in}}%
\pgfusepath{stroke}%
\end{pgfscope}%
\begin{pgfscope}%
\pgfpathrectangle{\pgfqpoint{0.636356in}{0.905046in}}{\pgfqpoint{3.020670in}{2.030398in}} %
\pgfusepath{clip}%
\pgfsetbuttcap%
\pgfsetroundjoin%
\pgfsetlinewidth{1.003750pt}%
\definecolor{currentstroke}{rgb}{0.000000,0.000000,0.000000}%
\pgfsetstrokecolor{currentstroke}%
\pgfsetdash{}{0pt}%
\pgfpathmoveto{\pgfqpoint{2.433654in}{1.167524in}}%
\pgfpathlineto{\pgfqpoint{2.433654in}{1.185265in}}%
\pgfusepath{stroke}%
\end{pgfscope}%
\begin{pgfscope}%
\pgfpathrectangle{\pgfqpoint{0.636356in}{0.905046in}}{\pgfqpoint{3.020670in}{2.030398in}} %
\pgfusepath{clip}%
\pgfsetbuttcap%
\pgfsetroundjoin%
\pgfsetlinewidth{1.003750pt}%
\definecolor{currentstroke}{rgb}{0.000000,0.000000,0.000000}%
\pgfsetstrokecolor{currentstroke}%
\pgfsetdash{}{0pt}%
\pgfpathmoveto{\pgfqpoint{2.463861in}{1.141286in}}%
\pgfpathlineto{\pgfqpoint{2.463861in}{1.158132in}}%
\pgfusepath{stroke}%
\end{pgfscope}%
\begin{pgfscope}%
\pgfpathrectangle{\pgfqpoint{0.636356in}{0.905046in}}{\pgfqpoint{3.020670in}{2.030398in}} %
\pgfusepath{clip}%
\pgfsetbuttcap%
\pgfsetroundjoin%
\pgfsetlinewidth{1.003750pt}%
\definecolor{currentstroke}{rgb}{0.000000,0.000000,0.000000}%
\pgfsetstrokecolor{currentstroke}%
\pgfsetdash{}{0pt}%
\pgfpathmoveto{\pgfqpoint{2.494068in}{1.140431in}}%
\pgfpathlineto{\pgfqpoint{2.494068in}{1.157247in}}%
\pgfusepath{stroke}%
\end{pgfscope}%
\begin{pgfscope}%
\pgfpathrectangle{\pgfqpoint{0.636356in}{0.905046in}}{\pgfqpoint{3.020670in}{2.030398in}} %
\pgfusepath{clip}%
\pgfsetbuttcap%
\pgfsetroundjoin%
\pgfsetlinewidth{1.003750pt}%
\definecolor{currentstroke}{rgb}{0.000000,0.000000,0.000000}%
\pgfsetstrokecolor{currentstroke}%
\pgfsetdash{}{0pt}%
\pgfpathmoveto{\pgfqpoint{2.524274in}{1.127037in}}%
\pgfpathlineto{\pgfqpoint{2.524274in}{1.143376in}}%
\pgfusepath{stroke}%
\end{pgfscope}%
\begin{pgfscope}%
\pgfpathrectangle{\pgfqpoint{0.636356in}{0.905046in}}{\pgfqpoint{3.020670in}{2.030398in}} %
\pgfusepath{clip}%
\pgfsetbuttcap%
\pgfsetroundjoin%
\pgfsetlinewidth{1.003750pt}%
\definecolor{currentstroke}{rgb}{0.000000,0.000000,0.000000}%
\pgfsetstrokecolor{currentstroke}%
\pgfsetdash{}{0pt}%
\pgfpathmoveto{\pgfqpoint{2.554481in}{1.126752in}}%
\pgfpathlineto{\pgfqpoint{2.554481in}{1.143081in}}%
\pgfusepath{stroke}%
\end{pgfscope}%
\begin{pgfscope}%
\pgfpathrectangle{\pgfqpoint{0.636356in}{0.905046in}}{\pgfqpoint{3.020670in}{2.030398in}} %
\pgfusepath{clip}%
\pgfsetbuttcap%
\pgfsetroundjoin%
\pgfsetlinewidth{1.003750pt}%
\definecolor{currentstroke}{rgb}{0.000000,0.000000,0.000000}%
\pgfsetstrokecolor{currentstroke}%
\pgfsetdash{}{0pt}%
\pgfpathmoveto{\pgfqpoint{2.584688in}{1.105394in}}%
\pgfpathlineto{\pgfqpoint{2.584688in}{1.120930in}}%
\pgfusepath{stroke}%
\end{pgfscope}%
\begin{pgfscope}%
\pgfpathrectangle{\pgfqpoint{0.636356in}{0.905046in}}{\pgfqpoint{3.020670in}{2.030398in}} %
\pgfusepath{clip}%
\pgfsetbuttcap%
\pgfsetroundjoin%
\pgfsetlinewidth{1.003750pt}%
\definecolor{currentstroke}{rgb}{0.000000,0.000000,0.000000}%
\pgfsetstrokecolor{currentstroke}%
\pgfsetdash{}{0pt}%
\pgfpathmoveto{\pgfqpoint{2.614894in}{1.114220in}}%
\pgfpathlineto{\pgfqpoint{2.614894in}{1.130088in}}%
\pgfusepath{stroke}%
\end{pgfscope}%
\begin{pgfscope}%
\pgfpathrectangle{\pgfqpoint{0.636356in}{0.905046in}}{\pgfqpoint{3.020670in}{2.030398in}} %
\pgfusepath{clip}%
\pgfsetbuttcap%
\pgfsetroundjoin%
\pgfsetlinewidth{1.003750pt}%
\definecolor{currentstroke}{rgb}{0.000000,0.000000,0.000000}%
\pgfsetstrokecolor{currentstroke}%
\pgfsetdash{}{0pt}%
\pgfpathmoveto{\pgfqpoint{2.645101in}{1.114220in}}%
\pgfpathlineto{\pgfqpoint{2.645101in}{1.130088in}}%
\pgfusepath{stroke}%
\end{pgfscope}%
\begin{pgfscope}%
\pgfpathrectangle{\pgfqpoint{0.636356in}{0.905046in}}{\pgfqpoint{3.020670in}{2.030398in}} %
\pgfusepath{clip}%
\pgfsetbuttcap%
\pgfsetroundjoin%
\pgfsetlinewidth{1.003750pt}%
\definecolor{currentstroke}{rgb}{0.000000,0.000000,0.000000}%
\pgfsetstrokecolor{currentstroke}%
\pgfsetdash{}{0pt}%
\pgfpathmoveto{\pgfqpoint{2.675308in}{1.099986in}}%
\pgfpathlineto{\pgfqpoint{2.675308in}{1.115316in}}%
\pgfusepath{stroke}%
\end{pgfscope}%
\begin{pgfscope}%
\pgfpathrectangle{\pgfqpoint{0.636356in}{0.905046in}}{\pgfqpoint{3.020670in}{2.030398in}} %
\pgfusepath{clip}%
\pgfsetbuttcap%
\pgfsetroundjoin%
\pgfsetlinewidth{1.003750pt}%
\definecolor{currentstroke}{rgb}{0.000000,0.000000,0.000000}%
\pgfsetstrokecolor{currentstroke}%
\pgfsetdash{}{0pt}%
\pgfpathmoveto{\pgfqpoint{2.705515in}{1.099702in}}%
\pgfpathlineto{\pgfqpoint{2.705515in}{1.115020in}}%
\pgfusepath{stroke}%
\end{pgfscope}%
\begin{pgfscope}%
\pgfpathrectangle{\pgfqpoint{0.636356in}{0.905046in}}{\pgfqpoint{3.020670in}{2.030398in}} %
\pgfusepath{clip}%
\pgfsetbuttcap%
\pgfsetroundjoin%
\pgfsetlinewidth{1.003750pt}%
\definecolor{currentstroke}{rgb}{0.000000,0.000000,0.000000}%
\pgfsetstrokecolor{currentstroke}%
\pgfsetdash{}{0pt}%
\pgfpathmoveto{\pgfqpoint{2.735721in}{1.089460in}}%
\pgfpathlineto{\pgfqpoint{2.735721in}{1.104378in}}%
\pgfusepath{stroke}%
\end{pgfscope}%
\begin{pgfscope}%
\pgfpathrectangle{\pgfqpoint{0.636356in}{0.905046in}}{\pgfqpoint{3.020670in}{2.030398in}} %
\pgfusepath{clip}%
\pgfsetbuttcap%
\pgfsetroundjoin%
\pgfsetlinewidth{1.003750pt}%
\definecolor{currentstroke}{rgb}{0.000000,0.000000,0.000000}%
\pgfsetstrokecolor{currentstroke}%
\pgfsetdash{}{0pt}%
\pgfpathmoveto{\pgfqpoint{2.765928in}{1.081782in}}%
\pgfpathlineto{\pgfqpoint{2.765928in}{1.096392in}}%
\pgfusepath{stroke}%
\end{pgfscope}%
\begin{pgfscope}%
\pgfpathrectangle{\pgfqpoint{0.636356in}{0.905046in}}{\pgfqpoint{3.020670in}{2.030398in}} %
\pgfusepath{clip}%
\pgfsetbuttcap%
\pgfsetroundjoin%
\pgfsetlinewidth{1.003750pt}%
\definecolor{currentstroke}{rgb}{0.000000,0.000000,0.000000}%
\pgfsetstrokecolor{currentstroke}%
\pgfsetdash{}{0pt}%
\pgfpathmoveto{\pgfqpoint{2.796135in}{1.058486in}}%
\pgfpathlineto{\pgfqpoint{2.796135in}{1.072119in}}%
\pgfusepath{stroke}%
\end{pgfscope}%
\begin{pgfscope}%
\pgfpathrectangle{\pgfqpoint{0.636356in}{0.905046in}}{\pgfqpoint{3.020670in}{2.030398in}} %
\pgfusepath{clip}%
\pgfsetbuttcap%
\pgfsetroundjoin%
\pgfsetlinewidth{1.003750pt}%
\definecolor{currentstroke}{rgb}{0.000000,0.000000,0.000000}%
\pgfsetstrokecolor{currentstroke}%
\pgfsetdash{}{0pt}%
\pgfpathmoveto{\pgfqpoint{2.826341in}{1.070698in}}%
\pgfpathlineto{\pgfqpoint{2.826341in}{1.084852in}}%
\pgfusepath{stroke}%
\end{pgfscope}%
\begin{pgfscope}%
\pgfpathrectangle{\pgfqpoint{0.636356in}{0.905046in}}{\pgfqpoint{3.020670in}{2.030398in}} %
\pgfusepath{clip}%
\pgfsetbuttcap%
\pgfsetroundjoin%
\pgfsetlinewidth{1.003750pt}%
\definecolor{currentstroke}{rgb}{0.000000,0.000000,0.000000}%
\pgfsetstrokecolor{currentstroke}%
\pgfsetdash{}{0pt}%
\pgfpathmoveto{\pgfqpoint{2.856548in}{1.049972in}}%
\pgfpathlineto{\pgfqpoint{2.856548in}{1.063230in}}%
\pgfusepath{stroke}%
\end{pgfscope}%
\begin{pgfscope}%
\pgfpathrectangle{\pgfqpoint{0.636356in}{0.905046in}}{\pgfqpoint{3.020670in}{2.030398in}} %
\pgfusepath{clip}%
\pgfsetbuttcap%
\pgfsetroundjoin%
\pgfsetlinewidth{1.003750pt}%
\definecolor{currentstroke}{rgb}{0.000000,0.000000,0.000000}%
\pgfsetstrokecolor{currentstroke}%
\pgfsetdash{}{0pt}%
\pgfpathmoveto{\pgfqpoint{2.886755in}{1.055080in}}%
\pgfpathlineto{\pgfqpoint{2.886755in}{1.068564in}}%
\pgfusepath{stroke}%
\end{pgfscope}%
\begin{pgfscope}%
\pgfpathrectangle{\pgfqpoint{0.636356in}{0.905046in}}{\pgfqpoint{3.020670in}{2.030398in}} %
\pgfusepath{clip}%
\pgfsetbuttcap%
\pgfsetroundjoin%
\pgfsetlinewidth{1.003750pt}%
\definecolor{currentstroke}{rgb}{0.000000,0.000000,0.000000}%
\pgfsetstrokecolor{currentstroke}%
\pgfsetdash{}{0pt}%
\pgfpathmoveto{\pgfqpoint{2.916961in}{1.052526in}}%
\pgfpathlineto{\pgfqpoint{2.916961in}{1.065897in}}%
\pgfusepath{stroke}%
\end{pgfscope}%
\begin{pgfscope}%
\pgfpathrectangle{\pgfqpoint{0.636356in}{0.905046in}}{\pgfqpoint{3.020670in}{2.030398in}} %
\pgfusepath{clip}%
\pgfsetbuttcap%
\pgfsetroundjoin%
\pgfsetlinewidth{1.003750pt}%
\definecolor{currentstroke}{rgb}{0.000000,0.000000,0.000000}%
\pgfsetstrokecolor{currentstroke}%
\pgfsetdash{}{0pt}%
\pgfpathmoveto{\pgfqpoint{2.947168in}{1.040613in}}%
\pgfpathlineto{\pgfqpoint{2.947168in}{1.053445in}}%
\pgfusepath{stroke}%
\end{pgfscope}%
\begin{pgfscope}%
\pgfpathrectangle{\pgfqpoint{0.636356in}{0.905046in}}{\pgfqpoint{3.020670in}{2.030398in}} %
\pgfusepath{clip}%
\pgfsetbuttcap%
\pgfsetroundjoin%
\pgfsetlinewidth{1.003750pt}%
\definecolor{currentstroke}{rgb}{0.000000,0.000000,0.000000}%
\pgfsetstrokecolor{currentstroke}%
\pgfsetdash{}{0pt}%
\pgfpathmoveto{\pgfqpoint{2.977375in}{1.049689in}}%
\pgfpathlineto{\pgfqpoint{2.977375in}{1.062933in}}%
\pgfusepath{stroke}%
\end{pgfscope}%
\begin{pgfscope}%
\pgfpathrectangle{\pgfqpoint{0.636356in}{0.905046in}}{\pgfqpoint{3.020670in}{2.030398in}} %
\pgfusepath{clip}%
\pgfsetbuttcap%
\pgfsetroundjoin%
\pgfsetlinewidth{1.003750pt}%
\definecolor{currentstroke}{rgb}{0.000000,0.000000,0.000000}%
\pgfsetstrokecolor{currentstroke}%
\pgfsetdash{}{0pt}%
\pgfpathmoveto{\pgfqpoint{3.007582in}{1.044299in}}%
\pgfpathlineto{\pgfqpoint{3.007582in}{1.057300in}}%
\pgfusepath{stroke}%
\end{pgfscope}%
\begin{pgfscope}%
\pgfpathrectangle{\pgfqpoint{0.636356in}{0.905046in}}{\pgfqpoint{3.020670in}{2.030398in}} %
\pgfusepath{clip}%
\pgfsetbuttcap%
\pgfsetroundjoin%
\pgfsetlinewidth{1.003750pt}%
\definecolor{currentstroke}{rgb}{0.000000,0.000000,0.000000}%
\pgfsetstrokecolor{currentstroke}%
\pgfsetdash{}{0pt}%
\pgfpathmoveto{\pgfqpoint{3.037788in}{1.042882in}}%
\pgfpathlineto{\pgfqpoint{3.037788in}{1.055818in}}%
\pgfusepath{stroke}%
\end{pgfscope}%
\begin{pgfscope}%
\pgfpathrectangle{\pgfqpoint{0.636356in}{0.905046in}}{\pgfqpoint{3.020670in}{2.030398in}} %
\pgfusepath{clip}%
\pgfsetbuttcap%
\pgfsetroundjoin%
\pgfsetlinewidth{1.003750pt}%
\definecolor{currentstroke}{rgb}{0.000000,0.000000,0.000000}%
\pgfsetstrokecolor{currentstroke}%
\pgfsetdash{}{0pt}%
\pgfpathmoveto{\pgfqpoint{3.067995in}{1.038062in}}%
\pgfpathlineto{\pgfqpoint{3.067995in}{1.050775in}}%
\pgfusepath{stroke}%
\end{pgfscope}%
\begin{pgfscope}%
\pgfpathrectangle{\pgfqpoint{0.636356in}{0.905046in}}{\pgfqpoint{3.020670in}{2.030398in}} %
\pgfusepath{clip}%
\pgfsetbuttcap%
\pgfsetroundjoin%
\pgfsetlinewidth{1.003750pt}%
\definecolor{currentstroke}{rgb}{0.000000,0.000000,0.000000}%
\pgfsetstrokecolor{currentstroke}%
\pgfsetdash{}{0pt}%
\pgfpathmoveto{\pgfqpoint{3.098202in}{1.023616in}}%
\pgfpathlineto{\pgfqpoint{3.098202in}{1.035635in}}%
\pgfusepath{stroke}%
\end{pgfscope}%
\begin{pgfscope}%
\pgfpathrectangle{\pgfqpoint{0.636356in}{0.905046in}}{\pgfqpoint{3.020670in}{2.030398in}} %
\pgfusepath{clip}%
\pgfsetbuttcap%
\pgfsetroundjoin%
\pgfsetlinewidth{1.003750pt}%
\definecolor{currentstroke}{rgb}{0.000000,0.000000,0.000000}%
\pgfsetstrokecolor{currentstroke}%
\pgfsetdash{}{0pt}%
\pgfpathmoveto{\pgfqpoint{3.128408in}{1.031828in}}%
\pgfpathlineto{\pgfqpoint{3.128408in}{1.044247in}}%
\pgfusepath{stroke}%
\end{pgfscope}%
\begin{pgfscope}%
\pgfpathrectangle{\pgfqpoint{0.636356in}{0.905046in}}{\pgfqpoint{3.020670in}{2.030398in}} %
\pgfusepath{clip}%
\pgfsetbuttcap%
\pgfsetroundjoin%
\pgfsetlinewidth{1.003750pt}%
\definecolor{currentstroke}{rgb}{0.000000,0.000000,0.000000}%
\pgfsetstrokecolor{currentstroke}%
\pgfsetdash{}{0pt}%
\pgfpathmoveto{\pgfqpoint{3.158615in}{1.021352in}}%
\pgfpathlineto{\pgfqpoint{3.158615in}{1.033258in}}%
\pgfusepath{stroke}%
\end{pgfscope}%
\begin{pgfscope}%
\pgfpathrectangle{\pgfqpoint{0.636356in}{0.905046in}}{\pgfqpoint{3.020670in}{2.030398in}} %
\pgfusepath{clip}%
\pgfsetbuttcap%
\pgfsetroundjoin%
\pgfsetlinewidth{1.003750pt}%
\definecolor{currentstroke}{rgb}{0.000000,0.000000,0.000000}%
\pgfsetstrokecolor{currentstroke}%
\pgfsetdash{}{0pt}%
\pgfpathmoveto{\pgfqpoint{3.188822in}{1.012583in}}%
\pgfpathlineto{\pgfqpoint{3.188822in}{1.024043in}}%
\pgfusepath{stroke}%
\end{pgfscope}%
\begin{pgfscope}%
\pgfpathrectangle{\pgfqpoint{0.636356in}{0.905046in}}{\pgfqpoint{3.020670in}{2.030398in}} %
\pgfusepath{clip}%
\pgfsetbuttcap%
\pgfsetroundjoin%
\pgfsetlinewidth{1.003750pt}%
\definecolor{currentstroke}{rgb}{0.000000,0.000000,0.000000}%
\pgfsetstrokecolor{currentstroke}%
\pgfsetdash{}{0pt}%
\pgfpathmoveto{\pgfqpoint{3.219028in}{1.018805in}}%
\pgfpathlineto{\pgfqpoint{3.219028in}{1.030584in}}%
\pgfusepath{stroke}%
\end{pgfscope}%
\begin{pgfscope}%
\pgfpathrectangle{\pgfqpoint{0.636356in}{0.905046in}}{\pgfqpoint{3.020670in}{2.030398in}} %
\pgfusepath{clip}%
\pgfsetbuttcap%
\pgfsetroundjoin%
\pgfsetlinewidth{1.003750pt}%
\definecolor{currentstroke}{rgb}{0.000000,0.000000,0.000000}%
\pgfsetstrokecolor{currentstroke}%
\pgfsetdash{}{0pt}%
\pgfpathmoveto{\pgfqpoint{3.249235in}{1.013432in}}%
\pgfpathlineto{\pgfqpoint{3.249235in}{1.024936in}}%
\pgfusepath{stroke}%
\end{pgfscope}%
\begin{pgfscope}%
\pgfpathrectangle{\pgfqpoint{0.636356in}{0.905046in}}{\pgfqpoint{3.020670in}{2.030398in}} %
\pgfusepath{clip}%
\pgfsetbuttcap%
\pgfsetroundjoin%
\pgfsetlinewidth{1.003750pt}%
\definecolor{currentstroke}{rgb}{0.000000,0.000000,0.000000}%
\pgfsetstrokecolor{currentstroke}%
\pgfsetdash{}{0pt}%
\pgfpathmoveto{\pgfqpoint{3.279442in}{1.008344in}}%
\pgfpathlineto{\pgfqpoint{3.279442in}{1.019581in}}%
\pgfusepath{stroke}%
\end{pgfscope}%
\begin{pgfscope}%
\pgfpathrectangle{\pgfqpoint{0.636356in}{0.905046in}}{\pgfqpoint{3.020670in}{2.030398in}} %
\pgfusepath{clip}%
\pgfsetbuttcap%
\pgfsetroundjoin%
\pgfsetlinewidth{1.003750pt}%
\definecolor{currentstroke}{rgb}{0.000000,0.000000,0.000000}%
\pgfsetstrokecolor{currentstroke}%
\pgfsetdash{}{0pt}%
\pgfpathmoveto{\pgfqpoint{3.309649in}{1.001283in}}%
\pgfpathlineto{\pgfqpoint{3.309649in}{1.012140in}}%
\pgfusepath{stroke}%
\end{pgfscope}%
\begin{pgfscope}%
\pgfpathrectangle{\pgfqpoint{0.636356in}{0.905046in}}{\pgfqpoint{3.020670in}{2.030398in}} %
\pgfusepath{clip}%
\pgfsetbuttcap%
\pgfsetroundjoin%
\pgfsetlinewidth{1.003750pt}%
\definecolor{currentstroke}{rgb}{0.000000,0.000000,0.000000}%
\pgfsetstrokecolor{currentstroke}%
\pgfsetdash{}{0pt}%
\pgfpathmoveto{\pgfqpoint{3.339855in}{1.014280in}}%
\pgfpathlineto{\pgfqpoint{3.339855in}{1.025828in}}%
\pgfusepath{stroke}%
\end{pgfscope}%
\begin{pgfscope}%
\pgfpathrectangle{\pgfqpoint{0.636356in}{0.905046in}}{\pgfqpoint{3.020670in}{2.030398in}} %
\pgfusepath{clip}%
\pgfsetbuttcap%
\pgfsetroundjoin%
\pgfsetlinewidth{1.003750pt}%
\definecolor{currentstroke}{rgb}{0.000000,0.000000,0.000000}%
\pgfsetstrokecolor{currentstroke}%
\pgfsetdash{}{0pt}%
\pgfpathmoveto{\pgfqpoint{3.370062in}{0.999871in}}%
\pgfpathlineto{\pgfqpoint{3.370062in}{1.010650in}}%
\pgfusepath{stroke}%
\end{pgfscope}%
\begin{pgfscope}%
\pgfpathrectangle{\pgfqpoint{0.636356in}{0.905046in}}{\pgfqpoint{3.020670in}{2.030398in}} %
\pgfusepath{clip}%
\pgfsetbuttcap%
\pgfsetroundjoin%
\pgfsetlinewidth{1.003750pt}%
\definecolor{currentstroke}{rgb}{0.000000,0.000000,0.000000}%
\pgfsetstrokecolor{currentstroke}%
\pgfsetdash{}{0pt}%
\pgfpathmoveto{\pgfqpoint{3.400269in}{1.002977in}}%
\pgfpathlineto{\pgfqpoint{3.400269in}{1.013926in}}%
\pgfusepath{stroke}%
\end{pgfscope}%
\begin{pgfscope}%
\pgfpathrectangle{\pgfqpoint{0.636356in}{0.905046in}}{\pgfqpoint{3.020670in}{2.030398in}} %
\pgfusepath{clip}%
\pgfsetbuttcap%
\pgfsetroundjoin%
\pgfsetlinewidth{1.003750pt}%
\definecolor{currentstroke}{rgb}{0.000000,0.000000,0.000000}%
\pgfsetstrokecolor{currentstroke}%
\pgfsetdash{}{0pt}%
\pgfpathmoveto{\pgfqpoint{3.430475in}{0.991127in}}%
\pgfpathlineto{\pgfqpoint{3.430475in}{1.001411in}}%
\pgfusepath{stroke}%
\end{pgfscope}%
\begin{pgfscope}%
\pgfpathrectangle{\pgfqpoint{0.636356in}{0.905046in}}{\pgfqpoint{3.020670in}{2.030398in}} %
\pgfusepath{clip}%
\pgfsetbuttcap%
\pgfsetroundjoin%
\pgfsetlinewidth{1.003750pt}%
\definecolor{currentstroke}{rgb}{0.000000,0.000000,0.000000}%
\pgfsetstrokecolor{currentstroke}%
\pgfsetdash{}{0pt}%
\pgfpathmoveto{\pgfqpoint{3.460682in}{0.990845in}}%
\pgfpathlineto{\pgfqpoint{3.460682in}{1.001113in}}%
\pgfusepath{stroke}%
\end{pgfscope}%
\begin{pgfscope}%
\pgfpathrectangle{\pgfqpoint{0.636356in}{0.905046in}}{\pgfqpoint{3.020670in}{2.030398in}} %
\pgfusepath{clip}%
\pgfsetbuttcap%
\pgfsetroundjoin%
\pgfsetlinewidth{1.003750pt}%
\definecolor{currentstroke}{rgb}{0.000000,0.000000,0.000000}%
\pgfsetstrokecolor{currentstroke}%
\pgfsetdash{}{0pt}%
\pgfpathmoveto{\pgfqpoint{3.490889in}{0.986056in}}%
\pgfpathlineto{\pgfqpoint{3.490889in}{0.996041in}}%
\pgfusepath{stroke}%
\end{pgfscope}%
\begin{pgfscope}%
\pgfpathrectangle{\pgfqpoint{0.636356in}{0.905046in}}{\pgfqpoint{3.020670in}{2.030398in}} %
\pgfusepath{clip}%
\pgfsetbuttcap%
\pgfsetroundjoin%
\pgfsetlinewidth{1.003750pt}%
\definecolor{currentstroke}{rgb}{0.000000,0.000000,0.000000}%
\pgfsetstrokecolor{currentstroke}%
\pgfsetdash{}{0pt}%
\pgfpathmoveto{\pgfqpoint{3.521095in}{0.992255in}}%
\pgfpathlineto{\pgfqpoint{3.521095in}{1.002604in}}%
\pgfusepath{stroke}%
\end{pgfscope}%
\begin{pgfscope}%
\pgfpathrectangle{\pgfqpoint{0.636356in}{0.905046in}}{\pgfqpoint{3.020670in}{2.030398in}} %
\pgfusepath{clip}%
\pgfsetbuttcap%
\pgfsetroundjoin%
\pgfsetlinewidth{1.003750pt}%
\definecolor{currentstroke}{rgb}{0.000000,0.000000,0.000000}%
\pgfsetstrokecolor{currentstroke}%
\pgfsetdash{}{0pt}%
\pgfpathmoveto{\pgfqpoint{3.551302in}{0.990564in}}%
\pgfpathlineto{\pgfqpoint{3.551302in}{1.000815in}}%
\pgfusepath{stroke}%
\end{pgfscope}%
\begin{pgfscope}%
\pgfpathrectangle{\pgfqpoint{0.636356in}{0.905046in}}{\pgfqpoint{3.020670in}{2.030398in}} %
\pgfusepath{clip}%
\pgfsetbuttcap%
\pgfsetroundjoin%
\pgfsetlinewidth{1.003750pt}%
\definecolor{currentstroke}{rgb}{0.000000,0.000000,0.000000}%
\pgfsetstrokecolor{currentstroke}%
\pgfsetdash{}{0pt}%
\pgfpathmoveto{\pgfqpoint{3.581509in}{0.981270in}}%
\pgfpathlineto{\pgfqpoint{3.581509in}{0.990964in}}%
\pgfusepath{stroke}%
\end{pgfscope}%
\begin{pgfscope}%
\pgfpathrectangle{\pgfqpoint{0.636356in}{0.905046in}}{\pgfqpoint{3.020670in}{2.030398in}} %
\pgfusepath{clip}%
\pgfsetbuttcap%
\pgfsetroundjoin%
\pgfsetlinewidth{1.003750pt}%
\definecolor{currentstroke}{rgb}{0.000000,0.000000,0.000000}%
\pgfsetstrokecolor{currentstroke}%
\pgfsetdash{}{0pt}%
\pgfpathmoveto{\pgfqpoint{3.611715in}{0.969186in}}%
\pgfpathlineto{\pgfqpoint{3.611715in}{0.978103in}}%
\pgfusepath{stroke}%
\end{pgfscope}%
\begin{pgfscope}%
\pgfpathrectangle{\pgfqpoint{0.636356in}{0.905046in}}{\pgfqpoint{3.020670in}{2.030398in}} %
\pgfusepath{clip}%
\pgfsetbuttcap%
\pgfsetroundjoin%
\pgfsetlinewidth{1.003750pt}%
\definecolor{currentstroke}{rgb}{0.000000,0.000000,0.000000}%
\pgfsetstrokecolor{currentstroke}%
\pgfsetdash{}{0pt}%
\pgfpathmoveto{\pgfqpoint{3.641922in}{0.972274in}}%
\pgfpathlineto{\pgfqpoint{3.641922in}{0.981396in}}%
\pgfusepath{stroke}%
\end{pgfscope}%
\begin{pgfscope}%
\pgfpathrectangle{\pgfqpoint{0.636356in}{0.905046in}}{\pgfqpoint{3.020670in}{2.030398in}} %
\pgfusepath{clip}%
\pgfsetbuttcap%
\pgfsetroundjoin%
\definecolor{currentfill}{rgb}{0.000000,0.000000,0.000000}%
\pgfsetfillcolor{currentfill}%
\pgfsetlinewidth{1.003750pt}%
\definecolor{currentstroke}{rgb}{0.000000,0.000000,0.000000}%
\pgfsetstrokecolor{currentstroke}%
\pgfsetdash{}{0pt}%
\pgfsys@defobject{currentmarker}{\pgfqpoint{-0.013889in}{-0.000000in}}{\pgfqpoint{0.013889in}{0.000000in}}{%
\pgfpathmoveto{\pgfqpoint{0.013889in}{-0.000000in}}%
\pgfpathlineto{\pgfqpoint{-0.013889in}{0.000000in}}%
\pgfusepath{stroke,fill}%
}%
\begin{pgfscope}%
\pgfsys@transformshift{0.651459in}{2.822523in}%
\pgfsys@useobject{currentmarker}{}%
\end{pgfscope}%
\begin{pgfscope}%
\pgfsys@transformshift{0.681666in}{2.827712in}%
\pgfsys@useobject{currentmarker}{}%
\end{pgfscope}%
\begin{pgfscope}%
\pgfsys@transformshift{0.711873in}{2.721344in}%
\pgfsys@useobject{currentmarker}{}%
\end{pgfscope}%
\begin{pgfscope}%
\pgfsys@transformshift{0.742079in}{2.662547in}%
\pgfsys@useobject{currentmarker}{}%
\end{pgfscope}%
\begin{pgfscope}%
\pgfsys@transformshift{0.772286in}{2.647848in}%
\pgfsys@useobject{currentmarker}{}%
\end{pgfscope}%
\begin{pgfscope}%
\pgfsys@transformshift{0.802493in}{2.582432in}%
\pgfsys@useobject{currentmarker}{}%
\end{pgfscope}%
\begin{pgfscope}%
\pgfsys@transformshift{0.832699in}{2.547277in}%
\pgfsys@useobject{currentmarker}{}%
\end{pgfscope}%
\begin{pgfscope}%
\pgfsys@transformshift{0.862906in}{2.471790in}%
\pgfsys@useobject{currentmarker}{}%
\end{pgfscope}%
\begin{pgfscope}%
\pgfsys@transformshift{0.893113in}{2.409565in}%
\pgfsys@useobject{currentmarker}{}%
\end{pgfscope}%
\begin{pgfscope}%
\pgfsys@transformshift{0.923319in}{2.352533in}%
\pgfsys@useobject{currentmarker}{}%
\end{pgfscope}%
\begin{pgfscope}%
\pgfsys@transformshift{0.953526in}{2.281975in}%
\pgfsys@useobject{currentmarker}{}%
\end{pgfscope}%
\begin{pgfscope}%
\pgfsys@transformshift{0.983733in}{2.282263in}%
\pgfsys@useobject{currentmarker}{}%
\end{pgfscope}%
\begin{pgfscope}%
\pgfsys@transformshift{1.013940in}{2.240222in}%
\pgfsys@useobject{currentmarker}{}%
\end{pgfscope}%
\begin{pgfscope}%
\pgfsys@transformshift{1.044146in}{2.178897in}%
\pgfsys@useobject{currentmarker}{}%
\end{pgfscope}%
\begin{pgfscope}%
\pgfsys@transformshift{1.074353in}{2.169685in}%
\pgfsys@useobject{currentmarker}{}%
\end{pgfscope}%
\begin{pgfscope}%
\pgfsys@transformshift{1.104560in}{1.708035in}%
\pgfsys@useobject{currentmarker}{}%
\end{pgfscope}%
\begin{pgfscope}%
\pgfsys@transformshift{1.134766in}{0.905046in}%
\pgfsys@useobject{currentmarker}{}%
\end{pgfscope}%
\begin{pgfscope}%
\pgfsys@transformshift{1.164973in}{0.905046in}%
\pgfsys@useobject{currentmarker}{}%
\end{pgfscope}%
\begin{pgfscope}%
\pgfsys@transformshift{1.195180in}{0.905046in}%
\pgfsys@useobject{currentmarker}{}%
\end{pgfscope}%
\begin{pgfscope}%
\pgfsys@transformshift{1.225386in}{0.905046in}%
\pgfsys@useobject{currentmarker}{}%
\end{pgfscope}%
\begin{pgfscope}%
\pgfsys@transformshift{1.255593in}{1.500444in}%
\pgfsys@useobject{currentmarker}{}%
\end{pgfscope}%
\begin{pgfscope}%
\pgfsys@transformshift{1.285800in}{1.897478in}%
\pgfsys@useobject{currentmarker}{}%
\end{pgfscope}%
\begin{pgfscope}%
\pgfsys@transformshift{1.316007in}{1.822139in}%
\pgfsys@useobject{currentmarker}{}%
\end{pgfscope}%
\begin{pgfscope}%
\pgfsys@transformshift{1.346213in}{1.820702in}%
\pgfsys@useobject{currentmarker}{}%
\end{pgfscope}%
\begin{pgfscope}%
\pgfsys@transformshift{1.376420in}{1.797416in}%
\pgfsys@useobject{currentmarker}{}%
\end{pgfscope}%
\begin{pgfscope}%
\pgfsys@transformshift{1.406627in}{1.770108in}%
\pgfsys@useobject{currentmarker}{}%
\end{pgfscope}%
\begin{pgfscope}%
\pgfsys@transformshift{1.436833in}{1.729011in}%
\pgfsys@useobject{currentmarker}{}%
\end{pgfscope}%
\begin{pgfscope}%
\pgfsys@transformshift{1.467040in}{1.713494in}%
\pgfsys@useobject{currentmarker}{}%
\end{pgfscope}%
\begin{pgfscope}%
\pgfsys@transformshift{1.497247in}{1.700277in}%
\pgfsys@useobject{currentmarker}{}%
\end{pgfscope}%
\begin{pgfscope}%
\pgfsys@transformshift{1.527453in}{1.671549in}%
\pgfsys@useobject{currentmarker}{}%
\end{pgfscope}%
\begin{pgfscope}%
\pgfsys@transformshift{1.557660in}{1.625306in}%
\pgfsys@useobject{currentmarker}{}%
\end{pgfscope}%
\begin{pgfscope}%
\pgfsys@transformshift{1.587867in}{1.607790in}%
\pgfsys@useobject{currentmarker}{}%
\end{pgfscope}%
\begin{pgfscope}%
\pgfsys@transformshift{1.618073in}{1.565874in}%
\pgfsys@useobject{currentmarker}{}%
\end{pgfscope}%
\begin{pgfscope}%
\pgfsys@transformshift{1.648280in}{1.570179in}%
\pgfsys@useobject{currentmarker}{}%
\end{pgfscope}%
\begin{pgfscope}%
\pgfsys@transformshift{1.678487in}{1.499870in}%
\pgfsys@useobject{currentmarker}{}%
\end{pgfscope}%
\begin{pgfscope}%
\pgfsys@transformshift{1.708694in}{1.482371in}%
\pgfsys@useobject{currentmarker}{}%
\end{pgfscope}%
\begin{pgfscope}%
\pgfsys@transformshift{1.738900in}{1.477782in}%
\pgfsys@useobject{currentmarker}{}%
\end{pgfscope}%
\begin{pgfscope}%
\pgfsys@transformshift{1.769107in}{1.439068in}%
\pgfsys@useobject{currentmarker}{}%
\end{pgfscope}%
\begin{pgfscope}%
\pgfsys@transformshift{1.799314in}{1.420720in}%
\pgfsys@useobject{currentmarker}{}%
\end{pgfscope}%
\begin{pgfscope}%
\pgfsys@transformshift{1.829520in}{1.404382in}%
\pgfsys@useobject{currentmarker}{}%
\end{pgfscope}%
\begin{pgfscope}%
\pgfsys@transformshift{1.859727in}{1.407534in}%
\pgfsys@useobject{currentmarker}{}%
\end{pgfscope}%
\begin{pgfscope}%
\pgfsys@transformshift{1.889934in}{1.374580in}%
\pgfsys@useobject{currentmarker}{}%
\end{pgfscope}%
\begin{pgfscope}%
\pgfsys@transformshift{1.920140in}{1.362835in}%
\pgfsys@useobject{currentmarker}{}%
\end{pgfscope}%
\begin{pgfscope}%
\pgfsys@transformshift{1.950347in}{1.353383in}%
\pgfsys@useobject{currentmarker}{}%
\end{pgfscope}%
\begin{pgfscope}%
\pgfsys@transformshift{1.980554in}{1.326751in}%
\pgfsys@useobject{currentmarker}{}%
\end{pgfscope}%
\begin{pgfscope}%
\pgfsys@transformshift{2.010761in}{1.300131in}%
\pgfsys@useobject{currentmarker}{}%
\end{pgfscope}%
\begin{pgfscope}%
\pgfsys@transformshift{2.040967in}{1.312152in}%
\pgfsys@useobject{currentmarker}{}%
\end{pgfscope}%
\begin{pgfscope}%
\pgfsys@transformshift{2.071174in}{1.297555in}%
\pgfsys@useobject{currentmarker}{}%
\end{pgfscope}%
\begin{pgfscope}%
\pgfsys@transformshift{2.101381in}{1.271234in}%
\pgfsys@useobject{currentmarker}{}%
\end{pgfscope}%
\begin{pgfscope}%
\pgfsys@transformshift{2.131587in}{1.251787in}%
\pgfsys@useobject{currentmarker}{}%
\end{pgfscope}%
\begin{pgfscope}%
\pgfsys@transformshift{2.161794in}{1.254075in}%
\pgfsys@useobject{currentmarker}{}%
\end{pgfscope}%
\begin{pgfscope}%
\pgfsys@transformshift{2.192001in}{1.220061in}%
\pgfsys@useobject{currentmarker}{}%
\end{pgfscope}%
\begin{pgfscope}%
\pgfsys@transformshift{2.222207in}{1.224061in}%
\pgfsys@useobject{currentmarker}{}%
\end{pgfscope}%
\begin{pgfscope}%
\pgfsys@transformshift{2.252414in}{1.208919in}%
\pgfsys@useobject{currentmarker}{}%
\end{pgfscope}%
\begin{pgfscope}%
\pgfsys@transformshift{2.282621in}{1.204921in}%
\pgfsys@useobject{currentmarker}{}%
\end{pgfscope}%
\begin{pgfscope}%
\pgfsys@transformshift{2.312828in}{1.172089in}%
\pgfsys@useobject{currentmarker}{}%
\end{pgfscope}%
\begin{pgfscope}%
\pgfsys@transformshift{2.343034in}{1.196353in}%
\pgfsys@useobject{currentmarker}{}%
\end{pgfscope}%
\begin{pgfscope}%
\pgfsys@transformshift{2.373241in}{1.179795in}%
\pgfsys@useobject{currentmarker}{}%
\end{pgfscope}%
\begin{pgfscope}%
\pgfsys@transformshift{2.403448in}{1.180366in}%
\pgfsys@useobject{currentmarker}{}%
\end{pgfscope}%
\begin{pgfscope}%
\pgfsys@transformshift{2.433654in}{1.167524in}%
\pgfsys@useobject{currentmarker}{}%
\end{pgfscope}%
\begin{pgfscope}%
\pgfsys@transformshift{2.463861in}{1.141286in}%
\pgfsys@useobject{currentmarker}{}%
\end{pgfscope}%
\begin{pgfscope}%
\pgfsys@transformshift{2.494068in}{1.140431in}%
\pgfsys@useobject{currentmarker}{}%
\end{pgfscope}%
\begin{pgfscope}%
\pgfsys@transformshift{2.524274in}{1.127037in}%
\pgfsys@useobject{currentmarker}{}%
\end{pgfscope}%
\begin{pgfscope}%
\pgfsys@transformshift{2.554481in}{1.126752in}%
\pgfsys@useobject{currentmarker}{}%
\end{pgfscope}%
\begin{pgfscope}%
\pgfsys@transformshift{2.584688in}{1.105394in}%
\pgfsys@useobject{currentmarker}{}%
\end{pgfscope}%
\begin{pgfscope}%
\pgfsys@transformshift{2.614894in}{1.114220in}%
\pgfsys@useobject{currentmarker}{}%
\end{pgfscope}%
\begin{pgfscope}%
\pgfsys@transformshift{2.645101in}{1.114220in}%
\pgfsys@useobject{currentmarker}{}%
\end{pgfscope}%
\begin{pgfscope}%
\pgfsys@transformshift{2.675308in}{1.099986in}%
\pgfsys@useobject{currentmarker}{}%
\end{pgfscope}%
\begin{pgfscope}%
\pgfsys@transformshift{2.705515in}{1.099702in}%
\pgfsys@useobject{currentmarker}{}%
\end{pgfscope}%
\begin{pgfscope}%
\pgfsys@transformshift{2.735721in}{1.089460in}%
\pgfsys@useobject{currentmarker}{}%
\end{pgfscope}%
\begin{pgfscope}%
\pgfsys@transformshift{2.765928in}{1.081782in}%
\pgfsys@useobject{currentmarker}{}%
\end{pgfscope}%
\begin{pgfscope}%
\pgfsys@transformshift{2.796135in}{1.058486in}%
\pgfsys@useobject{currentmarker}{}%
\end{pgfscope}%
\begin{pgfscope}%
\pgfsys@transformshift{2.826341in}{1.070698in}%
\pgfsys@useobject{currentmarker}{}%
\end{pgfscope}%
\begin{pgfscope}%
\pgfsys@transformshift{2.856548in}{1.049972in}%
\pgfsys@useobject{currentmarker}{}%
\end{pgfscope}%
\begin{pgfscope}%
\pgfsys@transformshift{2.886755in}{1.055080in}%
\pgfsys@useobject{currentmarker}{}%
\end{pgfscope}%
\begin{pgfscope}%
\pgfsys@transformshift{2.916961in}{1.052526in}%
\pgfsys@useobject{currentmarker}{}%
\end{pgfscope}%
\begin{pgfscope}%
\pgfsys@transformshift{2.947168in}{1.040613in}%
\pgfsys@useobject{currentmarker}{}%
\end{pgfscope}%
\begin{pgfscope}%
\pgfsys@transformshift{2.977375in}{1.049689in}%
\pgfsys@useobject{currentmarker}{}%
\end{pgfscope}%
\begin{pgfscope}%
\pgfsys@transformshift{3.007582in}{1.044299in}%
\pgfsys@useobject{currentmarker}{}%
\end{pgfscope}%
\begin{pgfscope}%
\pgfsys@transformshift{3.037788in}{1.042882in}%
\pgfsys@useobject{currentmarker}{}%
\end{pgfscope}%
\begin{pgfscope}%
\pgfsys@transformshift{3.067995in}{1.038062in}%
\pgfsys@useobject{currentmarker}{}%
\end{pgfscope}%
\begin{pgfscope}%
\pgfsys@transformshift{3.098202in}{1.023616in}%
\pgfsys@useobject{currentmarker}{}%
\end{pgfscope}%
\begin{pgfscope}%
\pgfsys@transformshift{3.128408in}{1.031828in}%
\pgfsys@useobject{currentmarker}{}%
\end{pgfscope}%
\begin{pgfscope}%
\pgfsys@transformshift{3.158615in}{1.021352in}%
\pgfsys@useobject{currentmarker}{}%
\end{pgfscope}%
\begin{pgfscope}%
\pgfsys@transformshift{3.188822in}{1.012583in}%
\pgfsys@useobject{currentmarker}{}%
\end{pgfscope}%
\begin{pgfscope}%
\pgfsys@transformshift{3.219028in}{1.018805in}%
\pgfsys@useobject{currentmarker}{}%
\end{pgfscope}%
\begin{pgfscope}%
\pgfsys@transformshift{3.249235in}{1.013432in}%
\pgfsys@useobject{currentmarker}{}%
\end{pgfscope}%
\begin{pgfscope}%
\pgfsys@transformshift{3.279442in}{1.008344in}%
\pgfsys@useobject{currentmarker}{}%
\end{pgfscope}%
\begin{pgfscope}%
\pgfsys@transformshift{3.309649in}{1.001283in}%
\pgfsys@useobject{currentmarker}{}%
\end{pgfscope}%
\begin{pgfscope}%
\pgfsys@transformshift{3.339855in}{1.014280in}%
\pgfsys@useobject{currentmarker}{}%
\end{pgfscope}%
\begin{pgfscope}%
\pgfsys@transformshift{3.370062in}{0.999871in}%
\pgfsys@useobject{currentmarker}{}%
\end{pgfscope}%
\begin{pgfscope}%
\pgfsys@transformshift{3.400269in}{1.002977in}%
\pgfsys@useobject{currentmarker}{}%
\end{pgfscope}%
\begin{pgfscope}%
\pgfsys@transformshift{3.430475in}{0.991127in}%
\pgfsys@useobject{currentmarker}{}%
\end{pgfscope}%
\begin{pgfscope}%
\pgfsys@transformshift{3.460682in}{0.990845in}%
\pgfsys@useobject{currentmarker}{}%
\end{pgfscope}%
\begin{pgfscope}%
\pgfsys@transformshift{3.490889in}{0.986056in}%
\pgfsys@useobject{currentmarker}{}%
\end{pgfscope}%
\begin{pgfscope}%
\pgfsys@transformshift{3.521095in}{0.992255in}%
\pgfsys@useobject{currentmarker}{}%
\end{pgfscope}%
\begin{pgfscope}%
\pgfsys@transformshift{3.551302in}{0.990564in}%
\pgfsys@useobject{currentmarker}{}%
\end{pgfscope}%
\begin{pgfscope}%
\pgfsys@transformshift{3.581509in}{0.981270in}%
\pgfsys@useobject{currentmarker}{}%
\end{pgfscope}%
\begin{pgfscope}%
\pgfsys@transformshift{3.611715in}{0.969186in}%
\pgfsys@useobject{currentmarker}{}%
\end{pgfscope}%
\begin{pgfscope}%
\pgfsys@transformshift{3.641922in}{0.972274in}%
\pgfsys@useobject{currentmarker}{}%
\end{pgfscope}%
\end{pgfscope}%
\begin{pgfscope}%
\pgfpathrectangle{\pgfqpoint{0.636356in}{0.905046in}}{\pgfqpoint{3.020670in}{2.030398in}} %
\pgfusepath{clip}%
\pgfsetbuttcap%
\pgfsetroundjoin%
\definecolor{currentfill}{rgb}{0.000000,0.000000,0.000000}%
\pgfsetfillcolor{currentfill}%
\pgfsetlinewidth{1.003750pt}%
\definecolor{currentstroke}{rgb}{0.000000,0.000000,0.000000}%
\pgfsetstrokecolor{currentstroke}%
\pgfsetdash{}{0pt}%
\pgfsys@defobject{currentmarker}{\pgfqpoint{-0.013889in}{-0.000000in}}{\pgfqpoint{0.013889in}{0.000000in}}{%
\pgfpathmoveto{\pgfqpoint{0.013889in}{-0.000000in}}%
\pgfpathlineto{\pgfqpoint{-0.013889in}{0.000000in}}%
\pgfusepath{stroke,fill}%
}%
\begin{pgfscope}%
\pgfsys@transformshift{0.651459in}{2.869980in}%
\pgfsys@useobject{currentmarker}{}%
\end{pgfscope}%
\begin{pgfscope}%
\pgfsys@transformshift{0.681666in}{2.875233in}%
\pgfsys@useobject{currentmarker}{}%
\end{pgfscope}%
\begin{pgfscope}%
\pgfsys@transformshift{0.711873in}{2.767539in}%
\pgfsys@useobject{currentmarker}{}%
\end{pgfscope}%
\begin{pgfscope}%
\pgfsys@transformshift{0.742079in}{2.707993in}%
\pgfsys@useobject{currentmarker}{}%
\end{pgfscope}%
\begin{pgfscope}%
\pgfsys@transformshift{0.772286in}{2.693106in}%
\pgfsys@useobject{currentmarker}{}%
\end{pgfscope}%
\begin{pgfscope}%
\pgfsys@transformshift{0.802493in}{2.626837in}%
\pgfsys@useobject{currentmarker}{}%
\end{pgfscope}%
\begin{pgfscope}%
\pgfsys@transformshift{0.832699in}{2.591217in}%
\pgfsys@useobject{currentmarker}{}%
\end{pgfscope}%
\begin{pgfscope}%
\pgfsys@transformshift{0.862906in}{2.514715in}%
\pgfsys@useobject{currentmarker}{}%
\end{pgfscope}%
\begin{pgfscope}%
\pgfsys@transformshift{0.893113in}{2.451635in}%
\pgfsys@useobject{currentmarker}{}%
\end{pgfscope}%
\begin{pgfscope}%
\pgfsys@transformshift{0.923319in}{2.393804in}%
\pgfsys@useobject{currentmarker}{}%
\end{pgfscope}%
\begin{pgfscope}%
\pgfsys@transformshift{0.953526in}{2.322235in}%
\pgfsys@useobject{currentmarker}{}%
\end{pgfscope}%
\begin{pgfscope}%
\pgfsys@transformshift{0.983733in}{2.322527in}%
\pgfsys@useobject{currentmarker}{}%
\end{pgfscope}%
\begin{pgfscope}%
\pgfsys@transformshift{1.013940in}{2.279871in}%
\pgfsys@useobject{currentmarker}{}%
\end{pgfscope}%
\begin{pgfscope}%
\pgfsys@transformshift{1.044146in}{2.217632in}%
\pgfsys@useobject{currentmarker}{}%
\end{pgfscope}%
\begin{pgfscope}%
\pgfsys@transformshift{1.074353in}{2.208280in}%
\pgfsys@useobject{currentmarker}{}%
\end{pgfscope}%
\begin{pgfscope}%
\pgfsys@transformshift{1.104560in}{1.738848in}%
\pgfsys@useobject{currentmarker}{}%
\end{pgfscope}%
\begin{pgfscope}%
\pgfsys@transformshift{1.134766in}{0.905379in}%
\pgfsys@useobject{currentmarker}{}%
\end{pgfscope}%
\begin{pgfscope}%
\pgfsys@transformshift{1.164973in}{0.905379in}%
\pgfsys@useobject{currentmarker}{}%
\end{pgfscope}%
\begin{pgfscope}%
\pgfsys@transformshift{1.195180in}{0.905379in}%
\pgfsys@useobject{currentmarker}{}%
\end{pgfscope}%
\begin{pgfscope}%
\pgfsys@transformshift{1.225386in}{0.905379in}%
\pgfsys@useobject{currentmarker}{}%
\end{pgfscope}%
\begin{pgfscope}%
\pgfsys@transformshift{1.255593in}{1.527017in}%
\pgfsys@useobject{currentmarker}{}%
\end{pgfscope}%
\begin{pgfscope}%
\pgfsys@transformshift{1.285800in}{1.931701in}%
\pgfsys@useobject{currentmarker}{}%
\end{pgfscope}%
\begin{pgfscope}%
\pgfsys@transformshift{1.316007in}{1.855049in}%
\pgfsys@useobject{currentmarker}{}%
\end{pgfscope}%
\begin{pgfscope}%
\pgfsys@transformshift{1.346213in}{1.853586in}%
\pgfsys@useobject{currentmarker}{}%
\end{pgfscope}%
\begin{pgfscope}%
\pgfsys@transformshift{1.376420in}{1.829883in}%
\pgfsys@useobject{currentmarker}{}%
\end{pgfscope}%
\begin{pgfscope}%
\pgfsys@transformshift{1.406627in}{1.802079in}%
\pgfsys@useobject{currentmarker}{}%
\end{pgfscope}%
\begin{pgfscope}%
\pgfsys@transformshift{1.436833in}{1.760220in}%
\pgfsys@useobject{currentmarker}{}%
\end{pgfscope}%
\begin{pgfscope}%
\pgfsys@transformshift{1.467040in}{1.744411in}%
\pgfsys@useobject{currentmarker}{}%
\end{pgfscope}%
\begin{pgfscope}%
\pgfsys@transformshift{1.497247in}{1.730943in}%
\pgfsys@useobject{currentmarker}{}%
\end{pgfscope}%
\begin{pgfscope}%
\pgfsys@transformshift{1.527453in}{1.701660in}%
\pgfsys@useobject{currentmarker}{}%
\end{pgfscope}%
\begin{pgfscope}%
\pgfsys@transformshift{1.557660in}{1.654504in}%
\pgfsys@useobject{currentmarker}{}%
\end{pgfscope}%
\begin{pgfscope}%
\pgfsys@transformshift{1.587867in}{1.636634in}%
\pgfsys@useobject{currentmarker}{}%
\end{pgfscope}%
\begin{pgfscope}%
\pgfsys@transformshift{1.618073in}{1.593853in}%
\pgfsys@useobject{currentmarker}{}%
\end{pgfscope}%
\begin{pgfscope}%
\pgfsys@transformshift{1.648280in}{1.598249in}%
\pgfsys@useobject{currentmarker}{}%
\end{pgfscope}%
\begin{pgfscope}%
\pgfsys@transformshift{1.678487in}{1.526431in}%
\pgfsys@useobject{currentmarker}{}%
\end{pgfscope}%
\begin{pgfscope}%
\pgfsys@transformshift{1.708694in}{1.508542in}%
\pgfsys@useobject{currentmarker}{}%
\end{pgfscope}%
\begin{pgfscope}%
\pgfsys@transformshift{1.738900in}{1.503850in}%
\pgfsys@useobject{currentmarker}{}%
\end{pgfscope}%
\begin{pgfscope}%
\pgfsys@transformshift{1.769107in}{1.464249in}%
\pgfsys@useobject{currentmarker}{}%
\end{pgfscope}%
\begin{pgfscope}%
\pgfsys@transformshift{1.799314in}{1.445470in}%
\pgfsys@useobject{currentmarker}{}%
\end{pgfscope}%
\begin{pgfscope}%
\pgfsys@transformshift{1.829520in}{1.428741in}%
\pgfsys@useobject{currentmarker}{}%
\end{pgfscope}%
\begin{pgfscope}%
\pgfsys@transformshift{1.859727in}{1.431970in}%
\pgfsys@useobject{currentmarker}{}%
\end{pgfscope}%
\begin{pgfscope}%
\pgfsys@transformshift{1.889934in}{1.398211in}%
\pgfsys@useobject{currentmarker}{}%
\end{pgfscope}%
\begin{pgfscope}%
\pgfsys@transformshift{1.920140in}{1.386171in}%
\pgfsys@useobject{currentmarker}{}%
\end{pgfscope}%
\begin{pgfscope}%
\pgfsys@transformshift{1.950347in}{1.376480in}%
\pgfsys@useobject{currentmarker}{}%
\end{pgfscope}%
\begin{pgfscope}%
\pgfsys@transformshift{1.980554in}{1.349161in}%
\pgfsys@useobject{currentmarker}{}%
\end{pgfscope}%
\begin{pgfscope}%
\pgfsys@transformshift{2.010761in}{1.321831in}%
\pgfsys@useobject{currentmarker}{}%
\end{pgfscope}%
\begin{pgfscope}%
\pgfsys@transformshift{2.040967in}{1.334175in}%
\pgfsys@useobject{currentmarker}{}%
\end{pgfscope}%
\begin{pgfscope}%
\pgfsys@transformshift{2.071174in}{1.319185in}%
\pgfsys@useobject{currentmarker}{}%
\end{pgfscope}%
\begin{pgfscope}%
\pgfsys@transformshift{2.101381in}{1.292136in}%
\pgfsys@useobject{currentmarker}{}%
\end{pgfscope}%
\begin{pgfscope}%
\pgfsys@transformshift{2.131587in}{1.272135in}%
\pgfsys@useobject{currentmarker}{}%
\end{pgfscope}%
\begin{pgfscope}%
\pgfsys@transformshift{2.161794in}{1.274488in}%
\pgfsys@useobject{currentmarker}{}%
\end{pgfscope}%
\begin{pgfscope}%
\pgfsys@transformshift{2.192001in}{1.239469in}%
\pgfsys@useobject{currentmarker}{}%
\end{pgfscope}%
\begin{pgfscope}%
\pgfsys@transformshift{2.222207in}{1.243590in}%
\pgfsys@useobject{currentmarker}{}%
\end{pgfscope}%
\begin{pgfscope}%
\pgfsys@transformshift{2.252414in}{1.227986in}%
\pgfsys@useobject{currentmarker}{}%
\end{pgfscope}%
\begin{pgfscope}%
\pgfsys@transformshift{2.282621in}{1.223863in}%
\pgfsys@useobject{currentmarker}{}%
\end{pgfscope}%
\begin{pgfscope}%
\pgfsys@transformshift{2.312828in}{1.189981in}%
\pgfsys@useobject{currentmarker}{}%
\end{pgfscope}%
\begin{pgfscope}%
\pgfsys@transformshift{2.343034in}{1.215027in}%
\pgfsys@useobject{currentmarker}{}%
\end{pgfscope}%
\begin{pgfscope}%
\pgfsys@transformshift{2.373241in}{1.197939in}%
\pgfsys@useobject{currentmarker}{}%
\end{pgfscope}%
\begin{pgfscope}%
\pgfsys@transformshift{2.403448in}{1.198528in}%
\pgfsys@useobject{currentmarker}{}%
\end{pgfscope}%
\begin{pgfscope}%
\pgfsys@transformshift{2.433654in}{1.185265in}%
\pgfsys@useobject{currentmarker}{}%
\end{pgfscope}%
\begin{pgfscope}%
\pgfsys@transformshift{2.463861in}{1.158132in}%
\pgfsys@useobject{currentmarker}{}%
\end{pgfscope}%
\begin{pgfscope}%
\pgfsys@transformshift{2.494068in}{1.157247in}%
\pgfsys@useobject{currentmarker}{}%
\end{pgfscope}%
\begin{pgfscope}%
\pgfsys@transformshift{2.524274in}{1.143376in}%
\pgfsys@useobject{currentmarker}{}%
\end{pgfscope}%
\begin{pgfscope}%
\pgfsys@transformshift{2.554481in}{1.143081in}%
\pgfsys@useobject{currentmarker}{}%
\end{pgfscope}%
\begin{pgfscope}%
\pgfsys@transformshift{2.584688in}{1.120930in}%
\pgfsys@useobject{currentmarker}{}%
\end{pgfscope}%
\begin{pgfscope}%
\pgfsys@transformshift{2.614894in}{1.130088in}%
\pgfsys@useobject{currentmarker}{}%
\end{pgfscope}%
\begin{pgfscope}%
\pgfsys@transformshift{2.645101in}{1.130088in}%
\pgfsys@useobject{currentmarker}{}%
\end{pgfscope}%
\begin{pgfscope}%
\pgfsys@transformshift{2.675308in}{1.115316in}%
\pgfsys@useobject{currentmarker}{}%
\end{pgfscope}%
\begin{pgfscope}%
\pgfsys@transformshift{2.705515in}{1.115020in}%
\pgfsys@useobject{currentmarker}{}%
\end{pgfscope}%
\begin{pgfscope}%
\pgfsys@transformshift{2.735721in}{1.104378in}%
\pgfsys@useobject{currentmarker}{}%
\end{pgfscope}%
\begin{pgfscope}%
\pgfsys@transformshift{2.765928in}{1.096392in}%
\pgfsys@useobject{currentmarker}{}%
\end{pgfscope}%
\begin{pgfscope}%
\pgfsys@transformshift{2.796135in}{1.072119in}%
\pgfsys@useobject{currentmarker}{}%
\end{pgfscope}%
\begin{pgfscope}%
\pgfsys@transformshift{2.826341in}{1.084852in}%
\pgfsys@useobject{currentmarker}{}%
\end{pgfscope}%
\begin{pgfscope}%
\pgfsys@transformshift{2.856548in}{1.063230in}%
\pgfsys@useobject{currentmarker}{}%
\end{pgfscope}%
\begin{pgfscope}%
\pgfsys@transformshift{2.886755in}{1.068564in}%
\pgfsys@useobject{currentmarker}{}%
\end{pgfscope}%
\begin{pgfscope}%
\pgfsys@transformshift{2.916961in}{1.065897in}%
\pgfsys@useobject{currentmarker}{}%
\end{pgfscope}%
\begin{pgfscope}%
\pgfsys@transformshift{2.947168in}{1.053445in}%
\pgfsys@useobject{currentmarker}{}%
\end{pgfscope}%
\begin{pgfscope}%
\pgfsys@transformshift{2.977375in}{1.062933in}%
\pgfsys@useobject{currentmarker}{}%
\end{pgfscope}%
\begin{pgfscope}%
\pgfsys@transformshift{3.007582in}{1.057300in}%
\pgfsys@useobject{currentmarker}{}%
\end{pgfscope}%
\begin{pgfscope}%
\pgfsys@transformshift{3.037788in}{1.055818in}%
\pgfsys@useobject{currentmarker}{}%
\end{pgfscope}%
\begin{pgfscope}%
\pgfsys@transformshift{3.067995in}{1.050775in}%
\pgfsys@useobject{currentmarker}{}%
\end{pgfscope}%
\begin{pgfscope}%
\pgfsys@transformshift{3.098202in}{1.035635in}%
\pgfsys@useobject{currentmarker}{}%
\end{pgfscope}%
\begin{pgfscope}%
\pgfsys@transformshift{3.128408in}{1.044247in}%
\pgfsys@useobject{currentmarker}{}%
\end{pgfscope}%
\begin{pgfscope}%
\pgfsys@transformshift{3.158615in}{1.033258in}%
\pgfsys@useobject{currentmarker}{}%
\end{pgfscope}%
\begin{pgfscope}%
\pgfsys@transformshift{3.188822in}{1.024043in}%
\pgfsys@useobject{currentmarker}{}%
\end{pgfscope}%
\begin{pgfscope}%
\pgfsys@transformshift{3.219028in}{1.030584in}%
\pgfsys@useobject{currentmarker}{}%
\end{pgfscope}%
\begin{pgfscope}%
\pgfsys@transformshift{3.249235in}{1.024936in}%
\pgfsys@useobject{currentmarker}{}%
\end{pgfscope}%
\begin{pgfscope}%
\pgfsys@transformshift{3.279442in}{1.019581in}%
\pgfsys@useobject{currentmarker}{}%
\end{pgfscope}%
\begin{pgfscope}%
\pgfsys@transformshift{3.309649in}{1.012140in}%
\pgfsys@useobject{currentmarker}{}%
\end{pgfscope}%
\begin{pgfscope}%
\pgfsys@transformshift{3.339855in}{1.025828in}%
\pgfsys@useobject{currentmarker}{}%
\end{pgfscope}%
\begin{pgfscope}%
\pgfsys@transformshift{3.370062in}{1.010650in}%
\pgfsys@useobject{currentmarker}{}%
\end{pgfscope}%
\begin{pgfscope}%
\pgfsys@transformshift{3.400269in}{1.013926in}%
\pgfsys@useobject{currentmarker}{}%
\end{pgfscope}%
\begin{pgfscope}%
\pgfsys@transformshift{3.430475in}{1.001411in}%
\pgfsys@useobject{currentmarker}{}%
\end{pgfscope}%
\begin{pgfscope}%
\pgfsys@transformshift{3.460682in}{1.001113in}%
\pgfsys@useobject{currentmarker}{}%
\end{pgfscope}%
\begin{pgfscope}%
\pgfsys@transformshift{3.490889in}{0.996041in}%
\pgfsys@useobject{currentmarker}{}%
\end{pgfscope}%
\begin{pgfscope}%
\pgfsys@transformshift{3.521095in}{1.002604in}%
\pgfsys@useobject{currentmarker}{}%
\end{pgfscope}%
\begin{pgfscope}%
\pgfsys@transformshift{3.551302in}{1.000815in}%
\pgfsys@useobject{currentmarker}{}%
\end{pgfscope}%
\begin{pgfscope}%
\pgfsys@transformshift{3.581509in}{0.990964in}%
\pgfsys@useobject{currentmarker}{}%
\end{pgfscope}%
\begin{pgfscope}%
\pgfsys@transformshift{3.611715in}{0.978103in}%
\pgfsys@useobject{currentmarker}{}%
\end{pgfscope}%
\begin{pgfscope}%
\pgfsys@transformshift{3.641922in}{0.981396in}%
\pgfsys@useobject{currentmarker}{}%
\end{pgfscope}%
\end{pgfscope}%
\begin{pgfscope}%
\pgfsetbuttcap%
\pgfsetroundjoin%
\definecolor{currentfill}{rgb}{0.000000,0.000000,0.000000}%
\pgfsetfillcolor{currentfill}%
\pgfsetlinewidth{1.003750pt}%
\definecolor{currentstroke}{rgb}{0.000000,0.000000,0.000000}%
\pgfsetstrokecolor{currentstroke}%
\pgfsetdash{}{0pt}%
\pgfsys@defobject{currentmarker}{\pgfqpoint{-0.010417in}{-0.010417in}}{\pgfqpoint{0.010417in}{0.010417in}}{%
\pgfpathmoveto{\pgfqpoint{0.000000in}{-0.010417in}}%
\pgfpathcurveto{\pgfqpoint{0.002763in}{-0.010417in}}{\pgfqpoint{0.005412in}{-0.009319in}}{\pgfqpoint{0.007366in}{-0.007366in}}%
\pgfpathcurveto{\pgfqpoint{0.009319in}{-0.005412in}}{\pgfqpoint{0.010417in}{-0.002763in}}{\pgfqpoint{0.010417in}{0.000000in}}%
\pgfpathcurveto{\pgfqpoint{0.010417in}{0.002763in}}{\pgfqpoint{0.009319in}{0.005412in}}{\pgfqpoint{0.007366in}{0.007366in}}%
\pgfpathcurveto{\pgfqpoint{0.005412in}{0.009319in}}{\pgfqpoint{0.002763in}{0.010417in}}{\pgfqpoint{0.000000in}{0.010417in}}%
\pgfpathcurveto{\pgfqpoint{-0.002763in}{0.010417in}}{\pgfqpoint{-0.005412in}{0.009319in}}{\pgfqpoint{-0.007366in}{0.007366in}}%
\pgfpathcurveto{\pgfqpoint{-0.009319in}{0.005412in}}{\pgfqpoint{-0.010417in}{0.002763in}}{\pgfqpoint{-0.010417in}{0.000000in}}%
\pgfpathcurveto{\pgfqpoint{-0.010417in}{-0.002763in}}{\pgfqpoint{-0.009319in}{-0.005412in}}{\pgfqpoint{-0.007366in}{-0.007366in}}%
\pgfpathcurveto{\pgfqpoint{-0.005412in}{-0.009319in}}{\pgfqpoint{-0.002763in}{-0.010417in}}{\pgfqpoint{0.000000in}{-0.010417in}}%
\pgfpathclose%
\pgfusepath{stroke,fill}%
}%
\begin{pgfscope}%
\pgfsys@transformshift{0.651459in}{2.846106in}%
\pgfsys@useobject{currentmarker}{}%
\end{pgfscope}%
\begin{pgfscope}%
\pgfsys@transformshift{0.681666in}{2.851327in}%
\pgfsys@useobject{currentmarker}{}%
\end{pgfscope}%
\begin{pgfscope}%
\pgfsys@transformshift{0.711873in}{2.744296in}%
\pgfsys@useobject{currentmarker}{}%
\end{pgfscope}%
\begin{pgfscope}%
\pgfsys@transformshift{0.742079in}{2.685125in}%
\pgfsys@useobject{currentmarker}{}%
\end{pgfscope}%
\begin{pgfscope}%
\pgfsys@transformshift{0.772286in}{2.670332in}%
\pgfsys@useobject{currentmarker}{}%
\end{pgfscope}%
\begin{pgfscope}%
\pgfsys@transformshift{0.802493in}{2.604489in}%
\pgfsys@useobject{currentmarker}{}%
\end{pgfscope}%
\begin{pgfscope}%
\pgfsys@transformshift{0.832699in}{2.569102in}%
\pgfsys@useobject{currentmarker}{}%
\end{pgfscope}%
\begin{pgfscope}%
\pgfsys@transformshift{0.862906in}{2.493107in}%
\pgfsys@useobject{currentmarker}{}%
\end{pgfscope}%
\begin{pgfscope}%
\pgfsys@transformshift{0.893113in}{2.430455in}%
\pgfsys@useobject{currentmarker}{}%
\end{pgfscope}%
\begin{pgfscope}%
\pgfsys@transformshift{0.923319in}{2.373024in}%
\pgfsys@useobject{currentmarker}{}%
\end{pgfscope}%
\begin{pgfscope}%
\pgfsys@transformshift{0.953526in}{2.301960in}%
\pgfsys@useobject{currentmarker}{}%
\end{pgfscope}%
\begin{pgfscope}%
\pgfsys@transformshift{0.983733in}{2.302250in}%
\pgfsys@useobject{currentmarker}{}%
\end{pgfscope}%
\begin{pgfscope}%
\pgfsys@transformshift{1.013940in}{2.259902in}%
\pgfsys@useobject{currentmarker}{}%
\end{pgfscope}%
\begin{pgfscope}%
\pgfsys@transformshift{1.044146in}{2.198120in}%
\pgfsys@useobject{currentmarker}{}%
\end{pgfscope}%
\begin{pgfscope}%
\pgfsys@transformshift{1.074353in}{2.188838in}%
\pgfsys@useobject{currentmarker}{}%
\end{pgfscope}%
\begin{pgfscope}%
\pgfsys@transformshift{1.104560in}{1.723297in}%
\pgfsys@useobject{currentmarker}{}%
\end{pgfscope}%
\begin{pgfscope}%
\pgfsys@transformshift{1.134766in}{0.905046in}%
\pgfsys@useobject{currentmarker}{}%
\end{pgfscope}%
\begin{pgfscope}%
\pgfsys@transformshift{1.164973in}{0.905046in}%
\pgfsys@useobject{currentmarker}{}%
\end{pgfscope}%
\begin{pgfscope}%
\pgfsys@transformshift{1.195180in}{0.905046in}%
\pgfsys@useobject{currentmarker}{}%
\end{pgfscope}%
\begin{pgfscope}%
\pgfsys@transformshift{1.225386in}{0.905046in}%
\pgfsys@useobject{currentmarker}{}%
\end{pgfscope}%
\begin{pgfscope}%
\pgfsys@transformshift{1.255593in}{1.513585in}%
\pgfsys@useobject{currentmarker}{}%
\end{pgfscope}%
\begin{pgfscope}%
\pgfsys@transformshift{1.285800in}{1.914444in}%
\pgfsys@useobject{currentmarker}{}%
\end{pgfscope}%
\begin{pgfscope}%
\pgfsys@transformshift{1.316007in}{1.838449in}%
\pgfsys@useobject{currentmarker}{}%
\end{pgfscope}%
\begin{pgfscope}%
\pgfsys@transformshift{1.346213in}{1.836999in}%
\pgfsys@useobject{currentmarker}{}%
\end{pgfscope}%
\begin{pgfscope}%
\pgfsys@transformshift{1.376420in}{1.813504in}%
\pgfsys@useobject{currentmarker}{}%
\end{pgfscope}%
\begin{pgfscope}%
\pgfsys@transformshift{1.406627in}{1.785949in}%
\pgfsys@useobject{currentmarker}{}%
\end{pgfscope}%
\begin{pgfscope}%
\pgfsys@transformshift{1.436833in}{1.744471in}%
\pgfsys@useobject{currentmarker}{}%
\end{pgfscope}%
\begin{pgfscope}%
\pgfsys@transformshift{1.467040in}{1.728808in}%
\pgfsys@useobject{currentmarker}{}%
\end{pgfscope}%
\begin{pgfscope}%
\pgfsys@transformshift{1.497247in}{1.715465in}%
\pgfsys@useobject{currentmarker}{}%
\end{pgfscope}%
\begin{pgfscope}%
\pgfsys@transformshift{1.527453in}{1.686459in}%
\pgfsys@useobject{currentmarker}{}%
\end{pgfscope}%
\begin{pgfscope}%
\pgfsys@transformshift{1.557660in}{1.639760in}%
\pgfsys@useobject{currentmarker}{}%
\end{pgfscope}%
\begin{pgfscope}%
\pgfsys@transformshift{1.587867in}{1.622067in}%
\pgfsys@useobject{currentmarker}{}%
\end{pgfscope}%
\begin{pgfscope}%
\pgfsys@transformshift{1.618073in}{1.579718in}%
\pgfsys@useobject{currentmarker}{}%
\end{pgfscope}%
\begin{pgfscope}%
\pgfsys@transformshift{1.648280in}{1.584069in}%
\pgfsys@useobject{currentmarker}{}%
\end{pgfscope}%
\begin{pgfscope}%
\pgfsys@transformshift{1.678487in}{1.513005in}%
\pgfsys@useobject{currentmarker}{}%
\end{pgfscope}%
\begin{pgfscope}%
\pgfsys@transformshift{1.708694in}{1.495312in}%
\pgfsys@useobject{currentmarker}{}%
\end{pgfscope}%
\begin{pgfscope}%
\pgfsys@transformshift{1.738900in}{1.490671in}%
\pgfsys@useobject{currentmarker}{}%
\end{pgfscope}%
\begin{pgfscope}%
\pgfsys@transformshift{1.769107in}{1.451513in}%
\pgfsys@useobject{currentmarker}{}%
\end{pgfscope}%
\begin{pgfscope}%
\pgfsys@transformshift{1.799314in}{1.432950in}%
\pgfsys@useobject{currentmarker}{}%
\end{pgfscope}%
\begin{pgfscope}%
\pgfsys@transformshift{1.829520in}{1.416416in}%
\pgfsys@useobject{currentmarker}{}%
\end{pgfscope}%
\begin{pgfscope}%
\pgfsys@transformshift{1.859727in}{1.419607in}%
\pgfsys@useobject{currentmarker}{}%
\end{pgfscope}%
\begin{pgfscope}%
\pgfsys@transformshift{1.889934in}{1.386251in}%
\pgfsys@useobject{currentmarker}{}%
\end{pgfscope}%
\begin{pgfscope}%
\pgfsys@transformshift{1.920140in}{1.374358in}%
\pgfsys@useobject{currentmarker}{}%
\end{pgfscope}%
\begin{pgfscope}%
\pgfsys@transformshift{1.950347in}{1.364786in}%
\pgfsys@useobject{currentmarker}{}%
\end{pgfscope}%
\begin{pgfscope}%
\pgfsys@transformshift{1.980554in}{1.337811in}%
\pgfsys@useobject{currentmarker}{}%
\end{pgfscope}%
\begin{pgfscope}%
\pgfsys@transformshift{2.010761in}{1.310836in}%
\pgfsys@useobject{currentmarker}{}%
\end{pgfscope}%
\begin{pgfscope}%
\pgfsys@transformshift{2.040967in}{1.323018in}%
\pgfsys@useobject{currentmarker}{}%
\end{pgfscope}%
\begin{pgfscope}%
\pgfsys@transformshift{2.071174in}{1.308225in}%
\pgfsys@useobject{currentmarker}{}%
\end{pgfscope}%
\begin{pgfscope}%
\pgfsys@transformshift{2.101381in}{1.281540in}%
\pgfsys@useobject{currentmarker}{}%
\end{pgfscope}%
\begin{pgfscope}%
\pgfsys@transformshift{2.131587in}{1.261816in}%
\pgfsys@useobject{currentmarker}{}%
\end{pgfscope}%
\begin{pgfscope}%
\pgfsys@transformshift{2.161794in}{1.264137in}%
\pgfsys@useobject{currentmarker}{}%
\end{pgfscope}%
\begin{pgfscope}%
\pgfsys@transformshift{2.192001in}{1.229620in}%
\pgfsys@useobject{currentmarker}{}%
\end{pgfscope}%
\begin{pgfscope}%
\pgfsys@transformshift{2.222207in}{1.233681in}%
\pgfsys@useobject{currentmarker}{}%
\end{pgfscope}%
\begin{pgfscope}%
\pgfsys@transformshift{2.252414in}{1.218308in}%
\pgfsys@useobject{currentmarker}{}%
\end{pgfscope}%
\begin{pgfscope}%
\pgfsys@transformshift{2.282621in}{1.214247in}%
\pgfsys@useobject{currentmarker}{}%
\end{pgfscope}%
\begin{pgfscope}%
\pgfsys@transformshift{2.312828in}{1.180890in}%
\pgfsys@useobject{currentmarker}{}%
\end{pgfscope}%
\begin{pgfscope}%
\pgfsys@transformshift{2.343034in}{1.205545in}%
\pgfsys@useobject{currentmarker}{}%
\end{pgfscope}%
\begin{pgfscope}%
\pgfsys@transformshift{2.373241in}{1.188722in}%
\pgfsys@useobject{currentmarker}{}%
\end{pgfscope}%
\begin{pgfscope}%
\pgfsys@transformshift{2.403448in}{1.189302in}%
\pgfsys@useobject{currentmarker}{}%
\end{pgfscope}%
\begin{pgfscope}%
\pgfsys@transformshift{2.433654in}{1.176249in}%
\pgfsys@useobject{currentmarker}{}%
\end{pgfscope}%
\begin{pgfscope}%
\pgfsys@transformshift{2.463861in}{1.149564in}%
\pgfsys@useobject{currentmarker}{}%
\end{pgfscope}%
\begin{pgfscope}%
\pgfsys@transformshift{2.494068in}{1.148694in}%
\pgfsys@useobject{currentmarker}{}%
\end{pgfscope}%
\begin{pgfscope}%
\pgfsys@transformshift{2.524274in}{1.135061in}%
\pgfsys@useobject{currentmarker}{}%
\end{pgfscope}%
\begin{pgfscope}%
\pgfsys@transformshift{2.554481in}{1.134771in}%
\pgfsys@useobject{currentmarker}{}%
\end{pgfscope}%
\begin{pgfscope}%
\pgfsys@transformshift{2.584688in}{1.113017in}%
\pgfsys@useobject{currentmarker}{}%
\end{pgfscope}%
\begin{pgfscope}%
\pgfsys@transformshift{2.614894in}{1.122009in}%
\pgfsys@useobject{currentmarker}{}%
\end{pgfscope}%
\begin{pgfscope}%
\pgfsys@transformshift{2.645101in}{1.122009in}%
\pgfsys@useobject{currentmarker}{}%
\end{pgfscope}%
\begin{pgfscope}%
\pgfsys@transformshift{2.675308in}{1.107506in}%
\pgfsys@useobject{currentmarker}{}%
\end{pgfscope}%
\begin{pgfscope}%
\pgfsys@transformshift{2.705515in}{1.107216in}%
\pgfsys@useobject{currentmarker}{}%
\end{pgfscope}%
\begin{pgfscope}%
\pgfsys@transformshift{2.735721in}{1.096774in}%
\pgfsys@useobject{currentmarker}{}%
\end{pgfscope}%
\begin{pgfscope}%
\pgfsys@transformshift{2.765928in}{1.088942in}%
\pgfsys@useobject{currentmarker}{}%
\end{pgfscope}%
\begin{pgfscope}%
\pgfsys@transformshift{2.796135in}{1.065158in}%
\pgfsys@useobject{currentmarker}{}%
\end{pgfscope}%
\begin{pgfscope}%
\pgfsys@transformshift{2.826341in}{1.077630in}%
\pgfsys@useobject{currentmarker}{}%
\end{pgfscope}%
\begin{pgfscope}%
\pgfsys@transformshift{2.856548in}{1.056456in}%
\pgfsys@useobject{currentmarker}{}%
\end{pgfscope}%
\begin{pgfscope}%
\pgfsys@transformshift{2.886755in}{1.061677in}%
\pgfsys@useobject{currentmarker}{}%
\end{pgfscope}%
\begin{pgfscope}%
\pgfsys@transformshift{2.916961in}{1.059066in}%
\pgfsys@useobject{currentmarker}{}%
\end{pgfscope}%
\begin{pgfscope}%
\pgfsys@transformshift{2.947168in}{1.046884in}%
\pgfsys@useobject{currentmarker}{}%
\end{pgfscope}%
\begin{pgfscope}%
\pgfsys@transformshift{2.977375in}{1.056166in}%
\pgfsys@useobject{currentmarker}{}%
\end{pgfscope}%
\begin{pgfscope}%
\pgfsys@transformshift{3.007582in}{1.050655in}%
\pgfsys@useobject{currentmarker}{}%
\end{pgfscope}%
\begin{pgfscope}%
\pgfsys@transformshift{3.037788in}{1.049205in}%
\pgfsys@useobject{currentmarker}{}%
\end{pgfscope}%
\begin{pgfscope}%
\pgfsys@transformshift{3.067995in}{1.044274in}%
\pgfsys@useobject{currentmarker}{}%
\end{pgfscope}%
\begin{pgfscope}%
\pgfsys@transformshift{3.098202in}{1.029481in}%
\pgfsys@useobject{currentmarker}{}%
\end{pgfscope}%
\begin{pgfscope}%
\pgfsys@transformshift{3.128408in}{1.037892in}%
\pgfsys@useobject{currentmarker}{}%
\end{pgfscope}%
\begin{pgfscope}%
\pgfsys@transformshift{3.158615in}{1.027160in}%
\pgfsys@useobject{currentmarker}{}%
\end{pgfscope}%
\begin{pgfscope}%
\pgfsys@transformshift{3.188822in}{1.018168in}%
\pgfsys@useobject{currentmarker}{}%
\end{pgfscope}%
\begin{pgfscope}%
\pgfsys@transformshift{3.219028in}{1.024550in}%
\pgfsys@useobject{currentmarker}{}%
\end{pgfscope}%
\begin{pgfscope}%
\pgfsys@transformshift{3.249235in}{1.019039in}%
\pgfsys@useobject{currentmarker}{}%
\end{pgfscope}%
\begin{pgfscope}%
\pgfsys@transformshift{3.279442in}{1.013818in}%
\pgfsys@useobject{currentmarker}{}%
\end{pgfscope}%
\begin{pgfscope}%
\pgfsys@transformshift{3.309649in}{1.006566in}%
\pgfsys@useobject{currentmarker}{}%
\end{pgfscope}%
\begin{pgfscope}%
\pgfsys@transformshift{3.339855in}{1.019909in}%
\pgfsys@useobject{currentmarker}{}%
\end{pgfscope}%
\begin{pgfscope}%
\pgfsys@transformshift{3.370062in}{1.005116in}%
\pgfsys@useobject{currentmarker}{}%
\end{pgfscope}%
\begin{pgfscope}%
\pgfsys@transformshift{3.400269in}{1.008307in}%
\pgfsys@useobject{currentmarker}{}%
\end{pgfscope}%
\begin{pgfscope}%
\pgfsys@transformshift{3.430475in}{0.996124in}%
\pgfsys@useobject{currentmarker}{}%
\end{pgfscope}%
\begin{pgfscope}%
\pgfsys@transformshift{3.460682in}{0.995834in}%
\pgfsys@useobject{currentmarker}{}%
\end{pgfscope}%
\begin{pgfscope}%
\pgfsys@transformshift{3.490889in}{0.990903in}%
\pgfsys@useobject{currentmarker}{}%
\end{pgfscope}%
\begin{pgfscope}%
\pgfsys@transformshift{3.521095in}{0.997284in}%
\pgfsys@useobject{currentmarker}{}%
\end{pgfscope}%
\begin{pgfscope}%
\pgfsys@transformshift{3.551302in}{0.995544in}%
\pgfsys@useobject{currentmarker}{}%
\end{pgfscope}%
\begin{pgfscope}%
\pgfsys@transformshift{3.581509in}{0.985972in}%
\pgfsys@useobject{currentmarker}{}%
\end{pgfscope}%
\begin{pgfscope}%
\pgfsys@transformshift{3.611715in}{0.973500in}%
\pgfsys@useobject{currentmarker}{}%
\end{pgfscope}%
\begin{pgfscope}%
\pgfsys@transformshift{3.641922in}{0.976690in}%
\pgfsys@useobject{currentmarker}{}%
\end{pgfscope}%
\end{pgfscope}%
\end{pgfpicture}%
\makeatother%
\endgroup%

  \caption{
    Fit of an exponential function to upper and lower mass sideband of the $B^0\to\APDzero\APmuon\Pmuon$ data sample before the application of the multivariate selection.
    The blinded region has been masked in the fit.
    This is used to find the absolute number of background events in a $\pm1\sigma$ window around the nominal $B^0$ mass for the calculation of the FOM.
  }
  \label{fig:bkginitial}
\end{figure}

\begin{figure}
  \centering
  %% Creator: Matplotlib, PGF backend
%%
%% To include the figure in your LaTeX document, write
%%   \input{<filename>.pgf}
%%
%% Make sure the required packages are loaded in your preamble
%%   \usepackage{pgf}
%%
%% Figures using additional raster images can only be included by \input if
%% they are in the same directory as the main LaTeX file. For loading figures
%% from other directories you can use the `import` package
%%   \usepackage{import}
%% and then include the figures with
%%   \import{<path to file>}{<filename>.pgf}
%%
%% Matplotlib used the following preamble
%%   \usepackage{fontspec}
%%   \setmainfont{DejaVu Serif}
%%   \setsansfont{DejaVu Sans}
%%   \setmonofont{DejaVu Sans Mono}
%%
\begingroup%
\makeatletter%
\begin{pgfpicture}%
\pgfpathrectangle{\pgfpointorigin}{\pgfqpoint{3.866420in}{2.576728in}}%
\pgfusepath{use as bounding box, clip}%
\begin{pgfscope}%
\pgfsetbuttcap%
\pgfsetmiterjoin%
\definecolor{currentfill}{rgb}{1.000000,1.000000,1.000000}%
\pgfsetfillcolor{currentfill}%
\pgfsetlinewidth{0.000000pt}%
\definecolor{currentstroke}{rgb}{1.000000,1.000000,1.000000}%
\pgfsetstrokecolor{currentstroke}%
\pgfsetdash{}{0pt}%
\pgfpathmoveto{\pgfqpoint{-0.000000in}{0.000000in}}%
\pgfpathlineto{\pgfqpoint{3.866420in}{0.000000in}}%
\pgfpathlineto{\pgfqpoint{3.866420in}{2.576728in}}%
\pgfpathlineto{\pgfqpoint{-0.000000in}{2.576728in}}%
\pgfpathclose%
\pgfusepath{fill}%
\end{pgfscope}%
\begin{pgfscope}%
\pgfsetbuttcap%
\pgfsetmiterjoin%
\definecolor{currentfill}{rgb}{1.000000,1.000000,1.000000}%
\pgfsetfillcolor{currentfill}%
\pgfsetlinewidth{0.000000pt}%
\definecolor{currentstroke}{rgb}{0.000000,0.000000,0.000000}%
\pgfsetstrokecolor{currentstroke}%
\pgfsetstrokeopacity{0.000000}%
\pgfsetdash{}{0pt}%
\pgfpathmoveto{\pgfqpoint{0.559026in}{0.417391in}}%
\pgfpathlineto{\pgfqpoint{3.781074in}{0.417391in}}%
\pgfpathlineto{\pgfqpoint{3.781074in}{2.472963in}}%
\pgfpathlineto{\pgfqpoint{0.559026in}{2.472963in}}%
\pgfpathclose%
\pgfusepath{fill}%
\end{pgfscope}%
\begin{pgfscope}%
\pgfpathrectangle{\pgfqpoint{0.559026in}{0.417391in}}{\pgfqpoint{3.222048in}{2.055572in}} %
\pgfusepath{clip}%
\pgfsetrectcap%
\pgfsetroundjoin%
\pgfsetlinewidth{1.003750pt}%
\definecolor{currentstroke}{rgb}{0.392157,0.584314,0.929412}%
\pgfsetstrokecolor{currentstroke}%
\pgfsetdash{}{0pt}%
\pgfpathmoveto{\pgfqpoint{3.614508in}{0.417821in}}%
\pgfpathlineto{\pgfqpoint{3.539936in}{0.421692in}}%
\pgfpathlineto{\pgfqpoint{3.533858in}{0.423842in}}%
\pgfpathlineto{\pgfqpoint{3.521355in}{0.425562in}}%
\pgfpathlineto{\pgfqpoint{3.520818in}{0.426423in}}%
\pgfpathlineto{\pgfqpoint{3.516458in}{0.426853in}}%
\pgfpathlineto{\pgfqpoint{3.515846in}{0.424558in}}%
\pgfpathlineto{\pgfqpoint{3.501662in}{0.426512in}}%
\pgfpathlineto{\pgfqpoint{3.499130in}{0.427164in}}%
\pgfpathlineto{\pgfqpoint{3.480326in}{0.430747in}}%
\pgfpathlineto{\pgfqpoint{3.479657in}{0.431399in}}%
\pgfpathlineto{\pgfqpoint{3.473319in}{0.433027in}}%
\pgfpathlineto{\pgfqpoint{3.442084in}{0.443777in}}%
\pgfpathlineto{\pgfqpoint{3.440022in}{0.445406in}}%
\pgfpathlineto{\pgfqpoint{3.434155in}{0.446709in}}%
\pgfpathlineto{\pgfqpoint{3.433438in}{0.448012in}}%
\pgfpathlineto{\pgfqpoint{3.430451in}{0.449967in}}%
\pgfpathlineto{\pgfqpoint{3.430275in}{0.450618in}}%
\pgfpathlineto{\pgfqpoint{3.426247in}{0.452573in}}%
\pgfpathlineto{\pgfqpoint{3.417510in}{0.459088in}}%
\pgfpathlineto{\pgfqpoint{3.409258in}{0.464300in}}%
\pgfpathlineto{\pgfqpoint{3.406103in}{0.467884in}}%
\pgfpathlineto{\pgfqpoint{3.402172in}{0.473747in}}%
\pgfpathlineto{\pgfqpoint{3.400122in}{0.474725in}}%
\pgfpathlineto{\pgfqpoint{3.396094in}{0.479611in}}%
\pgfpathlineto{\pgfqpoint{3.392065in}{0.481240in}}%
\pgfpathlineto{\pgfqpoint{3.390865in}{0.483846in}}%
\pgfpathlineto{\pgfqpoint{3.388121in}{0.487429in}}%
\pgfpathlineto{\pgfqpoint{3.387538in}{0.489709in}}%
\pgfpathlineto{\pgfqpoint{3.385123in}{0.491338in}}%
\pgfpathlineto{\pgfqpoint{3.373319in}{0.503391in}}%
\pgfpathlineto{\pgfqpoint{3.372751in}{0.504043in}}%
\pgfpathlineto{\pgfqpoint{3.371435in}{0.507626in}}%
\pgfpathlineto{\pgfqpoint{3.368195in}{0.511209in}}%
\pgfpathlineto{\pgfqpoint{3.365678in}{0.512838in}}%
\pgfpathlineto{\pgfqpoint{3.362549in}{0.516096in}}%
\pgfpathlineto{\pgfqpoint{3.359326in}{0.520331in}}%
\pgfpathlineto{\pgfqpoint{3.357608in}{0.523914in}}%
\pgfpathlineto{\pgfqpoint{3.350705in}{0.534338in}}%
\pgfpathlineto{\pgfqpoint{3.348670in}{0.536944in}}%
\pgfpathlineto{\pgfqpoint{3.346526in}{0.540528in}}%
\pgfpathlineto{\pgfqpoint{3.339652in}{0.546391in}}%
\pgfpathlineto{\pgfqpoint{3.338482in}{0.550952in}}%
\pgfpathlineto{\pgfqpoint{3.333789in}{0.560073in}}%
\pgfpathlineto{\pgfqpoint{3.332343in}{0.563657in}}%
\pgfpathlineto{\pgfqpoint{3.329518in}{0.569195in}}%
\pgfpathlineto{\pgfqpoint{3.321504in}{0.582551in}}%
\pgfpathlineto{\pgfqpoint{3.320057in}{0.586134in}}%
\pgfpathlineto{\pgfqpoint{3.318636in}{0.587763in}}%
\pgfpathlineto{\pgfqpoint{3.316294in}{0.591346in}}%
\pgfpathlineto{\pgfqpoint{3.315443in}{0.592649in}}%
\pgfpathlineto{\pgfqpoint{3.314220in}{0.598187in}}%
\pgfpathlineto{\pgfqpoint{3.313213in}{0.600142in}}%
\pgfpathlineto{\pgfqpoint{3.312039in}{0.603725in}}%
\pgfpathlineto{\pgfqpoint{3.310649in}{0.606983in}}%
\pgfpathlineto{\pgfqpoint{3.309339in}{0.609263in}}%
\pgfpathlineto{\pgfqpoint{3.308331in}{0.613172in}}%
\pgfpathlineto{\pgfqpoint{3.307729in}{0.613823in}}%
\pgfpathlineto{\pgfqpoint{3.306632in}{0.617733in}}%
\pgfpathlineto{\pgfqpoint{3.303341in}{0.628157in}}%
\pgfpathlineto{\pgfqpoint{3.302167in}{0.631089in}}%
\pgfpathlineto{\pgfqpoint{3.300920in}{0.633043in}}%
\pgfpathlineto{\pgfqpoint{3.298052in}{0.637930in}}%
\pgfpathlineto{\pgfqpoint{3.297037in}{0.641513in}}%
\pgfpathlineto{\pgfqpoint{3.296417in}{0.643468in}}%
\pgfpathlineto{\pgfqpoint{3.294818in}{0.645096in}}%
\pgfpathlineto{\pgfqpoint{3.293941in}{0.651286in}}%
\pgfpathlineto{\pgfqpoint{3.293161in}{0.652589in}}%
\pgfpathlineto{\pgfqpoint{3.292045in}{0.653892in}}%
\pgfpathlineto{\pgfqpoint{3.290779in}{0.657801in}}%
\pgfpathlineto{\pgfqpoint{3.290181in}{0.659430in}}%
\pgfpathlineto{\pgfqpoint{3.289061in}{0.662687in}}%
\pgfpathlineto{\pgfqpoint{3.283307in}{0.675392in}}%
\pgfpathlineto{\pgfqpoint{3.282124in}{0.676695in}}%
\pgfpathlineto{\pgfqpoint{3.280922in}{0.681581in}}%
\pgfpathlineto{\pgfqpoint{3.279795in}{0.683862in}}%
\pgfpathlineto{\pgfqpoint{3.276683in}{0.691680in}}%
\pgfpathlineto{\pgfqpoint{3.275998in}{0.693634in}}%
\pgfpathlineto{\pgfqpoint{3.273944in}{0.702430in}}%
\pgfpathlineto{\pgfqpoint{3.271792in}{0.706990in}}%
\pgfpathlineto{\pgfqpoint{3.269901in}{0.715786in}}%
\pgfpathlineto{\pgfqpoint{3.267327in}{0.722627in}}%
\pgfpathlineto{\pgfqpoint{3.266137in}{0.727839in}}%
\pgfpathlineto{\pgfqpoint{3.266037in}{0.700679in}}%
\pgfpathlineto{\pgfqpoint{3.264950in}{0.703048in}}%
\pgfpathlineto{\pgfqpoint{3.264360in}{0.705120in}}%
\pgfpathlineto{\pgfqpoint{3.262136in}{0.712816in}}%
\pgfpathlineto{\pgfqpoint{3.261270in}{0.713704in}}%
\pgfpathlineto{\pgfqpoint{3.260183in}{0.719921in}}%
\pgfpathlineto{\pgfqpoint{3.258861in}{0.721993in}}%
\pgfpathlineto{\pgfqpoint{3.257408in}{0.727913in}}%
\pgfpathlineto{\pgfqpoint{3.255463in}{0.733537in}}%
\pgfpathlineto{\pgfqpoint{3.255055in}{0.735017in}}%
\pgfpathlineto{\pgfqpoint{3.253431in}{0.740346in}}%
\pgfpathlineto{\pgfqpoint{3.251846in}{0.745970in}}%
\pgfpathlineto{\pgfqpoint{3.250670in}{0.749226in}}%
\pgfpathlineto{\pgfqpoint{3.250096in}{0.750706in}}%
\pgfpathlineto{\pgfqpoint{3.248472in}{0.755739in}}%
\pgfpathlineto{\pgfqpoint{3.244001in}{0.774092in}}%
\pgfpathlineto{\pgfqpoint{3.242539in}{0.776756in}}%
\pgfpathlineto{\pgfqpoint{3.240976in}{0.783860in}}%
\pgfpathlineto{\pgfqpoint{3.239213in}{0.788893in}}%
\pgfpathlineto{\pgfqpoint{3.238388in}{0.790669in}}%
\pgfpathlineto{\pgfqpoint{3.237336in}{0.794221in}}%
\pgfpathlineto{\pgfqpoint{3.235958in}{0.797181in}}%
\pgfpathlineto{\pgfqpoint{3.234797in}{0.801029in}}%
\pgfpathlineto{\pgfqpoint{3.213785in}{0.876810in}}%
\pgfpathlineto{\pgfqpoint{3.212707in}{0.883618in}}%
\pgfpathlineto{\pgfqpoint{3.212484in}{0.883914in}}%
\pgfpathlineto{\pgfqpoint{3.211463in}{0.890130in}}%
\pgfpathlineto{\pgfqpoint{3.206789in}{0.909076in}}%
\pgfpathlineto{\pgfqpoint{3.203235in}{0.929797in}}%
\pgfpathlineto{\pgfqpoint{3.202111in}{0.932757in}}%
\pgfpathlineto{\pgfqpoint{3.200762in}{0.940157in}}%
\pgfpathlineto{\pgfqpoint{3.197256in}{0.956142in}}%
\pgfpathlineto{\pgfqpoint{3.195430in}{0.959695in}}%
\pgfpathlineto{\pgfqpoint{3.194277in}{0.962359in}}%
\pgfpathlineto{\pgfqpoint{3.192273in}{0.968279in}}%
\pgfpathlineto{\pgfqpoint{3.190285in}{0.977456in}}%
\pgfpathlineto{\pgfqpoint{3.188173in}{0.987520in}}%
\pgfpathlineto{\pgfqpoint{3.187778in}{0.988704in}}%
\pgfpathlineto{\pgfqpoint{3.187721in}{0.951301in}}%
\pgfpathlineto{\pgfqpoint{3.186650in}{0.955727in}}%
\pgfpathlineto{\pgfqpoint{3.163964in}{1.050613in}}%
\pgfpathlineto{\pgfqpoint{3.162619in}{1.057529in}}%
\pgfpathlineto{\pgfqpoint{3.161394in}{1.063615in}}%
\pgfpathlineto{\pgfqpoint{3.159754in}{1.073574in}}%
\pgfpathlineto{\pgfqpoint{3.157683in}{1.082427in}}%
\pgfpathlineto{\pgfqpoint{3.156636in}{1.088513in}}%
\pgfpathlineto{\pgfqpoint{3.155346in}{1.092662in}}%
\pgfpathlineto{\pgfqpoint{3.154169in}{1.099025in}}%
\pgfpathlineto{\pgfqpoint{3.153280in}{1.102898in}}%
\pgfpathlineto{\pgfqpoint{3.140458in}{1.160162in}}%
\pgfpathlineto{\pgfqpoint{3.140434in}{1.121310in}}%
\pgfpathlineto{\pgfqpoint{3.139349in}{1.126553in}}%
\pgfpathlineto{\pgfqpoint{3.137631in}{1.133370in}}%
\pgfpathlineto{\pgfqpoint{3.135812in}{1.141497in}}%
\pgfpathlineto{\pgfqpoint{3.134669in}{1.145691in}}%
\pgfpathlineto{\pgfqpoint{3.133402in}{1.151721in}}%
\pgfpathlineto{\pgfqpoint{3.123664in}{1.199960in}}%
\pgfpathlineto{\pgfqpoint{3.122919in}{1.203892in}}%
\pgfpathlineto{\pgfqpoint{3.120907in}{1.212282in}}%
\pgfpathlineto{\pgfqpoint{3.119005in}{1.218574in}}%
\pgfpathlineto{\pgfqpoint{3.113115in}{1.249772in}}%
\pgfpathlineto{\pgfqpoint{3.113100in}{1.213103in}}%
\pgfpathlineto{\pgfqpoint{3.112007in}{1.217363in}}%
\pgfpathlineto{\pgfqpoint{3.111263in}{1.221624in}}%
\pgfpathlineto{\pgfqpoint{3.106251in}{1.244681in}}%
\pgfpathlineto{\pgfqpoint{3.104549in}{1.254706in}}%
\pgfpathlineto{\pgfqpoint{3.096309in}{1.295055in}}%
\pgfpathlineto{\pgfqpoint{3.095399in}{1.298313in}}%
\pgfpathlineto{\pgfqpoint{3.083843in}{1.363223in}}%
\pgfpathlineto{\pgfqpoint{3.082266in}{1.370992in}}%
\pgfpathlineto{\pgfqpoint{3.078030in}{1.398811in}}%
\pgfpathlineto{\pgfqpoint{3.077235in}{1.402320in}}%
\pgfpathlineto{\pgfqpoint{3.044336in}{1.573993in}}%
\pgfpathlineto{\pgfqpoint{3.042566in}{1.587025in}}%
\pgfpathlineto{\pgfqpoint{3.041279in}{1.592288in}}%
\pgfpathlineto{\pgfqpoint{3.039284in}{1.605320in}}%
\pgfpathlineto{\pgfqpoint{3.039257in}{1.559821in}}%
\pgfpathlineto{\pgfqpoint{3.038157in}{1.565846in}}%
\pgfpathlineto{\pgfqpoint{3.036577in}{1.573559in}}%
\pgfpathlineto{\pgfqpoint{3.034934in}{1.581512in}}%
\pgfpathlineto{\pgfqpoint{3.033648in}{1.587538in}}%
\pgfpathlineto{\pgfqpoint{3.031765in}{1.597902in}}%
\pgfpathlineto{\pgfqpoint{3.030781in}{1.601517in}}%
\pgfpathlineto{\pgfqpoint{3.027640in}{1.623209in}}%
\pgfpathlineto{\pgfqpoint{3.027076in}{1.626824in}}%
\pgfpathlineto{\pgfqpoint{3.025796in}{1.633331in}}%
\pgfpathlineto{\pgfqpoint{3.024262in}{1.642731in}}%
\pgfpathlineto{\pgfqpoint{3.023802in}{1.645623in}}%
\pgfpathlineto{\pgfqpoint{3.022530in}{1.651890in}}%
\pgfpathlineto{\pgfqpoint{3.022039in}{1.655746in}}%
\pgfpathlineto{\pgfqpoint{3.020771in}{1.662977in}}%
\pgfpathlineto{\pgfqpoint{3.020067in}{1.667797in}}%
\pgfpathlineto{\pgfqpoint{3.018699in}{1.674546in}}%
\pgfpathlineto{\pgfqpoint{3.010570in}{1.728052in}}%
\pgfpathlineto{\pgfqpoint{3.010262in}{1.730703in}}%
\pgfpathlineto{\pgfqpoint{3.004947in}{1.768302in}}%
\pgfpathlineto{\pgfqpoint{3.004935in}{1.722339in}}%
\pgfpathlineto{\pgfqpoint{3.003851in}{1.730255in}}%
\pgfpathlineto{\pgfqpoint{3.003225in}{1.734912in}}%
\pgfpathlineto{\pgfqpoint{3.001965in}{1.741198in}}%
\pgfpathlineto{\pgfqpoint{3.000866in}{1.750278in}}%
\pgfpathlineto{\pgfqpoint{2.999363in}{1.760289in}}%
\pgfpathlineto{\pgfqpoint{2.991844in}{1.807551in}}%
\pgfpathlineto{\pgfqpoint{2.991826in}{1.764879in}}%
\pgfpathlineto{\pgfqpoint{2.990780in}{1.770972in}}%
\pgfpathlineto{\pgfqpoint{2.989171in}{1.779096in}}%
\pgfpathlineto{\pgfqpoint{2.987390in}{1.788800in}}%
\pgfpathlineto{\pgfqpoint{2.987248in}{1.750369in}}%
\pgfpathlineto{\pgfqpoint{2.986208in}{1.757169in}}%
\pgfpathlineto{\pgfqpoint{2.985590in}{1.759582in}}%
\pgfpathlineto{\pgfqpoint{2.985521in}{1.724927in}}%
\pgfpathlineto{\pgfqpoint{2.984523in}{1.729200in}}%
\pgfpathlineto{\pgfqpoint{2.983784in}{1.731978in}}%
\pgfpathlineto{\pgfqpoint{2.982890in}{1.736679in}}%
\pgfpathlineto{\pgfqpoint{2.980918in}{1.747150in}}%
\pgfpathlineto{\pgfqpoint{2.979919in}{1.752919in}}%
\pgfpathlineto{\pgfqpoint{2.978653in}{1.758902in}}%
\pgfpathlineto{\pgfqpoint{2.978469in}{1.759544in}}%
\pgfpathlineto{\pgfqpoint{2.978366in}{1.727373in}}%
\pgfpathlineto{\pgfqpoint{2.977301in}{1.734672in}}%
\pgfpathlineto{\pgfqpoint{2.967055in}{1.789733in}}%
\pgfpathlineto{\pgfqpoint{2.967031in}{1.759210in}}%
\pgfpathlineto{\pgfqpoint{2.965985in}{1.766347in}}%
\pgfpathlineto{\pgfqpoint{2.965624in}{1.768793in}}%
\pgfpathlineto{\pgfqpoint{2.965595in}{1.740382in}}%
\pgfpathlineto{\pgfqpoint{2.964508in}{1.744574in}}%
\pgfpathlineto{\pgfqpoint{2.961313in}{1.760543in}}%
\pgfpathlineto{\pgfqpoint{2.956896in}{1.784296in}}%
\pgfpathlineto{\pgfqpoint{2.956097in}{1.786691in}}%
\pgfpathlineto{\pgfqpoint{2.954422in}{1.795673in}}%
\pgfpathlineto{\pgfqpoint{2.953071in}{1.804256in}}%
\pgfpathlineto{\pgfqpoint{2.951355in}{1.815035in}}%
\pgfpathlineto{\pgfqpoint{2.949908in}{1.820624in}}%
\pgfpathlineto{\pgfqpoint{2.947801in}{1.832601in}}%
\pgfpathlineto{\pgfqpoint{2.945956in}{1.841184in}}%
\pgfpathlineto{\pgfqpoint{2.943958in}{1.852162in}}%
\pgfpathlineto{\pgfqpoint{2.942519in}{1.858549in}}%
\pgfpathlineto{\pgfqpoint{2.939571in}{1.875516in}}%
\pgfpathlineto{\pgfqpoint{2.939204in}{1.877712in}}%
\pgfpathlineto{\pgfqpoint{2.936432in}{1.892882in}}%
\pgfpathlineto{\pgfqpoint{2.935891in}{1.895077in}}%
\pgfpathlineto{\pgfqpoint{2.930777in}{1.924619in}}%
\pgfpathlineto{\pgfqpoint{2.930476in}{1.926815in}}%
\pgfpathlineto{\pgfqpoint{2.928836in}{1.936995in}}%
\pgfpathlineto{\pgfqpoint{2.926409in}{1.953562in}}%
\pgfpathlineto{\pgfqpoint{2.925922in}{1.957754in}}%
\pgfpathlineto{\pgfqpoint{2.919591in}{1.996677in}}%
\pgfpathlineto{\pgfqpoint{2.918767in}{2.001068in}}%
\pgfpathlineto{\pgfqpoint{2.917880in}{2.005859in}}%
\pgfpathlineto{\pgfqpoint{2.917851in}{1.974363in}}%
\pgfpathlineto{\pgfqpoint{2.916759in}{1.981210in}}%
\pgfpathlineto{\pgfqpoint{2.914902in}{1.989428in}}%
\pgfpathlineto{\pgfqpoint{2.908675in}{2.023862in}}%
\pgfpathlineto{\pgfqpoint{2.902928in}{2.058100in}}%
\pgfpathlineto{\pgfqpoint{2.900771in}{2.073165in}}%
\pgfpathlineto{\pgfqpoint{2.899754in}{2.078839in}}%
\pgfpathlineto{\pgfqpoint{2.898155in}{2.085882in}}%
\pgfpathlineto{\pgfqpoint{2.896946in}{2.090382in}}%
\pgfpathlineto{\pgfqpoint{2.894254in}{2.105252in}}%
\pgfpathlineto{\pgfqpoint{2.894195in}{2.073566in}}%
\pgfpathlineto{\pgfqpoint{2.893134in}{2.078558in}}%
\pgfpathlineto{\pgfqpoint{2.892374in}{2.082397in}}%
\pgfpathlineto{\pgfqpoint{2.889039in}{2.103707in}}%
\pgfpathlineto{\pgfqpoint{2.887614in}{2.112537in}}%
\pgfpathlineto{\pgfqpoint{2.886373in}{2.120792in}}%
\pgfpathlineto{\pgfqpoint{2.885514in}{2.124824in}}%
\pgfpathlineto{\pgfqpoint{2.884335in}{2.130775in}}%
\pgfpathlineto{\pgfqpoint{2.883709in}{2.134231in}}%
\pgfpathlineto{\pgfqpoint{2.883335in}{2.136918in}}%
\pgfpathlineto{\pgfqpoint{2.883253in}{2.106254in}}%
\pgfpathlineto{\pgfqpoint{2.882185in}{2.113608in}}%
\pgfpathlineto{\pgfqpoint{2.879804in}{2.128315in}}%
\pgfpathlineto{\pgfqpoint{2.879774in}{2.099247in}}%
\pgfpathlineto{\pgfqpoint{2.878716in}{2.105363in}}%
\pgfpathlineto{\pgfqpoint{2.878099in}{2.107773in}}%
\pgfpathlineto{\pgfqpoint{2.876918in}{2.115187in}}%
\pgfpathlineto{\pgfqpoint{2.876185in}{2.119079in}}%
\pgfpathlineto{\pgfqpoint{2.875335in}{2.122601in}}%
\pgfpathlineto{\pgfqpoint{2.875296in}{2.094926in}}%
\pgfpathlineto{\pgfqpoint{2.874216in}{2.102220in}}%
\pgfpathlineto{\pgfqpoint{2.872787in}{2.109878in}}%
\pgfpathlineto{\pgfqpoint{2.870777in}{2.121548in}}%
\pgfpathlineto{\pgfqpoint{2.870218in}{2.125012in}}%
\pgfpathlineto{\pgfqpoint{2.869313in}{2.130118in}}%
\pgfpathlineto{\pgfqpoint{2.868134in}{2.138688in}}%
\pgfpathlineto{\pgfqpoint{2.867534in}{2.140329in}}%
\pgfpathlineto{\pgfqpoint{2.866349in}{2.147440in}}%
\pgfpathlineto{\pgfqpoint{2.861127in}{2.177344in}}%
\pgfpathlineto{\pgfqpoint{2.859677in}{2.185185in}}%
\pgfpathlineto{\pgfqpoint{2.858911in}{2.188831in}}%
\pgfpathlineto{\pgfqpoint{2.855130in}{2.209254in}}%
\pgfpathlineto{\pgfqpoint{2.853627in}{2.216000in}}%
\pgfpathlineto{\pgfqpoint{2.853580in}{2.188051in}}%
\pgfpathlineto{\pgfqpoint{2.852483in}{2.196308in}}%
\pgfpathlineto{\pgfqpoint{2.848636in}{2.220183in}}%
\pgfpathlineto{\pgfqpoint{2.848563in}{2.193304in}}%
\pgfpathlineto{\pgfqpoint{2.847521in}{2.199316in}}%
\pgfpathlineto{\pgfqpoint{2.845531in}{2.208865in}}%
\pgfpathlineto{\pgfqpoint{2.838254in}{2.246353in}}%
\pgfpathlineto{\pgfqpoint{2.836993in}{2.254487in}}%
\pgfpathlineto{\pgfqpoint{2.836592in}{2.257140in}}%
\pgfpathlineto{\pgfqpoint{2.835383in}{2.262444in}}%
\pgfpathlineto{\pgfqpoint{2.834655in}{2.268280in}}%
\pgfpathlineto{\pgfqpoint{2.834607in}{2.241751in}}%
\pgfpathlineto{\pgfqpoint{2.833522in}{2.248723in}}%
\pgfpathlineto{\pgfqpoint{2.833386in}{2.248897in}}%
\pgfpathlineto{\pgfqpoint{2.833377in}{2.223618in}}%
\pgfpathlineto{\pgfqpoint{2.832277in}{2.228946in}}%
\pgfpathlineto{\pgfqpoint{2.831707in}{2.233072in}}%
\pgfpathlineto{\pgfqpoint{2.831668in}{2.208900in}}%
\pgfpathlineto{\pgfqpoint{2.830619in}{2.215006in}}%
\pgfpathlineto{\pgfqpoint{2.829820in}{2.219755in}}%
\pgfpathlineto{\pgfqpoint{2.828955in}{2.226539in}}%
\pgfpathlineto{\pgfqpoint{2.828941in}{2.203276in}}%
\pgfpathlineto{\pgfqpoint{2.827854in}{2.209638in}}%
\pgfpathlineto{\pgfqpoint{2.820056in}{2.249149in}}%
\pgfpathlineto{\pgfqpoint{2.820021in}{2.226368in}}%
\pgfpathlineto{\pgfqpoint{2.818935in}{2.233147in}}%
\pgfpathlineto{\pgfqpoint{2.818488in}{2.235296in}}%
\pgfpathlineto{\pgfqpoint{2.817202in}{2.243398in}}%
\pgfpathlineto{\pgfqpoint{2.816750in}{2.246043in}}%
\pgfpathlineto{\pgfqpoint{2.816191in}{2.248027in}}%
\pgfpathlineto{\pgfqpoint{2.816182in}{2.225980in}}%
\pgfpathlineto{\pgfqpoint{2.815078in}{2.230390in}}%
\pgfpathlineto{\pgfqpoint{2.809953in}{2.257996in}}%
\pgfpathlineto{\pgfqpoint{2.809925in}{2.236505in}}%
\pgfpathlineto{\pgfqpoint{2.808837in}{2.241832in}}%
\pgfpathlineto{\pgfqpoint{2.807863in}{2.247967in}}%
\pgfpathlineto{\pgfqpoint{2.807859in}{2.227223in}}%
\pgfpathlineto{\pgfqpoint{2.806797in}{2.233128in}}%
\pgfpathlineto{\pgfqpoint{2.805880in}{2.239034in}}%
\pgfpathlineto{\pgfqpoint{2.803799in}{2.251324in}}%
\pgfpathlineto{\pgfqpoint{2.801971in}{2.260103in}}%
\pgfpathlineto{\pgfqpoint{2.801808in}{2.220932in}}%
\pgfpathlineto{\pgfqpoint{2.800845in}{2.226554in}}%
\pgfpathlineto{\pgfqpoint{2.800616in}{2.227491in}}%
\pgfpathlineto{\pgfqpoint{2.800363in}{2.192292in}}%
\pgfpathlineto{\pgfqpoint{2.799465in}{2.197187in}}%
\pgfpathlineto{\pgfqpoint{2.796545in}{2.210648in}}%
\pgfpathlineto{\pgfqpoint{2.796088in}{2.212637in}}%
\pgfpathlineto{\pgfqpoint{2.793878in}{2.222580in}}%
\pgfpathlineto{\pgfqpoint{2.793008in}{2.228240in}}%
\pgfpathlineto{\pgfqpoint{2.792953in}{2.210328in}}%
\pgfpathlineto{\pgfqpoint{2.791869in}{2.216841in}}%
\pgfpathlineto{\pgfqpoint{2.790128in}{2.224414in}}%
\pgfpathlineto{\pgfqpoint{2.786428in}{2.243043in}}%
\pgfpathlineto{\pgfqpoint{2.784006in}{2.256371in}}%
\pgfpathlineto{\pgfqpoint{2.784002in}{2.238627in}}%
\pgfpathlineto{\pgfqpoint{2.782894in}{2.243577in}}%
\pgfpathlineto{\pgfqpoint{2.781719in}{2.251527in}}%
\pgfpathlineto{\pgfqpoint{2.779390in}{2.263377in}}%
\pgfpathlineto{\pgfqpoint{2.779250in}{2.229758in}}%
\pgfpathlineto{\pgfqpoint{2.778260in}{2.234616in}}%
\pgfpathlineto{\pgfqpoint{2.777894in}{2.236972in}}%
\pgfpathlineto{\pgfqpoint{2.776746in}{2.242860in}}%
\pgfpathlineto{\pgfqpoint{2.772702in}{2.263912in}}%
\pgfpathlineto{\pgfqpoint{2.771628in}{2.268623in}}%
\pgfpathlineto{\pgfqpoint{2.768830in}{2.282314in}}%
\pgfpathlineto{\pgfqpoint{2.766286in}{2.293503in}}%
\pgfpathlineto{\pgfqpoint{2.765410in}{2.298950in}}%
\pgfpathlineto{\pgfqpoint{2.764385in}{2.271568in}}%
\pgfpathlineto{\pgfqpoint{2.764292in}{2.271858in}}%
\pgfpathlineto{\pgfqpoint{2.763640in}{2.274750in}}%
\pgfpathlineto{\pgfqpoint{2.763628in}{2.258772in}}%
\pgfpathlineto{\pgfqpoint{2.762521in}{2.264507in}}%
\pgfpathlineto{\pgfqpoint{2.760415in}{2.274112in}}%
\pgfpathlineto{\pgfqpoint{2.759504in}{2.247098in}}%
\pgfpathlineto{\pgfqpoint{2.759299in}{2.247802in}}%
\pgfpathlineto{\pgfqpoint{2.758464in}{2.251891in}}%
\pgfpathlineto{\pgfqpoint{2.757276in}{2.258658in}}%
\pgfpathlineto{\pgfqpoint{2.756242in}{2.262465in}}%
\pgfpathlineto{\pgfqpoint{2.756218in}{2.247693in}}%
\pgfpathlineto{\pgfqpoint{2.755140in}{2.252727in}}%
\pgfpathlineto{\pgfqpoint{2.754001in}{2.258321in}}%
\pgfpathlineto{\pgfqpoint{2.752459in}{2.266712in}}%
\pgfpathlineto{\pgfqpoint{2.751532in}{2.270627in}}%
\pgfpathlineto{\pgfqpoint{2.750111in}{2.276641in}}%
\pgfpathlineto{\pgfqpoint{2.748160in}{2.286849in}}%
\pgfpathlineto{\pgfqpoint{2.748137in}{2.272037in}}%
\pgfpathlineto{\pgfqpoint{2.747092in}{2.277448in}}%
\pgfpathlineto{\pgfqpoint{2.746321in}{2.281055in}}%
\pgfpathlineto{\pgfqpoint{2.746313in}{2.266573in}}%
\pgfpathlineto{\pgfqpoint{2.745247in}{2.271529in}}%
\pgfpathlineto{\pgfqpoint{2.745109in}{2.272217in}}%
\pgfpathlineto{\pgfqpoint{2.744033in}{2.249724in}}%
\pgfpathlineto{\pgfqpoint{2.743991in}{2.249859in}}%
\pgfpathlineto{\pgfqpoint{2.742695in}{2.253520in}}%
\pgfpathlineto{\pgfqpoint{2.737416in}{2.277112in}}%
\pgfpathlineto{\pgfqpoint{2.737401in}{2.263446in}}%
\pgfpathlineto{\pgfqpoint{2.736311in}{2.268022in}}%
\pgfpathlineto{\pgfqpoint{2.733613in}{2.279328in}}%
\pgfpathlineto{\pgfqpoint{2.731952in}{2.285116in}}%
\pgfpathlineto{\pgfqpoint{2.731502in}{2.261003in}}%
\pgfpathlineto{\pgfqpoint{2.730830in}{2.263524in}}%
\pgfpathlineto{\pgfqpoint{2.728569in}{2.274005in}}%
\pgfpathlineto{\pgfqpoint{2.727064in}{2.278914in}}%
\pgfpathlineto{\pgfqpoint{2.725822in}{2.284088in}}%
\pgfpathlineto{\pgfqpoint{2.723142in}{2.295896in}}%
\pgfpathlineto{\pgfqpoint{2.722445in}{2.298815in}}%
\pgfpathlineto{\pgfqpoint{2.721473in}{2.276555in}}%
\pgfpathlineto{\pgfqpoint{2.721230in}{2.278125in}}%
\pgfpathlineto{\pgfqpoint{2.720338in}{2.281396in}}%
\pgfpathlineto{\pgfqpoint{2.719878in}{2.284405in}}%
\pgfpathlineto{\pgfqpoint{2.719878in}{2.284405in}}%
\pgfpathlineto{\pgfqpoint{2.719878in}{2.284405in}}%
\pgfpathlineto{\pgfqpoint{2.719862in}{2.271817in}}%
\pgfpathlineto{\pgfqpoint{2.718763in}{2.276366in}}%
\pgfpathlineto{\pgfqpoint{2.715629in}{2.291571in}}%
\pgfpathlineto{\pgfqpoint{2.715455in}{2.267369in}}%
\pgfpathlineto{\pgfqpoint{2.714473in}{2.271987in}}%
\pgfpathlineto{\pgfqpoint{2.713449in}{2.276476in}}%
\pgfpathlineto{\pgfqpoint{2.713354in}{2.264528in}}%
\pgfpathlineto{\pgfqpoint{2.712290in}{2.269371in}}%
\pgfpathlineto{\pgfqpoint{2.707525in}{2.289123in}}%
\pgfpathlineto{\pgfqpoint{2.706941in}{2.291544in}}%
\pgfpathlineto{\pgfqpoint{2.705700in}{2.296386in}}%
\pgfpathlineto{\pgfqpoint{2.702463in}{2.310404in}}%
\pgfpathlineto{\pgfqpoint{2.701145in}{2.316266in}}%
\pgfpathlineto{\pgfqpoint{2.699702in}{2.320853in}}%
\pgfpathlineto{\pgfqpoint{2.699698in}{2.308808in}}%
\pgfpathlineto{\pgfqpoint{2.698606in}{2.314886in}}%
\pgfpathlineto{\pgfqpoint{2.694979in}{2.329828in}}%
\pgfpathlineto{\pgfqpoint{2.694968in}{2.317908in}}%
\pgfpathlineto{\pgfqpoint{2.693863in}{2.323948in}}%
\pgfpathlineto{\pgfqpoint{2.691778in}{2.332505in}}%
\pgfpathlineto{\pgfqpoint{2.690056in}{2.341440in}}%
\pgfpathlineto{\pgfqpoint{2.682630in}{2.371892in}}%
\pgfpathlineto{\pgfqpoint{2.681923in}{2.351493in}}%
\pgfpathlineto{\pgfqpoint{2.681487in}{2.353855in}}%
\pgfpathlineto{\pgfqpoint{2.681390in}{2.354352in}}%
\pgfpathlineto{\pgfqpoint{2.681379in}{2.342924in}}%
\pgfpathlineto{\pgfqpoint{2.680299in}{2.347496in}}%
\pgfpathlineto{\pgfqpoint{2.679151in}{2.352933in}}%
\pgfpathlineto{\pgfqpoint{2.677939in}{2.358617in}}%
\pgfpathlineto{\pgfqpoint{2.677020in}{2.361829in}}%
\pgfpathlineto{\pgfqpoint{2.676062in}{2.366648in}}%
\pgfpathlineto{\pgfqpoint{2.676052in}{2.355187in}}%
\pgfpathlineto{\pgfqpoint{2.674951in}{2.359241in}}%
\pgfpathlineto{\pgfqpoint{2.673859in}{2.363786in}}%
\pgfpathlineto{\pgfqpoint{2.673535in}{2.364769in}}%
\pgfpathlineto{\pgfqpoint{2.673521in}{2.353478in}}%
\pgfpathlineto{\pgfqpoint{2.672435in}{2.359706in}}%
\pgfpathlineto{\pgfqpoint{2.670107in}{2.368255in}}%
\pgfpathlineto{\pgfqpoint{2.670103in}{2.357099in}}%
\pgfpathlineto{\pgfqpoint{2.669005in}{2.361592in}}%
\pgfpathlineto{\pgfqpoint{2.666108in}{2.374464in}}%
\pgfpathlineto{\pgfqpoint{2.661760in}{2.390371in}}%
\pgfpathlineto{\pgfqpoint{2.661690in}{2.390857in}}%
\pgfpathlineto{\pgfqpoint{2.661690in}{2.390857in}}%
\pgfpathlineto{\pgfqpoint{2.661690in}{2.390857in}}%
\pgfpathlineto{\pgfqpoint{2.661674in}{2.379724in}}%
\pgfpathlineto{\pgfqpoint{2.660595in}{2.383708in}}%
\pgfpathlineto{\pgfqpoint{2.660522in}{2.384191in}}%
\pgfpathlineto{\pgfqpoint{2.660487in}{2.373244in}}%
\pgfpathlineto{\pgfqpoint{2.659389in}{2.377206in}}%
\pgfpathlineto{\pgfqpoint{2.655987in}{2.392335in}}%
\pgfpathlineto{\pgfqpoint{2.655943in}{2.381486in}}%
\pgfpathlineto{\pgfqpoint{2.654851in}{2.386143in}}%
\pgfpathlineto{\pgfqpoint{2.654252in}{2.389248in}}%
\pgfpathlineto{\pgfqpoint{2.654252in}{2.389248in}}%
\pgfpathlineto{\pgfqpoint{2.654252in}{2.389248in}}%
\pgfpathlineto{\pgfqpoint{2.653213in}{2.371366in}}%
\pgfpathlineto{\pgfqpoint{2.653122in}{2.372074in}}%
\pgfpathlineto{\pgfqpoint{2.652800in}{2.373374in}}%
\pgfpathlineto{\pgfqpoint{2.651711in}{2.327822in}}%
\pgfpathlineto{\pgfqpoint{2.651694in}{2.327937in}}%
\pgfpathlineto{\pgfqpoint{2.645373in}{2.352801in}}%
\pgfpathlineto{\pgfqpoint{2.644925in}{2.354413in}}%
\pgfpathlineto{\pgfqpoint{2.644797in}{2.355104in}}%
\pgfpathlineto{\pgfqpoint{2.644794in}{2.345580in}}%
\pgfpathlineto{\pgfqpoint{2.643686in}{2.348100in}}%
\pgfpathlineto{\pgfqpoint{2.640704in}{2.360127in}}%
\pgfpathlineto{\pgfqpoint{2.639058in}{2.369175in}}%
\pgfpathlineto{\pgfqpoint{2.639054in}{2.369290in}}%
\pgfpathlineto{\pgfqpoint{2.639054in}{2.369290in}}%
\pgfpathlineto{\pgfqpoint{2.639054in}{2.369290in}}%
\pgfpathlineto{\pgfqpoint{2.638334in}{2.351956in}}%
\pgfpathlineto{\pgfqpoint{2.637836in}{2.354678in}}%
\pgfpathlineto{\pgfqpoint{2.636311in}{2.360349in}}%
\pgfpathlineto{\pgfqpoint{2.635765in}{2.334126in}}%
\pgfpathlineto{\pgfqpoint{2.635179in}{2.335915in}}%
\pgfpathlineto{\pgfqpoint{2.634739in}{2.337369in}}%
\pgfpathlineto{\pgfqpoint{2.634730in}{2.328558in}}%
\pgfpathlineto{\pgfqpoint{2.633654in}{2.332676in}}%
\pgfpathlineto{\pgfqpoint{2.632497in}{2.336794in}}%
\pgfpathlineto{\pgfqpoint{2.632483in}{2.327970in}}%
\pgfpathlineto{\pgfqpoint{2.631388in}{2.332401in}}%
\pgfpathlineto{\pgfqpoint{2.631282in}{2.332845in}}%
\pgfpathlineto{\pgfqpoint{2.630651in}{2.317200in}}%
\pgfpathlineto{\pgfqpoint{2.630212in}{2.318956in}}%
\pgfpathlineto{\pgfqpoint{2.629595in}{2.320493in}}%
\pgfpathlineto{\pgfqpoint{2.628249in}{2.325763in}}%
\pgfpathlineto{\pgfqpoint{2.626752in}{2.331252in}}%
\pgfpathlineto{\pgfqpoint{2.625342in}{2.336741in}}%
\pgfpathlineto{\pgfqpoint{2.624449in}{2.339705in}}%
\pgfpathlineto{\pgfqpoint{2.624427in}{2.331145in}}%
\pgfpathlineto{\pgfqpoint{2.623356in}{2.336282in}}%
\pgfpathlineto{\pgfqpoint{2.617641in}{2.355518in}}%
\pgfpathlineto{\pgfqpoint{2.616893in}{2.323483in}}%
\pgfpathlineto{\pgfqpoint{2.616258in}{2.325953in}}%
\pgfpathlineto{\pgfqpoint{2.616022in}{2.326812in}}%
\pgfpathlineto{\pgfqpoint{2.615966in}{2.318730in}}%
\pgfpathlineto{\pgfqpoint{2.614934in}{2.322367in}}%
\pgfpathlineto{\pgfqpoint{2.614735in}{2.322794in}}%
\pgfpathlineto{\pgfqpoint{2.614716in}{2.314808in}}%
\pgfpathlineto{\pgfqpoint{2.613666in}{2.319068in}}%
\pgfpathlineto{\pgfqpoint{2.612576in}{2.323542in}}%
\pgfpathlineto{\pgfqpoint{2.612502in}{2.323861in}}%
\pgfpathlineto{\pgfqpoint{2.612502in}{2.323861in}}%
\pgfpathlineto{\pgfqpoint{2.612502in}{2.323861in}}%
\pgfpathlineto{\pgfqpoint{2.612244in}{2.308776in}}%
\pgfpathlineto{\pgfqpoint{2.611334in}{2.312050in}}%
\pgfpathlineto{\pgfqpoint{2.608822in}{2.320184in}}%
\pgfpathlineto{\pgfqpoint{2.608813in}{2.312435in}}%
\pgfpathlineto{\pgfqpoint{2.607707in}{2.316538in}}%
\pgfpathlineto{\pgfqpoint{2.606365in}{2.321903in}}%
\pgfpathlineto{\pgfqpoint{2.606362in}{2.314220in}}%
\pgfpathlineto{\pgfqpoint{2.605253in}{2.318725in}}%
\pgfpathlineto{\pgfqpoint{2.604559in}{2.322288in}}%
\pgfpathlineto{\pgfqpoint{2.601028in}{2.335280in}}%
\pgfpathlineto{\pgfqpoint{2.600965in}{2.327615in}}%
\pgfpathlineto{\pgfqpoint{2.599899in}{2.330328in}}%
\pgfpathlineto{\pgfqpoint{2.599441in}{2.332206in}}%
\pgfpathlineto{\pgfqpoint{2.599422in}{2.324624in}}%
\pgfpathlineto{\pgfqpoint{2.598393in}{2.328262in}}%
\pgfpathlineto{\pgfqpoint{2.598178in}{2.328470in}}%
\pgfpathlineto{\pgfqpoint{2.598167in}{2.320971in}}%
\pgfpathlineto{\pgfqpoint{2.597091in}{2.325112in}}%
\pgfpathlineto{\pgfqpoint{2.596013in}{2.329046in}}%
\pgfpathlineto{\pgfqpoint{2.595275in}{2.331945in}}%
\pgfpathlineto{\pgfqpoint{2.595238in}{2.324500in}}%
\pgfpathlineto{\pgfqpoint{2.594153in}{2.328110in}}%
\pgfpathlineto{\pgfqpoint{2.593710in}{2.330173in}}%
\pgfpathlineto{\pgfqpoint{2.593710in}{2.330173in}}%
\pgfpathlineto{\pgfqpoint{2.593710in}{2.330173in}}%
\pgfpathlineto{\pgfqpoint{2.593397in}{2.316751in}}%
\pgfpathlineto{\pgfqpoint{2.592625in}{2.319821in}}%
\pgfpathlineto{\pgfqpoint{2.591660in}{2.322687in}}%
\pgfpathlineto{\pgfqpoint{2.590354in}{2.326371in}}%
\pgfpathlineto{\pgfqpoint{2.589716in}{2.328316in}}%
\pgfpathlineto{\pgfqpoint{2.589713in}{2.321080in}}%
\pgfpathlineto{\pgfqpoint{2.588632in}{2.325260in}}%
\pgfpathlineto{\pgfqpoint{2.586634in}{2.331582in}}%
\pgfpathlineto{\pgfqpoint{2.586233in}{2.311299in}}%
\pgfpathlineto{\pgfqpoint{2.585516in}{2.313416in}}%
\pgfpathlineto{\pgfqpoint{2.583910in}{2.318961in}}%
\pgfpathlineto{\pgfqpoint{2.583868in}{2.312004in}}%
\pgfpathlineto{\pgfqpoint{2.582788in}{2.315420in}}%
\pgfpathlineto{\pgfqpoint{2.582325in}{2.317228in}}%
\pgfpathlineto{\pgfqpoint{2.578484in}{2.328780in}}%
\pgfpathlineto{\pgfqpoint{2.578265in}{2.315492in}}%
\pgfpathlineto{\pgfqpoint{2.577408in}{2.317686in}}%
\pgfpathlineto{\pgfqpoint{2.576941in}{2.319481in}}%
\pgfpathlineto{\pgfqpoint{2.576746in}{2.319980in}}%
\pgfpathlineto{\pgfqpoint{2.576684in}{2.313191in}}%
\pgfpathlineto{\pgfqpoint{2.575612in}{2.316272in}}%
\pgfpathlineto{\pgfqpoint{2.574161in}{2.320843in}}%
\pgfpathlineto{\pgfqpoint{2.573523in}{2.303323in}}%
\pgfpathlineto{\pgfqpoint{2.573013in}{2.305192in}}%
\pgfpathlineto{\pgfqpoint{2.571864in}{2.308928in}}%
\pgfpathlineto{\pgfqpoint{2.571855in}{2.302393in}}%
\pgfpathlineto{\pgfqpoint{2.570818in}{2.305333in}}%
\pgfpathlineto{\pgfqpoint{2.570279in}{2.306999in}}%
\pgfpathlineto{\pgfqpoint{2.569605in}{2.297329in}}%
\pgfpathlineto{\pgfqpoint{2.569144in}{2.298983in}}%
\pgfpathlineto{\pgfqpoint{2.566101in}{2.308131in}}%
\pgfpathlineto{\pgfqpoint{2.566069in}{2.301751in}}%
\pgfpathlineto{\pgfqpoint{2.564992in}{2.304660in}}%
\pgfpathlineto{\pgfqpoint{2.563166in}{2.310383in}}%
\pgfpathlineto{\pgfqpoint{2.561836in}{2.314166in}}%
\pgfpathlineto{\pgfqpoint{2.560008in}{2.320567in}}%
\pgfpathlineto{\pgfqpoint{2.559223in}{2.322895in}}%
\pgfpathlineto{\pgfqpoint{2.559183in}{2.323089in}}%
\pgfpathlineto{\pgfqpoint{2.559183in}{2.323089in}}%
\pgfpathlineto{\pgfqpoint{2.559183in}{2.323089in}}%
\pgfpathlineto{\pgfqpoint{2.558284in}{2.313093in}}%
\pgfpathlineto{\pgfqpoint{2.558069in}{2.313478in}}%
\pgfpathlineto{\pgfqpoint{2.557403in}{2.315502in}}%
\pgfpathlineto{\pgfqpoint{2.555971in}{2.320415in}}%
\pgfpathlineto{\pgfqpoint{2.555937in}{2.314139in}}%
\pgfpathlineto{\pgfqpoint{2.554905in}{2.317692in}}%
\pgfpathlineto{\pgfqpoint{2.554367in}{2.319612in}}%
\pgfpathlineto{\pgfqpoint{2.554356in}{2.313385in}}%
\pgfpathlineto{\pgfqpoint{2.553257in}{2.317884in}}%
\pgfpathlineto{\pgfqpoint{2.552811in}{2.319415in}}%
\pgfpathlineto{\pgfqpoint{2.552811in}{2.319415in}}%
\pgfpathlineto{\pgfqpoint{2.552811in}{2.319415in}}%
\pgfpathlineto{\pgfqpoint{2.552280in}{2.308072in}}%
\pgfpathlineto{\pgfqpoint{2.551639in}{2.310069in}}%
\pgfpathlineto{\pgfqpoint{2.550093in}{2.315014in}}%
\pgfpathlineto{\pgfqpoint{2.550086in}{2.308938in}}%
\pgfpathlineto{\pgfqpoint{2.549001in}{2.313204in}}%
\pgfpathlineto{\pgfqpoint{2.548775in}{2.314436in}}%
\pgfpathlineto{\pgfqpoint{2.548512in}{2.297942in}}%
\pgfpathlineto{\pgfqpoint{2.547683in}{2.300571in}}%
\pgfpathlineto{\pgfqpoint{2.547252in}{2.302731in}}%
\pgfpathlineto{\pgfqpoint{2.544257in}{2.313716in}}%
\pgfpathlineto{\pgfqpoint{2.543692in}{2.304591in}}%
\pgfpathlineto{\pgfqpoint{2.543090in}{2.306924in}}%
\pgfpathlineto{\pgfqpoint{2.542096in}{2.310283in}}%
\pgfpathlineto{\pgfqpoint{2.541834in}{2.299564in}}%
\pgfpathlineto{\pgfqpoint{2.540994in}{2.301697in}}%
\pgfpathlineto{\pgfqpoint{2.539600in}{2.306334in}}%
\pgfpathlineto{\pgfqpoint{2.539588in}{2.300620in}}%
\pgfpathlineto{\pgfqpoint{2.538513in}{2.303764in}}%
\pgfpathlineto{\pgfqpoint{2.537417in}{2.307185in}}%
\pgfpathlineto{\pgfqpoint{2.536231in}{2.310791in}}%
\pgfpathlineto{\pgfqpoint{2.536177in}{2.305102in}}%
\pgfpathlineto{\pgfqpoint{2.535203in}{2.308052in}}%
\pgfpathlineto{\pgfqpoint{2.534565in}{2.309804in}}%
\pgfpathlineto{\pgfqpoint{2.532474in}{2.316349in}}%
\pgfpathlineto{\pgfqpoint{2.532220in}{2.305529in}}%
\pgfpathlineto{\pgfqpoint{2.531325in}{2.309103in}}%
\pgfpathlineto{\pgfqpoint{2.530703in}{2.311302in}}%
\pgfpathlineto{\pgfqpoint{2.529482in}{2.315335in}}%
\pgfpathlineto{\pgfqpoint{2.527730in}{2.320833in}}%
\pgfpathlineto{\pgfqpoint{2.526542in}{2.307633in}}%
\pgfpathlineto{\pgfqpoint{2.526496in}{2.307814in}}%
\pgfpathlineto{\pgfqpoint{2.526194in}{2.308904in}}%
\pgfpathlineto{\pgfqpoint{2.525160in}{2.284976in}}%
\pgfpathlineto{\pgfqpoint{2.524833in}{2.285782in}}%
\pgfpathlineto{\pgfqpoint{2.523899in}{2.289095in}}%
\pgfpathlineto{\pgfqpoint{2.522439in}{2.293304in}}%
\pgfpathlineto{\pgfqpoint{2.521872in}{2.295633in}}%
\pgfpathlineto{\pgfqpoint{2.521793in}{2.285270in}}%
\pgfpathlineto{\pgfqpoint{2.520757in}{2.289010in}}%
\pgfpathlineto{\pgfqpoint{2.519269in}{2.292572in}}%
\pgfpathlineto{\pgfqpoint{2.518419in}{2.294888in}}%
\pgfpathlineto{\pgfqpoint{2.518419in}{2.294888in}}%
\pgfpathlineto{\pgfqpoint{2.518419in}{2.294888in}}%
\pgfpathlineto{\pgfqpoint{2.518345in}{2.289706in}}%
\pgfpathlineto{\pgfqpoint{2.517299in}{2.293169in}}%
\pgfpathlineto{\pgfqpoint{2.516789in}{2.294857in}}%
\pgfpathlineto{\pgfqpoint{2.515816in}{2.282098in}}%
\pgfpathlineto{\pgfqpoint{2.515750in}{2.282451in}}%
\pgfpathlineto{\pgfqpoint{2.515647in}{2.277516in}}%
\pgfpathlineto{\pgfqpoint{2.514617in}{2.280854in}}%
\pgfpathlineto{\pgfqpoint{2.514088in}{2.282172in}}%
\pgfpathlineto{\pgfqpoint{2.513301in}{2.284632in}}%
\pgfpathlineto{\pgfqpoint{2.513301in}{2.284632in}}%
\pgfpathlineto{\pgfqpoint{2.513301in}{2.284632in}}%
\pgfpathlineto{\pgfqpoint{2.513269in}{2.279721in}}%
\pgfpathlineto{\pgfqpoint{2.512247in}{2.282700in}}%
\pgfpathlineto{\pgfqpoint{2.511391in}{2.285328in}}%
\pgfpathlineto{\pgfqpoint{2.509171in}{2.292337in}}%
\pgfpathlineto{\pgfqpoint{2.508885in}{2.293038in}}%
\pgfpathlineto{\pgfqpoint{2.507730in}{2.286151in}}%
\pgfpathlineto{\pgfqpoint{2.507713in}{2.286238in}}%
\pgfpathlineto{\pgfqpoint{2.506903in}{2.288416in}}%
\pgfpathlineto{\pgfqpoint{2.505978in}{2.282001in}}%
\pgfpathlineto{\pgfqpoint{2.505810in}{2.282781in}}%
\pgfpathlineto{\pgfqpoint{2.503517in}{2.289977in}}%
\pgfpathlineto{\pgfqpoint{2.502643in}{2.283478in}}%
\pgfpathlineto{\pgfqpoint{2.502440in}{2.284340in}}%
\pgfpathlineto{\pgfqpoint{2.502344in}{2.284512in}}%
\pgfpathlineto{\pgfqpoint{2.502325in}{2.279717in}}%
\pgfpathlineto{\pgfqpoint{2.501311in}{2.283502in}}%
\pgfpathlineto{\pgfqpoint{2.500504in}{2.285395in}}%
\pgfpathlineto{\pgfqpoint{2.500500in}{2.280625in}}%
\pgfpathlineto{\pgfqpoint{2.499392in}{2.284572in}}%
\pgfpathlineto{\pgfqpoint{2.499085in}{2.285515in}}%
\pgfpathlineto{\pgfqpoint{2.499040in}{2.280773in}}%
\pgfpathlineto{\pgfqpoint{2.497964in}{2.283682in}}%
\pgfpathlineto{\pgfqpoint{2.496992in}{2.285907in}}%
\pgfpathlineto{\pgfqpoint{2.496989in}{2.281190in}}%
\pgfpathlineto{\pgfqpoint{2.495885in}{2.284178in}}%
\pgfpathlineto{\pgfqpoint{2.494909in}{2.286569in}}%
\pgfpathlineto{\pgfqpoint{2.493677in}{2.290069in}}%
\pgfpathlineto{\pgfqpoint{2.493661in}{2.285368in}}%
\pgfpathlineto{\pgfqpoint{2.492583in}{2.288348in}}%
\pgfpathlineto{\pgfqpoint{2.492510in}{2.288689in}}%
\pgfpathlineto{\pgfqpoint{2.492447in}{2.279808in}}%
\pgfpathlineto{\pgfqpoint{2.491435in}{2.282689in}}%
\pgfpathlineto{\pgfqpoint{2.490153in}{2.286247in}}%
\pgfpathlineto{\pgfqpoint{2.488933in}{2.289975in}}%
\pgfpathlineto{\pgfqpoint{2.487940in}{2.274725in}}%
\pgfpathlineto{\pgfqpoint{2.487827in}{2.274977in}}%
\pgfpathlineto{\pgfqpoint{2.486652in}{2.279004in}}%
\pgfpathlineto{\pgfqpoint{2.486226in}{2.279675in}}%
\pgfpathlineto{\pgfqpoint{2.486219in}{2.275178in}}%
\pgfpathlineto{\pgfqpoint{2.485177in}{2.278275in}}%
\pgfpathlineto{\pgfqpoint{2.485105in}{2.278359in}}%
\pgfpathlineto{\pgfqpoint{2.485090in}{2.273889in}}%
\pgfpathlineto{\pgfqpoint{2.483997in}{2.277062in}}%
\pgfpathlineto{\pgfqpoint{2.482179in}{2.282990in}}%
\pgfpathlineto{\pgfqpoint{2.481299in}{2.276769in}}%
\pgfpathlineto{\pgfqpoint{2.481065in}{2.277850in}}%
\pgfpathlineto{\pgfqpoint{2.480901in}{2.278348in}}%
\pgfpathlineto{\pgfqpoint{2.480802in}{2.273949in}}%
\pgfpathlineto{\pgfqpoint{2.479759in}{2.276851in}}%
\pgfpathlineto{\pgfqpoint{2.478194in}{2.281245in}}%
\pgfpathlineto{\pgfqpoint{2.477047in}{2.284147in}}%
\pgfpathlineto{\pgfqpoint{2.476113in}{2.286799in}}%
\pgfpathlineto{\pgfqpoint{2.476098in}{2.282404in}}%
\pgfpathlineto{\pgfqpoint{2.474999in}{2.285547in}}%
\pgfpathlineto{\pgfqpoint{2.472040in}{2.293983in}}%
\pgfpathlineto{\pgfqpoint{2.471041in}{2.282721in}}%
\pgfpathlineto{\pgfqpoint{2.470968in}{2.282803in}}%
\pgfpathlineto{\pgfqpoint{2.470269in}{2.285185in}}%
\pgfpathlineto{\pgfqpoint{2.470244in}{2.285349in}}%
\pgfpathlineto{\pgfqpoint{2.470244in}{2.285349in}}%
\pgfpathlineto{\pgfqpoint{2.470244in}{2.285349in}}%
\pgfpathlineto{\pgfqpoint{2.469655in}{2.277840in}}%
\pgfpathlineto{\pgfqpoint{2.469047in}{2.280211in}}%
\pgfpathlineto{\pgfqpoint{2.468607in}{2.280947in}}%
\pgfpathlineto{\pgfqpoint{2.467821in}{2.270570in}}%
\pgfpathlineto{\pgfqpoint{2.467382in}{2.271463in}}%
\pgfpathlineto{\pgfqpoint{2.466169in}{2.275442in}}%
\pgfpathlineto{\pgfqpoint{2.466142in}{2.267291in}}%
\pgfpathlineto{\pgfqpoint{2.465083in}{2.270767in}}%
\pgfpathlineto{\pgfqpoint{2.464798in}{2.271333in}}%
\pgfpathlineto{\pgfqpoint{2.464785in}{2.267212in}}%
\pgfpathlineto{\pgfqpoint{2.463689in}{2.269954in}}%
\pgfpathlineto{\pgfqpoint{2.462891in}{2.272293in}}%
\pgfpathlineto{\pgfqpoint{2.462891in}{2.272293in}}%
\pgfpathlineto{\pgfqpoint{2.462891in}{2.272293in}}%
\pgfpathlineto{\pgfqpoint{2.462889in}{2.268190in}}%
\pgfpathlineto{\pgfqpoint{2.461788in}{2.271570in}}%
\pgfpathlineto{\pgfqpoint{2.459178in}{2.277606in}}%
\pgfpathlineto{\pgfqpoint{2.458161in}{2.280423in}}%
\pgfpathlineto{\pgfqpoint{2.458141in}{2.276323in}}%
\pgfpathlineto{\pgfqpoint{2.457060in}{2.279936in}}%
\pgfpathlineto{\pgfqpoint{2.456611in}{2.281060in}}%
\pgfpathlineto{\pgfqpoint{2.456554in}{2.276978in}}%
\pgfpathlineto{\pgfqpoint{2.455525in}{2.280504in}}%
\pgfpathlineto{\pgfqpoint{2.454821in}{2.282667in}}%
\pgfpathlineto{\pgfqpoint{2.454320in}{2.275522in}}%
\pgfpathlineto{\pgfqpoint{2.453688in}{2.277038in}}%
\pgfpathlineto{\pgfqpoint{2.453228in}{2.278075in}}%
\pgfpathlineto{\pgfqpoint{2.453214in}{2.274058in}}%
\pgfpathlineto{\pgfqpoint{2.452146in}{2.277243in}}%
\pgfpathlineto{\pgfqpoint{2.451568in}{2.278755in}}%
\pgfpathlineto{\pgfqpoint{2.451540in}{2.274756in}}%
\pgfpathlineto{\pgfqpoint{2.450505in}{2.277457in}}%
\pgfpathlineto{\pgfqpoint{2.449223in}{2.281032in}}%
\pgfpathlineto{\pgfqpoint{2.449159in}{2.281349in}}%
\pgfpathlineto{\pgfqpoint{2.449159in}{2.281349in}}%
\pgfpathlineto{\pgfqpoint{2.449159in}{2.281349in}}%
\pgfpathlineto{\pgfqpoint{2.448962in}{2.273801in}}%
\pgfpathlineto{\pgfqpoint{2.448076in}{2.275858in}}%
\pgfpathlineto{\pgfqpoint{2.447112in}{2.278152in}}%
\pgfpathlineto{\pgfqpoint{2.446709in}{2.271084in}}%
\pgfpathlineto{\pgfqpoint{2.446021in}{2.272266in}}%
\pgfpathlineto{\pgfqpoint{2.444238in}{2.276440in}}%
\pgfpathlineto{\pgfqpoint{2.444214in}{2.272539in}}%
\pgfpathlineto{\pgfqpoint{2.443129in}{2.275447in}}%
\pgfpathlineto{\pgfqpoint{2.442800in}{2.276312in}}%
\pgfpathlineto{\pgfqpoint{2.442800in}{2.276312in}}%
\pgfpathlineto{\pgfqpoint{2.442800in}{2.276312in}}%
\pgfpathlineto{\pgfqpoint{2.442791in}{2.272430in}}%
\pgfpathlineto{\pgfqpoint{2.441763in}{2.274312in}}%
\pgfpathlineto{\pgfqpoint{2.441268in}{2.275724in}}%
\pgfpathlineto{\pgfqpoint{2.441207in}{2.276116in}}%
\pgfpathlineto{\pgfqpoint{2.441207in}{2.276116in}}%
\pgfpathlineto{\pgfqpoint{2.441207in}{2.276116in}}%
\pgfpathlineto{\pgfqpoint{2.440229in}{2.271226in}}%
\pgfpathlineto{\pgfqpoint{2.440019in}{2.271617in}}%
\pgfpathlineto{\pgfqpoint{2.438625in}{2.275210in}}%
\pgfpathlineto{\pgfqpoint{2.438590in}{2.271383in}}%
\pgfpathlineto{\pgfqpoint{2.437542in}{2.274267in}}%
\pgfpathlineto{\pgfqpoint{2.437040in}{2.275748in}}%
\pgfpathlineto{\pgfqpoint{2.436056in}{2.279022in}}%
\pgfpathlineto{\pgfqpoint{2.434947in}{2.270357in}}%
\pgfpathlineto{\pgfqpoint{2.434945in}{2.270435in}}%
\pgfpathlineto{\pgfqpoint{2.434428in}{2.271906in}}%
\pgfpathlineto{\pgfqpoint{2.434413in}{2.268155in}}%
\pgfpathlineto{\pgfqpoint{2.433339in}{2.271325in}}%
\pgfpathlineto{\pgfqpoint{2.429007in}{2.278857in}}%
\pgfpathlineto{\pgfqpoint{2.429002in}{2.275125in}}%
\pgfpathlineto{\pgfqpoint{2.427913in}{2.278359in}}%
\pgfpathlineto{\pgfqpoint{2.427609in}{2.279591in}}%
\pgfpathlineto{\pgfqpoint{2.427609in}{2.279591in}}%
\pgfpathlineto{\pgfqpoint{2.427609in}{2.279591in}}%
\pgfpathlineto{\pgfqpoint{2.426126in}{2.260153in}}%
\pgfpathlineto{\pgfqpoint{2.425923in}{2.260458in}}%
\pgfpathlineto{\pgfqpoint{2.423757in}{2.266470in}}%
\pgfpathlineto{\pgfqpoint{2.423374in}{2.263710in}}%
\pgfpathlineto{\pgfqpoint{2.422446in}{2.258928in}}%
\pgfpathlineto{\pgfqpoint{2.422238in}{2.259609in}}%
\pgfpathlineto{\pgfqpoint{2.421674in}{2.260896in}}%
\pgfpathlineto{\pgfqpoint{2.421674in}{2.260896in}}%
\pgfpathlineto{\pgfqpoint{2.421674in}{2.260896in}}%
\pgfpathlineto{\pgfqpoint{2.421664in}{2.257432in}}%
\pgfpathlineto{\pgfqpoint{2.420568in}{2.260302in}}%
\pgfpathlineto{\pgfqpoint{2.419008in}{2.264757in}}%
\pgfpathlineto{\pgfqpoint{2.414162in}{2.276159in}}%
\pgfpathlineto{\pgfqpoint{2.413727in}{2.266617in}}%
\pgfpathlineto{\pgfqpoint{2.412977in}{2.268869in}}%
\pgfpathlineto{\pgfqpoint{2.412589in}{2.269620in}}%
\pgfpathlineto{\pgfqpoint{2.412589in}{2.269620in}}%
\pgfpathlineto{\pgfqpoint{2.412589in}{2.269620in}}%
\pgfpathlineto{\pgfqpoint{2.412556in}{2.266124in}}%
\pgfpathlineto{\pgfqpoint{2.411532in}{2.268447in}}%
\pgfpathlineto{\pgfqpoint{2.410533in}{2.270320in}}%
\pgfpathlineto{\pgfqpoint{2.410118in}{2.271220in}}%
\pgfpathlineto{\pgfqpoint{2.410118in}{2.271220in}}%
\pgfpathlineto{\pgfqpoint{2.410118in}{2.271220in}}%
\pgfpathlineto{\pgfqpoint{2.409313in}{2.265765in}}%
\pgfpathlineto{\pgfqpoint{2.408986in}{2.266362in}}%
\pgfpathlineto{\pgfqpoint{2.408150in}{2.268453in}}%
\pgfpathlineto{\pgfqpoint{2.408105in}{2.265002in}}%
\pgfpathlineto{\pgfqpoint{2.407023in}{2.266790in}}%
\pgfpathlineto{\pgfqpoint{2.405324in}{2.271709in}}%
\pgfpathlineto{\pgfqpoint{2.405147in}{2.265438in}}%
\pgfpathlineto{\pgfqpoint{2.404214in}{2.267665in}}%
\pgfpathlineto{\pgfqpoint{2.401658in}{2.273976in}}%
\pgfpathlineto{\pgfqpoint{2.400749in}{2.273151in}}%
\pgfpathlineto{\pgfqpoint{2.400715in}{2.269748in}}%
\pgfpathlineto{\pgfqpoint{2.399695in}{2.272485in}}%
\pgfpathlineto{\pgfqpoint{2.399488in}{2.272928in}}%
\pgfpathlineto{\pgfqpoint{2.398721in}{2.264365in}}%
\pgfpathlineto{\pgfqpoint{2.398307in}{2.265836in}}%
\pgfpathlineto{\pgfqpoint{2.397758in}{2.267234in}}%
\pgfpathlineto{\pgfqpoint{2.397735in}{2.263896in}}%
\pgfpathlineto{\pgfqpoint{2.396838in}{2.267200in}}%
\pgfpathlineto{\pgfqpoint{2.396838in}{2.267200in}}%
\pgfpathlineto{\pgfqpoint{2.396161in}{2.268742in}}%
\pgfpathlineto{\pgfqpoint{2.395456in}{2.260569in}}%
\pgfpathlineto{\pgfqpoint{2.395015in}{2.261445in}}%
\pgfpathlineto{\pgfqpoint{2.393696in}{2.265097in}}%
\pgfpathlineto{\pgfqpoint{2.393480in}{2.265828in}}%
\pgfpathlineto{\pgfqpoint{2.393480in}{2.265828in}}%
\pgfpathlineto{\pgfqpoint{2.393480in}{2.265828in}}%
\pgfpathlineto{\pgfqpoint{2.392716in}{2.260808in}}%
\pgfpathlineto{\pgfqpoint{2.392374in}{2.261754in}}%
\pgfpathlineto{\pgfqpoint{2.391923in}{2.263137in}}%
\pgfpathlineto{\pgfqpoint{2.390956in}{2.259259in}}%
\pgfpathlineto{\pgfqpoint{2.390917in}{2.259332in}}%
\pgfpathlineto{\pgfqpoint{2.389992in}{2.261072in}}%
\pgfpathlineto{\pgfqpoint{2.389259in}{2.263103in}}%
\pgfpathlineto{\pgfqpoint{2.389259in}{2.263103in}}%
\pgfpathlineto{\pgfqpoint{2.389259in}{2.263103in}}%
\pgfpathlineto{\pgfqpoint{2.388527in}{2.255132in}}%
\pgfpathlineto{\pgfqpoint{2.388227in}{2.256070in}}%
\pgfpathlineto{\pgfqpoint{2.387542in}{2.257513in}}%
\pgfpathlineto{\pgfqpoint{2.383744in}{2.266747in}}%
\pgfpathlineto{\pgfqpoint{2.383165in}{2.265062in}}%
\pgfpathlineto{\pgfqpoint{2.382301in}{2.260509in}}%
\pgfpathlineto{\pgfqpoint{2.382062in}{2.261370in}}%
\pgfpathlineto{\pgfqpoint{2.381755in}{2.262088in}}%
\pgfpathlineto{\pgfqpoint{2.381288in}{2.253750in}}%
\pgfpathlineto{\pgfqpoint{2.380646in}{2.255535in}}%
\pgfpathlineto{\pgfqpoint{2.380580in}{2.255678in}}%
\pgfpathlineto{\pgfqpoint{2.380572in}{2.252642in}}%
\pgfpathlineto{\pgfqpoint{2.379475in}{2.254852in}}%
\pgfpathlineto{\pgfqpoint{2.378358in}{2.251095in}}%
\pgfpathlineto{\pgfqpoint{2.377532in}{2.252871in}}%
\pgfpathlineto{\pgfqpoint{2.376247in}{2.256423in}}%
\pgfpathlineto{\pgfqpoint{2.376124in}{2.256636in}}%
\pgfpathlineto{\pgfqpoint{2.375249in}{2.252698in}}%
\pgfpathlineto{\pgfqpoint{2.374935in}{2.253902in}}%
\pgfpathlineto{\pgfqpoint{2.374406in}{2.255389in}}%
\pgfpathlineto{\pgfqpoint{2.373443in}{2.251704in}}%
\pgfpathlineto{\pgfqpoint{2.373316in}{2.251845in}}%
\pgfpathlineto{\pgfqpoint{2.372105in}{2.254739in}}%
\pgfpathlineto{\pgfqpoint{2.371955in}{2.255092in}}%
\pgfpathlineto{\pgfqpoint{2.371955in}{2.255092in}}%
\pgfpathlineto{\pgfqpoint{2.371955in}{2.255092in}}%
\pgfpathlineto{\pgfqpoint{2.371134in}{2.251725in}}%
\pgfpathlineto{\pgfqpoint{2.370827in}{2.252851in}}%
\pgfpathlineto{\pgfqpoint{2.370333in}{2.254257in}}%
\pgfpathlineto{\pgfqpoint{2.369442in}{2.244429in}}%
\pgfpathlineto{\pgfqpoint{2.369209in}{2.245127in}}%
\pgfpathlineto{\pgfqpoint{2.368014in}{2.247434in}}%
\pgfpathlineto{\pgfqpoint{2.367952in}{2.244483in}}%
\pgfpathlineto{\pgfqpoint{2.366915in}{2.246855in}}%
\pgfpathlineto{\pgfqpoint{2.365747in}{2.249646in}}%
\pgfpathlineto{\pgfqpoint{2.365322in}{2.250483in}}%
\pgfpathlineto{\pgfqpoint{2.365005in}{2.245372in}}%
\pgfpathlineto{\pgfqpoint{2.364213in}{2.247528in}}%
\pgfpathlineto{\pgfqpoint{2.363693in}{2.248850in}}%
\pgfpathlineto{\pgfqpoint{2.362967in}{2.244961in}}%
\pgfpathlineto{\pgfqpoint{2.362595in}{2.245585in}}%
\pgfpathlineto{\pgfqpoint{2.361172in}{2.248705in}}%
\pgfpathlineto{\pgfqpoint{2.360554in}{2.247118in}}%
\pgfpathlineto{\pgfqpoint{2.360318in}{2.241906in}}%
\pgfpathlineto{\pgfqpoint{2.359474in}{2.244390in}}%
\pgfpathlineto{\pgfqpoint{2.359089in}{2.245701in}}%
\pgfpathlineto{\pgfqpoint{2.358779in}{2.246253in}}%
\pgfpathlineto{\pgfqpoint{2.358779in}{2.246253in}}%
\pgfpathlineto{\pgfqpoint{2.358779in}{2.246253in}}%
\pgfpathlineto{\pgfqpoint{2.358727in}{2.243385in}}%
\pgfpathlineto{\pgfqpoint{2.357751in}{2.245865in}}%
\pgfpathlineto{\pgfqpoint{2.356887in}{2.247588in}}%
\pgfpathlineto{\pgfqpoint{2.355705in}{2.250275in}}%
\pgfpathlineto{\pgfqpoint{2.354071in}{2.253582in}}%
\pgfpathlineto{\pgfqpoint{2.353732in}{2.254477in}}%
\pgfpathlineto{\pgfqpoint{2.353732in}{2.254477in}}%
\pgfpathlineto{\pgfqpoint{2.353732in}{2.254477in}}%
\pgfpathlineto{\pgfqpoint{2.352797in}{2.242655in}}%
\pgfpathlineto{\pgfqpoint{2.352611in}{2.243202in}}%
\pgfpathlineto{\pgfqpoint{2.351718in}{2.244433in}}%
\pgfpathlineto{\pgfqpoint{2.351587in}{2.244775in}}%
\pgfpathlineto{\pgfqpoint{2.351587in}{2.244775in}}%
\pgfpathlineto{\pgfqpoint{2.351587in}{2.244775in}}%
\pgfpathlineto{\pgfqpoint{2.351568in}{2.241967in}}%
\pgfpathlineto{\pgfqpoint{2.350475in}{2.244219in}}%
\pgfpathlineto{\pgfqpoint{2.349932in}{2.245652in}}%
\pgfpathlineto{\pgfqpoint{2.349922in}{2.242853in}}%
\pgfpathlineto{\pgfqpoint{2.348825in}{2.244625in}}%
\pgfpathlineto{\pgfqpoint{2.348025in}{2.246465in}}%
\pgfpathlineto{\pgfqpoint{2.348025in}{2.246465in}}%
\pgfpathlineto{\pgfqpoint{2.348025in}{2.246465in}}%
\pgfpathlineto{\pgfqpoint{2.347743in}{2.241234in}}%
\pgfpathlineto{\pgfqpoint{2.346909in}{2.243069in}}%
\pgfpathlineto{\pgfqpoint{2.345234in}{2.246262in}}%
\pgfpathlineto{\pgfqpoint{2.344972in}{2.246602in}}%
\pgfpathlineto{\pgfqpoint{2.344728in}{2.241814in}}%
\pgfpathlineto{\pgfqpoint{2.343860in}{2.244117in}}%
\pgfpathlineto{\pgfqpoint{2.343254in}{2.245471in}}%
\pgfpathlineto{\pgfqpoint{2.343249in}{2.242718in}}%
\pgfpathlineto{\pgfqpoint{2.342263in}{2.245423in}}%
\pgfpathlineto{\pgfqpoint{2.342263in}{2.245423in}}%
\pgfpathlineto{\pgfqpoint{2.341291in}{2.247791in}}%
\pgfpathlineto{\pgfqpoint{2.341095in}{2.245380in}}%
\pgfpathlineto{\pgfqpoint{2.338826in}{2.247770in}}%
\pgfpathlineto{\pgfqpoint{2.338810in}{2.245040in}}%
\pgfpathlineto{\pgfqpoint{2.337751in}{2.246589in}}%
\pgfpathlineto{\pgfqpoint{2.336887in}{2.248272in}}%
\pgfpathlineto{\pgfqpoint{2.336024in}{2.250023in}}%
\pgfpathlineto{\pgfqpoint{2.335554in}{2.245795in}}%
\pgfpathlineto{\pgfqpoint{2.334919in}{2.247205in}}%
\pgfpathlineto{\pgfqpoint{2.334606in}{2.247675in}}%
\pgfpathlineto{\pgfqpoint{2.334576in}{2.244972in}}%
\pgfpathlineto{\pgfqpoint{2.333603in}{2.247184in}}%
\pgfpathlineto{\pgfqpoint{2.332854in}{2.248122in}}%
\pgfpathlineto{\pgfqpoint{2.332815in}{2.245427in}}%
\pgfpathlineto{\pgfqpoint{2.331744in}{2.247770in}}%
\pgfpathlineto{\pgfqpoint{2.331240in}{2.248707in}}%
\pgfpathlineto{\pgfqpoint{2.331177in}{2.243411in}}%
\pgfpathlineto{\pgfqpoint{2.330144in}{2.246080in}}%
\pgfpathlineto{\pgfqpoint{2.329301in}{2.248283in}}%
\pgfpathlineto{\pgfqpoint{2.328791in}{2.249551in}}%
\pgfpathlineto{\pgfqpoint{2.328791in}{2.249551in}}%
\pgfpathlineto{\pgfqpoint{2.328791in}{2.249551in}}%
\pgfpathlineto{\pgfqpoint{2.328776in}{2.246879in}}%
\pgfpathlineto{\pgfqpoint{2.327695in}{2.248945in}}%
\pgfpathlineto{\pgfqpoint{2.325758in}{2.253410in}}%
\pgfpathlineto{\pgfqpoint{2.324585in}{2.256076in}}%
\pgfpathlineto{\pgfqpoint{2.324509in}{2.256143in}}%
\pgfpathlineto{\pgfqpoint{2.324492in}{2.253536in}}%
\pgfpathlineto{\pgfqpoint{2.323424in}{2.255599in}}%
\pgfpathlineto{\pgfqpoint{2.323091in}{2.256065in}}%
\pgfpathlineto{\pgfqpoint{2.322765in}{2.251280in}}%
\pgfpathlineto{\pgfqpoint{2.322007in}{2.252275in}}%
\pgfpathlineto{\pgfqpoint{2.320227in}{2.255526in}}%
\pgfpathlineto{\pgfqpoint{2.319153in}{2.252891in}}%
\pgfpathlineto{\pgfqpoint{2.319088in}{2.253024in}}%
\pgfpathlineto{\pgfqpoint{2.318591in}{2.253950in}}%
\pgfpathlineto{\pgfqpoint{2.318547in}{2.251322in}}%
\pgfpathlineto{\pgfqpoint{2.317484in}{2.253568in}}%
\pgfpathlineto{\pgfqpoint{2.316007in}{2.257004in}}%
\pgfpathlineto{\pgfqpoint{2.315989in}{2.254380in}}%
\pgfpathlineto{\pgfqpoint{2.314893in}{2.256359in}}%
\pgfpathlineto{\pgfqpoint{2.314531in}{2.256953in}}%
\pgfpathlineto{\pgfqpoint{2.313881in}{2.249929in}}%
\pgfpathlineto{\pgfqpoint{2.313219in}{2.251505in}}%
\pgfpathlineto{\pgfqpoint{2.312594in}{2.252885in}}%
\pgfpathlineto{\pgfqpoint{2.312594in}{2.252885in}}%
\pgfpathlineto{\pgfqpoint{2.312594in}{2.252885in}}%
\pgfpathlineto{\pgfqpoint{2.312498in}{2.247724in}}%
\pgfpathlineto{\pgfqpoint{2.311475in}{2.250083in}}%
\pgfpathlineto{\pgfqpoint{2.310535in}{2.252376in}}%
\pgfpathlineto{\pgfqpoint{2.310102in}{2.250853in}}%
\pgfpathlineto{\pgfqpoint{2.309115in}{2.245032in}}%
\pgfpathlineto{\pgfqpoint{2.308838in}{2.245553in}}%
\pgfpathlineto{\pgfqpoint{2.308219in}{2.246856in}}%
\pgfpathlineto{\pgfqpoint{2.307719in}{2.242650in}}%
\pgfpathlineto{\pgfqpoint{2.307155in}{2.244209in}}%
\pgfpathlineto{\pgfqpoint{2.306586in}{2.244988in}}%
\pgfpathlineto{\pgfqpoint{2.305837in}{2.241074in}}%
\pgfpathlineto{\pgfqpoint{2.305457in}{2.241981in}}%
\pgfpathlineto{\pgfqpoint{2.302779in}{2.246345in}}%
\pgfpathlineto{\pgfqpoint{2.301807in}{2.243822in}}%
\pgfpathlineto{\pgfqpoint{2.301714in}{2.244016in}}%
\pgfpathlineto{\pgfqpoint{2.301535in}{2.244339in}}%
\pgfpathlineto{\pgfqpoint{2.301052in}{2.239913in}}%
\pgfpathlineto{\pgfqpoint{2.300395in}{2.241586in}}%
\pgfpathlineto{\pgfqpoint{2.299213in}{2.243839in}}%
\pgfpathlineto{\pgfqpoint{2.298653in}{2.237515in}}%
\pgfpathlineto{\pgfqpoint{2.298072in}{2.239117in}}%
\pgfpathlineto{\pgfqpoint{2.298041in}{2.239246in}}%
\pgfpathlineto{\pgfqpoint{2.298041in}{2.239246in}}%
\pgfpathlineto{\pgfqpoint{2.298041in}{2.239246in}}%
\pgfpathlineto{\pgfqpoint{2.297237in}{2.230916in}}%
\pgfpathlineto{\pgfqpoint{2.296944in}{2.231489in}}%
\pgfpathlineto{\pgfqpoint{2.294168in}{2.237673in}}%
\pgfpathlineto{\pgfqpoint{2.294118in}{2.235270in}}%
\pgfpathlineto{\pgfqpoint{2.293090in}{2.237690in}}%
\pgfpathlineto{\pgfqpoint{2.293090in}{2.237690in}}%
\pgfpathlineto{\pgfqpoint{2.291737in}{2.239663in}}%
\pgfpathlineto{\pgfqpoint{2.290806in}{2.241510in}}%
\pgfpathlineto{\pgfqpoint{2.290267in}{2.242528in}}%
\pgfpathlineto{\pgfqpoint{2.290267in}{2.242528in}}%
\pgfpathlineto{\pgfqpoint{2.290267in}{2.242528in}}%
\pgfpathlineto{\pgfqpoint{2.290254in}{2.240126in}}%
\pgfpathlineto{\pgfqpoint{2.289256in}{2.242479in}}%
\pgfpathlineto{\pgfqpoint{2.289256in}{2.242479in}}%
\pgfpathlineto{\pgfqpoint{2.288994in}{2.243114in}}%
\pgfpathlineto{\pgfqpoint{2.288994in}{2.243114in}}%
\pgfpathlineto{\pgfqpoint{2.288994in}{2.243114in}}%
\pgfpathlineto{\pgfqpoint{2.288287in}{2.232686in}}%
\pgfpathlineto{\pgfqpoint{2.287839in}{2.233634in}}%
\pgfpathlineto{\pgfqpoint{2.286948in}{2.234645in}}%
\pgfpathlineto{\pgfqpoint{2.286676in}{2.227871in}}%
\pgfpathlineto{\pgfqpoint{2.285710in}{2.229633in}}%
\pgfpathlineto{\pgfqpoint{2.285479in}{2.229759in}}%
\pgfpathlineto{\pgfqpoint{2.285474in}{2.227433in}}%
\pgfpathlineto{\pgfqpoint{2.284509in}{2.229759in}}%
\pgfpathlineto{\pgfqpoint{2.284509in}{2.229759in}}%
\pgfpathlineto{\pgfqpoint{2.284051in}{2.230575in}}%
\pgfpathlineto{\pgfqpoint{2.283991in}{2.228255in}}%
\pgfpathlineto{\pgfqpoint{2.282939in}{2.230515in}}%
\pgfpathlineto{\pgfqpoint{2.282400in}{2.231456in}}%
\pgfpathlineto{\pgfqpoint{2.282400in}{2.231456in}}%
\pgfpathlineto{\pgfqpoint{2.282400in}{2.231456in}}%
\pgfpathlineto{\pgfqpoint{2.282185in}{2.224790in}}%
\pgfpathlineto{\pgfqpoint{2.281157in}{2.226666in}}%
\pgfpathlineto{\pgfqpoint{2.279433in}{2.229229in}}%
\pgfpathlineto{\pgfqpoint{2.278296in}{2.222415in}}%
\pgfpathlineto{\pgfqpoint{2.278293in}{2.222477in}}%
\pgfpathlineto{\pgfqpoint{2.278127in}{2.222912in}}%
\pgfpathlineto{\pgfqpoint{2.277369in}{2.219332in}}%
\pgfpathlineto{\pgfqpoint{2.277000in}{2.220573in}}%
\pgfpathlineto{\pgfqpoint{2.276517in}{2.221628in}}%
\pgfpathlineto{\pgfqpoint{2.276517in}{2.221628in}}%
\pgfpathlineto{\pgfqpoint{2.276517in}{2.221628in}}%
\pgfpathlineto{\pgfqpoint{2.276506in}{2.219382in}}%
\pgfpathlineto{\pgfqpoint{2.275657in}{2.221613in}}%
\pgfpathlineto{\pgfqpoint{2.275657in}{2.221613in}}%
\pgfpathlineto{\pgfqpoint{2.274973in}{2.223162in}}%
\pgfpathlineto{\pgfqpoint{2.274973in}{2.223162in}}%
\pgfpathlineto{\pgfqpoint{2.274973in}{2.223162in}}%
\pgfpathlineto{\pgfqpoint{2.273845in}{2.218624in}}%
\pgfpathlineto{\pgfqpoint{2.273790in}{2.218686in}}%
\pgfpathlineto{\pgfqpoint{2.270634in}{2.223189in}}%
\pgfpathlineto{\pgfqpoint{2.270525in}{2.221279in}}%
\pgfpathlineto{\pgfqpoint{2.270187in}{2.215534in}}%
\pgfpathlineto{\pgfqpoint{2.269382in}{2.217006in}}%
\pgfpathlineto{\pgfqpoint{2.267658in}{2.220750in}}%
\pgfpathlineto{\pgfqpoint{2.267130in}{2.219294in}}%
\pgfpathlineto{\pgfqpoint{2.266553in}{2.216645in}}%
\pgfpathlineto{\pgfqpoint{2.266042in}{2.217746in}}%
\pgfpathlineto{\pgfqpoint{2.265514in}{2.218418in}}%
\pgfpathlineto{\pgfqpoint{2.264489in}{2.215669in}}%
\pgfpathlineto{\pgfqpoint{2.264411in}{2.215974in}}%
\pgfpathlineto{\pgfqpoint{2.262565in}{2.218963in}}%
\pgfpathlineto{\pgfqpoint{2.262303in}{2.217411in}}%
\pgfpathlineto{\pgfqpoint{2.262115in}{2.211338in}}%
\pgfpathlineto{\pgfqpoint{2.261167in}{2.212977in}}%
\pgfpathlineto{\pgfqpoint{2.258188in}{2.216181in}}%
\pgfpathlineto{\pgfqpoint{2.256859in}{2.209760in}}%
\pgfpathlineto{\pgfqpoint{2.256708in}{2.210303in}}%
\pgfpathlineto{\pgfqpoint{2.256027in}{2.211691in}}%
\pgfpathlineto{\pgfqpoint{2.256027in}{2.211691in}}%
\pgfpathlineto{\pgfqpoint{2.256027in}{2.211691in}}%
\pgfpathlineto{\pgfqpoint{2.255997in}{2.209587in}}%
\pgfpathlineto{\pgfqpoint{2.254934in}{2.211275in}}%
\pgfpathlineto{\pgfqpoint{2.253129in}{2.214229in}}%
\pgfpathlineto{\pgfqpoint{2.252497in}{2.212850in}}%
\pgfpathlineto{\pgfqpoint{2.252453in}{2.210756in}}%
\pgfpathlineto{\pgfqpoint{2.251476in}{2.212861in}}%
\pgfpathlineto{\pgfqpoint{2.251476in}{2.212861in}}%
\pgfpathlineto{\pgfqpoint{2.249622in}{2.215688in}}%
\pgfpathlineto{\pgfqpoint{2.249480in}{2.214076in}}%
\pgfpathlineto{\pgfqpoint{2.248652in}{2.211292in}}%
\pgfpathlineto{\pgfqpoint{2.248341in}{2.211951in}}%
\pgfpathlineto{\pgfqpoint{2.247895in}{2.212670in}}%
\pgfpathlineto{\pgfqpoint{2.247723in}{2.208831in}}%
\pgfpathlineto{\pgfqpoint{2.246782in}{2.210086in}}%
\pgfpathlineto{\pgfqpoint{2.246083in}{2.211521in}}%
\pgfpathlineto{\pgfqpoint{2.246083in}{2.211521in}}%
\pgfpathlineto{\pgfqpoint{2.246083in}{2.211521in}}%
\pgfpathlineto{\pgfqpoint{2.245947in}{2.209460in}}%
\pgfpathlineto{\pgfqpoint{2.245043in}{2.210953in}}%
\pgfpathlineto{\pgfqpoint{2.244204in}{2.212625in}}%
\pgfpathlineto{\pgfqpoint{2.242957in}{2.214477in}}%
\pgfpathlineto{\pgfqpoint{2.242839in}{2.212417in}}%
\pgfpathlineto{\pgfqpoint{2.241850in}{2.214207in}}%
\pgfpathlineto{\pgfqpoint{2.239405in}{2.216323in}}%
\pgfpathlineto{\pgfqpoint{2.239208in}{2.214569in}}%
\pgfpathlineto{\pgfqpoint{2.238239in}{2.206017in}}%
\pgfpathlineto{\pgfqpoint{2.238013in}{2.206727in}}%
\pgfpathlineto{\pgfqpoint{2.237873in}{2.207023in}}%
\pgfpathlineto{\pgfqpoint{2.236979in}{2.202367in}}%
\pgfpathlineto{\pgfqpoint{2.236771in}{2.202721in}}%
\pgfpathlineto{\pgfqpoint{2.235548in}{2.204727in}}%
\pgfpathlineto{\pgfqpoint{2.235459in}{2.202733in}}%
\pgfpathlineto{\pgfqpoint{2.234784in}{2.204736in}}%
\pgfpathlineto{\pgfqpoint{2.234784in}{2.204736in}}%
\pgfpathlineto{\pgfqpoint{2.234240in}{2.205619in}}%
\pgfpathlineto{\pgfqpoint{2.233192in}{2.203174in}}%
\pgfpathlineto{\pgfqpoint{2.233113in}{2.203468in}}%
\pgfpathlineto{\pgfqpoint{2.231271in}{2.205819in}}%
\pgfpathlineto{\pgfqpoint{2.230648in}{2.206760in}}%
\pgfpathlineto{\pgfqpoint{2.230636in}{2.204837in}}%
\pgfpathlineto{\pgfqpoint{2.229571in}{2.206716in}}%
\pgfpathlineto{\pgfqpoint{2.226522in}{2.212059in}}%
\pgfpathlineto{\pgfqpoint{2.225975in}{2.212999in}}%
\pgfpathlineto{\pgfqpoint{2.225975in}{2.212999in}}%
\pgfpathlineto{\pgfqpoint{2.225975in}{2.212999in}}%
\pgfpathlineto{\pgfqpoint{2.225964in}{2.211014in}}%
\pgfpathlineto{\pgfqpoint{2.224917in}{2.212950in}}%
\pgfpathlineto{\pgfqpoint{2.224917in}{2.212950in}}%
\pgfpathlineto{\pgfqpoint{2.224064in}{2.215120in}}%
\pgfpathlineto{\pgfqpoint{2.223812in}{2.213607in}}%
\pgfpathlineto{\pgfqpoint{2.222882in}{2.215599in}}%
\pgfpathlineto{\pgfqpoint{2.222470in}{2.214031in}}%
\pgfpathlineto{\pgfqpoint{2.221464in}{2.207527in}}%
\pgfpathlineto{\pgfqpoint{2.221369in}{2.207585in}}%
\pgfpathlineto{\pgfqpoint{2.220842in}{2.208401in}}%
\pgfpathlineto{\pgfqpoint{2.219737in}{2.204439in}}%
\pgfpathlineto{\pgfqpoint{2.219606in}{2.204904in}}%
\pgfpathlineto{\pgfqpoint{2.218919in}{2.205949in}}%
\pgfpathlineto{\pgfqpoint{2.218681in}{2.200641in}}%
\pgfpathlineto{\pgfqpoint{2.217748in}{2.202551in}}%
\pgfpathlineto{\pgfqpoint{2.217085in}{2.203478in}}%
\pgfpathlineto{\pgfqpoint{2.216559in}{2.195006in}}%
\pgfpathlineto{\pgfqpoint{2.215935in}{2.195870in}}%
\pgfpathlineto{\pgfqpoint{2.214192in}{2.198231in}}%
\pgfpathlineto{\pgfqpoint{2.213913in}{2.195038in}}%
\pgfpathlineto{\pgfqpoint{2.213078in}{2.196704in}}%
\pgfpathlineto{\pgfqpoint{2.211334in}{2.199003in}}%
\pgfpathlineto{\pgfqpoint{2.210329in}{2.196741in}}%
\pgfpathlineto{\pgfqpoint{2.210262in}{2.197028in}}%
\pgfpathlineto{\pgfqpoint{2.208904in}{2.199666in}}%
\pgfpathlineto{\pgfqpoint{2.208792in}{2.199895in}}%
\pgfpathlineto{\pgfqpoint{2.208792in}{2.199895in}}%
\pgfpathlineto{\pgfqpoint{2.208792in}{2.199895in}}%
\pgfpathlineto{\pgfqpoint{2.208782in}{2.198024in}}%
\pgfpathlineto{\pgfqpoint{2.207681in}{2.199799in}}%
\pgfpathlineto{\pgfqpoint{2.206939in}{2.200945in}}%
\pgfpathlineto{\pgfqpoint{2.206939in}{2.200945in}}%
\pgfpathlineto{\pgfqpoint{2.206939in}{2.200945in}}%
\pgfpathlineto{\pgfqpoint{2.206439in}{2.197900in}}%
\pgfpathlineto{\pgfqpoint{2.205866in}{2.198586in}}%
\pgfpathlineto{\pgfqpoint{2.205057in}{2.199615in}}%
\pgfpathlineto{\pgfqpoint{2.203693in}{2.189341in}}%
\pgfpathlineto{\pgfqpoint{2.203650in}{2.189454in}}%
\pgfpathlineto{\pgfqpoint{2.203635in}{2.189511in}}%
\pgfpathlineto{\pgfqpoint{2.203635in}{2.189511in}}%
\pgfpathlineto{\pgfqpoint{2.203635in}{2.189511in}}%
\pgfpathlineto{\pgfqpoint{2.202792in}{2.187237in}}%
\pgfpathlineto{\pgfqpoint{2.202542in}{2.187803in}}%
\pgfpathlineto{\pgfqpoint{2.202286in}{2.188143in}}%
\pgfpathlineto{\pgfqpoint{2.202286in}{2.188143in}}%
\pgfpathlineto{\pgfqpoint{2.202286in}{2.188143in}}%
\pgfpathlineto{\pgfqpoint{2.201714in}{2.183520in}}%
\pgfpathlineto{\pgfqpoint{2.201216in}{2.184254in}}%
\pgfpathlineto{\pgfqpoint{2.200276in}{2.185496in}}%
\pgfpathlineto{\pgfqpoint{2.200186in}{2.183813in}}%
\pgfpathlineto{\pgfqpoint{2.199030in}{2.178316in}}%
\pgfpathlineto{\pgfqpoint{2.198821in}{2.178653in}}%
\pgfpathlineto{\pgfqpoint{2.197295in}{2.180788in}}%
\pgfpathlineto{\pgfqpoint{2.197100in}{2.179465in}}%
\pgfpathlineto{\pgfqpoint{2.197084in}{2.177699in}}%
\pgfpathlineto{\pgfqpoint{2.196030in}{2.179380in}}%
\pgfpathlineto{\pgfqpoint{2.194540in}{2.179802in}}%
\pgfpathlineto{\pgfqpoint{2.194537in}{2.178043in}}%
\pgfpathlineto{\pgfqpoint{2.193487in}{2.179610in}}%
\pgfpathlineto{\pgfqpoint{2.192860in}{2.180505in}}%
\pgfpathlineto{\pgfqpoint{2.192860in}{2.180505in}}%
\pgfpathlineto{\pgfqpoint{2.192860in}{2.180505in}}%
\pgfpathlineto{\pgfqpoint{2.192051in}{2.178338in}}%
\pgfpathlineto{\pgfqpoint{2.191754in}{2.179120in}}%
\pgfpathlineto{\pgfqpoint{2.191613in}{2.179287in}}%
\pgfpathlineto{\pgfqpoint{2.191388in}{2.175909in}}%
\pgfpathlineto{\pgfqpoint{2.190486in}{2.177526in}}%
\pgfpathlineto{\pgfqpoint{2.190223in}{2.178139in}}%
\pgfpathlineto{\pgfqpoint{2.190223in}{2.178139in}}%
\pgfpathlineto{\pgfqpoint{2.190223in}{2.178139in}}%
\pgfpathlineto{\pgfqpoint{2.189674in}{2.175167in}}%
\pgfpathlineto{\pgfqpoint{2.189107in}{2.176390in}}%
\pgfpathlineto{\pgfqpoint{2.188806in}{2.176891in}}%
\pgfpathlineto{\pgfqpoint{2.188806in}{2.176891in}}%
\pgfpathlineto{\pgfqpoint{2.188806in}{2.176891in}}%
\pgfpathlineto{\pgfqpoint{2.188782in}{2.175161in}}%
\pgfpathlineto{\pgfqpoint{2.187745in}{2.176883in}}%
\pgfpathlineto{\pgfqpoint{2.187745in}{2.176883in}}%
\pgfpathlineto{\pgfqpoint{2.187470in}{2.177550in}}%
\pgfpathlineto{\pgfqpoint{2.187470in}{2.177550in}}%
\pgfpathlineto{\pgfqpoint{2.187470in}{2.177550in}}%
\pgfpathlineto{\pgfqpoint{2.187459in}{2.175823in}}%
\pgfpathlineto{\pgfqpoint{2.186399in}{2.177543in}}%
\pgfpathlineto{\pgfqpoint{2.186399in}{2.177543in}}%
\pgfpathlineto{\pgfqpoint{2.186007in}{2.178376in}}%
\pgfpathlineto{\pgfqpoint{2.186007in}{2.178376in}}%
\pgfpathlineto{\pgfqpoint{2.186007in}{2.178376in}}%
\pgfpathlineto{\pgfqpoint{2.185358in}{2.176373in}}%
\pgfpathlineto{\pgfqpoint{2.184847in}{2.176982in}}%
\pgfpathlineto{\pgfqpoint{2.183533in}{2.178921in}}%
\pgfpathlineto{\pgfqpoint{2.182356in}{2.175330in}}%
\pgfpathlineto{\pgfqpoint{2.182000in}{2.175772in}}%
\pgfpathlineto{\pgfqpoint{2.181414in}{2.169593in}}%
\pgfpathlineto{\pgfqpoint{2.180793in}{2.170253in}}%
\pgfpathlineto{\pgfqpoint{2.180438in}{2.171023in}}%
\pgfpathlineto{\pgfqpoint{2.180438in}{2.171023in}}%
\pgfpathlineto{\pgfqpoint{2.180438in}{2.171023in}}%
\pgfpathlineto{\pgfqpoint{2.179585in}{2.167080in}}%
\pgfpathlineto{\pgfqpoint{2.179414in}{2.167299in}}%
\pgfpathlineto{\pgfqpoint{2.178938in}{2.168012in}}%
\pgfpathlineto{\pgfqpoint{2.178597in}{2.166834in}}%
\pgfpathlineto{\pgfqpoint{2.177771in}{2.159200in}}%
\pgfpathlineto{\pgfqpoint{2.177423in}{2.159909in}}%
\pgfpathlineto{\pgfqpoint{2.177206in}{2.160182in}}%
\pgfpathlineto{\pgfqpoint{2.177206in}{2.160182in}}%
\pgfpathlineto{\pgfqpoint{2.177206in}{2.160182in}}%
\pgfpathlineto{\pgfqpoint{2.177192in}{2.158538in}}%
\pgfpathlineto{\pgfqpoint{2.176196in}{2.160173in}}%
\pgfpathlineto{\pgfqpoint{2.176196in}{2.160173in}}%
\pgfpathlineto{\pgfqpoint{2.175603in}{2.161045in}}%
\pgfpathlineto{\pgfqpoint{2.175603in}{2.161045in}}%
\pgfpathlineto{\pgfqpoint{2.175603in}{2.161045in}}%
\pgfpathlineto{\pgfqpoint{2.174531in}{2.159235in}}%
\pgfpathlineto{\pgfqpoint{2.174509in}{2.159289in}}%
\pgfpathlineto{\pgfqpoint{2.173400in}{2.160921in}}%
\pgfpathlineto{\pgfqpoint{2.173085in}{2.159884in}}%
\pgfpathlineto{\pgfqpoint{2.171931in}{2.156631in}}%
\pgfpathlineto{\pgfqpoint{2.171929in}{2.156685in}}%
\pgfpathlineto{\pgfqpoint{2.169472in}{2.160425in}}%
\pgfpathlineto{\pgfqpoint{2.168501in}{2.153874in}}%
\pgfpathlineto{\pgfqpoint{2.168316in}{2.154090in}}%
\pgfpathlineto{\pgfqpoint{2.167787in}{2.154738in}}%
\pgfpathlineto{\pgfqpoint{2.167787in}{2.154738in}}%
\pgfpathlineto{\pgfqpoint{2.167787in}{2.154738in}}%
\pgfpathlineto{\pgfqpoint{2.167770in}{2.153137in}}%
\pgfpathlineto{\pgfqpoint{2.166744in}{2.154538in}}%
\pgfpathlineto{\pgfqpoint{2.166399in}{2.155024in}}%
\pgfpathlineto{\pgfqpoint{2.165533in}{2.152855in}}%
\pgfpathlineto{\pgfqpoint{2.165326in}{2.153124in}}%
\pgfpathlineto{\pgfqpoint{2.163456in}{2.156406in}}%
\pgfpathlineto{\pgfqpoint{2.163426in}{2.154867in}}%
\pgfpathlineto{\pgfqpoint{2.163345in}{2.153279in}}%
\pgfpathlineto{\pgfqpoint{2.162354in}{2.154675in}}%
\pgfpathlineto{\pgfqpoint{2.160211in}{2.157736in}}%
\pgfpathlineto{\pgfqpoint{2.159569in}{2.158703in}}%
\pgfpathlineto{\pgfqpoint{2.159569in}{2.158703in}}%
\pgfpathlineto{\pgfqpoint{2.159569in}{2.158703in}}%
\pgfpathlineto{\pgfqpoint{2.158856in}{2.156496in}}%
\pgfpathlineto{\pgfqpoint{2.158432in}{2.157460in}}%
\pgfpathlineto{\pgfqpoint{2.157778in}{2.158211in}}%
\pgfpathlineto{\pgfqpoint{2.157699in}{2.156629in}}%
\pgfpathlineto{\pgfqpoint{2.156669in}{2.158289in}}%
\pgfpathlineto{\pgfqpoint{2.156462in}{2.158557in}}%
\pgfpathlineto{\pgfqpoint{2.156462in}{2.158557in}}%
\pgfpathlineto{\pgfqpoint{2.156462in}{2.158557in}}%
\pgfpathlineto{\pgfqpoint{2.155791in}{2.154902in}}%
\pgfpathlineto{\pgfqpoint{2.155375in}{2.155489in}}%
\pgfpathlineto{\pgfqpoint{2.152504in}{2.159762in}}%
\pgfpathlineto{\pgfqpoint{2.151836in}{2.159151in}}%
\pgfpathlineto{\pgfqpoint{2.151018in}{2.155793in}}%
\pgfpathlineto{\pgfqpoint{2.150733in}{2.156165in}}%
\pgfpathlineto{\pgfqpoint{2.149802in}{2.157549in}}%
\pgfpathlineto{\pgfqpoint{2.149760in}{2.155992in}}%
\pgfpathlineto{\pgfqpoint{2.148978in}{2.152461in}}%
\pgfpathlineto{\pgfqpoint{2.148625in}{2.153097in}}%
\pgfpathlineto{\pgfqpoint{2.148115in}{2.153734in}}%
\pgfpathlineto{\pgfqpoint{2.147751in}{2.152563in}}%
\pgfpathlineto{\pgfqpoint{2.147095in}{2.147916in}}%
\pgfpathlineto{\pgfqpoint{2.146670in}{2.149183in}}%
\pgfpathlineto{\pgfqpoint{2.145619in}{2.148767in}}%
\pgfpathlineto{\pgfqpoint{2.144946in}{2.145009in}}%
\pgfpathlineto{\pgfqpoint{2.144334in}{2.145851in}}%
\pgfpathlineto{\pgfqpoint{2.143789in}{2.146798in}}%
\pgfpathlineto{\pgfqpoint{2.143781in}{2.145288in}}%
\pgfpathlineto{\pgfqpoint{2.143081in}{2.142858in}}%
\pgfpathlineto{\pgfqpoint{2.142708in}{2.143697in}}%
\pgfpathlineto{\pgfqpoint{2.142555in}{2.143907in}}%
\pgfpathlineto{\pgfqpoint{2.142221in}{2.138355in}}%
\pgfpathlineto{\pgfqpoint{2.141416in}{2.139714in}}%
\pgfpathlineto{\pgfqpoint{2.140386in}{2.141126in}}%
\pgfpathlineto{\pgfqpoint{2.139564in}{2.142276in}}%
\pgfpathlineto{\pgfqpoint{2.139564in}{2.142276in}}%
\pgfpathlineto{\pgfqpoint{2.139564in}{2.142276in}}%
\pgfpathlineto{\pgfqpoint{2.138518in}{2.136551in}}%
\pgfpathlineto{\pgfqpoint{2.138430in}{2.136655in}}%
\pgfpathlineto{\pgfqpoint{2.137297in}{2.137112in}}%
\pgfpathlineto{\pgfqpoint{2.136445in}{2.134963in}}%
\pgfpathlineto{\pgfqpoint{2.136235in}{2.135379in}}%
\pgfpathlineto{\pgfqpoint{2.136085in}{2.135690in}}%
\pgfpathlineto{\pgfqpoint{2.136085in}{2.135690in}}%
\pgfpathlineto{\pgfqpoint{2.136085in}{2.135690in}}%
\pgfpathlineto{\pgfqpoint{2.136062in}{2.134231in}}%
\pgfpathlineto{\pgfqpoint{2.135181in}{2.135684in}}%
\pgfpathlineto{\pgfqpoint{2.134924in}{2.135736in}}%
\pgfpathlineto{\pgfqpoint{2.134195in}{2.131174in}}%
\pgfpathlineto{\pgfqpoint{2.133759in}{2.131898in}}%
\pgfpathlineto{\pgfqpoint{2.133671in}{2.132415in}}%
\pgfpathlineto{\pgfqpoint{2.133671in}{2.132415in}}%
\pgfpathlineto{\pgfqpoint{2.133671in}{2.132415in}}%
\pgfpathlineto{\pgfqpoint{2.133657in}{2.130971in}}%
\pgfpathlineto{\pgfqpoint{2.132739in}{2.132418in}}%
\pgfpathlineto{\pgfqpoint{2.132739in}{2.132418in}}%
\pgfpathlineto{\pgfqpoint{2.132029in}{2.132883in}}%
\pgfpathlineto{\pgfqpoint{2.131724in}{2.130314in}}%
\pgfpathlineto{\pgfqpoint{2.130962in}{2.131501in}}%
\pgfpathlineto{\pgfqpoint{2.130880in}{2.131552in}}%
\pgfpathlineto{\pgfqpoint{2.130825in}{2.128738in}}%
\pgfpathlineto{\pgfqpoint{2.129813in}{2.130386in}}%
\pgfpathlineto{\pgfqpoint{2.128219in}{2.133115in}}%
\pgfpathlineto{\pgfqpoint{2.127106in}{2.134506in}}%
\pgfpathlineto{\pgfqpoint{2.125876in}{2.134926in}}%
\pgfpathlineto{\pgfqpoint{2.125126in}{2.129792in}}%
\pgfpathlineto{\pgfqpoint{2.124723in}{2.130151in}}%
\pgfpathlineto{\pgfqpoint{2.124618in}{2.130202in}}%
\pgfpathlineto{\pgfqpoint{2.123680in}{2.126937in}}%
\pgfpathlineto{\pgfqpoint{2.123552in}{2.127193in}}%
\pgfpathlineto{\pgfqpoint{2.123195in}{2.127654in}}%
\pgfpathlineto{\pgfqpoint{2.123195in}{2.127654in}}%
\pgfpathlineto{\pgfqpoint{2.123195in}{2.127654in}}%
\pgfpathlineto{\pgfqpoint{2.123172in}{2.126247in}}%
\pgfpathlineto{\pgfqpoint{2.122196in}{2.127627in}}%
\pgfpathlineto{\pgfqpoint{2.122196in}{2.127627in}}%
\pgfpathlineto{\pgfqpoint{2.121699in}{2.128138in}}%
\pgfpathlineto{\pgfqpoint{2.121699in}{2.128138in}}%
\pgfpathlineto{\pgfqpoint{2.121699in}{2.128138in}}%
\pgfpathlineto{\pgfqpoint{2.120877in}{2.124803in}}%
\pgfpathlineto{\pgfqpoint{2.120516in}{2.125313in}}%
\pgfpathlineto{\pgfqpoint{2.120219in}{2.125925in}}%
\pgfpathlineto{\pgfqpoint{2.120219in}{2.125925in}}%
\pgfpathlineto{\pgfqpoint{2.120219in}{2.125925in}}%
\pgfpathlineto{\pgfqpoint{2.119706in}{2.122309in}}%
\pgfpathlineto{\pgfqpoint{2.119068in}{2.123428in}}%
\pgfpathlineto{\pgfqpoint{2.118854in}{2.123835in}}%
\pgfpathlineto{\pgfqpoint{2.118854in}{2.123835in}}%
\pgfpathlineto{\pgfqpoint{2.118854in}{2.123835in}}%
\pgfpathlineto{\pgfqpoint{2.118533in}{2.121472in}}%
\pgfpathlineto{\pgfqpoint{2.117816in}{2.122437in}}%
\pgfpathlineto{\pgfqpoint{2.110547in}{2.126141in}}%
\pgfpathlineto{\pgfqpoint{2.110000in}{2.123957in}}%
\pgfpathlineto{\pgfqpoint{2.109424in}{2.124866in}}%
\pgfpathlineto{\pgfqpoint{2.108188in}{2.126835in}}%
\pgfpathlineto{\pgfqpoint{2.107917in}{2.125770in}}%
\pgfpathlineto{\pgfqpoint{2.107096in}{2.124154in}}%
\pgfpathlineto{\pgfqpoint{2.106821in}{2.124506in}}%
\pgfpathlineto{\pgfqpoint{2.104977in}{2.125211in}}%
\pgfpathlineto{\pgfqpoint{2.104923in}{2.123855in}}%
\pgfpathlineto{\pgfqpoint{2.103896in}{2.125213in}}%
\pgfpathlineto{\pgfqpoint{2.103896in}{2.125213in}}%
\pgfpathlineto{\pgfqpoint{2.103838in}{2.125313in}}%
\pgfpathlineto{\pgfqpoint{2.103838in}{2.125313in}}%
\pgfpathlineto{\pgfqpoint{2.103838in}{2.125313in}}%
\pgfpathlineto{\pgfqpoint{2.102902in}{2.122062in}}%
\pgfpathlineto{\pgfqpoint{2.102713in}{2.122263in}}%
\pgfpathlineto{\pgfqpoint{2.102612in}{2.122614in}}%
\pgfpathlineto{\pgfqpoint{2.102612in}{2.122614in}}%
\pgfpathlineto{\pgfqpoint{2.102612in}{2.122614in}}%
\pgfpathlineto{\pgfqpoint{2.102603in}{2.121269in}}%
\pgfpathlineto{\pgfqpoint{2.101515in}{2.122372in}}%
\pgfpathlineto{\pgfqpoint{2.101294in}{2.122572in}}%
\pgfpathlineto{\pgfqpoint{2.101294in}{2.122572in}}%
\pgfpathlineto{\pgfqpoint{2.101294in}{2.122572in}}%
\pgfpathlineto{\pgfqpoint{2.101273in}{2.121229in}}%
\pgfpathlineto{\pgfqpoint{2.100202in}{2.122481in}}%
\pgfpathlineto{\pgfqpoint{2.099602in}{2.122141in}}%
\pgfpathlineto{\pgfqpoint{2.099052in}{2.120117in}}%
\pgfpathlineto{\pgfqpoint{2.098510in}{2.120867in}}%
\pgfpathlineto{\pgfqpoint{2.098458in}{2.120917in}}%
\pgfpathlineto{\pgfqpoint{2.098458in}{2.120917in}}%
\pgfpathlineto{\pgfqpoint{2.098458in}{2.120917in}}%
\pgfpathlineto{\pgfqpoint{2.098309in}{2.117177in}}%
\pgfpathlineto{\pgfqpoint{2.097386in}{2.118373in}}%
\pgfpathlineto{\pgfqpoint{2.096180in}{2.118693in}}%
\pgfpathlineto{\pgfqpoint{2.096163in}{2.117420in}}%
\pgfpathlineto{\pgfqpoint{2.095360in}{2.118665in}}%
\pgfpathlineto{\pgfqpoint{2.095360in}{2.118665in}}%
\pgfpathlineto{\pgfqpoint{2.094046in}{2.120507in}}%
\pgfpathlineto{\pgfqpoint{2.092577in}{2.122150in}}%
\pgfpathlineto{\pgfqpoint{2.092361in}{2.121026in}}%
\pgfpathlineto{\pgfqpoint{2.091226in}{2.118366in}}%
\pgfpathlineto{\pgfqpoint{2.087198in}{2.121049in}}%
\pgfpathlineto{\pgfqpoint{2.086854in}{2.119077in}}%
\pgfpathlineto{\pgfqpoint{2.086077in}{2.120017in}}%
\pgfpathlineto{\pgfqpoint{2.085285in}{2.121056in}}%
\pgfpathlineto{\pgfqpoint{2.085285in}{2.121056in}}%
\pgfpathlineto{\pgfqpoint{2.085285in}{2.121056in}}%
\pgfpathlineto{\pgfqpoint{2.084104in}{2.114652in}}%
\pgfpathlineto{\pgfqpoint{2.083962in}{2.114947in}}%
\pgfpathlineto{\pgfqpoint{2.082652in}{2.116179in}}%
\pgfpathlineto{\pgfqpoint{2.082627in}{2.114890in}}%
\pgfpathlineto{\pgfqpoint{2.081663in}{2.113010in}}%
\pgfpathlineto{\pgfqpoint{2.081520in}{2.113108in}}%
\pgfpathlineto{\pgfqpoint{2.078062in}{2.115065in}}%
\pgfpathlineto{\pgfqpoint{2.077796in}{2.114180in}}%
\pgfpathlineto{\pgfqpoint{2.076828in}{2.110224in}}%
\pgfpathlineto{\pgfqpoint{2.076548in}{2.110810in}}%
\pgfpathlineto{\pgfqpoint{2.072322in}{2.114093in}}%
\pgfpathlineto{\pgfqpoint{2.072274in}{2.112831in}}%
\pgfpathlineto{\pgfqpoint{2.071486in}{2.114099in}}%
\pgfpathlineto{\pgfqpoint{2.071486in}{2.114099in}}%
\pgfpathlineto{\pgfqpoint{2.070318in}{2.114203in}}%
\pgfpathlineto{\pgfqpoint{2.070275in}{2.112944in}}%
\pgfpathlineto{\pgfqpoint{2.069705in}{2.114211in}}%
\pgfpathlineto{\pgfqpoint{2.069705in}{2.114211in}}%
\pgfpathlineto{\pgfqpoint{2.067563in}{2.115484in}}%
\pgfpathlineto{\pgfqpoint{2.067540in}{2.114278in}}%
\pgfpathlineto{\pgfqpoint{2.066735in}{2.112843in}}%
\pgfpathlineto{\pgfqpoint{2.066407in}{2.113183in}}%
\pgfpathlineto{\pgfqpoint{2.066269in}{2.113572in}}%
\pgfpathlineto{\pgfqpoint{2.066269in}{2.113572in}}%
\pgfpathlineto{\pgfqpoint{2.066269in}{2.113572in}}%
\pgfpathlineto{\pgfqpoint{2.065875in}{2.111417in}}%
\pgfpathlineto{\pgfqpoint{2.065142in}{2.111725in}}%
\pgfpathlineto{\pgfqpoint{2.063786in}{2.113275in}}%
\pgfpathlineto{\pgfqpoint{2.063736in}{2.112130in}}%
\pgfpathlineto{\pgfqpoint{2.062632in}{2.108492in}}%
\pgfpathlineto{\pgfqpoint{2.062572in}{2.108589in}}%
\pgfpathlineto{\pgfqpoint{2.062483in}{2.108830in}}%
\pgfpathlineto{\pgfqpoint{2.062483in}{2.108830in}}%
\pgfpathlineto{\pgfqpoint{2.062483in}{2.108830in}}%
\pgfpathlineto{\pgfqpoint{2.062474in}{2.107601in}}%
\pgfpathlineto{\pgfqpoint{2.061724in}{2.108807in}}%
\pgfpathlineto{\pgfqpoint{2.061724in}{2.108807in}}%
\pgfpathlineto{\pgfqpoint{2.061522in}{2.109048in}}%
\pgfpathlineto{\pgfqpoint{2.061522in}{2.109048in}}%
\pgfpathlineto{\pgfqpoint{2.061522in}{2.109048in}}%
\pgfpathlineto{\pgfqpoint{2.060465in}{2.106384in}}%
\pgfpathlineto{\pgfqpoint{2.060409in}{2.106432in}}%
\pgfpathlineto{\pgfqpoint{2.060221in}{2.106625in}}%
\pgfpathlineto{\pgfqpoint{2.060221in}{2.106625in}}%
\pgfpathlineto{\pgfqpoint{2.060221in}{2.106625in}}%
\pgfpathlineto{\pgfqpoint{2.059774in}{2.104715in}}%
\pgfpathlineto{\pgfqpoint{2.059235in}{2.105244in}}%
\pgfpathlineto{\pgfqpoint{2.058464in}{2.104893in}}%
\pgfpathlineto{\pgfqpoint{2.058417in}{2.102517in}}%
\pgfpathlineto{\pgfqpoint{2.057344in}{2.104052in}}%
\pgfpathlineto{\pgfqpoint{2.056034in}{2.105395in}}%
\pgfpathlineto{\pgfqpoint{2.055878in}{2.104376in}}%
\pgfpathlineto{\pgfqpoint{2.054801in}{2.102196in}}%
\pgfpathlineto{\pgfqpoint{2.054793in}{2.102244in}}%
\pgfpathlineto{\pgfqpoint{2.053130in}{2.103146in}}%
\pgfpathlineto{\pgfqpoint{2.053031in}{2.102090in}}%
\pgfpathlineto{\pgfqpoint{2.051872in}{2.097418in}}%
\pgfpathlineto{\pgfqpoint{2.051843in}{2.097465in}}%
\pgfpathlineto{\pgfqpoint{2.051546in}{2.098132in}}%
\pgfpathlineto{\pgfqpoint{2.051546in}{2.098132in}}%
\pgfpathlineto{\pgfqpoint{2.051546in}{2.098132in}}%
\pgfpathlineto{\pgfqpoint{2.050707in}{2.096619in}}%
\pgfpathlineto{\pgfqpoint{2.050405in}{2.096810in}}%
\pgfpathlineto{\pgfqpoint{2.050301in}{2.096905in}}%
\pgfpathlineto{\pgfqpoint{2.050301in}{2.096905in}}%
\pgfpathlineto{\pgfqpoint{2.050301in}{2.096905in}}%
\pgfpathlineto{\pgfqpoint{2.049069in}{2.094414in}}%
\pgfpathlineto{\pgfqpoint{2.048958in}{2.094509in}}%
\pgfpathlineto{\pgfqpoint{2.048958in}{2.094509in}}%
\pgfpathlineto{\pgfqpoint{2.048958in}{2.094509in}}%
\pgfpathlineto{\pgfqpoint{2.048930in}{2.093335in}}%
\pgfpathlineto{\pgfqpoint{2.047845in}{2.094425in}}%
\pgfpathlineto{\pgfqpoint{2.047626in}{2.094662in}}%
\pgfpathlineto{\pgfqpoint{2.047626in}{2.094662in}}%
\pgfpathlineto{\pgfqpoint{2.047626in}{2.094662in}}%
\pgfpathlineto{\pgfqpoint{2.046683in}{2.093077in}}%
\pgfpathlineto{\pgfqpoint{2.046481in}{2.093219in}}%
\pgfpathlineto{\pgfqpoint{2.045978in}{2.094355in}}%
\pgfpathlineto{\pgfqpoint{2.045419in}{2.093990in}}%
\pgfpathlineto{\pgfqpoint{2.044488in}{2.092746in}}%
\pgfpathlineto{\pgfqpoint{2.044329in}{2.092982in}}%
\pgfpathlineto{\pgfqpoint{2.043956in}{2.093549in}}%
\pgfpathlineto{\pgfqpoint{2.043956in}{2.093549in}}%
\pgfpathlineto{\pgfqpoint{2.043956in}{2.093549in}}%
\pgfpathlineto{\pgfqpoint{2.043203in}{2.090917in}}%
\pgfpathlineto{\pgfqpoint{2.042848in}{2.091200in}}%
\pgfpathlineto{\pgfqpoint{2.039249in}{2.093623in}}%
\pgfpathlineto{\pgfqpoint{2.039248in}{2.092467in}}%
\pgfpathlineto{\pgfqpoint{2.039213in}{2.091313in}}%
\pgfpathlineto{\pgfqpoint{2.038450in}{2.092489in}}%
\pgfpathlineto{\pgfqpoint{2.038450in}{2.092489in}}%
\pgfpathlineto{\pgfqpoint{2.037182in}{2.094135in}}%
\pgfpathlineto{\pgfqpoint{2.037051in}{2.093029in}}%
\pgfpathlineto{\pgfqpoint{2.036861in}{2.093170in}}%
\pgfpathlineto{\pgfqpoint{2.036861in}{2.093170in}}%
\pgfpathlineto{\pgfqpoint{2.036861in}{2.093170in}}%
\pgfpathlineto{\pgfqpoint{2.036063in}{2.091951in}}%
\pgfpathlineto{\pgfqpoint{2.035704in}{2.092373in}}%
\pgfpathlineto{\pgfqpoint{2.035177in}{2.093124in}}%
\pgfpathlineto{\pgfqpoint{2.035177in}{2.093124in}}%
\pgfpathlineto{\pgfqpoint{2.035177in}{2.093124in}}%
\pgfpathlineto{\pgfqpoint{2.035133in}{2.091977in}}%
\pgfpathlineto{\pgfqpoint{2.034464in}{2.093149in}}%
\pgfpathlineto{\pgfqpoint{2.034464in}{2.093149in}}%
\pgfpathlineto{\pgfqpoint{2.034340in}{2.093290in}}%
\pgfpathlineto{\pgfqpoint{2.034340in}{2.093290in}}%
\pgfpathlineto{\pgfqpoint{2.034340in}{2.093290in}}%
\pgfpathlineto{\pgfqpoint{2.033268in}{2.092078in}}%
\pgfpathlineto{\pgfqpoint{2.031227in}{2.093043in}}%
\pgfpathlineto{\pgfqpoint{2.029891in}{2.089849in}}%
\pgfpathlineto{\pgfqpoint{2.029883in}{2.089896in}}%
\pgfpathlineto{\pgfqpoint{2.029037in}{2.091063in}}%
\pgfpathlineto{\pgfqpoint{2.028994in}{2.089930in}}%
\pgfpathlineto{\pgfqpoint{2.027724in}{2.085930in}}%
\pgfpathlineto{\pgfqpoint{2.027688in}{2.086069in}}%
\pgfpathlineto{\pgfqpoint{2.027590in}{2.086209in}}%
\pgfpathlineto{\pgfqpoint{2.027590in}{2.086209in}}%
\pgfpathlineto{\pgfqpoint{2.027590in}{2.086209in}}%
\pgfpathlineto{\pgfqpoint{2.027137in}{2.083460in}}%
\pgfpathlineto{\pgfqpoint{2.026448in}{2.084341in}}%
\pgfpathlineto{\pgfqpoint{2.026275in}{2.084759in}}%
\pgfpathlineto{\pgfqpoint{2.026275in}{2.084759in}}%
\pgfpathlineto{\pgfqpoint{2.026275in}{2.084759in}}%
\pgfpathlineto{\pgfqpoint{2.025841in}{2.082765in}}%
\pgfpathlineto{\pgfqpoint{2.025186in}{2.083599in}}%
\pgfpathlineto{\pgfqpoint{2.023703in}{2.085498in}}%
\pgfpathlineto{\pgfqpoint{2.023467in}{2.085730in}}%
\pgfpathlineto{\pgfqpoint{2.023467in}{2.085730in}}%
\pgfpathlineto{\pgfqpoint{2.023467in}{2.085730in}}%
\pgfpathlineto{\pgfqpoint{2.022565in}{2.084620in}}%
\pgfpathlineto{\pgfqpoint{2.022349in}{2.084713in}}%
\pgfpathlineto{\pgfqpoint{2.022113in}{2.084898in}}%
\pgfpathlineto{\pgfqpoint{2.022113in}{2.084898in}}%
\pgfpathlineto{\pgfqpoint{2.022113in}{2.084898in}}%
\pgfpathlineto{\pgfqpoint{2.021377in}{2.081311in}}%
\pgfpathlineto{\pgfqpoint{2.020940in}{2.081726in}}%
\pgfpathlineto{\pgfqpoint{2.020824in}{2.081957in}}%
\pgfpathlineto{\pgfqpoint{2.020824in}{2.081957in}}%
\pgfpathlineto{\pgfqpoint{2.020824in}{2.081957in}}%
\pgfpathlineto{\pgfqpoint{2.019418in}{2.077908in}}%
\pgfpathlineto{\pgfqpoint{2.019388in}{2.078000in}}%
\pgfpathlineto{\pgfqpoint{2.017631in}{2.077614in}}%
\pgfpathlineto{\pgfqpoint{2.016790in}{2.075051in}}%
\pgfpathlineto{\pgfqpoint{2.016567in}{2.075509in}}%
\pgfpathlineto{\pgfqpoint{2.014590in}{2.076170in}}%
\pgfpathlineto{\pgfqpoint{2.013420in}{2.072074in}}%
\pgfpathlineto{\pgfqpoint{2.013245in}{2.072257in}}%
\pgfpathlineto{\pgfqpoint{2.013245in}{2.072257in}}%
\pgfpathlineto{\pgfqpoint{2.013245in}{2.072257in}}%
\pgfpathlineto{\pgfqpoint{2.013222in}{2.071188in}}%
\pgfpathlineto{\pgfqpoint{2.012275in}{2.072238in}}%
\pgfpathlineto{\pgfqpoint{2.012275in}{2.072238in}}%
\pgfpathlineto{\pgfqpoint{2.012200in}{2.072420in}}%
\pgfpathlineto{\pgfqpoint{2.012200in}{2.072420in}}%
\pgfpathlineto{\pgfqpoint{2.012200in}{2.072420in}}%
\pgfpathlineto{\pgfqpoint{2.011611in}{2.069633in}}%
\pgfpathlineto{\pgfqpoint{2.011065in}{2.070362in}}%
\pgfpathlineto{\pgfqpoint{2.010229in}{2.070028in}}%
\pgfpathlineto{\pgfqpoint{2.009834in}{2.067307in}}%
\pgfpathlineto{\pgfqpoint{2.009110in}{2.068488in}}%
\pgfpathlineto{\pgfqpoint{2.008365in}{2.069306in}}%
\pgfpathlineto{\pgfqpoint{2.008074in}{2.068386in}}%
\pgfpathlineto{\pgfqpoint{2.007928in}{2.064501in}}%
\pgfpathlineto{\pgfqpoint{2.006993in}{2.065497in}}%
\pgfpathlineto{\pgfqpoint{2.006061in}{2.065175in}}%
\pgfpathlineto{\pgfqpoint{2.005611in}{2.062454in}}%
\pgfpathlineto{\pgfqpoint{2.004933in}{2.062996in}}%
\pgfpathlineto{\pgfqpoint{2.004086in}{2.064396in}}%
\pgfpathlineto{\pgfqpoint{2.003950in}{2.063447in}}%
\pgfpathlineto{\pgfqpoint{2.003863in}{2.061418in}}%
\pgfpathlineto{\pgfqpoint{2.002836in}{2.062815in}}%
\pgfpathlineto{\pgfqpoint{2.001427in}{2.064799in}}%
\pgfpathlineto{\pgfqpoint{2.001321in}{2.063898in}}%
\pgfpathlineto{\pgfqpoint{2.000367in}{2.061612in}}%
\pgfpathlineto{\pgfqpoint{2.000351in}{2.061657in}}%
\pgfpathlineto{\pgfqpoint{1.998818in}{2.061437in}}%
\pgfpathlineto{\pgfqpoint{1.998780in}{2.059385in}}%
\pgfpathlineto{\pgfqpoint{1.997708in}{2.060327in}}%
\pgfpathlineto{\pgfqpoint{1.996500in}{2.061539in}}%
\pgfpathlineto{\pgfqpoint{1.996257in}{2.060784in}}%
\pgfpathlineto{\pgfqpoint{1.996111in}{2.060874in}}%
\pgfpathlineto{\pgfqpoint{1.996111in}{2.060874in}}%
\pgfpathlineto{\pgfqpoint{1.996111in}{2.060874in}}%
\pgfpathlineto{\pgfqpoint{1.995217in}{2.058394in}}%
\pgfpathlineto{\pgfqpoint{1.994939in}{2.058886in}}%
\pgfpathlineto{\pgfqpoint{1.993175in}{2.059346in}}%
\pgfpathlineto{\pgfqpoint{1.992836in}{2.057762in}}%
\pgfpathlineto{\pgfqpoint{1.992155in}{2.058477in}}%
\pgfpathlineto{\pgfqpoint{1.991964in}{2.058566in}}%
\pgfpathlineto{\pgfqpoint{1.991964in}{2.058566in}}%
\pgfpathlineto{\pgfqpoint{1.991964in}{2.058566in}}%
\pgfpathlineto{\pgfqpoint{1.991922in}{2.057553in}}%
\pgfpathlineto{\pgfqpoint{1.990924in}{2.058580in}}%
\pgfpathlineto{\pgfqpoint{1.990924in}{2.058580in}}%
\pgfpathlineto{\pgfqpoint{1.990317in}{2.059161in}}%
\pgfpathlineto{\pgfqpoint{1.990317in}{2.059161in}}%
\pgfpathlineto{\pgfqpoint{1.990317in}{2.059161in}}%
\pgfpathlineto{\pgfqpoint{1.990248in}{2.057228in}}%
\pgfpathlineto{\pgfqpoint{1.989225in}{2.058120in}}%
\pgfpathlineto{\pgfqpoint{1.989152in}{2.058254in}}%
\pgfpathlineto{\pgfqpoint{1.989152in}{2.058254in}}%
\pgfpathlineto{\pgfqpoint{1.989152in}{2.058254in}}%
\pgfpathlineto{\pgfqpoint{1.988845in}{2.056371in}}%
\pgfpathlineto{\pgfqpoint{1.987945in}{2.057396in}}%
\pgfpathlineto{\pgfqpoint{1.987643in}{2.057574in}}%
\pgfpathlineto{\pgfqpoint{1.986792in}{2.054539in}}%
\pgfpathlineto{\pgfqpoint{1.986511in}{2.054895in}}%
\pgfpathlineto{\pgfqpoint{1.986147in}{2.055562in}}%
\pgfpathlineto{\pgfqpoint{1.986147in}{2.055562in}}%
\pgfpathlineto{\pgfqpoint{1.986147in}{2.055562in}}%
\pgfpathlineto{\pgfqpoint{1.984616in}{2.051186in}}%
\pgfpathlineto{\pgfqpoint{1.984552in}{2.051319in}}%
\pgfpathlineto{\pgfqpoint{1.983626in}{2.051304in}}%
\pgfpathlineto{\pgfqpoint{1.983112in}{2.049462in}}%
\pgfpathlineto{\pgfqpoint{1.982407in}{2.050390in}}%
\pgfpathlineto{\pgfqpoint{1.981973in}{2.050832in}}%
\pgfpathlineto{\pgfqpoint{1.981958in}{2.049847in}}%
\pgfpathlineto{\pgfqpoint{1.981761in}{2.047034in}}%
\pgfpathlineto{\pgfqpoint{1.980795in}{2.047872in}}%
\pgfpathlineto{\pgfqpoint{1.980790in}{2.047916in}}%
\pgfpathlineto{\pgfqpoint{1.980790in}{2.047916in}}%
\pgfpathlineto{\pgfqpoint{1.980790in}{2.047916in}}%
\pgfpathlineto{\pgfqpoint{1.980741in}{2.046938in}}%
\pgfpathlineto{\pgfqpoint{1.979708in}{2.047863in}}%
\pgfpathlineto{\pgfqpoint{1.977530in}{2.048868in}}%
\pgfpathlineto{\pgfqpoint{1.976458in}{2.045364in}}%
\pgfpathlineto{\pgfqpoint{1.974990in}{2.045756in}}%
\pgfpathlineto{\pgfqpoint{1.974830in}{2.044921in}}%
\pgfpathlineto{\pgfqpoint{1.974274in}{2.041725in}}%
\pgfpathlineto{\pgfqpoint{1.973737in}{2.042294in}}%
\pgfpathlineto{\pgfqpoint{1.973438in}{2.042600in}}%
\pgfpathlineto{\pgfqpoint{1.973438in}{2.042600in}}%
\pgfpathlineto{\pgfqpoint{1.973438in}{2.042600in}}%
\pgfpathlineto{\pgfqpoint{1.973241in}{2.038903in}}%
\pgfpathlineto{\pgfqpoint{1.972205in}{2.040126in}}%
\pgfpathlineto{\pgfqpoint{1.972068in}{2.040300in}}%
\pgfpathlineto{\pgfqpoint{1.972068in}{2.040300in}}%
\pgfpathlineto{\pgfqpoint{1.972068in}{2.040300in}}%
\pgfpathlineto{\pgfqpoint{1.970600in}{2.036041in}}%
\pgfpathlineto{\pgfqpoint{1.970466in}{2.036215in}}%
\pgfpathlineto{\pgfqpoint{1.970102in}{2.036607in}}%
\pgfpathlineto{\pgfqpoint{1.970102in}{2.036607in}}%
\pgfpathlineto{\pgfqpoint{1.970102in}{2.036607in}}%
\pgfpathlineto{\pgfqpoint{1.969246in}{2.034908in}}%
\pgfpathlineto{\pgfqpoint{1.968974in}{2.035125in}}%
\pgfpathlineto{\pgfqpoint{1.968117in}{2.035054in}}%
\pgfpathlineto{\pgfqpoint{1.968091in}{2.034115in}}%
\pgfpathlineto{\pgfqpoint{1.967121in}{2.034939in}}%
\pgfpathlineto{\pgfqpoint{1.963448in}{2.035449in}}%
\pgfpathlineto{\pgfqpoint{1.963319in}{2.033666in}}%
\pgfpathlineto{\pgfqpoint{1.962328in}{2.034445in}}%
\pgfpathlineto{\pgfqpoint{1.962208in}{2.034531in}}%
\pgfpathlineto{\pgfqpoint{1.962208in}{2.034531in}}%
\pgfpathlineto{\pgfqpoint{1.962208in}{2.034531in}}%
\pgfpathlineto{\pgfqpoint{1.962200in}{2.033599in}}%
\pgfpathlineto{\pgfqpoint{1.961500in}{2.034507in}}%
\pgfpathlineto{\pgfqpoint{1.961500in}{2.034507in}}%
\pgfpathlineto{\pgfqpoint{1.960094in}{2.035088in}}%
\pgfpathlineto{\pgfqpoint{1.958908in}{2.033682in}}%
\pgfpathlineto{\pgfqpoint{1.958876in}{2.033725in}}%
\pgfpathlineto{\pgfqpoint{1.958121in}{2.034329in}}%
\pgfpathlineto{\pgfqpoint{1.957989in}{2.033791in}}%
\pgfpathlineto{\pgfqpoint{1.957363in}{2.031665in}}%
\pgfpathlineto{\pgfqpoint{1.956862in}{2.032052in}}%
\pgfpathlineto{\pgfqpoint{1.956759in}{2.032138in}}%
\pgfpathlineto{\pgfqpoint{1.956759in}{2.032138in}}%
\pgfpathlineto{\pgfqpoint{1.956759in}{2.032138in}}%
\pgfpathlineto{\pgfqpoint{1.955979in}{2.030282in}}%
\pgfpathlineto{\pgfqpoint{1.955706in}{2.030583in}}%
\pgfpathlineto{\pgfqpoint{1.955018in}{2.031572in}}%
\pgfpathlineto{\pgfqpoint{1.954661in}{2.030082in}}%
\pgfpathlineto{\pgfqpoint{1.953262in}{2.025082in}}%
\pgfpathlineto{\pgfqpoint{1.953151in}{2.025296in}}%
\pgfpathlineto{\pgfqpoint{1.951385in}{2.025240in}}%
\pgfpathlineto{\pgfqpoint{1.950387in}{2.022792in}}%
\pgfpathlineto{\pgfqpoint{1.950375in}{2.022835in}}%
\pgfpathlineto{\pgfqpoint{1.948284in}{2.024752in}}%
\pgfpathlineto{\pgfqpoint{1.948169in}{2.024923in}}%
\pgfpathlineto{\pgfqpoint{1.948169in}{2.024923in}}%
\pgfpathlineto{\pgfqpoint{1.948169in}{2.024923in}}%
\pgfpathlineto{\pgfqpoint{1.947764in}{2.023299in}}%
\pgfpathlineto{\pgfqpoint{1.947012in}{2.024151in}}%
\pgfpathlineto{\pgfqpoint{1.946235in}{2.024662in}}%
\pgfpathlineto{\pgfqpoint{1.945769in}{2.024022in}}%
\pgfpathlineto{\pgfqpoint{1.944684in}{2.022194in}}%
\pgfpathlineto{\pgfqpoint{1.944642in}{2.022279in}}%
\pgfpathlineto{\pgfqpoint{1.944556in}{2.022449in}}%
\pgfpathlineto{\pgfqpoint{1.944556in}{2.022449in}}%
\pgfpathlineto{\pgfqpoint{1.944556in}{2.022449in}}%
\pgfpathlineto{\pgfqpoint{1.943719in}{2.019618in}}%
\pgfpathlineto{\pgfqpoint{1.943496in}{2.019873in}}%
\pgfpathlineto{\pgfqpoint{1.941629in}{2.019884in}}%
\pgfpathlineto{\pgfqpoint{1.940561in}{2.017504in}}%
\pgfpathlineto{\pgfqpoint{1.939103in}{2.018316in}}%
\pgfpathlineto{\pgfqpoint{1.939091in}{2.017439in}}%
\pgfpathlineto{\pgfqpoint{1.938195in}{2.016449in}}%
\pgfpathlineto{\pgfqpoint{1.938018in}{2.016618in}}%
\pgfpathlineto{\pgfqpoint{1.936644in}{2.016715in}}%
\pgfpathlineto{\pgfqpoint{1.935724in}{2.014036in}}%
\pgfpathlineto{\pgfqpoint{1.935721in}{2.014078in}}%
\pgfpathlineto{\pgfqpoint{1.934498in}{2.013968in}}%
\pgfpathlineto{\pgfqpoint{1.934425in}{2.013103in}}%
\pgfpathlineto{\pgfqpoint{1.933447in}{2.014027in}}%
\pgfpathlineto{\pgfqpoint{1.933447in}{2.014027in}}%
\pgfpathlineto{\pgfqpoint{1.932801in}{2.014783in}}%
\pgfpathlineto{\pgfqpoint{1.932787in}{2.013918in}}%
\pgfpathlineto{\pgfqpoint{1.932787in}{2.013918in}}%
\pgfpathlineto{\pgfqpoint{1.932312in}{2.012655in}}%
\pgfpathlineto{\pgfqpoint{1.931661in}{2.013284in}}%
\pgfpathlineto{\pgfqpoint{1.931570in}{2.013409in}}%
\pgfpathlineto{\pgfqpoint{1.931570in}{2.013409in}}%
\pgfpathlineto{\pgfqpoint{1.931570in}{2.013409in}}%
\pgfpathlineto{\pgfqpoint{1.931534in}{2.011689in}}%
\pgfpathlineto{\pgfqpoint{1.930453in}{2.012652in}}%
\pgfpathlineto{\pgfqpoint{1.930069in}{2.012946in}}%
\pgfpathlineto{\pgfqpoint{1.930069in}{2.012946in}}%
\pgfpathlineto{\pgfqpoint{1.930069in}{2.012946in}}%
\pgfpathlineto{\pgfqpoint{1.929910in}{2.011439in}}%
\pgfpathlineto{\pgfqpoint{1.928952in}{2.012108in}}%
\pgfpathlineto{\pgfqpoint{1.928735in}{2.012317in}}%
\pgfpathlineto{\pgfqpoint{1.928735in}{2.012317in}}%
\pgfpathlineto{\pgfqpoint{1.928735in}{2.012317in}}%
\pgfpathlineto{\pgfqpoint{1.928637in}{2.010690in}}%
\pgfpathlineto{\pgfqpoint{1.927645in}{2.011047in}}%
\pgfpathlineto{\pgfqpoint{1.925722in}{2.010720in}}%
\pgfpathlineto{\pgfqpoint{1.924939in}{2.008700in}}%
\pgfpathlineto{\pgfqpoint{1.924720in}{2.008908in}}%
\pgfpathlineto{\pgfqpoint{1.920175in}{2.009479in}}%
\pgfpathlineto{\pgfqpoint{1.920141in}{2.008637in}}%
\pgfpathlineto{\pgfqpoint{1.919339in}{2.009509in}}%
\pgfpathlineto{\pgfqpoint{1.919339in}{2.009509in}}%
\pgfpathlineto{\pgfqpoint{1.919308in}{2.009633in}}%
\pgfpathlineto{\pgfqpoint{1.919308in}{2.009633in}}%
\pgfpathlineto{\pgfqpoint{1.919308in}{2.009633in}}%
\pgfpathlineto{\pgfqpoint{1.918262in}{2.007311in}}%
\pgfpathlineto{\pgfqpoint{1.918190in}{2.007394in}}%
\pgfpathlineto{\pgfqpoint{1.916723in}{2.007676in}}%
\pgfpathlineto{\pgfqpoint{1.916545in}{2.006840in}}%
\pgfpathlineto{\pgfqpoint{1.915524in}{2.007288in}}%
\pgfpathlineto{\pgfqpoint{1.915052in}{2.007660in}}%
\pgfpathlineto{\pgfqpoint{1.915052in}{2.007660in}}%
\pgfpathlineto{\pgfqpoint{1.915052in}{2.007660in}}%
\pgfpathlineto{\pgfqpoint{1.914214in}{2.006029in}}%
\pgfpathlineto{\pgfqpoint{1.913944in}{2.006484in}}%
\pgfpathlineto{\pgfqpoint{1.913111in}{2.007145in}}%
\pgfpathlineto{\pgfqpoint{1.912684in}{2.006726in}}%
\pgfpathlineto{\pgfqpoint{1.912141in}{2.004777in}}%
\pgfpathlineto{\pgfqpoint{1.911579in}{2.005354in}}%
\pgfpathlineto{\pgfqpoint{1.909889in}{2.006052in}}%
\pgfpathlineto{\pgfqpoint{1.909886in}{2.005226in}}%
\pgfpathlineto{\pgfqpoint{1.908864in}{2.006049in}}%
\pgfpathlineto{\pgfqpoint{1.906358in}{2.006169in}}%
\pgfpathlineto{\pgfqpoint{1.905600in}{2.005140in}}%
\pgfpathlineto{\pgfqpoint{1.905188in}{2.005716in}}%
\pgfpathlineto{\pgfqpoint{1.905112in}{2.005757in}}%
\pgfpathlineto{\pgfqpoint{1.905112in}{2.005757in}}%
\pgfpathlineto{\pgfqpoint{1.905112in}{2.005757in}}%
\pgfpathlineto{\pgfqpoint{1.905066in}{2.004935in}}%
\pgfpathlineto{\pgfqpoint{1.904355in}{2.005798in}}%
\pgfpathlineto{\pgfqpoint{1.904355in}{2.005798in}}%
\pgfpathlineto{\pgfqpoint{1.904084in}{2.006044in}}%
\pgfpathlineto{\pgfqpoint{1.904084in}{2.006044in}}%
\pgfpathlineto{\pgfqpoint{1.904084in}{2.006044in}}%
\pgfpathlineto{\pgfqpoint{1.903310in}{2.004202in}}%
\pgfpathlineto{\pgfqpoint{1.903029in}{2.004407in}}%
\pgfpathlineto{\pgfqpoint{1.902067in}{2.004410in}}%
\pgfpathlineto{\pgfqpoint{1.901068in}{2.002947in}}%
\pgfpathlineto{\pgfqpoint{1.900965in}{2.002988in}}%
\pgfpathlineto{\pgfqpoint{1.894965in}{2.004039in}}%
\pgfpathlineto{\pgfqpoint{1.894011in}{2.001504in}}%
\pgfpathlineto{\pgfqpoint{1.893760in}{2.001871in}}%
\pgfpathlineto{\pgfqpoint{1.893329in}{2.002319in}}%
\pgfpathlineto{\pgfqpoint{1.893329in}{2.002319in}}%
\pgfpathlineto{\pgfqpoint{1.893329in}{2.002319in}}%
\pgfpathlineto{\pgfqpoint{1.892392in}{2.000437in}}%
\pgfpathlineto{\pgfqpoint{1.892189in}{2.000519in}}%
\pgfpathlineto{\pgfqpoint{1.889607in}{2.001031in}}%
\pgfpathlineto{\pgfqpoint{1.889567in}{2.000231in}}%
\pgfpathlineto{\pgfqpoint{1.888509in}{2.001003in}}%
\pgfpathlineto{\pgfqpoint{1.888093in}{2.001368in}}%
\pgfpathlineto{\pgfqpoint{1.888093in}{2.001368in}}%
\pgfpathlineto{\pgfqpoint{1.888093in}{2.001368in}}%
\pgfpathlineto{\pgfqpoint{1.887162in}{1.998869in}}%
\pgfpathlineto{\pgfqpoint{1.886982in}{1.999031in}}%
\pgfpathlineto{\pgfqpoint{1.886957in}{1.999112in}}%
\pgfpathlineto{\pgfqpoint{1.886957in}{1.999112in}}%
\pgfpathlineto{\pgfqpoint{1.886957in}{1.999112in}}%
\pgfpathlineto{\pgfqpoint{1.886220in}{1.995758in}}%
\pgfpathlineto{\pgfqpoint{1.885886in}{1.996082in}}%
\pgfpathlineto{\pgfqpoint{1.884796in}{1.996061in}}%
\pgfpathlineto{\pgfqpoint{1.884498in}{1.993862in}}%
\pgfpathlineto{\pgfqpoint{1.883809in}{1.994548in}}%
\pgfpathlineto{\pgfqpoint{1.877739in}{1.994943in}}%
\pgfpathlineto{\pgfqpoint{1.876635in}{1.991854in}}%
\pgfpathlineto{\pgfqpoint{1.876542in}{1.991934in}}%
\pgfpathlineto{\pgfqpoint{1.876542in}{1.991934in}}%
\pgfpathlineto{\pgfqpoint{1.876542in}{1.991934in}}%
\pgfpathlineto{\pgfqpoint{1.876512in}{1.991159in}}%
\pgfpathlineto{\pgfqpoint{1.875701in}{1.991961in}}%
\pgfpathlineto{\pgfqpoint{1.875701in}{1.991961in}}%
\pgfpathlineto{\pgfqpoint{1.875678in}{1.992001in}}%
\pgfpathlineto{\pgfqpoint{1.875678in}{1.992001in}}%
\pgfpathlineto{\pgfqpoint{1.875678in}{1.992001in}}%
\pgfpathlineto{\pgfqpoint{1.875063in}{1.989431in}}%
\pgfpathlineto{\pgfqpoint{1.874591in}{1.989911in}}%
\pgfpathlineto{\pgfqpoint{1.874476in}{1.989992in}}%
\pgfpathlineto{\pgfqpoint{1.874476in}{1.989992in}}%
\pgfpathlineto{\pgfqpoint{1.874476in}{1.989992in}}%
\pgfpathlineto{\pgfqpoint{1.874158in}{1.988852in}}%
\pgfpathlineto{\pgfqpoint{1.873395in}{1.989412in}}%
\pgfpathlineto{\pgfqpoint{1.872490in}{1.990172in}}%
\pgfpathlineto{\pgfqpoint{1.872116in}{1.989563in}}%
\pgfpathlineto{\pgfqpoint{1.870920in}{1.986534in}}%
\pgfpathlineto{\pgfqpoint{1.870786in}{1.986654in}}%
\pgfpathlineto{\pgfqpoint{1.869863in}{1.987292in}}%
\pgfpathlineto{\pgfqpoint{1.869790in}{1.986569in}}%
\pgfpathlineto{\pgfqpoint{1.869731in}{1.985087in}}%
\pgfpathlineto{\pgfqpoint{1.868696in}{1.986322in}}%
\pgfpathlineto{\pgfqpoint{1.867738in}{1.987437in}}%
\pgfpathlineto{\pgfqpoint{1.867437in}{1.986276in}}%
\pgfpathlineto{\pgfqpoint{1.866524in}{1.984520in}}%
\pgfpathlineto{\pgfqpoint{1.866158in}{1.984838in}}%
\pgfpathlineto{\pgfqpoint{1.865345in}{1.984679in}}%
\pgfpathlineto{\pgfqpoint{1.864912in}{1.983250in}}%
\pgfpathlineto{\pgfqpoint{1.864128in}{1.983726in}}%
\pgfpathlineto{\pgfqpoint{1.863023in}{1.984758in}}%
\pgfpathlineto{\pgfqpoint{1.862761in}{1.984164in}}%
\pgfpathlineto{\pgfqpoint{1.862196in}{1.983295in}}%
\pgfpathlineto{\pgfqpoint{1.861626in}{1.983612in}}%
\pgfpathlineto{\pgfqpoint{1.861573in}{1.983651in}}%
\pgfpathlineto{\pgfqpoint{1.861573in}{1.983651in}}%
\pgfpathlineto{\pgfqpoint{1.861573in}{1.983651in}}%
\pgfpathlineto{\pgfqpoint{1.860730in}{1.982077in}}%
\pgfpathlineto{\pgfqpoint{1.860564in}{1.982195in}}%
\pgfpathlineto{\pgfqpoint{1.860016in}{1.981764in}}%
\pgfpathlineto{\pgfqpoint{1.859555in}{1.980747in}}%
\pgfpathlineto{\pgfqpoint{1.858905in}{1.981537in}}%
\pgfpathlineto{\pgfqpoint{1.858519in}{1.981932in}}%
\pgfpathlineto{\pgfqpoint{1.858366in}{1.981266in}}%
\pgfpathlineto{\pgfqpoint{1.857608in}{1.979472in}}%
\pgfpathlineto{\pgfqpoint{1.857317in}{1.979629in}}%
\pgfpathlineto{\pgfqpoint{1.852033in}{1.981710in}}%
\pgfpathlineto{\pgfqpoint{1.851806in}{1.981168in}}%
\pgfpathlineto{\pgfqpoint{1.851056in}{1.980087in}}%
\pgfpathlineto{\pgfqpoint{1.850648in}{1.980440in}}%
\pgfpathlineto{\pgfqpoint{1.847067in}{1.980826in}}%
\pgfpathlineto{\pgfqpoint{1.846473in}{1.979868in}}%
\pgfpathlineto{\pgfqpoint{1.845963in}{1.980378in}}%
\pgfpathlineto{\pgfqpoint{1.845593in}{1.980770in}}%
\pgfpathlineto{\pgfqpoint{1.845192in}{1.980233in}}%
\pgfpathlineto{\pgfqpoint{1.844187in}{1.978550in}}%
\pgfpathlineto{\pgfqpoint{1.843987in}{1.978668in}}%
\pgfpathlineto{\pgfqpoint{1.843385in}{1.979137in}}%
\pgfpathlineto{\pgfqpoint{1.843152in}{1.978487in}}%
\pgfpathlineto{\pgfqpoint{1.842405in}{1.977501in}}%
\pgfpathlineto{\pgfqpoint{1.842120in}{1.977853in}}%
\pgfpathlineto{\pgfqpoint{1.839981in}{1.978275in}}%
\pgfpathlineto{\pgfqpoint{1.839873in}{1.977706in}}%
\pgfpathlineto{\pgfqpoint{1.838751in}{1.972954in}}%
\pgfpathlineto{\pgfqpoint{1.838523in}{1.973109in}}%
\pgfpathlineto{\pgfqpoint{1.838476in}{1.973148in}}%
\pgfpathlineto{\pgfqpoint{1.838476in}{1.973148in}}%
\pgfpathlineto{\pgfqpoint{1.838476in}{1.973148in}}%
\pgfpathlineto{\pgfqpoint{1.838397in}{1.972432in}}%
\pgfpathlineto{\pgfqpoint{1.837374in}{1.973131in}}%
\pgfpathlineto{\pgfqpoint{1.837200in}{1.973404in}}%
\pgfpathlineto{\pgfqpoint{1.837200in}{1.973404in}}%
\pgfpathlineto{\pgfqpoint{1.837200in}{1.973404in}}%
\pgfpathlineto{\pgfqpoint{1.836299in}{1.972594in}}%
\pgfpathlineto{\pgfqpoint{1.836149in}{1.972672in}}%
\pgfpathlineto{\pgfqpoint{1.831306in}{1.974255in}}%
\pgfpathlineto{\pgfqpoint{1.831225in}{1.973543in}}%
\pgfpathlineto{\pgfqpoint{1.831201in}{1.972831in}}%
\pgfpathlineto{\pgfqpoint{1.830578in}{1.973528in}}%
\pgfpathlineto{\pgfqpoint{1.830578in}{1.973528in}}%
\pgfpathlineto{\pgfqpoint{1.830214in}{1.973993in}}%
\pgfpathlineto{\pgfqpoint{1.830156in}{1.973282in}}%
\pgfpathlineto{\pgfqpoint{1.830156in}{1.973282in}}%
\pgfpathlineto{\pgfqpoint{1.830135in}{1.972572in}}%
\pgfpathlineto{\pgfqpoint{1.829244in}{1.973307in}}%
\pgfpathlineto{\pgfqpoint{1.829244in}{1.973307in}}%
\pgfpathlineto{\pgfqpoint{1.829123in}{1.973384in}}%
\pgfpathlineto{\pgfqpoint{1.829123in}{1.973384in}}%
\pgfpathlineto{\pgfqpoint{1.829123in}{1.973384in}}%
\pgfpathlineto{\pgfqpoint{1.828810in}{1.971966in}}%
\pgfpathlineto{\pgfqpoint{1.827951in}{1.972933in}}%
\pgfpathlineto{\pgfqpoint{1.827654in}{1.973165in}}%
\pgfpathlineto{\pgfqpoint{1.827597in}{1.972534in}}%
\pgfpathlineto{\pgfqpoint{1.826542in}{1.971110in}}%
\pgfpathlineto{\pgfqpoint{1.826448in}{1.971226in}}%
\pgfpathlineto{\pgfqpoint{1.825464in}{1.972075in}}%
\pgfpathlineto{\pgfqpoint{1.825226in}{1.971563in}}%
\pgfpathlineto{\pgfqpoint{1.823615in}{1.972247in}}%
\pgfpathlineto{\pgfqpoint{1.823585in}{1.971544in}}%
\pgfpathlineto{\pgfqpoint{1.822590in}{1.969393in}}%
\pgfpathlineto{\pgfqpoint{1.822497in}{1.969547in}}%
\pgfpathlineto{\pgfqpoint{1.822384in}{1.969585in}}%
\pgfpathlineto{\pgfqpoint{1.822384in}{1.969585in}}%
\pgfpathlineto{\pgfqpoint{1.822384in}{1.969585in}}%
\pgfpathlineto{\pgfqpoint{1.822223in}{1.968342in}}%
\pgfpathlineto{\pgfqpoint{1.821284in}{1.968957in}}%
\pgfpathlineto{\pgfqpoint{1.817375in}{1.969170in}}%
\pgfpathlineto{\pgfqpoint{1.816361in}{1.966468in}}%
\pgfpathlineto{\pgfqpoint{1.816293in}{1.966506in}}%
\pgfpathlineto{\pgfqpoint{1.814311in}{1.966004in}}%
\pgfpathlineto{\pgfqpoint{1.813278in}{1.963978in}}%
\pgfpathlineto{\pgfqpoint{1.813146in}{1.964055in}}%
\pgfpathlineto{\pgfqpoint{1.812850in}{1.964093in}}%
\pgfpathlineto{\pgfqpoint{1.812781in}{1.963407in}}%
\pgfpathlineto{\pgfqpoint{1.812778in}{1.962722in}}%
\pgfpathlineto{\pgfqpoint{1.811976in}{1.963409in}}%
\pgfpathlineto{\pgfqpoint{1.811976in}{1.963409in}}%
\pgfpathlineto{\pgfqpoint{1.811800in}{1.963562in}}%
\pgfpathlineto{\pgfqpoint{1.811800in}{1.963562in}}%
\pgfpathlineto{\pgfqpoint{1.811800in}{1.963562in}}%
\pgfpathlineto{\pgfqpoint{1.810485in}{1.962274in}}%
\pgfpathlineto{\pgfqpoint{1.810406in}{1.962350in}}%
\pgfpathlineto{\pgfqpoint{1.809318in}{1.962240in}}%
\pgfpathlineto{\pgfqpoint{1.809260in}{1.961559in}}%
\pgfpathlineto{\pgfqpoint{1.808262in}{1.962168in}}%
\pgfpathlineto{\pgfqpoint{1.808112in}{1.962244in}}%
\pgfpathlineto{\pgfqpoint{1.808112in}{1.962244in}}%
\pgfpathlineto{\pgfqpoint{1.808112in}{1.962244in}}%
\pgfpathlineto{\pgfqpoint{1.807509in}{1.961417in}}%
\pgfpathlineto{\pgfqpoint{1.807091in}{1.961835in}}%
\pgfpathlineto{\pgfqpoint{1.806378in}{1.961498in}}%
\pgfpathlineto{\pgfqpoint{1.805519in}{1.960074in}}%
\pgfpathlineto{\pgfqpoint{1.805278in}{1.960491in}}%
\pgfpathlineto{\pgfqpoint{1.803158in}{1.960211in}}%
\pgfpathlineto{\pgfqpoint{1.801891in}{1.958277in}}%
\pgfpathlineto{\pgfqpoint{1.801812in}{1.958315in}}%
\pgfpathlineto{\pgfqpoint{1.801162in}{1.958769in}}%
\pgfpathlineto{\pgfqpoint{1.801115in}{1.958098in}}%
\pgfpathlineto{\pgfqpoint{1.800577in}{1.955613in}}%
\pgfpathlineto{\pgfqpoint{1.799714in}{1.956217in}}%
\pgfpathlineto{\pgfqpoint{1.799439in}{1.956482in}}%
\pgfpathlineto{\pgfqpoint{1.799439in}{1.956482in}}%
\pgfpathlineto{\pgfqpoint{1.799439in}{1.956482in}}%
\pgfpathlineto{\pgfqpoint{1.798446in}{1.954461in}}%
\pgfpathlineto{\pgfqpoint{1.798303in}{1.954649in}}%
\pgfpathlineto{\pgfqpoint{1.797964in}{1.955102in}}%
\pgfpathlineto{\pgfqpoint{1.797964in}{1.955102in}}%
\pgfpathlineto{\pgfqpoint{1.797964in}{1.955102in}}%
\pgfpathlineto{\pgfqpoint{1.797088in}{1.951621in}}%
\pgfpathlineto{\pgfqpoint{1.796867in}{1.951771in}}%
\pgfpathlineto{\pgfqpoint{1.796693in}{1.951959in}}%
\pgfpathlineto{\pgfqpoint{1.796693in}{1.951959in}}%
\pgfpathlineto{\pgfqpoint{1.796693in}{1.951959in}}%
\pgfpathlineto{\pgfqpoint{1.796343in}{1.950906in}}%
\pgfpathlineto{\pgfqpoint{1.795575in}{1.951394in}}%
\pgfpathlineto{\pgfqpoint{1.795175in}{1.951657in}}%
\pgfpathlineto{\pgfqpoint{1.794888in}{1.951262in}}%
\pgfpathlineto{\pgfqpoint{1.794888in}{1.951262in}}%
\pgfpathlineto{\pgfqpoint{1.793760in}{1.949728in}}%
\pgfpathlineto{\pgfqpoint{1.793305in}{1.950103in}}%
\pgfpathlineto{\pgfqpoint{1.792752in}{1.949170in}}%
\pgfpathlineto{\pgfqpoint{1.791559in}{1.946166in}}%
\pgfpathlineto{\pgfqpoint{1.791427in}{1.946240in}}%
\pgfpathlineto{\pgfqpoint{1.791237in}{1.946352in}}%
\pgfpathlineto{\pgfqpoint{1.791237in}{1.946352in}}%
\pgfpathlineto{\pgfqpoint{1.791237in}{1.946352in}}%
\pgfpathlineto{\pgfqpoint{1.790419in}{1.944785in}}%
\pgfpathlineto{\pgfqpoint{1.790329in}{1.944972in}}%
\pgfpathlineto{\pgfqpoint{1.789368in}{1.944166in}}%
\pgfpathlineto{\pgfqpoint{1.788537in}{1.943364in}}%
\pgfpathlineto{\pgfqpoint{1.788273in}{1.943513in}}%
\pgfpathlineto{\pgfqpoint{1.787338in}{1.943801in}}%
\pgfpathlineto{\pgfqpoint{1.787072in}{1.943159in}}%
\pgfpathlineto{\pgfqpoint{1.786038in}{1.942398in}}%
\pgfpathlineto{\pgfqpoint{1.785936in}{1.942510in}}%
\pgfpathlineto{\pgfqpoint{1.785732in}{1.942696in}}%
\pgfpathlineto{\pgfqpoint{1.785732in}{1.942696in}}%
\pgfpathlineto{\pgfqpoint{1.785732in}{1.942696in}}%
\pgfpathlineto{\pgfqpoint{1.784991in}{1.941900in}}%
\pgfpathlineto{\pgfqpoint{1.784701in}{1.942123in}}%
\pgfpathlineto{\pgfqpoint{1.781180in}{1.942103in}}%
\pgfpathlineto{\pgfqpoint{1.779848in}{1.940301in}}%
\pgfpathlineto{\pgfqpoint{1.779682in}{1.940375in}}%
\pgfpathlineto{\pgfqpoint{1.776687in}{1.940212in}}%
\pgfpathlineto{\pgfqpoint{1.776111in}{1.939319in}}%
\pgfpathlineto{\pgfqpoint{1.775578in}{1.939762in}}%
\pgfpathlineto{\pgfqpoint{1.774526in}{1.937984in}}%
\pgfpathlineto{\pgfqpoint{1.774423in}{1.938131in}}%
\pgfpathlineto{\pgfqpoint{1.772746in}{1.937983in}}%
\pgfpathlineto{\pgfqpoint{1.772730in}{1.937357in}}%
\pgfpathlineto{\pgfqpoint{1.772121in}{1.937983in}}%
\pgfpathlineto{\pgfqpoint{1.772121in}{1.937983in}}%
\pgfpathlineto{\pgfqpoint{1.771687in}{1.938241in}}%
\pgfpathlineto{\pgfqpoint{1.771687in}{1.938241in}}%
\pgfpathlineto{\pgfqpoint{1.771687in}{1.938241in}}%
\pgfpathlineto{\pgfqpoint{1.771670in}{1.937615in}}%
\pgfpathlineto{\pgfqpoint{1.770713in}{1.938242in}}%
\pgfpathlineto{\pgfqpoint{1.770713in}{1.938242in}}%
\pgfpathlineto{\pgfqpoint{1.770596in}{1.938389in}}%
\pgfpathlineto{\pgfqpoint{1.770596in}{1.938389in}}%
\pgfpathlineto{\pgfqpoint{1.770596in}{1.938389in}}%
\pgfpathlineto{\pgfqpoint{1.769804in}{1.937544in}}%
\pgfpathlineto{\pgfqpoint{1.769461in}{1.937985in}}%
\pgfpathlineto{\pgfqpoint{1.768944in}{1.938354in}}%
\pgfpathlineto{\pgfqpoint{1.768715in}{1.937289in}}%
\pgfpathlineto{\pgfqpoint{1.767942in}{1.935864in}}%
\pgfpathlineto{\pgfqpoint{1.767643in}{1.936195in}}%
\pgfpathlineto{\pgfqpoint{1.766456in}{1.935761in}}%
\pgfpathlineto{\pgfqpoint{1.765409in}{1.934020in}}%
\pgfpathlineto{\pgfqpoint{1.765303in}{1.934057in}}%
\pgfpathlineto{\pgfqpoint{1.763556in}{1.935340in}}%
\pgfpathlineto{\pgfqpoint{1.763424in}{1.934216in}}%
\pgfpathlineto{\pgfqpoint{1.763089in}{1.932480in}}%
\pgfpathlineto{\pgfqpoint{1.762402in}{1.932744in}}%
\pgfpathlineto{\pgfqpoint{1.759912in}{1.933235in}}%
\pgfpathlineto{\pgfqpoint{1.759809in}{1.932659in}}%
\pgfpathlineto{\pgfqpoint{1.759406in}{1.931082in}}%
\pgfpathlineto{\pgfqpoint{1.758648in}{1.931483in}}%
\pgfpathlineto{\pgfqpoint{1.756676in}{1.930896in}}%
\pgfpathlineto{\pgfqpoint{1.755605in}{1.929051in}}%
\pgfpathlineto{\pgfqpoint{1.753398in}{1.929004in}}%
\pgfpathlineto{\pgfqpoint{1.753126in}{1.927943in}}%
\pgfpathlineto{\pgfqpoint{1.752296in}{1.928596in}}%
\pgfpathlineto{\pgfqpoint{1.752152in}{1.928741in}}%
\pgfpathlineto{\pgfqpoint{1.752152in}{1.928741in}}%
\pgfpathlineto{\pgfqpoint{1.752152in}{1.928741in}}%
\pgfpathlineto{\pgfqpoint{1.751802in}{1.926479in}}%
\pgfpathlineto{\pgfqpoint{1.750982in}{1.926713in}}%
\pgfpathlineto{\pgfqpoint{1.750477in}{1.925804in}}%
\pgfpathlineto{\pgfqpoint{1.749884in}{1.926239in}}%
\pgfpathlineto{\pgfqpoint{1.747323in}{1.926471in}}%
\pgfpathlineto{\pgfqpoint{1.746787in}{1.924268in}}%
\pgfpathlineto{\pgfqpoint{1.746066in}{1.924846in}}%
\pgfpathlineto{\pgfqpoint{1.745303in}{1.924019in}}%
\pgfpathlineto{\pgfqpoint{1.744829in}{1.924343in}}%
\pgfpathlineto{\pgfqpoint{1.744791in}{1.924379in}}%
\pgfpathlineto{\pgfqpoint{1.744791in}{1.924379in}}%
\pgfpathlineto{\pgfqpoint{1.744791in}{1.924379in}}%
\pgfpathlineto{\pgfqpoint{1.744200in}{1.923446in}}%
\pgfpathlineto{\pgfqpoint{1.743604in}{1.923807in}}%
\pgfpathlineto{\pgfqpoint{1.742825in}{1.924527in}}%
\pgfpathlineto{\pgfqpoint{1.742625in}{1.924115in}}%
\pgfpathlineto{\pgfqpoint{1.742219in}{1.923596in}}%
\pgfpathlineto{\pgfqpoint{1.741432in}{1.922257in}}%
\pgfpathlineto{\pgfqpoint{1.741151in}{1.922545in}}%
\pgfpathlineto{\pgfqpoint{1.740754in}{1.922725in}}%
\pgfpathlineto{\pgfqpoint{1.740450in}{1.921691in}}%
\pgfpathlineto{\pgfqpoint{1.739641in}{1.919703in}}%
\pgfpathlineto{\pgfqpoint{1.738939in}{1.920170in}}%
\pgfpathlineto{\pgfqpoint{1.738337in}{1.920123in}}%
\pgfpathlineto{\pgfqpoint{1.738157in}{1.919645in}}%
\pgfpathlineto{\pgfqpoint{1.737785in}{1.918182in}}%
\pgfpathlineto{\pgfqpoint{1.737027in}{1.918683in}}%
\pgfpathlineto{\pgfqpoint{1.735711in}{1.919615in}}%
\pgfpathlineto{\pgfqpoint{1.735588in}{1.919651in}}%
\pgfpathlineto{\pgfqpoint{1.735588in}{1.919651in}}%
\pgfpathlineto{\pgfqpoint{1.735588in}{1.919651in}}%
\pgfpathlineto{\pgfqpoint{1.734733in}{1.918513in}}%
\pgfpathlineto{\pgfqpoint{1.734441in}{1.918692in}}%
\pgfpathlineto{\pgfqpoint{1.734152in}{1.918978in}}%
\pgfpathlineto{\pgfqpoint{1.734152in}{1.918978in}}%
\pgfpathlineto{\pgfqpoint{1.734152in}{1.918978in}}%
\pgfpathlineto{\pgfqpoint{1.733870in}{1.918031in}}%
\pgfpathlineto{\pgfqpoint{1.733092in}{1.918425in}}%
\pgfpathlineto{\pgfqpoint{1.731189in}{1.918373in}}%
\pgfpathlineto{\pgfqpoint{1.731149in}{1.917830in}}%
\pgfpathlineto{\pgfqpoint{1.730350in}{1.917174in}}%
\pgfpathlineto{\pgfqpoint{1.728498in}{1.912859in}}%
\pgfpathlineto{\pgfqpoint{1.728328in}{1.912930in}}%
\pgfpathlineto{\pgfqpoint{1.726974in}{1.912285in}}%
\pgfpathlineto{\pgfqpoint{1.726032in}{1.911643in}}%
\pgfpathlineto{\pgfqpoint{1.725814in}{1.911891in}}%
\pgfpathlineto{\pgfqpoint{1.725749in}{1.911927in}}%
\pgfpathlineto{\pgfqpoint{1.725749in}{1.911927in}}%
\pgfpathlineto{\pgfqpoint{1.725749in}{1.911927in}}%
\pgfpathlineto{\pgfqpoint{1.724850in}{1.910718in}}%
\pgfpathlineto{\pgfqpoint{1.724655in}{1.910896in}}%
\pgfpathlineto{\pgfqpoint{1.724594in}{1.910931in}}%
\pgfpathlineto{\pgfqpoint{1.724594in}{1.910931in}}%
\pgfpathlineto{\pgfqpoint{1.724594in}{1.910931in}}%
\pgfpathlineto{\pgfqpoint{1.723396in}{1.907823in}}%
\pgfpathlineto{\pgfqpoint{1.723344in}{1.907893in}}%
\pgfpathlineto{\pgfqpoint{1.723226in}{1.907999in}}%
\pgfpathlineto{\pgfqpoint{1.723226in}{1.907999in}}%
\pgfpathlineto{\pgfqpoint{1.723226in}{1.907999in}}%
\pgfpathlineto{\pgfqpoint{1.723007in}{1.905857in}}%
\pgfpathlineto{\pgfqpoint{1.722126in}{1.906457in}}%
\pgfpathlineto{\pgfqpoint{1.720915in}{1.905937in}}%
\pgfpathlineto{\pgfqpoint{1.719713in}{1.904512in}}%
\pgfpathlineto{\pgfqpoint{1.718816in}{1.903786in}}%
\pgfpathlineto{\pgfqpoint{1.717814in}{1.902233in}}%
\pgfpathlineto{\pgfqpoint{1.717642in}{1.902303in}}%
\pgfpathlineto{\pgfqpoint{1.717506in}{1.902338in}}%
\pgfpathlineto{\pgfqpoint{1.717498in}{1.901784in}}%
\pgfpathlineto{\pgfqpoint{1.717498in}{1.901784in}}%
\pgfpathlineto{\pgfqpoint{1.717089in}{1.900959in}}%
\pgfpathlineto{\pgfqpoint{1.716473in}{1.901275in}}%
\pgfpathlineto{\pgfqpoint{1.712179in}{1.900904in}}%
\pgfpathlineto{\pgfqpoint{1.710883in}{1.898825in}}%
\pgfpathlineto{\pgfqpoint{1.710782in}{1.898930in}}%
\pgfpathlineto{\pgfqpoint{1.709798in}{1.899013in}}%
\pgfpathlineto{\pgfqpoint{1.709747in}{1.898501in}}%
\pgfpathlineto{\pgfqpoint{1.708167in}{1.895493in}}%
\pgfpathlineto{\pgfqpoint{1.707398in}{1.894758in}}%
\pgfpathlineto{\pgfqpoint{1.706592in}{1.893189in}}%
\pgfpathlineto{\pgfqpoint{1.706114in}{1.893537in}}%
\pgfpathlineto{\pgfqpoint{1.706012in}{1.893572in}}%
\pgfpathlineto{\pgfqpoint{1.706012in}{1.893572in}}%
\pgfpathlineto{\pgfqpoint{1.706012in}{1.893572in}}%
\pgfpathlineto{\pgfqpoint{1.704891in}{1.892115in}}%
\pgfpathlineto{\pgfqpoint{1.702545in}{1.891791in}}%
\pgfpathlineto{\pgfqpoint{1.702536in}{1.891255in}}%
\pgfpathlineto{\pgfqpoint{1.701712in}{1.891776in}}%
\pgfpathlineto{\pgfqpoint{1.701712in}{1.891776in}}%
\pgfpathlineto{\pgfqpoint{1.700603in}{1.891504in}}%
\pgfpathlineto{\pgfqpoint{1.700599in}{1.890970in}}%
\pgfpathlineto{\pgfqpoint{1.699253in}{1.888296in}}%
\pgfpathlineto{\pgfqpoint{1.699213in}{1.888331in}}%
\pgfpathlineto{\pgfqpoint{1.697123in}{1.888124in}}%
\pgfpathlineto{\pgfqpoint{1.696352in}{1.886392in}}%
\pgfpathlineto{\pgfqpoint{1.696002in}{1.886668in}}%
\pgfpathlineto{\pgfqpoint{1.694553in}{1.886270in}}%
\pgfpathlineto{\pgfqpoint{1.692680in}{1.885830in}}%
\pgfpathlineto{\pgfqpoint{1.691158in}{1.883895in}}%
\pgfpathlineto{\pgfqpoint{1.690977in}{1.884066in}}%
\pgfpathlineto{\pgfqpoint{1.690913in}{1.884170in}}%
\pgfpathlineto{\pgfqpoint{1.690913in}{1.884170in}}%
\pgfpathlineto{\pgfqpoint{1.690913in}{1.884170in}}%
\pgfpathlineto{\pgfqpoint{1.690106in}{1.882361in}}%
\pgfpathlineto{\pgfqpoint{1.689706in}{1.882636in}}%
\pgfpathlineto{\pgfqpoint{1.688717in}{1.882043in}}%
\pgfpathlineto{\pgfqpoint{1.688704in}{1.881524in}}%
\pgfpathlineto{\pgfqpoint{1.687950in}{1.882039in}}%
\pgfpathlineto{\pgfqpoint{1.687950in}{1.882039in}}%
\pgfpathlineto{\pgfqpoint{1.686553in}{1.881722in}}%
\pgfpathlineto{\pgfqpoint{1.685776in}{1.880651in}}%
\pgfpathlineto{\pgfqpoint{1.685299in}{1.880993in}}%
\pgfpathlineto{\pgfqpoint{1.682026in}{1.880710in}}%
\pgfpathlineto{\pgfqpoint{1.681475in}{1.879340in}}%
\pgfpathlineto{\pgfqpoint{1.680904in}{1.879783in}}%
\pgfpathlineto{\pgfqpoint{1.679918in}{1.879168in}}%
\pgfpathlineto{\pgfqpoint{1.678601in}{1.877703in}}%
\pgfpathlineto{\pgfqpoint{1.676581in}{1.877773in}}%
\pgfpathlineto{\pgfqpoint{1.676559in}{1.877264in}}%
\pgfpathlineto{\pgfqpoint{1.676443in}{1.876281in}}%
\pgfpathlineto{\pgfqpoint{1.675430in}{1.876454in}}%
\pgfpathlineto{\pgfqpoint{1.675338in}{1.876522in}}%
\pgfpathlineto{\pgfqpoint{1.675338in}{1.876522in}}%
\pgfpathlineto{\pgfqpoint{1.675338in}{1.876522in}}%
\pgfpathlineto{\pgfqpoint{1.674176in}{1.875579in}}%
\pgfpathlineto{\pgfqpoint{1.674120in}{1.875613in}}%
\pgfpathlineto{\pgfqpoint{1.673979in}{1.875681in}}%
\pgfpathlineto{\pgfqpoint{1.673979in}{1.875681in}}%
\pgfpathlineto{\pgfqpoint{1.673979in}{1.875681in}}%
\pgfpathlineto{\pgfqpoint{1.672819in}{1.873332in}}%
\pgfpathlineto{\pgfqpoint{1.664946in}{1.874319in}}%
\pgfpathlineto{\pgfqpoint{1.664246in}{1.873724in}}%
\pgfpathlineto{\pgfqpoint{1.663836in}{1.874061in}}%
\pgfpathlineto{\pgfqpoint{1.663687in}{1.874230in}}%
\pgfpathlineto{\pgfqpoint{1.663687in}{1.874230in}}%
\pgfpathlineto{\pgfqpoint{1.663687in}{1.874230in}}%
\pgfpathlineto{\pgfqpoint{1.662371in}{1.873272in}}%
\pgfpathlineto{\pgfqpoint{1.662309in}{1.873475in}}%
\pgfpathlineto{\pgfqpoint{1.662235in}{1.873508in}}%
\pgfpathlineto{\pgfqpoint{1.662235in}{1.873508in}}%
\pgfpathlineto{\pgfqpoint{1.662235in}{1.873508in}}%
\pgfpathlineto{\pgfqpoint{1.660858in}{1.871064in}}%
\pgfpathlineto{\pgfqpoint{1.660766in}{1.871097in}}%
\pgfpathlineto{\pgfqpoint{1.660539in}{1.871367in}}%
\pgfpathlineto{\pgfqpoint{1.660413in}{1.870871in}}%
\pgfpathlineto{\pgfqpoint{1.660413in}{1.870871in}}%
\pgfpathlineto{\pgfqpoint{1.659640in}{1.870184in}}%
\pgfpathlineto{\pgfqpoint{1.659262in}{1.870352in}}%
\pgfpathlineto{\pgfqpoint{1.659175in}{1.870453in}}%
\pgfpathlineto{\pgfqpoint{1.659175in}{1.870453in}}%
\pgfpathlineto{\pgfqpoint{1.659175in}{1.870453in}}%
\pgfpathlineto{\pgfqpoint{1.657646in}{1.869352in}}%
\pgfpathlineto{\pgfqpoint{1.656792in}{1.868771in}}%
\pgfpathlineto{\pgfqpoint{1.656311in}{1.867465in}}%
\pgfpathlineto{\pgfqpoint{1.655661in}{1.867867in}}%
\pgfpathlineto{\pgfqpoint{1.655440in}{1.868035in}}%
\pgfpathlineto{\pgfqpoint{1.655440in}{1.868035in}}%
\pgfpathlineto{\pgfqpoint{1.655440in}{1.868035in}}%
\pgfpathlineto{\pgfqpoint{1.654538in}{1.867323in}}%
\pgfpathlineto{\pgfqpoint{1.654247in}{1.867423in}}%
\pgfpathlineto{\pgfqpoint{1.652647in}{1.867483in}}%
\pgfpathlineto{\pgfqpoint{1.651495in}{1.864640in}}%
\pgfpathlineto{\pgfqpoint{1.648665in}{1.864370in}}%
\pgfpathlineto{\pgfqpoint{1.648008in}{1.863286in}}%
\pgfpathlineto{\pgfqpoint{1.647571in}{1.863519in}}%
\pgfpathlineto{\pgfqpoint{1.647383in}{1.863686in}}%
\pgfpathlineto{\pgfqpoint{1.647342in}{1.863203in}}%
\pgfpathlineto{\pgfqpoint{1.647342in}{1.863203in}}%
\pgfpathlineto{\pgfqpoint{1.646716in}{1.861410in}}%
\pgfpathlineto{\pgfqpoint{1.645931in}{1.861876in}}%
\pgfpathlineto{\pgfqpoint{1.645713in}{1.862109in}}%
\pgfpathlineto{\pgfqpoint{1.645653in}{1.861662in}}%
\pgfpathlineto{\pgfqpoint{1.645653in}{1.861662in}}%
\pgfpathlineto{\pgfqpoint{1.645475in}{1.860702in}}%
\pgfpathlineto{\pgfqpoint{1.644380in}{1.861201in}}%
\pgfpathlineto{\pgfqpoint{1.644276in}{1.861267in}}%
\pgfpathlineto{\pgfqpoint{1.644276in}{1.861267in}}%
\pgfpathlineto{\pgfqpoint{1.644276in}{1.861267in}}%
\pgfpathlineto{\pgfqpoint{1.643539in}{1.859652in}}%
\pgfpathlineto{\pgfqpoint{1.643196in}{1.859784in}}%
\pgfpathlineto{\pgfqpoint{1.641857in}{1.859871in}}%
\pgfpathlineto{\pgfqpoint{1.641775in}{1.859427in}}%
\pgfpathlineto{\pgfqpoint{1.641491in}{1.858131in}}%
\pgfpathlineto{\pgfqpoint{1.640638in}{1.858318in}}%
\pgfpathlineto{\pgfqpoint{1.639787in}{1.856785in}}%
\pgfpathlineto{\pgfqpoint{1.639627in}{1.856884in}}%
\pgfpathlineto{\pgfqpoint{1.638083in}{1.856841in}}%
\pgfpathlineto{\pgfqpoint{1.635076in}{1.857582in}}%
\pgfpathlineto{\pgfqpoint{1.634584in}{1.856770in}}%
\pgfpathlineto{\pgfqpoint{1.633937in}{1.856968in}}%
\pgfpathlineto{\pgfqpoint{1.632770in}{1.856346in}}%
\pgfpathlineto{\pgfqpoint{1.631828in}{1.855061in}}%
\pgfpathlineto{\pgfqpoint{1.631701in}{1.855225in}}%
\pgfpathlineto{\pgfqpoint{1.631335in}{1.855456in}}%
\pgfpathlineto{\pgfqpoint{1.631217in}{1.854585in}}%
\pgfpathlineto{\pgfqpoint{1.630357in}{1.853445in}}%
\pgfpathlineto{\pgfqpoint{1.630011in}{1.853610in}}%
\pgfpathlineto{\pgfqpoint{1.629799in}{1.853708in}}%
\pgfpathlineto{\pgfqpoint{1.629799in}{1.853708in}}%
\pgfpathlineto{\pgfqpoint{1.629799in}{1.853708in}}%
\pgfpathlineto{\pgfqpoint{1.628680in}{1.852271in}}%
\pgfpathlineto{\pgfqpoint{1.628644in}{1.852304in}}%
\pgfpathlineto{\pgfqpoint{1.628506in}{1.852436in}}%
\pgfpathlineto{\pgfqpoint{1.628506in}{1.852436in}}%
\pgfpathlineto{\pgfqpoint{1.628506in}{1.852436in}}%
\pgfpathlineto{\pgfqpoint{1.627884in}{1.850742in}}%
\pgfpathlineto{\pgfqpoint{1.627460in}{1.851037in}}%
\pgfpathlineto{\pgfqpoint{1.627387in}{1.851070in}}%
\pgfpathlineto{\pgfqpoint{1.627387in}{1.851070in}}%
\pgfpathlineto{\pgfqpoint{1.627387in}{1.851070in}}%
\pgfpathlineto{\pgfqpoint{1.625899in}{1.847899in}}%
\pgfpathlineto{\pgfqpoint{1.625659in}{1.848096in}}%
\pgfpathlineto{\pgfqpoint{1.624714in}{1.847602in}}%
\pgfpathlineto{\pgfqpoint{1.624221in}{1.845571in}}%
\pgfpathlineto{\pgfqpoint{1.623624in}{1.845898in}}%
\pgfpathlineto{\pgfqpoint{1.622235in}{1.845605in}}%
\pgfpathlineto{\pgfqpoint{1.621118in}{1.844140in}}%
\pgfpathlineto{\pgfqpoint{1.620593in}{1.843979in}}%
\pgfpathlineto{\pgfqpoint{1.620505in}{1.843557in}}%
\pgfpathlineto{\pgfqpoint{1.619530in}{1.842685in}}%
\pgfpathlineto{\pgfqpoint{1.619311in}{1.842815in}}%
\pgfpathlineto{\pgfqpoint{1.618138in}{1.841945in}}%
\pgfpathlineto{\pgfqpoint{1.617032in}{1.840471in}}%
\pgfpathlineto{\pgfqpoint{1.617031in}{1.840503in}}%
\pgfpathlineto{\pgfqpoint{1.615134in}{1.840188in}}%
\pgfpathlineto{\pgfqpoint{1.615108in}{1.839739in}}%
\pgfpathlineto{\pgfqpoint{1.614247in}{1.838816in}}%
\pgfpathlineto{\pgfqpoint{1.614029in}{1.838978in}}%
\pgfpathlineto{\pgfqpoint{1.613420in}{1.838887in}}%
\pgfpathlineto{\pgfqpoint{1.613382in}{1.838440in}}%
\pgfpathlineto{\pgfqpoint{1.611560in}{1.835178in}}%
\pgfpathlineto{\pgfqpoint{1.610953in}{1.835565in}}%
\pgfpathlineto{\pgfqpoint{1.610829in}{1.835662in}}%
\pgfpathlineto{\pgfqpoint{1.610722in}{1.835316in}}%
\pgfpathlineto{\pgfqpoint{1.610722in}{1.835316in}}%
\pgfpathlineto{\pgfqpoint{1.609690in}{1.833966in}}%
\pgfpathlineto{\pgfqpoint{1.609631in}{1.833998in}}%
\pgfpathlineto{\pgfqpoint{1.608360in}{1.833696in}}%
\pgfpathlineto{\pgfqpoint{1.607265in}{1.831425in}}%
\pgfpathlineto{\pgfqpoint{1.607154in}{1.831521in}}%
\pgfpathlineto{\pgfqpoint{1.603841in}{1.831571in}}%
\pgfpathlineto{\pgfqpoint{1.603790in}{1.831134in}}%
\pgfpathlineto{\pgfqpoint{1.602808in}{1.827701in}}%
\pgfpathlineto{\pgfqpoint{1.602442in}{1.827925in}}%
\pgfpathlineto{\pgfqpoint{1.601023in}{1.827956in}}%
\pgfpathlineto{\pgfqpoint{1.600998in}{1.827523in}}%
\pgfpathlineto{\pgfqpoint{1.600182in}{1.826149in}}%
\pgfpathlineto{\pgfqpoint{1.599755in}{1.826373in}}%
\pgfpathlineto{\pgfqpoint{1.598386in}{1.826119in}}%
\pgfpathlineto{\pgfqpoint{1.598377in}{1.825689in}}%
\pgfpathlineto{\pgfqpoint{1.597792in}{1.826104in}}%
\pgfpathlineto{\pgfqpoint{1.597792in}{1.826104in}}%
\pgfpathlineto{\pgfqpoint{1.597782in}{1.826136in}}%
\pgfpathlineto{\pgfqpoint{1.597782in}{1.826136in}}%
\pgfpathlineto{\pgfqpoint{1.597782in}{1.826136in}}%
\pgfpathlineto{\pgfqpoint{1.597112in}{1.825199in}}%
\pgfpathlineto{\pgfqpoint{1.596831in}{1.825390in}}%
\pgfpathlineto{\pgfqpoint{1.592128in}{1.825257in}}%
\pgfpathlineto{\pgfqpoint{1.591014in}{1.823602in}}%
\pgfpathlineto{\pgfqpoint{1.589787in}{1.822411in}}%
\pgfpathlineto{\pgfqpoint{1.589697in}{1.822443in}}%
\pgfpathlineto{\pgfqpoint{1.589572in}{1.822602in}}%
\pgfpathlineto{\pgfqpoint{1.589572in}{1.822602in}}%
\pgfpathlineto{\pgfqpoint{1.589572in}{1.822602in}}%
\pgfpathlineto{\pgfqpoint{1.588625in}{1.820507in}}%
\pgfpathlineto{\pgfqpoint{1.588389in}{1.820634in}}%
\pgfpathlineto{\pgfqpoint{1.586236in}{1.820320in}}%
\pgfpathlineto{\pgfqpoint{1.583832in}{1.818361in}}%
\pgfpathlineto{\pgfqpoint{1.583442in}{1.818488in}}%
\pgfpathlineto{\pgfqpoint{1.583210in}{1.817683in}}%
\pgfpathlineto{\pgfqpoint{1.582149in}{1.816102in}}%
\pgfpathlineto{\pgfqpoint{1.582050in}{1.816134in}}%
\pgfpathlineto{\pgfqpoint{1.580926in}{1.815328in}}%
\pgfpathlineto{\pgfqpoint{1.579868in}{1.813284in}}%
\pgfpathlineto{\pgfqpoint{1.579792in}{1.813315in}}%
\pgfpathlineto{\pgfqpoint{1.579040in}{1.813787in}}%
\pgfpathlineto{\pgfqpoint{1.578656in}{1.813500in}}%
\pgfpathlineto{\pgfqpoint{1.577453in}{1.811594in}}%
\pgfpathlineto{\pgfqpoint{1.577370in}{1.811657in}}%
\pgfpathlineto{\pgfqpoint{1.576294in}{1.811842in}}%
\pgfpathlineto{\pgfqpoint{1.576264in}{1.811432in}}%
\pgfpathlineto{\pgfqpoint{1.574859in}{1.811208in}}%
\pgfpathlineto{\pgfqpoint{1.573864in}{1.810010in}}%
\pgfpathlineto{\pgfqpoint{1.573690in}{1.810198in}}%
\pgfpathlineto{\pgfqpoint{1.571351in}{1.810070in}}%
\pgfpathlineto{\pgfqpoint{1.570412in}{1.809318in}}%
\pgfpathlineto{\pgfqpoint{1.570067in}{1.809505in}}%
\pgfpathlineto{\pgfqpoint{1.570044in}{1.809536in}}%
\pgfpathlineto{\pgfqpoint{1.570044in}{1.809536in}}%
\pgfpathlineto{\pgfqpoint{1.570044in}{1.809536in}}%
\pgfpathlineto{\pgfqpoint{1.568554in}{1.807789in}}%
\pgfpathlineto{\pgfqpoint{1.568450in}{1.807820in}}%
\pgfpathlineto{\pgfqpoint{1.568357in}{1.807852in}}%
\pgfpathlineto{\pgfqpoint{1.568357in}{1.807852in}}%
\pgfpathlineto{\pgfqpoint{1.568357in}{1.807852in}}%
\pgfpathlineto{\pgfqpoint{1.567282in}{1.806643in}}%
\pgfpathlineto{\pgfqpoint{1.567264in}{1.806674in}}%
\pgfpathlineto{\pgfqpoint{1.563203in}{1.807056in}}%
\pgfpathlineto{\pgfqpoint{1.562069in}{1.805916in}}%
\pgfpathlineto{\pgfqpoint{1.561955in}{1.806009in}}%
\pgfpathlineto{\pgfqpoint{1.559210in}{1.805708in}}%
\pgfpathlineto{\pgfqpoint{1.557653in}{1.801061in}}%
\pgfpathlineto{\pgfqpoint{1.557578in}{1.801153in}}%
\pgfpathlineto{\pgfqpoint{1.557451in}{1.800758in}}%
\pgfpathlineto{\pgfqpoint{1.556529in}{1.801407in}}%
\pgfpathlineto{\pgfqpoint{1.556447in}{1.801500in}}%
\pgfpathlineto{\pgfqpoint{1.556447in}{1.801500in}}%
\pgfpathlineto{\pgfqpoint{1.556447in}{1.801500in}}%
\pgfpathlineto{\pgfqpoint{1.555541in}{1.799928in}}%
\pgfpathlineto{\pgfqpoint{1.555331in}{1.800051in}}%
\pgfpathlineto{\pgfqpoint{1.553993in}{1.799819in}}%
\pgfpathlineto{\pgfqpoint{1.553976in}{1.799426in}}%
\pgfpathlineto{\pgfqpoint{1.553544in}{1.798671in}}%
\pgfpathlineto{\pgfqpoint{1.552814in}{1.798979in}}%
\pgfpathlineto{\pgfqpoint{1.549800in}{1.799397in}}%
\pgfpathlineto{\pgfqpoint{1.549785in}{1.799005in}}%
\pgfpathlineto{\pgfqpoint{1.548448in}{1.796639in}}%
\pgfpathlineto{\pgfqpoint{1.548410in}{1.796700in}}%
\pgfpathlineto{\pgfqpoint{1.547079in}{1.796506in}}%
\pgfpathlineto{\pgfqpoint{1.547042in}{1.796116in}}%
\pgfpathlineto{\pgfqpoint{1.545626in}{1.792117in}}%
\pgfpathlineto{\pgfqpoint{1.545581in}{1.792179in}}%
\pgfpathlineto{\pgfqpoint{1.543179in}{1.792341in}}%
\pgfpathlineto{\pgfqpoint{1.543121in}{1.791957in}}%
\pgfpathlineto{\pgfqpoint{1.542275in}{1.791173in}}%
\pgfpathlineto{\pgfqpoint{1.541965in}{1.791387in}}%
\pgfpathlineto{\pgfqpoint{1.539859in}{1.791277in}}%
\pgfpathlineto{\pgfqpoint{1.538824in}{1.790329in}}%
\pgfpathlineto{\pgfqpoint{1.538769in}{1.790421in}}%
\pgfpathlineto{\pgfqpoint{1.538603in}{1.789658in}}%
\pgfpathlineto{\pgfqpoint{1.537661in}{1.789857in}}%
\pgfpathlineto{\pgfqpoint{1.536559in}{1.789082in}}%
\pgfpathlineto{\pgfqpoint{1.536411in}{1.788323in}}%
\pgfpathlineto{\pgfqpoint{1.535510in}{1.788841in}}%
\pgfpathlineto{\pgfqpoint{1.531835in}{1.788408in}}%
\pgfpathlineto{\pgfqpoint{1.531300in}{1.787865in}}%
\pgfpathlineto{\pgfqpoint{1.530703in}{1.788078in}}%
\pgfpathlineto{\pgfqpoint{1.529669in}{1.787779in}}%
\pgfpathlineto{\pgfqpoint{1.529568in}{1.787432in}}%
\pgfpathlineto{\pgfqpoint{1.528427in}{1.785197in}}%
\pgfpathlineto{\pgfqpoint{1.526988in}{1.785864in}}%
\pgfpathlineto{\pgfqpoint{1.526915in}{1.785489in}}%
\pgfpathlineto{\pgfqpoint{1.525932in}{1.785792in}}%
\pgfpathlineto{\pgfqpoint{1.523617in}{1.785295in}}%
\pgfpathlineto{\pgfqpoint{1.521951in}{1.784660in}}%
\pgfpathlineto{\pgfqpoint{1.521932in}{1.784720in}}%
\pgfpathlineto{\pgfqpoint{1.521888in}{1.784811in}}%
\pgfpathlineto{\pgfqpoint{1.521888in}{1.784811in}}%
\pgfpathlineto{\pgfqpoint{1.521888in}{1.784811in}}%
\pgfpathlineto{\pgfqpoint{1.520650in}{1.783866in}}%
\pgfpathlineto{\pgfqpoint{1.519773in}{1.783265in}}%
\pgfpathlineto{\pgfqpoint{1.519002in}{1.781716in}}%
\pgfpathlineto{\pgfqpoint{1.518559in}{1.781896in}}%
\pgfpathlineto{\pgfqpoint{1.515112in}{1.781475in}}%
\pgfpathlineto{\pgfqpoint{1.514098in}{1.780057in}}%
\pgfpathlineto{\pgfqpoint{1.514000in}{1.780117in}}%
\pgfpathlineto{\pgfqpoint{1.512601in}{1.779467in}}%
\pgfpathlineto{\pgfqpoint{1.511591in}{1.778424in}}%
\pgfpathlineto{\pgfqpoint{1.511550in}{1.778454in}}%
\pgfpathlineto{\pgfqpoint{1.509613in}{1.777870in}}%
\pgfpathlineto{\pgfqpoint{1.508449in}{1.776380in}}%
\pgfpathlineto{\pgfqpoint{1.508400in}{1.776410in}}%
\pgfpathlineto{\pgfqpoint{1.505926in}{1.775857in}}%
\pgfpathlineto{\pgfqpoint{1.505615in}{1.775525in}}%
\pgfpathlineto{\pgfqpoint{1.504840in}{1.776034in}}%
\pgfpathlineto{\pgfqpoint{1.503204in}{1.776117in}}%
\pgfpathlineto{\pgfqpoint{1.503199in}{1.775756in}}%
\pgfpathlineto{\pgfqpoint{1.502753in}{1.776114in}}%
\pgfpathlineto{\pgfqpoint{1.502753in}{1.776114in}}%
\pgfpathlineto{\pgfqpoint{1.501303in}{1.775930in}}%
\pgfpathlineto{\pgfqpoint{1.501279in}{1.775569in}}%
\pgfpathlineto{\pgfqpoint{1.500332in}{1.773738in}}%
\pgfpathlineto{\pgfqpoint{1.500124in}{1.773857in}}%
\pgfpathlineto{\pgfqpoint{1.498994in}{1.772065in}}%
\pgfpathlineto{\pgfqpoint{1.498981in}{1.772095in}}%
\pgfpathlineto{\pgfqpoint{1.496996in}{1.771440in}}%
\pgfpathlineto{\pgfqpoint{1.495854in}{1.769722in}}%
\pgfpathlineto{\pgfqpoint{1.495781in}{1.769752in}}%
\pgfpathlineto{\pgfqpoint{1.495082in}{1.768897in}}%
\pgfpathlineto{\pgfqpoint{1.493938in}{1.768221in}}%
\pgfpathlineto{\pgfqpoint{1.492466in}{1.767577in}}%
\pgfpathlineto{\pgfqpoint{1.491794in}{1.766317in}}%
\pgfpathlineto{\pgfqpoint{1.491429in}{1.766494in}}%
\pgfpathlineto{\pgfqpoint{1.490686in}{1.765678in}}%
\pgfpathlineto{\pgfqpoint{1.489561in}{1.764572in}}%
\pgfpathlineto{\pgfqpoint{1.489337in}{1.764690in}}%
\pgfpathlineto{\pgfqpoint{1.489292in}{1.764779in}}%
\pgfpathlineto{\pgfqpoint{1.489292in}{1.764779in}}%
\pgfpathlineto{\pgfqpoint{1.489292in}{1.764779in}}%
\pgfpathlineto{\pgfqpoint{1.488310in}{1.763648in}}%
\pgfpathlineto{\pgfqpoint{1.488271in}{1.763736in}}%
\pgfpathlineto{\pgfqpoint{1.487567in}{1.763104in}}%
\pgfpathlineto{\pgfqpoint{1.487562in}{1.762756in}}%
\pgfpathlineto{\pgfqpoint{1.486846in}{1.761345in}}%
\pgfpathlineto{\pgfqpoint{1.486352in}{1.761639in}}%
\pgfpathlineto{\pgfqpoint{1.485459in}{1.762110in}}%
\pgfpathlineto{\pgfqpoint{1.485138in}{1.761160in}}%
\pgfpathlineto{\pgfqpoint{1.481756in}{1.760521in}}%
\pgfpathlineto{\pgfqpoint{1.480912in}{1.759437in}}%
\pgfpathlineto{\pgfqpoint{1.480534in}{1.759613in}}%
\pgfpathlineto{\pgfqpoint{1.478140in}{1.759178in}}%
\pgfpathlineto{\pgfqpoint{1.476552in}{1.757223in}}%
\pgfpathlineto{\pgfqpoint{1.476514in}{1.757282in}}%
\pgfpathlineto{\pgfqpoint{1.475865in}{1.756522in}}%
\pgfpathlineto{\pgfqpoint{1.475106in}{1.755513in}}%
\pgfpathlineto{\pgfqpoint{1.474771in}{1.755718in}}%
\pgfpathlineto{\pgfqpoint{1.473430in}{1.755167in}}%
\pgfpathlineto{\pgfqpoint{1.472695in}{1.754531in}}%
\pgfpathlineto{\pgfqpoint{1.472308in}{1.754677in}}%
\pgfpathlineto{\pgfqpoint{1.471417in}{1.754718in}}%
\pgfpathlineto{\pgfqpoint{1.471326in}{1.754409in}}%
\pgfpathlineto{\pgfqpoint{1.471081in}{1.753089in}}%
\pgfpathlineto{\pgfqpoint{1.470207in}{1.753438in}}%
\pgfpathlineto{\pgfqpoint{1.467773in}{1.754121in}}%
\pgfpathlineto{\pgfqpoint{1.467734in}{1.753111in}}%
\pgfpathlineto{\pgfqpoint{1.466366in}{1.752934in}}%
\pgfpathlineto{\pgfqpoint{1.466348in}{1.752992in}}%
\pgfpathlineto{\pgfqpoint{1.465533in}{1.752277in}}%
\pgfpathlineto{\pgfqpoint{1.464770in}{1.751985in}}%
\pgfpathlineto{\pgfqpoint{1.464460in}{1.752101in}}%
\pgfpathlineto{\pgfqpoint{1.463553in}{1.751316in}}%
\pgfpathlineto{\pgfqpoint{1.462624in}{1.749608in}}%
\pgfpathlineto{\pgfqpoint{1.460083in}{1.748934in}}%
\pgfpathlineto{\pgfqpoint{1.459479in}{1.748115in}}%
\pgfpathlineto{\pgfqpoint{1.458934in}{1.748346in}}%
\pgfpathlineto{\pgfqpoint{1.458536in}{1.748606in}}%
\pgfpathlineto{\pgfqpoint{1.458536in}{1.748606in}}%
\pgfpathlineto{\pgfqpoint{1.458536in}{1.748606in}}%
\pgfpathlineto{\pgfqpoint{1.457596in}{1.747962in}}%
\pgfpathlineto{\pgfqpoint{1.457480in}{1.748049in}}%
\pgfpathlineto{\pgfqpoint{1.455276in}{1.747626in}}%
\pgfpathlineto{\pgfqpoint{1.454274in}{1.746001in}}%
\pgfpathlineto{\pgfqpoint{1.453998in}{1.746231in}}%
\pgfpathlineto{\pgfqpoint{1.453958in}{1.746260in}}%
\pgfpathlineto{\pgfqpoint{1.453958in}{1.746260in}}%
\pgfpathlineto{\pgfqpoint{1.453958in}{1.746260in}}%
\pgfpathlineto{\pgfqpoint{1.453923in}{1.745933in}}%
\pgfpathlineto{\pgfqpoint{1.452873in}{1.746278in}}%
\pgfpathlineto{\pgfqpoint{1.451548in}{1.745132in}}%
\pgfpathlineto{\pgfqpoint{1.451516in}{1.745161in}}%
\pgfpathlineto{\pgfqpoint{1.449921in}{1.744700in}}%
\pgfpathlineto{\pgfqpoint{1.449907in}{1.744403in}}%
\pgfpathlineto{\pgfqpoint{1.448978in}{1.744154in}}%
\pgfpathlineto{\pgfqpoint{1.448775in}{1.744240in}}%
\pgfpathlineto{\pgfqpoint{1.448395in}{1.744001in}}%
\pgfpathlineto{\pgfqpoint{1.448335in}{1.743352in}}%
\pgfpathlineto{\pgfqpoint{1.447658in}{1.741903in}}%
\pgfpathlineto{\pgfqpoint{1.447331in}{1.742133in}}%
\pgfpathlineto{\pgfqpoint{1.442147in}{1.741413in}}%
\pgfpathlineto{\pgfqpoint{1.440507in}{1.739358in}}%
\pgfpathlineto{\pgfqpoint{1.439033in}{1.738770in}}%
\pgfpathlineto{\pgfqpoint{1.436845in}{1.737336in}}%
\pgfpathlineto{\pgfqpoint{1.436740in}{1.737479in}}%
\pgfpathlineto{\pgfqpoint{1.434983in}{1.737038in}}%
\pgfpathlineto{\pgfqpoint{1.433931in}{1.735389in}}%
\pgfpathlineto{\pgfqpoint{1.431658in}{1.734748in}}%
\pgfpathlineto{\pgfqpoint{1.431137in}{1.733574in}}%
\pgfpathlineto{\pgfqpoint{1.430669in}{1.733773in}}%
\pgfpathlineto{\pgfqpoint{1.428629in}{1.732599in}}%
\pgfpathlineto{\pgfqpoint{1.427351in}{1.731972in}}%
\pgfpathlineto{\pgfqpoint{1.426786in}{1.731121in}}%
\pgfpathlineto{\pgfqpoint{1.426347in}{1.731290in}}%
\pgfpathlineto{\pgfqpoint{1.421214in}{1.730835in}}%
\pgfpathlineto{\pgfqpoint{1.420652in}{1.730158in}}%
\pgfpathlineto{\pgfqpoint{1.420090in}{1.730469in}}%
\pgfpathlineto{\pgfqpoint{1.417120in}{1.730217in}}%
\pgfpathlineto{\pgfqpoint{1.415954in}{1.729124in}}%
\pgfpathlineto{\pgfqpoint{1.415817in}{1.729208in}}%
\pgfpathlineto{\pgfqpoint{1.412387in}{1.728571in}}%
\pgfpathlineto{\pgfqpoint{1.411477in}{1.727374in}}%
\pgfpathlineto{\pgfqpoint{1.411146in}{1.727598in}}%
\pgfpathlineto{\pgfqpoint{1.407529in}{1.725775in}}%
\pgfpathlineto{\pgfqpoint{1.407499in}{1.725887in}}%
\pgfpathlineto{\pgfqpoint{1.407499in}{1.725887in}}%
\pgfpathlineto{\pgfqpoint{1.407499in}{1.725887in}}%
\pgfpathlineto{\pgfqpoint{1.404965in}{1.723523in}}%
\pgfpathlineto{\pgfqpoint{1.404876in}{1.723635in}}%
\pgfpathlineto{\pgfqpoint{1.402430in}{1.723348in}}%
\pgfpathlineto{\pgfqpoint{1.400801in}{1.722147in}}%
\pgfpathlineto{\pgfqpoint{1.400742in}{1.722203in}}%
\pgfpathlineto{\pgfqpoint{1.397947in}{1.722171in}}%
\pgfpathlineto{\pgfqpoint{1.397924in}{1.721871in}}%
\pgfpathlineto{\pgfqpoint{1.397917in}{1.721571in}}%
\pgfpathlineto{\pgfqpoint{1.397218in}{1.721877in}}%
\pgfpathlineto{\pgfqpoint{1.397218in}{1.721877in}}%
\pgfpathlineto{\pgfqpoint{1.397117in}{1.721933in}}%
\pgfpathlineto{\pgfqpoint{1.397117in}{1.721933in}}%
\pgfpathlineto{\pgfqpoint{1.397117in}{1.721933in}}%
\pgfpathlineto{\pgfqpoint{1.396732in}{1.721500in}}%
\pgfpathlineto{\pgfqpoint{1.396130in}{1.721668in}}%
\pgfpathlineto{\pgfqpoint{1.393456in}{1.720381in}}%
\pgfpathlineto{\pgfqpoint{1.393411in}{1.720436in}}%
\pgfpathlineto{\pgfqpoint{1.392708in}{1.720062in}}%
\pgfpathlineto{\pgfqpoint{1.392654in}{1.719764in}}%
\pgfpathlineto{\pgfqpoint{1.391265in}{1.718369in}}%
\pgfpathlineto{\pgfqpoint{1.391225in}{1.718397in}}%
\pgfpathlineto{\pgfqpoint{1.391157in}{1.718425in}}%
\pgfpathlineto{\pgfqpoint{1.391157in}{1.718425in}}%
\pgfpathlineto{\pgfqpoint{1.391157in}{1.718425in}}%
\pgfpathlineto{\pgfqpoint{1.390165in}{1.717165in}}%
\pgfpathlineto{\pgfqpoint{1.390048in}{1.717249in}}%
\pgfpathlineto{\pgfqpoint{1.389287in}{1.716962in}}%
\pgfpathlineto{\pgfqpoint{1.389173in}{1.716695in}}%
\pgfpathlineto{\pgfqpoint{1.387996in}{1.716013in}}%
\pgfpathlineto{\pgfqpoint{1.387926in}{1.716041in}}%
\pgfpathlineto{\pgfqpoint{1.386929in}{1.716051in}}%
\pgfpathlineto{\pgfqpoint{1.386713in}{1.715812in}}%
\pgfpathlineto{\pgfqpoint{1.385731in}{1.715096in}}%
\pgfpathlineto{\pgfqpoint{1.385459in}{1.715207in}}%
\pgfpathlineto{\pgfqpoint{1.382972in}{1.714973in}}%
\pgfpathlineto{\pgfqpoint{1.382921in}{1.714680in}}%
\pgfpathlineto{\pgfqpoint{1.381586in}{1.714216in}}%
\pgfpathlineto{\pgfqpoint{1.381538in}{1.714244in}}%
\pgfpathlineto{\pgfqpoint{1.379602in}{1.713710in}}%
\pgfpathlineto{\pgfqpoint{1.377126in}{1.713453in}}%
\pgfpathlineto{\pgfqpoint{1.377108in}{1.713163in}}%
\pgfpathlineto{\pgfqpoint{1.376798in}{1.713465in}}%
\pgfpathlineto{\pgfqpoint{1.376798in}{1.713465in}}%
\pgfpathlineto{\pgfqpoint{1.375597in}{1.713352in}}%
\pgfpathlineto{\pgfqpoint{1.373533in}{1.713623in}}%
\pgfpathlineto{\pgfqpoint{1.373465in}{1.712753in}}%
\pgfpathlineto{\pgfqpoint{1.372611in}{1.711636in}}%
\pgfpathlineto{\pgfqpoint{1.372348in}{1.711691in}}%
\pgfpathlineto{\pgfqpoint{1.371190in}{1.712158in}}%
\pgfpathlineto{\pgfqpoint{1.370954in}{1.711319in}}%
\pgfpathlineto{\pgfqpoint{1.369870in}{1.710962in}}%
\pgfpathlineto{\pgfqpoint{1.368351in}{1.710195in}}%
\pgfpathlineto{\pgfqpoint{1.367089in}{1.707701in}}%
\pgfpathlineto{\pgfqpoint{1.361230in}{1.707121in}}%
\pgfpathlineto{\pgfqpoint{1.360558in}{1.705867in}}%
\pgfpathlineto{\pgfqpoint{1.360045in}{1.706085in}}%
\pgfpathlineto{\pgfqpoint{1.358799in}{1.705536in}}%
\pgfpathlineto{\pgfqpoint{1.358214in}{1.704880in}}%
\pgfpathlineto{\pgfqpoint{1.357756in}{1.705179in}}%
\pgfpathlineto{\pgfqpoint{1.356517in}{1.704324in}}%
\pgfpathlineto{\pgfqpoint{1.355280in}{1.702910in}}%
\pgfpathlineto{\pgfqpoint{1.355202in}{1.702964in}}%
\pgfpathlineto{\pgfqpoint{1.352272in}{1.702460in}}%
\pgfpathlineto{\pgfqpoint{1.350844in}{1.701525in}}%
\pgfpathlineto{\pgfqpoint{1.350756in}{1.701606in}}%
\pgfpathlineto{\pgfqpoint{1.350705in}{1.701633in}}%
\pgfpathlineto{\pgfqpoint{1.350705in}{1.701633in}}%
\pgfpathlineto{\pgfqpoint{1.350705in}{1.701633in}}%
\pgfpathlineto{\pgfqpoint{1.349694in}{1.701095in}}%
\pgfpathlineto{\pgfqpoint{1.346791in}{1.700436in}}%
\pgfpathlineto{\pgfqpoint{1.346021in}{1.699902in}}%
\pgfpathlineto{\pgfqpoint{1.345735in}{1.699983in}}%
\pgfpathlineto{\pgfqpoint{1.345370in}{1.699841in}}%
\pgfpathlineto{\pgfqpoint{1.345134in}{1.699065in}}%
\pgfpathlineto{\pgfqpoint{1.343960in}{1.697955in}}%
\pgfpathlineto{\pgfqpoint{1.343950in}{1.697982in}}%
\pgfpathlineto{\pgfqpoint{1.343950in}{1.697982in}}%
\pgfpathlineto{\pgfqpoint{1.343950in}{1.697982in}}%
\pgfpathlineto{\pgfqpoint{1.342736in}{1.696790in}}%
\pgfpathlineto{\pgfqpoint{1.342632in}{1.696870in}}%
\pgfpathlineto{\pgfqpoint{1.340360in}{1.694618in}}%
\pgfpathlineto{\pgfqpoint{1.340229in}{1.694698in}}%
\pgfpathlineto{\pgfqpoint{1.339463in}{1.693769in}}%
\pgfpathlineto{\pgfqpoint{1.336429in}{1.691426in}}%
\pgfpathlineto{\pgfqpoint{1.335041in}{1.691178in}}%
\pgfpathlineto{\pgfqpoint{1.331942in}{1.688325in}}%
\pgfpathlineto{\pgfqpoint{1.331902in}{1.688352in}}%
\pgfpathlineto{\pgfqpoint{1.330848in}{1.687522in}}%
\pgfpathlineto{\pgfqpoint{1.329344in}{1.686828in}}%
\pgfpathlineto{\pgfqpoint{1.327700in}{1.686217in}}%
\pgfpathlineto{\pgfqpoint{1.325922in}{1.686058in}}%
\pgfpathlineto{\pgfqpoint{1.324969in}{1.686112in}}%
\pgfpathlineto{\pgfqpoint{1.322215in}{1.685373in}}%
\pgfpathlineto{\pgfqpoint{1.320694in}{1.684581in}}%
\pgfpathlineto{\pgfqpoint{1.320678in}{1.684635in}}%
\pgfpathlineto{\pgfqpoint{1.319980in}{1.683766in}}%
\pgfpathlineto{\pgfqpoint{1.319579in}{1.683292in}}%
\pgfpathlineto{\pgfqpoint{1.318849in}{1.683451in}}%
\pgfpathlineto{\pgfqpoint{1.317508in}{1.682666in}}%
\pgfpathlineto{\pgfqpoint{1.317277in}{1.682851in}}%
\pgfpathlineto{\pgfqpoint{1.313673in}{1.680929in}}%
\pgfpathlineto{\pgfqpoint{1.310922in}{1.681306in}}%
\pgfpathlineto{\pgfqpoint{1.309962in}{1.680763in}}%
\pgfpathlineto{\pgfqpoint{1.309845in}{1.680789in}}%
\pgfpathlineto{\pgfqpoint{1.305272in}{1.680396in}}%
\pgfpathlineto{\pgfqpoint{1.303989in}{1.678822in}}%
\pgfpathlineto{\pgfqpoint{1.302843in}{1.678363in}}%
\pgfpathlineto{\pgfqpoint{1.302781in}{1.678132in}}%
\pgfpathlineto{\pgfqpoint{1.301889in}{1.677313in}}%
\pgfpathlineto{\pgfqpoint{1.301697in}{1.677444in}}%
\pgfpathlineto{\pgfqpoint{1.300414in}{1.676811in}}%
\pgfpathlineto{\pgfqpoint{1.300411in}{1.676555in}}%
\pgfpathlineto{\pgfqpoint{1.299597in}{1.676023in}}%
\pgfpathlineto{\pgfqpoint{1.299372in}{1.676154in}}%
\pgfpathlineto{\pgfqpoint{1.297892in}{1.675499in}}%
\pgfpathlineto{\pgfqpoint{1.295895in}{1.674572in}}%
\pgfpathlineto{\pgfqpoint{1.293261in}{1.673910in}}%
\pgfpathlineto{\pgfqpoint{1.292900in}{1.673254in}}%
\pgfpathlineto{\pgfqpoint{1.292144in}{1.673359in}}%
\pgfpathlineto{\pgfqpoint{1.290906in}{1.672329in}}%
\pgfpathlineto{\pgfqpoint{1.290825in}{1.672355in}}%
\pgfpathlineto{\pgfqpoint{1.288870in}{1.671060in}}%
\pgfpathlineto{\pgfqpoint{1.287653in}{1.670315in}}%
\pgfpathlineto{\pgfqpoint{1.286380in}{1.669875in}}%
\pgfpathlineto{\pgfqpoint{1.286371in}{1.669901in}}%
\pgfpathlineto{\pgfqpoint{1.281529in}{1.669607in}}%
\pgfpathlineto{\pgfqpoint{1.281223in}{1.668712in}}%
\pgfpathlineto{\pgfqpoint{1.280408in}{1.668998in}}%
\pgfpathlineto{\pgfqpoint{1.277413in}{1.668636in}}%
\pgfpathlineto{\pgfqpoint{1.275644in}{1.667757in}}%
\pgfpathlineto{\pgfqpoint{1.274869in}{1.666947in}}%
\pgfpathlineto{\pgfqpoint{1.274127in}{1.665775in}}%
\pgfpathlineto{\pgfqpoint{1.273766in}{1.665905in}}%
\pgfpathlineto{\pgfqpoint{1.271352in}{1.665177in}}%
\pgfpathlineto{\pgfqpoint{1.269919in}{1.663137in}}%
\pgfpathlineto{\pgfqpoint{1.269805in}{1.663215in}}%
\pgfpathlineto{\pgfqpoint{1.268740in}{1.662676in}}%
\pgfpathlineto{\pgfqpoint{1.245893in}{1.647400in}}%
\pgfpathlineto{\pgfqpoint{1.245875in}{1.647426in}}%
\pgfpathlineto{\pgfqpoint{1.245027in}{1.647446in}}%
\pgfpathlineto{\pgfqpoint{1.244804in}{1.647057in}}%
\pgfpathlineto{\pgfqpoint{1.240624in}{1.646622in}}%
\pgfpathlineto{\pgfqpoint{1.239700in}{1.645613in}}%
\pgfpathlineto{\pgfqpoint{1.239376in}{1.645689in}}%
\pgfpathlineto{\pgfqpoint{1.238189in}{1.645246in}}%
\pgfpathlineto{\pgfqpoint{1.236885in}{1.644091in}}%
\pgfpathlineto{\pgfqpoint{1.235421in}{1.643551in}}%
\pgfpathlineto{\pgfqpoint{1.234306in}{1.642580in}}%
\pgfpathlineto{\pgfqpoint{1.232620in}{1.642069in}}%
\pgfpathlineto{\pgfqpoint{1.231540in}{1.640625in}}%
\pgfpathlineto{\pgfqpoint{1.230259in}{1.639867in}}%
\pgfpathlineto{\pgfqpoint{1.229035in}{1.639060in}}%
\pgfpathlineto{\pgfqpoint{1.228991in}{1.639110in}}%
\pgfpathlineto{\pgfqpoint{1.228220in}{1.639135in}}%
\pgfpathlineto{\pgfqpoint{1.227899in}{1.638608in}}%
\pgfpathlineto{\pgfqpoint{1.225071in}{1.638106in}}%
\pgfpathlineto{\pgfqpoint{1.223835in}{1.637482in}}%
\pgfpathlineto{\pgfqpoint{1.223609in}{1.637682in}}%
\pgfpathlineto{\pgfqpoint{1.223532in}{1.637732in}}%
\pgfpathlineto{\pgfqpoint{1.223532in}{1.637732in}}%
\pgfpathlineto{\pgfqpoint{1.223532in}{1.637732in}}%
\pgfpathlineto{\pgfqpoint{1.221781in}{1.636662in}}%
\pgfpathlineto{\pgfqpoint{1.221743in}{1.636687in}}%
\pgfpathlineto{\pgfqpoint{1.220569in}{1.635843in}}%
\pgfpathlineto{\pgfqpoint{1.218780in}{1.634409in}}%
\pgfpathlineto{\pgfqpoint{1.218652in}{1.634583in}}%
\pgfpathlineto{\pgfqpoint{1.216644in}{1.634367in}}%
\pgfpathlineto{\pgfqpoint{1.215382in}{1.633925in}}%
\pgfpathlineto{\pgfqpoint{1.212978in}{1.633094in}}%
\pgfpathlineto{\pgfqpoint{1.210503in}{1.630506in}}%
\pgfpathlineto{\pgfqpoint{1.207923in}{1.630249in}}%
\pgfpathlineto{\pgfqpoint{1.206487in}{1.629720in}}%
\pgfpathlineto{\pgfqpoint{1.206479in}{1.629745in}}%
\pgfpathlineto{\pgfqpoint{1.204832in}{1.629246in}}%
\pgfpathlineto{\pgfqpoint{1.204809in}{1.629028in}}%
\pgfpathlineto{\pgfqpoint{1.204799in}{1.628810in}}%
\pgfpathlineto{\pgfqpoint{1.204173in}{1.629058in}}%
\pgfpathlineto{\pgfqpoint{1.204173in}{1.629058in}}%
\pgfpathlineto{\pgfqpoint{1.202840in}{1.628725in}}%
\pgfpathlineto{\pgfqpoint{1.190707in}{1.621819in}}%
\pgfpathlineto{\pgfqpoint{1.190011in}{1.620976in}}%
\pgfpathlineto{\pgfqpoint{1.189615in}{1.621074in}}%
\pgfpathlineto{\pgfqpoint{1.188834in}{1.620299in}}%
\pgfpathlineto{\pgfqpoint{1.188827in}{1.620087in}}%
\pgfpathlineto{\pgfqpoint{1.187050in}{1.619469in}}%
\pgfpathlineto{\pgfqpoint{1.187033in}{1.619494in}}%
\pgfpathlineto{\pgfqpoint{1.187033in}{1.619494in}}%
\pgfpathlineto{\pgfqpoint{1.187033in}{1.619494in}}%
\pgfpathlineto{\pgfqpoint{1.185605in}{1.618756in}}%
\pgfpathlineto{\pgfqpoint{1.185552in}{1.618780in}}%
\pgfpathlineto{\pgfqpoint{1.185552in}{1.618780in}}%
\pgfpathlineto{\pgfqpoint{1.185552in}{1.618780in}}%
\pgfpathlineto{\pgfqpoint{1.184743in}{1.618246in}}%
\pgfpathlineto{\pgfqpoint{1.184502in}{1.618319in}}%
\pgfpathlineto{\pgfqpoint{1.182116in}{1.617814in}}%
\pgfpathlineto{\pgfqpoint{1.178228in}{1.616019in}}%
\pgfpathlineto{\pgfqpoint{1.175587in}{1.615314in}}%
\pgfpathlineto{\pgfqpoint{1.173546in}{1.614627in}}%
\pgfpathlineto{\pgfqpoint{1.172224in}{1.613992in}}%
\pgfpathlineto{\pgfqpoint{1.172211in}{1.613785in}}%
\pgfpathlineto{\pgfqpoint{1.171643in}{1.613541in}}%
\pgfpathlineto{\pgfqpoint{1.171437in}{1.613613in}}%
\pgfpathlineto{\pgfqpoint{1.169861in}{1.612665in}}%
\pgfpathlineto{\pgfqpoint{1.169483in}{1.612120in}}%
\pgfpathlineto{\pgfqpoint{1.168668in}{1.612483in}}%
\pgfpathlineto{\pgfqpoint{1.166820in}{1.612035in}}%
\pgfpathlineto{\pgfqpoint{1.166810in}{1.612060in}}%
\pgfpathlineto{\pgfqpoint{1.163946in}{1.611577in}}%
\pgfpathlineto{\pgfqpoint{1.163924in}{1.611396in}}%
\pgfpathlineto{\pgfqpoint{1.162774in}{1.610254in}}%
\pgfpathlineto{\pgfqpoint{1.160642in}{1.609415in}}%
\pgfpathlineto{\pgfqpoint{1.159414in}{1.609142in}}%
\pgfpathlineto{\pgfqpoint{1.159316in}{1.609239in}}%
\pgfpathlineto{\pgfqpoint{1.157875in}{1.608680in}}%
\pgfpathlineto{\pgfqpoint{1.156742in}{1.608419in}}%
\pgfpathlineto{\pgfqpoint{1.156725in}{1.608443in}}%
\pgfpathlineto{\pgfqpoint{1.154799in}{1.607935in}}%
\pgfpathlineto{\pgfqpoint{1.153898in}{1.606713in}}%
\pgfpathlineto{\pgfqpoint{1.153820in}{1.606737in}}%
\pgfpathlineto{\pgfqpoint{1.150570in}{1.606096in}}%
\pgfpathlineto{\pgfqpoint{1.149483in}{1.605026in}}%
\pgfpathlineto{\pgfqpoint{1.148160in}{1.603369in}}%
\pgfpathlineto{\pgfqpoint{1.148133in}{1.603393in}}%
\pgfpathlineto{\pgfqpoint{1.146916in}{1.603282in}}%
\pgfpathlineto{\pgfqpoint{1.144262in}{1.602709in}}%
\pgfpathlineto{\pgfqpoint{1.143325in}{1.602105in}}%
\pgfpathlineto{\pgfqpoint{1.143035in}{1.602248in}}%
\pgfpathlineto{\pgfqpoint{1.140700in}{1.601314in}}%
\pgfpathlineto{\pgfqpoint{1.139543in}{1.600840in}}%
\pgfpathlineto{\pgfqpoint{1.138654in}{1.599973in}}%
\pgfpathlineto{\pgfqpoint{1.127702in}{1.595234in}}%
\pgfpathlineto{\pgfqpoint{1.126522in}{1.594648in}}%
\pgfpathlineto{\pgfqpoint{1.126520in}{1.594455in}}%
\pgfpathlineto{\pgfqpoint{1.124858in}{1.592927in}}%
\pgfpathlineto{\pgfqpoint{1.123098in}{1.591673in}}%
\pgfpathlineto{\pgfqpoint{1.123064in}{1.591696in}}%
\pgfpathlineto{\pgfqpoint{1.118529in}{1.591249in}}%
\pgfpathlineto{\pgfqpoint{1.114737in}{1.589167in}}%
\pgfpathlineto{\pgfqpoint{1.114695in}{1.589190in}}%
\pgfpathlineto{\pgfqpoint{1.114631in}{1.589001in}}%
\pgfpathlineto{\pgfqpoint{1.114631in}{1.589001in}}%
\pgfpathlineto{\pgfqpoint{1.113043in}{1.588003in}}%
\pgfpathlineto{\pgfqpoint{1.108873in}{1.587901in}}%
\pgfpathlineto{\pgfqpoint{1.107748in}{1.587569in}}%
\pgfpathlineto{\pgfqpoint{1.106221in}{1.587614in}}%
\pgfpathlineto{\pgfqpoint{1.106206in}{1.587426in}}%
\pgfpathlineto{\pgfqpoint{1.103404in}{1.587116in}}%
\pgfpathlineto{\pgfqpoint{1.097525in}{1.583425in}}%
\pgfpathlineto{\pgfqpoint{1.096292in}{1.582336in}}%
\pgfpathlineto{\pgfqpoint{1.094215in}{1.581575in}}%
\pgfpathlineto{\pgfqpoint{1.089957in}{1.580042in}}%
\pgfpathlineto{\pgfqpoint{1.087925in}{1.579385in}}%
\pgfpathlineto{\pgfqpoint{1.086693in}{1.578815in}}%
\pgfpathlineto{\pgfqpoint{1.085629in}{1.577859in}}%
\pgfpathlineto{\pgfqpoint{1.083655in}{1.577113in}}%
\pgfpathlineto{\pgfqpoint{1.080417in}{1.576424in}}%
\pgfpathlineto{\pgfqpoint{1.079282in}{1.576134in}}%
\pgfpathlineto{\pgfqpoint{1.077772in}{1.575665in}}%
\pgfpathlineto{\pgfqpoint{1.069285in}{1.571814in}}%
\pgfpathlineto{\pgfqpoint{1.068005in}{1.570974in}}%
\pgfpathlineto{\pgfqpoint{1.066841in}{1.569582in}}%
\pgfpathlineto{\pgfqpoint{1.063944in}{1.568608in}}%
\pgfpathlineto{\pgfqpoint{1.063778in}{1.568653in}}%
\pgfpathlineto{\pgfqpoint{1.063778in}{1.568653in}}%
\pgfpathlineto{\pgfqpoint{1.063778in}{1.568653in}}%
\pgfpathlineto{\pgfqpoint{1.062490in}{1.568108in}}%
\pgfpathlineto{\pgfqpoint{1.058705in}{1.567339in}}%
\pgfpathlineto{\pgfqpoint{1.057767in}{1.567041in}}%
\pgfpathlineto{\pgfqpoint{1.057606in}{1.567155in}}%
\pgfpathlineto{\pgfqpoint{1.056256in}{1.566179in}}%
\pgfpathlineto{\pgfqpoint{1.055123in}{1.564591in}}%
\pgfpathlineto{\pgfqpoint{1.055080in}{1.564614in}}%
\pgfpathlineto{\pgfqpoint{1.052242in}{1.563806in}}%
\pgfpathlineto{\pgfqpoint{1.047871in}{1.561989in}}%
\pgfpathlineto{\pgfqpoint{1.043165in}{1.561147in}}%
\pgfpathlineto{\pgfqpoint{1.041811in}{1.560091in}}%
\pgfpathlineto{\pgfqpoint{1.041749in}{1.560114in}}%
\pgfpathlineto{\pgfqpoint{1.039470in}{1.559346in}}%
\pgfpathlineto{\pgfqpoint{1.035645in}{1.556712in}}%
\pgfpathlineto{\pgfqpoint{1.034666in}{1.555787in}}%
\pgfpathlineto{\pgfqpoint{1.032599in}{1.554900in}}%
\pgfpathlineto{\pgfqpoint{1.031335in}{1.554228in}}%
\pgfpathlineto{\pgfqpoint{1.029768in}{1.552822in}}%
\pgfpathlineto{\pgfqpoint{1.026751in}{1.552259in}}%
\pgfpathlineto{\pgfqpoint{1.024286in}{1.550710in}}%
\pgfpathlineto{\pgfqpoint{1.022682in}{1.549975in}}%
\pgfpathlineto{\pgfqpoint{1.021574in}{1.549680in}}%
\pgfpathlineto{\pgfqpoint{1.021503in}{1.549702in}}%
\pgfpathlineto{\pgfqpoint{1.020153in}{1.548980in}}%
\pgfpathlineto{\pgfqpoint{1.018586in}{1.547118in}}%
\pgfpathlineto{\pgfqpoint{1.018095in}{1.547229in}}%
\pgfpathlineto{\pgfqpoint{1.017805in}{1.546946in}}%
\pgfpathlineto{\pgfqpoint{1.017805in}{1.546946in}}%
\pgfpathlineto{\pgfqpoint{1.016756in}{1.546699in}}%
\pgfpathlineto{\pgfqpoint{1.016705in}{1.546722in}}%
\pgfpathlineto{\pgfqpoint{1.015244in}{1.545792in}}%
\pgfpathlineto{\pgfqpoint{1.013221in}{1.544985in}}%
\pgfpathlineto{\pgfqpoint{1.011336in}{1.543900in}}%
\pgfpathlineto{\pgfqpoint{1.009878in}{1.543835in}}%
\pgfpathlineto{\pgfqpoint{1.006658in}{1.542890in}}%
\pgfpathlineto{\pgfqpoint{1.002605in}{1.540906in}}%
\pgfpathlineto{\pgfqpoint{1.002587in}{1.540929in}}%
\pgfpathlineto{\pgfqpoint{1.000843in}{1.540183in}}%
\pgfpathlineto{\pgfqpoint{0.999655in}{1.539517in}}%
\pgfpathlineto{\pgfqpoint{0.996809in}{1.539337in}}%
\pgfpathlineto{\pgfqpoint{0.996775in}{1.539018in}}%
\pgfpathlineto{\pgfqpoint{0.995682in}{1.538810in}}%
\pgfpathlineto{\pgfqpoint{0.995672in}{1.538832in}}%
\pgfpathlineto{\pgfqpoint{0.992748in}{1.538359in}}%
\pgfpathlineto{\pgfqpoint{0.991475in}{1.537447in}}%
\pgfpathlineto{\pgfqpoint{0.989406in}{1.536759in}}%
\pgfpathlineto{\pgfqpoint{0.988011in}{1.536234in}}%
\pgfpathlineto{\pgfqpoint{0.987976in}{1.536256in}}%
\pgfpathlineto{\pgfqpoint{0.987332in}{1.536274in}}%
\pgfpathlineto{\pgfqpoint{0.987071in}{1.535959in}}%
\pgfpathlineto{\pgfqpoint{0.987071in}{1.535959in}}%
\pgfpathlineto{\pgfqpoint{0.984524in}{1.534066in}}%
\pgfpathlineto{\pgfqpoint{0.984510in}{1.534088in}}%
\pgfpathlineto{\pgfqpoint{0.984510in}{1.534088in}}%
\pgfpathlineto{\pgfqpoint{0.984510in}{1.534088in}}%
\pgfpathlineto{\pgfqpoint{0.983075in}{1.533145in}}%
\pgfpathlineto{\pgfqpoint{0.983073in}{1.533167in}}%
\pgfpathlineto{\pgfqpoint{0.970791in}{1.529796in}}%
\pgfpathlineto{\pgfqpoint{0.968982in}{1.529068in}}%
\pgfpathlineto{\pgfqpoint{0.966858in}{1.528715in}}%
\pgfpathlineto{\pgfqpoint{0.966848in}{1.528736in}}%
\pgfpathlineto{\pgfqpoint{0.964968in}{1.527882in}}%
\pgfpathlineto{\pgfqpoint{0.964117in}{1.527445in}}%
\pgfpathlineto{\pgfqpoint{0.963995in}{1.527466in}}%
\pgfpathlineto{\pgfqpoint{0.959080in}{1.525683in}}%
\pgfpathlineto{\pgfqpoint{0.957666in}{1.525359in}}%
\pgfpathlineto{\pgfqpoint{0.956447in}{1.524517in}}%
\pgfpathlineto{\pgfqpoint{0.954047in}{1.523270in}}%
\pgfpathlineto{\pgfqpoint{0.951215in}{1.521753in}}%
\pgfpathlineto{\pgfqpoint{0.948406in}{1.521309in}}%
\pgfpathlineto{\pgfqpoint{0.947543in}{1.521119in}}%
\pgfpathlineto{\pgfqpoint{0.947248in}{1.521226in}}%
\pgfpathlineto{\pgfqpoint{0.944887in}{1.521229in}}%
\pgfpathlineto{\pgfqpoint{0.943455in}{1.520467in}}%
\pgfpathlineto{\pgfqpoint{0.942375in}{1.519748in}}%
\pgfpathlineto{\pgfqpoint{0.937122in}{1.516514in}}%
\pgfpathlineto{\pgfqpoint{0.935199in}{1.515559in}}%
\pgfpathlineto{\pgfqpoint{0.932652in}{1.514565in}}%
\pgfpathlineto{\pgfqpoint{0.932632in}{1.514587in}}%
\pgfpathlineto{\pgfqpoint{0.930699in}{1.513976in}}%
\pgfpathlineto{\pgfqpoint{0.924152in}{1.510107in}}%
\pgfpathlineto{\pgfqpoint{0.923195in}{1.509209in}}%
\pgfpathlineto{\pgfqpoint{0.921741in}{1.508812in}}%
\pgfpathlineto{\pgfqpoint{0.872437in}{1.488260in}}%
\pgfpathlineto{\pgfqpoint{0.869693in}{1.487529in}}%
\pgfpathlineto{\pgfqpoint{0.868659in}{1.486462in}}%
\pgfpathlineto{\pgfqpoint{0.868587in}{1.486503in}}%
\pgfpathlineto{\pgfqpoint{0.866231in}{1.485971in}}%
\pgfpathlineto{\pgfqpoint{0.866225in}{1.485839in}}%
\pgfpathlineto{\pgfqpoint{0.863263in}{1.483846in}}%
\pgfpathlineto{\pgfqpoint{0.861211in}{1.483410in}}%
\pgfpathlineto{\pgfqpoint{0.853467in}{1.480162in}}%
\pgfpathlineto{\pgfqpoint{0.851171in}{1.479471in}}%
\pgfpathlineto{\pgfqpoint{0.850001in}{1.478822in}}%
\pgfpathlineto{\pgfqpoint{0.849893in}{1.478863in}}%
\pgfpathlineto{\pgfqpoint{0.848131in}{1.478594in}}%
\pgfpathlineto{\pgfqpoint{0.846207in}{1.477854in}}%
\pgfpathlineto{\pgfqpoint{0.846205in}{1.477726in}}%
\pgfpathlineto{\pgfqpoint{0.845281in}{1.477041in}}%
\pgfpathlineto{\pgfqpoint{0.845055in}{1.477102in}}%
\pgfpathlineto{\pgfqpoint{0.841971in}{1.475939in}}%
\pgfpathlineto{\pgfqpoint{0.839622in}{1.475370in}}%
\pgfpathlineto{\pgfqpoint{0.837940in}{1.474433in}}%
\pgfpathlineto{\pgfqpoint{0.837932in}{1.474453in}}%
\pgfpathlineto{\pgfqpoint{0.833123in}{1.473461in}}%
\pgfpathlineto{\pgfqpoint{0.831632in}{1.472597in}}%
\pgfpathlineto{\pgfqpoint{0.829741in}{1.472046in}}%
\pgfpathlineto{\pgfqpoint{0.828720in}{1.471396in}}%
\pgfpathlineto{\pgfqpoint{0.828716in}{1.471271in}}%
\pgfpathlineto{\pgfqpoint{0.826935in}{1.470306in}}%
\pgfpathlineto{\pgfqpoint{0.822964in}{1.469730in}}%
\pgfpathlineto{\pgfqpoint{0.818238in}{1.468990in}}%
\pgfpathlineto{\pgfqpoint{0.817195in}{1.468696in}}%
\pgfpathlineto{\pgfqpoint{0.817123in}{1.468716in}}%
\pgfpathlineto{\pgfqpoint{0.814816in}{1.467863in}}%
\pgfpathlineto{\pgfqpoint{0.810896in}{1.465767in}}%
\pgfpathlineto{\pgfqpoint{0.808962in}{1.464845in}}%
\pgfpathlineto{\pgfqpoint{0.807491in}{1.464251in}}%
\pgfpathlineto{\pgfqpoint{0.805433in}{1.463760in}}%
\pgfpathlineto{\pgfqpoint{0.803150in}{1.462946in}}%
\pgfpathlineto{\pgfqpoint{0.800185in}{1.462738in}}%
\pgfpathlineto{\pgfqpoint{0.762469in}{1.448956in}}%
\pgfpathlineto{\pgfqpoint{0.760199in}{1.447932in}}%
\pgfpathlineto{\pgfqpoint{0.756349in}{1.447068in}}%
\pgfpathlineto{\pgfqpoint{0.754827in}{1.446749in}}%
\pgfpathlineto{\pgfqpoint{0.754027in}{1.445897in}}%
\pgfpathlineto{\pgfqpoint{0.752092in}{1.444769in}}%
\pgfpathlineto{\pgfqpoint{0.749748in}{1.444121in}}%
\pgfpathlineto{\pgfqpoint{0.748357in}{1.443375in}}%
\pgfpathlineto{\pgfqpoint{0.746657in}{1.442619in}}%
\pgfpathlineto{\pgfqpoint{0.745139in}{1.442306in}}%
\pgfpathlineto{\pgfqpoint{0.743909in}{1.441453in}}%
\pgfpathlineto{\pgfqpoint{0.742845in}{1.441181in}}%
\pgfpathlineto{\pgfqpoint{0.741098in}{1.440336in}}%
\pgfpathlineto{\pgfqpoint{0.739814in}{1.439727in}}%
\pgfpathlineto{\pgfqpoint{0.739787in}{1.439746in}}%
\pgfpathlineto{\pgfqpoint{0.736545in}{1.437955in}}%
\pgfpathlineto{\pgfqpoint{0.735510in}{1.437922in}}%
\pgfpathlineto{\pgfqpoint{0.735434in}{1.437811in}}%
\pgfpathlineto{\pgfqpoint{0.733074in}{1.437017in}}%
\pgfpathlineto{\pgfqpoint{0.724151in}{1.434326in}}%
\pgfpathlineto{\pgfqpoint{0.724150in}{1.434345in}}%
\pgfpathlineto{\pgfqpoint{0.722580in}{1.433526in}}%
\pgfpathlineto{\pgfqpoint{0.721722in}{1.432844in}}%
\pgfpathlineto{\pgfqpoint{0.721559in}{1.432901in}}%
\pgfpathlineto{\pgfqpoint{0.698556in}{1.423606in}}%
\pgfpathlineto{\pgfqpoint{0.697412in}{1.423091in}}%
\pgfpathlineto{\pgfqpoint{0.694937in}{1.421666in}}%
\pgfpathlineto{\pgfqpoint{0.691161in}{1.420890in}}%
\pgfpathlineto{\pgfqpoint{0.690012in}{1.420278in}}%
\pgfpathlineto{\pgfqpoint{0.686000in}{1.419002in}}%
\pgfpathlineto{\pgfqpoint{0.676271in}{1.415412in}}%
\pgfpathlineto{\pgfqpoint{0.674704in}{1.414546in}}%
\pgfpathlineto{\pgfqpoint{0.652976in}{1.407182in}}%
\pgfpathlineto{\pgfqpoint{0.651660in}{1.406863in}}%
\pgfpathlineto{\pgfqpoint{0.648962in}{1.406120in}}%
\pgfpathlineto{\pgfqpoint{0.646936in}{1.404989in}}%
\pgfpathlineto{\pgfqpoint{0.645063in}{1.404253in}}%
\pgfpathlineto{\pgfqpoint{0.638713in}{1.402018in}}%
\pgfpathlineto{\pgfqpoint{0.637398in}{1.401581in}}%
\pgfpathlineto{\pgfqpoint{0.634479in}{1.401003in}}%
\pgfpathlineto{\pgfqpoint{0.634476in}{1.400907in}}%
\pgfpathlineto{\pgfqpoint{0.630621in}{1.399447in}}%
\pgfpathlineto{\pgfqpoint{0.627190in}{1.398590in}}%
\pgfpathlineto{\pgfqpoint{0.622564in}{1.396912in}}%
\pgfpathlineto{\pgfqpoint{0.618624in}{1.394917in}}%
\pgfpathlineto{\pgfqpoint{0.616448in}{1.393821in}}%
\pgfpathlineto{\pgfqpoint{0.613651in}{1.393194in}}%
\pgfpathlineto{\pgfqpoint{0.610794in}{1.391786in}}%
\pgfpathlineto{\pgfqpoint{0.609822in}{1.391634in}}%
\pgfpathlineto{\pgfqpoint{0.609764in}{1.391448in}}%
\pgfpathlineto{\pgfqpoint{0.608098in}{1.390962in}}%
\pgfpathlineto{\pgfqpoint{0.593714in}{1.385741in}}%
\pgfpathlineto{\pgfqpoint{0.592314in}{1.385050in}}%
\pgfpathlineto{\pgfqpoint{0.588236in}{1.384376in}}%
\pgfpathlineto{\pgfqpoint{0.568818in}{1.377901in}}%
\pgfpathlineto{\pgfqpoint{0.566509in}{1.376970in}}%
\pgfpathlineto{\pgfqpoint{0.565401in}{1.376550in}}%
\pgfpathlineto{\pgfqpoint{0.565323in}{1.376585in}}%
\pgfpathlineto{\pgfqpoint{0.561507in}{1.375505in}}%
\pgfpathlineto{\pgfqpoint{0.556762in}{1.374603in}}%
\pgfpathlineto{\pgfqpoint{0.549026in}{1.372256in}}%
\pgfpathlineto{\pgfqpoint{0.549026in}{1.372256in}}%
\pgfusepath{stroke}%
\end{pgfscope}%
\begin{pgfscope}%
\pgfpathrectangle{\pgfqpoint{0.559026in}{0.417391in}}{\pgfqpoint{3.222048in}{2.055572in}} %
\pgfusepath{clip}%
\pgfsetrectcap%
\pgfsetroundjoin%
\pgfsetlinewidth{1.003750pt}%
\definecolor{currentstroke}{rgb}{0.000000,0.000000,0.000000}%
\pgfsetstrokecolor{currentstroke}%
\pgfsetdash{}{0pt}%
\pgfpathmoveto{\pgfqpoint{2.655987in}{0.417391in}}%
\pgfpathlineto{\pgfqpoint{2.655987in}{2.472963in}}%
\pgfusepath{stroke}%
\end{pgfscope}%
\begin{pgfscope}%
\pgfsetrectcap%
\pgfsetmiterjoin%
\pgfsetlinewidth{1.003750pt}%
\definecolor{currentstroke}{rgb}{0.000000,0.000000,0.000000}%
\pgfsetstrokecolor{currentstroke}%
\pgfsetdash{}{0pt}%
\pgfpathmoveto{\pgfqpoint{0.559026in}{2.472963in}}%
\pgfpathlineto{\pgfqpoint{3.781074in}{2.472963in}}%
\pgfusepath{stroke}%
\end{pgfscope}%
\begin{pgfscope}%
\pgfsetrectcap%
\pgfsetmiterjoin%
\pgfsetlinewidth{1.003750pt}%
\definecolor{currentstroke}{rgb}{0.000000,0.000000,0.000000}%
\pgfsetstrokecolor{currentstroke}%
\pgfsetdash{}{0pt}%
\pgfpathmoveto{\pgfqpoint{3.781074in}{0.417391in}}%
\pgfpathlineto{\pgfqpoint{3.781074in}{2.472963in}}%
\pgfusepath{stroke}%
\end{pgfscope}%
\begin{pgfscope}%
\pgfsetrectcap%
\pgfsetmiterjoin%
\pgfsetlinewidth{1.003750pt}%
\definecolor{currentstroke}{rgb}{0.000000,0.000000,0.000000}%
\pgfsetstrokecolor{currentstroke}%
\pgfsetdash{}{0pt}%
\pgfpathmoveto{\pgfqpoint{0.559026in}{0.417391in}}%
\pgfpathlineto{\pgfqpoint{3.781074in}{0.417391in}}%
\pgfusepath{stroke}%
\end{pgfscope}%
\begin{pgfscope}%
\pgfsetrectcap%
\pgfsetmiterjoin%
\pgfsetlinewidth{1.003750pt}%
\definecolor{currentstroke}{rgb}{0.000000,0.000000,0.000000}%
\pgfsetstrokecolor{currentstroke}%
\pgfsetdash{}{0pt}%
\pgfpathmoveto{\pgfqpoint{0.559026in}{0.417391in}}%
\pgfpathlineto{\pgfqpoint{0.559026in}{2.472963in}}%
\pgfusepath{stroke}%
\end{pgfscope}%
\begin{pgfscope}%
\pgfsetbuttcap%
\pgfsetroundjoin%
\definecolor{currentfill}{rgb}{0.000000,0.000000,0.000000}%
\pgfsetfillcolor{currentfill}%
\pgfsetlinewidth{0.501875pt}%
\definecolor{currentstroke}{rgb}{0.000000,0.000000,0.000000}%
\pgfsetstrokecolor{currentstroke}%
\pgfsetdash{}{0pt}%
\pgfsys@defobject{currentmarker}{\pgfqpoint{0.000000in}{0.000000in}}{\pgfqpoint{0.000000in}{0.069444in}}{%
\pgfpathmoveto{\pgfqpoint{0.000000in}{0.000000in}}%
\pgfpathlineto{\pgfqpoint{0.000000in}{0.069444in}}%
\pgfusepath{stroke,fill}%
}%
\begin{pgfscope}%
\pgfsys@transformshift{0.559026in}{0.417391in}%
\pgfsys@useobject{currentmarker}{}%
\end{pgfscope}%
\end{pgfscope}%
\begin{pgfscope}%
\pgfsetbuttcap%
\pgfsetroundjoin%
\definecolor{currentfill}{rgb}{0.000000,0.000000,0.000000}%
\pgfsetfillcolor{currentfill}%
\pgfsetlinewidth{0.501875pt}%
\definecolor{currentstroke}{rgb}{0.000000,0.000000,0.000000}%
\pgfsetstrokecolor{currentstroke}%
\pgfsetdash{}{0pt}%
\pgfsys@defobject{currentmarker}{\pgfqpoint{0.000000in}{-0.069444in}}{\pgfqpoint{0.000000in}{0.000000in}}{%
\pgfpathmoveto{\pgfqpoint{0.000000in}{0.000000in}}%
\pgfpathlineto{\pgfqpoint{0.000000in}{-0.069444in}}%
\pgfusepath{stroke,fill}%
}%
\begin{pgfscope}%
\pgfsys@transformshift{0.559026in}{2.472963in}%
\pgfsys@useobject{currentmarker}{}%
\end{pgfscope}%
\end{pgfscope}%
\begin{pgfscope}%
\pgftext[x=0.559026in,y=0.347947in,,top]{\rmfamily\fontsize{8.000000}{9.600000}\selectfont 0}%
\end{pgfscope}%
\begin{pgfscope}%
\pgfsetbuttcap%
\pgfsetroundjoin%
\definecolor{currentfill}{rgb}{0.000000,0.000000,0.000000}%
\pgfsetfillcolor{currentfill}%
\pgfsetlinewidth{0.501875pt}%
\definecolor{currentstroke}{rgb}{0.000000,0.000000,0.000000}%
\pgfsetstrokecolor{currentstroke}%
\pgfsetdash{}{0pt}%
\pgfsys@defobject{currentmarker}{\pgfqpoint{0.000000in}{0.000000in}}{\pgfqpoint{0.000000in}{0.069444in}}{%
\pgfpathmoveto{\pgfqpoint{0.000000in}{0.000000in}}%
\pgfpathlineto{\pgfqpoint{0.000000in}{0.069444in}}%
\pgfusepath{stroke,fill}%
}%
\begin{pgfscope}%
\pgfsys@transformshift{1.096034in}{0.417391in}%
\pgfsys@useobject{currentmarker}{}%
\end{pgfscope}%
\end{pgfscope}%
\begin{pgfscope}%
\pgfsetbuttcap%
\pgfsetroundjoin%
\definecolor{currentfill}{rgb}{0.000000,0.000000,0.000000}%
\pgfsetfillcolor{currentfill}%
\pgfsetlinewidth{0.501875pt}%
\definecolor{currentstroke}{rgb}{0.000000,0.000000,0.000000}%
\pgfsetstrokecolor{currentstroke}%
\pgfsetdash{}{0pt}%
\pgfsys@defobject{currentmarker}{\pgfqpoint{0.000000in}{-0.069444in}}{\pgfqpoint{0.000000in}{0.000000in}}{%
\pgfpathmoveto{\pgfqpoint{0.000000in}{0.000000in}}%
\pgfpathlineto{\pgfqpoint{0.000000in}{-0.069444in}}%
\pgfusepath{stroke,fill}%
}%
\begin{pgfscope}%
\pgfsys@transformshift{1.096034in}{2.472963in}%
\pgfsys@useobject{currentmarker}{}%
\end{pgfscope}%
\end{pgfscope}%
\begin{pgfscope}%
\pgftext[x=1.096034in,y=0.347947in,,top]{\rmfamily\fontsize{8.000000}{9.600000}\selectfont 1}%
\end{pgfscope}%
\begin{pgfscope}%
\pgfsetbuttcap%
\pgfsetroundjoin%
\definecolor{currentfill}{rgb}{0.000000,0.000000,0.000000}%
\pgfsetfillcolor{currentfill}%
\pgfsetlinewidth{0.501875pt}%
\definecolor{currentstroke}{rgb}{0.000000,0.000000,0.000000}%
\pgfsetstrokecolor{currentstroke}%
\pgfsetdash{}{0pt}%
\pgfsys@defobject{currentmarker}{\pgfqpoint{0.000000in}{0.000000in}}{\pgfqpoint{0.000000in}{0.069444in}}{%
\pgfpathmoveto{\pgfqpoint{0.000000in}{0.000000in}}%
\pgfpathlineto{\pgfqpoint{0.000000in}{0.069444in}}%
\pgfusepath{stroke,fill}%
}%
\begin{pgfscope}%
\pgfsys@transformshift{1.633042in}{0.417391in}%
\pgfsys@useobject{currentmarker}{}%
\end{pgfscope}%
\end{pgfscope}%
\begin{pgfscope}%
\pgfsetbuttcap%
\pgfsetroundjoin%
\definecolor{currentfill}{rgb}{0.000000,0.000000,0.000000}%
\pgfsetfillcolor{currentfill}%
\pgfsetlinewidth{0.501875pt}%
\definecolor{currentstroke}{rgb}{0.000000,0.000000,0.000000}%
\pgfsetstrokecolor{currentstroke}%
\pgfsetdash{}{0pt}%
\pgfsys@defobject{currentmarker}{\pgfqpoint{0.000000in}{-0.069444in}}{\pgfqpoint{0.000000in}{0.000000in}}{%
\pgfpathmoveto{\pgfqpoint{0.000000in}{0.000000in}}%
\pgfpathlineto{\pgfqpoint{0.000000in}{-0.069444in}}%
\pgfusepath{stroke,fill}%
}%
\begin{pgfscope}%
\pgfsys@transformshift{1.633042in}{2.472963in}%
\pgfsys@useobject{currentmarker}{}%
\end{pgfscope}%
\end{pgfscope}%
\begin{pgfscope}%
\pgftext[x=1.633042in,y=0.347947in,,top]{\rmfamily\fontsize{8.000000}{9.600000}\selectfont 2}%
\end{pgfscope}%
\begin{pgfscope}%
\pgfsetbuttcap%
\pgfsetroundjoin%
\definecolor{currentfill}{rgb}{0.000000,0.000000,0.000000}%
\pgfsetfillcolor{currentfill}%
\pgfsetlinewidth{0.501875pt}%
\definecolor{currentstroke}{rgb}{0.000000,0.000000,0.000000}%
\pgfsetstrokecolor{currentstroke}%
\pgfsetdash{}{0pt}%
\pgfsys@defobject{currentmarker}{\pgfqpoint{0.000000in}{0.000000in}}{\pgfqpoint{0.000000in}{0.069444in}}{%
\pgfpathmoveto{\pgfqpoint{0.000000in}{0.000000in}}%
\pgfpathlineto{\pgfqpoint{0.000000in}{0.069444in}}%
\pgfusepath{stroke,fill}%
}%
\begin{pgfscope}%
\pgfsys@transformshift{2.170050in}{0.417391in}%
\pgfsys@useobject{currentmarker}{}%
\end{pgfscope}%
\end{pgfscope}%
\begin{pgfscope}%
\pgfsetbuttcap%
\pgfsetroundjoin%
\definecolor{currentfill}{rgb}{0.000000,0.000000,0.000000}%
\pgfsetfillcolor{currentfill}%
\pgfsetlinewidth{0.501875pt}%
\definecolor{currentstroke}{rgb}{0.000000,0.000000,0.000000}%
\pgfsetstrokecolor{currentstroke}%
\pgfsetdash{}{0pt}%
\pgfsys@defobject{currentmarker}{\pgfqpoint{0.000000in}{-0.069444in}}{\pgfqpoint{0.000000in}{0.000000in}}{%
\pgfpathmoveto{\pgfqpoint{0.000000in}{0.000000in}}%
\pgfpathlineto{\pgfqpoint{0.000000in}{-0.069444in}}%
\pgfusepath{stroke,fill}%
}%
\begin{pgfscope}%
\pgfsys@transformshift{2.170050in}{2.472963in}%
\pgfsys@useobject{currentmarker}{}%
\end{pgfscope}%
\end{pgfscope}%
\begin{pgfscope}%
\pgftext[x=2.170050in,y=0.347947in,,top]{\rmfamily\fontsize{8.000000}{9.600000}\selectfont 3}%
\end{pgfscope}%
\begin{pgfscope}%
\pgfsetbuttcap%
\pgfsetroundjoin%
\definecolor{currentfill}{rgb}{0.000000,0.000000,0.000000}%
\pgfsetfillcolor{currentfill}%
\pgfsetlinewidth{0.501875pt}%
\definecolor{currentstroke}{rgb}{0.000000,0.000000,0.000000}%
\pgfsetstrokecolor{currentstroke}%
\pgfsetdash{}{0pt}%
\pgfsys@defobject{currentmarker}{\pgfqpoint{0.000000in}{0.000000in}}{\pgfqpoint{0.000000in}{0.069444in}}{%
\pgfpathmoveto{\pgfqpoint{0.000000in}{0.000000in}}%
\pgfpathlineto{\pgfqpoint{0.000000in}{0.069444in}}%
\pgfusepath{stroke,fill}%
}%
\begin{pgfscope}%
\pgfsys@transformshift{2.707058in}{0.417391in}%
\pgfsys@useobject{currentmarker}{}%
\end{pgfscope}%
\end{pgfscope}%
\begin{pgfscope}%
\pgfsetbuttcap%
\pgfsetroundjoin%
\definecolor{currentfill}{rgb}{0.000000,0.000000,0.000000}%
\pgfsetfillcolor{currentfill}%
\pgfsetlinewidth{0.501875pt}%
\definecolor{currentstroke}{rgb}{0.000000,0.000000,0.000000}%
\pgfsetstrokecolor{currentstroke}%
\pgfsetdash{}{0pt}%
\pgfsys@defobject{currentmarker}{\pgfqpoint{0.000000in}{-0.069444in}}{\pgfqpoint{0.000000in}{0.000000in}}{%
\pgfpathmoveto{\pgfqpoint{0.000000in}{0.000000in}}%
\pgfpathlineto{\pgfqpoint{0.000000in}{-0.069444in}}%
\pgfusepath{stroke,fill}%
}%
\begin{pgfscope}%
\pgfsys@transformshift{2.707058in}{2.472963in}%
\pgfsys@useobject{currentmarker}{}%
\end{pgfscope}%
\end{pgfscope}%
\begin{pgfscope}%
\pgftext[x=2.707058in,y=0.347947in,,top]{\rmfamily\fontsize{8.000000}{9.600000}\selectfont 4}%
\end{pgfscope}%
\begin{pgfscope}%
\pgfsetbuttcap%
\pgfsetroundjoin%
\definecolor{currentfill}{rgb}{0.000000,0.000000,0.000000}%
\pgfsetfillcolor{currentfill}%
\pgfsetlinewidth{0.501875pt}%
\definecolor{currentstroke}{rgb}{0.000000,0.000000,0.000000}%
\pgfsetstrokecolor{currentstroke}%
\pgfsetdash{}{0pt}%
\pgfsys@defobject{currentmarker}{\pgfqpoint{0.000000in}{0.000000in}}{\pgfqpoint{0.000000in}{0.069444in}}{%
\pgfpathmoveto{\pgfqpoint{0.000000in}{0.000000in}}%
\pgfpathlineto{\pgfqpoint{0.000000in}{0.069444in}}%
\pgfusepath{stroke,fill}%
}%
\begin{pgfscope}%
\pgfsys@transformshift{3.244066in}{0.417391in}%
\pgfsys@useobject{currentmarker}{}%
\end{pgfscope}%
\end{pgfscope}%
\begin{pgfscope}%
\pgfsetbuttcap%
\pgfsetroundjoin%
\definecolor{currentfill}{rgb}{0.000000,0.000000,0.000000}%
\pgfsetfillcolor{currentfill}%
\pgfsetlinewidth{0.501875pt}%
\definecolor{currentstroke}{rgb}{0.000000,0.000000,0.000000}%
\pgfsetstrokecolor{currentstroke}%
\pgfsetdash{}{0pt}%
\pgfsys@defobject{currentmarker}{\pgfqpoint{0.000000in}{-0.069444in}}{\pgfqpoint{0.000000in}{0.000000in}}{%
\pgfpathmoveto{\pgfqpoint{0.000000in}{0.000000in}}%
\pgfpathlineto{\pgfqpoint{0.000000in}{-0.069444in}}%
\pgfusepath{stroke,fill}%
}%
\begin{pgfscope}%
\pgfsys@transformshift{3.244066in}{2.472963in}%
\pgfsys@useobject{currentmarker}{}%
\end{pgfscope}%
\end{pgfscope}%
\begin{pgfscope}%
\pgftext[x=3.244066in,y=0.347947in,,top]{\rmfamily\fontsize{8.000000}{9.600000}\selectfont 5}%
\end{pgfscope}%
\begin{pgfscope}%
\pgfsetbuttcap%
\pgfsetroundjoin%
\definecolor{currentfill}{rgb}{0.000000,0.000000,0.000000}%
\pgfsetfillcolor{currentfill}%
\pgfsetlinewidth{0.501875pt}%
\definecolor{currentstroke}{rgb}{0.000000,0.000000,0.000000}%
\pgfsetstrokecolor{currentstroke}%
\pgfsetdash{}{0pt}%
\pgfsys@defobject{currentmarker}{\pgfqpoint{0.000000in}{0.000000in}}{\pgfqpoint{0.000000in}{0.069444in}}{%
\pgfpathmoveto{\pgfqpoint{0.000000in}{0.000000in}}%
\pgfpathlineto{\pgfqpoint{0.000000in}{0.069444in}}%
\pgfusepath{stroke,fill}%
}%
\begin{pgfscope}%
\pgfsys@transformshift{3.781074in}{0.417391in}%
\pgfsys@useobject{currentmarker}{}%
\end{pgfscope}%
\end{pgfscope}%
\begin{pgfscope}%
\pgfsetbuttcap%
\pgfsetroundjoin%
\definecolor{currentfill}{rgb}{0.000000,0.000000,0.000000}%
\pgfsetfillcolor{currentfill}%
\pgfsetlinewidth{0.501875pt}%
\definecolor{currentstroke}{rgb}{0.000000,0.000000,0.000000}%
\pgfsetstrokecolor{currentstroke}%
\pgfsetdash{}{0pt}%
\pgfsys@defobject{currentmarker}{\pgfqpoint{0.000000in}{-0.069444in}}{\pgfqpoint{0.000000in}{0.000000in}}{%
\pgfpathmoveto{\pgfqpoint{0.000000in}{0.000000in}}%
\pgfpathlineto{\pgfqpoint{0.000000in}{-0.069444in}}%
\pgfusepath{stroke,fill}%
}%
\begin{pgfscope}%
\pgfsys@transformshift{3.781074in}{2.472963in}%
\pgfsys@useobject{currentmarker}{}%
\end{pgfscope}%
\end{pgfscope}%
\begin{pgfscope}%
\pgftext[x=3.781074in,y=0.347947in,,top]{\rmfamily\fontsize{8.000000}{9.600000}\selectfont 6}%
\end{pgfscope}%
\begin{pgfscope}%
\pgftext[x=2.170050in,y=0.170972in,,top]{\rmfamily\fontsize{9.000000}{10.800000}\selectfont classifier threshold}%
\end{pgfscope}%
\begin{pgfscope}%
\pgfsetbuttcap%
\pgfsetroundjoin%
\definecolor{currentfill}{rgb}{0.000000,0.000000,0.000000}%
\pgfsetfillcolor{currentfill}%
\pgfsetlinewidth{0.501875pt}%
\definecolor{currentstroke}{rgb}{0.000000,0.000000,0.000000}%
\pgfsetstrokecolor{currentstroke}%
\pgfsetdash{}{0pt}%
\pgfsys@defobject{currentmarker}{\pgfqpoint{0.000000in}{0.000000in}}{\pgfqpoint{0.069444in}{0.000000in}}{%
\pgfpathmoveto{\pgfqpoint{0.000000in}{0.000000in}}%
\pgfpathlineto{\pgfqpoint{0.069444in}{0.000000in}}%
\pgfusepath{stroke,fill}%
}%
\begin{pgfscope}%
\pgfsys@transformshift{0.559026in}{0.417391in}%
\pgfsys@useobject{currentmarker}{}%
\end{pgfscope}%
\end{pgfscope}%
\begin{pgfscope}%
\pgfsetbuttcap%
\pgfsetroundjoin%
\definecolor{currentfill}{rgb}{0.000000,0.000000,0.000000}%
\pgfsetfillcolor{currentfill}%
\pgfsetlinewidth{0.501875pt}%
\definecolor{currentstroke}{rgb}{0.000000,0.000000,0.000000}%
\pgfsetstrokecolor{currentstroke}%
\pgfsetdash{}{0pt}%
\pgfsys@defobject{currentmarker}{\pgfqpoint{-0.069444in}{0.000000in}}{\pgfqpoint{0.000000in}{0.000000in}}{%
\pgfpathmoveto{\pgfqpoint{0.000000in}{0.000000in}}%
\pgfpathlineto{\pgfqpoint{-0.069444in}{0.000000in}}%
\pgfusepath{stroke,fill}%
}%
\begin{pgfscope}%
\pgfsys@transformshift{3.781074in}{0.417391in}%
\pgfsys@useobject{currentmarker}{}%
\end{pgfscope}%
\end{pgfscope}%
\begin{pgfscope}%
\pgftext[x=0.489582in,y=0.417391in,right,]{\rmfamily\fontsize{8.000000}{9.600000}\selectfont 0.00}%
\end{pgfscope}%
\begin{pgfscope}%
\pgfsetbuttcap%
\pgfsetroundjoin%
\definecolor{currentfill}{rgb}{0.000000,0.000000,0.000000}%
\pgfsetfillcolor{currentfill}%
\pgfsetlinewidth{0.501875pt}%
\definecolor{currentstroke}{rgb}{0.000000,0.000000,0.000000}%
\pgfsetstrokecolor{currentstroke}%
\pgfsetdash{}{0pt}%
\pgfsys@defobject{currentmarker}{\pgfqpoint{0.000000in}{0.000000in}}{\pgfqpoint{0.069444in}{0.000000in}}{%
\pgfpathmoveto{\pgfqpoint{0.000000in}{0.000000in}}%
\pgfpathlineto{\pgfqpoint{0.069444in}{0.000000in}}%
\pgfusepath{stroke,fill}%
}%
\begin{pgfscope}%
\pgfsys@transformshift{0.559026in}{0.674337in}%
\pgfsys@useobject{currentmarker}{}%
\end{pgfscope}%
\end{pgfscope}%
\begin{pgfscope}%
\pgfsetbuttcap%
\pgfsetroundjoin%
\definecolor{currentfill}{rgb}{0.000000,0.000000,0.000000}%
\pgfsetfillcolor{currentfill}%
\pgfsetlinewidth{0.501875pt}%
\definecolor{currentstroke}{rgb}{0.000000,0.000000,0.000000}%
\pgfsetstrokecolor{currentstroke}%
\pgfsetdash{}{0pt}%
\pgfsys@defobject{currentmarker}{\pgfqpoint{-0.069444in}{0.000000in}}{\pgfqpoint{0.000000in}{0.000000in}}{%
\pgfpathmoveto{\pgfqpoint{0.000000in}{0.000000in}}%
\pgfpathlineto{\pgfqpoint{-0.069444in}{0.000000in}}%
\pgfusepath{stroke,fill}%
}%
\begin{pgfscope}%
\pgfsys@transformshift{3.781074in}{0.674337in}%
\pgfsys@useobject{currentmarker}{}%
\end{pgfscope}%
\end{pgfscope}%
\begin{pgfscope}%
\pgftext[x=0.489582in,y=0.674337in,right,]{\rmfamily\fontsize{8.000000}{9.600000}\selectfont 0.01}%
\end{pgfscope}%
\begin{pgfscope}%
\pgfsetbuttcap%
\pgfsetroundjoin%
\definecolor{currentfill}{rgb}{0.000000,0.000000,0.000000}%
\pgfsetfillcolor{currentfill}%
\pgfsetlinewidth{0.501875pt}%
\definecolor{currentstroke}{rgb}{0.000000,0.000000,0.000000}%
\pgfsetstrokecolor{currentstroke}%
\pgfsetdash{}{0pt}%
\pgfsys@defobject{currentmarker}{\pgfqpoint{0.000000in}{0.000000in}}{\pgfqpoint{0.069444in}{0.000000in}}{%
\pgfpathmoveto{\pgfqpoint{0.000000in}{0.000000in}}%
\pgfpathlineto{\pgfqpoint{0.069444in}{0.000000in}}%
\pgfusepath{stroke,fill}%
}%
\begin{pgfscope}%
\pgfsys@transformshift{0.559026in}{0.931284in}%
\pgfsys@useobject{currentmarker}{}%
\end{pgfscope}%
\end{pgfscope}%
\begin{pgfscope}%
\pgfsetbuttcap%
\pgfsetroundjoin%
\definecolor{currentfill}{rgb}{0.000000,0.000000,0.000000}%
\pgfsetfillcolor{currentfill}%
\pgfsetlinewidth{0.501875pt}%
\definecolor{currentstroke}{rgb}{0.000000,0.000000,0.000000}%
\pgfsetstrokecolor{currentstroke}%
\pgfsetdash{}{0pt}%
\pgfsys@defobject{currentmarker}{\pgfqpoint{-0.069444in}{0.000000in}}{\pgfqpoint{0.000000in}{0.000000in}}{%
\pgfpathmoveto{\pgfqpoint{0.000000in}{0.000000in}}%
\pgfpathlineto{\pgfqpoint{-0.069444in}{0.000000in}}%
\pgfusepath{stroke,fill}%
}%
\begin{pgfscope}%
\pgfsys@transformshift{3.781074in}{0.931284in}%
\pgfsys@useobject{currentmarker}{}%
\end{pgfscope}%
\end{pgfscope}%
\begin{pgfscope}%
\pgftext[x=0.489582in,y=0.931284in,right,]{\rmfamily\fontsize{8.000000}{9.600000}\selectfont 0.02}%
\end{pgfscope}%
\begin{pgfscope}%
\pgfsetbuttcap%
\pgfsetroundjoin%
\definecolor{currentfill}{rgb}{0.000000,0.000000,0.000000}%
\pgfsetfillcolor{currentfill}%
\pgfsetlinewidth{0.501875pt}%
\definecolor{currentstroke}{rgb}{0.000000,0.000000,0.000000}%
\pgfsetstrokecolor{currentstroke}%
\pgfsetdash{}{0pt}%
\pgfsys@defobject{currentmarker}{\pgfqpoint{0.000000in}{0.000000in}}{\pgfqpoint{0.069444in}{0.000000in}}{%
\pgfpathmoveto{\pgfqpoint{0.000000in}{0.000000in}}%
\pgfpathlineto{\pgfqpoint{0.069444in}{0.000000in}}%
\pgfusepath{stroke,fill}%
}%
\begin{pgfscope}%
\pgfsys@transformshift{0.559026in}{1.188230in}%
\pgfsys@useobject{currentmarker}{}%
\end{pgfscope}%
\end{pgfscope}%
\begin{pgfscope}%
\pgfsetbuttcap%
\pgfsetroundjoin%
\definecolor{currentfill}{rgb}{0.000000,0.000000,0.000000}%
\pgfsetfillcolor{currentfill}%
\pgfsetlinewidth{0.501875pt}%
\definecolor{currentstroke}{rgb}{0.000000,0.000000,0.000000}%
\pgfsetstrokecolor{currentstroke}%
\pgfsetdash{}{0pt}%
\pgfsys@defobject{currentmarker}{\pgfqpoint{-0.069444in}{0.000000in}}{\pgfqpoint{0.000000in}{0.000000in}}{%
\pgfpathmoveto{\pgfqpoint{0.000000in}{0.000000in}}%
\pgfpathlineto{\pgfqpoint{-0.069444in}{0.000000in}}%
\pgfusepath{stroke,fill}%
}%
\begin{pgfscope}%
\pgfsys@transformshift{3.781074in}{1.188230in}%
\pgfsys@useobject{currentmarker}{}%
\end{pgfscope}%
\end{pgfscope}%
\begin{pgfscope}%
\pgftext[x=0.489582in,y=1.188230in,right,]{\rmfamily\fontsize{8.000000}{9.600000}\selectfont 0.03}%
\end{pgfscope}%
\begin{pgfscope}%
\pgfsetbuttcap%
\pgfsetroundjoin%
\definecolor{currentfill}{rgb}{0.000000,0.000000,0.000000}%
\pgfsetfillcolor{currentfill}%
\pgfsetlinewidth{0.501875pt}%
\definecolor{currentstroke}{rgb}{0.000000,0.000000,0.000000}%
\pgfsetstrokecolor{currentstroke}%
\pgfsetdash{}{0pt}%
\pgfsys@defobject{currentmarker}{\pgfqpoint{0.000000in}{0.000000in}}{\pgfqpoint{0.069444in}{0.000000in}}{%
\pgfpathmoveto{\pgfqpoint{0.000000in}{0.000000in}}%
\pgfpathlineto{\pgfqpoint{0.069444in}{0.000000in}}%
\pgfusepath{stroke,fill}%
}%
\begin{pgfscope}%
\pgfsys@transformshift{0.559026in}{1.445177in}%
\pgfsys@useobject{currentmarker}{}%
\end{pgfscope}%
\end{pgfscope}%
\begin{pgfscope}%
\pgfsetbuttcap%
\pgfsetroundjoin%
\definecolor{currentfill}{rgb}{0.000000,0.000000,0.000000}%
\pgfsetfillcolor{currentfill}%
\pgfsetlinewidth{0.501875pt}%
\definecolor{currentstroke}{rgb}{0.000000,0.000000,0.000000}%
\pgfsetstrokecolor{currentstroke}%
\pgfsetdash{}{0pt}%
\pgfsys@defobject{currentmarker}{\pgfqpoint{-0.069444in}{0.000000in}}{\pgfqpoint{0.000000in}{0.000000in}}{%
\pgfpathmoveto{\pgfqpoint{0.000000in}{0.000000in}}%
\pgfpathlineto{\pgfqpoint{-0.069444in}{0.000000in}}%
\pgfusepath{stroke,fill}%
}%
\begin{pgfscope}%
\pgfsys@transformshift{3.781074in}{1.445177in}%
\pgfsys@useobject{currentmarker}{}%
\end{pgfscope}%
\end{pgfscope}%
\begin{pgfscope}%
\pgftext[x=0.489582in,y=1.445177in,right,]{\rmfamily\fontsize{8.000000}{9.600000}\selectfont 0.04}%
\end{pgfscope}%
\begin{pgfscope}%
\pgfsetbuttcap%
\pgfsetroundjoin%
\definecolor{currentfill}{rgb}{0.000000,0.000000,0.000000}%
\pgfsetfillcolor{currentfill}%
\pgfsetlinewidth{0.501875pt}%
\definecolor{currentstroke}{rgb}{0.000000,0.000000,0.000000}%
\pgfsetstrokecolor{currentstroke}%
\pgfsetdash{}{0pt}%
\pgfsys@defobject{currentmarker}{\pgfqpoint{0.000000in}{0.000000in}}{\pgfqpoint{0.069444in}{0.000000in}}{%
\pgfpathmoveto{\pgfqpoint{0.000000in}{0.000000in}}%
\pgfpathlineto{\pgfqpoint{0.069444in}{0.000000in}}%
\pgfusepath{stroke,fill}%
}%
\begin{pgfscope}%
\pgfsys@transformshift{0.559026in}{1.702123in}%
\pgfsys@useobject{currentmarker}{}%
\end{pgfscope}%
\end{pgfscope}%
\begin{pgfscope}%
\pgfsetbuttcap%
\pgfsetroundjoin%
\definecolor{currentfill}{rgb}{0.000000,0.000000,0.000000}%
\pgfsetfillcolor{currentfill}%
\pgfsetlinewidth{0.501875pt}%
\definecolor{currentstroke}{rgb}{0.000000,0.000000,0.000000}%
\pgfsetstrokecolor{currentstroke}%
\pgfsetdash{}{0pt}%
\pgfsys@defobject{currentmarker}{\pgfqpoint{-0.069444in}{0.000000in}}{\pgfqpoint{0.000000in}{0.000000in}}{%
\pgfpathmoveto{\pgfqpoint{0.000000in}{0.000000in}}%
\pgfpathlineto{\pgfqpoint{-0.069444in}{0.000000in}}%
\pgfusepath{stroke,fill}%
}%
\begin{pgfscope}%
\pgfsys@transformshift{3.781074in}{1.702123in}%
\pgfsys@useobject{currentmarker}{}%
\end{pgfscope}%
\end{pgfscope}%
\begin{pgfscope}%
\pgftext[x=0.489582in,y=1.702123in,right,]{\rmfamily\fontsize{8.000000}{9.600000}\selectfont 0.05}%
\end{pgfscope}%
\begin{pgfscope}%
\pgfsetbuttcap%
\pgfsetroundjoin%
\definecolor{currentfill}{rgb}{0.000000,0.000000,0.000000}%
\pgfsetfillcolor{currentfill}%
\pgfsetlinewidth{0.501875pt}%
\definecolor{currentstroke}{rgb}{0.000000,0.000000,0.000000}%
\pgfsetstrokecolor{currentstroke}%
\pgfsetdash{}{0pt}%
\pgfsys@defobject{currentmarker}{\pgfqpoint{0.000000in}{0.000000in}}{\pgfqpoint{0.069444in}{0.000000in}}{%
\pgfpathmoveto{\pgfqpoint{0.000000in}{0.000000in}}%
\pgfpathlineto{\pgfqpoint{0.069444in}{0.000000in}}%
\pgfusepath{stroke,fill}%
}%
\begin{pgfscope}%
\pgfsys@transformshift{0.559026in}{1.959070in}%
\pgfsys@useobject{currentmarker}{}%
\end{pgfscope}%
\end{pgfscope}%
\begin{pgfscope}%
\pgfsetbuttcap%
\pgfsetroundjoin%
\definecolor{currentfill}{rgb}{0.000000,0.000000,0.000000}%
\pgfsetfillcolor{currentfill}%
\pgfsetlinewidth{0.501875pt}%
\definecolor{currentstroke}{rgb}{0.000000,0.000000,0.000000}%
\pgfsetstrokecolor{currentstroke}%
\pgfsetdash{}{0pt}%
\pgfsys@defobject{currentmarker}{\pgfqpoint{-0.069444in}{0.000000in}}{\pgfqpoint{0.000000in}{0.000000in}}{%
\pgfpathmoveto{\pgfqpoint{0.000000in}{0.000000in}}%
\pgfpathlineto{\pgfqpoint{-0.069444in}{0.000000in}}%
\pgfusepath{stroke,fill}%
}%
\begin{pgfscope}%
\pgfsys@transformshift{3.781074in}{1.959070in}%
\pgfsys@useobject{currentmarker}{}%
\end{pgfscope}%
\end{pgfscope}%
\begin{pgfscope}%
\pgftext[x=0.489582in,y=1.959070in,right,]{\rmfamily\fontsize{8.000000}{9.600000}\selectfont 0.06}%
\end{pgfscope}%
\begin{pgfscope}%
\pgfsetbuttcap%
\pgfsetroundjoin%
\definecolor{currentfill}{rgb}{0.000000,0.000000,0.000000}%
\pgfsetfillcolor{currentfill}%
\pgfsetlinewidth{0.501875pt}%
\definecolor{currentstroke}{rgb}{0.000000,0.000000,0.000000}%
\pgfsetstrokecolor{currentstroke}%
\pgfsetdash{}{0pt}%
\pgfsys@defobject{currentmarker}{\pgfqpoint{0.000000in}{0.000000in}}{\pgfqpoint{0.069444in}{0.000000in}}{%
\pgfpathmoveto{\pgfqpoint{0.000000in}{0.000000in}}%
\pgfpathlineto{\pgfqpoint{0.069444in}{0.000000in}}%
\pgfusepath{stroke,fill}%
}%
\begin{pgfscope}%
\pgfsys@transformshift{0.559026in}{2.216016in}%
\pgfsys@useobject{currentmarker}{}%
\end{pgfscope}%
\end{pgfscope}%
\begin{pgfscope}%
\pgfsetbuttcap%
\pgfsetroundjoin%
\definecolor{currentfill}{rgb}{0.000000,0.000000,0.000000}%
\pgfsetfillcolor{currentfill}%
\pgfsetlinewidth{0.501875pt}%
\definecolor{currentstroke}{rgb}{0.000000,0.000000,0.000000}%
\pgfsetstrokecolor{currentstroke}%
\pgfsetdash{}{0pt}%
\pgfsys@defobject{currentmarker}{\pgfqpoint{-0.069444in}{0.000000in}}{\pgfqpoint{0.000000in}{0.000000in}}{%
\pgfpathmoveto{\pgfqpoint{0.000000in}{0.000000in}}%
\pgfpathlineto{\pgfqpoint{-0.069444in}{0.000000in}}%
\pgfusepath{stroke,fill}%
}%
\begin{pgfscope}%
\pgfsys@transformshift{3.781074in}{2.216016in}%
\pgfsys@useobject{currentmarker}{}%
\end{pgfscope}%
\end{pgfscope}%
\begin{pgfscope}%
\pgftext[x=0.489582in,y=2.216016in,right,]{\rmfamily\fontsize{8.000000}{9.600000}\selectfont 0.07}%
\end{pgfscope}%
\begin{pgfscope}%
\pgfsetbuttcap%
\pgfsetroundjoin%
\definecolor{currentfill}{rgb}{0.000000,0.000000,0.000000}%
\pgfsetfillcolor{currentfill}%
\pgfsetlinewidth{0.501875pt}%
\definecolor{currentstroke}{rgb}{0.000000,0.000000,0.000000}%
\pgfsetstrokecolor{currentstroke}%
\pgfsetdash{}{0pt}%
\pgfsys@defobject{currentmarker}{\pgfqpoint{0.000000in}{0.000000in}}{\pgfqpoint{0.069444in}{0.000000in}}{%
\pgfpathmoveto{\pgfqpoint{0.000000in}{0.000000in}}%
\pgfpathlineto{\pgfqpoint{0.069444in}{0.000000in}}%
\pgfusepath{stroke,fill}%
}%
\begin{pgfscope}%
\pgfsys@transformshift{0.559026in}{2.472963in}%
\pgfsys@useobject{currentmarker}{}%
\end{pgfscope}%
\end{pgfscope}%
\begin{pgfscope}%
\pgfsetbuttcap%
\pgfsetroundjoin%
\definecolor{currentfill}{rgb}{0.000000,0.000000,0.000000}%
\pgfsetfillcolor{currentfill}%
\pgfsetlinewidth{0.501875pt}%
\definecolor{currentstroke}{rgb}{0.000000,0.000000,0.000000}%
\pgfsetstrokecolor{currentstroke}%
\pgfsetdash{}{0pt}%
\pgfsys@defobject{currentmarker}{\pgfqpoint{-0.069444in}{0.000000in}}{\pgfqpoint{0.000000in}{0.000000in}}{%
\pgfpathmoveto{\pgfqpoint{0.000000in}{0.000000in}}%
\pgfpathlineto{\pgfqpoint{-0.069444in}{0.000000in}}%
\pgfusepath{stroke,fill}%
}%
\begin{pgfscope}%
\pgfsys@transformshift{3.781074in}{2.472963in}%
\pgfsys@useobject{currentmarker}{}%
\end{pgfscope}%
\end{pgfscope}%
\begin{pgfscope}%
\pgftext[x=0.489582in,y=2.472963in,right,]{\rmfamily\fontsize{8.000000}{9.600000}\selectfont 0.08}%
\end{pgfscope}%
\begin{pgfscope}%
\pgftext[x=0.172742in,y=1.445177in,,bottom,rotate=90.000000]{\rmfamily\fontsize{9.000000}{10.800000}\selectfont Punzi figure of merit (a=3)}%
\end{pgfscope}%
\end{pgfpicture}%
\makeatother%
\endgroup%

  \caption{
    Plot of the Punzi figure of merit with $a = 3$ depending on the chosen threshold on the classifier response.
    The value that maximises the FOM is found to be $3.905$ (marked in black).
  }
  \label{fig:fom}
\end{figure}

\begin{figure}	
	\centering
	\begin{subfigure}[t]{0.49\textwidth}
		\centering
    %\includegraphics[width=\textwidth]{store/variables/SIG_BKG_B_DiraAngle.pdf}
    %% Creator: Matplotlib, PGF backend
%%
%% To include the figure in your LaTeX document, write
%%   \input{<filename>.pgf}
%%
%% Make sure the required packages are loaded in your preamble
%%   \usepackage{pgf}
%%
%% Figures using additional raster images can only be included by \input if
%% they are in the same directory as the main LaTeX file. For loading figures
%% from other directories you can use the `import` package
%%   \usepackage{import}
%% and then include the figures with
%%   \import{<path to file>}{<filename>.pgf}
%%
%% Matplotlib used the following preamble
%%   \usepackage{fontspec}
%%   \setmainfont{DejaVu Serif}
%%   \setsansfont{DejaVu Sans}
%%   \setmonofont{DejaVu Sans Mono}
%%
\begingroup%
\makeatletter%
\begin{pgfpicture}%
\pgfpathrectangle{\pgfpointorigin}{\pgfqpoint{2.676066in}{1.736191in}}%
\pgfusepath{use as bounding box, clip}%
\begin{pgfscope}%
\pgfsetbuttcap%
\pgfsetmiterjoin%
\definecolor{currentfill}{rgb}{1.000000,1.000000,1.000000}%
\pgfsetfillcolor{currentfill}%
\pgfsetlinewidth{0.000000pt}%
\definecolor{currentstroke}{rgb}{1.000000,1.000000,1.000000}%
\pgfsetstrokecolor{currentstroke}%
\pgfsetdash{}{0pt}%
\pgfpathmoveto{\pgfqpoint{0.000000in}{0.000000in}}%
\pgfpathlineto{\pgfqpoint{2.676066in}{0.000000in}}%
\pgfpathlineto{\pgfqpoint{2.676066in}{1.736191in}}%
\pgfpathlineto{\pgfqpoint{0.000000in}{1.736191in}}%
\pgfpathclose%
\pgfusepath{fill}%
\end{pgfscope}%
\begin{pgfscope}%
\pgfsetbuttcap%
\pgfsetmiterjoin%
\definecolor{currentfill}{rgb}{1.000000,1.000000,1.000000}%
\pgfsetfillcolor{currentfill}%
\pgfsetlinewidth{0.000000pt}%
\definecolor{currentstroke}{rgb}{0.000000,0.000000,0.000000}%
\pgfsetstrokecolor{currentstroke}%
\pgfsetstrokeopacity{0.000000}%
\pgfsetdash{}{0pt}%
\pgfpathmoveto{\pgfqpoint{0.331521in}{0.422640in}}%
\pgfpathlineto{\pgfqpoint{2.467022in}{0.422640in}}%
\pgfpathlineto{\pgfqpoint{2.467022in}{1.632426in}}%
\pgfpathlineto{\pgfqpoint{0.331521in}{1.632426in}}%
\pgfpathclose%
\pgfusepath{fill}%
\end{pgfscope}%
\begin{pgfscope}%
\pgfpathrectangle{\pgfqpoint{0.331521in}{0.422640in}}{\pgfqpoint{2.135501in}{1.209786in}} %
\pgfusepath{clip}%
\pgfsetbuttcap%
\pgfsetmiterjoin%
\definecolor{currentfill}{rgb}{0.215686,0.470588,0.749020}%
\pgfsetfillcolor{currentfill}%
\pgfsetlinewidth{0.000000pt}%
\definecolor{currentstroke}{rgb}{0.000000,0.000000,0.000000}%
\pgfsetstrokecolor{currentstroke}%
\pgfsetdash{}{0pt}%
\pgfpathmoveto{\pgfqpoint{0.331521in}{0.422640in}}%
\pgfpathlineto{\pgfqpoint{0.331521in}{1.029401in}}%
\pgfpathlineto{\pgfqpoint{0.374231in}{1.029401in}}%
\pgfpathlineto{\pgfqpoint{0.374231in}{1.477153in}}%
\pgfpathlineto{\pgfqpoint{0.416941in}{1.477153in}}%
\pgfpathlineto{\pgfqpoint{0.416941in}{1.499513in}}%
\pgfpathlineto{\pgfqpoint{0.459651in}{1.499513in}}%
\pgfpathlineto{\pgfqpoint{0.459651in}{1.331332in}}%
\pgfpathlineto{\pgfqpoint{0.502361in}{1.331332in}}%
\pgfpathlineto{\pgfqpoint{0.502361in}{1.216522in}}%
\pgfpathlineto{\pgfqpoint{0.545071in}{1.216522in}}%
\pgfpathlineto{\pgfqpoint{0.545071in}{1.098210in}}%
\pgfpathlineto{\pgfqpoint{0.587781in}{1.098210in}}%
\pgfpathlineto{\pgfqpoint{0.587781in}{1.052586in}}%
\pgfpathlineto{\pgfqpoint{0.630491in}{1.052586in}}%
\pgfpathlineto{\pgfqpoint{0.630491in}{0.960937in}}%
\pgfpathlineto{\pgfqpoint{0.673201in}{0.960937in}}%
\pgfpathlineto{\pgfqpoint{0.673201in}{0.898936in}}%
\pgfpathlineto{\pgfqpoint{0.715911in}{0.898936in}}%
\pgfpathlineto{\pgfqpoint{0.715911in}{0.820865in}}%
\pgfpathlineto{\pgfqpoint{0.758621in}{0.820865in}}%
\pgfpathlineto{\pgfqpoint{0.758621in}{0.792454in}}%
\pgfpathlineto{\pgfqpoint{0.801331in}{0.792454in}}%
\pgfpathlineto{\pgfqpoint{0.801331in}{0.765894in}}%
\pgfpathlineto{\pgfqpoint{0.844041in}{0.765894in}}%
\pgfpathlineto{\pgfqpoint{0.844041in}{0.693471in}}%
\pgfpathlineto{\pgfqpoint{0.886751in}{0.693471in}}%
\pgfpathlineto{\pgfqpoint{0.886751in}{0.680718in}}%
\pgfpathlineto{\pgfqpoint{0.929461in}{0.680718in}}%
\pgfpathlineto{\pgfqpoint{0.929461in}{0.640292in}}%
\pgfpathlineto{\pgfqpoint{0.972171in}{0.640292in}}%
\pgfpathlineto{\pgfqpoint{0.972171in}{0.625181in}}%
\pgfpathlineto{\pgfqpoint{1.014881in}{0.625181in}}%
\pgfpathlineto{\pgfqpoint{1.014881in}{0.602254in}}%
\pgfpathlineto{\pgfqpoint{1.057591in}{0.602254in}}%
\pgfpathlineto{\pgfqpoint{1.057591in}{0.574050in}}%
\pgfpathlineto{\pgfqpoint{1.100301in}{0.574050in}}%
\pgfpathlineto{\pgfqpoint{1.100301in}{0.563518in}}%
\pgfpathlineto{\pgfqpoint{1.143011in}{0.563518in}}%
\pgfpathlineto{\pgfqpoint{1.143011in}{0.550465in}}%
\pgfpathlineto{\pgfqpoint{1.185721in}{0.550465in}}%
\pgfpathlineto{\pgfqpoint{1.185721in}{0.541238in}}%
\pgfpathlineto{\pgfqpoint{1.228432in}{0.541238in}}%
\pgfpathlineto{\pgfqpoint{1.228432in}{0.523736in}}%
\pgfpathlineto{\pgfqpoint{1.271142in}{0.523736in}}%
\pgfpathlineto{\pgfqpoint{1.271142in}{0.526813in}}%
\pgfpathlineto{\pgfqpoint{1.313852in}{0.526813in}}%
\pgfpathlineto{\pgfqpoint{1.313852in}{0.512250in}}%
\pgfpathlineto{\pgfqpoint{1.356562in}{0.512250in}}%
\pgfpathlineto{\pgfqpoint{1.356562in}{0.509644in}}%
\pgfpathlineto{\pgfqpoint{1.399272in}{0.509644in}}%
\pgfpathlineto{\pgfqpoint{1.399272in}{0.502943in}}%
\pgfpathlineto{\pgfqpoint{1.441982in}{0.502943in}}%
\pgfpathlineto{\pgfqpoint{1.441982in}{0.478997in}}%
\pgfpathlineto{\pgfqpoint{1.484692in}{0.478997in}}%
\pgfpathlineto{\pgfqpoint{1.484692in}{0.487333in}}%
\pgfpathlineto{\pgfqpoint{1.527402in}{0.487333in}}%
\pgfpathlineto{\pgfqpoint{1.527402in}{0.481178in}}%
\pgfpathlineto{\pgfqpoint{1.570112in}{0.481178in}}%
\pgfpathlineto{\pgfqpoint{1.570112in}{0.477368in}}%
\pgfpathlineto{\pgfqpoint{1.612822in}{0.477368in}}%
\pgfpathlineto{\pgfqpoint{1.612822in}{0.471314in}}%
\pgfpathlineto{\pgfqpoint{1.655532in}{0.471314in}}%
\pgfpathlineto{\pgfqpoint{1.655532in}{0.463996in}}%
\pgfpathlineto{\pgfqpoint{1.698242in}{0.463996in}}%
\pgfpathlineto{\pgfqpoint{1.698242in}{0.466704in}}%
\pgfpathlineto{\pgfqpoint{1.740952in}{0.466704in}}%
\pgfpathlineto{\pgfqpoint{1.740952in}{0.466014in}}%
\pgfpathlineto{\pgfqpoint{1.783662in}{0.466014in}}%
\pgfpathlineto{\pgfqpoint{1.783662in}{0.460567in}}%
\pgfpathlineto{\pgfqpoint{1.826372in}{0.460567in}}%
\pgfpathlineto{\pgfqpoint{1.826372in}{0.463224in}}%
\pgfpathlineto{\pgfqpoint{1.869082in}{0.463224in}}%
\pgfpathlineto{\pgfqpoint{1.869082in}{0.454152in}}%
\pgfpathlineto{\pgfqpoint{1.911792in}{0.454152in}}%
\pgfpathlineto{\pgfqpoint{1.911792in}{0.451705in}}%
\pgfpathlineto{\pgfqpoint{1.954502in}{0.451705in}}%
\pgfpathlineto{\pgfqpoint{1.954502in}{0.456370in}}%
\pgfpathlineto{\pgfqpoint{1.997212in}{0.456370in}}%
\pgfpathlineto{\pgfqpoint{1.997212in}{0.452002in}}%
\pgfpathlineto{\pgfqpoint{2.039922in}{0.452002in}}%
\pgfpathlineto{\pgfqpoint{2.039922in}{0.453564in}}%
\pgfpathlineto{\pgfqpoint{2.082632in}{0.453564in}}%
\pgfpathlineto{\pgfqpoint{2.082632in}{0.443427in}}%
\pgfpathlineto{\pgfqpoint{2.125342in}{0.443427in}}%
\pgfpathlineto{\pgfqpoint{2.125342in}{0.446625in}}%
\pgfpathlineto{\pgfqpoint{2.168052in}{0.446625in}}%
\pgfpathlineto{\pgfqpoint{2.168052in}{0.446700in}}%
\pgfpathlineto{\pgfqpoint{2.210762in}{0.446700in}}%
\pgfpathlineto{\pgfqpoint{2.210762in}{0.439602in}}%
\pgfpathlineto{\pgfqpoint{2.253472in}{0.439602in}}%
\pgfpathlineto{\pgfqpoint{2.253472in}{0.439767in}}%
\pgfpathlineto{\pgfqpoint{2.296182in}{0.439767in}}%
\pgfpathlineto{\pgfqpoint{2.296182in}{0.439397in}}%
\pgfpathlineto{\pgfqpoint{2.338892in}{0.439397in}}%
\pgfpathlineto{\pgfqpoint{2.338892in}{0.435033in}}%
\pgfpathlineto{\pgfqpoint{2.381602in}{0.435033in}}%
\pgfpathlineto{\pgfqpoint{2.381602in}{0.432018in}}%
\pgfpathlineto{\pgfqpoint{2.424312in}{0.432018in}}%
\pgfpathlineto{\pgfqpoint{2.424312in}{0.436233in}}%
\pgfpathlineto{\pgfqpoint{2.467022in}{0.436233in}}%
\pgfpathlineto{\pgfqpoint{2.467022in}{0.422640in}}%
\pgfpathlineto{\pgfqpoint{2.424312in}{0.422640in}}%
\pgfpathlineto{\pgfqpoint{2.424312in}{0.422640in}}%
\pgfpathlineto{\pgfqpoint{2.381602in}{0.422640in}}%
\pgfpathlineto{\pgfqpoint{2.381602in}{0.422640in}}%
\pgfpathlineto{\pgfqpoint{2.338892in}{0.422640in}}%
\pgfpathlineto{\pgfqpoint{2.338892in}{0.422640in}}%
\pgfpathlineto{\pgfqpoint{2.296182in}{0.422640in}}%
\pgfpathlineto{\pgfqpoint{2.296182in}{0.422640in}}%
\pgfpathlineto{\pgfqpoint{2.253472in}{0.422640in}}%
\pgfpathlineto{\pgfqpoint{2.253472in}{0.422640in}}%
\pgfpathlineto{\pgfqpoint{2.210762in}{0.422640in}}%
\pgfpathlineto{\pgfqpoint{2.210762in}{0.422640in}}%
\pgfpathlineto{\pgfqpoint{2.168052in}{0.422640in}}%
\pgfpathlineto{\pgfqpoint{2.168052in}{0.422640in}}%
\pgfpathlineto{\pgfqpoint{2.125342in}{0.422640in}}%
\pgfpathlineto{\pgfqpoint{2.125342in}{0.422640in}}%
\pgfpathlineto{\pgfqpoint{2.082632in}{0.422640in}}%
\pgfpathlineto{\pgfqpoint{2.082632in}{0.422640in}}%
\pgfpathlineto{\pgfqpoint{2.039922in}{0.422640in}}%
\pgfpathlineto{\pgfqpoint{2.039922in}{0.422640in}}%
\pgfpathlineto{\pgfqpoint{1.997212in}{0.422640in}}%
\pgfpathlineto{\pgfqpoint{1.997212in}{0.422640in}}%
\pgfpathlineto{\pgfqpoint{1.954502in}{0.422640in}}%
\pgfpathlineto{\pgfqpoint{1.954502in}{0.422640in}}%
\pgfpathlineto{\pgfqpoint{1.911792in}{0.422640in}}%
\pgfpathlineto{\pgfqpoint{1.911792in}{0.422640in}}%
\pgfpathlineto{\pgfqpoint{1.869082in}{0.422640in}}%
\pgfpathlineto{\pgfqpoint{1.869082in}{0.422640in}}%
\pgfpathlineto{\pgfqpoint{1.826372in}{0.422640in}}%
\pgfpathlineto{\pgfqpoint{1.826372in}{0.422640in}}%
\pgfpathlineto{\pgfqpoint{1.783662in}{0.422640in}}%
\pgfpathlineto{\pgfqpoint{1.783662in}{0.422640in}}%
\pgfpathlineto{\pgfqpoint{1.740952in}{0.422640in}}%
\pgfpathlineto{\pgfqpoint{1.740952in}{0.422640in}}%
\pgfpathlineto{\pgfqpoint{1.698242in}{0.422640in}}%
\pgfpathlineto{\pgfqpoint{1.698242in}{0.422640in}}%
\pgfpathlineto{\pgfqpoint{1.655532in}{0.422640in}}%
\pgfpathlineto{\pgfqpoint{1.655532in}{0.422640in}}%
\pgfpathlineto{\pgfqpoint{1.612822in}{0.422640in}}%
\pgfpathlineto{\pgfqpoint{1.612822in}{0.422640in}}%
\pgfpathlineto{\pgfqpoint{1.570112in}{0.422640in}}%
\pgfpathlineto{\pgfqpoint{1.570112in}{0.422640in}}%
\pgfpathlineto{\pgfqpoint{1.527402in}{0.422640in}}%
\pgfpathlineto{\pgfqpoint{1.527402in}{0.422640in}}%
\pgfpathlineto{\pgfqpoint{1.484692in}{0.422640in}}%
\pgfpathlineto{\pgfqpoint{1.484692in}{0.422640in}}%
\pgfpathlineto{\pgfqpoint{1.441982in}{0.422640in}}%
\pgfpathlineto{\pgfqpoint{1.441982in}{0.422640in}}%
\pgfpathlineto{\pgfqpoint{1.399272in}{0.422640in}}%
\pgfpathlineto{\pgfqpoint{1.399272in}{0.422640in}}%
\pgfpathlineto{\pgfqpoint{1.356562in}{0.422640in}}%
\pgfpathlineto{\pgfqpoint{1.356562in}{0.422640in}}%
\pgfpathlineto{\pgfqpoint{1.313852in}{0.422640in}}%
\pgfpathlineto{\pgfqpoint{1.313852in}{0.422640in}}%
\pgfpathlineto{\pgfqpoint{1.271142in}{0.422640in}}%
\pgfpathlineto{\pgfqpoint{1.271142in}{0.422640in}}%
\pgfpathlineto{\pgfqpoint{1.228432in}{0.422640in}}%
\pgfpathlineto{\pgfqpoint{1.228432in}{0.422640in}}%
\pgfpathlineto{\pgfqpoint{1.185721in}{0.422640in}}%
\pgfpathlineto{\pgfqpoint{1.185721in}{0.422640in}}%
\pgfpathlineto{\pgfqpoint{1.143011in}{0.422640in}}%
\pgfpathlineto{\pgfqpoint{1.143011in}{0.422640in}}%
\pgfpathlineto{\pgfqpoint{1.100301in}{0.422640in}}%
\pgfpathlineto{\pgfqpoint{1.100301in}{0.422640in}}%
\pgfpathlineto{\pgfqpoint{1.057591in}{0.422640in}}%
\pgfpathlineto{\pgfqpoint{1.057591in}{0.422640in}}%
\pgfpathlineto{\pgfqpoint{1.014881in}{0.422640in}}%
\pgfpathlineto{\pgfqpoint{1.014881in}{0.422640in}}%
\pgfpathlineto{\pgfqpoint{0.972171in}{0.422640in}}%
\pgfpathlineto{\pgfqpoint{0.972171in}{0.422640in}}%
\pgfpathlineto{\pgfqpoint{0.929461in}{0.422640in}}%
\pgfpathlineto{\pgfqpoint{0.929461in}{0.422640in}}%
\pgfpathlineto{\pgfqpoint{0.886751in}{0.422640in}}%
\pgfpathlineto{\pgfqpoint{0.886751in}{0.422640in}}%
\pgfpathlineto{\pgfqpoint{0.844041in}{0.422640in}}%
\pgfpathlineto{\pgfqpoint{0.844041in}{0.422640in}}%
\pgfpathlineto{\pgfqpoint{0.801331in}{0.422640in}}%
\pgfpathlineto{\pgfqpoint{0.801331in}{0.422640in}}%
\pgfpathlineto{\pgfqpoint{0.758621in}{0.422640in}}%
\pgfpathlineto{\pgfqpoint{0.758621in}{0.422640in}}%
\pgfpathlineto{\pgfqpoint{0.715911in}{0.422640in}}%
\pgfpathlineto{\pgfqpoint{0.715911in}{0.422640in}}%
\pgfpathlineto{\pgfqpoint{0.673201in}{0.422640in}}%
\pgfpathlineto{\pgfqpoint{0.673201in}{0.422640in}}%
\pgfpathlineto{\pgfqpoint{0.630491in}{0.422640in}}%
\pgfpathlineto{\pgfqpoint{0.630491in}{0.422640in}}%
\pgfpathlineto{\pgfqpoint{0.587781in}{0.422640in}}%
\pgfpathlineto{\pgfqpoint{0.587781in}{0.422640in}}%
\pgfpathlineto{\pgfqpoint{0.545071in}{0.422640in}}%
\pgfpathlineto{\pgfqpoint{0.545071in}{0.422640in}}%
\pgfpathlineto{\pgfqpoint{0.502361in}{0.422640in}}%
\pgfpathlineto{\pgfqpoint{0.502361in}{0.422640in}}%
\pgfpathlineto{\pgfqpoint{0.459651in}{0.422640in}}%
\pgfpathlineto{\pgfqpoint{0.459651in}{0.422640in}}%
\pgfpathlineto{\pgfqpoint{0.416941in}{0.422640in}}%
\pgfpathlineto{\pgfqpoint{0.416941in}{0.422640in}}%
\pgfpathlineto{\pgfqpoint{0.374231in}{0.422640in}}%
\pgfpathlineto{\pgfqpoint{0.374231in}{0.422640in}}%
\pgfpathlineto{\pgfqpoint{0.331521in}{0.422640in}}%
\pgfusepath{fill}%
\end{pgfscope}%
\begin{pgfscope}%
\pgfpathrectangle{\pgfqpoint{0.331521in}{0.422640in}}{\pgfqpoint{2.135501in}{1.209786in}} %
\pgfusepath{clip}%
\pgfsetbuttcap%
\pgfsetmiterjoin%
\pgfsetlinewidth{0.501875pt}%
\definecolor{currentstroke}{rgb}{1.000000,0.000000,0.000000}%
\pgfsetstrokecolor{currentstroke}%
\pgfsetdash{}{0pt}%
\pgfpathmoveto{\pgfqpoint{0.331521in}{0.422640in}}%
\pgfpathlineto{\pgfqpoint{0.331521in}{0.454983in}}%
\pgfpathlineto{\pgfqpoint{0.374231in}{0.454983in}}%
\pgfpathlineto{\pgfqpoint{0.374231in}{0.508557in}}%
\pgfpathlineto{\pgfqpoint{0.416941in}{0.508557in}}%
\pgfpathlineto{\pgfqpoint{0.416941in}{0.555351in}}%
\pgfpathlineto{\pgfqpoint{0.459651in}{0.555351in}}%
\pgfpathlineto{\pgfqpoint{0.459651in}{0.591727in}}%
\pgfpathlineto{\pgfqpoint{0.502361in}{0.591727in}}%
\pgfpathlineto{\pgfqpoint{0.502361in}{0.626841in}}%
\pgfpathlineto{\pgfqpoint{0.545071in}{0.626841in}}%
\pgfpathlineto{\pgfqpoint{0.545071in}{0.652329in}}%
\pgfpathlineto{\pgfqpoint{0.587781in}{0.652329in}}%
\pgfpathlineto{\pgfqpoint{0.587781in}{0.672695in}}%
\pgfpathlineto{\pgfqpoint{0.630491in}{0.672695in}}%
\pgfpathlineto{\pgfqpoint{0.630491in}{0.690710in}}%
\pgfpathlineto{\pgfqpoint{0.673201in}{0.690710in}}%
\pgfpathlineto{\pgfqpoint{0.673201in}{0.691007in}}%
\pgfpathlineto{\pgfqpoint{0.715911in}{0.691007in}}%
\pgfpathlineto{\pgfqpoint{0.715911in}{0.704196in}}%
\pgfpathlineto{\pgfqpoint{0.758621in}{0.704196in}}%
\pgfpathlineto{\pgfqpoint{0.758621in}{0.708601in}}%
\pgfpathlineto{\pgfqpoint{0.801331in}{0.708601in}}%
\pgfpathlineto{\pgfqpoint{0.801331in}{0.711793in}}%
\pgfpathlineto{\pgfqpoint{0.844041in}{0.711793in}}%
\pgfpathlineto{\pgfqpoint{0.844041in}{0.709071in}}%
\pgfpathlineto{\pgfqpoint{0.886751in}{0.709071in}}%
\pgfpathlineto{\pgfqpoint{0.886751in}{0.711595in}}%
\pgfpathlineto{\pgfqpoint{0.929461in}{0.711595in}}%
\pgfpathlineto{\pgfqpoint{0.929461in}{0.715159in}}%
\pgfpathlineto{\pgfqpoint{0.972171in}{0.715159in}}%
\pgfpathlineto{\pgfqpoint{0.972171in}{0.712065in}}%
\pgfpathlineto{\pgfqpoint{1.014881in}{0.712065in}}%
\pgfpathlineto{\pgfqpoint{1.014881in}{0.709789in}}%
\pgfpathlineto{\pgfqpoint{1.057591in}{0.709789in}}%
\pgfpathlineto{\pgfqpoint{1.057591in}{0.704840in}}%
\pgfpathlineto{\pgfqpoint{1.100301in}{0.704840in}}%
\pgfpathlineto{\pgfqpoint{1.100301in}{0.706671in}}%
\pgfpathlineto{\pgfqpoint{1.143011in}{0.706671in}}%
\pgfpathlineto{\pgfqpoint{1.143011in}{0.695758in}}%
\pgfpathlineto{\pgfqpoint{1.185721in}{0.695758in}}%
\pgfpathlineto{\pgfqpoint{1.185721in}{0.692318in}}%
\pgfpathlineto{\pgfqpoint{1.228432in}{0.692318in}}%
\pgfpathlineto{\pgfqpoint{1.228432in}{0.686627in}}%
\pgfpathlineto{\pgfqpoint{1.271142in}{0.686627in}}%
\pgfpathlineto{\pgfqpoint{1.271142in}{0.682445in}}%
\pgfpathlineto{\pgfqpoint{1.313852in}{0.682445in}}%
\pgfpathlineto{\pgfqpoint{1.313852in}{0.679129in}}%
\pgfpathlineto{\pgfqpoint{1.356562in}{0.679129in}}%
\pgfpathlineto{\pgfqpoint{1.356562in}{0.672101in}}%
\pgfpathlineto{\pgfqpoint{1.399272in}{0.672101in}}%
\pgfpathlineto{\pgfqpoint{1.399272in}{0.668686in}}%
\pgfpathlineto{\pgfqpoint{1.441982in}{0.668686in}}%
\pgfpathlineto{\pgfqpoint{1.441982in}{0.660768in}}%
\pgfpathlineto{\pgfqpoint{1.484692in}{0.660768in}}%
\pgfpathlineto{\pgfqpoint{1.484692in}{0.656264in}}%
\pgfpathlineto{\pgfqpoint{1.527402in}{0.656264in}}%
\pgfpathlineto{\pgfqpoint{1.527402in}{0.657427in}}%
\pgfpathlineto{\pgfqpoint{1.570112in}{0.657427in}}%
\pgfpathlineto{\pgfqpoint{1.570112in}{0.653171in}}%
\pgfpathlineto{\pgfqpoint{1.612822in}{0.653171in}}%
\pgfpathlineto{\pgfqpoint{1.612822in}{0.639066in}}%
\pgfpathlineto{\pgfqpoint{1.655532in}{0.639066in}}%
\pgfpathlineto{\pgfqpoint{1.655532in}{0.632879in}}%
\pgfpathlineto{\pgfqpoint{1.698242in}{0.632879in}}%
\pgfpathlineto{\pgfqpoint{1.698242in}{0.629266in}}%
\pgfpathlineto{\pgfqpoint{1.740952in}{0.629266in}}%
\pgfpathlineto{\pgfqpoint{1.740952in}{0.628796in}}%
\pgfpathlineto{\pgfqpoint{1.783662in}{0.628796in}}%
\pgfpathlineto{\pgfqpoint{1.783662in}{0.620803in}}%
\pgfpathlineto{\pgfqpoint{1.826372in}{0.620803in}}%
\pgfpathlineto{\pgfqpoint{1.826372in}{0.622164in}}%
\pgfpathlineto{\pgfqpoint{1.869082in}{0.622164in}}%
\pgfpathlineto{\pgfqpoint{1.869082in}{0.612860in}}%
\pgfpathlineto{\pgfqpoint{1.911792in}{0.612860in}}%
\pgfpathlineto{\pgfqpoint{1.911792in}{0.608604in}}%
\pgfpathlineto{\pgfqpoint{1.954502in}{0.608604in}}%
\pgfpathlineto{\pgfqpoint{1.954502in}{0.605040in}}%
\pgfpathlineto{\pgfqpoint{1.997212in}{0.605040in}}%
\pgfpathlineto{\pgfqpoint{1.997212in}{0.599151in}}%
\pgfpathlineto{\pgfqpoint{2.039922in}{0.599151in}}%
\pgfpathlineto{\pgfqpoint{2.039922in}{0.597443in}}%
\pgfpathlineto{\pgfqpoint{2.082632in}{0.597443in}}%
\pgfpathlineto{\pgfqpoint{2.082632in}{0.593435in}}%
\pgfpathlineto{\pgfqpoint{2.125342in}{0.593435in}}%
\pgfpathlineto{\pgfqpoint{2.125342in}{0.589451in}}%
\pgfpathlineto{\pgfqpoint{2.168052in}{0.589451in}}%
\pgfpathlineto{\pgfqpoint{2.168052in}{0.584378in}}%
\pgfpathlineto{\pgfqpoint{2.210762in}{0.584378in}}%
\pgfpathlineto{\pgfqpoint{2.210762in}{0.582522in}}%
\pgfpathlineto{\pgfqpoint{2.253472in}{0.582522in}}%
\pgfpathlineto{\pgfqpoint{2.253472in}{0.574776in}}%
\pgfpathlineto{\pgfqpoint{2.296182in}{0.574776in}}%
\pgfpathlineto{\pgfqpoint{2.296182in}{0.575692in}}%
\pgfpathlineto{\pgfqpoint{2.338892in}{0.575692in}}%
\pgfpathlineto{\pgfqpoint{2.338892in}{0.566858in}}%
\pgfpathlineto{\pgfqpoint{2.381602in}{0.566858in}}%
\pgfpathlineto{\pgfqpoint{2.381602in}{0.566041in}}%
\pgfpathlineto{\pgfqpoint{2.424312in}{0.566041in}}%
\pgfpathlineto{\pgfqpoint{2.424312in}{0.561760in}}%
\pgfpathlineto{\pgfqpoint{2.467022in}{0.561760in}}%
\pgfpathlineto{\pgfqpoint{2.467022in}{0.422640in}}%
\pgfusepath{stroke}%
\end{pgfscope}%
\begin{pgfscope}%
\pgfsetrectcap%
\pgfsetmiterjoin%
\pgfsetlinewidth{1.003750pt}%
\definecolor{currentstroke}{rgb}{0.000000,0.000000,0.000000}%
\pgfsetstrokecolor{currentstroke}%
\pgfsetdash{}{0pt}%
\pgfpathmoveto{\pgfqpoint{0.331521in}{1.632426in}}%
\pgfpathlineto{\pgfqpoint{2.467022in}{1.632426in}}%
\pgfusepath{stroke}%
\end{pgfscope}%
\begin{pgfscope}%
\pgfsetrectcap%
\pgfsetmiterjoin%
\pgfsetlinewidth{1.003750pt}%
\definecolor{currentstroke}{rgb}{0.000000,0.000000,0.000000}%
\pgfsetstrokecolor{currentstroke}%
\pgfsetdash{}{0pt}%
\pgfpathmoveto{\pgfqpoint{2.467022in}{0.422640in}}%
\pgfpathlineto{\pgfqpoint{2.467022in}{1.632426in}}%
\pgfusepath{stroke}%
\end{pgfscope}%
\begin{pgfscope}%
\pgfsetrectcap%
\pgfsetmiterjoin%
\pgfsetlinewidth{1.003750pt}%
\definecolor{currentstroke}{rgb}{0.000000,0.000000,0.000000}%
\pgfsetstrokecolor{currentstroke}%
\pgfsetdash{}{0pt}%
\pgfpathmoveto{\pgfqpoint{0.331521in}{0.422640in}}%
\pgfpathlineto{\pgfqpoint{2.467022in}{0.422640in}}%
\pgfusepath{stroke}%
\end{pgfscope}%
\begin{pgfscope}%
\pgfsetrectcap%
\pgfsetmiterjoin%
\pgfsetlinewidth{1.003750pt}%
\definecolor{currentstroke}{rgb}{0.000000,0.000000,0.000000}%
\pgfsetstrokecolor{currentstroke}%
\pgfsetdash{}{0pt}%
\pgfpathmoveto{\pgfqpoint{0.331521in}{0.422640in}}%
\pgfpathlineto{\pgfqpoint{0.331521in}{1.632426in}}%
\pgfusepath{stroke}%
\end{pgfscope}%
\begin{pgfscope}%
\pgfsetbuttcap%
\pgfsetroundjoin%
\definecolor{currentfill}{rgb}{0.000000,0.000000,0.000000}%
\pgfsetfillcolor{currentfill}%
\pgfsetlinewidth{0.501875pt}%
\definecolor{currentstroke}{rgb}{0.000000,0.000000,0.000000}%
\pgfsetstrokecolor{currentstroke}%
\pgfsetdash{}{0pt}%
\pgfsys@defobject{currentmarker}{\pgfqpoint{0.000000in}{0.000000in}}{\pgfqpoint{0.000000in}{0.069444in}}{%
\pgfpathmoveto{\pgfqpoint{0.000000in}{0.000000in}}%
\pgfpathlineto{\pgfqpoint{0.000000in}{0.069444in}}%
\pgfusepath{stroke,fill}%
}%
\begin{pgfscope}%
\pgfsys@transformshift{0.331521in}{0.422640in}%
\pgfsys@useobject{currentmarker}{}%
\end{pgfscope}%
\end{pgfscope}%
\begin{pgfscope}%
\pgfsetbuttcap%
\pgfsetroundjoin%
\definecolor{currentfill}{rgb}{0.000000,0.000000,0.000000}%
\pgfsetfillcolor{currentfill}%
\pgfsetlinewidth{0.501875pt}%
\definecolor{currentstroke}{rgb}{0.000000,0.000000,0.000000}%
\pgfsetstrokecolor{currentstroke}%
\pgfsetdash{}{0pt}%
\pgfsys@defobject{currentmarker}{\pgfqpoint{0.000000in}{-0.069444in}}{\pgfqpoint{0.000000in}{0.000000in}}{%
\pgfpathmoveto{\pgfqpoint{0.000000in}{0.000000in}}%
\pgfpathlineto{\pgfqpoint{0.000000in}{-0.069444in}}%
\pgfusepath{stroke,fill}%
}%
\begin{pgfscope}%
\pgfsys@transformshift{0.331521in}{1.632426in}%
\pgfsys@useobject{currentmarker}{}%
\end{pgfscope}%
\end{pgfscope}%
\begin{pgfscope}%
\pgftext[x=0.331521in,y=0.353196in,,top]{\rmfamily\fontsize{8.000000}{9.600000}\selectfont 0.000}%
\end{pgfscope}%
\begin{pgfscope}%
\pgfsetbuttcap%
\pgfsetroundjoin%
\definecolor{currentfill}{rgb}{0.000000,0.000000,0.000000}%
\pgfsetfillcolor{currentfill}%
\pgfsetlinewidth{0.501875pt}%
\definecolor{currentstroke}{rgb}{0.000000,0.000000,0.000000}%
\pgfsetstrokecolor{currentstroke}%
\pgfsetdash{}{0pt}%
\pgfsys@defobject{currentmarker}{\pgfqpoint{0.000000in}{0.000000in}}{\pgfqpoint{0.000000in}{0.069444in}}{%
\pgfpathmoveto{\pgfqpoint{0.000000in}{0.000000in}}%
\pgfpathlineto{\pgfqpoint{0.000000in}{0.069444in}}%
\pgfusepath{stroke,fill}%
}%
\begin{pgfscope}%
\pgfsys@transformshift{0.636593in}{0.422640in}%
\pgfsys@useobject{currentmarker}{}%
\end{pgfscope}%
\end{pgfscope}%
\begin{pgfscope}%
\pgfsetbuttcap%
\pgfsetroundjoin%
\definecolor{currentfill}{rgb}{0.000000,0.000000,0.000000}%
\pgfsetfillcolor{currentfill}%
\pgfsetlinewidth{0.501875pt}%
\definecolor{currentstroke}{rgb}{0.000000,0.000000,0.000000}%
\pgfsetstrokecolor{currentstroke}%
\pgfsetdash{}{0pt}%
\pgfsys@defobject{currentmarker}{\pgfqpoint{0.000000in}{-0.069444in}}{\pgfqpoint{0.000000in}{0.000000in}}{%
\pgfpathmoveto{\pgfqpoint{0.000000in}{0.000000in}}%
\pgfpathlineto{\pgfqpoint{0.000000in}{-0.069444in}}%
\pgfusepath{stroke,fill}%
}%
\begin{pgfscope}%
\pgfsys@transformshift{0.636593in}{1.632426in}%
\pgfsys@useobject{currentmarker}{}%
\end{pgfscope}%
\end{pgfscope}%
\begin{pgfscope}%
\pgftext[x=0.636593in,y=0.353196in,,top]{\rmfamily\fontsize{8.000000}{9.600000}\selectfont 0.002}%
\end{pgfscope}%
\begin{pgfscope}%
\pgfsetbuttcap%
\pgfsetroundjoin%
\definecolor{currentfill}{rgb}{0.000000,0.000000,0.000000}%
\pgfsetfillcolor{currentfill}%
\pgfsetlinewidth{0.501875pt}%
\definecolor{currentstroke}{rgb}{0.000000,0.000000,0.000000}%
\pgfsetstrokecolor{currentstroke}%
\pgfsetdash{}{0pt}%
\pgfsys@defobject{currentmarker}{\pgfqpoint{0.000000in}{0.000000in}}{\pgfqpoint{0.000000in}{0.069444in}}{%
\pgfpathmoveto{\pgfqpoint{0.000000in}{0.000000in}}%
\pgfpathlineto{\pgfqpoint{0.000000in}{0.069444in}}%
\pgfusepath{stroke,fill}%
}%
\begin{pgfscope}%
\pgfsys@transformshift{0.941664in}{0.422640in}%
\pgfsys@useobject{currentmarker}{}%
\end{pgfscope}%
\end{pgfscope}%
\begin{pgfscope}%
\pgfsetbuttcap%
\pgfsetroundjoin%
\definecolor{currentfill}{rgb}{0.000000,0.000000,0.000000}%
\pgfsetfillcolor{currentfill}%
\pgfsetlinewidth{0.501875pt}%
\definecolor{currentstroke}{rgb}{0.000000,0.000000,0.000000}%
\pgfsetstrokecolor{currentstroke}%
\pgfsetdash{}{0pt}%
\pgfsys@defobject{currentmarker}{\pgfqpoint{0.000000in}{-0.069444in}}{\pgfqpoint{0.000000in}{0.000000in}}{%
\pgfpathmoveto{\pgfqpoint{0.000000in}{0.000000in}}%
\pgfpathlineto{\pgfqpoint{0.000000in}{-0.069444in}}%
\pgfusepath{stroke,fill}%
}%
\begin{pgfscope}%
\pgfsys@transformshift{0.941664in}{1.632426in}%
\pgfsys@useobject{currentmarker}{}%
\end{pgfscope}%
\end{pgfscope}%
\begin{pgfscope}%
\pgftext[x=0.941664in,y=0.353196in,,top]{\rmfamily\fontsize{8.000000}{9.600000}\selectfont 0.004}%
\end{pgfscope}%
\begin{pgfscope}%
\pgfsetbuttcap%
\pgfsetroundjoin%
\definecolor{currentfill}{rgb}{0.000000,0.000000,0.000000}%
\pgfsetfillcolor{currentfill}%
\pgfsetlinewidth{0.501875pt}%
\definecolor{currentstroke}{rgb}{0.000000,0.000000,0.000000}%
\pgfsetstrokecolor{currentstroke}%
\pgfsetdash{}{0pt}%
\pgfsys@defobject{currentmarker}{\pgfqpoint{0.000000in}{0.000000in}}{\pgfqpoint{0.000000in}{0.069444in}}{%
\pgfpathmoveto{\pgfqpoint{0.000000in}{0.000000in}}%
\pgfpathlineto{\pgfqpoint{0.000000in}{0.069444in}}%
\pgfusepath{stroke,fill}%
}%
\begin{pgfscope}%
\pgfsys@transformshift{1.246736in}{0.422640in}%
\pgfsys@useobject{currentmarker}{}%
\end{pgfscope}%
\end{pgfscope}%
\begin{pgfscope}%
\pgfsetbuttcap%
\pgfsetroundjoin%
\definecolor{currentfill}{rgb}{0.000000,0.000000,0.000000}%
\pgfsetfillcolor{currentfill}%
\pgfsetlinewidth{0.501875pt}%
\definecolor{currentstroke}{rgb}{0.000000,0.000000,0.000000}%
\pgfsetstrokecolor{currentstroke}%
\pgfsetdash{}{0pt}%
\pgfsys@defobject{currentmarker}{\pgfqpoint{0.000000in}{-0.069444in}}{\pgfqpoint{0.000000in}{0.000000in}}{%
\pgfpathmoveto{\pgfqpoint{0.000000in}{0.000000in}}%
\pgfpathlineto{\pgfqpoint{0.000000in}{-0.069444in}}%
\pgfusepath{stroke,fill}%
}%
\begin{pgfscope}%
\pgfsys@transformshift{1.246736in}{1.632426in}%
\pgfsys@useobject{currentmarker}{}%
\end{pgfscope}%
\end{pgfscope}%
\begin{pgfscope}%
\pgftext[x=1.246736in,y=0.353196in,,top]{\rmfamily\fontsize{8.000000}{9.600000}\selectfont 0.006}%
\end{pgfscope}%
\begin{pgfscope}%
\pgfsetbuttcap%
\pgfsetroundjoin%
\definecolor{currentfill}{rgb}{0.000000,0.000000,0.000000}%
\pgfsetfillcolor{currentfill}%
\pgfsetlinewidth{0.501875pt}%
\definecolor{currentstroke}{rgb}{0.000000,0.000000,0.000000}%
\pgfsetstrokecolor{currentstroke}%
\pgfsetdash{}{0pt}%
\pgfsys@defobject{currentmarker}{\pgfqpoint{0.000000in}{0.000000in}}{\pgfqpoint{0.000000in}{0.069444in}}{%
\pgfpathmoveto{\pgfqpoint{0.000000in}{0.000000in}}%
\pgfpathlineto{\pgfqpoint{0.000000in}{0.069444in}}%
\pgfusepath{stroke,fill}%
}%
\begin{pgfscope}%
\pgfsys@transformshift{1.551807in}{0.422640in}%
\pgfsys@useobject{currentmarker}{}%
\end{pgfscope}%
\end{pgfscope}%
\begin{pgfscope}%
\pgfsetbuttcap%
\pgfsetroundjoin%
\definecolor{currentfill}{rgb}{0.000000,0.000000,0.000000}%
\pgfsetfillcolor{currentfill}%
\pgfsetlinewidth{0.501875pt}%
\definecolor{currentstroke}{rgb}{0.000000,0.000000,0.000000}%
\pgfsetstrokecolor{currentstroke}%
\pgfsetdash{}{0pt}%
\pgfsys@defobject{currentmarker}{\pgfqpoint{0.000000in}{-0.069444in}}{\pgfqpoint{0.000000in}{0.000000in}}{%
\pgfpathmoveto{\pgfqpoint{0.000000in}{0.000000in}}%
\pgfpathlineto{\pgfqpoint{0.000000in}{-0.069444in}}%
\pgfusepath{stroke,fill}%
}%
\begin{pgfscope}%
\pgfsys@transformshift{1.551807in}{1.632426in}%
\pgfsys@useobject{currentmarker}{}%
\end{pgfscope}%
\end{pgfscope}%
\begin{pgfscope}%
\pgftext[x=1.551807in,y=0.353196in,,top]{\rmfamily\fontsize{8.000000}{9.600000}\selectfont 0.008}%
\end{pgfscope}%
\begin{pgfscope}%
\pgfsetbuttcap%
\pgfsetroundjoin%
\definecolor{currentfill}{rgb}{0.000000,0.000000,0.000000}%
\pgfsetfillcolor{currentfill}%
\pgfsetlinewidth{0.501875pt}%
\definecolor{currentstroke}{rgb}{0.000000,0.000000,0.000000}%
\pgfsetstrokecolor{currentstroke}%
\pgfsetdash{}{0pt}%
\pgfsys@defobject{currentmarker}{\pgfqpoint{0.000000in}{0.000000in}}{\pgfqpoint{0.000000in}{0.069444in}}{%
\pgfpathmoveto{\pgfqpoint{0.000000in}{0.000000in}}%
\pgfpathlineto{\pgfqpoint{0.000000in}{0.069444in}}%
\pgfusepath{stroke,fill}%
}%
\begin{pgfscope}%
\pgfsys@transformshift{1.856879in}{0.422640in}%
\pgfsys@useobject{currentmarker}{}%
\end{pgfscope}%
\end{pgfscope}%
\begin{pgfscope}%
\pgfsetbuttcap%
\pgfsetroundjoin%
\definecolor{currentfill}{rgb}{0.000000,0.000000,0.000000}%
\pgfsetfillcolor{currentfill}%
\pgfsetlinewidth{0.501875pt}%
\definecolor{currentstroke}{rgb}{0.000000,0.000000,0.000000}%
\pgfsetstrokecolor{currentstroke}%
\pgfsetdash{}{0pt}%
\pgfsys@defobject{currentmarker}{\pgfqpoint{0.000000in}{-0.069444in}}{\pgfqpoint{0.000000in}{0.000000in}}{%
\pgfpathmoveto{\pgfqpoint{0.000000in}{0.000000in}}%
\pgfpathlineto{\pgfqpoint{0.000000in}{-0.069444in}}%
\pgfusepath{stroke,fill}%
}%
\begin{pgfscope}%
\pgfsys@transformshift{1.856879in}{1.632426in}%
\pgfsys@useobject{currentmarker}{}%
\end{pgfscope}%
\end{pgfscope}%
\begin{pgfscope}%
\pgftext[x=1.856879in,y=0.353196in,,top]{\rmfamily\fontsize{8.000000}{9.600000}\selectfont 0.010}%
\end{pgfscope}%
\begin{pgfscope}%
\pgfsetbuttcap%
\pgfsetroundjoin%
\definecolor{currentfill}{rgb}{0.000000,0.000000,0.000000}%
\pgfsetfillcolor{currentfill}%
\pgfsetlinewidth{0.501875pt}%
\definecolor{currentstroke}{rgb}{0.000000,0.000000,0.000000}%
\pgfsetstrokecolor{currentstroke}%
\pgfsetdash{}{0pt}%
\pgfsys@defobject{currentmarker}{\pgfqpoint{0.000000in}{0.000000in}}{\pgfqpoint{0.000000in}{0.069444in}}{%
\pgfpathmoveto{\pgfqpoint{0.000000in}{0.000000in}}%
\pgfpathlineto{\pgfqpoint{0.000000in}{0.069444in}}%
\pgfusepath{stroke,fill}%
}%
\begin{pgfscope}%
\pgfsys@transformshift{2.161950in}{0.422640in}%
\pgfsys@useobject{currentmarker}{}%
\end{pgfscope}%
\end{pgfscope}%
\begin{pgfscope}%
\pgfsetbuttcap%
\pgfsetroundjoin%
\definecolor{currentfill}{rgb}{0.000000,0.000000,0.000000}%
\pgfsetfillcolor{currentfill}%
\pgfsetlinewidth{0.501875pt}%
\definecolor{currentstroke}{rgb}{0.000000,0.000000,0.000000}%
\pgfsetstrokecolor{currentstroke}%
\pgfsetdash{}{0pt}%
\pgfsys@defobject{currentmarker}{\pgfqpoint{0.000000in}{-0.069444in}}{\pgfqpoint{0.000000in}{0.000000in}}{%
\pgfpathmoveto{\pgfqpoint{0.000000in}{0.000000in}}%
\pgfpathlineto{\pgfqpoint{0.000000in}{-0.069444in}}%
\pgfusepath{stroke,fill}%
}%
\begin{pgfscope}%
\pgfsys@transformshift{2.161950in}{1.632426in}%
\pgfsys@useobject{currentmarker}{}%
\end{pgfscope}%
\end{pgfscope}%
\begin{pgfscope}%
\pgftext[x=2.161950in,y=0.353196in,,top]{\rmfamily\fontsize{8.000000}{9.600000}\selectfont 0.012}%
\end{pgfscope}%
\begin{pgfscope}%
\pgfsetbuttcap%
\pgfsetroundjoin%
\definecolor{currentfill}{rgb}{0.000000,0.000000,0.000000}%
\pgfsetfillcolor{currentfill}%
\pgfsetlinewidth{0.501875pt}%
\definecolor{currentstroke}{rgb}{0.000000,0.000000,0.000000}%
\pgfsetstrokecolor{currentstroke}%
\pgfsetdash{}{0pt}%
\pgfsys@defobject{currentmarker}{\pgfqpoint{0.000000in}{0.000000in}}{\pgfqpoint{0.000000in}{0.069444in}}{%
\pgfpathmoveto{\pgfqpoint{0.000000in}{0.000000in}}%
\pgfpathlineto{\pgfqpoint{0.000000in}{0.069444in}}%
\pgfusepath{stroke,fill}%
}%
\begin{pgfscope}%
\pgfsys@transformshift{2.467022in}{0.422640in}%
\pgfsys@useobject{currentmarker}{}%
\end{pgfscope}%
\end{pgfscope}%
\begin{pgfscope}%
\pgfsetbuttcap%
\pgfsetroundjoin%
\definecolor{currentfill}{rgb}{0.000000,0.000000,0.000000}%
\pgfsetfillcolor{currentfill}%
\pgfsetlinewidth{0.501875pt}%
\definecolor{currentstroke}{rgb}{0.000000,0.000000,0.000000}%
\pgfsetstrokecolor{currentstroke}%
\pgfsetdash{}{0pt}%
\pgfsys@defobject{currentmarker}{\pgfqpoint{0.000000in}{-0.069444in}}{\pgfqpoint{0.000000in}{0.000000in}}{%
\pgfpathmoveto{\pgfqpoint{0.000000in}{0.000000in}}%
\pgfpathlineto{\pgfqpoint{0.000000in}{-0.069444in}}%
\pgfusepath{stroke,fill}%
}%
\begin{pgfscope}%
\pgfsys@transformshift{2.467022in}{1.632426in}%
\pgfsys@useobject{currentmarker}{}%
\end{pgfscope}%
\end{pgfscope}%
\begin{pgfscope}%
\pgftext[x=2.467022in,y=0.353196in,,top]{\rmfamily\fontsize{8.000000}{9.600000}\selectfont 0.014}%
\end{pgfscope}%
\begin{pgfscope}%
\pgftext[x=1.399272in,y=0.176221in,,top]{\rmfamily\fontsize{9.000000}{10.800000}\selectfont \(\displaystyle \mathrm{cos}(\mathrm{DIRA\ angle})\)}%
\end{pgfscope}%
\begin{pgfscope}%
\pgfsetbuttcap%
\pgfsetroundjoin%
\definecolor{currentfill}{rgb}{0.000000,0.000000,0.000000}%
\pgfsetfillcolor{currentfill}%
\pgfsetlinewidth{0.501875pt}%
\definecolor{currentstroke}{rgb}{0.000000,0.000000,0.000000}%
\pgfsetstrokecolor{currentstroke}%
\pgfsetdash{}{0pt}%
\pgfsys@defobject{currentmarker}{\pgfqpoint{0.000000in}{0.000000in}}{\pgfqpoint{0.069444in}{0.000000in}}{%
\pgfpathmoveto{\pgfqpoint{0.000000in}{0.000000in}}%
\pgfpathlineto{\pgfqpoint{0.069444in}{0.000000in}}%
\pgfusepath{stroke,fill}%
}%
\begin{pgfscope}%
\pgfsys@transformshift{0.331521in}{0.422640in}%
\pgfsys@useobject{currentmarker}{}%
\end{pgfscope}%
\end{pgfscope}%
\begin{pgfscope}%
\pgfsetbuttcap%
\pgfsetroundjoin%
\definecolor{currentfill}{rgb}{0.000000,0.000000,0.000000}%
\pgfsetfillcolor{currentfill}%
\pgfsetlinewidth{0.501875pt}%
\definecolor{currentstroke}{rgb}{0.000000,0.000000,0.000000}%
\pgfsetstrokecolor{currentstroke}%
\pgfsetdash{}{0pt}%
\pgfsys@defobject{currentmarker}{\pgfqpoint{-0.069444in}{0.000000in}}{\pgfqpoint{0.000000in}{0.000000in}}{%
\pgfpathmoveto{\pgfqpoint{0.000000in}{0.000000in}}%
\pgfpathlineto{\pgfqpoint{-0.069444in}{0.000000in}}%
\pgfusepath{stroke,fill}%
}%
\begin{pgfscope}%
\pgfsys@transformshift{2.467022in}{0.422640in}%
\pgfsys@useobject{currentmarker}{}%
\end{pgfscope}%
\end{pgfscope}%
\begin{pgfscope}%
\pgftext[x=0.262077in,y=0.422640in,right,]{\rmfamily\fontsize{8.000000}{9.600000}\selectfont 0}%
\end{pgfscope}%
\begin{pgfscope}%
\pgfsetbuttcap%
\pgfsetroundjoin%
\definecolor{currentfill}{rgb}{0.000000,0.000000,0.000000}%
\pgfsetfillcolor{currentfill}%
\pgfsetlinewidth{0.501875pt}%
\definecolor{currentstroke}{rgb}{0.000000,0.000000,0.000000}%
\pgfsetstrokecolor{currentstroke}%
\pgfsetdash{}{0pt}%
\pgfsys@defobject{currentmarker}{\pgfqpoint{0.000000in}{0.000000in}}{\pgfqpoint{0.069444in}{0.000000in}}{%
\pgfpathmoveto{\pgfqpoint{0.000000in}{0.000000in}}%
\pgfpathlineto{\pgfqpoint{0.069444in}{0.000000in}}%
\pgfusepath{stroke,fill}%
}%
\begin{pgfscope}%
\pgfsys@transformshift{0.331521in}{0.573863in}%
\pgfsys@useobject{currentmarker}{}%
\end{pgfscope}%
\end{pgfscope}%
\begin{pgfscope}%
\pgfsetbuttcap%
\pgfsetroundjoin%
\definecolor{currentfill}{rgb}{0.000000,0.000000,0.000000}%
\pgfsetfillcolor{currentfill}%
\pgfsetlinewidth{0.501875pt}%
\definecolor{currentstroke}{rgb}{0.000000,0.000000,0.000000}%
\pgfsetstrokecolor{currentstroke}%
\pgfsetdash{}{0pt}%
\pgfsys@defobject{currentmarker}{\pgfqpoint{-0.069444in}{0.000000in}}{\pgfqpoint{0.000000in}{0.000000in}}{%
\pgfpathmoveto{\pgfqpoint{0.000000in}{0.000000in}}%
\pgfpathlineto{\pgfqpoint{-0.069444in}{0.000000in}}%
\pgfusepath{stroke,fill}%
}%
\begin{pgfscope}%
\pgfsys@transformshift{2.467022in}{0.573863in}%
\pgfsys@useobject{currentmarker}{}%
\end{pgfscope}%
\end{pgfscope}%
\begin{pgfscope}%
\pgftext[x=0.262077in,y=0.573863in,right,]{\rmfamily\fontsize{8.000000}{9.600000}\selectfont 50}%
\end{pgfscope}%
\begin{pgfscope}%
\pgfsetbuttcap%
\pgfsetroundjoin%
\definecolor{currentfill}{rgb}{0.000000,0.000000,0.000000}%
\pgfsetfillcolor{currentfill}%
\pgfsetlinewidth{0.501875pt}%
\definecolor{currentstroke}{rgb}{0.000000,0.000000,0.000000}%
\pgfsetstrokecolor{currentstroke}%
\pgfsetdash{}{0pt}%
\pgfsys@defobject{currentmarker}{\pgfqpoint{0.000000in}{0.000000in}}{\pgfqpoint{0.069444in}{0.000000in}}{%
\pgfpathmoveto{\pgfqpoint{0.000000in}{0.000000in}}%
\pgfpathlineto{\pgfqpoint{0.069444in}{0.000000in}}%
\pgfusepath{stroke,fill}%
}%
\begin{pgfscope}%
\pgfsys@transformshift{0.331521in}{0.725087in}%
\pgfsys@useobject{currentmarker}{}%
\end{pgfscope}%
\end{pgfscope}%
\begin{pgfscope}%
\pgfsetbuttcap%
\pgfsetroundjoin%
\definecolor{currentfill}{rgb}{0.000000,0.000000,0.000000}%
\pgfsetfillcolor{currentfill}%
\pgfsetlinewidth{0.501875pt}%
\definecolor{currentstroke}{rgb}{0.000000,0.000000,0.000000}%
\pgfsetstrokecolor{currentstroke}%
\pgfsetdash{}{0pt}%
\pgfsys@defobject{currentmarker}{\pgfqpoint{-0.069444in}{0.000000in}}{\pgfqpoint{0.000000in}{0.000000in}}{%
\pgfpathmoveto{\pgfqpoint{0.000000in}{0.000000in}}%
\pgfpathlineto{\pgfqpoint{-0.069444in}{0.000000in}}%
\pgfusepath{stroke,fill}%
}%
\begin{pgfscope}%
\pgfsys@transformshift{2.467022in}{0.725087in}%
\pgfsys@useobject{currentmarker}{}%
\end{pgfscope}%
\end{pgfscope}%
\begin{pgfscope}%
\pgftext[x=0.262077in,y=0.725087in,right,]{\rmfamily\fontsize{8.000000}{9.600000}\selectfont 100}%
\end{pgfscope}%
\begin{pgfscope}%
\pgfsetbuttcap%
\pgfsetroundjoin%
\definecolor{currentfill}{rgb}{0.000000,0.000000,0.000000}%
\pgfsetfillcolor{currentfill}%
\pgfsetlinewidth{0.501875pt}%
\definecolor{currentstroke}{rgb}{0.000000,0.000000,0.000000}%
\pgfsetstrokecolor{currentstroke}%
\pgfsetdash{}{0pt}%
\pgfsys@defobject{currentmarker}{\pgfqpoint{0.000000in}{0.000000in}}{\pgfqpoint{0.069444in}{0.000000in}}{%
\pgfpathmoveto{\pgfqpoint{0.000000in}{0.000000in}}%
\pgfpathlineto{\pgfqpoint{0.069444in}{0.000000in}}%
\pgfusepath{stroke,fill}%
}%
\begin{pgfscope}%
\pgfsys@transformshift{0.331521in}{0.876310in}%
\pgfsys@useobject{currentmarker}{}%
\end{pgfscope}%
\end{pgfscope}%
\begin{pgfscope}%
\pgfsetbuttcap%
\pgfsetroundjoin%
\definecolor{currentfill}{rgb}{0.000000,0.000000,0.000000}%
\pgfsetfillcolor{currentfill}%
\pgfsetlinewidth{0.501875pt}%
\definecolor{currentstroke}{rgb}{0.000000,0.000000,0.000000}%
\pgfsetstrokecolor{currentstroke}%
\pgfsetdash{}{0pt}%
\pgfsys@defobject{currentmarker}{\pgfqpoint{-0.069444in}{0.000000in}}{\pgfqpoint{0.000000in}{0.000000in}}{%
\pgfpathmoveto{\pgfqpoint{0.000000in}{0.000000in}}%
\pgfpathlineto{\pgfqpoint{-0.069444in}{0.000000in}}%
\pgfusepath{stroke,fill}%
}%
\begin{pgfscope}%
\pgfsys@transformshift{2.467022in}{0.876310in}%
\pgfsys@useobject{currentmarker}{}%
\end{pgfscope}%
\end{pgfscope}%
\begin{pgfscope}%
\pgftext[x=0.262077in,y=0.876310in,right,]{\rmfamily\fontsize{8.000000}{9.600000}\selectfont 150}%
\end{pgfscope}%
\begin{pgfscope}%
\pgfsetbuttcap%
\pgfsetroundjoin%
\definecolor{currentfill}{rgb}{0.000000,0.000000,0.000000}%
\pgfsetfillcolor{currentfill}%
\pgfsetlinewidth{0.501875pt}%
\definecolor{currentstroke}{rgb}{0.000000,0.000000,0.000000}%
\pgfsetstrokecolor{currentstroke}%
\pgfsetdash{}{0pt}%
\pgfsys@defobject{currentmarker}{\pgfqpoint{0.000000in}{0.000000in}}{\pgfqpoint{0.069444in}{0.000000in}}{%
\pgfpathmoveto{\pgfqpoint{0.000000in}{0.000000in}}%
\pgfpathlineto{\pgfqpoint{0.069444in}{0.000000in}}%
\pgfusepath{stroke,fill}%
}%
\begin{pgfscope}%
\pgfsys@transformshift{0.331521in}{1.027533in}%
\pgfsys@useobject{currentmarker}{}%
\end{pgfscope}%
\end{pgfscope}%
\begin{pgfscope}%
\pgfsetbuttcap%
\pgfsetroundjoin%
\definecolor{currentfill}{rgb}{0.000000,0.000000,0.000000}%
\pgfsetfillcolor{currentfill}%
\pgfsetlinewidth{0.501875pt}%
\definecolor{currentstroke}{rgb}{0.000000,0.000000,0.000000}%
\pgfsetstrokecolor{currentstroke}%
\pgfsetdash{}{0pt}%
\pgfsys@defobject{currentmarker}{\pgfqpoint{-0.069444in}{0.000000in}}{\pgfqpoint{0.000000in}{0.000000in}}{%
\pgfpathmoveto{\pgfqpoint{0.000000in}{0.000000in}}%
\pgfpathlineto{\pgfqpoint{-0.069444in}{0.000000in}}%
\pgfusepath{stroke,fill}%
}%
\begin{pgfscope}%
\pgfsys@transformshift{2.467022in}{1.027533in}%
\pgfsys@useobject{currentmarker}{}%
\end{pgfscope}%
\end{pgfscope}%
\begin{pgfscope}%
\pgftext[x=0.262077in,y=1.027533in,right,]{\rmfamily\fontsize{8.000000}{9.600000}\selectfont 200}%
\end{pgfscope}%
\begin{pgfscope}%
\pgfsetbuttcap%
\pgfsetroundjoin%
\definecolor{currentfill}{rgb}{0.000000,0.000000,0.000000}%
\pgfsetfillcolor{currentfill}%
\pgfsetlinewidth{0.501875pt}%
\definecolor{currentstroke}{rgb}{0.000000,0.000000,0.000000}%
\pgfsetstrokecolor{currentstroke}%
\pgfsetdash{}{0pt}%
\pgfsys@defobject{currentmarker}{\pgfqpoint{0.000000in}{0.000000in}}{\pgfqpoint{0.069444in}{0.000000in}}{%
\pgfpathmoveto{\pgfqpoint{0.000000in}{0.000000in}}%
\pgfpathlineto{\pgfqpoint{0.069444in}{0.000000in}}%
\pgfusepath{stroke,fill}%
}%
\begin{pgfscope}%
\pgfsys@transformshift{0.331521in}{1.178756in}%
\pgfsys@useobject{currentmarker}{}%
\end{pgfscope}%
\end{pgfscope}%
\begin{pgfscope}%
\pgfsetbuttcap%
\pgfsetroundjoin%
\definecolor{currentfill}{rgb}{0.000000,0.000000,0.000000}%
\pgfsetfillcolor{currentfill}%
\pgfsetlinewidth{0.501875pt}%
\definecolor{currentstroke}{rgb}{0.000000,0.000000,0.000000}%
\pgfsetstrokecolor{currentstroke}%
\pgfsetdash{}{0pt}%
\pgfsys@defobject{currentmarker}{\pgfqpoint{-0.069444in}{0.000000in}}{\pgfqpoint{0.000000in}{0.000000in}}{%
\pgfpathmoveto{\pgfqpoint{0.000000in}{0.000000in}}%
\pgfpathlineto{\pgfqpoint{-0.069444in}{0.000000in}}%
\pgfusepath{stroke,fill}%
}%
\begin{pgfscope}%
\pgfsys@transformshift{2.467022in}{1.178756in}%
\pgfsys@useobject{currentmarker}{}%
\end{pgfscope}%
\end{pgfscope}%
\begin{pgfscope}%
\pgftext[x=0.262077in,y=1.178756in,right,]{\rmfamily\fontsize{8.000000}{9.600000}\selectfont 250}%
\end{pgfscope}%
\begin{pgfscope}%
\pgfsetbuttcap%
\pgfsetroundjoin%
\definecolor{currentfill}{rgb}{0.000000,0.000000,0.000000}%
\pgfsetfillcolor{currentfill}%
\pgfsetlinewidth{0.501875pt}%
\definecolor{currentstroke}{rgb}{0.000000,0.000000,0.000000}%
\pgfsetstrokecolor{currentstroke}%
\pgfsetdash{}{0pt}%
\pgfsys@defobject{currentmarker}{\pgfqpoint{0.000000in}{0.000000in}}{\pgfqpoint{0.069444in}{0.000000in}}{%
\pgfpathmoveto{\pgfqpoint{0.000000in}{0.000000in}}%
\pgfpathlineto{\pgfqpoint{0.069444in}{0.000000in}}%
\pgfusepath{stroke,fill}%
}%
\begin{pgfscope}%
\pgfsys@transformshift{0.331521in}{1.329980in}%
\pgfsys@useobject{currentmarker}{}%
\end{pgfscope}%
\end{pgfscope}%
\begin{pgfscope}%
\pgfsetbuttcap%
\pgfsetroundjoin%
\definecolor{currentfill}{rgb}{0.000000,0.000000,0.000000}%
\pgfsetfillcolor{currentfill}%
\pgfsetlinewidth{0.501875pt}%
\definecolor{currentstroke}{rgb}{0.000000,0.000000,0.000000}%
\pgfsetstrokecolor{currentstroke}%
\pgfsetdash{}{0pt}%
\pgfsys@defobject{currentmarker}{\pgfqpoint{-0.069444in}{0.000000in}}{\pgfqpoint{0.000000in}{0.000000in}}{%
\pgfpathmoveto{\pgfqpoint{0.000000in}{0.000000in}}%
\pgfpathlineto{\pgfqpoint{-0.069444in}{0.000000in}}%
\pgfusepath{stroke,fill}%
}%
\begin{pgfscope}%
\pgfsys@transformshift{2.467022in}{1.329980in}%
\pgfsys@useobject{currentmarker}{}%
\end{pgfscope}%
\end{pgfscope}%
\begin{pgfscope}%
\pgftext[x=0.262077in,y=1.329980in,right,]{\rmfamily\fontsize{8.000000}{9.600000}\selectfont 300}%
\end{pgfscope}%
\begin{pgfscope}%
\pgfsetbuttcap%
\pgfsetroundjoin%
\definecolor{currentfill}{rgb}{0.000000,0.000000,0.000000}%
\pgfsetfillcolor{currentfill}%
\pgfsetlinewidth{0.501875pt}%
\definecolor{currentstroke}{rgb}{0.000000,0.000000,0.000000}%
\pgfsetstrokecolor{currentstroke}%
\pgfsetdash{}{0pt}%
\pgfsys@defobject{currentmarker}{\pgfqpoint{0.000000in}{0.000000in}}{\pgfqpoint{0.069444in}{0.000000in}}{%
\pgfpathmoveto{\pgfqpoint{0.000000in}{0.000000in}}%
\pgfpathlineto{\pgfqpoint{0.069444in}{0.000000in}}%
\pgfusepath{stroke,fill}%
}%
\begin{pgfscope}%
\pgfsys@transformshift{0.331521in}{1.481203in}%
\pgfsys@useobject{currentmarker}{}%
\end{pgfscope}%
\end{pgfscope}%
\begin{pgfscope}%
\pgfsetbuttcap%
\pgfsetroundjoin%
\definecolor{currentfill}{rgb}{0.000000,0.000000,0.000000}%
\pgfsetfillcolor{currentfill}%
\pgfsetlinewidth{0.501875pt}%
\definecolor{currentstroke}{rgb}{0.000000,0.000000,0.000000}%
\pgfsetstrokecolor{currentstroke}%
\pgfsetdash{}{0pt}%
\pgfsys@defobject{currentmarker}{\pgfqpoint{-0.069444in}{0.000000in}}{\pgfqpoint{0.000000in}{0.000000in}}{%
\pgfpathmoveto{\pgfqpoint{0.000000in}{0.000000in}}%
\pgfpathlineto{\pgfqpoint{-0.069444in}{0.000000in}}%
\pgfusepath{stroke,fill}%
}%
\begin{pgfscope}%
\pgfsys@transformshift{2.467022in}{1.481203in}%
\pgfsys@useobject{currentmarker}{}%
\end{pgfscope}%
\end{pgfscope}%
\begin{pgfscope}%
\pgftext[x=0.262077in,y=1.481203in,right,]{\rmfamily\fontsize{8.000000}{9.600000}\selectfont 350}%
\end{pgfscope}%
\begin{pgfscope}%
\pgfsetbuttcap%
\pgfsetroundjoin%
\definecolor{currentfill}{rgb}{0.000000,0.000000,0.000000}%
\pgfsetfillcolor{currentfill}%
\pgfsetlinewidth{0.501875pt}%
\definecolor{currentstroke}{rgb}{0.000000,0.000000,0.000000}%
\pgfsetstrokecolor{currentstroke}%
\pgfsetdash{}{0pt}%
\pgfsys@defobject{currentmarker}{\pgfqpoint{0.000000in}{0.000000in}}{\pgfqpoint{0.069444in}{0.000000in}}{%
\pgfpathmoveto{\pgfqpoint{0.000000in}{0.000000in}}%
\pgfpathlineto{\pgfqpoint{0.069444in}{0.000000in}}%
\pgfusepath{stroke,fill}%
}%
\begin{pgfscope}%
\pgfsys@transformshift{0.331521in}{1.632426in}%
\pgfsys@useobject{currentmarker}{}%
\end{pgfscope}%
\end{pgfscope}%
\begin{pgfscope}%
\pgfsetbuttcap%
\pgfsetroundjoin%
\definecolor{currentfill}{rgb}{0.000000,0.000000,0.000000}%
\pgfsetfillcolor{currentfill}%
\pgfsetlinewidth{0.501875pt}%
\definecolor{currentstroke}{rgb}{0.000000,0.000000,0.000000}%
\pgfsetstrokecolor{currentstroke}%
\pgfsetdash{}{0pt}%
\pgfsys@defobject{currentmarker}{\pgfqpoint{-0.069444in}{0.000000in}}{\pgfqpoint{0.000000in}{0.000000in}}{%
\pgfpathmoveto{\pgfqpoint{0.000000in}{0.000000in}}%
\pgfpathlineto{\pgfqpoint{-0.069444in}{0.000000in}}%
\pgfusepath{stroke,fill}%
}%
\begin{pgfscope}%
\pgfsys@transformshift{2.467022in}{1.632426in}%
\pgfsys@useobject{currentmarker}{}%
\end{pgfscope}%
\end{pgfscope}%
\begin{pgfscope}%
\pgftext[x=0.262077in,y=1.632426in,right,]{\rmfamily\fontsize{8.000000}{9.600000}\selectfont 400}%
\end{pgfscope}%
\end{pgfpicture}%
\makeatother%
\endgroup%

	\end{subfigure}
	\begin{subfigure}[t]{0.49\textwidth}
		\centering
    %\includegraphics[width=\textwidth]{store/variables/SIG_BKG_B_ENDVERTEX_CHI2_NDOF.pdf}
    %% Creator: Matplotlib, PGF backend
%%
%% To include the figure in your LaTeX document, write
%%   \input{<filename>.pgf}
%%
%% Make sure the required packages are loaded in your preamble
%%   \usepackage{pgf}
%%
%% Figures using additional raster images can only be included by \input if
%% they are in the same directory as the main LaTeX file. For loading figures
%% from other directories you can use the `import` package
%%   \usepackage{import}
%% and then include the figures with
%%   \import{<path to file>}{<filename>.pgf}
%%
%% Matplotlib used the following preamble
%%   \usepackage{fontspec}
%%   \setmainfont{DejaVu Serif}
%%   \setsansfont{DejaVu Sans}
%%   \setmonofont{DejaVu Sans Mono}
%%
\begingroup%
\makeatletter%
\begin{pgfpicture}%
\pgfpathrectangle{\pgfpointorigin}{\pgfqpoint{2.678086in}{1.718727in}}%
\pgfusepath{use as bounding box, clip}%
\begin{pgfscope}%
\pgfsetbuttcap%
\pgfsetmiterjoin%
\definecolor{currentfill}{rgb}{1.000000,1.000000,1.000000}%
\pgfsetfillcolor{currentfill}%
\pgfsetlinewidth{0.000000pt}%
\definecolor{currentstroke}{rgb}{1.000000,1.000000,1.000000}%
\pgfsetstrokecolor{currentstroke}%
\pgfsetdash{}{0pt}%
\pgfpathmoveto{\pgfqpoint{0.000000in}{-0.000000in}}%
\pgfpathlineto{\pgfqpoint{2.678086in}{-0.000000in}}%
\pgfpathlineto{\pgfqpoint{2.678086in}{1.718727in}}%
\pgfpathlineto{\pgfqpoint{0.000000in}{1.718727in}}%
\pgfpathclose%
\pgfusepath{fill}%
\end{pgfscope}%
\begin{pgfscope}%
\pgfsetbuttcap%
\pgfsetmiterjoin%
\definecolor{currentfill}{rgb}{1.000000,1.000000,1.000000}%
\pgfsetfillcolor{currentfill}%
\pgfsetlinewidth{0.000000pt}%
\definecolor{currentstroke}{rgb}{0.000000,0.000000,0.000000}%
\pgfsetstrokecolor{currentstroke}%
\pgfsetstrokeopacity{0.000000}%
\pgfsetdash{}{0pt}%
\pgfpathmoveto{\pgfqpoint{0.296148in}{0.441418in}}%
\pgfpathlineto{\pgfqpoint{2.592740in}{0.441418in}}%
\pgfpathlineto{\pgfqpoint{2.592740in}{1.614961in}}%
\pgfpathlineto{\pgfqpoint{0.296148in}{1.614961in}}%
\pgfpathclose%
\pgfusepath{fill}%
\end{pgfscope}%
\begin{pgfscope}%
\pgfpathrectangle{\pgfqpoint{0.296148in}{0.441418in}}{\pgfqpoint{2.296592in}{1.173543in}} %
\pgfusepath{clip}%
\pgfsetbuttcap%
\pgfsetmiterjoin%
\definecolor{currentfill}{rgb}{0.215686,0.470588,0.749020}%
\pgfsetfillcolor{currentfill}%
\pgfsetlinewidth{0.000000pt}%
\definecolor{currentstroke}{rgb}{0.000000,0.000000,0.000000}%
\pgfsetstrokecolor{currentstroke}%
\pgfsetdash{}{0pt}%
\pgfpathmoveto{\pgfqpoint{0.296368in}{0.441418in}}%
\pgfpathlineto{\pgfqpoint{0.296368in}{1.098733in}}%
\pgfpathlineto{\pgfqpoint{0.342249in}{1.098733in}}%
\pgfpathlineto{\pgfqpoint{0.342249in}{1.402896in}}%
\pgfpathlineto{\pgfqpoint{0.388129in}{1.402896in}}%
\pgfpathlineto{\pgfqpoint{0.388129in}{1.462991in}}%
\pgfpathlineto{\pgfqpoint{0.434010in}{1.462991in}}%
\pgfpathlineto{\pgfqpoint{0.434010in}{1.460585in}}%
\pgfpathlineto{\pgfqpoint{0.479890in}{1.460585in}}%
\pgfpathlineto{\pgfqpoint{0.479890in}{1.366568in}}%
\pgfpathlineto{\pgfqpoint{0.525771in}{1.366568in}}%
\pgfpathlineto{\pgfqpoint{0.525771in}{1.341194in}}%
\pgfpathlineto{\pgfqpoint{0.571652in}{1.341194in}}%
\pgfpathlineto{\pgfqpoint{0.571652in}{1.232427in}}%
\pgfpathlineto{\pgfqpoint{0.617532in}{1.232427in}}%
\pgfpathlineto{\pgfqpoint{0.617532in}{1.114696in}}%
\pgfpathlineto{\pgfqpoint{0.663413in}{1.114696in}}%
\pgfpathlineto{\pgfqpoint{0.663413in}{1.091001in}}%
\pgfpathlineto{\pgfqpoint{0.709293in}{1.091001in}}%
\pgfpathlineto{\pgfqpoint{0.709293in}{1.033005in}}%
\pgfpathlineto{\pgfqpoint{0.755174in}{1.033005in}}%
\pgfpathlineto{\pgfqpoint{0.755174in}{0.929308in}}%
\pgfpathlineto{\pgfqpoint{0.801054in}{0.929308in}}%
\pgfpathlineto{\pgfqpoint{0.801054in}{0.862763in}}%
\pgfpathlineto{\pgfqpoint{0.846935in}{0.862763in}}%
\pgfpathlineto{\pgfqpoint{0.846935in}{0.820487in}}%
\pgfpathlineto{\pgfqpoint{0.892815in}{0.820487in}}%
\pgfpathlineto{\pgfqpoint{0.892815in}{0.757204in}}%
\pgfpathlineto{\pgfqpoint{0.938696in}{0.757204in}}%
\pgfpathlineto{\pgfqpoint{0.938696in}{0.730248in}}%
\pgfpathlineto{\pgfqpoint{0.984577in}{0.730248in}}%
\pgfpathlineto{\pgfqpoint{0.984577in}{0.691708in}}%
\pgfpathlineto{\pgfqpoint{1.030457in}{0.691708in}}%
\pgfpathlineto{\pgfqpoint{1.030457in}{0.666340in}}%
\pgfpathlineto{\pgfqpoint{1.076338in}{0.666340in}}%
\pgfpathlineto{\pgfqpoint{1.076338in}{0.623472in}}%
\pgfpathlineto{\pgfqpoint{1.122218in}{0.623472in}}%
\pgfpathlineto{\pgfqpoint{1.122218in}{0.640388in}}%
\pgfpathlineto{\pgfqpoint{1.168099in}{0.640388in}}%
\pgfpathlineto{\pgfqpoint{1.168099in}{0.604016in}}%
\pgfpathlineto{\pgfqpoint{1.213979in}{0.604016in}}%
\pgfpathlineto{\pgfqpoint{1.213979in}{0.568967in}}%
\pgfpathlineto{\pgfqpoint{1.259860in}{0.568967in}}%
\pgfpathlineto{\pgfqpoint{1.259860in}{0.559977in}}%
\pgfpathlineto{\pgfqpoint{1.305740in}{0.559977in}}%
\pgfpathlineto{\pgfqpoint{1.305740in}{0.537866in}}%
\pgfpathlineto{\pgfqpoint{1.351621in}{0.537866in}}%
\pgfpathlineto{\pgfqpoint{1.351621in}{0.518586in}}%
\pgfpathlineto{\pgfqpoint{1.397502in}{0.518586in}}%
\pgfpathlineto{\pgfqpoint{1.397502in}{0.528381in}}%
\pgfpathlineto{\pgfqpoint{1.443382in}{0.528381in}}%
\pgfpathlineto{\pgfqpoint{1.443382in}{0.519109in}}%
\pgfpathlineto{\pgfqpoint{1.489263in}{0.519109in}}%
\pgfpathlineto{\pgfqpoint{1.489263in}{0.505142in}}%
\pgfpathlineto{\pgfqpoint{1.535143in}{0.505142in}}%
\pgfpathlineto{\pgfqpoint{1.535143in}{0.503899in}}%
\pgfpathlineto{\pgfqpoint{1.581024in}{0.503899in}}%
\pgfpathlineto{\pgfqpoint{1.581024in}{0.480799in}}%
\pgfpathlineto{\pgfqpoint{1.626904in}{0.480799in}}%
\pgfpathlineto{\pgfqpoint{1.626904in}{0.498384in}}%
\pgfpathlineto{\pgfqpoint{1.672785in}{0.498384in}}%
\pgfpathlineto{\pgfqpoint{1.672785in}{0.482902in}}%
\pgfpathlineto{\pgfqpoint{1.718665in}{0.482902in}}%
\pgfpathlineto{\pgfqpoint{1.718665in}{0.478170in}}%
\pgfpathlineto{\pgfqpoint{1.764546in}{0.478170in}}%
\pgfpathlineto{\pgfqpoint{1.764546in}{0.469119in}}%
\pgfpathlineto{\pgfqpoint{1.810427in}{0.469119in}}%
\pgfpathlineto{\pgfqpoint{1.810427in}{0.463954in}}%
\pgfpathlineto{\pgfqpoint{1.856307in}{0.463954in}}%
\pgfpathlineto{\pgfqpoint{1.856307in}{0.458755in}}%
\pgfpathlineto{\pgfqpoint{1.902188in}{0.458755in}}%
\pgfpathlineto{\pgfqpoint{1.902188in}{0.461093in}}%
\pgfpathlineto{\pgfqpoint{1.948068in}{0.461093in}}%
\pgfpathlineto{\pgfqpoint{1.948068in}{0.461995in}}%
\pgfpathlineto{\pgfqpoint{1.993949in}{0.461995in}}%
\pgfpathlineto{\pgfqpoint{1.993949in}{0.463209in}}%
\pgfpathlineto{\pgfqpoint{2.039829in}{0.463209in}}%
\pgfpathlineto{\pgfqpoint{2.039829in}{0.458847in}}%
\pgfpathlineto{\pgfqpoint{2.085710in}{0.458847in}}%
\pgfpathlineto{\pgfqpoint{2.085710in}{0.457666in}}%
\pgfpathlineto{\pgfqpoint{2.131590in}{0.457666in}}%
\pgfpathlineto{\pgfqpoint{2.131590in}{0.451195in}}%
\pgfpathlineto{\pgfqpoint{2.177471in}{0.451195in}}%
\pgfpathlineto{\pgfqpoint{2.177471in}{0.451030in}}%
\pgfpathlineto{\pgfqpoint{2.223352in}{0.451030in}}%
\pgfpathlineto{\pgfqpoint{2.223352in}{0.451429in}}%
\pgfpathlineto{\pgfqpoint{2.269232in}{0.451429in}}%
\pgfpathlineto{\pgfqpoint{2.269232in}{0.455977in}}%
\pgfpathlineto{\pgfqpoint{2.315113in}{0.455977in}}%
\pgfpathlineto{\pgfqpoint{2.315113in}{0.449781in}}%
\pgfpathlineto{\pgfqpoint{2.360993in}{0.449781in}}%
\pgfpathlineto{\pgfqpoint{2.360993in}{0.451176in}}%
\pgfpathlineto{\pgfqpoint{2.406874in}{0.451176in}}%
\pgfpathlineto{\pgfqpoint{2.406874in}{0.445863in}}%
\pgfpathlineto{\pgfqpoint{2.452754in}{0.445863in}}%
\pgfpathlineto{\pgfqpoint{2.452754in}{0.449302in}}%
\pgfpathlineto{\pgfqpoint{2.498635in}{0.449302in}}%
\pgfpathlineto{\pgfqpoint{2.498635in}{0.450017in}}%
\pgfpathlineto{\pgfqpoint{2.544515in}{0.450017in}}%
\pgfpathlineto{\pgfqpoint{2.544515in}{0.446374in}}%
\pgfpathlineto{\pgfqpoint{2.590396in}{0.446374in}}%
\pgfpathlineto{\pgfqpoint{2.590396in}{0.441418in}}%
\pgfpathlineto{\pgfqpoint{2.544515in}{0.441418in}}%
\pgfpathlineto{\pgfqpoint{2.544515in}{0.441418in}}%
\pgfpathlineto{\pgfqpoint{2.498635in}{0.441418in}}%
\pgfpathlineto{\pgfqpoint{2.498635in}{0.441418in}}%
\pgfpathlineto{\pgfqpoint{2.452754in}{0.441418in}}%
\pgfpathlineto{\pgfqpoint{2.452754in}{0.441418in}}%
\pgfpathlineto{\pgfqpoint{2.406874in}{0.441418in}}%
\pgfpathlineto{\pgfqpoint{2.406874in}{0.441418in}}%
\pgfpathlineto{\pgfqpoint{2.360993in}{0.441418in}}%
\pgfpathlineto{\pgfqpoint{2.360993in}{0.441418in}}%
\pgfpathlineto{\pgfqpoint{2.315113in}{0.441418in}}%
\pgfpathlineto{\pgfqpoint{2.315113in}{0.441418in}}%
\pgfpathlineto{\pgfqpoint{2.269232in}{0.441418in}}%
\pgfpathlineto{\pgfqpoint{2.269232in}{0.441418in}}%
\pgfpathlineto{\pgfqpoint{2.223352in}{0.441418in}}%
\pgfpathlineto{\pgfqpoint{2.223352in}{0.441418in}}%
\pgfpathlineto{\pgfqpoint{2.177471in}{0.441418in}}%
\pgfpathlineto{\pgfqpoint{2.177471in}{0.441418in}}%
\pgfpathlineto{\pgfqpoint{2.131590in}{0.441418in}}%
\pgfpathlineto{\pgfqpoint{2.131590in}{0.441418in}}%
\pgfpathlineto{\pgfqpoint{2.085710in}{0.441418in}}%
\pgfpathlineto{\pgfqpoint{2.085710in}{0.441418in}}%
\pgfpathlineto{\pgfqpoint{2.039829in}{0.441418in}}%
\pgfpathlineto{\pgfqpoint{2.039829in}{0.441418in}}%
\pgfpathlineto{\pgfqpoint{1.993949in}{0.441418in}}%
\pgfpathlineto{\pgfqpoint{1.993949in}{0.441418in}}%
\pgfpathlineto{\pgfqpoint{1.948068in}{0.441418in}}%
\pgfpathlineto{\pgfqpoint{1.948068in}{0.441418in}}%
\pgfpathlineto{\pgfqpoint{1.902188in}{0.441418in}}%
\pgfpathlineto{\pgfqpoint{1.902188in}{0.441418in}}%
\pgfpathlineto{\pgfqpoint{1.856307in}{0.441418in}}%
\pgfpathlineto{\pgfqpoint{1.856307in}{0.441418in}}%
\pgfpathlineto{\pgfqpoint{1.810427in}{0.441418in}}%
\pgfpathlineto{\pgfqpoint{1.810427in}{0.441418in}}%
\pgfpathlineto{\pgfqpoint{1.764546in}{0.441418in}}%
\pgfpathlineto{\pgfqpoint{1.764546in}{0.441418in}}%
\pgfpathlineto{\pgfqpoint{1.718665in}{0.441418in}}%
\pgfpathlineto{\pgfqpoint{1.718665in}{0.441418in}}%
\pgfpathlineto{\pgfqpoint{1.672785in}{0.441418in}}%
\pgfpathlineto{\pgfqpoint{1.672785in}{0.441418in}}%
\pgfpathlineto{\pgfqpoint{1.626904in}{0.441418in}}%
\pgfpathlineto{\pgfqpoint{1.626904in}{0.441418in}}%
\pgfpathlineto{\pgfqpoint{1.581024in}{0.441418in}}%
\pgfpathlineto{\pgfqpoint{1.581024in}{0.441418in}}%
\pgfpathlineto{\pgfqpoint{1.535143in}{0.441418in}}%
\pgfpathlineto{\pgfqpoint{1.535143in}{0.441418in}}%
\pgfpathlineto{\pgfqpoint{1.489263in}{0.441418in}}%
\pgfpathlineto{\pgfqpoint{1.489263in}{0.441418in}}%
\pgfpathlineto{\pgfqpoint{1.443382in}{0.441418in}}%
\pgfpathlineto{\pgfqpoint{1.443382in}{0.441418in}}%
\pgfpathlineto{\pgfqpoint{1.397502in}{0.441418in}}%
\pgfpathlineto{\pgfqpoint{1.397502in}{0.441418in}}%
\pgfpathlineto{\pgfqpoint{1.351621in}{0.441418in}}%
\pgfpathlineto{\pgfqpoint{1.351621in}{0.441418in}}%
\pgfpathlineto{\pgfqpoint{1.305740in}{0.441418in}}%
\pgfpathlineto{\pgfqpoint{1.305740in}{0.441418in}}%
\pgfpathlineto{\pgfqpoint{1.259860in}{0.441418in}}%
\pgfpathlineto{\pgfqpoint{1.259860in}{0.441418in}}%
\pgfpathlineto{\pgfqpoint{1.213979in}{0.441418in}}%
\pgfpathlineto{\pgfqpoint{1.213979in}{0.441418in}}%
\pgfpathlineto{\pgfqpoint{1.168099in}{0.441418in}}%
\pgfpathlineto{\pgfqpoint{1.168099in}{0.441418in}}%
\pgfpathlineto{\pgfqpoint{1.122218in}{0.441418in}}%
\pgfpathlineto{\pgfqpoint{1.122218in}{0.441418in}}%
\pgfpathlineto{\pgfqpoint{1.076338in}{0.441418in}}%
\pgfpathlineto{\pgfqpoint{1.076338in}{0.441418in}}%
\pgfpathlineto{\pgfqpoint{1.030457in}{0.441418in}}%
\pgfpathlineto{\pgfqpoint{1.030457in}{0.441418in}}%
\pgfpathlineto{\pgfqpoint{0.984577in}{0.441418in}}%
\pgfpathlineto{\pgfqpoint{0.984577in}{0.441418in}}%
\pgfpathlineto{\pgfqpoint{0.938696in}{0.441418in}}%
\pgfpathlineto{\pgfqpoint{0.938696in}{0.441418in}}%
\pgfpathlineto{\pgfqpoint{0.892815in}{0.441418in}}%
\pgfpathlineto{\pgfqpoint{0.892815in}{0.441418in}}%
\pgfpathlineto{\pgfqpoint{0.846935in}{0.441418in}}%
\pgfpathlineto{\pgfqpoint{0.846935in}{0.441418in}}%
\pgfpathlineto{\pgfqpoint{0.801054in}{0.441418in}}%
\pgfpathlineto{\pgfqpoint{0.801054in}{0.441418in}}%
\pgfpathlineto{\pgfqpoint{0.755174in}{0.441418in}}%
\pgfpathlineto{\pgfqpoint{0.755174in}{0.441418in}}%
\pgfpathlineto{\pgfqpoint{0.709293in}{0.441418in}}%
\pgfpathlineto{\pgfqpoint{0.709293in}{0.441418in}}%
\pgfpathlineto{\pgfqpoint{0.663413in}{0.441418in}}%
\pgfpathlineto{\pgfqpoint{0.663413in}{0.441418in}}%
\pgfpathlineto{\pgfqpoint{0.617532in}{0.441418in}}%
\pgfpathlineto{\pgfqpoint{0.617532in}{0.441418in}}%
\pgfpathlineto{\pgfqpoint{0.571652in}{0.441418in}}%
\pgfpathlineto{\pgfqpoint{0.571652in}{0.441418in}}%
\pgfpathlineto{\pgfqpoint{0.525771in}{0.441418in}}%
\pgfpathlineto{\pgfqpoint{0.525771in}{0.441418in}}%
\pgfpathlineto{\pgfqpoint{0.479890in}{0.441418in}}%
\pgfpathlineto{\pgfqpoint{0.479890in}{0.441418in}}%
\pgfpathlineto{\pgfqpoint{0.434010in}{0.441418in}}%
\pgfpathlineto{\pgfqpoint{0.434010in}{0.441418in}}%
\pgfpathlineto{\pgfqpoint{0.388129in}{0.441418in}}%
\pgfpathlineto{\pgfqpoint{0.388129in}{0.441418in}}%
\pgfpathlineto{\pgfqpoint{0.342249in}{0.441418in}}%
\pgfpathlineto{\pgfqpoint{0.342249in}{0.441418in}}%
\pgfpathlineto{\pgfqpoint{0.296368in}{0.441418in}}%
\pgfusepath{fill}%
\end{pgfscope}%
\begin{pgfscope}%
\pgfpathrectangle{\pgfqpoint{0.296148in}{0.441418in}}{\pgfqpoint{2.296592in}{1.173543in}} %
\pgfusepath{clip}%
\pgfsetbuttcap%
\pgfsetmiterjoin%
\pgfsetlinewidth{0.501875pt}%
\definecolor{currentstroke}{rgb}{1.000000,0.000000,0.000000}%
\pgfsetstrokecolor{currentstroke}%
\pgfsetdash{}{0pt}%
\pgfpathmoveto{\pgfqpoint{0.296368in}{0.441418in}}%
\pgfpathlineto{\pgfqpoint{0.296368in}{0.553029in}}%
\pgfpathlineto{\pgfqpoint{0.342249in}{0.553029in}}%
\pgfpathlineto{\pgfqpoint{0.342249in}{0.626106in}}%
\pgfpathlineto{\pgfqpoint{0.388129in}{0.626106in}}%
\pgfpathlineto{\pgfqpoint{0.388129in}{0.656246in}}%
\pgfpathlineto{\pgfqpoint{0.434010in}{0.656246in}}%
\pgfpathlineto{\pgfqpoint{0.434010in}{0.680726in}}%
\pgfpathlineto{\pgfqpoint{0.479890in}{0.680726in}}%
\pgfpathlineto{\pgfqpoint{0.479890in}{0.695225in}}%
\pgfpathlineto{\pgfqpoint{0.525771in}{0.695225in}}%
\pgfpathlineto{\pgfqpoint{0.525771in}{0.704091in}}%
\pgfpathlineto{\pgfqpoint{0.571652in}{0.704091in}}%
\pgfpathlineto{\pgfqpoint{0.571652in}{0.710699in}}%
\pgfpathlineto{\pgfqpoint{0.617532in}{0.710699in}}%
\pgfpathlineto{\pgfqpoint{0.617532in}{0.718143in}}%
\pgfpathlineto{\pgfqpoint{0.663413in}{0.718143in}}%
\pgfpathlineto{\pgfqpoint{0.663413in}{0.713208in}}%
\pgfpathlineto{\pgfqpoint{0.709293in}{0.713208in}}%
\pgfpathlineto{\pgfqpoint{0.709293in}{0.721099in}}%
\pgfpathlineto{\pgfqpoint{0.755174in}{0.721099in}}%
\pgfpathlineto{\pgfqpoint{0.755174in}{0.722827in}}%
\pgfpathlineto{\pgfqpoint{0.801054in}{0.722827in}}%
\pgfpathlineto{\pgfqpoint{0.801054in}{0.723357in}}%
\pgfpathlineto{\pgfqpoint{0.846935in}{0.723357in}}%
\pgfpathlineto{\pgfqpoint{0.846935in}{0.724305in}}%
\pgfpathlineto{\pgfqpoint{0.892815in}{0.724305in}}%
\pgfpathlineto{\pgfqpoint{0.892815in}{0.719844in}}%
\pgfpathlineto{\pgfqpoint{0.938696in}{0.719844in}}%
\pgfpathlineto{\pgfqpoint{0.938696in}{0.722995in}}%
\pgfpathlineto{\pgfqpoint{0.984577in}{0.722995in}}%
\pgfpathlineto{\pgfqpoint{0.984577in}{0.724249in}}%
\pgfpathlineto{\pgfqpoint{1.030457in}{0.724249in}}%
\pgfpathlineto{\pgfqpoint{1.030457in}{0.725922in}}%
\pgfpathlineto{\pgfqpoint{1.076338in}{0.725922in}}%
\pgfpathlineto{\pgfqpoint{1.076338in}{0.727763in}}%
\pgfpathlineto{\pgfqpoint{1.122218in}{0.727763in}}%
\pgfpathlineto{\pgfqpoint{1.122218in}{0.722186in}}%
\pgfpathlineto{\pgfqpoint{1.168099in}{0.722186in}}%
\pgfpathlineto{\pgfqpoint{1.168099in}{0.721043in}}%
\pgfpathlineto{\pgfqpoint{1.213979in}{0.721043in}}%
\pgfpathlineto{\pgfqpoint{1.213979in}{0.727735in}}%
\pgfpathlineto{\pgfqpoint{1.259860in}{0.727735in}}%
\pgfpathlineto{\pgfqpoint{1.259860in}{0.724445in}}%
\pgfpathlineto{\pgfqpoint{1.305740in}{0.724445in}}%
\pgfpathlineto{\pgfqpoint{1.305740in}{0.720430in}}%
\pgfpathlineto{\pgfqpoint{1.351621in}{0.720430in}}%
\pgfpathlineto{\pgfqpoint{1.351621in}{0.721629in}}%
\pgfpathlineto{\pgfqpoint{1.397502in}{0.721629in}}%
\pgfpathlineto{\pgfqpoint{1.397502in}{0.721740in}}%
\pgfpathlineto{\pgfqpoint{1.443382in}{0.721740in}}%
\pgfpathlineto{\pgfqpoint{1.443382in}{0.712567in}}%
\pgfpathlineto{\pgfqpoint{1.489263in}{0.712567in}}%
\pgfpathlineto{\pgfqpoint{1.489263in}{0.709193in}}%
\pgfpathlineto{\pgfqpoint{1.535143in}{0.709193in}}%
\pgfpathlineto{\pgfqpoint{1.535143in}{0.703617in}}%
\pgfpathlineto{\pgfqpoint{1.581024in}{0.703617in}}%
\pgfpathlineto{\pgfqpoint{1.581024in}{0.696368in}}%
\pgfpathlineto{\pgfqpoint{1.626904in}{0.696368in}}%
\pgfpathlineto{\pgfqpoint{1.626904in}{0.696675in}}%
\pgfpathlineto{\pgfqpoint{1.672785in}{0.696675in}}%
\pgfpathlineto{\pgfqpoint{1.672785in}{0.690429in}}%
\pgfpathlineto{\pgfqpoint{1.718665in}{0.690429in}}%
\pgfpathlineto{\pgfqpoint{1.718665in}{0.685745in}}%
\pgfpathlineto{\pgfqpoint{1.764546in}{0.685745in}}%
\pgfpathlineto{\pgfqpoint{1.764546in}{0.680866in}}%
\pgfpathlineto{\pgfqpoint{1.810427in}{0.680866in}}%
\pgfpathlineto{\pgfqpoint{1.810427in}{0.679304in}}%
\pgfpathlineto{\pgfqpoint{1.856307in}{0.679304in}}%
\pgfpathlineto{\pgfqpoint{1.856307in}{0.673505in}}%
\pgfpathlineto{\pgfqpoint{1.902188in}{0.673505in}}%
\pgfpathlineto{\pgfqpoint{1.902188in}{0.675261in}}%
\pgfpathlineto{\pgfqpoint{1.948068in}{0.675261in}}%
\pgfpathlineto{\pgfqpoint{1.948068in}{0.667399in}}%
\pgfpathlineto{\pgfqpoint{1.993949in}{0.667399in}}%
\pgfpathlineto{\pgfqpoint{1.993949in}{0.656441in}}%
\pgfpathlineto{\pgfqpoint{2.039829in}{0.656441in}}%
\pgfpathlineto{\pgfqpoint{2.039829in}{0.661349in}}%
\pgfpathlineto{\pgfqpoint{2.085710in}{0.661349in}}%
\pgfpathlineto{\pgfqpoint{2.085710in}{0.658254in}}%
\pgfpathlineto{\pgfqpoint{2.131590in}{0.658254in}}%
\pgfpathlineto{\pgfqpoint{2.131590in}{0.654322in}}%
\pgfpathlineto{\pgfqpoint{2.177471in}{0.654322in}}%
\pgfpathlineto{\pgfqpoint{2.177471in}{0.647631in}}%
\pgfpathlineto{\pgfqpoint{2.223352in}{0.647631in}}%
\pgfpathlineto{\pgfqpoint{2.223352in}{0.647073in}}%
\pgfpathlineto{\pgfqpoint{2.269232in}{0.647073in}}%
\pgfpathlineto{\pgfqpoint{2.269232in}{0.646460in}}%
\pgfpathlineto{\pgfqpoint{2.315113in}{0.646460in}}%
\pgfpathlineto{\pgfqpoint{2.315113in}{0.650726in}}%
\pgfpathlineto{\pgfqpoint{2.360993in}{0.650726in}}%
\pgfpathlineto{\pgfqpoint{2.360993in}{0.639935in}}%
\pgfpathlineto{\pgfqpoint{2.406874in}{0.639935in}}%
\pgfpathlineto{\pgfqpoint{2.406874in}{0.637259in}}%
\pgfpathlineto{\pgfqpoint{2.452754in}{0.637259in}}%
\pgfpathlineto{\pgfqpoint{2.452754in}{0.640382in}}%
\pgfpathlineto{\pgfqpoint{2.498635in}{0.640382in}}%
\pgfpathlineto{\pgfqpoint{2.498635in}{0.635725in}}%
\pgfpathlineto{\pgfqpoint{2.544515in}{0.635725in}}%
\pgfpathlineto{\pgfqpoint{2.544515in}{0.633467in}}%
\pgfpathlineto{\pgfqpoint{2.590396in}{0.633467in}}%
\pgfpathlineto{\pgfqpoint{2.590396in}{0.441418in}}%
\pgfusepath{stroke}%
\end{pgfscope}%
\begin{pgfscope}%
\pgfsetrectcap%
\pgfsetmiterjoin%
\pgfsetlinewidth{1.003750pt}%
\definecolor{currentstroke}{rgb}{0.000000,0.000000,0.000000}%
\pgfsetstrokecolor{currentstroke}%
\pgfsetdash{}{0pt}%
\pgfpathmoveto{\pgfqpoint{0.296148in}{1.614961in}}%
\pgfpathlineto{\pgfqpoint{2.592740in}{1.614961in}}%
\pgfusepath{stroke}%
\end{pgfscope}%
\begin{pgfscope}%
\pgfsetrectcap%
\pgfsetmiterjoin%
\pgfsetlinewidth{1.003750pt}%
\definecolor{currentstroke}{rgb}{0.000000,0.000000,0.000000}%
\pgfsetstrokecolor{currentstroke}%
\pgfsetdash{}{0pt}%
\pgfpathmoveto{\pgfqpoint{2.592740in}{0.441418in}}%
\pgfpathlineto{\pgfqpoint{2.592740in}{1.614961in}}%
\pgfusepath{stroke}%
\end{pgfscope}%
\begin{pgfscope}%
\pgfsetrectcap%
\pgfsetmiterjoin%
\pgfsetlinewidth{1.003750pt}%
\definecolor{currentstroke}{rgb}{0.000000,0.000000,0.000000}%
\pgfsetstrokecolor{currentstroke}%
\pgfsetdash{}{0pt}%
\pgfpathmoveto{\pgfqpoint{0.296148in}{0.441418in}}%
\pgfpathlineto{\pgfqpoint{2.592740in}{0.441418in}}%
\pgfusepath{stroke}%
\end{pgfscope}%
\begin{pgfscope}%
\pgfsetrectcap%
\pgfsetmiterjoin%
\pgfsetlinewidth{1.003750pt}%
\definecolor{currentstroke}{rgb}{0.000000,0.000000,0.000000}%
\pgfsetstrokecolor{currentstroke}%
\pgfsetdash{}{0pt}%
\pgfpathmoveto{\pgfqpoint{0.296148in}{0.441418in}}%
\pgfpathlineto{\pgfqpoint{0.296148in}{1.614961in}}%
\pgfusepath{stroke}%
\end{pgfscope}%
\begin{pgfscope}%
\pgfsetbuttcap%
\pgfsetroundjoin%
\definecolor{currentfill}{rgb}{0.000000,0.000000,0.000000}%
\pgfsetfillcolor{currentfill}%
\pgfsetlinewidth{0.501875pt}%
\definecolor{currentstroke}{rgb}{0.000000,0.000000,0.000000}%
\pgfsetstrokecolor{currentstroke}%
\pgfsetdash{}{0pt}%
\pgfsys@defobject{currentmarker}{\pgfqpoint{0.000000in}{0.000000in}}{\pgfqpoint{0.000000in}{0.069444in}}{%
\pgfpathmoveto{\pgfqpoint{0.000000in}{0.000000in}}%
\pgfpathlineto{\pgfqpoint{0.000000in}{0.069444in}}%
\pgfusepath{stroke,fill}%
}%
\begin{pgfscope}%
\pgfsys@transformshift{0.296148in}{0.441418in}%
\pgfsys@useobject{currentmarker}{}%
\end{pgfscope}%
\end{pgfscope}%
\begin{pgfscope}%
\pgfsetbuttcap%
\pgfsetroundjoin%
\definecolor{currentfill}{rgb}{0.000000,0.000000,0.000000}%
\pgfsetfillcolor{currentfill}%
\pgfsetlinewidth{0.501875pt}%
\definecolor{currentstroke}{rgb}{0.000000,0.000000,0.000000}%
\pgfsetstrokecolor{currentstroke}%
\pgfsetdash{}{0pt}%
\pgfsys@defobject{currentmarker}{\pgfqpoint{0.000000in}{-0.069444in}}{\pgfqpoint{0.000000in}{0.000000in}}{%
\pgfpathmoveto{\pgfqpoint{0.000000in}{0.000000in}}%
\pgfpathlineto{\pgfqpoint{0.000000in}{-0.069444in}}%
\pgfusepath{stroke,fill}%
}%
\begin{pgfscope}%
\pgfsys@transformshift{0.296148in}{1.614961in}%
\pgfsys@useobject{currentmarker}{}%
\end{pgfscope}%
\end{pgfscope}%
\begin{pgfscope}%
\pgftext[x=0.296148in,y=0.371974in,,top]{\rmfamily\fontsize{8.000000}{9.600000}\selectfont 0}%
\end{pgfscope}%
\begin{pgfscope}%
\pgfsetbuttcap%
\pgfsetroundjoin%
\definecolor{currentfill}{rgb}{0.000000,0.000000,0.000000}%
\pgfsetfillcolor{currentfill}%
\pgfsetlinewidth{0.501875pt}%
\definecolor{currentstroke}{rgb}{0.000000,0.000000,0.000000}%
\pgfsetstrokecolor{currentstroke}%
\pgfsetdash{}{0pt}%
\pgfsys@defobject{currentmarker}{\pgfqpoint{0.000000in}{0.000000in}}{\pgfqpoint{0.000000in}{0.069444in}}{%
\pgfpathmoveto{\pgfqpoint{0.000000in}{0.000000in}}%
\pgfpathlineto{\pgfqpoint{0.000000in}{0.069444in}}%
\pgfusepath{stroke,fill}%
}%
\begin{pgfscope}%
\pgfsys@transformshift{0.583222in}{0.441418in}%
\pgfsys@useobject{currentmarker}{}%
\end{pgfscope}%
\end{pgfscope}%
\begin{pgfscope}%
\pgfsetbuttcap%
\pgfsetroundjoin%
\definecolor{currentfill}{rgb}{0.000000,0.000000,0.000000}%
\pgfsetfillcolor{currentfill}%
\pgfsetlinewidth{0.501875pt}%
\definecolor{currentstroke}{rgb}{0.000000,0.000000,0.000000}%
\pgfsetstrokecolor{currentstroke}%
\pgfsetdash{}{0pt}%
\pgfsys@defobject{currentmarker}{\pgfqpoint{0.000000in}{-0.069444in}}{\pgfqpoint{0.000000in}{0.000000in}}{%
\pgfpathmoveto{\pgfqpoint{0.000000in}{0.000000in}}%
\pgfpathlineto{\pgfqpoint{0.000000in}{-0.069444in}}%
\pgfusepath{stroke,fill}%
}%
\begin{pgfscope}%
\pgfsys@transformshift{0.583222in}{1.614961in}%
\pgfsys@useobject{currentmarker}{}%
\end{pgfscope}%
\end{pgfscope}%
\begin{pgfscope}%
\pgftext[x=0.583222in,y=0.371974in,,top]{\rmfamily\fontsize{8.000000}{9.600000}\selectfont 1}%
\end{pgfscope}%
\begin{pgfscope}%
\pgfsetbuttcap%
\pgfsetroundjoin%
\definecolor{currentfill}{rgb}{0.000000,0.000000,0.000000}%
\pgfsetfillcolor{currentfill}%
\pgfsetlinewidth{0.501875pt}%
\definecolor{currentstroke}{rgb}{0.000000,0.000000,0.000000}%
\pgfsetstrokecolor{currentstroke}%
\pgfsetdash{}{0pt}%
\pgfsys@defobject{currentmarker}{\pgfqpoint{0.000000in}{0.000000in}}{\pgfqpoint{0.000000in}{0.069444in}}{%
\pgfpathmoveto{\pgfqpoint{0.000000in}{0.000000in}}%
\pgfpathlineto{\pgfqpoint{0.000000in}{0.069444in}}%
\pgfusepath{stroke,fill}%
}%
\begin{pgfscope}%
\pgfsys@transformshift{0.870296in}{0.441418in}%
\pgfsys@useobject{currentmarker}{}%
\end{pgfscope}%
\end{pgfscope}%
\begin{pgfscope}%
\pgfsetbuttcap%
\pgfsetroundjoin%
\definecolor{currentfill}{rgb}{0.000000,0.000000,0.000000}%
\pgfsetfillcolor{currentfill}%
\pgfsetlinewidth{0.501875pt}%
\definecolor{currentstroke}{rgb}{0.000000,0.000000,0.000000}%
\pgfsetstrokecolor{currentstroke}%
\pgfsetdash{}{0pt}%
\pgfsys@defobject{currentmarker}{\pgfqpoint{0.000000in}{-0.069444in}}{\pgfqpoint{0.000000in}{0.000000in}}{%
\pgfpathmoveto{\pgfqpoint{0.000000in}{0.000000in}}%
\pgfpathlineto{\pgfqpoint{0.000000in}{-0.069444in}}%
\pgfusepath{stroke,fill}%
}%
\begin{pgfscope}%
\pgfsys@transformshift{0.870296in}{1.614961in}%
\pgfsys@useobject{currentmarker}{}%
\end{pgfscope}%
\end{pgfscope}%
\begin{pgfscope}%
\pgftext[x=0.870296in,y=0.371974in,,top]{\rmfamily\fontsize{8.000000}{9.600000}\selectfont 2}%
\end{pgfscope}%
\begin{pgfscope}%
\pgfsetbuttcap%
\pgfsetroundjoin%
\definecolor{currentfill}{rgb}{0.000000,0.000000,0.000000}%
\pgfsetfillcolor{currentfill}%
\pgfsetlinewidth{0.501875pt}%
\definecolor{currentstroke}{rgb}{0.000000,0.000000,0.000000}%
\pgfsetstrokecolor{currentstroke}%
\pgfsetdash{}{0pt}%
\pgfsys@defobject{currentmarker}{\pgfqpoint{0.000000in}{0.000000in}}{\pgfqpoint{0.000000in}{0.069444in}}{%
\pgfpathmoveto{\pgfqpoint{0.000000in}{0.000000in}}%
\pgfpathlineto{\pgfqpoint{0.000000in}{0.069444in}}%
\pgfusepath{stroke,fill}%
}%
\begin{pgfscope}%
\pgfsys@transformshift{1.157370in}{0.441418in}%
\pgfsys@useobject{currentmarker}{}%
\end{pgfscope}%
\end{pgfscope}%
\begin{pgfscope}%
\pgfsetbuttcap%
\pgfsetroundjoin%
\definecolor{currentfill}{rgb}{0.000000,0.000000,0.000000}%
\pgfsetfillcolor{currentfill}%
\pgfsetlinewidth{0.501875pt}%
\definecolor{currentstroke}{rgb}{0.000000,0.000000,0.000000}%
\pgfsetstrokecolor{currentstroke}%
\pgfsetdash{}{0pt}%
\pgfsys@defobject{currentmarker}{\pgfqpoint{0.000000in}{-0.069444in}}{\pgfqpoint{0.000000in}{0.000000in}}{%
\pgfpathmoveto{\pgfqpoint{0.000000in}{0.000000in}}%
\pgfpathlineto{\pgfqpoint{0.000000in}{-0.069444in}}%
\pgfusepath{stroke,fill}%
}%
\begin{pgfscope}%
\pgfsys@transformshift{1.157370in}{1.614961in}%
\pgfsys@useobject{currentmarker}{}%
\end{pgfscope}%
\end{pgfscope}%
\begin{pgfscope}%
\pgftext[x=1.157370in,y=0.371974in,,top]{\rmfamily\fontsize{8.000000}{9.600000}\selectfont 3}%
\end{pgfscope}%
\begin{pgfscope}%
\pgfsetbuttcap%
\pgfsetroundjoin%
\definecolor{currentfill}{rgb}{0.000000,0.000000,0.000000}%
\pgfsetfillcolor{currentfill}%
\pgfsetlinewidth{0.501875pt}%
\definecolor{currentstroke}{rgb}{0.000000,0.000000,0.000000}%
\pgfsetstrokecolor{currentstroke}%
\pgfsetdash{}{0pt}%
\pgfsys@defobject{currentmarker}{\pgfqpoint{0.000000in}{0.000000in}}{\pgfqpoint{0.000000in}{0.069444in}}{%
\pgfpathmoveto{\pgfqpoint{0.000000in}{0.000000in}}%
\pgfpathlineto{\pgfqpoint{0.000000in}{0.069444in}}%
\pgfusepath{stroke,fill}%
}%
\begin{pgfscope}%
\pgfsys@transformshift{1.444444in}{0.441418in}%
\pgfsys@useobject{currentmarker}{}%
\end{pgfscope}%
\end{pgfscope}%
\begin{pgfscope}%
\pgfsetbuttcap%
\pgfsetroundjoin%
\definecolor{currentfill}{rgb}{0.000000,0.000000,0.000000}%
\pgfsetfillcolor{currentfill}%
\pgfsetlinewidth{0.501875pt}%
\definecolor{currentstroke}{rgb}{0.000000,0.000000,0.000000}%
\pgfsetstrokecolor{currentstroke}%
\pgfsetdash{}{0pt}%
\pgfsys@defobject{currentmarker}{\pgfqpoint{0.000000in}{-0.069444in}}{\pgfqpoint{0.000000in}{0.000000in}}{%
\pgfpathmoveto{\pgfqpoint{0.000000in}{0.000000in}}%
\pgfpathlineto{\pgfqpoint{0.000000in}{-0.069444in}}%
\pgfusepath{stroke,fill}%
}%
\begin{pgfscope}%
\pgfsys@transformshift{1.444444in}{1.614961in}%
\pgfsys@useobject{currentmarker}{}%
\end{pgfscope}%
\end{pgfscope}%
\begin{pgfscope}%
\pgftext[x=1.444444in,y=0.371974in,,top]{\rmfamily\fontsize{8.000000}{9.600000}\selectfont 4}%
\end{pgfscope}%
\begin{pgfscope}%
\pgfsetbuttcap%
\pgfsetroundjoin%
\definecolor{currentfill}{rgb}{0.000000,0.000000,0.000000}%
\pgfsetfillcolor{currentfill}%
\pgfsetlinewidth{0.501875pt}%
\definecolor{currentstroke}{rgb}{0.000000,0.000000,0.000000}%
\pgfsetstrokecolor{currentstroke}%
\pgfsetdash{}{0pt}%
\pgfsys@defobject{currentmarker}{\pgfqpoint{0.000000in}{0.000000in}}{\pgfqpoint{0.000000in}{0.069444in}}{%
\pgfpathmoveto{\pgfqpoint{0.000000in}{0.000000in}}%
\pgfpathlineto{\pgfqpoint{0.000000in}{0.069444in}}%
\pgfusepath{stroke,fill}%
}%
\begin{pgfscope}%
\pgfsys@transformshift{1.731518in}{0.441418in}%
\pgfsys@useobject{currentmarker}{}%
\end{pgfscope}%
\end{pgfscope}%
\begin{pgfscope}%
\pgfsetbuttcap%
\pgfsetroundjoin%
\definecolor{currentfill}{rgb}{0.000000,0.000000,0.000000}%
\pgfsetfillcolor{currentfill}%
\pgfsetlinewidth{0.501875pt}%
\definecolor{currentstroke}{rgb}{0.000000,0.000000,0.000000}%
\pgfsetstrokecolor{currentstroke}%
\pgfsetdash{}{0pt}%
\pgfsys@defobject{currentmarker}{\pgfqpoint{0.000000in}{-0.069444in}}{\pgfqpoint{0.000000in}{0.000000in}}{%
\pgfpathmoveto{\pgfqpoint{0.000000in}{0.000000in}}%
\pgfpathlineto{\pgfqpoint{0.000000in}{-0.069444in}}%
\pgfusepath{stroke,fill}%
}%
\begin{pgfscope}%
\pgfsys@transformshift{1.731518in}{1.614961in}%
\pgfsys@useobject{currentmarker}{}%
\end{pgfscope}%
\end{pgfscope}%
\begin{pgfscope}%
\pgftext[x=1.731518in,y=0.371974in,,top]{\rmfamily\fontsize{8.000000}{9.600000}\selectfont 5}%
\end{pgfscope}%
\begin{pgfscope}%
\pgfsetbuttcap%
\pgfsetroundjoin%
\definecolor{currentfill}{rgb}{0.000000,0.000000,0.000000}%
\pgfsetfillcolor{currentfill}%
\pgfsetlinewidth{0.501875pt}%
\definecolor{currentstroke}{rgb}{0.000000,0.000000,0.000000}%
\pgfsetstrokecolor{currentstroke}%
\pgfsetdash{}{0pt}%
\pgfsys@defobject{currentmarker}{\pgfqpoint{0.000000in}{0.000000in}}{\pgfqpoint{0.000000in}{0.069444in}}{%
\pgfpathmoveto{\pgfqpoint{0.000000in}{0.000000in}}%
\pgfpathlineto{\pgfqpoint{0.000000in}{0.069444in}}%
\pgfusepath{stroke,fill}%
}%
\begin{pgfscope}%
\pgfsys@transformshift{2.018592in}{0.441418in}%
\pgfsys@useobject{currentmarker}{}%
\end{pgfscope}%
\end{pgfscope}%
\begin{pgfscope}%
\pgfsetbuttcap%
\pgfsetroundjoin%
\definecolor{currentfill}{rgb}{0.000000,0.000000,0.000000}%
\pgfsetfillcolor{currentfill}%
\pgfsetlinewidth{0.501875pt}%
\definecolor{currentstroke}{rgb}{0.000000,0.000000,0.000000}%
\pgfsetstrokecolor{currentstroke}%
\pgfsetdash{}{0pt}%
\pgfsys@defobject{currentmarker}{\pgfqpoint{0.000000in}{-0.069444in}}{\pgfqpoint{0.000000in}{0.000000in}}{%
\pgfpathmoveto{\pgfqpoint{0.000000in}{0.000000in}}%
\pgfpathlineto{\pgfqpoint{0.000000in}{-0.069444in}}%
\pgfusepath{stroke,fill}%
}%
\begin{pgfscope}%
\pgfsys@transformshift{2.018592in}{1.614961in}%
\pgfsys@useobject{currentmarker}{}%
\end{pgfscope}%
\end{pgfscope}%
\begin{pgfscope}%
\pgftext[x=2.018592in,y=0.371974in,,top]{\rmfamily\fontsize{8.000000}{9.600000}\selectfont 6}%
\end{pgfscope}%
\begin{pgfscope}%
\pgfsetbuttcap%
\pgfsetroundjoin%
\definecolor{currentfill}{rgb}{0.000000,0.000000,0.000000}%
\pgfsetfillcolor{currentfill}%
\pgfsetlinewidth{0.501875pt}%
\definecolor{currentstroke}{rgb}{0.000000,0.000000,0.000000}%
\pgfsetstrokecolor{currentstroke}%
\pgfsetdash{}{0pt}%
\pgfsys@defobject{currentmarker}{\pgfqpoint{0.000000in}{0.000000in}}{\pgfqpoint{0.000000in}{0.069444in}}{%
\pgfpathmoveto{\pgfqpoint{0.000000in}{0.000000in}}%
\pgfpathlineto{\pgfqpoint{0.000000in}{0.069444in}}%
\pgfusepath{stroke,fill}%
}%
\begin{pgfscope}%
\pgfsys@transformshift{2.305666in}{0.441418in}%
\pgfsys@useobject{currentmarker}{}%
\end{pgfscope}%
\end{pgfscope}%
\begin{pgfscope}%
\pgfsetbuttcap%
\pgfsetroundjoin%
\definecolor{currentfill}{rgb}{0.000000,0.000000,0.000000}%
\pgfsetfillcolor{currentfill}%
\pgfsetlinewidth{0.501875pt}%
\definecolor{currentstroke}{rgb}{0.000000,0.000000,0.000000}%
\pgfsetstrokecolor{currentstroke}%
\pgfsetdash{}{0pt}%
\pgfsys@defobject{currentmarker}{\pgfqpoint{0.000000in}{-0.069444in}}{\pgfqpoint{0.000000in}{0.000000in}}{%
\pgfpathmoveto{\pgfqpoint{0.000000in}{0.000000in}}%
\pgfpathlineto{\pgfqpoint{0.000000in}{-0.069444in}}%
\pgfusepath{stroke,fill}%
}%
\begin{pgfscope}%
\pgfsys@transformshift{2.305666in}{1.614961in}%
\pgfsys@useobject{currentmarker}{}%
\end{pgfscope}%
\end{pgfscope}%
\begin{pgfscope}%
\pgftext[x=2.305666in,y=0.371974in,,top]{\rmfamily\fontsize{8.000000}{9.600000}\selectfont 7}%
\end{pgfscope}%
\begin{pgfscope}%
\pgfsetbuttcap%
\pgfsetroundjoin%
\definecolor{currentfill}{rgb}{0.000000,0.000000,0.000000}%
\pgfsetfillcolor{currentfill}%
\pgfsetlinewidth{0.501875pt}%
\definecolor{currentstroke}{rgb}{0.000000,0.000000,0.000000}%
\pgfsetstrokecolor{currentstroke}%
\pgfsetdash{}{0pt}%
\pgfsys@defobject{currentmarker}{\pgfqpoint{0.000000in}{0.000000in}}{\pgfqpoint{0.000000in}{0.069444in}}{%
\pgfpathmoveto{\pgfqpoint{0.000000in}{0.000000in}}%
\pgfpathlineto{\pgfqpoint{0.000000in}{0.069444in}}%
\pgfusepath{stroke,fill}%
}%
\begin{pgfscope}%
\pgfsys@transformshift{2.592740in}{0.441418in}%
\pgfsys@useobject{currentmarker}{}%
\end{pgfscope}%
\end{pgfscope}%
\begin{pgfscope}%
\pgfsetbuttcap%
\pgfsetroundjoin%
\definecolor{currentfill}{rgb}{0.000000,0.000000,0.000000}%
\pgfsetfillcolor{currentfill}%
\pgfsetlinewidth{0.501875pt}%
\definecolor{currentstroke}{rgb}{0.000000,0.000000,0.000000}%
\pgfsetstrokecolor{currentstroke}%
\pgfsetdash{}{0pt}%
\pgfsys@defobject{currentmarker}{\pgfqpoint{0.000000in}{-0.069444in}}{\pgfqpoint{0.000000in}{0.000000in}}{%
\pgfpathmoveto{\pgfqpoint{0.000000in}{0.000000in}}%
\pgfpathlineto{\pgfqpoint{0.000000in}{-0.069444in}}%
\pgfusepath{stroke,fill}%
}%
\begin{pgfscope}%
\pgfsys@transformshift{2.592740in}{1.614961in}%
\pgfsys@useobject{currentmarker}{}%
\end{pgfscope}%
\end{pgfscope}%
\begin{pgfscope}%
\pgftext[x=2.592740in,y=0.371974in,,top]{\rmfamily\fontsize{8.000000}{9.600000}\selectfont 8}%
\end{pgfscope}%
\begin{pgfscope}%
\pgftext[x=1.444444in,y=0.194999in,,top]{\rmfamily\fontsize{9.000000}{10.800000}\selectfont \(\displaystyle B^0\ \mathrm{vertex}\ \chi^2 / \mathrm{ndf}\)}%
\end{pgfscope}%
\begin{pgfscope}%
\pgfsetbuttcap%
\pgfsetroundjoin%
\definecolor{currentfill}{rgb}{0.000000,0.000000,0.000000}%
\pgfsetfillcolor{currentfill}%
\pgfsetlinewidth{0.501875pt}%
\definecolor{currentstroke}{rgb}{0.000000,0.000000,0.000000}%
\pgfsetstrokecolor{currentstroke}%
\pgfsetdash{}{0pt}%
\pgfsys@defobject{currentmarker}{\pgfqpoint{0.000000in}{0.000000in}}{\pgfqpoint{0.069444in}{0.000000in}}{%
\pgfpathmoveto{\pgfqpoint{0.000000in}{0.000000in}}%
\pgfpathlineto{\pgfqpoint{0.069444in}{0.000000in}}%
\pgfusepath{stroke,fill}%
}%
\begin{pgfscope}%
\pgfsys@transformshift{0.296148in}{0.441418in}%
\pgfsys@useobject{currentmarker}{}%
\end{pgfscope}%
\end{pgfscope}%
\begin{pgfscope}%
\pgfsetbuttcap%
\pgfsetroundjoin%
\definecolor{currentfill}{rgb}{0.000000,0.000000,0.000000}%
\pgfsetfillcolor{currentfill}%
\pgfsetlinewidth{0.501875pt}%
\definecolor{currentstroke}{rgb}{0.000000,0.000000,0.000000}%
\pgfsetstrokecolor{currentstroke}%
\pgfsetdash{}{0pt}%
\pgfsys@defobject{currentmarker}{\pgfqpoint{-0.069444in}{0.000000in}}{\pgfqpoint{0.000000in}{0.000000in}}{%
\pgfpathmoveto{\pgfqpoint{0.000000in}{0.000000in}}%
\pgfpathlineto{\pgfqpoint{-0.069444in}{0.000000in}}%
\pgfusepath{stroke,fill}%
}%
\begin{pgfscope}%
\pgfsys@transformshift{2.592740in}{0.441418in}%
\pgfsys@useobject{currentmarker}{}%
\end{pgfscope}%
\end{pgfscope}%
\begin{pgfscope}%
\pgftext[x=0.226704in,y=0.441418in,right,]{\rmfamily\fontsize{8.000000}{9.600000}\selectfont 0.0}%
\end{pgfscope}%
\begin{pgfscope}%
\pgfsetbuttcap%
\pgfsetroundjoin%
\definecolor{currentfill}{rgb}{0.000000,0.000000,0.000000}%
\pgfsetfillcolor{currentfill}%
\pgfsetlinewidth{0.501875pt}%
\definecolor{currentstroke}{rgb}{0.000000,0.000000,0.000000}%
\pgfsetstrokecolor{currentstroke}%
\pgfsetdash{}{0pt}%
\pgfsys@defobject{currentmarker}{\pgfqpoint{0.000000in}{0.000000in}}{\pgfqpoint{0.069444in}{0.000000in}}{%
\pgfpathmoveto{\pgfqpoint{0.000000in}{0.000000in}}%
\pgfpathlineto{\pgfqpoint{0.069444in}{0.000000in}}%
\pgfusepath{stroke,fill}%
}%
\begin{pgfscope}%
\pgfsys@transformshift{0.296148in}{0.637009in}%
\pgfsys@useobject{currentmarker}{}%
\end{pgfscope}%
\end{pgfscope}%
\begin{pgfscope}%
\pgfsetbuttcap%
\pgfsetroundjoin%
\definecolor{currentfill}{rgb}{0.000000,0.000000,0.000000}%
\pgfsetfillcolor{currentfill}%
\pgfsetlinewidth{0.501875pt}%
\definecolor{currentstroke}{rgb}{0.000000,0.000000,0.000000}%
\pgfsetstrokecolor{currentstroke}%
\pgfsetdash{}{0pt}%
\pgfsys@defobject{currentmarker}{\pgfqpoint{-0.069444in}{0.000000in}}{\pgfqpoint{0.000000in}{0.000000in}}{%
\pgfpathmoveto{\pgfqpoint{0.000000in}{0.000000in}}%
\pgfpathlineto{\pgfqpoint{-0.069444in}{0.000000in}}%
\pgfusepath{stroke,fill}%
}%
\begin{pgfscope}%
\pgfsys@transformshift{2.592740in}{0.637009in}%
\pgfsys@useobject{currentmarker}{}%
\end{pgfscope}%
\end{pgfscope}%
\begin{pgfscope}%
\pgftext[x=0.226704in,y=0.637009in,right,]{\rmfamily\fontsize{8.000000}{9.600000}\selectfont 0.1}%
\end{pgfscope}%
\begin{pgfscope}%
\pgfsetbuttcap%
\pgfsetroundjoin%
\definecolor{currentfill}{rgb}{0.000000,0.000000,0.000000}%
\pgfsetfillcolor{currentfill}%
\pgfsetlinewidth{0.501875pt}%
\definecolor{currentstroke}{rgb}{0.000000,0.000000,0.000000}%
\pgfsetstrokecolor{currentstroke}%
\pgfsetdash{}{0pt}%
\pgfsys@defobject{currentmarker}{\pgfqpoint{0.000000in}{0.000000in}}{\pgfqpoint{0.069444in}{0.000000in}}{%
\pgfpathmoveto{\pgfqpoint{0.000000in}{0.000000in}}%
\pgfpathlineto{\pgfqpoint{0.069444in}{0.000000in}}%
\pgfusepath{stroke,fill}%
}%
\begin{pgfscope}%
\pgfsys@transformshift{0.296148in}{0.832599in}%
\pgfsys@useobject{currentmarker}{}%
\end{pgfscope}%
\end{pgfscope}%
\begin{pgfscope}%
\pgfsetbuttcap%
\pgfsetroundjoin%
\definecolor{currentfill}{rgb}{0.000000,0.000000,0.000000}%
\pgfsetfillcolor{currentfill}%
\pgfsetlinewidth{0.501875pt}%
\definecolor{currentstroke}{rgb}{0.000000,0.000000,0.000000}%
\pgfsetstrokecolor{currentstroke}%
\pgfsetdash{}{0pt}%
\pgfsys@defobject{currentmarker}{\pgfqpoint{-0.069444in}{0.000000in}}{\pgfqpoint{0.000000in}{0.000000in}}{%
\pgfpathmoveto{\pgfqpoint{0.000000in}{0.000000in}}%
\pgfpathlineto{\pgfqpoint{-0.069444in}{0.000000in}}%
\pgfusepath{stroke,fill}%
}%
\begin{pgfscope}%
\pgfsys@transformshift{2.592740in}{0.832599in}%
\pgfsys@useobject{currentmarker}{}%
\end{pgfscope}%
\end{pgfscope}%
\begin{pgfscope}%
\pgftext[x=0.226704in,y=0.832599in,right,]{\rmfamily\fontsize{8.000000}{9.600000}\selectfont 0.2}%
\end{pgfscope}%
\begin{pgfscope}%
\pgfsetbuttcap%
\pgfsetroundjoin%
\definecolor{currentfill}{rgb}{0.000000,0.000000,0.000000}%
\pgfsetfillcolor{currentfill}%
\pgfsetlinewidth{0.501875pt}%
\definecolor{currentstroke}{rgb}{0.000000,0.000000,0.000000}%
\pgfsetstrokecolor{currentstroke}%
\pgfsetdash{}{0pt}%
\pgfsys@defobject{currentmarker}{\pgfqpoint{0.000000in}{0.000000in}}{\pgfqpoint{0.069444in}{0.000000in}}{%
\pgfpathmoveto{\pgfqpoint{0.000000in}{0.000000in}}%
\pgfpathlineto{\pgfqpoint{0.069444in}{0.000000in}}%
\pgfusepath{stroke,fill}%
}%
\begin{pgfscope}%
\pgfsys@transformshift{0.296148in}{1.028190in}%
\pgfsys@useobject{currentmarker}{}%
\end{pgfscope}%
\end{pgfscope}%
\begin{pgfscope}%
\pgfsetbuttcap%
\pgfsetroundjoin%
\definecolor{currentfill}{rgb}{0.000000,0.000000,0.000000}%
\pgfsetfillcolor{currentfill}%
\pgfsetlinewidth{0.501875pt}%
\definecolor{currentstroke}{rgb}{0.000000,0.000000,0.000000}%
\pgfsetstrokecolor{currentstroke}%
\pgfsetdash{}{0pt}%
\pgfsys@defobject{currentmarker}{\pgfqpoint{-0.069444in}{0.000000in}}{\pgfqpoint{0.000000in}{0.000000in}}{%
\pgfpathmoveto{\pgfqpoint{0.000000in}{0.000000in}}%
\pgfpathlineto{\pgfqpoint{-0.069444in}{0.000000in}}%
\pgfusepath{stroke,fill}%
}%
\begin{pgfscope}%
\pgfsys@transformshift{2.592740in}{1.028190in}%
\pgfsys@useobject{currentmarker}{}%
\end{pgfscope}%
\end{pgfscope}%
\begin{pgfscope}%
\pgftext[x=0.226704in,y=1.028190in,right,]{\rmfamily\fontsize{8.000000}{9.600000}\selectfont 0.3}%
\end{pgfscope}%
\begin{pgfscope}%
\pgfsetbuttcap%
\pgfsetroundjoin%
\definecolor{currentfill}{rgb}{0.000000,0.000000,0.000000}%
\pgfsetfillcolor{currentfill}%
\pgfsetlinewidth{0.501875pt}%
\definecolor{currentstroke}{rgb}{0.000000,0.000000,0.000000}%
\pgfsetstrokecolor{currentstroke}%
\pgfsetdash{}{0pt}%
\pgfsys@defobject{currentmarker}{\pgfqpoint{0.000000in}{0.000000in}}{\pgfqpoint{0.069444in}{0.000000in}}{%
\pgfpathmoveto{\pgfqpoint{0.000000in}{0.000000in}}%
\pgfpathlineto{\pgfqpoint{0.069444in}{0.000000in}}%
\pgfusepath{stroke,fill}%
}%
\begin{pgfscope}%
\pgfsys@transformshift{0.296148in}{1.223780in}%
\pgfsys@useobject{currentmarker}{}%
\end{pgfscope}%
\end{pgfscope}%
\begin{pgfscope}%
\pgfsetbuttcap%
\pgfsetroundjoin%
\definecolor{currentfill}{rgb}{0.000000,0.000000,0.000000}%
\pgfsetfillcolor{currentfill}%
\pgfsetlinewidth{0.501875pt}%
\definecolor{currentstroke}{rgb}{0.000000,0.000000,0.000000}%
\pgfsetstrokecolor{currentstroke}%
\pgfsetdash{}{0pt}%
\pgfsys@defobject{currentmarker}{\pgfqpoint{-0.069444in}{0.000000in}}{\pgfqpoint{0.000000in}{0.000000in}}{%
\pgfpathmoveto{\pgfqpoint{0.000000in}{0.000000in}}%
\pgfpathlineto{\pgfqpoint{-0.069444in}{0.000000in}}%
\pgfusepath{stroke,fill}%
}%
\begin{pgfscope}%
\pgfsys@transformshift{2.592740in}{1.223780in}%
\pgfsys@useobject{currentmarker}{}%
\end{pgfscope}%
\end{pgfscope}%
\begin{pgfscope}%
\pgftext[x=0.226704in,y=1.223780in,right,]{\rmfamily\fontsize{8.000000}{9.600000}\selectfont 0.4}%
\end{pgfscope}%
\begin{pgfscope}%
\pgfsetbuttcap%
\pgfsetroundjoin%
\definecolor{currentfill}{rgb}{0.000000,0.000000,0.000000}%
\pgfsetfillcolor{currentfill}%
\pgfsetlinewidth{0.501875pt}%
\definecolor{currentstroke}{rgb}{0.000000,0.000000,0.000000}%
\pgfsetstrokecolor{currentstroke}%
\pgfsetdash{}{0pt}%
\pgfsys@defobject{currentmarker}{\pgfqpoint{0.000000in}{0.000000in}}{\pgfqpoint{0.069444in}{0.000000in}}{%
\pgfpathmoveto{\pgfqpoint{0.000000in}{0.000000in}}%
\pgfpathlineto{\pgfqpoint{0.069444in}{0.000000in}}%
\pgfusepath{stroke,fill}%
}%
\begin{pgfscope}%
\pgfsys@transformshift{0.296148in}{1.419371in}%
\pgfsys@useobject{currentmarker}{}%
\end{pgfscope}%
\end{pgfscope}%
\begin{pgfscope}%
\pgfsetbuttcap%
\pgfsetroundjoin%
\definecolor{currentfill}{rgb}{0.000000,0.000000,0.000000}%
\pgfsetfillcolor{currentfill}%
\pgfsetlinewidth{0.501875pt}%
\definecolor{currentstroke}{rgb}{0.000000,0.000000,0.000000}%
\pgfsetstrokecolor{currentstroke}%
\pgfsetdash{}{0pt}%
\pgfsys@defobject{currentmarker}{\pgfqpoint{-0.069444in}{0.000000in}}{\pgfqpoint{0.000000in}{0.000000in}}{%
\pgfpathmoveto{\pgfqpoint{0.000000in}{0.000000in}}%
\pgfpathlineto{\pgfqpoint{-0.069444in}{0.000000in}}%
\pgfusepath{stroke,fill}%
}%
\begin{pgfscope}%
\pgfsys@transformshift{2.592740in}{1.419371in}%
\pgfsys@useobject{currentmarker}{}%
\end{pgfscope}%
\end{pgfscope}%
\begin{pgfscope}%
\pgftext[x=0.226704in,y=1.419371in,right,]{\rmfamily\fontsize{8.000000}{9.600000}\selectfont 0.5}%
\end{pgfscope}%
\begin{pgfscope}%
\pgfsetbuttcap%
\pgfsetroundjoin%
\definecolor{currentfill}{rgb}{0.000000,0.000000,0.000000}%
\pgfsetfillcolor{currentfill}%
\pgfsetlinewidth{0.501875pt}%
\definecolor{currentstroke}{rgb}{0.000000,0.000000,0.000000}%
\pgfsetstrokecolor{currentstroke}%
\pgfsetdash{}{0pt}%
\pgfsys@defobject{currentmarker}{\pgfqpoint{0.000000in}{0.000000in}}{\pgfqpoint{0.069444in}{0.000000in}}{%
\pgfpathmoveto{\pgfqpoint{0.000000in}{0.000000in}}%
\pgfpathlineto{\pgfqpoint{0.069444in}{0.000000in}}%
\pgfusepath{stroke,fill}%
}%
\begin{pgfscope}%
\pgfsys@transformshift{0.296148in}{1.614961in}%
\pgfsys@useobject{currentmarker}{}%
\end{pgfscope}%
\end{pgfscope}%
\begin{pgfscope}%
\pgfsetbuttcap%
\pgfsetroundjoin%
\definecolor{currentfill}{rgb}{0.000000,0.000000,0.000000}%
\pgfsetfillcolor{currentfill}%
\pgfsetlinewidth{0.501875pt}%
\definecolor{currentstroke}{rgb}{0.000000,0.000000,0.000000}%
\pgfsetstrokecolor{currentstroke}%
\pgfsetdash{}{0pt}%
\pgfsys@defobject{currentmarker}{\pgfqpoint{-0.069444in}{0.000000in}}{\pgfqpoint{0.000000in}{0.000000in}}{%
\pgfpathmoveto{\pgfqpoint{0.000000in}{0.000000in}}%
\pgfpathlineto{\pgfqpoint{-0.069444in}{0.000000in}}%
\pgfusepath{stroke,fill}%
}%
\begin{pgfscope}%
\pgfsys@transformshift{2.592740in}{1.614961in}%
\pgfsys@useobject{currentmarker}{}%
\end{pgfscope}%
\end{pgfscope}%
\begin{pgfscope}%
\pgftext[x=0.226704in,y=1.614961in,right,]{\rmfamily\fontsize{8.000000}{9.600000}\selectfont 0.6}%
\end{pgfscope}%
\end{pgfpicture}%
\makeatother%
\endgroup%

	\end{subfigure}
  
	\begin{subfigure}[t]{0.49\textwidth}
		\centering
    %\includegraphics[width=\textwidth]{store/variables/SIG_BKG_B_ISOLATION_BDT_Soft.pdf}
    %% Creator: Matplotlib, PGF backend
%%
%% To include the figure in your LaTeX document, write
%%   \input{<filename>.pgf}
%%
%% Make sure the required packages are loaded in your preamble
%%   \usepackage{pgf}
%%
%% Figures using additional raster images can only be included by \input if
%% they are in the same directory as the main LaTeX file. For loading figures
%% from other directories you can use the `import` package
%%   \usepackage{import}
%% and then include the figures with
%%   \import{<path to file>}{<filename>.pgf}
%%
%% Matplotlib used the following preamble
%%   \usepackage{fontspec}
%%   \setmainfont{DejaVu Serif}
%%   \setsansfont{DejaVu Sans}
%%   \setmonofont{DejaVu Sans Mono}
%%
\begingroup%
\makeatletter%
\begin{pgfpicture}%
\pgfpathrectangle{\pgfpointorigin}{\pgfqpoint{2.680146in}{1.787280in}}%
\pgfusepath{use as bounding box, clip}%
\begin{pgfscope}%
\pgfsetbuttcap%
\pgfsetmiterjoin%
\definecolor{currentfill}{rgb}{1.000000,1.000000,1.000000}%
\pgfsetfillcolor{currentfill}%
\pgfsetlinewidth{0.000000pt}%
\definecolor{currentstroke}{rgb}{1.000000,1.000000,1.000000}%
\pgfsetstrokecolor{currentstroke}%
\pgfsetdash{}{0pt}%
\pgfpathmoveto{\pgfqpoint{0.000000in}{0.000000in}}%
\pgfpathlineto{\pgfqpoint{2.680146in}{0.000000in}}%
\pgfpathlineto{\pgfqpoint{2.680146in}{1.787280in}}%
\pgfpathlineto{\pgfqpoint{0.000000in}{1.787280in}}%
\pgfpathclose%
\pgfusepath{fill}%
\end{pgfscope}%
\begin{pgfscope}%
\pgfsetbuttcap%
\pgfsetmiterjoin%
\definecolor{currentfill}{rgb}{1.000000,1.000000,1.000000}%
\pgfsetfillcolor{currentfill}%
\pgfsetlinewidth{0.000000pt}%
\definecolor{currentstroke}{rgb}{0.000000,0.000000,0.000000}%
\pgfsetstrokecolor{currentstroke}%
\pgfsetstrokeopacity{0.000000}%
\pgfsetdash{}{0pt}%
\pgfpathmoveto{\pgfqpoint{0.296148in}{0.417391in}}%
\pgfpathlineto{\pgfqpoint{2.541794in}{0.417391in}}%
\pgfpathlineto{\pgfqpoint{2.541794in}{1.683515in}}%
\pgfpathlineto{\pgfqpoint{0.296148in}{1.683515in}}%
\pgfpathclose%
\pgfusepath{fill}%
\end{pgfscope}%
\begin{pgfscope}%
\pgfpathrectangle{\pgfqpoint{0.296148in}{0.417391in}}{\pgfqpoint{2.245646in}{1.266124in}} %
\pgfusepath{clip}%
\pgfsetbuttcap%
\pgfsetmiterjoin%
\definecolor{currentfill}{rgb}{0.215686,0.470588,0.749020}%
\pgfsetfillcolor{currentfill}%
\pgfsetlinewidth{0.000000pt}%
\definecolor{currentstroke}{rgb}{0.000000,0.000000,0.000000}%
\pgfsetstrokecolor{currentstroke}%
\pgfsetdash{}{0pt}%
\pgfpathmoveto{\pgfqpoint{0.296148in}{0.417391in}}%
\pgfpathlineto{\pgfqpoint{0.296148in}{1.582487in}}%
\pgfpathlineto{\pgfqpoint{0.335506in}{1.582487in}}%
\pgfpathlineto{\pgfqpoint{0.335506in}{0.417391in}}%
\pgfpathlineto{\pgfqpoint{0.374864in}{0.417391in}}%
\pgfpathlineto{\pgfqpoint{0.374864in}{0.417391in}}%
\pgfpathlineto{\pgfqpoint{0.414221in}{0.417391in}}%
\pgfpathlineto{\pgfqpoint{0.414221in}{0.417391in}}%
\pgfpathlineto{\pgfqpoint{0.453579in}{0.417391in}}%
\pgfpathlineto{\pgfqpoint{0.453579in}{0.417391in}}%
\pgfpathlineto{\pgfqpoint{0.492937in}{0.417391in}}%
\pgfpathlineto{\pgfqpoint{0.492937in}{0.417391in}}%
\pgfpathlineto{\pgfqpoint{0.532295in}{0.417391in}}%
\pgfpathlineto{\pgfqpoint{0.532295in}{0.417391in}}%
\pgfpathlineto{\pgfqpoint{0.571652in}{0.417391in}}%
\pgfpathlineto{\pgfqpoint{0.571652in}{0.417391in}}%
\pgfpathlineto{\pgfqpoint{0.611010in}{0.417391in}}%
\pgfpathlineto{\pgfqpoint{0.611010in}{0.417391in}}%
\pgfpathlineto{\pgfqpoint{0.650368in}{0.417391in}}%
\pgfpathlineto{\pgfqpoint{0.650368in}{0.417391in}}%
\pgfpathlineto{\pgfqpoint{0.689726in}{0.417391in}}%
\pgfpathlineto{\pgfqpoint{0.689726in}{0.417391in}}%
\pgfpathlineto{\pgfqpoint{0.729083in}{0.417391in}}%
\pgfpathlineto{\pgfqpoint{0.729083in}{0.417391in}}%
\pgfpathlineto{\pgfqpoint{0.768441in}{0.417391in}}%
\pgfpathlineto{\pgfqpoint{0.768441in}{0.417391in}}%
\pgfpathlineto{\pgfqpoint{0.807799in}{0.417391in}}%
\pgfpathlineto{\pgfqpoint{0.807799in}{0.417391in}}%
\pgfpathlineto{\pgfqpoint{0.847157in}{0.417391in}}%
\pgfpathlineto{\pgfqpoint{0.847157in}{0.417391in}}%
\pgfpathlineto{\pgfqpoint{0.886514in}{0.417391in}}%
\pgfpathlineto{\pgfqpoint{0.886514in}{0.417391in}}%
\pgfpathlineto{\pgfqpoint{0.925872in}{0.417391in}}%
\pgfpathlineto{\pgfqpoint{0.925872in}{0.417391in}}%
\pgfpathlineto{\pgfqpoint{0.965230in}{0.417391in}}%
\pgfpathlineto{\pgfqpoint{0.965230in}{0.417391in}}%
\pgfpathlineto{\pgfqpoint{1.004588in}{0.417391in}}%
\pgfpathlineto{\pgfqpoint{1.004588in}{0.417391in}}%
\pgfpathlineto{\pgfqpoint{1.043945in}{0.417391in}}%
\pgfpathlineto{\pgfqpoint{1.043945in}{0.417982in}}%
\pgfpathlineto{\pgfqpoint{1.083303in}{0.417982in}}%
\pgfpathlineto{\pgfqpoint{1.083303in}{0.421396in}}%
\pgfpathlineto{\pgfqpoint{1.122661in}{0.421396in}}%
\pgfpathlineto{\pgfqpoint{1.122661in}{0.420672in}}%
\pgfpathlineto{\pgfqpoint{1.162019in}{0.420672in}}%
\pgfpathlineto{\pgfqpoint{1.162019in}{0.446120in}}%
\pgfpathlineto{\pgfqpoint{1.201376in}{0.446120in}}%
\pgfpathlineto{\pgfqpoint{1.201376in}{0.447147in}}%
\pgfpathlineto{\pgfqpoint{1.240734in}{0.447147in}}%
\pgfpathlineto{\pgfqpoint{1.240734in}{0.470594in}}%
\pgfpathlineto{\pgfqpoint{1.280092in}{0.470594in}}%
\pgfpathlineto{\pgfqpoint{1.280092in}{0.491846in}}%
\pgfpathlineto{\pgfqpoint{1.319450in}{0.491846in}}%
\pgfpathlineto{\pgfqpoint{1.319450in}{0.488356in}}%
\pgfpathlineto{\pgfqpoint{1.358807in}{0.488356in}}%
\pgfpathlineto{\pgfqpoint{1.358807in}{0.502542in}}%
\pgfpathlineto{\pgfqpoint{1.398165in}{0.502542in}}%
\pgfpathlineto{\pgfqpoint{1.398165in}{0.540628in}}%
\pgfpathlineto{\pgfqpoint{1.437523in}{0.540628in}}%
\pgfpathlineto{\pgfqpoint{1.437523in}{0.533769in}}%
\pgfpathlineto{\pgfqpoint{1.476881in}{0.533769in}}%
\pgfpathlineto{\pgfqpoint{1.476881in}{0.597861in}}%
\pgfpathlineto{\pgfqpoint{1.516238in}{0.597861in}}%
\pgfpathlineto{\pgfqpoint{1.516238in}{0.631768in}}%
\pgfpathlineto{\pgfqpoint{1.555596in}{0.631768in}}%
\pgfpathlineto{\pgfqpoint{1.555596in}{0.716623in}}%
\pgfpathlineto{\pgfqpoint{1.594954in}{0.716623in}}%
\pgfpathlineto{\pgfqpoint{1.594954in}{0.632627in}}%
\pgfpathlineto{\pgfqpoint{1.634312in}{0.632627in}}%
\pgfpathlineto{\pgfqpoint{1.634312in}{0.669400in}}%
\pgfpathlineto{\pgfqpoint{1.673669in}{0.669400in}}%
\pgfpathlineto{\pgfqpoint{1.673669in}{0.710122in}}%
\pgfpathlineto{\pgfqpoint{1.713027in}{0.710122in}}%
\pgfpathlineto{\pgfqpoint{1.713027in}{0.900395in}}%
\pgfpathlineto{\pgfqpoint{1.752385in}{0.900395in}}%
\pgfpathlineto{\pgfqpoint{1.752385in}{0.752467in}}%
\pgfpathlineto{\pgfqpoint{1.791743in}{0.752467in}}%
\pgfpathlineto{\pgfqpoint{1.791743in}{0.763555in}}%
\pgfpathlineto{\pgfqpoint{1.831100in}{0.763555in}}%
\pgfpathlineto{\pgfqpoint{1.831100in}{0.586430in}}%
\pgfpathlineto{\pgfqpoint{1.870458in}{0.586430in}}%
\pgfpathlineto{\pgfqpoint{1.870458in}{0.543561in}}%
\pgfpathlineto{\pgfqpoint{1.909816in}{0.543561in}}%
\pgfpathlineto{\pgfqpoint{1.909816in}{0.696370in}}%
\pgfpathlineto{\pgfqpoint{1.949174in}{0.696370in}}%
\pgfpathlineto{\pgfqpoint{1.949174in}{0.496186in}}%
\pgfpathlineto{\pgfqpoint{1.988531in}{0.496186in}}%
\pgfpathlineto{\pgfqpoint{1.988531in}{0.533302in}}%
\pgfpathlineto{\pgfqpoint{2.027889in}{0.533302in}}%
\pgfpathlineto{\pgfqpoint{2.027889in}{0.512744in}}%
\pgfpathlineto{\pgfqpoint{2.067247in}{0.512744in}}%
\pgfpathlineto{\pgfqpoint{2.067247in}{0.461049in}}%
\pgfpathlineto{\pgfqpoint{2.106605in}{0.461049in}}%
\pgfpathlineto{\pgfqpoint{2.106605in}{0.438389in}}%
\pgfpathlineto{\pgfqpoint{2.145962in}{0.438389in}}%
\pgfpathlineto{\pgfqpoint{2.145962in}{0.448918in}}%
\pgfpathlineto{\pgfqpoint{2.185320in}{0.448918in}}%
\pgfpathlineto{\pgfqpoint{2.185320in}{0.431456in}}%
\pgfpathlineto{\pgfqpoint{2.224678in}{0.431456in}}%
\pgfpathlineto{\pgfqpoint{2.224678in}{0.420977in}}%
\pgfpathlineto{\pgfqpoint{2.264036in}{0.420977in}}%
\pgfpathlineto{\pgfqpoint{2.264036in}{0.417391in}}%
\pgfpathlineto{\pgfqpoint{2.224678in}{0.417391in}}%
\pgfpathlineto{\pgfqpoint{2.224678in}{0.417391in}}%
\pgfpathlineto{\pgfqpoint{2.185320in}{0.417391in}}%
\pgfpathlineto{\pgfqpoint{2.185320in}{0.417391in}}%
\pgfpathlineto{\pgfqpoint{2.145962in}{0.417391in}}%
\pgfpathlineto{\pgfqpoint{2.145962in}{0.417391in}}%
\pgfpathlineto{\pgfqpoint{2.106605in}{0.417391in}}%
\pgfpathlineto{\pgfqpoint{2.106605in}{0.417391in}}%
\pgfpathlineto{\pgfqpoint{2.067247in}{0.417391in}}%
\pgfpathlineto{\pgfqpoint{2.067247in}{0.417391in}}%
\pgfpathlineto{\pgfqpoint{2.027889in}{0.417391in}}%
\pgfpathlineto{\pgfqpoint{2.027889in}{0.417391in}}%
\pgfpathlineto{\pgfqpoint{1.988531in}{0.417391in}}%
\pgfpathlineto{\pgfqpoint{1.988531in}{0.417391in}}%
\pgfpathlineto{\pgfqpoint{1.949174in}{0.417391in}}%
\pgfpathlineto{\pgfqpoint{1.949174in}{0.417391in}}%
\pgfpathlineto{\pgfqpoint{1.909816in}{0.417391in}}%
\pgfpathlineto{\pgfqpoint{1.909816in}{0.417391in}}%
\pgfpathlineto{\pgfqpoint{1.870458in}{0.417391in}}%
\pgfpathlineto{\pgfqpoint{1.870458in}{0.417391in}}%
\pgfpathlineto{\pgfqpoint{1.831100in}{0.417391in}}%
\pgfpathlineto{\pgfqpoint{1.831100in}{0.417391in}}%
\pgfpathlineto{\pgfqpoint{1.791743in}{0.417391in}}%
\pgfpathlineto{\pgfqpoint{1.791743in}{0.417391in}}%
\pgfpathlineto{\pgfqpoint{1.752385in}{0.417391in}}%
\pgfpathlineto{\pgfqpoint{1.752385in}{0.417391in}}%
\pgfpathlineto{\pgfqpoint{1.713027in}{0.417391in}}%
\pgfpathlineto{\pgfqpoint{1.713027in}{0.417391in}}%
\pgfpathlineto{\pgfqpoint{1.673669in}{0.417391in}}%
\pgfpathlineto{\pgfqpoint{1.673669in}{0.417391in}}%
\pgfpathlineto{\pgfqpoint{1.634312in}{0.417391in}}%
\pgfpathlineto{\pgfqpoint{1.634312in}{0.417391in}}%
\pgfpathlineto{\pgfqpoint{1.594954in}{0.417391in}}%
\pgfpathlineto{\pgfqpoint{1.594954in}{0.417391in}}%
\pgfpathlineto{\pgfqpoint{1.555596in}{0.417391in}}%
\pgfpathlineto{\pgfqpoint{1.555596in}{0.417391in}}%
\pgfpathlineto{\pgfqpoint{1.516238in}{0.417391in}}%
\pgfpathlineto{\pgfqpoint{1.516238in}{0.417391in}}%
\pgfpathlineto{\pgfqpoint{1.476881in}{0.417391in}}%
\pgfpathlineto{\pgfqpoint{1.476881in}{0.417391in}}%
\pgfpathlineto{\pgfqpoint{1.437523in}{0.417391in}}%
\pgfpathlineto{\pgfqpoint{1.437523in}{0.417391in}}%
\pgfpathlineto{\pgfqpoint{1.398165in}{0.417391in}}%
\pgfpathlineto{\pgfqpoint{1.398165in}{0.417391in}}%
\pgfpathlineto{\pgfqpoint{1.358807in}{0.417391in}}%
\pgfpathlineto{\pgfqpoint{1.358807in}{0.417391in}}%
\pgfpathlineto{\pgfqpoint{1.319450in}{0.417391in}}%
\pgfpathlineto{\pgfqpoint{1.319450in}{0.417391in}}%
\pgfpathlineto{\pgfqpoint{1.280092in}{0.417391in}}%
\pgfpathlineto{\pgfqpoint{1.280092in}{0.417391in}}%
\pgfpathlineto{\pgfqpoint{1.240734in}{0.417391in}}%
\pgfpathlineto{\pgfqpoint{1.240734in}{0.417391in}}%
\pgfpathlineto{\pgfqpoint{1.201376in}{0.417391in}}%
\pgfpathlineto{\pgfqpoint{1.201376in}{0.417391in}}%
\pgfpathlineto{\pgfqpoint{1.162019in}{0.417391in}}%
\pgfpathlineto{\pgfqpoint{1.162019in}{0.417391in}}%
\pgfpathlineto{\pgfqpoint{1.122661in}{0.417391in}}%
\pgfpathlineto{\pgfqpoint{1.122661in}{0.417391in}}%
\pgfpathlineto{\pgfqpoint{1.083303in}{0.417391in}}%
\pgfpathlineto{\pgfqpoint{1.083303in}{0.417391in}}%
\pgfpathlineto{\pgfqpoint{1.043945in}{0.417391in}}%
\pgfpathlineto{\pgfqpoint{1.043945in}{0.417391in}}%
\pgfpathlineto{\pgfqpoint{1.004588in}{0.417391in}}%
\pgfpathlineto{\pgfqpoint{1.004588in}{0.417391in}}%
\pgfpathlineto{\pgfqpoint{0.965230in}{0.417391in}}%
\pgfpathlineto{\pgfqpoint{0.965230in}{0.417391in}}%
\pgfpathlineto{\pgfqpoint{0.925872in}{0.417391in}}%
\pgfpathlineto{\pgfqpoint{0.925872in}{0.417391in}}%
\pgfpathlineto{\pgfqpoint{0.886514in}{0.417391in}}%
\pgfpathlineto{\pgfqpoint{0.886514in}{0.417391in}}%
\pgfpathlineto{\pgfqpoint{0.847157in}{0.417391in}}%
\pgfpathlineto{\pgfqpoint{0.847157in}{0.417391in}}%
\pgfpathlineto{\pgfqpoint{0.807799in}{0.417391in}}%
\pgfpathlineto{\pgfqpoint{0.807799in}{0.417391in}}%
\pgfpathlineto{\pgfqpoint{0.768441in}{0.417391in}}%
\pgfpathlineto{\pgfqpoint{0.768441in}{0.417391in}}%
\pgfpathlineto{\pgfqpoint{0.729083in}{0.417391in}}%
\pgfpathlineto{\pgfqpoint{0.729083in}{0.417391in}}%
\pgfpathlineto{\pgfqpoint{0.689726in}{0.417391in}}%
\pgfpathlineto{\pgfqpoint{0.689726in}{0.417391in}}%
\pgfpathlineto{\pgfqpoint{0.650368in}{0.417391in}}%
\pgfpathlineto{\pgfqpoint{0.650368in}{0.417391in}}%
\pgfpathlineto{\pgfqpoint{0.611010in}{0.417391in}}%
\pgfpathlineto{\pgfqpoint{0.611010in}{0.417391in}}%
\pgfpathlineto{\pgfqpoint{0.571652in}{0.417391in}}%
\pgfpathlineto{\pgfqpoint{0.571652in}{0.417391in}}%
\pgfpathlineto{\pgfqpoint{0.532295in}{0.417391in}}%
\pgfpathlineto{\pgfqpoint{0.532295in}{0.417391in}}%
\pgfpathlineto{\pgfqpoint{0.492937in}{0.417391in}}%
\pgfpathlineto{\pgfqpoint{0.492937in}{0.417391in}}%
\pgfpathlineto{\pgfqpoint{0.453579in}{0.417391in}}%
\pgfpathlineto{\pgfqpoint{0.453579in}{0.417391in}}%
\pgfpathlineto{\pgfqpoint{0.414221in}{0.417391in}}%
\pgfpathlineto{\pgfqpoint{0.414221in}{0.417391in}}%
\pgfpathlineto{\pgfqpoint{0.374864in}{0.417391in}}%
\pgfpathlineto{\pgfqpoint{0.374864in}{0.417391in}}%
\pgfpathlineto{\pgfqpoint{0.335506in}{0.417391in}}%
\pgfpathlineto{\pgfqpoint{0.335506in}{0.417391in}}%
\pgfpathlineto{\pgfqpoint{0.296148in}{0.417391in}}%
\pgfusepath{fill}%
\end{pgfscope}%
\begin{pgfscope}%
\pgfpathrectangle{\pgfqpoint{0.296148in}{0.417391in}}{\pgfqpoint{2.245646in}{1.266124in}} %
\pgfusepath{clip}%
\pgfsetbuttcap%
\pgfsetmiterjoin%
\pgfsetlinewidth{0.501875pt}%
\definecolor{currentstroke}{rgb}{1.000000,0.000000,0.000000}%
\pgfsetstrokecolor{currentstroke}%
\pgfsetdash{}{0pt}%
\pgfpathmoveto{\pgfqpoint{0.296148in}{0.417391in}}%
\pgfpathlineto{\pgfqpoint{0.296148in}{0.533703in}}%
\pgfpathlineto{\pgfqpoint{0.335506in}{0.533703in}}%
\pgfpathlineto{\pgfqpoint{0.335506in}{0.417391in}}%
\pgfpathlineto{\pgfqpoint{0.374864in}{0.417391in}}%
\pgfpathlineto{\pgfqpoint{0.374864in}{0.417391in}}%
\pgfpathlineto{\pgfqpoint{0.414221in}{0.417391in}}%
\pgfpathlineto{\pgfqpoint{0.414221in}{0.417391in}}%
\pgfpathlineto{\pgfqpoint{0.453579in}{0.417391in}}%
\pgfpathlineto{\pgfqpoint{0.453579in}{0.417391in}}%
\pgfpathlineto{\pgfqpoint{0.492937in}{0.417391in}}%
\pgfpathlineto{\pgfqpoint{0.492937in}{0.417391in}}%
\pgfpathlineto{\pgfqpoint{0.532295in}{0.417391in}}%
\pgfpathlineto{\pgfqpoint{0.532295in}{0.417391in}}%
\pgfpathlineto{\pgfqpoint{0.571652in}{0.417391in}}%
\pgfpathlineto{\pgfqpoint{0.571652in}{0.417391in}}%
\pgfpathlineto{\pgfqpoint{0.611010in}{0.417391in}}%
\pgfpathlineto{\pgfqpoint{0.611010in}{0.417391in}}%
\pgfpathlineto{\pgfqpoint{0.650368in}{0.417391in}}%
\pgfpathlineto{\pgfqpoint{0.650368in}{0.417391in}}%
\pgfpathlineto{\pgfqpoint{0.689726in}{0.417391in}}%
\pgfpathlineto{\pgfqpoint{0.689726in}{0.417391in}}%
\pgfpathlineto{\pgfqpoint{0.729083in}{0.417391in}}%
\pgfpathlineto{\pgfqpoint{0.729083in}{0.417391in}}%
\pgfpathlineto{\pgfqpoint{0.768441in}{0.417391in}}%
\pgfpathlineto{\pgfqpoint{0.768441in}{0.417391in}}%
\pgfpathlineto{\pgfqpoint{0.807799in}{0.417391in}}%
\pgfpathlineto{\pgfqpoint{0.807799in}{0.417391in}}%
\pgfpathlineto{\pgfqpoint{0.847157in}{0.417391in}}%
\pgfpathlineto{\pgfqpoint{0.847157in}{0.417391in}}%
\pgfpathlineto{\pgfqpoint{0.886514in}{0.417391in}}%
\pgfpathlineto{\pgfqpoint{0.886514in}{0.417391in}}%
\pgfpathlineto{\pgfqpoint{0.925872in}{0.417391in}}%
\pgfpathlineto{\pgfqpoint{0.925872in}{0.417391in}}%
\pgfpathlineto{\pgfqpoint{0.965230in}{0.417391in}}%
\pgfpathlineto{\pgfqpoint{0.965230in}{0.417391in}}%
\pgfpathlineto{\pgfqpoint{1.004588in}{0.417391in}}%
\pgfpathlineto{\pgfqpoint{1.004588in}{0.417391in}}%
\pgfpathlineto{\pgfqpoint{1.043945in}{0.417391in}}%
\pgfpathlineto{\pgfqpoint{1.043945in}{0.417428in}}%
\pgfpathlineto{\pgfqpoint{1.083303in}{0.417428in}}%
\pgfpathlineto{\pgfqpoint{1.083303in}{0.417756in}}%
\pgfpathlineto{\pgfqpoint{1.122661in}{0.417756in}}%
\pgfpathlineto{\pgfqpoint{1.122661in}{0.417683in}}%
\pgfpathlineto{\pgfqpoint{1.162019in}{0.417683in}}%
\pgfpathlineto{\pgfqpoint{1.162019in}{0.420278in}}%
\pgfpathlineto{\pgfqpoint{1.201376in}{0.420278in}}%
\pgfpathlineto{\pgfqpoint{1.201376in}{0.419974in}}%
\pgfpathlineto{\pgfqpoint{1.240734in}{0.419974in}}%
\pgfpathlineto{\pgfqpoint{1.240734in}{0.422982in}}%
\pgfpathlineto{\pgfqpoint{1.280092in}{0.422982in}}%
\pgfpathlineto{\pgfqpoint{1.280092in}{0.424858in}}%
\pgfpathlineto{\pgfqpoint{1.319450in}{0.424858in}}%
\pgfpathlineto{\pgfqpoint{1.319450in}{0.425260in}}%
\pgfpathlineto{\pgfqpoint{1.358807in}{0.425260in}}%
\pgfpathlineto{\pgfqpoint{1.358807in}{0.426710in}}%
\pgfpathlineto{\pgfqpoint{1.398165in}{0.426710in}}%
\pgfpathlineto{\pgfqpoint{1.398165in}{0.432923in}}%
\pgfpathlineto{\pgfqpoint{1.437523in}{0.432923in}}%
\pgfpathlineto{\pgfqpoint{1.437523in}{0.434092in}}%
\pgfpathlineto{\pgfqpoint{1.476881in}{0.434092in}}%
\pgfpathlineto{\pgfqpoint{1.476881in}{0.447821in}}%
\pgfpathlineto{\pgfqpoint{1.516238in}{0.447821in}}%
\pgfpathlineto{\pgfqpoint{1.516238in}{0.464376in}}%
\pgfpathlineto{\pgfqpoint{1.555596in}{0.464376in}}%
\pgfpathlineto{\pgfqpoint{1.555596in}{0.500459in}}%
\pgfpathlineto{\pgfqpoint{1.594954in}{0.500459in}}%
\pgfpathlineto{\pgfqpoint{1.594954in}{0.489727in}}%
\pgfpathlineto{\pgfqpoint{1.634312in}{0.489727in}}%
\pgfpathlineto{\pgfqpoint{1.634312in}{0.533228in}}%
\pgfpathlineto{\pgfqpoint{1.673669in}{0.533228in}}%
\pgfpathlineto{\pgfqpoint{1.673669in}{0.574829in}}%
\pgfpathlineto{\pgfqpoint{1.713027in}{0.574829in}}%
\pgfpathlineto{\pgfqpoint{1.713027in}{0.879059in}}%
\pgfpathlineto{\pgfqpoint{1.752385in}{0.879059in}}%
\pgfpathlineto{\pgfqpoint{1.752385in}{0.891521in}}%
\pgfpathlineto{\pgfqpoint{1.791743in}{0.891521in}}%
\pgfpathlineto{\pgfqpoint{1.791743in}{0.998465in}}%
\pgfpathlineto{\pgfqpoint{1.831100in}{0.998465in}}%
\pgfpathlineto{\pgfqpoint{1.831100in}{0.786379in}}%
\pgfpathlineto{\pgfqpoint{1.870458in}{0.786379in}}%
\pgfpathlineto{\pgfqpoint{1.870458in}{0.730769in}}%
\pgfpathlineto{\pgfqpoint{1.909816in}{0.730769in}}%
\pgfpathlineto{\pgfqpoint{1.909816in}{1.179085in}}%
\pgfpathlineto{\pgfqpoint{1.949174in}{1.179085in}}%
\pgfpathlineto{\pgfqpoint{1.949174in}{0.737615in}}%
\pgfpathlineto{\pgfqpoint{1.988531in}{0.737615in}}%
\pgfpathlineto{\pgfqpoint{1.988531in}{0.775196in}}%
\pgfpathlineto{\pgfqpoint{2.027889in}{0.775196in}}%
\pgfpathlineto{\pgfqpoint{2.027889in}{0.782591in}}%
\pgfpathlineto{\pgfqpoint{2.067247in}{0.782591in}}%
\pgfpathlineto{\pgfqpoint{2.067247in}{0.750796in}}%
\pgfpathlineto{\pgfqpoint{2.106605in}{0.750796in}}%
\pgfpathlineto{\pgfqpoint{2.106605in}{0.505892in}}%
\pgfpathlineto{\pgfqpoint{2.145962in}{0.505892in}}%
\pgfpathlineto{\pgfqpoint{2.145962in}{0.536590in}}%
\pgfpathlineto{\pgfqpoint{2.185320in}{0.536590in}}%
\pgfpathlineto{\pgfqpoint{2.185320in}{0.518720in}}%
\pgfpathlineto{\pgfqpoint{2.224678in}{0.518720in}}%
\pgfpathlineto{\pgfqpoint{2.224678in}{0.430974in}}%
\pgfpathlineto{\pgfqpoint{2.264036in}{0.430974in}}%
\pgfpathlineto{\pgfqpoint{2.264036in}{0.417391in}}%
\pgfusepath{stroke}%
\end{pgfscope}%
\begin{pgfscope}%
\pgfsetrectcap%
\pgfsetmiterjoin%
\pgfsetlinewidth{1.003750pt}%
\definecolor{currentstroke}{rgb}{0.000000,0.000000,0.000000}%
\pgfsetstrokecolor{currentstroke}%
\pgfsetdash{}{0pt}%
\pgfpathmoveto{\pgfqpoint{0.296148in}{1.683515in}}%
\pgfpathlineto{\pgfqpoint{2.541794in}{1.683515in}}%
\pgfusepath{stroke}%
\end{pgfscope}%
\begin{pgfscope}%
\pgfsetrectcap%
\pgfsetmiterjoin%
\pgfsetlinewidth{1.003750pt}%
\definecolor{currentstroke}{rgb}{0.000000,0.000000,0.000000}%
\pgfsetstrokecolor{currentstroke}%
\pgfsetdash{}{0pt}%
\pgfpathmoveto{\pgfqpoint{2.541794in}{0.417391in}}%
\pgfpathlineto{\pgfqpoint{2.541794in}{1.683515in}}%
\pgfusepath{stroke}%
\end{pgfscope}%
\begin{pgfscope}%
\pgfsetrectcap%
\pgfsetmiterjoin%
\pgfsetlinewidth{1.003750pt}%
\definecolor{currentstroke}{rgb}{0.000000,0.000000,0.000000}%
\pgfsetstrokecolor{currentstroke}%
\pgfsetdash{}{0pt}%
\pgfpathmoveto{\pgfqpoint{0.296148in}{0.417391in}}%
\pgfpathlineto{\pgfqpoint{2.541794in}{0.417391in}}%
\pgfusepath{stroke}%
\end{pgfscope}%
\begin{pgfscope}%
\pgfsetrectcap%
\pgfsetmiterjoin%
\pgfsetlinewidth{1.003750pt}%
\definecolor{currentstroke}{rgb}{0.000000,0.000000,0.000000}%
\pgfsetstrokecolor{currentstroke}%
\pgfsetdash{}{0pt}%
\pgfpathmoveto{\pgfqpoint{0.296148in}{0.417391in}}%
\pgfpathlineto{\pgfqpoint{0.296148in}{1.683515in}}%
\pgfusepath{stroke}%
\end{pgfscope}%
\begin{pgfscope}%
\pgfsetbuttcap%
\pgfsetroundjoin%
\definecolor{currentfill}{rgb}{0.000000,0.000000,0.000000}%
\pgfsetfillcolor{currentfill}%
\pgfsetlinewidth{0.501875pt}%
\definecolor{currentstroke}{rgb}{0.000000,0.000000,0.000000}%
\pgfsetstrokecolor{currentstroke}%
\pgfsetdash{}{0pt}%
\pgfsys@defobject{currentmarker}{\pgfqpoint{0.000000in}{0.000000in}}{\pgfqpoint{0.000000in}{0.069444in}}{%
\pgfpathmoveto{\pgfqpoint{0.000000in}{0.000000in}}%
\pgfpathlineto{\pgfqpoint{0.000000in}{0.069444in}}%
\pgfusepath{stroke,fill}%
}%
\begin{pgfscope}%
\pgfsys@transformshift{0.296148in}{0.417391in}%
\pgfsys@useobject{currentmarker}{}%
\end{pgfscope}%
\end{pgfscope}%
\begin{pgfscope}%
\pgfsetbuttcap%
\pgfsetroundjoin%
\definecolor{currentfill}{rgb}{0.000000,0.000000,0.000000}%
\pgfsetfillcolor{currentfill}%
\pgfsetlinewidth{0.501875pt}%
\definecolor{currentstroke}{rgb}{0.000000,0.000000,0.000000}%
\pgfsetstrokecolor{currentstroke}%
\pgfsetdash{}{0pt}%
\pgfsys@defobject{currentmarker}{\pgfqpoint{0.000000in}{-0.069444in}}{\pgfqpoint{0.000000in}{0.000000in}}{%
\pgfpathmoveto{\pgfqpoint{0.000000in}{0.000000in}}%
\pgfpathlineto{\pgfqpoint{0.000000in}{-0.069444in}}%
\pgfusepath{stroke,fill}%
}%
\begin{pgfscope}%
\pgfsys@transformshift{0.296148in}{1.683515in}%
\pgfsys@useobject{currentmarker}{}%
\end{pgfscope}%
\end{pgfscope}%
\begin{pgfscope}%
\pgftext[x=0.296148in,y=0.347947in,,top]{\rmfamily\fontsize{8.000000}{9.600000}\selectfont −2.0}%
\end{pgfscope}%
\begin{pgfscope}%
\pgfsetbuttcap%
\pgfsetroundjoin%
\definecolor{currentfill}{rgb}{0.000000,0.000000,0.000000}%
\pgfsetfillcolor{currentfill}%
\pgfsetlinewidth{0.501875pt}%
\definecolor{currentstroke}{rgb}{0.000000,0.000000,0.000000}%
\pgfsetstrokecolor{currentstroke}%
\pgfsetdash{}{0pt}%
\pgfsys@defobject{currentmarker}{\pgfqpoint{0.000000in}{0.000000in}}{\pgfqpoint{0.000000in}{0.069444in}}{%
\pgfpathmoveto{\pgfqpoint{0.000000in}{0.000000in}}%
\pgfpathlineto{\pgfqpoint{0.000000in}{0.069444in}}%
\pgfusepath{stroke,fill}%
}%
\begin{pgfscope}%
\pgfsys@transformshift{0.670422in}{0.417391in}%
\pgfsys@useobject{currentmarker}{}%
\end{pgfscope}%
\end{pgfscope}%
\begin{pgfscope}%
\pgfsetbuttcap%
\pgfsetroundjoin%
\definecolor{currentfill}{rgb}{0.000000,0.000000,0.000000}%
\pgfsetfillcolor{currentfill}%
\pgfsetlinewidth{0.501875pt}%
\definecolor{currentstroke}{rgb}{0.000000,0.000000,0.000000}%
\pgfsetstrokecolor{currentstroke}%
\pgfsetdash{}{0pt}%
\pgfsys@defobject{currentmarker}{\pgfqpoint{0.000000in}{-0.069444in}}{\pgfqpoint{0.000000in}{0.000000in}}{%
\pgfpathmoveto{\pgfqpoint{0.000000in}{0.000000in}}%
\pgfpathlineto{\pgfqpoint{0.000000in}{-0.069444in}}%
\pgfusepath{stroke,fill}%
}%
\begin{pgfscope}%
\pgfsys@transformshift{0.670422in}{1.683515in}%
\pgfsys@useobject{currentmarker}{}%
\end{pgfscope}%
\end{pgfscope}%
\begin{pgfscope}%
\pgftext[x=0.670422in,y=0.347947in,,top]{\rmfamily\fontsize{8.000000}{9.600000}\selectfont −1.5}%
\end{pgfscope}%
\begin{pgfscope}%
\pgfsetbuttcap%
\pgfsetroundjoin%
\definecolor{currentfill}{rgb}{0.000000,0.000000,0.000000}%
\pgfsetfillcolor{currentfill}%
\pgfsetlinewidth{0.501875pt}%
\definecolor{currentstroke}{rgb}{0.000000,0.000000,0.000000}%
\pgfsetstrokecolor{currentstroke}%
\pgfsetdash{}{0pt}%
\pgfsys@defobject{currentmarker}{\pgfqpoint{0.000000in}{0.000000in}}{\pgfqpoint{0.000000in}{0.069444in}}{%
\pgfpathmoveto{\pgfqpoint{0.000000in}{0.000000in}}%
\pgfpathlineto{\pgfqpoint{0.000000in}{0.069444in}}%
\pgfusepath{stroke,fill}%
}%
\begin{pgfscope}%
\pgfsys@transformshift{1.044697in}{0.417391in}%
\pgfsys@useobject{currentmarker}{}%
\end{pgfscope}%
\end{pgfscope}%
\begin{pgfscope}%
\pgfsetbuttcap%
\pgfsetroundjoin%
\definecolor{currentfill}{rgb}{0.000000,0.000000,0.000000}%
\pgfsetfillcolor{currentfill}%
\pgfsetlinewidth{0.501875pt}%
\definecolor{currentstroke}{rgb}{0.000000,0.000000,0.000000}%
\pgfsetstrokecolor{currentstroke}%
\pgfsetdash{}{0pt}%
\pgfsys@defobject{currentmarker}{\pgfqpoint{0.000000in}{-0.069444in}}{\pgfqpoint{0.000000in}{0.000000in}}{%
\pgfpathmoveto{\pgfqpoint{0.000000in}{0.000000in}}%
\pgfpathlineto{\pgfqpoint{0.000000in}{-0.069444in}}%
\pgfusepath{stroke,fill}%
}%
\begin{pgfscope}%
\pgfsys@transformshift{1.044697in}{1.683515in}%
\pgfsys@useobject{currentmarker}{}%
\end{pgfscope}%
\end{pgfscope}%
\begin{pgfscope}%
\pgftext[x=1.044697in,y=0.347947in,,top]{\rmfamily\fontsize{8.000000}{9.600000}\selectfont −1.0}%
\end{pgfscope}%
\begin{pgfscope}%
\pgfsetbuttcap%
\pgfsetroundjoin%
\definecolor{currentfill}{rgb}{0.000000,0.000000,0.000000}%
\pgfsetfillcolor{currentfill}%
\pgfsetlinewidth{0.501875pt}%
\definecolor{currentstroke}{rgb}{0.000000,0.000000,0.000000}%
\pgfsetstrokecolor{currentstroke}%
\pgfsetdash{}{0pt}%
\pgfsys@defobject{currentmarker}{\pgfqpoint{0.000000in}{0.000000in}}{\pgfqpoint{0.000000in}{0.069444in}}{%
\pgfpathmoveto{\pgfqpoint{0.000000in}{0.000000in}}%
\pgfpathlineto{\pgfqpoint{0.000000in}{0.069444in}}%
\pgfusepath{stroke,fill}%
}%
\begin{pgfscope}%
\pgfsys@transformshift{1.418971in}{0.417391in}%
\pgfsys@useobject{currentmarker}{}%
\end{pgfscope}%
\end{pgfscope}%
\begin{pgfscope}%
\pgfsetbuttcap%
\pgfsetroundjoin%
\definecolor{currentfill}{rgb}{0.000000,0.000000,0.000000}%
\pgfsetfillcolor{currentfill}%
\pgfsetlinewidth{0.501875pt}%
\definecolor{currentstroke}{rgb}{0.000000,0.000000,0.000000}%
\pgfsetstrokecolor{currentstroke}%
\pgfsetdash{}{0pt}%
\pgfsys@defobject{currentmarker}{\pgfqpoint{0.000000in}{-0.069444in}}{\pgfqpoint{0.000000in}{0.000000in}}{%
\pgfpathmoveto{\pgfqpoint{0.000000in}{0.000000in}}%
\pgfpathlineto{\pgfqpoint{0.000000in}{-0.069444in}}%
\pgfusepath{stroke,fill}%
}%
\begin{pgfscope}%
\pgfsys@transformshift{1.418971in}{1.683515in}%
\pgfsys@useobject{currentmarker}{}%
\end{pgfscope}%
\end{pgfscope}%
\begin{pgfscope}%
\pgftext[x=1.418971in,y=0.347947in,,top]{\rmfamily\fontsize{8.000000}{9.600000}\selectfont −0.5}%
\end{pgfscope}%
\begin{pgfscope}%
\pgfsetbuttcap%
\pgfsetroundjoin%
\definecolor{currentfill}{rgb}{0.000000,0.000000,0.000000}%
\pgfsetfillcolor{currentfill}%
\pgfsetlinewidth{0.501875pt}%
\definecolor{currentstroke}{rgb}{0.000000,0.000000,0.000000}%
\pgfsetstrokecolor{currentstroke}%
\pgfsetdash{}{0pt}%
\pgfsys@defobject{currentmarker}{\pgfqpoint{0.000000in}{0.000000in}}{\pgfqpoint{0.000000in}{0.069444in}}{%
\pgfpathmoveto{\pgfqpoint{0.000000in}{0.000000in}}%
\pgfpathlineto{\pgfqpoint{0.000000in}{0.069444in}}%
\pgfusepath{stroke,fill}%
}%
\begin{pgfscope}%
\pgfsys@transformshift{1.793245in}{0.417391in}%
\pgfsys@useobject{currentmarker}{}%
\end{pgfscope}%
\end{pgfscope}%
\begin{pgfscope}%
\pgfsetbuttcap%
\pgfsetroundjoin%
\definecolor{currentfill}{rgb}{0.000000,0.000000,0.000000}%
\pgfsetfillcolor{currentfill}%
\pgfsetlinewidth{0.501875pt}%
\definecolor{currentstroke}{rgb}{0.000000,0.000000,0.000000}%
\pgfsetstrokecolor{currentstroke}%
\pgfsetdash{}{0pt}%
\pgfsys@defobject{currentmarker}{\pgfqpoint{0.000000in}{-0.069444in}}{\pgfqpoint{0.000000in}{0.000000in}}{%
\pgfpathmoveto{\pgfqpoint{0.000000in}{0.000000in}}%
\pgfpathlineto{\pgfqpoint{0.000000in}{-0.069444in}}%
\pgfusepath{stroke,fill}%
}%
\begin{pgfscope}%
\pgfsys@transformshift{1.793245in}{1.683515in}%
\pgfsys@useobject{currentmarker}{}%
\end{pgfscope}%
\end{pgfscope}%
\begin{pgfscope}%
\pgftext[x=1.793245in,y=0.347947in,,top]{\rmfamily\fontsize{8.000000}{9.600000}\selectfont 0.0}%
\end{pgfscope}%
\begin{pgfscope}%
\pgfsetbuttcap%
\pgfsetroundjoin%
\definecolor{currentfill}{rgb}{0.000000,0.000000,0.000000}%
\pgfsetfillcolor{currentfill}%
\pgfsetlinewidth{0.501875pt}%
\definecolor{currentstroke}{rgb}{0.000000,0.000000,0.000000}%
\pgfsetstrokecolor{currentstroke}%
\pgfsetdash{}{0pt}%
\pgfsys@defobject{currentmarker}{\pgfqpoint{0.000000in}{0.000000in}}{\pgfqpoint{0.000000in}{0.069444in}}{%
\pgfpathmoveto{\pgfqpoint{0.000000in}{0.000000in}}%
\pgfpathlineto{\pgfqpoint{0.000000in}{0.069444in}}%
\pgfusepath{stroke,fill}%
}%
\begin{pgfscope}%
\pgfsys@transformshift{2.167520in}{0.417391in}%
\pgfsys@useobject{currentmarker}{}%
\end{pgfscope}%
\end{pgfscope}%
\begin{pgfscope}%
\pgfsetbuttcap%
\pgfsetroundjoin%
\definecolor{currentfill}{rgb}{0.000000,0.000000,0.000000}%
\pgfsetfillcolor{currentfill}%
\pgfsetlinewidth{0.501875pt}%
\definecolor{currentstroke}{rgb}{0.000000,0.000000,0.000000}%
\pgfsetstrokecolor{currentstroke}%
\pgfsetdash{}{0pt}%
\pgfsys@defobject{currentmarker}{\pgfqpoint{0.000000in}{-0.069444in}}{\pgfqpoint{0.000000in}{0.000000in}}{%
\pgfpathmoveto{\pgfqpoint{0.000000in}{0.000000in}}%
\pgfpathlineto{\pgfqpoint{0.000000in}{-0.069444in}}%
\pgfusepath{stroke,fill}%
}%
\begin{pgfscope}%
\pgfsys@transformshift{2.167520in}{1.683515in}%
\pgfsys@useobject{currentmarker}{}%
\end{pgfscope}%
\end{pgfscope}%
\begin{pgfscope}%
\pgftext[x=2.167520in,y=0.347947in,,top]{\rmfamily\fontsize{8.000000}{9.600000}\selectfont 0.5}%
\end{pgfscope}%
\begin{pgfscope}%
\pgfsetbuttcap%
\pgfsetroundjoin%
\definecolor{currentfill}{rgb}{0.000000,0.000000,0.000000}%
\pgfsetfillcolor{currentfill}%
\pgfsetlinewidth{0.501875pt}%
\definecolor{currentstroke}{rgb}{0.000000,0.000000,0.000000}%
\pgfsetstrokecolor{currentstroke}%
\pgfsetdash{}{0pt}%
\pgfsys@defobject{currentmarker}{\pgfqpoint{0.000000in}{0.000000in}}{\pgfqpoint{0.000000in}{0.069444in}}{%
\pgfpathmoveto{\pgfqpoint{0.000000in}{0.000000in}}%
\pgfpathlineto{\pgfqpoint{0.000000in}{0.069444in}}%
\pgfusepath{stroke,fill}%
}%
\begin{pgfscope}%
\pgfsys@transformshift{2.541794in}{0.417391in}%
\pgfsys@useobject{currentmarker}{}%
\end{pgfscope}%
\end{pgfscope}%
\begin{pgfscope}%
\pgfsetbuttcap%
\pgfsetroundjoin%
\definecolor{currentfill}{rgb}{0.000000,0.000000,0.000000}%
\pgfsetfillcolor{currentfill}%
\pgfsetlinewidth{0.501875pt}%
\definecolor{currentstroke}{rgb}{0.000000,0.000000,0.000000}%
\pgfsetstrokecolor{currentstroke}%
\pgfsetdash{}{0pt}%
\pgfsys@defobject{currentmarker}{\pgfqpoint{0.000000in}{-0.069444in}}{\pgfqpoint{0.000000in}{0.000000in}}{%
\pgfpathmoveto{\pgfqpoint{0.000000in}{0.000000in}}%
\pgfpathlineto{\pgfqpoint{0.000000in}{-0.069444in}}%
\pgfusepath{stroke,fill}%
}%
\begin{pgfscope}%
\pgfsys@transformshift{2.541794in}{1.683515in}%
\pgfsys@useobject{currentmarker}{}%
\end{pgfscope}%
\end{pgfscope}%
\begin{pgfscope}%
\pgftext[x=2.541794in,y=0.347947in,,top]{\rmfamily\fontsize{8.000000}{9.600000}\selectfont 1.0}%
\end{pgfscope}%
\begin{pgfscope}%
\pgftext[x=1.418971in,y=0.170972in,,top]{\rmfamily\fontsize{9.000000}{10.800000}\selectfont muon isolation BDT response}%
\end{pgfscope}%
\begin{pgfscope}%
\pgfsetbuttcap%
\pgfsetroundjoin%
\definecolor{currentfill}{rgb}{0.000000,0.000000,0.000000}%
\pgfsetfillcolor{currentfill}%
\pgfsetlinewidth{0.501875pt}%
\definecolor{currentstroke}{rgb}{0.000000,0.000000,0.000000}%
\pgfsetstrokecolor{currentstroke}%
\pgfsetdash{}{0pt}%
\pgfsys@defobject{currentmarker}{\pgfqpoint{0.000000in}{0.000000in}}{\pgfqpoint{0.069444in}{0.000000in}}{%
\pgfpathmoveto{\pgfqpoint{0.000000in}{0.000000in}}%
\pgfpathlineto{\pgfqpoint{0.069444in}{0.000000in}}%
\pgfusepath{stroke,fill}%
}%
\begin{pgfscope}%
\pgfsys@transformshift{0.296148in}{0.417391in}%
\pgfsys@useobject{currentmarker}{}%
\end{pgfscope}%
\end{pgfscope}%
\begin{pgfscope}%
\pgfsetbuttcap%
\pgfsetroundjoin%
\definecolor{currentfill}{rgb}{0.000000,0.000000,0.000000}%
\pgfsetfillcolor{currentfill}%
\pgfsetlinewidth{0.501875pt}%
\definecolor{currentstroke}{rgb}{0.000000,0.000000,0.000000}%
\pgfsetstrokecolor{currentstroke}%
\pgfsetdash{}{0pt}%
\pgfsys@defobject{currentmarker}{\pgfqpoint{-0.069444in}{0.000000in}}{\pgfqpoint{0.000000in}{0.000000in}}{%
\pgfpathmoveto{\pgfqpoint{0.000000in}{0.000000in}}%
\pgfpathlineto{\pgfqpoint{-0.069444in}{0.000000in}}%
\pgfusepath{stroke,fill}%
}%
\begin{pgfscope}%
\pgfsys@transformshift{2.541794in}{0.417391in}%
\pgfsys@useobject{currentmarker}{}%
\end{pgfscope}%
\end{pgfscope}%
\begin{pgfscope}%
\pgftext[x=0.226704in,y=0.417391in,right,]{\rmfamily\fontsize{8.000000}{9.600000}\selectfont 0.0}%
\end{pgfscope}%
\begin{pgfscope}%
\pgfsetbuttcap%
\pgfsetroundjoin%
\definecolor{currentfill}{rgb}{0.000000,0.000000,0.000000}%
\pgfsetfillcolor{currentfill}%
\pgfsetlinewidth{0.501875pt}%
\definecolor{currentstroke}{rgb}{0.000000,0.000000,0.000000}%
\pgfsetstrokecolor{currentstroke}%
\pgfsetdash{}{0pt}%
\pgfsys@defobject{currentmarker}{\pgfqpoint{0.000000in}{0.000000in}}{\pgfqpoint{0.069444in}{0.000000in}}{%
\pgfpathmoveto{\pgfqpoint{0.000000in}{0.000000in}}%
\pgfpathlineto{\pgfqpoint{0.069444in}{0.000000in}}%
\pgfusepath{stroke,fill}%
}%
\begin{pgfscope}%
\pgfsys@transformshift{0.296148in}{0.558071in}%
\pgfsys@useobject{currentmarker}{}%
\end{pgfscope}%
\end{pgfscope}%
\begin{pgfscope}%
\pgfsetbuttcap%
\pgfsetroundjoin%
\definecolor{currentfill}{rgb}{0.000000,0.000000,0.000000}%
\pgfsetfillcolor{currentfill}%
\pgfsetlinewidth{0.501875pt}%
\definecolor{currentstroke}{rgb}{0.000000,0.000000,0.000000}%
\pgfsetstrokecolor{currentstroke}%
\pgfsetdash{}{0pt}%
\pgfsys@defobject{currentmarker}{\pgfqpoint{-0.069444in}{0.000000in}}{\pgfqpoint{0.000000in}{0.000000in}}{%
\pgfpathmoveto{\pgfqpoint{0.000000in}{0.000000in}}%
\pgfpathlineto{\pgfqpoint{-0.069444in}{0.000000in}}%
\pgfusepath{stroke,fill}%
}%
\begin{pgfscope}%
\pgfsys@transformshift{2.541794in}{0.558071in}%
\pgfsys@useobject{currentmarker}{}%
\end{pgfscope}%
\end{pgfscope}%
\begin{pgfscope}%
\pgftext[x=0.226704in,y=0.558071in,right,]{\rmfamily\fontsize{8.000000}{9.600000}\selectfont 0.5}%
\end{pgfscope}%
\begin{pgfscope}%
\pgfsetbuttcap%
\pgfsetroundjoin%
\definecolor{currentfill}{rgb}{0.000000,0.000000,0.000000}%
\pgfsetfillcolor{currentfill}%
\pgfsetlinewidth{0.501875pt}%
\definecolor{currentstroke}{rgb}{0.000000,0.000000,0.000000}%
\pgfsetstrokecolor{currentstroke}%
\pgfsetdash{}{0pt}%
\pgfsys@defobject{currentmarker}{\pgfqpoint{0.000000in}{0.000000in}}{\pgfqpoint{0.069444in}{0.000000in}}{%
\pgfpathmoveto{\pgfqpoint{0.000000in}{0.000000in}}%
\pgfpathlineto{\pgfqpoint{0.069444in}{0.000000in}}%
\pgfusepath{stroke,fill}%
}%
\begin{pgfscope}%
\pgfsys@transformshift{0.296148in}{0.698752in}%
\pgfsys@useobject{currentmarker}{}%
\end{pgfscope}%
\end{pgfscope}%
\begin{pgfscope}%
\pgfsetbuttcap%
\pgfsetroundjoin%
\definecolor{currentfill}{rgb}{0.000000,0.000000,0.000000}%
\pgfsetfillcolor{currentfill}%
\pgfsetlinewidth{0.501875pt}%
\definecolor{currentstroke}{rgb}{0.000000,0.000000,0.000000}%
\pgfsetstrokecolor{currentstroke}%
\pgfsetdash{}{0pt}%
\pgfsys@defobject{currentmarker}{\pgfqpoint{-0.069444in}{0.000000in}}{\pgfqpoint{0.000000in}{0.000000in}}{%
\pgfpathmoveto{\pgfqpoint{0.000000in}{0.000000in}}%
\pgfpathlineto{\pgfqpoint{-0.069444in}{0.000000in}}%
\pgfusepath{stroke,fill}%
}%
\begin{pgfscope}%
\pgfsys@transformshift{2.541794in}{0.698752in}%
\pgfsys@useobject{currentmarker}{}%
\end{pgfscope}%
\end{pgfscope}%
\begin{pgfscope}%
\pgftext[x=0.226704in,y=0.698752in,right,]{\rmfamily\fontsize{8.000000}{9.600000}\selectfont 1.0}%
\end{pgfscope}%
\begin{pgfscope}%
\pgfsetbuttcap%
\pgfsetroundjoin%
\definecolor{currentfill}{rgb}{0.000000,0.000000,0.000000}%
\pgfsetfillcolor{currentfill}%
\pgfsetlinewidth{0.501875pt}%
\definecolor{currentstroke}{rgb}{0.000000,0.000000,0.000000}%
\pgfsetstrokecolor{currentstroke}%
\pgfsetdash{}{0pt}%
\pgfsys@defobject{currentmarker}{\pgfqpoint{0.000000in}{0.000000in}}{\pgfqpoint{0.069444in}{0.000000in}}{%
\pgfpathmoveto{\pgfqpoint{0.000000in}{0.000000in}}%
\pgfpathlineto{\pgfqpoint{0.069444in}{0.000000in}}%
\pgfusepath{stroke,fill}%
}%
\begin{pgfscope}%
\pgfsys@transformshift{0.296148in}{0.839432in}%
\pgfsys@useobject{currentmarker}{}%
\end{pgfscope}%
\end{pgfscope}%
\begin{pgfscope}%
\pgfsetbuttcap%
\pgfsetroundjoin%
\definecolor{currentfill}{rgb}{0.000000,0.000000,0.000000}%
\pgfsetfillcolor{currentfill}%
\pgfsetlinewidth{0.501875pt}%
\definecolor{currentstroke}{rgb}{0.000000,0.000000,0.000000}%
\pgfsetstrokecolor{currentstroke}%
\pgfsetdash{}{0pt}%
\pgfsys@defobject{currentmarker}{\pgfqpoint{-0.069444in}{0.000000in}}{\pgfqpoint{0.000000in}{0.000000in}}{%
\pgfpathmoveto{\pgfqpoint{0.000000in}{0.000000in}}%
\pgfpathlineto{\pgfqpoint{-0.069444in}{0.000000in}}%
\pgfusepath{stroke,fill}%
}%
\begin{pgfscope}%
\pgfsys@transformshift{2.541794in}{0.839432in}%
\pgfsys@useobject{currentmarker}{}%
\end{pgfscope}%
\end{pgfscope}%
\begin{pgfscope}%
\pgftext[x=0.226704in,y=0.839432in,right,]{\rmfamily\fontsize{8.000000}{9.600000}\selectfont 1.5}%
\end{pgfscope}%
\begin{pgfscope}%
\pgfsetbuttcap%
\pgfsetroundjoin%
\definecolor{currentfill}{rgb}{0.000000,0.000000,0.000000}%
\pgfsetfillcolor{currentfill}%
\pgfsetlinewidth{0.501875pt}%
\definecolor{currentstroke}{rgb}{0.000000,0.000000,0.000000}%
\pgfsetstrokecolor{currentstroke}%
\pgfsetdash{}{0pt}%
\pgfsys@defobject{currentmarker}{\pgfqpoint{0.000000in}{0.000000in}}{\pgfqpoint{0.069444in}{0.000000in}}{%
\pgfpathmoveto{\pgfqpoint{0.000000in}{0.000000in}}%
\pgfpathlineto{\pgfqpoint{0.069444in}{0.000000in}}%
\pgfusepath{stroke,fill}%
}%
\begin{pgfscope}%
\pgfsys@transformshift{0.296148in}{0.980113in}%
\pgfsys@useobject{currentmarker}{}%
\end{pgfscope}%
\end{pgfscope}%
\begin{pgfscope}%
\pgfsetbuttcap%
\pgfsetroundjoin%
\definecolor{currentfill}{rgb}{0.000000,0.000000,0.000000}%
\pgfsetfillcolor{currentfill}%
\pgfsetlinewidth{0.501875pt}%
\definecolor{currentstroke}{rgb}{0.000000,0.000000,0.000000}%
\pgfsetstrokecolor{currentstroke}%
\pgfsetdash{}{0pt}%
\pgfsys@defobject{currentmarker}{\pgfqpoint{-0.069444in}{0.000000in}}{\pgfqpoint{0.000000in}{0.000000in}}{%
\pgfpathmoveto{\pgfqpoint{0.000000in}{0.000000in}}%
\pgfpathlineto{\pgfqpoint{-0.069444in}{0.000000in}}%
\pgfusepath{stroke,fill}%
}%
\begin{pgfscope}%
\pgfsys@transformshift{2.541794in}{0.980113in}%
\pgfsys@useobject{currentmarker}{}%
\end{pgfscope}%
\end{pgfscope}%
\begin{pgfscope}%
\pgftext[x=0.226704in,y=0.980113in,right,]{\rmfamily\fontsize{8.000000}{9.600000}\selectfont 2.0}%
\end{pgfscope}%
\begin{pgfscope}%
\pgfsetbuttcap%
\pgfsetroundjoin%
\definecolor{currentfill}{rgb}{0.000000,0.000000,0.000000}%
\pgfsetfillcolor{currentfill}%
\pgfsetlinewidth{0.501875pt}%
\definecolor{currentstroke}{rgb}{0.000000,0.000000,0.000000}%
\pgfsetstrokecolor{currentstroke}%
\pgfsetdash{}{0pt}%
\pgfsys@defobject{currentmarker}{\pgfqpoint{0.000000in}{0.000000in}}{\pgfqpoint{0.069444in}{0.000000in}}{%
\pgfpathmoveto{\pgfqpoint{0.000000in}{0.000000in}}%
\pgfpathlineto{\pgfqpoint{0.069444in}{0.000000in}}%
\pgfusepath{stroke,fill}%
}%
\begin{pgfscope}%
\pgfsys@transformshift{0.296148in}{1.120793in}%
\pgfsys@useobject{currentmarker}{}%
\end{pgfscope}%
\end{pgfscope}%
\begin{pgfscope}%
\pgfsetbuttcap%
\pgfsetroundjoin%
\definecolor{currentfill}{rgb}{0.000000,0.000000,0.000000}%
\pgfsetfillcolor{currentfill}%
\pgfsetlinewidth{0.501875pt}%
\definecolor{currentstroke}{rgb}{0.000000,0.000000,0.000000}%
\pgfsetstrokecolor{currentstroke}%
\pgfsetdash{}{0pt}%
\pgfsys@defobject{currentmarker}{\pgfqpoint{-0.069444in}{0.000000in}}{\pgfqpoint{0.000000in}{0.000000in}}{%
\pgfpathmoveto{\pgfqpoint{0.000000in}{0.000000in}}%
\pgfpathlineto{\pgfqpoint{-0.069444in}{0.000000in}}%
\pgfusepath{stroke,fill}%
}%
\begin{pgfscope}%
\pgfsys@transformshift{2.541794in}{1.120793in}%
\pgfsys@useobject{currentmarker}{}%
\end{pgfscope}%
\end{pgfscope}%
\begin{pgfscope}%
\pgftext[x=0.226704in,y=1.120793in,right,]{\rmfamily\fontsize{8.000000}{9.600000}\selectfont 2.5}%
\end{pgfscope}%
\begin{pgfscope}%
\pgfsetbuttcap%
\pgfsetroundjoin%
\definecolor{currentfill}{rgb}{0.000000,0.000000,0.000000}%
\pgfsetfillcolor{currentfill}%
\pgfsetlinewidth{0.501875pt}%
\definecolor{currentstroke}{rgb}{0.000000,0.000000,0.000000}%
\pgfsetstrokecolor{currentstroke}%
\pgfsetdash{}{0pt}%
\pgfsys@defobject{currentmarker}{\pgfqpoint{0.000000in}{0.000000in}}{\pgfqpoint{0.069444in}{0.000000in}}{%
\pgfpathmoveto{\pgfqpoint{0.000000in}{0.000000in}}%
\pgfpathlineto{\pgfqpoint{0.069444in}{0.000000in}}%
\pgfusepath{stroke,fill}%
}%
\begin{pgfscope}%
\pgfsys@transformshift{0.296148in}{1.261474in}%
\pgfsys@useobject{currentmarker}{}%
\end{pgfscope}%
\end{pgfscope}%
\begin{pgfscope}%
\pgfsetbuttcap%
\pgfsetroundjoin%
\definecolor{currentfill}{rgb}{0.000000,0.000000,0.000000}%
\pgfsetfillcolor{currentfill}%
\pgfsetlinewidth{0.501875pt}%
\definecolor{currentstroke}{rgb}{0.000000,0.000000,0.000000}%
\pgfsetstrokecolor{currentstroke}%
\pgfsetdash{}{0pt}%
\pgfsys@defobject{currentmarker}{\pgfqpoint{-0.069444in}{0.000000in}}{\pgfqpoint{0.000000in}{0.000000in}}{%
\pgfpathmoveto{\pgfqpoint{0.000000in}{0.000000in}}%
\pgfpathlineto{\pgfqpoint{-0.069444in}{0.000000in}}%
\pgfusepath{stroke,fill}%
}%
\begin{pgfscope}%
\pgfsys@transformshift{2.541794in}{1.261474in}%
\pgfsys@useobject{currentmarker}{}%
\end{pgfscope}%
\end{pgfscope}%
\begin{pgfscope}%
\pgftext[x=0.226704in,y=1.261474in,right,]{\rmfamily\fontsize{8.000000}{9.600000}\selectfont 3.0}%
\end{pgfscope}%
\begin{pgfscope}%
\pgfsetbuttcap%
\pgfsetroundjoin%
\definecolor{currentfill}{rgb}{0.000000,0.000000,0.000000}%
\pgfsetfillcolor{currentfill}%
\pgfsetlinewidth{0.501875pt}%
\definecolor{currentstroke}{rgb}{0.000000,0.000000,0.000000}%
\pgfsetstrokecolor{currentstroke}%
\pgfsetdash{}{0pt}%
\pgfsys@defobject{currentmarker}{\pgfqpoint{0.000000in}{0.000000in}}{\pgfqpoint{0.069444in}{0.000000in}}{%
\pgfpathmoveto{\pgfqpoint{0.000000in}{0.000000in}}%
\pgfpathlineto{\pgfqpoint{0.069444in}{0.000000in}}%
\pgfusepath{stroke,fill}%
}%
\begin{pgfscope}%
\pgfsys@transformshift{0.296148in}{1.402154in}%
\pgfsys@useobject{currentmarker}{}%
\end{pgfscope}%
\end{pgfscope}%
\begin{pgfscope}%
\pgfsetbuttcap%
\pgfsetroundjoin%
\definecolor{currentfill}{rgb}{0.000000,0.000000,0.000000}%
\pgfsetfillcolor{currentfill}%
\pgfsetlinewidth{0.501875pt}%
\definecolor{currentstroke}{rgb}{0.000000,0.000000,0.000000}%
\pgfsetstrokecolor{currentstroke}%
\pgfsetdash{}{0pt}%
\pgfsys@defobject{currentmarker}{\pgfqpoint{-0.069444in}{0.000000in}}{\pgfqpoint{0.000000in}{0.000000in}}{%
\pgfpathmoveto{\pgfqpoint{0.000000in}{0.000000in}}%
\pgfpathlineto{\pgfqpoint{-0.069444in}{0.000000in}}%
\pgfusepath{stroke,fill}%
}%
\begin{pgfscope}%
\pgfsys@transformshift{2.541794in}{1.402154in}%
\pgfsys@useobject{currentmarker}{}%
\end{pgfscope}%
\end{pgfscope}%
\begin{pgfscope}%
\pgftext[x=0.226704in,y=1.402154in,right,]{\rmfamily\fontsize{8.000000}{9.600000}\selectfont 3.5}%
\end{pgfscope}%
\begin{pgfscope}%
\pgfsetbuttcap%
\pgfsetroundjoin%
\definecolor{currentfill}{rgb}{0.000000,0.000000,0.000000}%
\pgfsetfillcolor{currentfill}%
\pgfsetlinewidth{0.501875pt}%
\definecolor{currentstroke}{rgb}{0.000000,0.000000,0.000000}%
\pgfsetstrokecolor{currentstroke}%
\pgfsetdash{}{0pt}%
\pgfsys@defobject{currentmarker}{\pgfqpoint{0.000000in}{0.000000in}}{\pgfqpoint{0.069444in}{0.000000in}}{%
\pgfpathmoveto{\pgfqpoint{0.000000in}{0.000000in}}%
\pgfpathlineto{\pgfqpoint{0.069444in}{0.000000in}}%
\pgfusepath{stroke,fill}%
}%
\begin{pgfscope}%
\pgfsys@transformshift{0.296148in}{1.542835in}%
\pgfsys@useobject{currentmarker}{}%
\end{pgfscope}%
\end{pgfscope}%
\begin{pgfscope}%
\pgfsetbuttcap%
\pgfsetroundjoin%
\definecolor{currentfill}{rgb}{0.000000,0.000000,0.000000}%
\pgfsetfillcolor{currentfill}%
\pgfsetlinewidth{0.501875pt}%
\definecolor{currentstroke}{rgb}{0.000000,0.000000,0.000000}%
\pgfsetstrokecolor{currentstroke}%
\pgfsetdash{}{0pt}%
\pgfsys@defobject{currentmarker}{\pgfqpoint{-0.069444in}{0.000000in}}{\pgfqpoint{0.000000in}{0.000000in}}{%
\pgfpathmoveto{\pgfqpoint{0.000000in}{0.000000in}}%
\pgfpathlineto{\pgfqpoint{-0.069444in}{0.000000in}}%
\pgfusepath{stroke,fill}%
}%
\begin{pgfscope}%
\pgfsys@transformshift{2.541794in}{1.542835in}%
\pgfsys@useobject{currentmarker}{}%
\end{pgfscope}%
\end{pgfscope}%
\begin{pgfscope}%
\pgftext[x=0.226704in,y=1.542835in,right,]{\rmfamily\fontsize{8.000000}{9.600000}\selectfont 4.0}%
\end{pgfscope}%
\begin{pgfscope}%
\pgfsetbuttcap%
\pgfsetroundjoin%
\definecolor{currentfill}{rgb}{0.000000,0.000000,0.000000}%
\pgfsetfillcolor{currentfill}%
\pgfsetlinewidth{0.501875pt}%
\definecolor{currentstroke}{rgb}{0.000000,0.000000,0.000000}%
\pgfsetstrokecolor{currentstroke}%
\pgfsetdash{}{0pt}%
\pgfsys@defobject{currentmarker}{\pgfqpoint{0.000000in}{0.000000in}}{\pgfqpoint{0.069444in}{0.000000in}}{%
\pgfpathmoveto{\pgfqpoint{0.000000in}{0.000000in}}%
\pgfpathlineto{\pgfqpoint{0.069444in}{0.000000in}}%
\pgfusepath{stroke,fill}%
}%
\begin{pgfscope}%
\pgfsys@transformshift{0.296148in}{1.683515in}%
\pgfsys@useobject{currentmarker}{}%
\end{pgfscope}%
\end{pgfscope}%
\begin{pgfscope}%
\pgfsetbuttcap%
\pgfsetroundjoin%
\definecolor{currentfill}{rgb}{0.000000,0.000000,0.000000}%
\pgfsetfillcolor{currentfill}%
\pgfsetlinewidth{0.501875pt}%
\definecolor{currentstroke}{rgb}{0.000000,0.000000,0.000000}%
\pgfsetstrokecolor{currentstroke}%
\pgfsetdash{}{0pt}%
\pgfsys@defobject{currentmarker}{\pgfqpoint{-0.069444in}{0.000000in}}{\pgfqpoint{0.000000in}{0.000000in}}{%
\pgfpathmoveto{\pgfqpoint{0.000000in}{0.000000in}}%
\pgfpathlineto{\pgfqpoint{-0.069444in}{0.000000in}}%
\pgfusepath{stroke,fill}%
}%
\begin{pgfscope}%
\pgfsys@transformshift{2.541794in}{1.683515in}%
\pgfsys@useobject{currentmarker}{}%
\end{pgfscope}%
\end{pgfscope}%
\begin{pgfscope}%
\pgftext[x=0.226704in,y=1.683515in,right,]{\rmfamily\fontsize{8.000000}{9.600000}\selectfont 4.5}%
\end{pgfscope}%
\end{pgfpicture}%
\makeatother%
\endgroup%

	\end{subfigure}
	\begin{subfigure}[t]{0.49\textwidth}
		\centering
    %\includegraphics[width=\textwidth]{store/variables/SIG_BKG_B_TAU.pdf}
    %% Creator: Matplotlib, PGF backend
%%
%% To include the figure in your LaTeX document, write
%%   \input{<filename>.pgf}
%%
%% Make sure the required packages are loaded in your preamble
%%   \usepackage{pgf}
%%
%% Figures using additional raster images can only be included by \input if
%% they are in the same directory as the main LaTeX file. For loading figures
%% from other directories you can use the `import` package
%%   \usepackage{import}
%% and then include the figures with
%%   \import{<path to file>}{<filename>.pgf}
%%
%% Matplotlib used the following preamble
%%   \usepackage{fontspec}
%%   \setmainfont{DejaVu Serif}
%%   \setsansfont{DejaVu Sans}
%%   \setmonofont{DejaVu Sans Mono}
%%
\begingroup%
\makeatletter%
\begin{pgfpicture}%
\pgfpathrectangle{\pgfpointorigin}{\pgfqpoint{2.681192in}{1.741309in}}%
\pgfusepath{use as bounding box, clip}%
\begin{pgfscope}%
\pgfsetbuttcap%
\pgfsetmiterjoin%
\definecolor{currentfill}{rgb}{1.000000,1.000000,1.000000}%
\pgfsetfillcolor{currentfill}%
\pgfsetlinewidth{0.000000pt}%
\definecolor{currentstroke}{rgb}{1.000000,1.000000,1.000000}%
\pgfsetstrokecolor{currentstroke}%
\pgfsetdash{}{0pt}%
\pgfpathmoveto{\pgfqpoint{0.000000in}{0.000000in}}%
\pgfpathlineto{\pgfqpoint{2.681192in}{0.000000in}}%
\pgfpathlineto{\pgfqpoint{2.681192in}{1.741309in}}%
\pgfpathlineto{\pgfqpoint{0.000000in}{1.741309in}}%
\pgfpathclose%
\pgfusepath{fill}%
\end{pgfscope}%
\begin{pgfscope}%
\pgfsetbuttcap%
\pgfsetmiterjoin%
\definecolor{currentfill}{rgb}{1.000000,1.000000,1.000000}%
\pgfsetfillcolor{currentfill}%
\pgfsetlinewidth{0.000000pt}%
\definecolor{currentstroke}{rgb}{0.000000,0.000000,0.000000}%
\pgfsetstrokecolor{currentstroke}%
\pgfsetstrokeopacity{0.000000}%
\pgfsetdash{}{0pt}%
\pgfpathmoveto{\pgfqpoint{0.296148in}{0.417391in}}%
\pgfpathlineto{\pgfqpoint{2.560500in}{0.417391in}}%
\pgfpathlineto{\pgfqpoint{2.560500in}{1.637544in}}%
\pgfpathlineto{\pgfqpoint{0.296148in}{1.637544in}}%
\pgfpathclose%
\pgfusepath{fill}%
\end{pgfscope}%
\begin{pgfscope}%
\pgfpathrectangle{\pgfqpoint{0.296148in}{0.417391in}}{\pgfqpoint{2.264352in}{1.220153in}} %
\pgfusepath{clip}%
\pgfsetbuttcap%
\pgfsetmiterjoin%
\definecolor{currentfill}{rgb}{0.215686,0.470588,0.749020}%
\pgfsetfillcolor{currentfill}%
\pgfsetlinewidth{0.000000pt}%
\definecolor{currentstroke}{rgb}{0.000000,0.000000,0.000000}%
\pgfsetstrokecolor{currentstroke}%
\pgfsetdash{}{0pt}%
\pgfpathmoveto{\pgfqpoint{0.296148in}{0.417391in}}%
\pgfpathlineto{\pgfqpoint{0.296148in}{0.417850in}}%
\pgfpathlineto{\pgfqpoint{0.341435in}{0.417850in}}%
\pgfpathlineto{\pgfqpoint{0.341435in}{0.508081in}}%
\pgfpathlineto{\pgfqpoint{0.386722in}{0.508081in}}%
\pgfpathlineto{\pgfqpoint{0.386722in}{0.750997in}}%
\pgfpathlineto{\pgfqpoint{0.432009in}{0.750997in}}%
\pgfpathlineto{\pgfqpoint{0.432009in}{0.932353in}}%
\pgfpathlineto{\pgfqpoint{0.477296in}{0.932353in}}%
\pgfpathlineto{\pgfqpoint{0.477296in}{1.019799in}}%
\pgfpathlineto{\pgfqpoint{0.522583in}{1.019799in}}%
\pgfpathlineto{\pgfqpoint{0.522583in}{0.951632in}}%
\pgfpathlineto{\pgfqpoint{0.567870in}{0.951632in}}%
\pgfpathlineto{\pgfqpoint{0.567870in}{0.944193in}}%
\pgfpathlineto{\pgfqpoint{0.613157in}{0.944193in}}%
\pgfpathlineto{\pgfqpoint{0.613157in}{0.871158in}}%
\pgfpathlineto{\pgfqpoint{0.658444in}{0.871158in}}%
\pgfpathlineto{\pgfqpoint{0.658444in}{0.845561in}}%
\pgfpathlineto{\pgfqpoint{0.703731in}{0.845561in}}%
\pgfpathlineto{\pgfqpoint{0.703731in}{0.805951in}}%
\pgfpathlineto{\pgfqpoint{0.749018in}{0.805951in}}%
\pgfpathlineto{\pgfqpoint{0.749018in}{0.755201in}}%
\pgfpathlineto{\pgfqpoint{0.794305in}{0.755201in}}%
\pgfpathlineto{\pgfqpoint{0.794305in}{0.752452in}}%
\pgfpathlineto{\pgfqpoint{0.839592in}{0.752452in}}%
\pgfpathlineto{\pgfqpoint{0.839592in}{0.681970in}}%
\pgfpathlineto{\pgfqpoint{0.884880in}{0.681970in}}%
\pgfpathlineto{\pgfqpoint{0.884880in}{0.666979in}}%
\pgfpathlineto{\pgfqpoint{0.930167in}{0.666979in}}%
\pgfpathlineto{\pgfqpoint{0.930167in}{0.643782in}}%
\pgfpathlineto{\pgfqpoint{0.975454in}{0.643782in}}%
\pgfpathlineto{\pgfqpoint{0.975454in}{0.599505in}}%
\pgfpathlineto{\pgfqpoint{1.020741in}{0.599505in}}%
\pgfpathlineto{\pgfqpoint{1.020741in}{0.594509in}}%
\pgfpathlineto{\pgfqpoint{1.066028in}{0.594509in}}%
\pgfpathlineto{\pgfqpoint{1.066028in}{0.554596in}}%
\pgfpathlineto{\pgfqpoint{1.111315in}{0.554596in}}%
\pgfpathlineto{\pgfqpoint{1.111315in}{0.533456in}}%
\pgfpathlineto{\pgfqpoint{1.156602in}{0.533456in}}%
\pgfpathlineto{\pgfqpoint{1.156602in}{0.540651in}}%
\pgfpathlineto{\pgfqpoint{1.201889in}{0.540651in}}%
\pgfpathlineto{\pgfqpoint{1.201889in}{0.517712in}}%
\pgfpathlineto{\pgfqpoint{1.247176in}{0.517712in}}%
\pgfpathlineto{\pgfqpoint{1.247176in}{0.496615in}}%
\pgfpathlineto{\pgfqpoint{1.292463in}{0.496615in}}%
\pgfpathlineto{\pgfqpoint{1.292463in}{0.487336in}}%
\pgfpathlineto{\pgfqpoint{1.337750in}{0.487336in}}%
\pgfpathlineto{\pgfqpoint{1.337750in}{0.476898in}}%
\pgfpathlineto{\pgfqpoint{1.383037in}{0.476898in}}%
\pgfpathlineto{\pgfqpoint{1.383037in}{0.473921in}}%
\pgfpathlineto{\pgfqpoint{1.428324in}{0.473921in}}%
\pgfpathlineto{\pgfqpoint{1.428324in}{0.473352in}}%
\pgfpathlineto{\pgfqpoint{1.473611in}{0.473352in}}%
\pgfpathlineto{\pgfqpoint{1.473611in}{0.456105in}}%
\pgfpathlineto{\pgfqpoint{1.518898in}{0.456105in}}%
\pgfpathlineto{\pgfqpoint{1.518898in}{0.457368in}}%
\pgfpathlineto{\pgfqpoint{1.564185in}{0.457368in}}%
\pgfpathlineto{\pgfqpoint{1.564185in}{0.445392in}}%
\pgfpathlineto{\pgfqpoint{1.609472in}{0.445392in}}%
\pgfpathlineto{\pgfqpoint{1.609472in}{0.446079in}}%
\pgfpathlineto{\pgfqpoint{1.654759in}{0.446079in}}%
\pgfpathlineto{\pgfqpoint{1.654759in}{0.448259in}}%
\pgfpathlineto{\pgfqpoint{1.700046in}{0.448259in}}%
\pgfpathlineto{\pgfqpoint{1.700046in}{0.436724in}}%
\pgfpathlineto{\pgfqpoint{1.745333in}{0.436724in}}%
\pgfpathlineto{\pgfqpoint{1.745333in}{0.441248in}}%
\pgfpathlineto{\pgfqpoint{1.790620in}{0.441248in}}%
\pgfpathlineto{\pgfqpoint{1.790620in}{0.439532in}}%
\pgfpathlineto{\pgfqpoint{1.835907in}{0.439532in}}%
\pgfpathlineto{\pgfqpoint{1.835907in}{0.433670in}}%
\pgfpathlineto{\pgfqpoint{1.881194in}{0.433670in}}%
\pgfpathlineto{\pgfqpoint{1.881194in}{0.427053in}}%
\pgfpathlineto{\pgfqpoint{1.926481in}{0.427053in}}%
\pgfpathlineto{\pgfqpoint{1.926481in}{0.428663in}}%
\pgfpathlineto{\pgfqpoint{1.971768in}{0.428663in}}%
\pgfpathlineto{\pgfqpoint{1.971768in}{0.427664in}}%
\pgfpathlineto{\pgfqpoint{2.017055in}{0.427664in}}%
\pgfpathlineto{\pgfqpoint{2.017055in}{0.427797in}}%
\pgfpathlineto{\pgfqpoint{2.062342in}{0.427797in}}%
\pgfpathlineto{\pgfqpoint{2.062342in}{0.430482in}}%
\pgfpathlineto{\pgfqpoint{2.107630in}{0.430482in}}%
\pgfpathlineto{\pgfqpoint{2.107630in}{0.422161in}}%
\pgfpathlineto{\pgfqpoint{2.152917in}{0.422161in}}%
\pgfpathlineto{\pgfqpoint{2.152917in}{0.422547in}}%
\pgfpathlineto{\pgfqpoint{2.198204in}{0.422547in}}%
\pgfpathlineto{\pgfqpoint{2.198204in}{0.420452in}}%
\pgfpathlineto{\pgfqpoint{2.243491in}{0.420452in}}%
\pgfpathlineto{\pgfqpoint{2.243491in}{0.421119in}}%
\pgfpathlineto{\pgfqpoint{2.288778in}{0.421119in}}%
\pgfpathlineto{\pgfqpoint{2.288778in}{0.418872in}}%
\pgfpathlineto{\pgfqpoint{2.334065in}{0.418872in}}%
\pgfpathlineto{\pgfqpoint{2.334065in}{0.421062in}}%
\pgfpathlineto{\pgfqpoint{2.379352in}{0.421062in}}%
\pgfpathlineto{\pgfqpoint{2.379352in}{0.421225in}}%
\pgfpathlineto{\pgfqpoint{2.424639in}{0.421225in}}%
\pgfpathlineto{\pgfqpoint{2.424639in}{0.418698in}}%
\pgfpathlineto{\pgfqpoint{2.469926in}{0.418698in}}%
\pgfpathlineto{\pgfqpoint{2.469926in}{0.420042in}}%
\pgfpathlineto{\pgfqpoint{2.515213in}{0.420042in}}%
\pgfpathlineto{\pgfqpoint{2.515213in}{0.419452in}}%
\pgfpathlineto{\pgfqpoint{2.560500in}{0.419452in}}%
\pgfpathlineto{\pgfqpoint{2.560500in}{0.417391in}}%
\pgfpathlineto{\pgfqpoint{2.515213in}{0.417391in}}%
\pgfpathlineto{\pgfqpoint{2.515213in}{0.417391in}}%
\pgfpathlineto{\pgfqpoint{2.469926in}{0.417391in}}%
\pgfpathlineto{\pgfqpoint{2.469926in}{0.417391in}}%
\pgfpathlineto{\pgfqpoint{2.424639in}{0.417391in}}%
\pgfpathlineto{\pgfqpoint{2.424639in}{0.417391in}}%
\pgfpathlineto{\pgfqpoint{2.379352in}{0.417391in}}%
\pgfpathlineto{\pgfqpoint{2.379352in}{0.417391in}}%
\pgfpathlineto{\pgfqpoint{2.334065in}{0.417391in}}%
\pgfpathlineto{\pgfqpoint{2.334065in}{0.417391in}}%
\pgfpathlineto{\pgfqpoint{2.288778in}{0.417391in}}%
\pgfpathlineto{\pgfqpoint{2.288778in}{0.417391in}}%
\pgfpathlineto{\pgfqpoint{2.243491in}{0.417391in}}%
\pgfpathlineto{\pgfqpoint{2.243491in}{0.417391in}}%
\pgfpathlineto{\pgfqpoint{2.198204in}{0.417391in}}%
\pgfpathlineto{\pgfqpoint{2.198204in}{0.417391in}}%
\pgfpathlineto{\pgfqpoint{2.152917in}{0.417391in}}%
\pgfpathlineto{\pgfqpoint{2.152917in}{0.417391in}}%
\pgfpathlineto{\pgfqpoint{2.107630in}{0.417391in}}%
\pgfpathlineto{\pgfqpoint{2.107630in}{0.417391in}}%
\pgfpathlineto{\pgfqpoint{2.062342in}{0.417391in}}%
\pgfpathlineto{\pgfqpoint{2.062342in}{0.417391in}}%
\pgfpathlineto{\pgfqpoint{2.017055in}{0.417391in}}%
\pgfpathlineto{\pgfqpoint{2.017055in}{0.417391in}}%
\pgfpathlineto{\pgfqpoint{1.971768in}{0.417391in}}%
\pgfpathlineto{\pgfqpoint{1.971768in}{0.417391in}}%
\pgfpathlineto{\pgfqpoint{1.926481in}{0.417391in}}%
\pgfpathlineto{\pgfqpoint{1.926481in}{0.417391in}}%
\pgfpathlineto{\pgfqpoint{1.881194in}{0.417391in}}%
\pgfpathlineto{\pgfqpoint{1.881194in}{0.417391in}}%
\pgfpathlineto{\pgfqpoint{1.835907in}{0.417391in}}%
\pgfpathlineto{\pgfqpoint{1.835907in}{0.417391in}}%
\pgfpathlineto{\pgfqpoint{1.790620in}{0.417391in}}%
\pgfpathlineto{\pgfqpoint{1.790620in}{0.417391in}}%
\pgfpathlineto{\pgfqpoint{1.745333in}{0.417391in}}%
\pgfpathlineto{\pgfqpoint{1.745333in}{0.417391in}}%
\pgfpathlineto{\pgfqpoint{1.700046in}{0.417391in}}%
\pgfpathlineto{\pgfqpoint{1.700046in}{0.417391in}}%
\pgfpathlineto{\pgfqpoint{1.654759in}{0.417391in}}%
\pgfpathlineto{\pgfqpoint{1.654759in}{0.417391in}}%
\pgfpathlineto{\pgfqpoint{1.609472in}{0.417391in}}%
\pgfpathlineto{\pgfqpoint{1.609472in}{0.417391in}}%
\pgfpathlineto{\pgfqpoint{1.564185in}{0.417391in}}%
\pgfpathlineto{\pgfqpoint{1.564185in}{0.417391in}}%
\pgfpathlineto{\pgfqpoint{1.518898in}{0.417391in}}%
\pgfpathlineto{\pgfqpoint{1.518898in}{0.417391in}}%
\pgfpathlineto{\pgfqpoint{1.473611in}{0.417391in}}%
\pgfpathlineto{\pgfqpoint{1.473611in}{0.417391in}}%
\pgfpathlineto{\pgfqpoint{1.428324in}{0.417391in}}%
\pgfpathlineto{\pgfqpoint{1.428324in}{0.417391in}}%
\pgfpathlineto{\pgfqpoint{1.383037in}{0.417391in}}%
\pgfpathlineto{\pgfqpoint{1.383037in}{0.417391in}}%
\pgfpathlineto{\pgfqpoint{1.337750in}{0.417391in}}%
\pgfpathlineto{\pgfqpoint{1.337750in}{0.417391in}}%
\pgfpathlineto{\pgfqpoint{1.292463in}{0.417391in}}%
\pgfpathlineto{\pgfqpoint{1.292463in}{0.417391in}}%
\pgfpathlineto{\pgfqpoint{1.247176in}{0.417391in}}%
\pgfpathlineto{\pgfqpoint{1.247176in}{0.417391in}}%
\pgfpathlineto{\pgfqpoint{1.201889in}{0.417391in}}%
\pgfpathlineto{\pgfqpoint{1.201889in}{0.417391in}}%
\pgfpathlineto{\pgfqpoint{1.156602in}{0.417391in}}%
\pgfpathlineto{\pgfqpoint{1.156602in}{0.417391in}}%
\pgfpathlineto{\pgfqpoint{1.111315in}{0.417391in}}%
\pgfpathlineto{\pgfqpoint{1.111315in}{0.417391in}}%
\pgfpathlineto{\pgfqpoint{1.066028in}{0.417391in}}%
\pgfpathlineto{\pgfqpoint{1.066028in}{0.417391in}}%
\pgfpathlineto{\pgfqpoint{1.020741in}{0.417391in}}%
\pgfpathlineto{\pgfqpoint{1.020741in}{0.417391in}}%
\pgfpathlineto{\pgfqpoint{0.975454in}{0.417391in}}%
\pgfpathlineto{\pgfqpoint{0.975454in}{0.417391in}}%
\pgfpathlineto{\pgfqpoint{0.930167in}{0.417391in}}%
\pgfpathlineto{\pgfqpoint{0.930167in}{0.417391in}}%
\pgfpathlineto{\pgfqpoint{0.884880in}{0.417391in}}%
\pgfpathlineto{\pgfqpoint{0.884880in}{0.417391in}}%
\pgfpathlineto{\pgfqpoint{0.839592in}{0.417391in}}%
\pgfpathlineto{\pgfqpoint{0.839592in}{0.417391in}}%
\pgfpathlineto{\pgfqpoint{0.794305in}{0.417391in}}%
\pgfpathlineto{\pgfqpoint{0.794305in}{0.417391in}}%
\pgfpathlineto{\pgfqpoint{0.749018in}{0.417391in}}%
\pgfpathlineto{\pgfqpoint{0.749018in}{0.417391in}}%
\pgfpathlineto{\pgfqpoint{0.703731in}{0.417391in}}%
\pgfpathlineto{\pgfqpoint{0.703731in}{0.417391in}}%
\pgfpathlineto{\pgfqpoint{0.658444in}{0.417391in}}%
\pgfpathlineto{\pgfqpoint{0.658444in}{0.417391in}}%
\pgfpathlineto{\pgfqpoint{0.613157in}{0.417391in}}%
\pgfpathlineto{\pgfqpoint{0.613157in}{0.417391in}}%
\pgfpathlineto{\pgfqpoint{0.567870in}{0.417391in}}%
\pgfpathlineto{\pgfqpoint{0.567870in}{0.417391in}}%
\pgfpathlineto{\pgfqpoint{0.522583in}{0.417391in}}%
\pgfpathlineto{\pgfqpoint{0.522583in}{0.417391in}}%
\pgfpathlineto{\pgfqpoint{0.477296in}{0.417391in}}%
\pgfpathlineto{\pgfqpoint{0.477296in}{0.417391in}}%
\pgfpathlineto{\pgfqpoint{0.432009in}{0.417391in}}%
\pgfpathlineto{\pgfqpoint{0.432009in}{0.417391in}}%
\pgfpathlineto{\pgfqpoint{0.386722in}{0.417391in}}%
\pgfpathlineto{\pgfqpoint{0.386722in}{0.417391in}}%
\pgfpathlineto{\pgfqpoint{0.341435in}{0.417391in}}%
\pgfpathlineto{\pgfqpoint{0.341435in}{0.417391in}}%
\pgfpathlineto{\pgfqpoint{0.296148in}{0.417391in}}%
\pgfusepath{fill}%
\end{pgfscope}%
\begin{pgfscope}%
\pgfpathrectangle{\pgfqpoint{0.296148in}{0.417391in}}{\pgfqpoint{2.264352in}{1.220153in}} %
\pgfusepath{clip}%
\pgfsetbuttcap%
\pgfsetmiterjoin%
\pgfsetlinewidth{0.501875pt}%
\definecolor{currentstroke}{rgb}{1.000000,0.000000,0.000000}%
\pgfsetstrokecolor{currentstroke}%
\pgfsetdash{}{0pt}%
\pgfpathmoveto{\pgfqpoint{0.296148in}{0.417391in}}%
\pgfpathlineto{\pgfqpoint{0.296148in}{0.418366in}}%
\pgfpathlineto{\pgfqpoint{0.341435in}{0.418366in}}%
\pgfpathlineto{\pgfqpoint{0.341435in}{0.512442in}}%
\pgfpathlineto{\pgfqpoint{0.386722in}{0.512442in}}%
\pgfpathlineto{\pgfqpoint{0.386722in}{1.102706in}}%
\pgfpathlineto{\pgfqpoint{0.432009in}{1.102706in}}%
\pgfpathlineto{\pgfqpoint{0.432009in}{1.605829in}}%
\pgfpathlineto{\pgfqpoint{0.477296in}{1.605829in}}%
\pgfpathlineto{\pgfqpoint{0.477296in}{1.579032in}}%
\pgfpathlineto{\pgfqpoint{0.522583in}{1.579032in}}%
\pgfpathlineto{\pgfqpoint{0.522583in}{1.345242in}}%
\pgfpathlineto{\pgfqpoint{0.567870in}{1.345242in}}%
\pgfpathlineto{\pgfqpoint{0.567870in}{1.113078in}}%
\pgfpathlineto{\pgfqpoint{0.613157in}{1.113078in}}%
\pgfpathlineto{\pgfqpoint{0.613157in}{0.918006in}}%
\pgfpathlineto{\pgfqpoint{0.658444in}{0.918006in}}%
\pgfpathlineto{\pgfqpoint{0.658444in}{0.780444in}}%
\pgfpathlineto{\pgfqpoint{0.703731in}{0.780444in}}%
\pgfpathlineto{\pgfqpoint{0.703731in}{0.679526in}}%
\pgfpathlineto{\pgfqpoint{0.749018in}{0.679526in}}%
\pgfpathlineto{\pgfqpoint{0.749018in}{0.612014in}}%
\pgfpathlineto{\pgfqpoint{0.794305in}{0.612014in}}%
\pgfpathlineto{\pgfqpoint{0.794305in}{0.563126in}}%
\pgfpathlineto{\pgfqpoint{0.839592in}{0.563126in}}%
\pgfpathlineto{\pgfqpoint{0.839592in}{0.527614in}}%
\pgfpathlineto{\pgfqpoint{0.884880in}{0.527614in}}%
\pgfpathlineto{\pgfqpoint{0.884880in}{0.500027in}}%
\pgfpathlineto{\pgfqpoint{0.930167in}{0.500027in}}%
\pgfpathlineto{\pgfqpoint{0.930167in}{0.480212in}}%
\pgfpathlineto{\pgfqpoint{0.975454in}{0.480212in}}%
\pgfpathlineto{\pgfqpoint{0.975454in}{0.469205in}}%
\pgfpathlineto{\pgfqpoint{1.020741in}{0.469205in}}%
\pgfpathlineto{\pgfqpoint{1.020741in}{0.457006in}}%
\pgfpathlineto{\pgfqpoint{1.066028in}{0.457006in}}%
\pgfpathlineto{\pgfqpoint{1.066028in}{0.448213in}}%
\pgfpathlineto{\pgfqpoint{1.111315in}{0.448213in}}%
\pgfpathlineto{\pgfqpoint{1.111315in}{0.443708in}}%
\pgfpathlineto{\pgfqpoint{1.156602in}{0.443708in}}%
\pgfpathlineto{\pgfqpoint{1.156602in}{0.438708in}}%
\pgfpathlineto{\pgfqpoint{1.201889in}{0.438708in}}%
\pgfpathlineto{\pgfqpoint{1.201889in}{0.435395in}}%
\pgfpathlineto{\pgfqpoint{1.247176in}{0.435395in}}%
\pgfpathlineto{\pgfqpoint{1.247176in}{0.432887in}}%
\pgfpathlineto{\pgfqpoint{1.292463in}{0.432887in}}%
\pgfpathlineto{\pgfqpoint{1.292463in}{0.429373in}}%
\pgfpathlineto{\pgfqpoint{1.337750in}{0.429373in}}%
\pgfpathlineto{\pgfqpoint{1.337750in}{0.427933in}}%
\pgfpathlineto{\pgfqpoint{1.383037in}{0.427933in}}%
\pgfpathlineto{\pgfqpoint{1.383037in}{0.426354in}}%
\pgfpathlineto{\pgfqpoint{1.428324in}{0.426354in}}%
\pgfpathlineto{\pgfqpoint{1.428324in}{0.424589in}}%
\pgfpathlineto{\pgfqpoint{1.473611in}{0.424589in}}%
\pgfpathlineto{\pgfqpoint{1.473611in}{0.424559in}}%
\pgfpathlineto{\pgfqpoint{1.518898in}{0.424559in}}%
\pgfpathlineto{\pgfqpoint{1.518898in}{0.422825in}}%
\pgfpathlineto{\pgfqpoint{1.564185in}{0.422825in}}%
\pgfpathlineto{\pgfqpoint{1.564185in}{0.422422in}}%
\pgfpathlineto{\pgfqpoint{1.609472in}{0.422422in}}%
\pgfpathlineto{\pgfqpoint{1.609472in}{0.421849in}}%
\pgfpathlineto{\pgfqpoint{1.654759in}{0.421849in}}%
\pgfpathlineto{\pgfqpoint{1.654759in}{0.421308in}}%
\pgfpathlineto{\pgfqpoint{1.700046in}{0.421308in}}%
\pgfpathlineto{\pgfqpoint{1.700046in}{0.420998in}}%
\pgfpathlineto{\pgfqpoint{1.745333in}{0.420998in}}%
\pgfpathlineto{\pgfqpoint{1.745333in}{0.420673in}}%
\pgfpathlineto{\pgfqpoint{1.790620in}{0.420673in}}%
\pgfpathlineto{\pgfqpoint{1.790620in}{0.420147in}}%
\pgfpathlineto{\pgfqpoint{1.835907in}{0.420147in}}%
\pgfpathlineto{\pgfqpoint{1.835907in}{0.420147in}}%
\pgfpathlineto{\pgfqpoint{1.881194in}{0.420147in}}%
\pgfpathlineto{\pgfqpoint{1.881194in}{0.419574in}}%
\pgfpathlineto{\pgfqpoint{1.926481in}{0.419574in}}%
\pgfpathlineto{\pgfqpoint{1.926481in}{0.419589in}}%
\pgfpathlineto{\pgfqpoint{1.971768in}{0.419589in}}%
\pgfpathlineto{\pgfqpoint{1.971768in}{0.419620in}}%
\pgfpathlineto{\pgfqpoint{2.017055in}{0.419620in}}%
\pgfpathlineto{\pgfqpoint{2.017055in}{0.419388in}}%
\pgfpathlineto{\pgfqpoint{2.062342in}{0.419388in}}%
\pgfpathlineto{\pgfqpoint{2.062342in}{0.419063in}}%
\pgfpathlineto{\pgfqpoint{2.107630in}{0.419063in}}%
\pgfpathlineto{\pgfqpoint{2.107630in}{0.418815in}}%
\pgfpathlineto{\pgfqpoint{2.152917in}{0.418815in}}%
\pgfpathlineto{\pgfqpoint{2.152917in}{0.418924in}}%
\pgfpathlineto{\pgfqpoint{2.198204in}{0.418924in}}%
\pgfpathlineto{\pgfqpoint{2.198204in}{0.418614in}}%
\pgfpathlineto{\pgfqpoint{2.243491in}{0.418614in}}%
\pgfpathlineto{\pgfqpoint{2.243491in}{0.418660in}}%
\pgfpathlineto{\pgfqpoint{2.288778in}{0.418660in}}%
\pgfpathlineto{\pgfqpoint{2.288778in}{0.418645in}}%
\pgfpathlineto{\pgfqpoint{2.334065in}{0.418645in}}%
\pgfpathlineto{\pgfqpoint{2.334065in}{0.418304in}}%
\pgfpathlineto{\pgfqpoint{2.379352in}{0.418304in}}%
\pgfpathlineto{\pgfqpoint{2.379352in}{0.418444in}}%
\pgfpathlineto{\pgfqpoint{2.424639in}{0.418444in}}%
\pgfpathlineto{\pgfqpoint{2.424639in}{0.418382in}}%
\pgfpathlineto{\pgfqpoint{2.469926in}{0.418382in}}%
\pgfpathlineto{\pgfqpoint{2.469926in}{0.418258in}}%
\pgfpathlineto{\pgfqpoint{2.515213in}{0.418258in}}%
\pgfpathlineto{\pgfqpoint{2.515213in}{0.417933in}}%
\pgfpathlineto{\pgfqpoint{2.560500in}{0.417933in}}%
\pgfpathlineto{\pgfqpoint{2.560500in}{0.417391in}}%
\pgfusepath{stroke}%
\end{pgfscope}%
\begin{pgfscope}%
\pgfsetrectcap%
\pgfsetmiterjoin%
\pgfsetlinewidth{1.003750pt}%
\definecolor{currentstroke}{rgb}{0.000000,0.000000,0.000000}%
\pgfsetstrokecolor{currentstroke}%
\pgfsetdash{}{0pt}%
\pgfpathmoveto{\pgfqpoint{0.296148in}{1.637544in}}%
\pgfpathlineto{\pgfqpoint{2.560500in}{1.637544in}}%
\pgfusepath{stroke}%
\end{pgfscope}%
\begin{pgfscope}%
\pgfsetrectcap%
\pgfsetmiterjoin%
\pgfsetlinewidth{1.003750pt}%
\definecolor{currentstroke}{rgb}{0.000000,0.000000,0.000000}%
\pgfsetstrokecolor{currentstroke}%
\pgfsetdash{}{0pt}%
\pgfpathmoveto{\pgfqpoint{2.560500in}{0.417391in}}%
\pgfpathlineto{\pgfqpoint{2.560500in}{1.637544in}}%
\pgfusepath{stroke}%
\end{pgfscope}%
\begin{pgfscope}%
\pgfsetrectcap%
\pgfsetmiterjoin%
\pgfsetlinewidth{1.003750pt}%
\definecolor{currentstroke}{rgb}{0.000000,0.000000,0.000000}%
\pgfsetstrokecolor{currentstroke}%
\pgfsetdash{}{0pt}%
\pgfpathmoveto{\pgfqpoint{0.296148in}{0.417391in}}%
\pgfpathlineto{\pgfqpoint{2.560500in}{0.417391in}}%
\pgfusepath{stroke}%
\end{pgfscope}%
\begin{pgfscope}%
\pgfsetrectcap%
\pgfsetmiterjoin%
\pgfsetlinewidth{1.003750pt}%
\definecolor{currentstroke}{rgb}{0.000000,0.000000,0.000000}%
\pgfsetstrokecolor{currentstroke}%
\pgfsetdash{}{0pt}%
\pgfpathmoveto{\pgfqpoint{0.296148in}{0.417391in}}%
\pgfpathlineto{\pgfqpoint{0.296148in}{1.637544in}}%
\pgfusepath{stroke}%
\end{pgfscope}%
\begin{pgfscope}%
\pgfsetbuttcap%
\pgfsetroundjoin%
\definecolor{currentfill}{rgb}{0.000000,0.000000,0.000000}%
\pgfsetfillcolor{currentfill}%
\pgfsetlinewidth{0.501875pt}%
\definecolor{currentstroke}{rgb}{0.000000,0.000000,0.000000}%
\pgfsetstrokecolor{currentstroke}%
\pgfsetdash{}{0pt}%
\pgfsys@defobject{currentmarker}{\pgfqpoint{0.000000in}{0.000000in}}{\pgfqpoint{0.000000in}{0.069444in}}{%
\pgfpathmoveto{\pgfqpoint{0.000000in}{0.000000in}}%
\pgfpathlineto{\pgfqpoint{0.000000in}{0.069444in}}%
\pgfusepath{stroke,fill}%
}%
\begin{pgfscope}%
\pgfsys@transformshift{0.296148in}{0.417391in}%
\pgfsys@useobject{currentmarker}{}%
\end{pgfscope}%
\end{pgfscope}%
\begin{pgfscope}%
\pgfsetbuttcap%
\pgfsetroundjoin%
\definecolor{currentfill}{rgb}{0.000000,0.000000,0.000000}%
\pgfsetfillcolor{currentfill}%
\pgfsetlinewidth{0.501875pt}%
\definecolor{currentstroke}{rgb}{0.000000,0.000000,0.000000}%
\pgfsetstrokecolor{currentstroke}%
\pgfsetdash{}{0pt}%
\pgfsys@defobject{currentmarker}{\pgfqpoint{0.000000in}{-0.069444in}}{\pgfqpoint{0.000000in}{0.000000in}}{%
\pgfpathmoveto{\pgfqpoint{0.000000in}{0.000000in}}%
\pgfpathlineto{\pgfqpoint{0.000000in}{-0.069444in}}%
\pgfusepath{stroke,fill}%
}%
\begin{pgfscope}%
\pgfsys@transformshift{0.296148in}{1.637544in}%
\pgfsys@useobject{currentmarker}{}%
\end{pgfscope}%
\end{pgfscope}%
\begin{pgfscope}%
\pgftext[x=0.296148in,y=0.347947in,,top]{\rmfamily\fontsize{8.000000}{9.600000}\selectfont 0}%
\end{pgfscope}%
\begin{pgfscope}%
\pgfsetbuttcap%
\pgfsetroundjoin%
\definecolor{currentfill}{rgb}{0.000000,0.000000,0.000000}%
\pgfsetfillcolor{currentfill}%
\pgfsetlinewidth{0.501875pt}%
\definecolor{currentstroke}{rgb}{0.000000,0.000000,0.000000}%
\pgfsetstrokecolor{currentstroke}%
\pgfsetdash{}{0pt}%
\pgfsys@defobject{currentmarker}{\pgfqpoint{0.000000in}{0.000000in}}{\pgfqpoint{0.000000in}{0.069444in}}{%
\pgfpathmoveto{\pgfqpoint{0.000000in}{0.000000in}}%
\pgfpathlineto{\pgfqpoint{0.000000in}{0.069444in}}%
\pgfusepath{stroke,fill}%
}%
\begin{pgfscope}%
\pgfsys@transformshift{0.749018in}{0.417391in}%
\pgfsys@useobject{currentmarker}{}%
\end{pgfscope}%
\end{pgfscope}%
\begin{pgfscope}%
\pgfsetbuttcap%
\pgfsetroundjoin%
\definecolor{currentfill}{rgb}{0.000000,0.000000,0.000000}%
\pgfsetfillcolor{currentfill}%
\pgfsetlinewidth{0.501875pt}%
\definecolor{currentstroke}{rgb}{0.000000,0.000000,0.000000}%
\pgfsetstrokecolor{currentstroke}%
\pgfsetdash{}{0pt}%
\pgfsys@defobject{currentmarker}{\pgfqpoint{0.000000in}{-0.069444in}}{\pgfqpoint{0.000000in}{0.000000in}}{%
\pgfpathmoveto{\pgfqpoint{0.000000in}{0.000000in}}%
\pgfpathlineto{\pgfqpoint{0.000000in}{-0.069444in}}%
\pgfusepath{stroke,fill}%
}%
\begin{pgfscope}%
\pgfsys@transformshift{0.749018in}{1.637544in}%
\pgfsys@useobject{currentmarker}{}%
\end{pgfscope}%
\end{pgfscope}%
\begin{pgfscope}%
\pgftext[x=0.749018in,y=0.347947in,,top]{\rmfamily\fontsize{8.000000}{9.600000}\selectfont 2}%
\end{pgfscope}%
\begin{pgfscope}%
\pgfsetbuttcap%
\pgfsetroundjoin%
\definecolor{currentfill}{rgb}{0.000000,0.000000,0.000000}%
\pgfsetfillcolor{currentfill}%
\pgfsetlinewidth{0.501875pt}%
\definecolor{currentstroke}{rgb}{0.000000,0.000000,0.000000}%
\pgfsetstrokecolor{currentstroke}%
\pgfsetdash{}{0pt}%
\pgfsys@defobject{currentmarker}{\pgfqpoint{0.000000in}{0.000000in}}{\pgfqpoint{0.000000in}{0.069444in}}{%
\pgfpathmoveto{\pgfqpoint{0.000000in}{0.000000in}}%
\pgfpathlineto{\pgfqpoint{0.000000in}{0.069444in}}%
\pgfusepath{stroke,fill}%
}%
\begin{pgfscope}%
\pgfsys@transformshift{1.201889in}{0.417391in}%
\pgfsys@useobject{currentmarker}{}%
\end{pgfscope}%
\end{pgfscope}%
\begin{pgfscope}%
\pgfsetbuttcap%
\pgfsetroundjoin%
\definecolor{currentfill}{rgb}{0.000000,0.000000,0.000000}%
\pgfsetfillcolor{currentfill}%
\pgfsetlinewidth{0.501875pt}%
\definecolor{currentstroke}{rgb}{0.000000,0.000000,0.000000}%
\pgfsetstrokecolor{currentstroke}%
\pgfsetdash{}{0pt}%
\pgfsys@defobject{currentmarker}{\pgfqpoint{0.000000in}{-0.069444in}}{\pgfqpoint{0.000000in}{0.000000in}}{%
\pgfpathmoveto{\pgfqpoint{0.000000in}{0.000000in}}%
\pgfpathlineto{\pgfqpoint{0.000000in}{-0.069444in}}%
\pgfusepath{stroke,fill}%
}%
\begin{pgfscope}%
\pgfsys@transformshift{1.201889in}{1.637544in}%
\pgfsys@useobject{currentmarker}{}%
\end{pgfscope}%
\end{pgfscope}%
\begin{pgfscope}%
\pgftext[x=1.201889in,y=0.347947in,,top]{\rmfamily\fontsize{8.000000}{9.600000}\selectfont 4}%
\end{pgfscope}%
\begin{pgfscope}%
\pgfsetbuttcap%
\pgfsetroundjoin%
\definecolor{currentfill}{rgb}{0.000000,0.000000,0.000000}%
\pgfsetfillcolor{currentfill}%
\pgfsetlinewidth{0.501875pt}%
\definecolor{currentstroke}{rgb}{0.000000,0.000000,0.000000}%
\pgfsetstrokecolor{currentstroke}%
\pgfsetdash{}{0pt}%
\pgfsys@defobject{currentmarker}{\pgfqpoint{0.000000in}{0.000000in}}{\pgfqpoint{0.000000in}{0.069444in}}{%
\pgfpathmoveto{\pgfqpoint{0.000000in}{0.000000in}}%
\pgfpathlineto{\pgfqpoint{0.000000in}{0.069444in}}%
\pgfusepath{stroke,fill}%
}%
\begin{pgfscope}%
\pgfsys@transformshift{1.654759in}{0.417391in}%
\pgfsys@useobject{currentmarker}{}%
\end{pgfscope}%
\end{pgfscope}%
\begin{pgfscope}%
\pgfsetbuttcap%
\pgfsetroundjoin%
\definecolor{currentfill}{rgb}{0.000000,0.000000,0.000000}%
\pgfsetfillcolor{currentfill}%
\pgfsetlinewidth{0.501875pt}%
\definecolor{currentstroke}{rgb}{0.000000,0.000000,0.000000}%
\pgfsetstrokecolor{currentstroke}%
\pgfsetdash{}{0pt}%
\pgfsys@defobject{currentmarker}{\pgfqpoint{0.000000in}{-0.069444in}}{\pgfqpoint{0.000000in}{0.000000in}}{%
\pgfpathmoveto{\pgfqpoint{0.000000in}{0.000000in}}%
\pgfpathlineto{\pgfqpoint{0.000000in}{-0.069444in}}%
\pgfusepath{stroke,fill}%
}%
\begin{pgfscope}%
\pgfsys@transformshift{1.654759in}{1.637544in}%
\pgfsys@useobject{currentmarker}{}%
\end{pgfscope}%
\end{pgfscope}%
\begin{pgfscope}%
\pgftext[x=1.654759in,y=0.347947in,,top]{\rmfamily\fontsize{8.000000}{9.600000}\selectfont 6}%
\end{pgfscope}%
\begin{pgfscope}%
\pgfsetbuttcap%
\pgfsetroundjoin%
\definecolor{currentfill}{rgb}{0.000000,0.000000,0.000000}%
\pgfsetfillcolor{currentfill}%
\pgfsetlinewidth{0.501875pt}%
\definecolor{currentstroke}{rgb}{0.000000,0.000000,0.000000}%
\pgfsetstrokecolor{currentstroke}%
\pgfsetdash{}{0pt}%
\pgfsys@defobject{currentmarker}{\pgfqpoint{0.000000in}{0.000000in}}{\pgfqpoint{0.000000in}{0.069444in}}{%
\pgfpathmoveto{\pgfqpoint{0.000000in}{0.000000in}}%
\pgfpathlineto{\pgfqpoint{0.000000in}{0.069444in}}%
\pgfusepath{stroke,fill}%
}%
\begin{pgfscope}%
\pgfsys@transformshift{2.107630in}{0.417391in}%
\pgfsys@useobject{currentmarker}{}%
\end{pgfscope}%
\end{pgfscope}%
\begin{pgfscope}%
\pgfsetbuttcap%
\pgfsetroundjoin%
\definecolor{currentfill}{rgb}{0.000000,0.000000,0.000000}%
\pgfsetfillcolor{currentfill}%
\pgfsetlinewidth{0.501875pt}%
\definecolor{currentstroke}{rgb}{0.000000,0.000000,0.000000}%
\pgfsetstrokecolor{currentstroke}%
\pgfsetdash{}{0pt}%
\pgfsys@defobject{currentmarker}{\pgfqpoint{0.000000in}{-0.069444in}}{\pgfqpoint{0.000000in}{0.000000in}}{%
\pgfpathmoveto{\pgfqpoint{0.000000in}{0.000000in}}%
\pgfpathlineto{\pgfqpoint{0.000000in}{-0.069444in}}%
\pgfusepath{stroke,fill}%
}%
\begin{pgfscope}%
\pgfsys@transformshift{2.107630in}{1.637544in}%
\pgfsys@useobject{currentmarker}{}%
\end{pgfscope}%
\end{pgfscope}%
\begin{pgfscope}%
\pgftext[x=2.107630in,y=0.347947in,,top]{\rmfamily\fontsize{8.000000}{9.600000}\selectfont 8}%
\end{pgfscope}%
\begin{pgfscope}%
\pgfsetbuttcap%
\pgfsetroundjoin%
\definecolor{currentfill}{rgb}{0.000000,0.000000,0.000000}%
\pgfsetfillcolor{currentfill}%
\pgfsetlinewidth{0.501875pt}%
\definecolor{currentstroke}{rgb}{0.000000,0.000000,0.000000}%
\pgfsetstrokecolor{currentstroke}%
\pgfsetdash{}{0pt}%
\pgfsys@defobject{currentmarker}{\pgfqpoint{0.000000in}{0.000000in}}{\pgfqpoint{0.000000in}{0.069444in}}{%
\pgfpathmoveto{\pgfqpoint{0.000000in}{0.000000in}}%
\pgfpathlineto{\pgfqpoint{0.000000in}{0.069444in}}%
\pgfusepath{stroke,fill}%
}%
\begin{pgfscope}%
\pgfsys@transformshift{2.560500in}{0.417391in}%
\pgfsys@useobject{currentmarker}{}%
\end{pgfscope}%
\end{pgfscope}%
\begin{pgfscope}%
\pgfsetbuttcap%
\pgfsetroundjoin%
\definecolor{currentfill}{rgb}{0.000000,0.000000,0.000000}%
\pgfsetfillcolor{currentfill}%
\pgfsetlinewidth{0.501875pt}%
\definecolor{currentstroke}{rgb}{0.000000,0.000000,0.000000}%
\pgfsetstrokecolor{currentstroke}%
\pgfsetdash{}{0pt}%
\pgfsys@defobject{currentmarker}{\pgfqpoint{0.000000in}{-0.069444in}}{\pgfqpoint{0.000000in}{0.000000in}}{%
\pgfpathmoveto{\pgfqpoint{0.000000in}{0.000000in}}%
\pgfpathlineto{\pgfqpoint{0.000000in}{-0.069444in}}%
\pgfusepath{stroke,fill}%
}%
\begin{pgfscope}%
\pgfsys@transformshift{2.560500in}{1.637544in}%
\pgfsys@useobject{currentmarker}{}%
\end{pgfscope}%
\end{pgfscope}%
\begin{pgfscope}%
\pgftext[x=2.560500in,y=0.347947in,,top]{\rmfamily\fontsize{8.000000}{9.600000}\selectfont 10}%
\end{pgfscope}%
\begin{pgfscope}%
\pgftext[x=1.428324in,y=0.170972in,,top]{\rmfamily\fontsize{9.000000}{10.800000}\selectfont \(\displaystyle t_{B^0}\)}%
\end{pgfscope}%
\begin{pgfscope}%
\pgfsetbuttcap%
\pgfsetroundjoin%
\definecolor{currentfill}{rgb}{0.000000,0.000000,0.000000}%
\pgfsetfillcolor{currentfill}%
\pgfsetlinewidth{0.501875pt}%
\definecolor{currentstroke}{rgb}{0.000000,0.000000,0.000000}%
\pgfsetstrokecolor{currentstroke}%
\pgfsetdash{}{0pt}%
\pgfsys@defobject{currentmarker}{\pgfqpoint{0.000000in}{0.000000in}}{\pgfqpoint{0.069444in}{0.000000in}}{%
\pgfpathmoveto{\pgfqpoint{0.000000in}{0.000000in}}%
\pgfpathlineto{\pgfqpoint{0.069444in}{0.000000in}}%
\pgfusepath{stroke,fill}%
}%
\begin{pgfscope}%
\pgfsys@transformshift{0.296148in}{0.417391in}%
\pgfsys@useobject{currentmarker}{}%
\end{pgfscope}%
\end{pgfscope}%
\begin{pgfscope}%
\pgfsetbuttcap%
\pgfsetroundjoin%
\definecolor{currentfill}{rgb}{0.000000,0.000000,0.000000}%
\pgfsetfillcolor{currentfill}%
\pgfsetlinewidth{0.501875pt}%
\definecolor{currentstroke}{rgb}{0.000000,0.000000,0.000000}%
\pgfsetstrokecolor{currentstroke}%
\pgfsetdash{}{0pt}%
\pgfsys@defobject{currentmarker}{\pgfqpoint{-0.069444in}{0.000000in}}{\pgfqpoint{0.000000in}{0.000000in}}{%
\pgfpathmoveto{\pgfqpoint{0.000000in}{0.000000in}}%
\pgfpathlineto{\pgfqpoint{-0.069444in}{0.000000in}}%
\pgfusepath{stroke,fill}%
}%
\begin{pgfscope}%
\pgfsys@transformshift{2.560500in}{0.417391in}%
\pgfsys@useobject{currentmarker}{}%
\end{pgfscope}%
\end{pgfscope}%
\begin{pgfscope}%
\pgftext[x=0.226704in,y=0.417391in,right,]{\rmfamily\fontsize{8.000000}{9.600000}\selectfont 0.0}%
\end{pgfscope}%
\begin{pgfscope}%
\pgfsetbuttcap%
\pgfsetroundjoin%
\definecolor{currentfill}{rgb}{0.000000,0.000000,0.000000}%
\pgfsetfillcolor{currentfill}%
\pgfsetlinewidth{0.501875pt}%
\definecolor{currentstroke}{rgb}{0.000000,0.000000,0.000000}%
\pgfsetstrokecolor{currentstroke}%
\pgfsetdash{}{0pt}%
\pgfsys@defobject{currentmarker}{\pgfqpoint{0.000000in}{0.000000in}}{\pgfqpoint{0.069444in}{0.000000in}}{%
\pgfpathmoveto{\pgfqpoint{0.000000in}{0.000000in}}%
\pgfpathlineto{\pgfqpoint{0.069444in}{0.000000in}}%
\pgfusepath{stroke,fill}%
}%
\begin{pgfscope}%
\pgfsys@transformshift{0.296148in}{0.552964in}%
\pgfsys@useobject{currentmarker}{}%
\end{pgfscope}%
\end{pgfscope}%
\begin{pgfscope}%
\pgfsetbuttcap%
\pgfsetroundjoin%
\definecolor{currentfill}{rgb}{0.000000,0.000000,0.000000}%
\pgfsetfillcolor{currentfill}%
\pgfsetlinewidth{0.501875pt}%
\definecolor{currentstroke}{rgb}{0.000000,0.000000,0.000000}%
\pgfsetstrokecolor{currentstroke}%
\pgfsetdash{}{0pt}%
\pgfsys@defobject{currentmarker}{\pgfqpoint{-0.069444in}{0.000000in}}{\pgfqpoint{0.000000in}{0.000000in}}{%
\pgfpathmoveto{\pgfqpoint{0.000000in}{0.000000in}}%
\pgfpathlineto{\pgfqpoint{-0.069444in}{0.000000in}}%
\pgfusepath{stroke,fill}%
}%
\begin{pgfscope}%
\pgfsys@transformshift{2.560500in}{0.552964in}%
\pgfsys@useobject{currentmarker}{}%
\end{pgfscope}%
\end{pgfscope}%
\begin{pgfscope}%
\pgftext[x=0.226704in,y=0.552964in,right,]{\rmfamily\fontsize{8.000000}{9.600000}\selectfont 0.1}%
\end{pgfscope}%
\begin{pgfscope}%
\pgfsetbuttcap%
\pgfsetroundjoin%
\definecolor{currentfill}{rgb}{0.000000,0.000000,0.000000}%
\pgfsetfillcolor{currentfill}%
\pgfsetlinewidth{0.501875pt}%
\definecolor{currentstroke}{rgb}{0.000000,0.000000,0.000000}%
\pgfsetstrokecolor{currentstroke}%
\pgfsetdash{}{0pt}%
\pgfsys@defobject{currentmarker}{\pgfqpoint{0.000000in}{0.000000in}}{\pgfqpoint{0.069444in}{0.000000in}}{%
\pgfpathmoveto{\pgfqpoint{0.000000in}{0.000000in}}%
\pgfpathlineto{\pgfqpoint{0.069444in}{0.000000in}}%
\pgfusepath{stroke,fill}%
}%
\begin{pgfscope}%
\pgfsys@transformshift{0.296148in}{0.688536in}%
\pgfsys@useobject{currentmarker}{}%
\end{pgfscope}%
\end{pgfscope}%
\begin{pgfscope}%
\pgfsetbuttcap%
\pgfsetroundjoin%
\definecolor{currentfill}{rgb}{0.000000,0.000000,0.000000}%
\pgfsetfillcolor{currentfill}%
\pgfsetlinewidth{0.501875pt}%
\definecolor{currentstroke}{rgb}{0.000000,0.000000,0.000000}%
\pgfsetstrokecolor{currentstroke}%
\pgfsetdash{}{0pt}%
\pgfsys@defobject{currentmarker}{\pgfqpoint{-0.069444in}{0.000000in}}{\pgfqpoint{0.000000in}{0.000000in}}{%
\pgfpathmoveto{\pgfqpoint{0.000000in}{0.000000in}}%
\pgfpathlineto{\pgfqpoint{-0.069444in}{0.000000in}}%
\pgfusepath{stroke,fill}%
}%
\begin{pgfscope}%
\pgfsys@transformshift{2.560500in}{0.688536in}%
\pgfsys@useobject{currentmarker}{}%
\end{pgfscope}%
\end{pgfscope}%
\begin{pgfscope}%
\pgftext[x=0.226704in,y=0.688536in,right,]{\rmfamily\fontsize{8.000000}{9.600000}\selectfont 0.2}%
\end{pgfscope}%
\begin{pgfscope}%
\pgfsetbuttcap%
\pgfsetroundjoin%
\definecolor{currentfill}{rgb}{0.000000,0.000000,0.000000}%
\pgfsetfillcolor{currentfill}%
\pgfsetlinewidth{0.501875pt}%
\definecolor{currentstroke}{rgb}{0.000000,0.000000,0.000000}%
\pgfsetstrokecolor{currentstroke}%
\pgfsetdash{}{0pt}%
\pgfsys@defobject{currentmarker}{\pgfqpoint{0.000000in}{0.000000in}}{\pgfqpoint{0.069444in}{0.000000in}}{%
\pgfpathmoveto{\pgfqpoint{0.000000in}{0.000000in}}%
\pgfpathlineto{\pgfqpoint{0.069444in}{0.000000in}}%
\pgfusepath{stroke,fill}%
}%
\begin{pgfscope}%
\pgfsys@transformshift{0.296148in}{0.824109in}%
\pgfsys@useobject{currentmarker}{}%
\end{pgfscope}%
\end{pgfscope}%
\begin{pgfscope}%
\pgfsetbuttcap%
\pgfsetroundjoin%
\definecolor{currentfill}{rgb}{0.000000,0.000000,0.000000}%
\pgfsetfillcolor{currentfill}%
\pgfsetlinewidth{0.501875pt}%
\definecolor{currentstroke}{rgb}{0.000000,0.000000,0.000000}%
\pgfsetstrokecolor{currentstroke}%
\pgfsetdash{}{0pt}%
\pgfsys@defobject{currentmarker}{\pgfqpoint{-0.069444in}{0.000000in}}{\pgfqpoint{0.000000in}{0.000000in}}{%
\pgfpathmoveto{\pgfqpoint{0.000000in}{0.000000in}}%
\pgfpathlineto{\pgfqpoint{-0.069444in}{0.000000in}}%
\pgfusepath{stroke,fill}%
}%
\begin{pgfscope}%
\pgfsys@transformshift{2.560500in}{0.824109in}%
\pgfsys@useobject{currentmarker}{}%
\end{pgfscope}%
\end{pgfscope}%
\begin{pgfscope}%
\pgftext[x=0.226704in,y=0.824109in,right,]{\rmfamily\fontsize{8.000000}{9.600000}\selectfont 0.3}%
\end{pgfscope}%
\begin{pgfscope}%
\pgfsetbuttcap%
\pgfsetroundjoin%
\definecolor{currentfill}{rgb}{0.000000,0.000000,0.000000}%
\pgfsetfillcolor{currentfill}%
\pgfsetlinewidth{0.501875pt}%
\definecolor{currentstroke}{rgb}{0.000000,0.000000,0.000000}%
\pgfsetstrokecolor{currentstroke}%
\pgfsetdash{}{0pt}%
\pgfsys@defobject{currentmarker}{\pgfqpoint{0.000000in}{0.000000in}}{\pgfqpoint{0.069444in}{0.000000in}}{%
\pgfpathmoveto{\pgfqpoint{0.000000in}{0.000000in}}%
\pgfpathlineto{\pgfqpoint{0.069444in}{0.000000in}}%
\pgfusepath{stroke,fill}%
}%
\begin{pgfscope}%
\pgfsys@transformshift{0.296148in}{0.959681in}%
\pgfsys@useobject{currentmarker}{}%
\end{pgfscope}%
\end{pgfscope}%
\begin{pgfscope}%
\pgfsetbuttcap%
\pgfsetroundjoin%
\definecolor{currentfill}{rgb}{0.000000,0.000000,0.000000}%
\pgfsetfillcolor{currentfill}%
\pgfsetlinewidth{0.501875pt}%
\definecolor{currentstroke}{rgb}{0.000000,0.000000,0.000000}%
\pgfsetstrokecolor{currentstroke}%
\pgfsetdash{}{0pt}%
\pgfsys@defobject{currentmarker}{\pgfqpoint{-0.069444in}{0.000000in}}{\pgfqpoint{0.000000in}{0.000000in}}{%
\pgfpathmoveto{\pgfqpoint{0.000000in}{0.000000in}}%
\pgfpathlineto{\pgfqpoint{-0.069444in}{0.000000in}}%
\pgfusepath{stroke,fill}%
}%
\begin{pgfscope}%
\pgfsys@transformshift{2.560500in}{0.959681in}%
\pgfsys@useobject{currentmarker}{}%
\end{pgfscope}%
\end{pgfscope}%
\begin{pgfscope}%
\pgftext[x=0.226704in,y=0.959681in,right,]{\rmfamily\fontsize{8.000000}{9.600000}\selectfont 0.4}%
\end{pgfscope}%
\begin{pgfscope}%
\pgfsetbuttcap%
\pgfsetroundjoin%
\definecolor{currentfill}{rgb}{0.000000,0.000000,0.000000}%
\pgfsetfillcolor{currentfill}%
\pgfsetlinewidth{0.501875pt}%
\definecolor{currentstroke}{rgb}{0.000000,0.000000,0.000000}%
\pgfsetstrokecolor{currentstroke}%
\pgfsetdash{}{0pt}%
\pgfsys@defobject{currentmarker}{\pgfqpoint{0.000000in}{0.000000in}}{\pgfqpoint{0.069444in}{0.000000in}}{%
\pgfpathmoveto{\pgfqpoint{0.000000in}{0.000000in}}%
\pgfpathlineto{\pgfqpoint{0.069444in}{0.000000in}}%
\pgfusepath{stroke,fill}%
}%
\begin{pgfscope}%
\pgfsys@transformshift{0.296148in}{1.095254in}%
\pgfsys@useobject{currentmarker}{}%
\end{pgfscope}%
\end{pgfscope}%
\begin{pgfscope}%
\pgfsetbuttcap%
\pgfsetroundjoin%
\definecolor{currentfill}{rgb}{0.000000,0.000000,0.000000}%
\pgfsetfillcolor{currentfill}%
\pgfsetlinewidth{0.501875pt}%
\definecolor{currentstroke}{rgb}{0.000000,0.000000,0.000000}%
\pgfsetstrokecolor{currentstroke}%
\pgfsetdash{}{0pt}%
\pgfsys@defobject{currentmarker}{\pgfqpoint{-0.069444in}{0.000000in}}{\pgfqpoint{0.000000in}{0.000000in}}{%
\pgfpathmoveto{\pgfqpoint{0.000000in}{0.000000in}}%
\pgfpathlineto{\pgfqpoint{-0.069444in}{0.000000in}}%
\pgfusepath{stroke,fill}%
}%
\begin{pgfscope}%
\pgfsys@transformshift{2.560500in}{1.095254in}%
\pgfsys@useobject{currentmarker}{}%
\end{pgfscope}%
\end{pgfscope}%
\begin{pgfscope}%
\pgftext[x=0.226704in,y=1.095254in,right,]{\rmfamily\fontsize{8.000000}{9.600000}\selectfont 0.5}%
\end{pgfscope}%
\begin{pgfscope}%
\pgfsetbuttcap%
\pgfsetroundjoin%
\definecolor{currentfill}{rgb}{0.000000,0.000000,0.000000}%
\pgfsetfillcolor{currentfill}%
\pgfsetlinewidth{0.501875pt}%
\definecolor{currentstroke}{rgb}{0.000000,0.000000,0.000000}%
\pgfsetstrokecolor{currentstroke}%
\pgfsetdash{}{0pt}%
\pgfsys@defobject{currentmarker}{\pgfqpoint{0.000000in}{0.000000in}}{\pgfqpoint{0.069444in}{0.000000in}}{%
\pgfpathmoveto{\pgfqpoint{0.000000in}{0.000000in}}%
\pgfpathlineto{\pgfqpoint{0.069444in}{0.000000in}}%
\pgfusepath{stroke,fill}%
}%
\begin{pgfscope}%
\pgfsys@transformshift{0.296148in}{1.230826in}%
\pgfsys@useobject{currentmarker}{}%
\end{pgfscope}%
\end{pgfscope}%
\begin{pgfscope}%
\pgfsetbuttcap%
\pgfsetroundjoin%
\definecolor{currentfill}{rgb}{0.000000,0.000000,0.000000}%
\pgfsetfillcolor{currentfill}%
\pgfsetlinewidth{0.501875pt}%
\definecolor{currentstroke}{rgb}{0.000000,0.000000,0.000000}%
\pgfsetstrokecolor{currentstroke}%
\pgfsetdash{}{0pt}%
\pgfsys@defobject{currentmarker}{\pgfqpoint{-0.069444in}{0.000000in}}{\pgfqpoint{0.000000in}{0.000000in}}{%
\pgfpathmoveto{\pgfqpoint{0.000000in}{0.000000in}}%
\pgfpathlineto{\pgfqpoint{-0.069444in}{0.000000in}}%
\pgfusepath{stroke,fill}%
}%
\begin{pgfscope}%
\pgfsys@transformshift{2.560500in}{1.230826in}%
\pgfsys@useobject{currentmarker}{}%
\end{pgfscope}%
\end{pgfscope}%
\begin{pgfscope}%
\pgftext[x=0.226704in,y=1.230826in,right,]{\rmfamily\fontsize{8.000000}{9.600000}\selectfont 0.6}%
\end{pgfscope}%
\begin{pgfscope}%
\pgfsetbuttcap%
\pgfsetroundjoin%
\definecolor{currentfill}{rgb}{0.000000,0.000000,0.000000}%
\pgfsetfillcolor{currentfill}%
\pgfsetlinewidth{0.501875pt}%
\definecolor{currentstroke}{rgb}{0.000000,0.000000,0.000000}%
\pgfsetstrokecolor{currentstroke}%
\pgfsetdash{}{0pt}%
\pgfsys@defobject{currentmarker}{\pgfqpoint{0.000000in}{0.000000in}}{\pgfqpoint{0.069444in}{0.000000in}}{%
\pgfpathmoveto{\pgfqpoint{0.000000in}{0.000000in}}%
\pgfpathlineto{\pgfqpoint{0.069444in}{0.000000in}}%
\pgfusepath{stroke,fill}%
}%
\begin{pgfscope}%
\pgfsys@transformshift{0.296148in}{1.366399in}%
\pgfsys@useobject{currentmarker}{}%
\end{pgfscope}%
\end{pgfscope}%
\begin{pgfscope}%
\pgfsetbuttcap%
\pgfsetroundjoin%
\definecolor{currentfill}{rgb}{0.000000,0.000000,0.000000}%
\pgfsetfillcolor{currentfill}%
\pgfsetlinewidth{0.501875pt}%
\definecolor{currentstroke}{rgb}{0.000000,0.000000,0.000000}%
\pgfsetstrokecolor{currentstroke}%
\pgfsetdash{}{0pt}%
\pgfsys@defobject{currentmarker}{\pgfqpoint{-0.069444in}{0.000000in}}{\pgfqpoint{0.000000in}{0.000000in}}{%
\pgfpathmoveto{\pgfqpoint{0.000000in}{0.000000in}}%
\pgfpathlineto{\pgfqpoint{-0.069444in}{0.000000in}}%
\pgfusepath{stroke,fill}%
}%
\begin{pgfscope}%
\pgfsys@transformshift{2.560500in}{1.366399in}%
\pgfsys@useobject{currentmarker}{}%
\end{pgfscope}%
\end{pgfscope}%
\begin{pgfscope}%
\pgftext[x=0.226704in,y=1.366399in,right,]{\rmfamily\fontsize{8.000000}{9.600000}\selectfont 0.7}%
\end{pgfscope}%
\begin{pgfscope}%
\pgfsetbuttcap%
\pgfsetroundjoin%
\definecolor{currentfill}{rgb}{0.000000,0.000000,0.000000}%
\pgfsetfillcolor{currentfill}%
\pgfsetlinewidth{0.501875pt}%
\definecolor{currentstroke}{rgb}{0.000000,0.000000,0.000000}%
\pgfsetstrokecolor{currentstroke}%
\pgfsetdash{}{0pt}%
\pgfsys@defobject{currentmarker}{\pgfqpoint{0.000000in}{0.000000in}}{\pgfqpoint{0.069444in}{0.000000in}}{%
\pgfpathmoveto{\pgfqpoint{0.000000in}{0.000000in}}%
\pgfpathlineto{\pgfqpoint{0.069444in}{0.000000in}}%
\pgfusepath{stroke,fill}%
}%
\begin{pgfscope}%
\pgfsys@transformshift{0.296148in}{1.501971in}%
\pgfsys@useobject{currentmarker}{}%
\end{pgfscope}%
\end{pgfscope}%
\begin{pgfscope}%
\pgfsetbuttcap%
\pgfsetroundjoin%
\definecolor{currentfill}{rgb}{0.000000,0.000000,0.000000}%
\pgfsetfillcolor{currentfill}%
\pgfsetlinewidth{0.501875pt}%
\definecolor{currentstroke}{rgb}{0.000000,0.000000,0.000000}%
\pgfsetstrokecolor{currentstroke}%
\pgfsetdash{}{0pt}%
\pgfsys@defobject{currentmarker}{\pgfqpoint{-0.069444in}{0.000000in}}{\pgfqpoint{0.000000in}{0.000000in}}{%
\pgfpathmoveto{\pgfqpoint{0.000000in}{0.000000in}}%
\pgfpathlineto{\pgfqpoint{-0.069444in}{0.000000in}}%
\pgfusepath{stroke,fill}%
}%
\begin{pgfscope}%
\pgfsys@transformshift{2.560500in}{1.501971in}%
\pgfsys@useobject{currentmarker}{}%
\end{pgfscope}%
\end{pgfscope}%
\begin{pgfscope}%
\pgftext[x=0.226704in,y=1.501971in,right,]{\rmfamily\fontsize{8.000000}{9.600000}\selectfont 0.8}%
\end{pgfscope}%
\begin{pgfscope}%
\pgfsetbuttcap%
\pgfsetroundjoin%
\definecolor{currentfill}{rgb}{0.000000,0.000000,0.000000}%
\pgfsetfillcolor{currentfill}%
\pgfsetlinewidth{0.501875pt}%
\definecolor{currentstroke}{rgb}{0.000000,0.000000,0.000000}%
\pgfsetstrokecolor{currentstroke}%
\pgfsetdash{}{0pt}%
\pgfsys@defobject{currentmarker}{\pgfqpoint{0.000000in}{0.000000in}}{\pgfqpoint{0.069444in}{0.000000in}}{%
\pgfpathmoveto{\pgfqpoint{0.000000in}{0.000000in}}%
\pgfpathlineto{\pgfqpoint{0.069444in}{0.000000in}}%
\pgfusepath{stroke,fill}%
}%
\begin{pgfscope}%
\pgfsys@transformshift{0.296148in}{1.637544in}%
\pgfsys@useobject{currentmarker}{}%
\end{pgfscope}%
\end{pgfscope}%
\begin{pgfscope}%
\pgfsetbuttcap%
\pgfsetroundjoin%
\definecolor{currentfill}{rgb}{0.000000,0.000000,0.000000}%
\pgfsetfillcolor{currentfill}%
\pgfsetlinewidth{0.501875pt}%
\definecolor{currentstroke}{rgb}{0.000000,0.000000,0.000000}%
\pgfsetstrokecolor{currentstroke}%
\pgfsetdash{}{0pt}%
\pgfsys@defobject{currentmarker}{\pgfqpoint{-0.069444in}{0.000000in}}{\pgfqpoint{0.000000in}{0.000000in}}{%
\pgfpathmoveto{\pgfqpoint{0.000000in}{0.000000in}}%
\pgfpathlineto{\pgfqpoint{-0.069444in}{0.000000in}}%
\pgfusepath{stroke,fill}%
}%
\begin{pgfscope}%
\pgfsys@transformshift{2.560500in}{1.637544in}%
\pgfsys@useobject{currentmarker}{}%
\end{pgfscope}%
\end{pgfscope}%
\begin{pgfscope}%
\pgftext[x=0.226704in,y=1.637544in,right,]{\rmfamily\fontsize{8.000000}{9.600000}\selectfont 0.9}%
\end{pgfscope}%
\end{pgfpicture}%
\makeatother%
\endgroup%

	\end{subfigure}

	\begin{subfigure}[t]{0.49\textwidth}
		\centering
    %\includegraphics[width=\textwidth]{store/variables/SIG_BKG_Kplus_PIDK.pdf}
    %% Creator: Matplotlib, PGF backend
%%
%% To include the figure in your LaTeX document, write
%%   \input{<filename>.pgf}
%%
%% Make sure the required packages are loaded in your preamble
%%   \usepackage{pgf}
%%
%% Figures using additional raster images can only be included by \input if
%% they are in the same directory as the main LaTeX file. For loading figures
%% from other directories you can use the `import` package
%%   \usepackage{import}
%% and then include the figures with
%%   \import{<path to file>}{<filename>.pgf}
%%
%% Matplotlib used the following preamble
%%   \usepackage{fontspec}
%%   \setmainfont{DejaVu Serif}
%%   \setsansfont{DejaVu Sans}
%%   \setmonofont{DejaVu Sans Mono}
%%
\begingroup%
\makeatletter%
\begin{pgfpicture}%
\pgfpathrectangle{\pgfpointorigin}{\pgfqpoint{2.681597in}{1.719349in}}%
\pgfusepath{use as bounding box, clip}%
\begin{pgfscope}%
\pgfsetbuttcap%
\pgfsetmiterjoin%
\definecolor{currentfill}{rgb}{1.000000,1.000000,1.000000}%
\pgfsetfillcolor{currentfill}%
\pgfsetlinewidth{0.000000pt}%
\definecolor{currentstroke}{rgb}{1.000000,1.000000,1.000000}%
\pgfsetstrokecolor{currentstroke}%
\pgfsetdash{}{0pt}%
\pgfpathmoveto{\pgfqpoint{0.000000in}{0.000000in}}%
\pgfpathlineto{\pgfqpoint{2.681597in}{0.000000in}}%
\pgfpathlineto{\pgfqpoint{2.681597in}{1.719349in}}%
\pgfpathlineto{\pgfqpoint{0.000000in}{1.719349in}}%
\pgfpathclose%
\pgfusepath{fill}%
\end{pgfscope}%
\begin{pgfscope}%
\pgfsetbuttcap%
\pgfsetmiterjoin%
\definecolor{currentfill}{rgb}{1.000000,1.000000,1.000000}%
\pgfsetfillcolor{currentfill}%
\pgfsetlinewidth{0.000000pt}%
\definecolor{currentstroke}{rgb}{0.000000,0.000000,0.000000}%
\pgfsetstrokecolor{currentstroke}%
\pgfsetstrokeopacity{0.000000}%
\pgfsetdash{}{0pt}%
\pgfpathmoveto{\pgfqpoint{0.366840in}{0.449983in}}%
\pgfpathlineto{\pgfqpoint{2.525558in}{0.449983in}}%
\pgfpathlineto{\pgfqpoint{2.525558in}{1.615583in}}%
\pgfpathlineto{\pgfqpoint{0.366840in}{1.615583in}}%
\pgfpathclose%
\pgfusepath{fill}%
\end{pgfscope}%
\begin{pgfscope}%
\pgfpathrectangle{\pgfqpoint{0.366840in}{0.449983in}}{\pgfqpoint{2.158718in}{1.165600in}} %
\pgfusepath{clip}%
\pgfsetbuttcap%
\pgfsetmiterjoin%
\definecolor{currentfill}{rgb}{0.215686,0.470588,0.749020}%
\pgfsetfillcolor{currentfill}%
\pgfsetlinewidth{0.000000pt}%
\definecolor{currentstroke}{rgb}{0.000000,0.000000,0.000000}%
\pgfsetstrokecolor{currentstroke}%
\pgfsetdash{}{0pt}%
\pgfpathmoveto{\pgfqpoint{0.434300in}{0.449983in}}%
\pgfpathlineto{\pgfqpoint{0.434300in}{0.449983in}}%
\pgfpathlineto{\pgfqpoint{0.473427in}{0.449983in}}%
\pgfpathlineto{\pgfqpoint{0.473427in}{0.449983in}}%
\pgfpathlineto{\pgfqpoint{0.512554in}{0.449983in}}%
\pgfpathlineto{\pgfqpoint{0.512554in}{0.449983in}}%
\pgfpathlineto{\pgfqpoint{0.551681in}{0.449983in}}%
\pgfpathlineto{\pgfqpoint{0.551681in}{0.511441in}}%
\pgfpathlineto{\pgfqpoint{0.590807in}{0.511441in}}%
\pgfpathlineto{\pgfqpoint{0.590807in}{0.671767in}}%
\pgfpathlineto{\pgfqpoint{0.629934in}{0.671767in}}%
\pgfpathlineto{\pgfqpoint{0.629934in}{0.839197in}}%
\pgfpathlineto{\pgfqpoint{0.669061in}{0.839197in}}%
\pgfpathlineto{\pgfqpoint{0.669061in}{0.840213in}}%
\pgfpathlineto{\pgfqpoint{0.708188in}{0.840213in}}%
\pgfpathlineto{\pgfqpoint{0.708188in}{0.852197in}}%
\pgfpathlineto{\pgfqpoint{0.747314in}{0.852197in}}%
\pgfpathlineto{\pgfqpoint{0.747314in}{0.882025in}}%
\pgfpathlineto{\pgfqpoint{0.786441in}{0.882025in}}%
\pgfpathlineto{\pgfqpoint{0.786441in}{0.938926in}}%
\pgfpathlineto{\pgfqpoint{0.825568in}{0.938926in}}%
\pgfpathlineto{\pgfqpoint{0.825568in}{0.976800in}}%
\pgfpathlineto{\pgfqpoint{0.864695in}{0.976800in}}%
\pgfpathlineto{\pgfqpoint{0.864695in}{0.994132in}}%
\pgfpathlineto{\pgfqpoint{0.903821in}{0.994132in}}%
\pgfpathlineto{\pgfqpoint{0.903821in}{0.954684in}}%
\pgfpathlineto{\pgfqpoint{0.942948in}{0.954684in}}%
\pgfpathlineto{\pgfqpoint{0.942948in}{0.960909in}}%
\pgfpathlineto{\pgfqpoint{0.982075in}{0.960909in}}%
\pgfpathlineto{\pgfqpoint{0.982075in}{0.898990in}}%
\pgfpathlineto{\pgfqpoint{1.021202in}{0.898990in}}%
\pgfpathlineto{\pgfqpoint{1.021202in}{0.807338in}}%
\pgfpathlineto{\pgfqpoint{1.060328in}{0.807338in}}%
\pgfpathlineto{\pgfqpoint{1.060328in}{0.759219in}}%
\pgfpathlineto{\pgfqpoint{1.099455in}{0.759219in}}%
\pgfpathlineto{\pgfqpoint{1.099455in}{0.674082in}}%
\pgfpathlineto{\pgfqpoint{1.138582in}{0.674082in}}%
\pgfpathlineto{\pgfqpoint{1.138582in}{0.629983in}}%
\pgfpathlineto{\pgfqpoint{1.177709in}{0.629983in}}%
\pgfpathlineto{\pgfqpoint{1.177709in}{0.600998in}}%
\pgfpathlineto{\pgfqpoint{1.216836in}{0.600998in}}%
\pgfpathlineto{\pgfqpoint{1.216836in}{0.570768in}}%
\pgfpathlineto{\pgfqpoint{1.255962in}{0.570768in}}%
\pgfpathlineto{\pgfqpoint{1.255962in}{0.557524in}}%
\pgfpathlineto{\pgfqpoint{1.295089in}{0.557524in}}%
\pgfpathlineto{\pgfqpoint{1.295089in}{0.522661in}}%
\pgfpathlineto{\pgfqpoint{1.334216in}{0.522661in}}%
\pgfpathlineto{\pgfqpoint{1.334216in}{0.513330in}}%
\pgfpathlineto{\pgfqpoint{1.373343in}{0.513330in}}%
\pgfpathlineto{\pgfqpoint{1.373343in}{0.495268in}}%
\pgfpathlineto{\pgfqpoint{1.412469in}{0.495268in}}%
\pgfpathlineto{\pgfqpoint{1.412469in}{0.492922in}}%
\pgfpathlineto{\pgfqpoint{1.451596in}{0.492922in}}%
\pgfpathlineto{\pgfqpoint{1.451596in}{0.474556in}}%
\pgfpathlineto{\pgfqpoint{1.490723in}{0.474556in}}%
\pgfpathlineto{\pgfqpoint{1.490723in}{0.470055in}}%
\pgfpathlineto{\pgfqpoint{1.529850in}{0.470055in}}%
\pgfpathlineto{\pgfqpoint{1.529850in}{0.464826in}}%
\pgfpathlineto{\pgfqpoint{1.568976in}{0.464826in}}%
\pgfpathlineto{\pgfqpoint{1.568976in}{0.462225in}}%
\pgfpathlineto{\pgfqpoint{1.608103in}{0.462225in}}%
\pgfpathlineto{\pgfqpoint{1.608103in}{0.456226in}}%
\pgfpathlineto{\pgfqpoint{1.647230in}{0.456226in}}%
\pgfpathlineto{\pgfqpoint{1.647230in}{0.457637in}}%
\pgfpathlineto{\pgfqpoint{1.686357in}{0.457637in}}%
\pgfpathlineto{\pgfqpoint{1.686357in}{0.453558in}}%
\pgfpathlineto{\pgfqpoint{1.725483in}{0.453558in}}%
\pgfpathlineto{\pgfqpoint{1.725483in}{0.451836in}}%
\pgfpathlineto{\pgfqpoint{1.764610in}{0.451836in}}%
\pgfpathlineto{\pgfqpoint{1.764610in}{0.453709in}}%
\pgfpathlineto{\pgfqpoint{1.803737in}{0.453709in}}%
\pgfpathlineto{\pgfqpoint{1.803737in}{0.452567in}}%
\pgfpathlineto{\pgfqpoint{1.842864in}{0.452567in}}%
\pgfpathlineto{\pgfqpoint{1.842864in}{0.452219in}}%
\pgfpathlineto{\pgfqpoint{1.881990in}{0.452219in}}%
\pgfpathlineto{\pgfqpoint{1.881990in}{0.451927in}}%
\pgfpathlineto{\pgfqpoint{1.921117in}{0.451927in}}%
\pgfpathlineto{\pgfqpoint{1.921117in}{0.450755in}}%
\pgfpathlineto{\pgfqpoint{1.960244in}{0.450755in}}%
\pgfpathlineto{\pgfqpoint{1.960244in}{0.450307in}}%
\pgfpathlineto{\pgfqpoint{1.999371in}{0.450307in}}%
\pgfpathlineto{\pgfqpoint{1.999371in}{0.450299in}}%
\pgfpathlineto{\pgfqpoint{2.038498in}{0.450299in}}%
\pgfpathlineto{\pgfqpoint{2.038498in}{0.450011in}}%
\pgfpathlineto{\pgfqpoint{2.077624in}{0.450011in}}%
\pgfpathlineto{\pgfqpoint{2.077624in}{0.450017in}}%
\pgfpathlineto{\pgfqpoint{2.116751in}{0.450017in}}%
\pgfpathlineto{\pgfqpoint{2.116751in}{0.449983in}}%
\pgfpathlineto{\pgfqpoint{2.155878in}{0.449983in}}%
\pgfpathlineto{\pgfqpoint{2.155878in}{0.449983in}}%
\pgfpathlineto{\pgfqpoint{2.195005in}{0.449983in}}%
\pgfpathlineto{\pgfqpoint{2.195005in}{0.450042in}}%
\pgfpathlineto{\pgfqpoint{2.234131in}{0.450042in}}%
\pgfpathlineto{\pgfqpoint{2.234131in}{0.450008in}}%
\pgfpathlineto{\pgfqpoint{2.273258in}{0.450008in}}%
\pgfpathlineto{\pgfqpoint{2.273258in}{0.449983in}}%
\pgfpathlineto{\pgfqpoint{2.312385in}{0.449983in}}%
\pgfpathlineto{\pgfqpoint{2.312385in}{0.449983in}}%
\pgfpathlineto{\pgfqpoint{2.351512in}{0.449983in}}%
\pgfpathlineto{\pgfqpoint{2.351512in}{0.449983in}}%
\pgfpathlineto{\pgfqpoint{2.390638in}{0.449983in}}%
\pgfpathlineto{\pgfqpoint{2.390638in}{0.449983in}}%
\pgfpathlineto{\pgfqpoint{2.351512in}{0.449983in}}%
\pgfpathlineto{\pgfqpoint{2.351512in}{0.449983in}}%
\pgfpathlineto{\pgfqpoint{2.312385in}{0.449983in}}%
\pgfpathlineto{\pgfqpoint{2.312385in}{0.449983in}}%
\pgfpathlineto{\pgfqpoint{2.273258in}{0.449983in}}%
\pgfpathlineto{\pgfqpoint{2.273258in}{0.449983in}}%
\pgfpathlineto{\pgfqpoint{2.234131in}{0.449983in}}%
\pgfpathlineto{\pgfqpoint{2.234131in}{0.449983in}}%
\pgfpathlineto{\pgfqpoint{2.195005in}{0.449983in}}%
\pgfpathlineto{\pgfqpoint{2.195005in}{0.449983in}}%
\pgfpathlineto{\pgfqpoint{2.155878in}{0.449983in}}%
\pgfpathlineto{\pgfqpoint{2.155878in}{0.449983in}}%
\pgfpathlineto{\pgfqpoint{2.116751in}{0.449983in}}%
\pgfpathlineto{\pgfqpoint{2.116751in}{0.449983in}}%
\pgfpathlineto{\pgfqpoint{2.077624in}{0.449983in}}%
\pgfpathlineto{\pgfqpoint{2.077624in}{0.449983in}}%
\pgfpathlineto{\pgfqpoint{2.038498in}{0.449983in}}%
\pgfpathlineto{\pgfqpoint{2.038498in}{0.449983in}}%
\pgfpathlineto{\pgfqpoint{1.999371in}{0.449983in}}%
\pgfpathlineto{\pgfqpoint{1.999371in}{0.449983in}}%
\pgfpathlineto{\pgfqpoint{1.960244in}{0.449983in}}%
\pgfpathlineto{\pgfqpoint{1.960244in}{0.449983in}}%
\pgfpathlineto{\pgfqpoint{1.921117in}{0.449983in}}%
\pgfpathlineto{\pgfqpoint{1.921117in}{0.449983in}}%
\pgfpathlineto{\pgfqpoint{1.881990in}{0.449983in}}%
\pgfpathlineto{\pgfqpoint{1.881990in}{0.449983in}}%
\pgfpathlineto{\pgfqpoint{1.842864in}{0.449983in}}%
\pgfpathlineto{\pgfqpoint{1.842864in}{0.449983in}}%
\pgfpathlineto{\pgfqpoint{1.803737in}{0.449983in}}%
\pgfpathlineto{\pgfqpoint{1.803737in}{0.449983in}}%
\pgfpathlineto{\pgfqpoint{1.764610in}{0.449983in}}%
\pgfpathlineto{\pgfqpoint{1.764610in}{0.449983in}}%
\pgfpathlineto{\pgfqpoint{1.725483in}{0.449983in}}%
\pgfpathlineto{\pgfqpoint{1.725483in}{0.449983in}}%
\pgfpathlineto{\pgfqpoint{1.686357in}{0.449983in}}%
\pgfpathlineto{\pgfqpoint{1.686357in}{0.449983in}}%
\pgfpathlineto{\pgfqpoint{1.647230in}{0.449983in}}%
\pgfpathlineto{\pgfqpoint{1.647230in}{0.449983in}}%
\pgfpathlineto{\pgfqpoint{1.608103in}{0.449983in}}%
\pgfpathlineto{\pgfqpoint{1.608103in}{0.449983in}}%
\pgfpathlineto{\pgfqpoint{1.568976in}{0.449983in}}%
\pgfpathlineto{\pgfqpoint{1.568976in}{0.449983in}}%
\pgfpathlineto{\pgfqpoint{1.529850in}{0.449983in}}%
\pgfpathlineto{\pgfqpoint{1.529850in}{0.449983in}}%
\pgfpathlineto{\pgfqpoint{1.490723in}{0.449983in}}%
\pgfpathlineto{\pgfqpoint{1.490723in}{0.449983in}}%
\pgfpathlineto{\pgfqpoint{1.451596in}{0.449983in}}%
\pgfpathlineto{\pgfqpoint{1.451596in}{0.449983in}}%
\pgfpathlineto{\pgfqpoint{1.412469in}{0.449983in}}%
\pgfpathlineto{\pgfqpoint{1.412469in}{0.449983in}}%
\pgfpathlineto{\pgfqpoint{1.373343in}{0.449983in}}%
\pgfpathlineto{\pgfqpoint{1.373343in}{0.449983in}}%
\pgfpathlineto{\pgfqpoint{1.334216in}{0.449983in}}%
\pgfpathlineto{\pgfqpoint{1.334216in}{0.449983in}}%
\pgfpathlineto{\pgfqpoint{1.295089in}{0.449983in}}%
\pgfpathlineto{\pgfqpoint{1.295089in}{0.449983in}}%
\pgfpathlineto{\pgfqpoint{1.255962in}{0.449983in}}%
\pgfpathlineto{\pgfqpoint{1.255962in}{0.449983in}}%
\pgfpathlineto{\pgfqpoint{1.216836in}{0.449983in}}%
\pgfpathlineto{\pgfqpoint{1.216836in}{0.449983in}}%
\pgfpathlineto{\pgfqpoint{1.177709in}{0.449983in}}%
\pgfpathlineto{\pgfqpoint{1.177709in}{0.449983in}}%
\pgfpathlineto{\pgfqpoint{1.138582in}{0.449983in}}%
\pgfpathlineto{\pgfqpoint{1.138582in}{0.449983in}}%
\pgfpathlineto{\pgfqpoint{1.099455in}{0.449983in}}%
\pgfpathlineto{\pgfqpoint{1.099455in}{0.449983in}}%
\pgfpathlineto{\pgfqpoint{1.060328in}{0.449983in}}%
\pgfpathlineto{\pgfqpoint{1.060328in}{0.449983in}}%
\pgfpathlineto{\pgfqpoint{1.021202in}{0.449983in}}%
\pgfpathlineto{\pgfqpoint{1.021202in}{0.449983in}}%
\pgfpathlineto{\pgfqpoint{0.982075in}{0.449983in}}%
\pgfpathlineto{\pgfqpoint{0.982075in}{0.449983in}}%
\pgfpathlineto{\pgfqpoint{0.942948in}{0.449983in}}%
\pgfpathlineto{\pgfqpoint{0.942948in}{0.449983in}}%
\pgfpathlineto{\pgfqpoint{0.903821in}{0.449983in}}%
\pgfpathlineto{\pgfqpoint{0.903821in}{0.449983in}}%
\pgfpathlineto{\pgfqpoint{0.864695in}{0.449983in}}%
\pgfpathlineto{\pgfqpoint{0.864695in}{0.449983in}}%
\pgfpathlineto{\pgfqpoint{0.825568in}{0.449983in}}%
\pgfpathlineto{\pgfqpoint{0.825568in}{0.449983in}}%
\pgfpathlineto{\pgfqpoint{0.786441in}{0.449983in}}%
\pgfpathlineto{\pgfqpoint{0.786441in}{0.449983in}}%
\pgfpathlineto{\pgfqpoint{0.747314in}{0.449983in}}%
\pgfpathlineto{\pgfqpoint{0.747314in}{0.449983in}}%
\pgfpathlineto{\pgfqpoint{0.708188in}{0.449983in}}%
\pgfpathlineto{\pgfqpoint{0.708188in}{0.449983in}}%
\pgfpathlineto{\pgfqpoint{0.669061in}{0.449983in}}%
\pgfpathlineto{\pgfqpoint{0.669061in}{0.449983in}}%
\pgfpathlineto{\pgfqpoint{0.629934in}{0.449983in}}%
\pgfpathlineto{\pgfqpoint{0.629934in}{0.449983in}}%
\pgfpathlineto{\pgfqpoint{0.590807in}{0.449983in}}%
\pgfpathlineto{\pgfqpoint{0.590807in}{0.449983in}}%
\pgfpathlineto{\pgfqpoint{0.551681in}{0.449983in}}%
\pgfpathlineto{\pgfqpoint{0.551681in}{0.449983in}}%
\pgfpathlineto{\pgfqpoint{0.512554in}{0.449983in}}%
\pgfpathlineto{\pgfqpoint{0.512554in}{0.449983in}}%
\pgfpathlineto{\pgfqpoint{0.473427in}{0.449983in}}%
\pgfpathlineto{\pgfqpoint{0.473427in}{0.449983in}}%
\pgfpathlineto{\pgfqpoint{0.434300in}{0.449983in}}%
\pgfusepath{fill}%
\end{pgfscope}%
\begin{pgfscope}%
\pgfpathrectangle{\pgfqpoint{0.366840in}{0.449983in}}{\pgfqpoint{2.158718in}{1.165600in}} %
\pgfusepath{clip}%
\pgfsetbuttcap%
\pgfsetmiterjoin%
\pgfsetlinewidth{0.501875pt}%
\definecolor{currentstroke}{rgb}{1.000000,0.000000,0.000000}%
\pgfsetstrokecolor{currentstroke}%
\pgfsetdash{}{0pt}%
\pgfpathmoveto{\pgfqpoint{0.434300in}{0.449983in}}%
\pgfpathlineto{\pgfqpoint{0.434300in}{0.449983in}}%
\pgfpathlineto{\pgfqpoint{0.473427in}{0.449983in}}%
\pgfpathlineto{\pgfqpoint{0.473427in}{0.449983in}}%
\pgfpathlineto{\pgfqpoint{0.512554in}{0.449983in}}%
\pgfpathlineto{\pgfqpoint{0.512554in}{0.449983in}}%
\pgfpathlineto{\pgfqpoint{0.551681in}{0.449983in}}%
\pgfpathlineto{\pgfqpoint{0.551681in}{0.841444in}}%
\pgfpathlineto{\pgfqpoint{0.590807in}{0.841444in}}%
\pgfpathlineto{\pgfqpoint{0.590807in}{1.308055in}}%
\pgfpathlineto{\pgfqpoint{0.629934in}{1.308055in}}%
\pgfpathlineto{\pgfqpoint{0.629934in}{1.443137in}}%
\pgfpathlineto{\pgfqpoint{0.669061in}{1.443137in}}%
\pgfpathlineto{\pgfqpoint{0.669061in}{1.281322in}}%
\pgfpathlineto{\pgfqpoint{0.708188in}{1.281322in}}%
\pgfpathlineto{\pgfqpoint{0.708188in}{1.110098in}}%
\pgfpathlineto{\pgfqpoint{0.747314in}{1.110098in}}%
\pgfpathlineto{\pgfqpoint{0.747314in}{0.978249in}}%
\pgfpathlineto{\pgfqpoint{0.786441in}{0.978249in}}%
\pgfpathlineto{\pgfqpoint{0.786441in}{0.899987in}}%
\pgfpathlineto{\pgfqpoint{0.825568in}{0.899987in}}%
\pgfpathlineto{\pgfqpoint{0.825568in}{0.839797in}}%
\pgfpathlineto{\pgfqpoint{0.864695in}{0.839797in}}%
\pgfpathlineto{\pgfqpoint{0.864695in}{0.779896in}}%
\pgfpathlineto{\pgfqpoint{0.903821in}{0.779896in}}%
\pgfpathlineto{\pgfqpoint{0.903821in}{0.721886in}}%
\pgfpathlineto{\pgfqpoint{0.942948in}{0.721886in}}%
\pgfpathlineto{\pgfqpoint{0.942948in}{0.666026in}}%
\pgfpathlineto{\pgfqpoint{0.982075in}{0.666026in}}%
\pgfpathlineto{\pgfqpoint{0.982075in}{0.622199in}}%
\pgfpathlineto{\pgfqpoint{1.021202in}{0.622199in}}%
\pgfpathlineto{\pgfqpoint{1.021202in}{0.585614in}}%
\pgfpathlineto{\pgfqpoint{1.060328in}{0.585614in}}%
\pgfpathlineto{\pgfqpoint{1.060328in}{0.553117in}}%
\pgfpathlineto{\pgfqpoint{1.099455in}{0.553117in}}%
\pgfpathlineto{\pgfqpoint{1.099455in}{0.531295in}}%
\pgfpathlineto{\pgfqpoint{1.138582in}{0.531295in}}%
\pgfpathlineto{\pgfqpoint{1.138582in}{0.513041in}}%
\pgfpathlineto{\pgfqpoint{1.177709in}{0.513041in}}%
\pgfpathlineto{\pgfqpoint{1.177709in}{0.499621in}}%
\pgfpathlineto{\pgfqpoint{1.216836in}{0.499621in}}%
\pgfpathlineto{\pgfqpoint{1.216836in}{0.489663in}}%
\pgfpathlineto{\pgfqpoint{1.255962in}{0.489663in}}%
\pgfpathlineto{\pgfqpoint{1.255962in}{0.480041in}}%
\pgfpathlineto{\pgfqpoint{1.295089in}{0.480041in}}%
\pgfpathlineto{\pgfqpoint{1.295089in}{0.474383in}}%
\pgfpathlineto{\pgfqpoint{1.334216in}{0.474383in}}%
\pgfpathlineto{\pgfqpoint{1.334216in}{0.468649in}}%
\pgfpathlineto{\pgfqpoint{1.373343in}{0.468649in}}%
\pgfpathlineto{\pgfqpoint{1.373343in}{0.465004in}}%
\pgfpathlineto{\pgfqpoint{1.412469in}{0.465004in}}%
\pgfpathlineto{\pgfqpoint{1.412469in}{0.461085in}}%
\pgfpathlineto{\pgfqpoint{1.451596in}{0.461085in}}%
\pgfpathlineto{\pgfqpoint{1.451596in}{0.459057in}}%
\pgfpathlineto{\pgfqpoint{1.490723in}{0.459057in}}%
\pgfpathlineto{\pgfqpoint{1.490723in}{0.456602in}}%
\pgfpathlineto{\pgfqpoint{1.529850in}{0.456602in}}%
\pgfpathlineto{\pgfqpoint{1.529850in}{0.455046in}}%
\pgfpathlineto{\pgfqpoint{1.568976in}{0.455046in}}%
\pgfpathlineto{\pgfqpoint{1.568976in}{0.453552in}}%
\pgfpathlineto{\pgfqpoint{1.608103in}{0.453552in}}%
\pgfpathlineto{\pgfqpoint{1.608103in}{0.453033in}}%
\pgfpathlineto{\pgfqpoint{1.647230in}{0.453033in}}%
\pgfpathlineto{\pgfqpoint{1.647230in}{0.452256in}}%
\pgfpathlineto{\pgfqpoint{1.686357in}{0.452256in}}%
\pgfpathlineto{\pgfqpoint{1.686357in}{0.451432in}}%
\pgfpathlineto{\pgfqpoint{1.725483in}{0.451432in}}%
\pgfpathlineto{\pgfqpoint{1.725483in}{0.450837in}}%
\pgfpathlineto{\pgfqpoint{1.764610in}{0.450837in}}%
\pgfpathlineto{\pgfqpoint{1.764610in}{0.450624in}}%
\pgfpathlineto{\pgfqpoint{1.803737in}{0.450624in}}%
\pgfpathlineto{\pgfqpoint{1.803737in}{0.450532in}}%
\pgfpathlineto{\pgfqpoint{1.842864in}{0.450532in}}%
\pgfpathlineto{\pgfqpoint{1.842864in}{0.450410in}}%
\pgfpathlineto{\pgfqpoint{1.881990in}{0.450410in}}%
\pgfpathlineto{\pgfqpoint{1.881990in}{0.450365in}}%
\pgfpathlineto{\pgfqpoint{1.921117in}{0.450365in}}%
\pgfpathlineto{\pgfqpoint{1.921117in}{0.450304in}}%
\pgfpathlineto{\pgfqpoint{1.960244in}{0.450304in}}%
\pgfpathlineto{\pgfqpoint{1.960244in}{0.450090in}}%
\pgfpathlineto{\pgfqpoint{1.999371in}{0.450090in}}%
\pgfpathlineto{\pgfqpoint{1.999371in}{0.450136in}}%
\pgfpathlineto{\pgfqpoint{2.038498in}{0.450136in}}%
\pgfpathlineto{\pgfqpoint{2.038498in}{0.450090in}}%
\pgfpathlineto{\pgfqpoint{2.077624in}{0.450090in}}%
\pgfpathlineto{\pgfqpoint{2.077624in}{0.450014in}}%
\pgfpathlineto{\pgfqpoint{2.116751in}{0.450014in}}%
\pgfpathlineto{\pgfqpoint{2.116751in}{0.450060in}}%
\pgfpathlineto{\pgfqpoint{2.155878in}{0.450060in}}%
\pgfpathlineto{\pgfqpoint{2.155878in}{0.450044in}}%
\pgfpathlineto{\pgfqpoint{2.195005in}{0.450044in}}%
\pgfpathlineto{\pgfqpoint{2.195005in}{0.449999in}}%
\pgfpathlineto{\pgfqpoint{2.234131in}{0.449999in}}%
\pgfpathlineto{\pgfqpoint{2.234131in}{0.450014in}}%
\pgfpathlineto{\pgfqpoint{2.273258in}{0.450014in}}%
\pgfpathlineto{\pgfqpoint{2.273258in}{0.449983in}}%
\pgfpathlineto{\pgfqpoint{2.312385in}{0.449983in}}%
\pgfpathlineto{\pgfqpoint{2.312385in}{0.449983in}}%
\pgfpathlineto{\pgfqpoint{2.351512in}{0.449983in}}%
\pgfpathlineto{\pgfqpoint{2.351512in}{0.449999in}}%
\pgfpathlineto{\pgfqpoint{2.390638in}{0.449999in}}%
\pgfpathlineto{\pgfqpoint{2.390638in}{0.449983in}}%
\pgfusepath{stroke}%
\end{pgfscope}%
\begin{pgfscope}%
\pgfsetrectcap%
\pgfsetmiterjoin%
\pgfsetlinewidth{1.003750pt}%
\definecolor{currentstroke}{rgb}{0.000000,0.000000,0.000000}%
\pgfsetstrokecolor{currentstroke}%
\pgfsetdash{}{0pt}%
\pgfpathmoveto{\pgfqpoint{0.366840in}{1.615583in}}%
\pgfpathlineto{\pgfqpoint{2.525558in}{1.615583in}}%
\pgfusepath{stroke}%
\end{pgfscope}%
\begin{pgfscope}%
\pgfsetrectcap%
\pgfsetmiterjoin%
\pgfsetlinewidth{1.003750pt}%
\definecolor{currentstroke}{rgb}{0.000000,0.000000,0.000000}%
\pgfsetstrokecolor{currentstroke}%
\pgfsetdash{}{0pt}%
\pgfpathmoveto{\pgfqpoint{2.525558in}{0.449983in}}%
\pgfpathlineto{\pgfqpoint{2.525558in}{1.615583in}}%
\pgfusepath{stroke}%
\end{pgfscope}%
\begin{pgfscope}%
\pgfsetrectcap%
\pgfsetmiterjoin%
\pgfsetlinewidth{1.003750pt}%
\definecolor{currentstroke}{rgb}{0.000000,0.000000,0.000000}%
\pgfsetstrokecolor{currentstroke}%
\pgfsetdash{}{0pt}%
\pgfpathmoveto{\pgfqpoint{0.366840in}{0.449983in}}%
\pgfpathlineto{\pgfqpoint{2.525558in}{0.449983in}}%
\pgfusepath{stroke}%
\end{pgfscope}%
\begin{pgfscope}%
\pgfsetrectcap%
\pgfsetmiterjoin%
\pgfsetlinewidth{1.003750pt}%
\definecolor{currentstroke}{rgb}{0.000000,0.000000,0.000000}%
\pgfsetstrokecolor{currentstroke}%
\pgfsetdash{}{0pt}%
\pgfpathmoveto{\pgfqpoint{0.366840in}{0.449983in}}%
\pgfpathlineto{\pgfqpoint{0.366840in}{1.615583in}}%
\pgfusepath{stroke}%
\end{pgfscope}%
\begin{pgfscope}%
\pgfsetbuttcap%
\pgfsetroundjoin%
\definecolor{currentfill}{rgb}{0.000000,0.000000,0.000000}%
\pgfsetfillcolor{currentfill}%
\pgfsetlinewidth{0.501875pt}%
\definecolor{currentstroke}{rgb}{0.000000,0.000000,0.000000}%
\pgfsetstrokecolor{currentstroke}%
\pgfsetdash{}{0pt}%
\pgfsys@defobject{currentmarker}{\pgfqpoint{0.000000in}{0.000000in}}{\pgfqpoint{0.000000in}{0.069444in}}{%
\pgfpathmoveto{\pgfqpoint{0.000000in}{0.000000in}}%
\pgfpathlineto{\pgfqpoint{0.000000in}{0.069444in}}%
\pgfusepath{stroke,fill}%
}%
\begin{pgfscope}%
\pgfsys@transformshift{0.366840in}{0.449983in}%
\pgfsys@useobject{currentmarker}{}%
\end{pgfscope}%
\end{pgfscope}%
\begin{pgfscope}%
\pgfsetbuttcap%
\pgfsetroundjoin%
\definecolor{currentfill}{rgb}{0.000000,0.000000,0.000000}%
\pgfsetfillcolor{currentfill}%
\pgfsetlinewidth{0.501875pt}%
\definecolor{currentstroke}{rgb}{0.000000,0.000000,0.000000}%
\pgfsetstrokecolor{currentstroke}%
\pgfsetdash{}{0pt}%
\pgfsys@defobject{currentmarker}{\pgfqpoint{0.000000in}{-0.069444in}}{\pgfqpoint{0.000000in}{0.000000in}}{%
\pgfpathmoveto{\pgfqpoint{0.000000in}{0.000000in}}%
\pgfpathlineto{\pgfqpoint{0.000000in}{-0.069444in}}%
\pgfusepath{stroke,fill}%
}%
\begin{pgfscope}%
\pgfsys@transformshift{0.366840in}{1.615583in}%
\pgfsys@useobject{currentmarker}{}%
\end{pgfscope}%
\end{pgfscope}%
\begin{pgfscope}%
\pgftext[x=0.366840in,y=0.380539in,,top]{\rmfamily\fontsize{8.000000}{9.600000}\selectfont −20}%
\end{pgfscope}%
\begin{pgfscope}%
\pgfsetbuttcap%
\pgfsetroundjoin%
\definecolor{currentfill}{rgb}{0.000000,0.000000,0.000000}%
\pgfsetfillcolor{currentfill}%
\pgfsetlinewidth{0.501875pt}%
\definecolor{currentstroke}{rgb}{0.000000,0.000000,0.000000}%
\pgfsetstrokecolor{currentstroke}%
\pgfsetdash{}{0pt}%
\pgfsys@defobject{currentmarker}{\pgfqpoint{0.000000in}{0.000000in}}{\pgfqpoint{0.000000in}{0.069444in}}{%
\pgfpathmoveto{\pgfqpoint{0.000000in}{0.000000in}}%
\pgfpathlineto{\pgfqpoint{0.000000in}{0.069444in}}%
\pgfusepath{stroke,fill}%
}%
\begin{pgfscope}%
\pgfsys@transformshift{0.636680in}{0.449983in}%
\pgfsys@useobject{currentmarker}{}%
\end{pgfscope}%
\end{pgfscope}%
\begin{pgfscope}%
\pgfsetbuttcap%
\pgfsetroundjoin%
\definecolor{currentfill}{rgb}{0.000000,0.000000,0.000000}%
\pgfsetfillcolor{currentfill}%
\pgfsetlinewidth{0.501875pt}%
\definecolor{currentstroke}{rgb}{0.000000,0.000000,0.000000}%
\pgfsetstrokecolor{currentstroke}%
\pgfsetdash{}{0pt}%
\pgfsys@defobject{currentmarker}{\pgfqpoint{0.000000in}{-0.069444in}}{\pgfqpoint{0.000000in}{0.000000in}}{%
\pgfpathmoveto{\pgfqpoint{0.000000in}{0.000000in}}%
\pgfpathlineto{\pgfqpoint{0.000000in}{-0.069444in}}%
\pgfusepath{stroke,fill}%
}%
\begin{pgfscope}%
\pgfsys@transformshift{0.636680in}{1.615583in}%
\pgfsys@useobject{currentmarker}{}%
\end{pgfscope}%
\end{pgfscope}%
\begin{pgfscope}%
\pgftext[x=0.636680in,y=0.380539in,,top]{\rmfamily\fontsize{8.000000}{9.600000}\selectfont 0}%
\end{pgfscope}%
\begin{pgfscope}%
\pgfsetbuttcap%
\pgfsetroundjoin%
\definecolor{currentfill}{rgb}{0.000000,0.000000,0.000000}%
\pgfsetfillcolor{currentfill}%
\pgfsetlinewidth{0.501875pt}%
\definecolor{currentstroke}{rgb}{0.000000,0.000000,0.000000}%
\pgfsetstrokecolor{currentstroke}%
\pgfsetdash{}{0pt}%
\pgfsys@defobject{currentmarker}{\pgfqpoint{0.000000in}{0.000000in}}{\pgfqpoint{0.000000in}{0.069444in}}{%
\pgfpathmoveto{\pgfqpoint{0.000000in}{0.000000in}}%
\pgfpathlineto{\pgfqpoint{0.000000in}{0.069444in}}%
\pgfusepath{stroke,fill}%
}%
\begin{pgfscope}%
\pgfsys@transformshift{0.906520in}{0.449983in}%
\pgfsys@useobject{currentmarker}{}%
\end{pgfscope}%
\end{pgfscope}%
\begin{pgfscope}%
\pgfsetbuttcap%
\pgfsetroundjoin%
\definecolor{currentfill}{rgb}{0.000000,0.000000,0.000000}%
\pgfsetfillcolor{currentfill}%
\pgfsetlinewidth{0.501875pt}%
\definecolor{currentstroke}{rgb}{0.000000,0.000000,0.000000}%
\pgfsetstrokecolor{currentstroke}%
\pgfsetdash{}{0pt}%
\pgfsys@defobject{currentmarker}{\pgfqpoint{0.000000in}{-0.069444in}}{\pgfqpoint{0.000000in}{0.000000in}}{%
\pgfpathmoveto{\pgfqpoint{0.000000in}{0.000000in}}%
\pgfpathlineto{\pgfqpoint{0.000000in}{-0.069444in}}%
\pgfusepath{stroke,fill}%
}%
\begin{pgfscope}%
\pgfsys@transformshift{0.906520in}{1.615583in}%
\pgfsys@useobject{currentmarker}{}%
\end{pgfscope}%
\end{pgfscope}%
\begin{pgfscope}%
\pgftext[x=0.906520in,y=0.380539in,,top]{\rmfamily\fontsize{8.000000}{9.600000}\selectfont 20}%
\end{pgfscope}%
\begin{pgfscope}%
\pgfsetbuttcap%
\pgfsetroundjoin%
\definecolor{currentfill}{rgb}{0.000000,0.000000,0.000000}%
\pgfsetfillcolor{currentfill}%
\pgfsetlinewidth{0.501875pt}%
\definecolor{currentstroke}{rgb}{0.000000,0.000000,0.000000}%
\pgfsetstrokecolor{currentstroke}%
\pgfsetdash{}{0pt}%
\pgfsys@defobject{currentmarker}{\pgfqpoint{0.000000in}{0.000000in}}{\pgfqpoint{0.000000in}{0.069444in}}{%
\pgfpathmoveto{\pgfqpoint{0.000000in}{0.000000in}}%
\pgfpathlineto{\pgfqpoint{0.000000in}{0.069444in}}%
\pgfusepath{stroke,fill}%
}%
\begin{pgfscope}%
\pgfsys@transformshift{1.176360in}{0.449983in}%
\pgfsys@useobject{currentmarker}{}%
\end{pgfscope}%
\end{pgfscope}%
\begin{pgfscope}%
\pgfsetbuttcap%
\pgfsetroundjoin%
\definecolor{currentfill}{rgb}{0.000000,0.000000,0.000000}%
\pgfsetfillcolor{currentfill}%
\pgfsetlinewidth{0.501875pt}%
\definecolor{currentstroke}{rgb}{0.000000,0.000000,0.000000}%
\pgfsetstrokecolor{currentstroke}%
\pgfsetdash{}{0pt}%
\pgfsys@defobject{currentmarker}{\pgfqpoint{0.000000in}{-0.069444in}}{\pgfqpoint{0.000000in}{0.000000in}}{%
\pgfpathmoveto{\pgfqpoint{0.000000in}{0.000000in}}%
\pgfpathlineto{\pgfqpoint{0.000000in}{-0.069444in}}%
\pgfusepath{stroke,fill}%
}%
\begin{pgfscope}%
\pgfsys@transformshift{1.176360in}{1.615583in}%
\pgfsys@useobject{currentmarker}{}%
\end{pgfscope}%
\end{pgfscope}%
\begin{pgfscope}%
\pgftext[x=1.176360in,y=0.380539in,,top]{\rmfamily\fontsize{8.000000}{9.600000}\selectfont 40}%
\end{pgfscope}%
\begin{pgfscope}%
\pgfsetbuttcap%
\pgfsetroundjoin%
\definecolor{currentfill}{rgb}{0.000000,0.000000,0.000000}%
\pgfsetfillcolor{currentfill}%
\pgfsetlinewidth{0.501875pt}%
\definecolor{currentstroke}{rgb}{0.000000,0.000000,0.000000}%
\pgfsetstrokecolor{currentstroke}%
\pgfsetdash{}{0pt}%
\pgfsys@defobject{currentmarker}{\pgfqpoint{0.000000in}{0.000000in}}{\pgfqpoint{0.000000in}{0.069444in}}{%
\pgfpathmoveto{\pgfqpoint{0.000000in}{0.000000in}}%
\pgfpathlineto{\pgfqpoint{0.000000in}{0.069444in}}%
\pgfusepath{stroke,fill}%
}%
\begin{pgfscope}%
\pgfsys@transformshift{1.446199in}{0.449983in}%
\pgfsys@useobject{currentmarker}{}%
\end{pgfscope}%
\end{pgfscope}%
\begin{pgfscope}%
\pgfsetbuttcap%
\pgfsetroundjoin%
\definecolor{currentfill}{rgb}{0.000000,0.000000,0.000000}%
\pgfsetfillcolor{currentfill}%
\pgfsetlinewidth{0.501875pt}%
\definecolor{currentstroke}{rgb}{0.000000,0.000000,0.000000}%
\pgfsetstrokecolor{currentstroke}%
\pgfsetdash{}{0pt}%
\pgfsys@defobject{currentmarker}{\pgfqpoint{0.000000in}{-0.069444in}}{\pgfqpoint{0.000000in}{0.000000in}}{%
\pgfpathmoveto{\pgfqpoint{0.000000in}{0.000000in}}%
\pgfpathlineto{\pgfqpoint{0.000000in}{-0.069444in}}%
\pgfusepath{stroke,fill}%
}%
\begin{pgfscope}%
\pgfsys@transformshift{1.446199in}{1.615583in}%
\pgfsys@useobject{currentmarker}{}%
\end{pgfscope}%
\end{pgfscope}%
\begin{pgfscope}%
\pgftext[x=1.446199in,y=0.380539in,,top]{\rmfamily\fontsize{8.000000}{9.600000}\selectfont 60}%
\end{pgfscope}%
\begin{pgfscope}%
\pgfsetbuttcap%
\pgfsetroundjoin%
\definecolor{currentfill}{rgb}{0.000000,0.000000,0.000000}%
\pgfsetfillcolor{currentfill}%
\pgfsetlinewidth{0.501875pt}%
\definecolor{currentstroke}{rgb}{0.000000,0.000000,0.000000}%
\pgfsetstrokecolor{currentstroke}%
\pgfsetdash{}{0pt}%
\pgfsys@defobject{currentmarker}{\pgfqpoint{0.000000in}{0.000000in}}{\pgfqpoint{0.000000in}{0.069444in}}{%
\pgfpathmoveto{\pgfqpoint{0.000000in}{0.000000in}}%
\pgfpathlineto{\pgfqpoint{0.000000in}{0.069444in}}%
\pgfusepath{stroke,fill}%
}%
\begin{pgfscope}%
\pgfsys@transformshift{1.716039in}{0.449983in}%
\pgfsys@useobject{currentmarker}{}%
\end{pgfscope}%
\end{pgfscope}%
\begin{pgfscope}%
\pgfsetbuttcap%
\pgfsetroundjoin%
\definecolor{currentfill}{rgb}{0.000000,0.000000,0.000000}%
\pgfsetfillcolor{currentfill}%
\pgfsetlinewidth{0.501875pt}%
\definecolor{currentstroke}{rgb}{0.000000,0.000000,0.000000}%
\pgfsetstrokecolor{currentstroke}%
\pgfsetdash{}{0pt}%
\pgfsys@defobject{currentmarker}{\pgfqpoint{0.000000in}{-0.069444in}}{\pgfqpoint{0.000000in}{0.000000in}}{%
\pgfpathmoveto{\pgfqpoint{0.000000in}{0.000000in}}%
\pgfpathlineto{\pgfqpoint{0.000000in}{-0.069444in}}%
\pgfusepath{stroke,fill}%
}%
\begin{pgfscope}%
\pgfsys@transformshift{1.716039in}{1.615583in}%
\pgfsys@useobject{currentmarker}{}%
\end{pgfscope}%
\end{pgfscope}%
\begin{pgfscope}%
\pgftext[x=1.716039in,y=0.380539in,,top]{\rmfamily\fontsize{8.000000}{9.600000}\selectfont 80}%
\end{pgfscope}%
\begin{pgfscope}%
\pgfsetbuttcap%
\pgfsetroundjoin%
\definecolor{currentfill}{rgb}{0.000000,0.000000,0.000000}%
\pgfsetfillcolor{currentfill}%
\pgfsetlinewidth{0.501875pt}%
\definecolor{currentstroke}{rgb}{0.000000,0.000000,0.000000}%
\pgfsetstrokecolor{currentstroke}%
\pgfsetdash{}{0pt}%
\pgfsys@defobject{currentmarker}{\pgfqpoint{0.000000in}{0.000000in}}{\pgfqpoint{0.000000in}{0.069444in}}{%
\pgfpathmoveto{\pgfqpoint{0.000000in}{0.000000in}}%
\pgfpathlineto{\pgfqpoint{0.000000in}{0.069444in}}%
\pgfusepath{stroke,fill}%
}%
\begin{pgfscope}%
\pgfsys@transformshift{1.985879in}{0.449983in}%
\pgfsys@useobject{currentmarker}{}%
\end{pgfscope}%
\end{pgfscope}%
\begin{pgfscope}%
\pgfsetbuttcap%
\pgfsetroundjoin%
\definecolor{currentfill}{rgb}{0.000000,0.000000,0.000000}%
\pgfsetfillcolor{currentfill}%
\pgfsetlinewidth{0.501875pt}%
\definecolor{currentstroke}{rgb}{0.000000,0.000000,0.000000}%
\pgfsetstrokecolor{currentstroke}%
\pgfsetdash{}{0pt}%
\pgfsys@defobject{currentmarker}{\pgfqpoint{0.000000in}{-0.069444in}}{\pgfqpoint{0.000000in}{0.000000in}}{%
\pgfpathmoveto{\pgfqpoint{0.000000in}{0.000000in}}%
\pgfpathlineto{\pgfqpoint{0.000000in}{-0.069444in}}%
\pgfusepath{stroke,fill}%
}%
\begin{pgfscope}%
\pgfsys@transformshift{1.985879in}{1.615583in}%
\pgfsys@useobject{currentmarker}{}%
\end{pgfscope}%
\end{pgfscope}%
\begin{pgfscope}%
\pgftext[x=1.985879in,y=0.380539in,,top]{\rmfamily\fontsize{8.000000}{9.600000}\selectfont 100}%
\end{pgfscope}%
\begin{pgfscope}%
\pgfsetbuttcap%
\pgfsetroundjoin%
\definecolor{currentfill}{rgb}{0.000000,0.000000,0.000000}%
\pgfsetfillcolor{currentfill}%
\pgfsetlinewidth{0.501875pt}%
\definecolor{currentstroke}{rgb}{0.000000,0.000000,0.000000}%
\pgfsetstrokecolor{currentstroke}%
\pgfsetdash{}{0pt}%
\pgfsys@defobject{currentmarker}{\pgfqpoint{0.000000in}{0.000000in}}{\pgfqpoint{0.000000in}{0.069444in}}{%
\pgfpathmoveto{\pgfqpoint{0.000000in}{0.000000in}}%
\pgfpathlineto{\pgfqpoint{0.000000in}{0.069444in}}%
\pgfusepath{stroke,fill}%
}%
\begin{pgfscope}%
\pgfsys@transformshift{2.255719in}{0.449983in}%
\pgfsys@useobject{currentmarker}{}%
\end{pgfscope}%
\end{pgfscope}%
\begin{pgfscope}%
\pgfsetbuttcap%
\pgfsetroundjoin%
\definecolor{currentfill}{rgb}{0.000000,0.000000,0.000000}%
\pgfsetfillcolor{currentfill}%
\pgfsetlinewidth{0.501875pt}%
\definecolor{currentstroke}{rgb}{0.000000,0.000000,0.000000}%
\pgfsetstrokecolor{currentstroke}%
\pgfsetdash{}{0pt}%
\pgfsys@defobject{currentmarker}{\pgfqpoint{0.000000in}{-0.069444in}}{\pgfqpoint{0.000000in}{0.000000in}}{%
\pgfpathmoveto{\pgfqpoint{0.000000in}{0.000000in}}%
\pgfpathlineto{\pgfqpoint{0.000000in}{-0.069444in}}%
\pgfusepath{stroke,fill}%
}%
\begin{pgfscope}%
\pgfsys@transformshift{2.255719in}{1.615583in}%
\pgfsys@useobject{currentmarker}{}%
\end{pgfscope}%
\end{pgfscope}%
\begin{pgfscope}%
\pgftext[x=2.255719in,y=0.380539in,,top]{\rmfamily\fontsize{8.000000}{9.600000}\selectfont 120}%
\end{pgfscope}%
\begin{pgfscope}%
\pgfsetbuttcap%
\pgfsetroundjoin%
\definecolor{currentfill}{rgb}{0.000000,0.000000,0.000000}%
\pgfsetfillcolor{currentfill}%
\pgfsetlinewidth{0.501875pt}%
\definecolor{currentstroke}{rgb}{0.000000,0.000000,0.000000}%
\pgfsetstrokecolor{currentstroke}%
\pgfsetdash{}{0pt}%
\pgfsys@defobject{currentmarker}{\pgfqpoint{0.000000in}{0.000000in}}{\pgfqpoint{0.000000in}{0.069444in}}{%
\pgfpathmoveto{\pgfqpoint{0.000000in}{0.000000in}}%
\pgfpathlineto{\pgfqpoint{0.000000in}{0.069444in}}%
\pgfusepath{stroke,fill}%
}%
\begin{pgfscope}%
\pgfsys@transformshift{2.525558in}{0.449983in}%
\pgfsys@useobject{currentmarker}{}%
\end{pgfscope}%
\end{pgfscope}%
\begin{pgfscope}%
\pgfsetbuttcap%
\pgfsetroundjoin%
\definecolor{currentfill}{rgb}{0.000000,0.000000,0.000000}%
\pgfsetfillcolor{currentfill}%
\pgfsetlinewidth{0.501875pt}%
\definecolor{currentstroke}{rgb}{0.000000,0.000000,0.000000}%
\pgfsetstrokecolor{currentstroke}%
\pgfsetdash{}{0pt}%
\pgfsys@defobject{currentmarker}{\pgfqpoint{0.000000in}{-0.069444in}}{\pgfqpoint{0.000000in}{0.000000in}}{%
\pgfpathmoveto{\pgfqpoint{0.000000in}{0.000000in}}%
\pgfpathlineto{\pgfqpoint{0.000000in}{-0.069444in}}%
\pgfusepath{stroke,fill}%
}%
\begin{pgfscope}%
\pgfsys@transformshift{2.525558in}{1.615583in}%
\pgfsys@useobject{currentmarker}{}%
\end{pgfscope}%
\end{pgfscope}%
\begin{pgfscope}%
\pgftext[x=2.525558in,y=0.380539in,,top]{\rmfamily\fontsize{8.000000}{9.600000}\selectfont 140}%
\end{pgfscope}%
\begin{pgfscope}%
\pgftext[x=1.446199in,y=0.203564in,,top]{\rmfamily\fontsize{9.000000}{10.800000}\selectfont \(\displaystyle \mathrm{DLL}_{K/\pi}(K^+)\)}%
\end{pgfscope}%
\begin{pgfscope}%
\pgfsetbuttcap%
\pgfsetroundjoin%
\definecolor{currentfill}{rgb}{0.000000,0.000000,0.000000}%
\pgfsetfillcolor{currentfill}%
\pgfsetlinewidth{0.501875pt}%
\definecolor{currentstroke}{rgb}{0.000000,0.000000,0.000000}%
\pgfsetstrokecolor{currentstroke}%
\pgfsetdash{}{0pt}%
\pgfsys@defobject{currentmarker}{\pgfqpoint{0.000000in}{0.000000in}}{\pgfqpoint{0.069444in}{0.000000in}}{%
\pgfpathmoveto{\pgfqpoint{0.000000in}{0.000000in}}%
\pgfpathlineto{\pgfqpoint{0.069444in}{0.000000in}}%
\pgfusepath{stroke,fill}%
}%
\begin{pgfscope}%
\pgfsys@transformshift{0.366840in}{0.449983in}%
\pgfsys@useobject{currentmarker}{}%
\end{pgfscope}%
\end{pgfscope}%
\begin{pgfscope}%
\pgfsetbuttcap%
\pgfsetroundjoin%
\definecolor{currentfill}{rgb}{0.000000,0.000000,0.000000}%
\pgfsetfillcolor{currentfill}%
\pgfsetlinewidth{0.501875pt}%
\definecolor{currentstroke}{rgb}{0.000000,0.000000,0.000000}%
\pgfsetstrokecolor{currentstroke}%
\pgfsetdash{}{0pt}%
\pgfsys@defobject{currentmarker}{\pgfqpoint{-0.069444in}{0.000000in}}{\pgfqpoint{0.000000in}{0.000000in}}{%
\pgfpathmoveto{\pgfqpoint{0.000000in}{0.000000in}}%
\pgfpathlineto{\pgfqpoint{-0.069444in}{0.000000in}}%
\pgfusepath{stroke,fill}%
}%
\begin{pgfscope}%
\pgfsys@transformshift{2.525558in}{0.449983in}%
\pgfsys@useobject{currentmarker}{}%
\end{pgfscope}%
\end{pgfscope}%
\begin{pgfscope}%
\pgftext[x=0.297396in,y=0.449983in,right,]{\rmfamily\fontsize{8.000000}{9.600000}\selectfont 0.00}%
\end{pgfscope}%
\begin{pgfscope}%
\pgfsetbuttcap%
\pgfsetroundjoin%
\definecolor{currentfill}{rgb}{0.000000,0.000000,0.000000}%
\pgfsetfillcolor{currentfill}%
\pgfsetlinewidth{0.501875pt}%
\definecolor{currentstroke}{rgb}{0.000000,0.000000,0.000000}%
\pgfsetstrokecolor{currentstroke}%
\pgfsetdash{}{0pt}%
\pgfsys@defobject{currentmarker}{\pgfqpoint{0.000000in}{0.000000in}}{\pgfqpoint{0.069444in}{0.000000in}}{%
\pgfpathmoveto{\pgfqpoint{0.000000in}{0.000000in}}%
\pgfpathlineto{\pgfqpoint{0.069444in}{0.000000in}}%
\pgfusepath{stroke,fill}%
}%
\begin{pgfscope}%
\pgfsys@transformshift{0.366840in}{0.644250in}%
\pgfsys@useobject{currentmarker}{}%
\end{pgfscope}%
\end{pgfscope}%
\begin{pgfscope}%
\pgfsetbuttcap%
\pgfsetroundjoin%
\definecolor{currentfill}{rgb}{0.000000,0.000000,0.000000}%
\pgfsetfillcolor{currentfill}%
\pgfsetlinewidth{0.501875pt}%
\definecolor{currentstroke}{rgb}{0.000000,0.000000,0.000000}%
\pgfsetstrokecolor{currentstroke}%
\pgfsetdash{}{0pt}%
\pgfsys@defobject{currentmarker}{\pgfqpoint{-0.069444in}{0.000000in}}{\pgfqpoint{0.000000in}{0.000000in}}{%
\pgfpathmoveto{\pgfqpoint{0.000000in}{0.000000in}}%
\pgfpathlineto{\pgfqpoint{-0.069444in}{0.000000in}}%
\pgfusepath{stroke,fill}%
}%
\begin{pgfscope}%
\pgfsys@transformshift{2.525558in}{0.644250in}%
\pgfsys@useobject{currentmarker}{}%
\end{pgfscope}%
\end{pgfscope}%
\begin{pgfscope}%
\pgftext[x=0.297396in,y=0.644250in,right,]{\rmfamily\fontsize{8.000000}{9.600000}\selectfont 0.01}%
\end{pgfscope}%
\begin{pgfscope}%
\pgfsetbuttcap%
\pgfsetroundjoin%
\definecolor{currentfill}{rgb}{0.000000,0.000000,0.000000}%
\pgfsetfillcolor{currentfill}%
\pgfsetlinewidth{0.501875pt}%
\definecolor{currentstroke}{rgb}{0.000000,0.000000,0.000000}%
\pgfsetstrokecolor{currentstroke}%
\pgfsetdash{}{0pt}%
\pgfsys@defobject{currentmarker}{\pgfqpoint{0.000000in}{0.000000in}}{\pgfqpoint{0.069444in}{0.000000in}}{%
\pgfpathmoveto{\pgfqpoint{0.000000in}{0.000000in}}%
\pgfpathlineto{\pgfqpoint{0.069444in}{0.000000in}}%
\pgfusepath{stroke,fill}%
}%
\begin{pgfscope}%
\pgfsys@transformshift{0.366840in}{0.838517in}%
\pgfsys@useobject{currentmarker}{}%
\end{pgfscope}%
\end{pgfscope}%
\begin{pgfscope}%
\pgfsetbuttcap%
\pgfsetroundjoin%
\definecolor{currentfill}{rgb}{0.000000,0.000000,0.000000}%
\pgfsetfillcolor{currentfill}%
\pgfsetlinewidth{0.501875pt}%
\definecolor{currentstroke}{rgb}{0.000000,0.000000,0.000000}%
\pgfsetstrokecolor{currentstroke}%
\pgfsetdash{}{0pt}%
\pgfsys@defobject{currentmarker}{\pgfqpoint{-0.069444in}{0.000000in}}{\pgfqpoint{0.000000in}{0.000000in}}{%
\pgfpathmoveto{\pgfqpoint{0.000000in}{0.000000in}}%
\pgfpathlineto{\pgfqpoint{-0.069444in}{0.000000in}}%
\pgfusepath{stroke,fill}%
}%
\begin{pgfscope}%
\pgfsys@transformshift{2.525558in}{0.838517in}%
\pgfsys@useobject{currentmarker}{}%
\end{pgfscope}%
\end{pgfscope}%
\begin{pgfscope}%
\pgftext[x=0.297396in,y=0.838517in,right,]{\rmfamily\fontsize{8.000000}{9.600000}\selectfont 0.02}%
\end{pgfscope}%
\begin{pgfscope}%
\pgfsetbuttcap%
\pgfsetroundjoin%
\definecolor{currentfill}{rgb}{0.000000,0.000000,0.000000}%
\pgfsetfillcolor{currentfill}%
\pgfsetlinewidth{0.501875pt}%
\definecolor{currentstroke}{rgb}{0.000000,0.000000,0.000000}%
\pgfsetstrokecolor{currentstroke}%
\pgfsetdash{}{0pt}%
\pgfsys@defobject{currentmarker}{\pgfqpoint{0.000000in}{0.000000in}}{\pgfqpoint{0.069444in}{0.000000in}}{%
\pgfpathmoveto{\pgfqpoint{0.000000in}{0.000000in}}%
\pgfpathlineto{\pgfqpoint{0.069444in}{0.000000in}}%
\pgfusepath{stroke,fill}%
}%
\begin{pgfscope}%
\pgfsys@transformshift{0.366840in}{1.032783in}%
\pgfsys@useobject{currentmarker}{}%
\end{pgfscope}%
\end{pgfscope}%
\begin{pgfscope}%
\pgfsetbuttcap%
\pgfsetroundjoin%
\definecolor{currentfill}{rgb}{0.000000,0.000000,0.000000}%
\pgfsetfillcolor{currentfill}%
\pgfsetlinewidth{0.501875pt}%
\definecolor{currentstroke}{rgb}{0.000000,0.000000,0.000000}%
\pgfsetstrokecolor{currentstroke}%
\pgfsetdash{}{0pt}%
\pgfsys@defobject{currentmarker}{\pgfqpoint{-0.069444in}{0.000000in}}{\pgfqpoint{0.000000in}{0.000000in}}{%
\pgfpathmoveto{\pgfqpoint{0.000000in}{0.000000in}}%
\pgfpathlineto{\pgfqpoint{-0.069444in}{0.000000in}}%
\pgfusepath{stroke,fill}%
}%
\begin{pgfscope}%
\pgfsys@transformshift{2.525558in}{1.032783in}%
\pgfsys@useobject{currentmarker}{}%
\end{pgfscope}%
\end{pgfscope}%
\begin{pgfscope}%
\pgftext[x=0.297396in,y=1.032783in,right,]{\rmfamily\fontsize{8.000000}{9.600000}\selectfont 0.03}%
\end{pgfscope}%
\begin{pgfscope}%
\pgfsetbuttcap%
\pgfsetroundjoin%
\definecolor{currentfill}{rgb}{0.000000,0.000000,0.000000}%
\pgfsetfillcolor{currentfill}%
\pgfsetlinewidth{0.501875pt}%
\definecolor{currentstroke}{rgb}{0.000000,0.000000,0.000000}%
\pgfsetstrokecolor{currentstroke}%
\pgfsetdash{}{0pt}%
\pgfsys@defobject{currentmarker}{\pgfqpoint{0.000000in}{0.000000in}}{\pgfqpoint{0.069444in}{0.000000in}}{%
\pgfpathmoveto{\pgfqpoint{0.000000in}{0.000000in}}%
\pgfpathlineto{\pgfqpoint{0.069444in}{0.000000in}}%
\pgfusepath{stroke,fill}%
}%
\begin{pgfscope}%
\pgfsys@transformshift{0.366840in}{1.227050in}%
\pgfsys@useobject{currentmarker}{}%
\end{pgfscope}%
\end{pgfscope}%
\begin{pgfscope}%
\pgfsetbuttcap%
\pgfsetroundjoin%
\definecolor{currentfill}{rgb}{0.000000,0.000000,0.000000}%
\pgfsetfillcolor{currentfill}%
\pgfsetlinewidth{0.501875pt}%
\definecolor{currentstroke}{rgb}{0.000000,0.000000,0.000000}%
\pgfsetstrokecolor{currentstroke}%
\pgfsetdash{}{0pt}%
\pgfsys@defobject{currentmarker}{\pgfqpoint{-0.069444in}{0.000000in}}{\pgfqpoint{0.000000in}{0.000000in}}{%
\pgfpathmoveto{\pgfqpoint{0.000000in}{0.000000in}}%
\pgfpathlineto{\pgfqpoint{-0.069444in}{0.000000in}}%
\pgfusepath{stroke,fill}%
}%
\begin{pgfscope}%
\pgfsys@transformshift{2.525558in}{1.227050in}%
\pgfsys@useobject{currentmarker}{}%
\end{pgfscope}%
\end{pgfscope}%
\begin{pgfscope}%
\pgftext[x=0.297396in,y=1.227050in,right,]{\rmfamily\fontsize{8.000000}{9.600000}\selectfont 0.04}%
\end{pgfscope}%
\begin{pgfscope}%
\pgfsetbuttcap%
\pgfsetroundjoin%
\definecolor{currentfill}{rgb}{0.000000,0.000000,0.000000}%
\pgfsetfillcolor{currentfill}%
\pgfsetlinewidth{0.501875pt}%
\definecolor{currentstroke}{rgb}{0.000000,0.000000,0.000000}%
\pgfsetstrokecolor{currentstroke}%
\pgfsetdash{}{0pt}%
\pgfsys@defobject{currentmarker}{\pgfqpoint{0.000000in}{0.000000in}}{\pgfqpoint{0.069444in}{0.000000in}}{%
\pgfpathmoveto{\pgfqpoint{0.000000in}{0.000000in}}%
\pgfpathlineto{\pgfqpoint{0.069444in}{0.000000in}}%
\pgfusepath{stroke,fill}%
}%
\begin{pgfscope}%
\pgfsys@transformshift{0.366840in}{1.421317in}%
\pgfsys@useobject{currentmarker}{}%
\end{pgfscope}%
\end{pgfscope}%
\begin{pgfscope}%
\pgfsetbuttcap%
\pgfsetroundjoin%
\definecolor{currentfill}{rgb}{0.000000,0.000000,0.000000}%
\pgfsetfillcolor{currentfill}%
\pgfsetlinewidth{0.501875pt}%
\definecolor{currentstroke}{rgb}{0.000000,0.000000,0.000000}%
\pgfsetstrokecolor{currentstroke}%
\pgfsetdash{}{0pt}%
\pgfsys@defobject{currentmarker}{\pgfqpoint{-0.069444in}{0.000000in}}{\pgfqpoint{0.000000in}{0.000000in}}{%
\pgfpathmoveto{\pgfqpoint{0.000000in}{0.000000in}}%
\pgfpathlineto{\pgfqpoint{-0.069444in}{0.000000in}}%
\pgfusepath{stroke,fill}%
}%
\begin{pgfscope}%
\pgfsys@transformshift{2.525558in}{1.421317in}%
\pgfsys@useobject{currentmarker}{}%
\end{pgfscope}%
\end{pgfscope}%
\begin{pgfscope}%
\pgftext[x=0.297396in,y=1.421317in,right,]{\rmfamily\fontsize{8.000000}{9.600000}\selectfont 0.05}%
\end{pgfscope}%
\begin{pgfscope}%
\pgfsetbuttcap%
\pgfsetroundjoin%
\definecolor{currentfill}{rgb}{0.000000,0.000000,0.000000}%
\pgfsetfillcolor{currentfill}%
\pgfsetlinewidth{0.501875pt}%
\definecolor{currentstroke}{rgb}{0.000000,0.000000,0.000000}%
\pgfsetstrokecolor{currentstroke}%
\pgfsetdash{}{0pt}%
\pgfsys@defobject{currentmarker}{\pgfqpoint{0.000000in}{0.000000in}}{\pgfqpoint{0.069444in}{0.000000in}}{%
\pgfpathmoveto{\pgfqpoint{0.000000in}{0.000000in}}%
\pgfpathlineto{\pgfqpoint{0.069444in}{0.000000in}}%
\pgfusepath{stroke,fill}%
}%
\begin{pgfscope}%
\pgfsys@transformshift{0.366840in}{1.615583in}%
\pgfsys@useobject{currentmarker}{}%
\end{pgfscope}%
\end{pgfscope}%
\begin{pgfscope}%
\pgfsetbuttcap%
\pgfsetroundjoin%
\definecolor{currentfill}{rgb}{0.000000,0.000000,0.000000}%
\pgfsetfillcolor{currentfill}%
\pgfsetlinewidth{0.501875pt}%
\definecolor{currentstroke}{rgb}{0.000000,0.000000,0.000000}%
\pgfsetstrokecolor{currentstroke}%
\pgfsetdash{}{0pt}%
\pgfsys@defobject{currentmarker}{\pgfqpoint{-0.069444in}{0.000000in}}{\pgfqpoint{0.000000in}{0.000000in}}{%
\pgfpathmoveto{\pgfqpoint{0.000000in}{0.000000in}}%
\pgfpathlineto{\pgfqpoint{-0.069444in}{0.000000in}}%
\pgfusepath{stroke,fill}%
}%
\begin{pgfscope}%
\pgfsys@transformshift{2.525558in}{1.615583in}%
\pgfsys@useobject{currentmarker}{}%
\end{pgfscope}%
\end{pgfscope}%
\begin{pgfscope}%
\pgftext[x=0.297396in,y=1.615583in,right,]{\rmfamily\fontsize{8.000000}{9.600000}\selectfont 0.06}%
\end{pgfscope}%
\end{pgfpicture}%
\makeatother%
\endgroup%

	\end{subfigure}
	\begin{subfigure}[t]{0.49\textwidth}
		\centering
    %\includegraphics[width=\textwidth]{store/variables/SIG_BKG_Kplus_PIDmu.pdf}
    %% Creator: Matplotlib, PGF backend
%%
%% To include the figure in your LaTeX document, write
%%   \input{<filename>.pgf}
%%
%% Make sure the required packages are loaded in your preamble
%%   \usepackage{pgf}
%%
%% Figures using additional raster images can only be included by \input if
%% they are in the same directory as the main LaTeX file. For loading figures
%% from other directories you can use the `import` package
%%   \usepackage{import}
%% and then include the figures with
%%   \import{<path to file>}{<filename>.pgf}
%%
%% Matplotlib used the following preamble
%%   \usepackage{fontspec}
%%   \setmainfont{DejaVu Serif}
%%   \setsansfont{DejaVu Sans}
%%   \setmonofont{DejaVu Sans Mono}
%%
\begingroup%
\makeatletter%
\begin{pgfpicture}%
\pgfpathrectangle{\pgfpointorigin}{\pgfqpoint{2.682342in}{1.719349in}}%
\pgfusepath{use as bounding box, clip}%
\begin{pgfscope}%
\pgfsetbuttcap%
\pgfsetmiterjoin%
\definecolor{currentfill}{rgb}{1.000000,1.000000,1.000000}%
\pgfsetfillcolor{currentfill}%
\pgfsetlinewidth{0.000000pt}%
\definecolor{currentstroke}{rgb}{1.000000,1.000000,1.000000}%
\pgfsetstrokecolor{currentstroke}%
\pgfsetdash{}{0pt}%
\pgfpathmoveto{\pgfqpoint{0.000000in}{0.000000in}}%
\pgfpathlineto{\pgfqpoint{2.682342in}{0.000000in}}%
\pgfpathlineto{\pgfqpoint{2.682342in}{1.719349in}}%
\pgfpathlineto{\pgfqpoint{0.000000in}{1.719349in}}%
\pgfpathclose%
\pgfusepath{fill}%
\end{pgfscope}%
\begin{pgfscope}%
\pgfsetbuttcap%
\pgfsetmiterjoin%
\definecolor{currentfill}{rgb}{1.000000,1.000000,1.000000}%
\pgfsetfillcolor{currentfill}%
\pgfsetlinewidth{0.000000pt}%
\definecolor{currentstroke}{rgb}{0.000000,0.000000,0.000000}%
\pgfsetstrokecolor{currentstroke}%
\pgfsetstrokeopacity{0.000000}%
\pgfsetdash{}{0pt}%
\pgfpathmoveto{\pgfqpoint{0.366840in}{0.449983in}}%
\pgfpathlineto{\pgfqpoint{2.561650in}{0.449983in}}%
\pgfpathlineto{\pgfqpoint{2.561650in}{1.615583in}}%
\pgfpathlineto{\pgfqpoint{0.366840in}{1.615583in}}%
\pgfpathclose%
\pgfusepath{fill}%
\end{pgfscope}%
\begin{pgfscope}%
\pgfpathrectangle{\pgfqpoint{0.366840in}{0.449983in}}{\pgfqpoint{2.194810in}{1.165600in}} %
\pgfusepath{clip}%
\pgfsetbuttcap%
\pgfsetmiterjoin%
\definecolor{currentfill}{rgb}{0.215686,0.470588,0.749020}%
\pgfsetfillcolor{currentfill}%
\pgfsetlinewidth{0.000000pt}%
\definecolor{currentstroke}{rgb}{0.000000,0.000000,0.000000}%
\pgfsetstrokecolor{currentstroke}%
\pgfsetdash{}{0pt}%
\pgfpathmoveto{\pgfqpoint{0.366840in}{0.449983in}}%
\pgfpathlineto{\pgfqpoint{0.366840in}{0.449983in}}%
\pgfpathlineto{\pgfqpoint{0.410736in}{0.449983in}}%
\pgfpathlineto{\pgfqpoint{0.410736in}{0.449983in}}%
\pgfpathlineto{\pgfqpoint{0.454633in}{0.449983in}}%
\pgfpathlineto{\pgfqpoint{0.454633in}{0.449983in}}%
\pgfpathlineto{\pgfqpoint{0.498529in}{0.449983in}}%
\pgfpathlineto{\pgfqpoint{0.498529in}{0.450814in}}%
\pgfpathlineto{\pgfqpoint{0.542425in}{0.450814in}}%
\pgfpathlineto{\pgfqpoint{0.542425in}{0.460681in}}%
\pgfpathlineto{\pgfqpoint{0.586321in}{0.460681in}}%
\pgfpathlineto{\pgfqpoint{0.586321in}{0.471320in}}%
\pgfpathlineto{\pgfqpoint{0.630217in}{0.471320in}}%
\pgfpathlineto{\pgfqpoint{0.630217in}{0.485337in}}%
\pgfpathlineto{\pgfqpoint{0.674114in}{0.485337in}}%
\pgfpathlineto{\pgfqpoint{0.674114in}{0.521827in}}%
\pgfpathlineto{\pgfqpoint{0.718010in}{0.521827in}}%
\pgfpathlineto{\pgfqpoint{0.718010in}{0.544645in}}%
\pgfpathlineto{\pgfqpoint{0.761906in}{0.544645in}}%
\pgfpathlineto{\pgfqpoint{0.761906in}{0.569388in}}%
\pgfpathlineto{\pgfqpoint{0.805802in}{0.569388in}}%
\pgfpathlineto{\pgfqpoint{0.805802in}{0.645880in}}%
\pgfpathlineto{\pgfqpoint{0.849698in}{0.645880in}}%
\pgfpathlineto{\pgfqpoint{0.849698in}{0.761746in}}%
\pgfpathlineto{\pgfqpoint{0.893595in}{0.761746in}}%
\pgfpathlineto{\pgfqpoint{0.893595in}{0.849120in}}%
\pgfpathlineto{\pgfqpoint{0.937491in}{0.849120in}}%
\pgfpathlineto{\pgfqpoint{0.937491in}{1.114872in}}%
\pgfpathlineto{\pgfqpoint{0.981387in}{1.114872in}}%
\pgfpathlineto{\pgfqpoint{0.981387in}{1.132451in}}%
\pgfpathlineto{\pgfqpoint{1.025283in}{1.132451in}}%
\pgfpathlineto{\pgfqpoint{1.025283in}{1.256867in}}%
\pgfpathlineto{\pgfqpoint{1.069179in}{1.256867in}}%
\pgfpathlineto{\pgfqpoint{1.069179in}{1.381010in}}%
\pgfpathlineto{\pgfqpoint{1.113076in}{1.381010in}}%
\pgfpathlineto{\pgfqpoint{1.113076in}{1.491918in}}%
\pgfpathlineto{\pgfqpoint{1.156972in}{1.491918in}}%
\pgfpathlineto{\pgfqpoint{1.156972in}{1.506411in}}%
\pgfpathlineto{\pgfqpoint{1.200868in}{1.506411in}}%
\pgfpathlineto{\pgfqpoint{1.200868in}{1.536868in}}%
\pgfpathlineto{\pgfqpoint{1.244764in}{1.536868in}}%
\pgfpathlineto{\pgfqpoint{1.244764in}{1.480368in}}%
\pgfpathlineto{\pgfqpoint{1.288660in}{1.480368in}}%
\pgfpathlineto{\pgfqpoint{1.288660in}{1.462128in}}%
\pgfpathlineto{\pgfqpoint{1.332556in}{1.462128in}}%
\pgfpathlineto{\pgfqpoint{1.332556in}{1.246544in}}%
\pgfpathlineto{\pgfqpoint{1.376453in}{1.246544in}}%
\pgfpathlineto{\pgfqpoint{1.376453in}{1.098044in}}%
\pgfpathlineto{\pgfqpoint{1.420349in}{1.098044in}}%
\pgfpathlineto{\pgfqpoint{1.420349in}{1.216048in}}%
\pgfpathlineto{\pgfqpoint{1.464245in}{1.216048in}}%
\pgfpathlineto{\pgfqpoint{1.464245in}{1.013062in}}%
\pgfpathlineto{\pgfqpoint{1.508141in}{1.013062in}}%
\pgfpathlineto{\pgfqpoint{1.508141in}{0.805886in}}%
\pgfpathlineto{\pgfqpoint{1.552037in}{0.805886in}}%
\pgfpathlineto{\pgfqpoint{1.552037in}{0.710865in}}%
\pgfpathlineto{\pgfqpoint{1.595934in}{0.710865in}}%
\pgfpathlineto{\pgfqpoint{1.595934in}{0.648720in}}%
\pgfpathlineto{\pgfqpoint{1.639830in}{0.648720in}}%
\pgfpathlineto{\pgfqpoint{1.639830in}{0.598279in}}%
\pgfpathlineto{\pgfqpoint{1.683726in}{0.598279in}}%
\pgfpathlineto{\pgfqpoint{1.683726in}{0.583065in}}%
\pgfpathlineto{\pgfqpoint{1.727622in}{0.583065in}}%
\pgfpathlineto{\pgfqpoint{1.727622in}{0.574189in}}%
\pgfpathlineto{\pgfqpoint{1.771518in}{0.574189in}}%
\pgfpathlineto{\pgfqpoint{1.771518in}{0.545736in}}%
\pgfpathlineto{\pgfqpoint{1.815415in}{0.545736in}}%
\pgfpathlineto{\pgfqpoint{1.815415in}{0.496399in}}%
\pgfpathlineto{\pgfqpoint{1.859311in}{0.496399in}}%
\pgfpathlineto{\pgfqpoint{1.859311in}{0.487744in}}%
\pgfpathlineto{\pgfqpoint{1.903207in}{0.487744in}}%
\pgfpathlineto{\pgfqpoint{1.903207in}{0.497658in}}%
\pgfpathlineto{\pgfqpoint{1.947103in}{0.497658in}}%
\pgfpathlineto{\pgfqpoint{1.947103in}{0.481362in}}%
\pgfpathlineto{\pgfqpoint{1.990999in}{0.481362in}}%
\pgfpathlineto{\pgfqpoint{1.990999in}{0.460237in}}%
\pgfpathlineto{\pgfqpoint{2.034896in}{0.460237in}}%
\pgfpathlineto{\pgfqpoint{2.034896in}{0.458754in}}%
\pgfpathlineto{\pgfqpoint{2.078792in}{0.458754in}}%
\pgfpathlineto{\pgfqpoint{2.078792in}{0.457255in}}%
\pgfpathlineto{\pgfqpoint{2.122688in}{0.457255in}}%
\pgfpathlineto{\pgfqpoint{2.122688in}{0.457356in}}%
\pgfpathlineto{\pgfqpoint{2.166584in}{0.457356in}}%
\pgfpathlineto{\pgfqpoint{2.166584in}{0.454963in}}%
\pgfpathlineto{\pgfqpoint{2.210480in}{0.454963in}}%
\pgfpathlineto{\pgfqpoint{2.210480in}{0.455884in}}%
\pgfpathlineto{\pgfqpoint{2.254377in}{0.455884in}}%
\pgfpathlineto{\pgfqpoint{2.254377in}{0.451768in}}%
\pgfpathlineto{\pgfqpoint{2.298273in}{0.451768in}}%
\pgfpathlineto{\pgfqpoint{2.298273in}{0.451014in}}%
\pgfpathlineto{\pgfqpoint{2.342169in}{0.451014in}}%
\pgfpathlineto{\pgfqpoint{2.342169in}{0.450179in}}%
\pgfpathlineto{\pgfqpoint{2.386065in}{0.450179in}}%
\pgfpathlineto{\pgfqpoint{2.386065in}{0.450750in}}%
\pgfpathlineto{\pgfqpoint{2.429961in}{0.450750in}}%
\pgfpathlineto{\pgfqpoint{2.429961in}{0.450035in}}%
\pgfpathlineto{\pgfqpoint{2.473857in}{0.450035in}}%
\pgfpathlineto{\pgfqpoint{2.473857in}{0.449983in}}%
\pgfpathlineto{\pgfqpoint{2.517754in}{0.449983in}}%
\pgfpathlineto{\pgfqpoint{2.517754in}{0.449983in}}%
\pgfpathlineto{\pgfqpoint{2.561650in}{0.449983in}}%
\pgfpathlineto{\pgfqpoint{2.561650in}{0.449983in}}%
\pgfpathlineto{\pgfqpoint{2.517754in}{0.449983in}}%
\pgfpathlineto{\pgfqpoint{2.517754in}{0.449983in}}%
\pgfpathlineto{\pgfqpoint{2.473857in}{0.449983in}}%
\pgfpathlineto{\pgfqpoint{2.473857in}{0.449983in}}%
\pgfpathlineto{\pgfqpoint{2.429961in}{0.449983in}}%
\pgfpathlineto{\pgfqpoint{2.429961in}{0.449983in}}%
\pgfpathlineto{\pgfqpoint{2.386065in}{0.449983in}}%
\pgfpathlineto{\pgfqpoint{2.386065in}{0.449983in}}%
\pgfpathlineto{\pgfqpoint{2.342169in}{0.449983in}}%
\pgfpathlineto{\pgfqpoint{2.342169in}{0.449983in}}%
\pgfpathlineto{\pgfqpoint{2.298273in}{0.449983in}}%
\pgfpathlineto{\pgfqpoint{2.298273in}{0.449983in}}%
\pgfpathlineto{\pgfqpoint{2.254377in}{0.449983in}}%
\pgfpathlineto{\pgfqpoint{2.254377in}{0.449983in}}%
\pgfpathlineto{\pgfqpoint{2.210480in}{0.449983in}}%
\pgfpathlineto{\pgfqpoint{2.210480in}{0.449983in}}%
\pgfpathlineto{\pgfqpoint{2.166584in}{0.449983in}}%
\pgfpathlineto{\pgfqpoint{2.166584in}{0.449983in}}%
\pgfpathlineto{\pgfqpoint{2.122688in}{0.449983in}}%
\pgfpathlineto{\pgfqpoint{2.122688in}{0.449983in}}%
\pgfpathlineto{\pgfqpoint{2.078792in}{0.449983in}}%
\pgfpathlineto{\pgfqpoint{2.078792in}{0.449983in}}%
\pgfpathlineto{\pgfqpoint{2.034896in}{0.449983in}}%
\pgfpathlineto{\pgfqpoint{2.034896in}{0.449983in}}%
\pgfpathlineto{\pgfqpoint{1.990999in}{0.449983in}}%
\pgfpathlineto{\pgfqpoint{1.990999in}{0.449983in}}%
\pgfpathlineto{\pgfqpoint{1.947103in}{0.449983in}}%
\pgfpathlineto{\pgfqpoint{1.947103in}{0.449983in}}%
\pgfpathlineto{\pgfqpoint{1.903207in}{0.449983in}}%
\pgfpathlineto{\pgfqpoint{1.903207in}{0.449983in}}%
\pgfpathlineto{\pgfqpoint{1.859311in}{0.449983in}}%
\pgfpathlineto{\pgfqpoint{1.859311in}{0.449983in}}%
\pgfpathlineto{\pgfqpoint{1.815415in}{0.449983in}}%
\pgfpathlineto{\pgfqpoint{1.815415in}{0.449983in}}%
\pgfpathlineto{\pgfqpoint{1.771518in}{0.449983in}}%
\pgfpathlineto{\pgfqpoint{1.771518in}{0.449983in}}%
\pgfpathlineto{\pgfqpoint{1.727622in}{0.449983in}}%
\pgfpathlineto{\pgfqpoint{1.727622in}{0.449983in}}%
\pgfpathlineto{\pgfqpoint{1.683726in}{0.449983in}}%
\pgfpathlineto{\pgfqpoint{1.683726in}{0.449983in}}%
\pgfpathlineto{\pgfqpoint{1.639830in}{0.449983in}}%
\pgfpathlineto{\pgfqpoint{1.639830in}{0.449983in}}%
\pgfpathlineto{\pgfqpoint{1.595934in}{0.449983in}}%
\pgfpathlineto{\pgfqpoint{1.595934in}{0.449983in}}%
\pgfpathlineto{\pgfqpoint{1.552037in}{0.449983in}}%
\pgfpathlineto{\pgfqpoint{1.552037in}{0.449983in}}%
\pgfpathlineto{\pgfqpoint{1.508141in}{0.449983in}}%
\pgfpathlineto{\pgfqpoint{1.508141in}{0.449983in}}%
\pgfpathlineto{\pgfqpoint{1.464245in}{0.449983in}}%
\pgfpathlineto{\pgfqpoint{1.464245in}{0.449983in}}%
\pgfpathlineto{\pgfqpoint{1.420349in}{0.449983in}}%
\pgfpathlineto{\pgfqpoint{1.420349in}{0.449983in}}%
\pgfpathlineto{\pgfqpoint{1.376453in}{0.449983in}}%
\pgfpathlineto{\pgfqpoint{1.376453in}{0.449983in}}%
\pgfpathlineto{\pgfqpoint{1.332556in}{0.449983in}}%
\pgfpathlineto{\pgfqpoint{1.332556in}{0.449983in}}%
\pgfpathlineto{\pgfqpoint{1.288660in}{0.449983in}}%
\pgfpathlineto{\pgfqpoint{1.288660in}{0.449983in}}%
\pgfpathlineto{\pgfqpoint{1.244764in}{0.449983in}}%
\pgfpathlineto{\pgfqpoint{1.244764in}{0.449983in}}%
\pgfpathlineto{\pgfqpoint{1.200868in}{0.449983in}}%
\pgfpathlineto{\pgfqpoint{1.200868in}{0.449983in}}%
\pgfpathlineto{\pgfqpoint{1.156972in}{0.449983in}}%
\pgfpathlineto{\pgfqpoint{1.156972in}{0.449983in}}%
\pgfpathlineto{\pgfqpoint{1.113076in}{0.449983in}}%
\pgfpathlineto{\pgfqpoint{1.113076in}{0.449983in}}%
\pgfpathlineto{\pgfqpoint{1.069179in}{0.449983in}}%
\pgfpathlineto{\pgfqpoint{1.069179in}{0.449983in}}%
\pgfpathlineto{\pgfqpoint{1.025283in}{0.449983in}}%
\pgfpathlineto{\pgfqpoint{1.025283in}{0.449983in}}%
\pgfpathlineto{\pgfqpoint{0.981387in}{0.449983in}}%
\pgfpathlineto{\pgfqpoint{0.981387in}{0.449983in}}%
\pgfpathlineto{\pgfqpoint{0.937491in}{0.449983in}}%
\pgfpathlineto{\pgfqpoint{0.937491in}{0.449983in}}%
\pgfpathlineto{\pgfqpoint{0.893595in}{0.449983in}}%
\pgfpathlineto{\pgfqpoint{0.893595in}{0.449983in}}%
\pgfpathlineto{\pgfqpoint{0.849698in}{0.449983in}}%
\pgfpathlineto{\pgfqpoint{0.849698in}{0.449983in}}%
\pgfpathlineto{\pgfqpoint{0.805802in}{0.449983in}}%
\pgfpathlineto{\pgfqpoint{0.805802in}{0.449983in}}%
\pgfpathlineto{\pgfqpoint{0.761906in}{0.449983in}}%
\pgfpathlineto{\pgfqpoint{0.761906in}{0.449983in}}%
\pgfpathlineto{\pgfqpoint{0.718010in}{0.449983in}}%
\pgfpathlineto{\pgfqpoint{0.718010in}{0.449983in}}%
\pgfpathlineto{\pgfqpoint{0.674114in}{0.449983in}}%
\pgfpathlineto{\pgfqpoint{0.674114in}{0.449983in}}%
\pgfpathlineto{\pgfqpoint{0.630217in}{0.449983in}}%
\pgfpathlineto{\pgfqpoint{0.630217in}{0.449983in}}%
\pgfpathlineto{\pgfqpoint{0.586321in}{0.449983in}}%
\pgfpathlineto{\pgfqpoint{0.586321in}{0.449983in}}%
\pgfpathlineto{\pgfqpoint{0.542425in}{0.449983in}}%
\pgfpathlineto{\pgfqpoint{0.542425in}{0.449983in}}%
\pgfpathlineto{\pgfqpoint{0.498529in}{0.449983in}}%
\pgfpathlineto{\pgfqpoint{0.498529in}{0.449983in}}%
\pgfpathlineto{\pgfqpoint{0.454633in}{0.449983in}}%
\pgfpathlineto{\pgfqpoint{0.454633in}{0.449983in}}%
\pgfpathlineto{\pgfqpoint{0.410736in}{0.449983in}}%
\pgfpathlineto{\pgfqpoint{0.410736in}{0.449983in}}%
\pgfpathlineto{\pgfqpoint{0.366840in}{0.449983in}}%
\pgfusepath{fill}%
\end{pgfscope}%
\begin{pgfscope}%
\pgfpathrectangle{\pgfqpoint{0.366840in}{0.449983in}}{\pgfqpoint{2.194810in}{1.165600in}} %
\pgfusepath{clip}%
\pgfsetbuttcap%
\pgfsetmiterjoin%
\pgfsetlinewidth{0.501875pt}%
\definecolor{currentstroke}{rgb}{1.000000,0.000000,0.000000}%
\pgfsetstrokecolor{currentstroke}%
\pgfsetdash{}{0pt}%
\pgfpathmoveto{\pgfqpoint{0.366840in}{0.449983in}}%
\pgfpathlineto{\pgfqpoint{0.366840in}{0.449983in}}%
\pgfpathlineto{\pgfqpoint{0.410736in}{0.449983in}}%
\pgfpathlineto{\pgfqpoint{0.410736in}{0.449983in}}%
\pgfpathlineto{\pgfqpoint{0.454633in}{0.449983in}}%
\pgfpathlineto{\pgfqpoint{0.454633in}{0.450110in}}%
\pgfpathlineto{\pgfqpoint{0.498529in}{0.450110in}}%
\pgfpathlineto{\pgfqpoint{0.498529in}{0.450584in}}%
\pgfpathlineto{\pgfqpoint{0.542425in}{0.450584in}}%
\pgfpathlineto{\pgfqpoint{0.542425in}{0.464862in}}%
\pgfpathlineto{\pgfqpoint{0.586321in}{0.464862in}}%
\pgfpathlineto{\pgfqpoint{0.586321in}{0.514108in}}%
\pgfpathlineto{\pgfqpoint{0.630217in}{0.514108in}}%
\pgfpathlineto{\pgfqpoint{0.630217in}{0.544054in}}%
\pgfpathlineto{\pgfqpoint{0.674114in}{0.544054in}}%
\pgfpathlineto{\pgfqpoint{0.674114in}{0.554131in}}%
\pgfpathlineto{\pgfqpoint{0.718010in}{0.554131in}}%
\pgfpathlineto{\pgfqpoint{0.718010in}{0.576874in}}%
\pgfpathlineto{\pgfqpoint{0.761906in}{0.576874in}}%
\pgfpathlineto{\pgfqpoint{0.761906in}{0.641504in}}%
\pgfpathlineto{\pgfqpoint{0.805802in}{0.641504in}}%
\pgfpathlineto{\pgfqpoint{0.805802in}{0.803269in}}%
\pgfpathlineto{\pgfqpoint{0.849698in}{0.803269in}}%
\pgfpathlineto{\pgfqpoint{0.849698in}{0.800616in}}%
\pgfpathlineto{\pgfqpoint{0.893595in}{0.800616in}}%
\pgfpathlineto{\pgfqpoint{0.893595in}{0.867299in}}%
\pgfpathlineto{\pgfqpoint{0.937491in}{0.867299in}}%
\pgfpathlineto{\pgfqpoint{0.937491in}{1.092462in}}%
\pgfpathlineto{\pgfqpoint{0.981387in}{1.092462in}}%
\pgfpathlineto{\pgfqpoint{0.981387in}{1.103486in}}%
\pgfpathlineto{\pgfqpoint{1.025283in}{1.103486in}}%
\pgfpathlineto{\pgfqpoint{1.025283in}{1.147710in}}%
\pgfpathlineto{\pgfqpoint{1.069179in}{1.147710in}}%
\pgfpathlineto{\pgfqpoint{1.069179in}{1.176108in}}%
\pgfpathlineto{\pgfqpoint{1.113076in}{1.176108in}}%
\pgfpathlineto{\pgfqpoint{1.113076in}{1.201600in}}%
\pgfpathlineto{\pgfqpoint{1.156972in}{1.201600in}}%
\pgfpathlineto{\pgfqpoint{1.156972in}{1.218468in}}%
\pgfpathlineto{\pgfqpoint{1.200868in}{1.218468in}}%
\pgfpathlineto{\pgfqpoint{1.200868in}{1.212814in}}%
\pgfpathlineto{\pgfqpoint{1.244764in}{1.212814in}}%
\pgfpathlineto{\pgfqpoint{1.244764in}{1.207128in}}%
\pgfpathlineto{\pgfqpoint{1.288660in}{1.207128in}}%
\pgfpathlineto{\pgfqpoint{1.288660in}{1.186343in}}%
\pgfpathlineto{\pgfqpoint{1.332556in}{1.186343in}}%
\pgfpathlineto{\pgfqpoint{1.332556in}{1.129610in}}%
\pgfpathlineto{\pgfqpoint{1.376453in}{1.129610in}}%
\pgfpathlineto{\pgfqpoint{1.376453in}{1.147868in}}%
\pgfpathlineto{\pgfqpoint{1.420349in}{1.147868in}}%
\pgfpathlineto{\pgfqpoint{1.420349in}{1.454055in}}%
\pgfpathlineto{\pgfqpoint{1.464245in}{1.454055in}}%
\pgfpathlineto{\pgfqpoint{1.464245in}{1.131000in}}%
\pgfpathlineto{\pgfqpoint{1.508141in}{1.131000in}}%
\pgfpathlineto{\pgfqpoint{1.508141in}{0.908964in}}%
\pgfpathlineto{\pgfqpoint{1.552037in}{0.908964in}}%
\pgfpathlineto{\pgfqpoint{1.552037in}{0.832362in}}%
\pgfpathlineto{\pgfqpoint{1.595934in}{0.832362in}}%
\pgfpathlineto{\pgfqpoint{1.595934in}{0.763404in}}%
\pgfpathlineto{\pgfqpoint{1.639830in}{0.763404in}}%
\pgfpathlineto{\pgfqpoint{1.639830in}{0.705060in}}%
\pgfpathlineto{\pgfqpoint{1.683726in}{0.705060in}}%
\pgfpathlineto{\pgfqpoint{1.683726in}{0.669113in}}%
\pgfpathlineto{\pgfqpoint{1.727622in}{0.669113in}}%
\pgfpathlineto{\pgfqpoint{1.727622in}{0.648643in}}%
\pgfpathlineto{\pgfqpoint{1.771518in}{0.648643in}}%
\pgfpathlineto{\pgfqpoint{1.771518in}{0.608494in}}%
\pgfpathlineto{\pgfqpoint{1.815415in}{0.608494in}}%
\pgfpathlineto{\pgfqpoint{1.815415in}{0.571125in}}%
\pgfpathlineto{\pgfqpoint{1.859311in}{0.571125in}}%
\pgfpathlineto{\pgfqpoint{1.859311in}{0.543643in}}%
\pgfpathlineto{\pgfqpoint{1.903207in}{0.543643in}}%
\pgfpathlineto{\pgfqpoint{1.903207in}{0.527502in}}%
\pgfpathlineto{\pgfqpoint{1.947103in}{0.527502in}}%
\pgfpathlineto{\pgfqpoint{1.947103in}{0.528102in}}%
\pgfpathlineto{\pgfqpoint{1.990999in}{0.528102in}}%
\pgfpathlineto{\pgfqpoint{1.990999in}{0.504316in}}%
\pgfpathlineto{\pgfqpoint{2.034896in}{0.504316in}}%
\pgfpathlineto{\pgfqpoint{2.034896in}{0.490891in}}%
\pgfpathlineto{\pgfqpoint{2.078792in}{0.490891in}}%
\pgfpathlineto{\pgfqpoint{2.078792in}{0.482930in}}%
\pgfpathlineto{\pgfqpoint{2.122688in}{0.482930in}}%
\pgfpathlineto{\pgfqpoint{2.122688in}{0.476676in}}%
\pgfpathlineto{\pgfqpoint{2.166584in}{0.476676in}}%
\pgfpathlineto{\pgfqpoint{2.166584in}{0.474022in}}%
\pgfpathlineto{\pgfqpoint{2.210480in}{0.474022in}}%
\pgfpathlineto{\pgfqpoint{2.210480in}{0.474180in}}%
\pgfpathlineto{\pgfqpoint{2.254377in}{0.474180in}}%
\pgfpathlineto{\pgfqpoint{2.254377in}{0.469568in}}%
\pgfpathlineto{\pgfqpoint{2.298273in}{0.469568in}}%
\pgfpathlineto{\pgfqpoint{2.298273in}{0.461608in}}%
\pgfpathlineto{\pgfqpoint{2.342169in}{0.461608in}}%
\pgfpathlineto{\pgfqpoint{2.342169in}{0.455543in}}%
\pgfpathlineto{\pgfqpoint{2.386065in}{0.455543in}}%
\pgfpathlineto{\pgfqpoint{2.386065in}{0.452542in}}%
\pgfpathlineto{\pgfqpoint{2.429961in}{0.452542in}}%
\pgfpathlineto{\pgfqpoint{2.429961in}{0.450584in}}%
\pgfpathlineto{\pgfqpoint{2.473857in}{0.450584in}}%
\pgfpathlineto{\pgfqpoint{2.473857in}{0.450078in}}%
\pgfpathlineto{\pgfqpoint{2.517754in}{0.450078in}}%
\pgfpathlineto{\pgfqpoint{2.517754in}{0.449983in}}%
\pgfpathlineto{\pgfqpoint{2.561650in}{0.449983in}}%
\pgfpathlineto{\pgfqpoint{2.561650in}{0.449983in}}%
\pgfusepath{stroke}%
\end{pgfscope}%
\begin{pgfscope}%
\pgfsetrectcap%
\pgfsetmiterjoin%
\pgfsetlinewidth{1.003750pt}%
\definecolor{currentstroke}{rgb}{0.000000,0.000000,0.000000}%
\pgfsetstrokecolor{currentstroke}%
\pgfsetdash{}{0pt}%
\pgfpathmoveto{\pgfqpoint{0.366840in}{1.615583in}}%
\pgfpathlineto{\pgfqpoint{2.561650in}{1.615583in}}%
\pgfusepath{stroke}%
\end{pgfscope}%
\begin{pgfscope}%
\pgfsetrectcap%
\pgfsetmiterjoin%
\pgfsetlinewidth{1.003750pt}%
\definecolor{currentstroke}{rgb}{0.000000,0.000000,0.000000}%
\pgfsetstrokecolor{currentstroke}%
\pgfsetdash{}{0pt}%
\pgfpathmoveto{\pgfqpoint{2.561650in}{0.449983in}}%
\pgfpathlineto{\pgfqpoint{2.561650in}{1.615583in}}%
\pgfusepath{stroke}%
\end{pgfscope}%
\begin{pgfscope}%
\pgfsetrectcap%
\pgfsetmiterjoin%
\pgfsetlinewidth{1.003750pt}%
\definecolor{currentstroke}{rgb}{0.000000,0.000000,0.000000}%
\pgfsetstrokecolor{currentstroke}%
\pgfsetdash{}{0pt}%
\pgfpathmoveto{\pgfqpoint{0.366840in}{0.449983in}}%
\pgfpathlineto{\pgfqpoint{2.561650in}{0.449983in}}%
\pgfusepath{stroke}%
\end{pgfscope}%
\begin{pgfscope}%
\pgfsetrectcap%
\pgfsetmiterjoin%
\pgfsetlinewidth{1.003750pt}%
\definecolor{currentstroke}{rgb}{0.000000,0.000000,0.000000}%
\pgfsetstrokecolor{currentstroke}%
\pgfsetdash{}{0pt}%
\pgfpathmoveto{\pgfqpoint{0.366840in}{0.449983in}}%
\pgfpathlineto{\pgfqpoint{0.366840in}{1.615583in}}%
\pgfusepath{stroke}%
\end{pgfscope}%
\begin{pgfscope}%
\pgfsetbuttcap%
\pgfsetroundjoin%
\definecolor{currentfill}{rgb}{0.000000,0.000000,0.000000}%
\pgfsetfillcolor{currentfill}%
\pgfsetlinewidth{0.501875pt}%
\definecolor{currentstroke}{rgb}{0.000000,0.000000,0.000000}%
\pgfsetstrokecolor{currentstroke}%
\pgfsetdash{}{0pt}%
\pgfsys@defobject{currentmarker}{\pgfqpoint{0.000000in}{0.000000in}}{\pgfqpoint{0.000000in}{0.069444in}}{%
\pgfpathmoveto{\pgfqpoint{0.000000in}{0.000000in}}%
\pgfpathlineto{\pgfqpoint{0.000000in}{0.069444in}}%
\pgfusepath{stroke,fill}%
}%
\begin{pgfscope}%
\pgfsys@transformshift{0.366840in}{0.449983in}%
\pgfsys@useobject{currentmarker}{}%
\end{pgfscope}%
\end{pgfscope}%
\begin{pgfscope}%
\pgfsetbuttcap%
\pgfsetroundjoin%
\definecolor{currentfill}{rgb}{0.000000,0.000000,0.000000}%
\pgfsetfillcolor{currentfill}%
\pgfsetlinewidth{0.501875pt}%
\definecolor{currentstroke}{rgb}{0.000000,0.000000,0.000000}%
\pgfsetstrokecolor{currentstroke}%
\pgfsetdash{}{0pt}%
\pgfsys@defobject{currentmarker}{\pgfqpoint{0.000000in}{-0.069444in}}{\pgfqpoint{0.000000in}{0.000000in}}{%
\pgfpathmoveto{\pgfqpoint{0.000000in}{0.000000in}}%
\pgfpathlineto{\pgfqpoint{0.000000in}{-0.069444in}}%
\pgfusepath{stroke,fill}%
}%
\begin{pgfscope}%
\pgfsys@transformshift{0.366840in}{1.615583in}%
\pgfsys@useobject{currentmarker}{}%
\end{pgfscope}%
\end{pgfscope}%
\begin{pgfscope}%
\pgftext[x=0.366840in,y=0.380539in,,top]{\rmfamily\fontsize{8.000000}{9.600000}\selectfont −15}%
\end{pgfscope}%
\begin{pgfscope}%
\pgfsetbuttcap%
\pgfsetroundjoin%
\definecolor{currentfill}{rgb}{0.000000,0.000000,0.000000}%
\pgfsetfillcolor{currentfill}%
\pgfsetlinewidth{0.501875pt}%
\definecolor{currentstroke}{rgb}{0.000000,0.000000,0.000000}%
\pgfsetstrokecolor{currentstroke}%
\pgfsetdash{}{0pt}%
\pgfsys@defobject{currentmarker}{\pgfqpoint{0.000000in}{0.000000in}}{\pgfqpoint{0.000000in}{0.069444in}}{%
\pgfpathmoveto{\pgfqpoint{0.000000in}{0.000000in}}%
\pgfpathlineto{\pgfqpoint{0.000000in}{0.069444in}}%
\pgfusepath{stroke,fill}%
}%
\begin{pgfscope}%
\pgfsys@transformshift{0.732642in}{0.449983in}%
\pgfsys@useobject{currentmarker}{}%
\end{pgfscope}%
\end{pgfscope}%
\begin{pgfscope}%
\pgfsetbuttcap%
\pgfsetroundjoin%
\definecolor{currentfill}{rgb}{0.000000,0.000000,0.000000}%
\pgfsetfillcolor{currentfill}%
\pgfsetlinewidth{0.501875pt}%
\definecolor{currentstroke}{rgb}{0.000000,0.000000,0.000000}%
\pgfsetstrokecolor{currentstroke}%
\pgfsetdash{}{0pt}%
\pgfsys@defobject{currentmarker}{\pgfqpoint{0.000000in}{-0.069444in}}{\pgfqpoint{0.000000in}{0.000000in}}{%
\pgfpathmoveto{\pgfqpoint{0.000000in}{0.000000in}}%
\pgfpathlineto{\pgfqpoint{0.000000in}{-0.069444in}}%
\pgfusepath{stroke,fill}%
}%
\begin{pgfscope}%
\pgfsys@transformshift{0.732642in}{1.615583in}%
\pgfsys@useobject{currentmarker}{}%
\end{pgfscope}%
\end{pgfscope}%
\begin{pgfscope}%
\pgftext[x=0.732642in,y=0.380539in,,top]{\rmfamily\fontsize{8.000000}{9.600000}\selectfont −10}%
\end{pgfscope}%
\begin{pgfscope}%
\pgfsetbuttcap%
\pgfsetroundjoin%
\definecolor{currentfill}{rgb}{0.000000,0.000000,0.000000}%
\pgfsetfillcolor{currentfill}%
\pgfsetlinewidth{0.501875pt}%
\definecolor{currentstroke}{rgb}{0.000000,0.000000,0.000000}%
\pgfsetstrokecolor{currentstroke}%
\pgfsetdash{}{0pt}%
\pgfsys@defobject{currentmarker}{\pgfqpoint{0.000000in}{0.000000in}}{\pgfqpoint{0.000000in}{0.069444in}}{%
\pgfpathmoveto{\pgfqpoint{0.000000in}{0.000000in}}%
\pgfpathlineto{\pgfqpoint{0.000000in}{0.069444in}}%
\pgfusepath{stroke,fill}%
}%
\begin{pgfscope}%
\pgfsys@transformshift{1.098443in}{0.449983in}%
\pgfsys@useobject{currentmarker}{}%
\end{pgfscope}%
\end{pgfscope}%
\begin{pgfscope}%
\pgfsetbuttcap%
\pgfsetroundjoin%
\definecolor{currentfill}{rgb}{0.000000,0.000000,0.000000}%
\pgfsetfillcolor{currentfill}%
\pgfsetlinewidth{0.501875pt}%
\definecolor{currentstroke}{rgb}{0.000000,0.000000,0.000000}%
\pgfsetstrokecolor{currentstroke}%
\pgfsetdash{}{0pt}%
\pgfsys@defobject{currentmarker}{\pgfqpoint{0.000000in}{-0.069444in}}{\pgfqpoint{0.000000in}{0.000000in}}{%
\pgfpathmoveto{\pgfqpoint{0.000000in}{0.000000in}}%
\pgfpathlineto{\pgfqpoint{0.000000in}{-0.069444in}}%
\pgfusepath{stroke,fill}%
}%
\begin{pgfscope}%
\pgfsys@transformshift{1.098443in}{1.615583in}%
\pgfsys@useobject{currentmarker}{}%
\end{pgfscope}%
\end{pgfscope}%
\begin{pgfscope}%
\pgftext[x=1.098443in,y=0.380539in,,top]{\rmfamily\fontsize{8.000000}{9.600000}\selectfont −5}%
\end{pgfscope}%
\begin{pgfscope}%
\pgfsetbuttcap%
\pgfsetroundjoin%
\definecolor{currentfill}{rgb}{0.000000,0.000000,0.000000}%
\pgfsetfillcolor{currentfill}%
\pgfsetlinewidth{0.501875pt}%
\definecolor{currentstroke}{rgb}{0.000000,0.000000,0.000000}%
\pgfsetstrokecolor{currentstroke}%
\pgfsetdash{}{0pt}%
\pgfsys@defobject{currentmarker}{\pgfqpoint{0.000000in}{0.000000in}}{\pgfqpoint{0.000000in}{0.069444in}}{%
\pgfpathmoveto{\pgfqpoint{0.000000in}{0.000000in}}%
\pgfpathlineto{\pgfqpoint{0.000000in}{0.069444in}}%
\pgfusepath{stroke,fill}%
}%
\begin{pgfscope}%
\pgfsys@transformshift{1.464245in}{0.449983in}%
\pgfsys@useobject{currentmarker}{}%
\end{pgfscope}%
\end{pgfscope}%
\begin{pgfscope}%
\pgfsetbuttcap%
\pgfsetroundjoin%
\definecolor{currentfill}{rgb}{0.000000,0.000000,0.000000}%
\pgfsetfillcolor{currentfill}%
\pgfsetlinewidth{0.501875pt}%
\definecolor{currentstroke}{rgb}{0.000000,0.000000,0.000000}%
\pgfsetstrokecolor{currentstroke}%
\pgfsetdash{}{0pt}%
\pgfsys@defobject{currentmarker}{\pgfqpoint{0.000000in}{-0.069444in}}{\pgfqpoint{0.000000in}{0.000000in}}{%
\pgfpathmoveto{\pgfqpoint{0.000000in}{0.000000in}}%
\pgfpathlineto{\pgfqpoint{0.000000in}{-0.069444in}}%
\pgfusepath{stroke,fill}%
}%
\begin{pgfscope}%
\pgfsys@transformshift{1.464245in}{1.615583in}%
\pgfsys@useobject{currentmarker}{}%
\end{pgfscope}%
\end{pgfscope}%
\begin{pgfscope}%
\pgftext[x=1.464245in,y=0.380539in,,top]{\rmfamily\fontsize{8.000000}{9.600000}\selectfont 0}%
\end{pgfscope}%
\begin{pgfscope}%
\pgfsetbuttcap%
\pgfsetroundjoin%
\definecolor{currentfill}{rgb}{0.000000,0.000000,0.000000}%
\pgfsetfillcolor{currentfill}%
\pgfsetlinewidth{0.501875pt}%
\definecolor{currentstroke}{rgb}{0.000000,0.000000,0.000000}%
\pgfsetstrokecolor{currentstroke}%
\pgfsetdash{}{0pt}%
\pgfsys@defobject{currentmarker}{\pgfqpoint{0.000000in}{0.000000in}}{\pgfqpoint{0.000000in}{0.069444in}}{%
\pgfpathmoveto{\pgfqpoint{0.000000in}{0.000000in}}%
\pgfpathlineto{\pgfqpoint{0.000000in}{0.069444in}}%
\pgfusepath{stroke,fill}%
}%
\begin{pgfscope}%
\pgfsys@transformshift{1.830047in}{0.449983in}%
\pgfsys@useobject{currentmarker}{}%
\end{pgfscope}%
\end{pgfscope}%
\begin{pgfscope}%
\pgfsetbuttcap%
\pgfsetroundjoin%
\definecolor{currentfill}{rgb}{0.000000,0.000000,0.000000}%
\pgfsetfillcolor{currentfill}%
\pgfsetlinewidth{0.501875pt}%
\definecolor{currentstroke}{rgb}{0.000000,0.000000,0.000000}%
\pgfsetstrokecolor{currentstroke}%
\pgfsetdash{}{0pt}%
\pgfsys@defobject{currentmarker}{\pgfqpoint{0.000000in}{-0.069444in}}{\pgfqpoint{0.000000in}{0.000000in}}{%
\pgfpathmoveto{\pgfqpoint{0.000000in}{0.000000in}}%
\pgfpathlineto{\pgfqpoint{0.000000in}{-0.069444in}}%
\pgfusepath{stroke,fill}%
}%
\begin{pgfscope}%
\pgfsys@transformshift{1.830047in}{1.615583in}%
\pgfsys@useobject{currentmarker}{}%
\end{pgfscope}%
\end{pgfscope}%
\begin{pgfscope}%
\pgftext[x=1.830047in,y=0.380539in,,top]{\rmfamily\fontsize{8.000000}{9.600000}\selectfont 5}%
\end{pgfscope}%
\begin{pgfscope}%
\pgfsetbuttcap%
\pgfsetroundjoin%
\definecolor{currentfill}{rgb}{0.000000,0.000000,0.000000}%
\pgfsetfillcolor{currentfill}%
\pgfsetlinewidth{0.501875pt}%
\definecolor{currentstroke}{rgb}{0.000000,0.000000,0.000000}%
\pgfsetstrokecolor{currentstroke}%
\pgfsetdash{}{0pt}%
\pgfsys@defobject{currentmarker}{\pgfqpoint{0.000000in}{0.000000in}}{\pgfqpoint{0.000000in}{0.069444in}}{%
\pgfpathmoveto{\pgfqpoint{0.000000in}{0.000000in}}%
\pgfpathlineto{\pgfqpoint{0.000000in}{0.069444in}}%
\pgfusepath{stroke,fill}%
}%
\begin{pgfscope}%
\pgfsys@transformshift{2.195848in}{0.449983in}%
\pgfsys@useobject{currentmarker}{}%
\end{pgfscope}%
\end{pgfscope}%
\begin{pgfscope}%
\pgfsetbuttcap%
\pgfsetroundjoin%
\definecolor{currentfill}{rgb}{0.000000,0.000000,0.000000}%
\pgfsetfillcolor{currentfill}%
\pgfsetlinewidth{0.501875pt}%
\definecolor{currentstroke}{rgb}{0.000000,0.000000,0.000000}%
\pgfsetstrokecolor{currentstroke}%
\pgfsetdash{}{0pt}%
\pgfsys@defobject{currentmarker}{\pgfqpoint{0.000000in}{-0.069444in}}{\pgfqpoint{0.000000in}{0.000000in}}{%
\pgfpathmoveto{\pgfqpoint{0.000000in}{0.000000in}}%
\pgfpathlineto{\pgfqpoint{0.000000in}{-0.069444in}}%
\pgfusepath{stroke,fill}%
}%
\begin{pgfscope}%
\pgfsys@transformshift{2.195848in}{1.615583in}%
\pgfsys@useobject{currentmarker}{}%
\end{pgfscope}%
\end{pgfscope}%
\begin{pgfscope}%
\pgftext[x=2.195848in,y=0.380539in,,top]{\rmfamily\fontsize{8.000000}{9.600000}\selectfont 10}%
\end{pgfscope}%
\begin{pgfscope}%
\pgfsetbuttcap%
\pgfsetroundjoin%
\definecolor{currentfill}{rgb}{0.000000,0.000000,0.000000}%
\pgfsetfillcolor{currentfill}%
\pgfsetlinewidth{0.501875pt}%
\definecolor{currentstroke}{rgb}{0.000000,0.000000,0.000000}%
\pgfsetstrokecolor{currentstroke}%
\pgfsetdash{}{0pt}%
\pgfsys@defobject{currentmarker}{\pgfqpoint{0.000000in}{0.000000in}}{\pgfqpoint{0.000000in}{0.069444in}}{%
\pgfpathmoveto{\pgfqpoint{0.000000in}{0.000000in}}%
\pgfpathlineto{\pgfqpoint{0.000000in}{0.069444in}}%
\pgfusepath{stroke,fill}%
}%
\begin{pgfscope}%
\pgfsys@transformshift{2.561650in}{0.449983in}%
\pgfsys@useobject{currentmarker}{}%
\end{pgfscope}%
\end{pgfscope}%
\begin{pgfscope}%
\pgfsetbuttcap%
\pgfsetroundjoin%
\definecolor{currentfill}{rgb}{0.000000,0.000000,0.000000}%
\pgfsetfillcolor{currentfill}%
\pgfsetlinewidth{0.501875pt}%
\definecolor{currentstroke}{rgb}{0.000000,0.000000,0.000000}%
\pgfsetstrokecolor{currentstroke}%
\pgfsetdash{}{0pt}%
\pgfsys@defobject{currentmarker}{\pgfqpoint{0.000000in}{-0.069444in}}{\pgfqpoint{0.000000in}{0.000000in}}{%
\pgfpathmoveto{\pgfqpoint{0.000000in}{0.000000in}}%
\pgfpathlineto{\pgfqpoint{0.000000in}{-0.069444in}}%
\pgfusepath{stroke,fill}%
}%
\begin{pgfscope}%
\pgfsys@transformshift{2.561650in}{1.615583in}%
\pgfsys@useobject{currentmarker}{}%
\end{pgfscope}%
\end{pgfscope}%
\begin{pgfscope}%
\pgftext[x=2.561650in,y=0.380539in,,top]{\rmfamily\fontsize{8.000000}{9.600000}\selectfont 15}%
\end{pgfscope}%
\begin{pgfscope}%
\pgftext[x=1.464245in,y=0.203564in,,top]{\rmfamily\fontsize{9.000000}{10.800000}\selectfont \(\displaystyle \mathrm{DLL}_{\mu/\pi}(K^+)\)}%
\end{pgfscope}%
\begin{pgfscope}%
\pgfsetbuttcap%
\pgfsetroundjoin%
\definecolor{currentfill}{rgb}{0.000000,0.000000,0.000000}%
\pgfsetfillcolor{currentfill}%
\pgfsetlinewidth{0.501875pt}%
\definecolor{currentstroke}{rgb}{0.000000,0.000000,0.000000}%
\pgfsetstrokecolor{currentstroke}%
\pgfsetdash{}{0pt}%
\pgfsys@defobject{currentmarker}{\pgfqpoint{0.000000in}{0.000000in}}{\pgfqpoint{0.069444in}{0.000000in}}{%
\pgfpathmoveto{\pgfqpoint{0.000000in}{0.000000in}}%
\pgfpathlineto{\pgfqpoint{0.069444in}{0.000000in}}%
\pgfusepath{stroke,fill}%
}%
\begin{pgfscope}%
\pgfsys@transformshift{0.366840in}{0.449983in}%
\pgfsys@useobject{currentmarker}{}%
\end{pgfscope}%
\end{pgfscope}%
\begin{pgfscope}%
\pgfsetbuttcap%
\pgfsetroundjoin%
\definecolor{currentfill}{rgb}{0.000000,0.000000,0.000000}%
\pgfsetfillcolor{currentfill}%
\pgfsetlinewidth{0.501875pt}%
\definecolor{currentstroke}{rgb}{0.000000,0.000000,0.000000}%
\pgfsetstrokecolor{currentstroke}%
\pgfsetdash{}{0pt}%
\pgfsys@defobject{currentmarker}{\pgfqpoint{-0.069444in}{0.000000in}}{\pgfqpoint{0.000000in}{0.000000in}}{%
\pgfpathmoveto{\pgfqpoint{0.000000in}{0.000000in}}%
\pgfpathlineto{\pgfqpoint{-0.069444in}{0.000000in}}%
\pgfusepath{stroke,fill}%
}%
\begin{pgfscope}%
\pgfsys@transformshift{2.561650in}{0.449983in}%
\pgfsys@useobject{currentmarker}{}%
\end{pgfscope}%
\end{pgfscope}%
\begin{pgfscope}%
\pgftext[x=0.297396in,y=0.449983in,right,]{\rmfamily\fontsize{8.000000}{9.600000}\selectfont 0.00}%
\end{pgfscope}%
\begin{pgfscope}%
\pgfsetbuttcap%
\pgfsetroundjoin%
\definecolor{currentfill}{rgb}{0.000000,0.000000,0.000000}%
\pgfsetfillcolor{currentfill}%
\pgfsetlinewidth{0.501875pt}%
\definecolor{currentstroke}{rgb}{0.000000,0.000000,0.000000}%
\pgfsetstrokecolor{currentstroke}%
\pgfsetdash{}{0pt}%
\pgfsys@defobject{currentmarker}{\pgfqpoint{0.000000in}{0.000000in}}{\pgfqpoint{0.069444in}{0.000000in}}{%
\pgfpathmoveto{\pgfqpoint{0.000000in}{0.000000in}}%
\pgfpathlineto{\pgfqpoint{0.069444in}{0.000000in}}%
\pgfusepath{stroke,fill}%
}%
\begin{pgfscope}%
\pgfsys@transformshift{0.366840in}{0.616498in}%
\pgfsys@useobject{currentmarker}{}%
\end{pgfscope}%
\end{pgfscope}%
\begin{pgfscope}%
\pgfsetbuttcap%
\pgfsetroundjoin%
\definecolor{currentfill}{rgb}{0.000000,0.000000,0.000000}%
\pgfsetfillcolor{currentfill}%
\pgfsetlinewidth{0.501875pt}%
\definecolor{currentstroke}{rgb}{0.000000,0.000000,0.000000}%
\pgfsetstrokecolor{currentstroke}%
\pgfsetdash{}{0pt}%
\pgfsys@defobject{currentmarker}{\pgfqpoint{-0.069444in}{0.000000in}}{\pgfqpoint{0.000000in}{0.000000in}}{%
\pgfpathmoveto{\pgfqpoint{0.000000in}{0.000000in}}%
\pgfpathlineto{\pgfqpoint{-0.069444in}{0.000000in}}%
\pgfusepath{stroke,fill}%
}%
\begin{pgfscope}%
\pgfsys@transformshift{2.561650in}{0.616498in}%
\pgfsys@useobject{currentmarker}{}%
\end{pgfscope}%
\end{pgfscope}%
\begin{pgfscope}%
\pgftext[x=0.297396in,y=0.616498in,right,]{\rmfamily\fontsize{8.000000}{9.600000}\selectfont 0.02}%
\end{pgfscope}%
\begin{pgfscope}%
\pgfsetbuttcap%
\pgfsetroundjoin%
\definecolor{currentfill}{rgb}{0.000000,0.000000,0.000000}%
\pgfsetfillcolor{currentfill}%
\pgfsetlinewidth{0.501875pt}%
\definecolor{currentstroke}{rgb}{0.000000,0.000000,0.000000}%
\pgfsetstrokecolor{currentstroke}%
\pgfsetdash{}{0pt}%
\pgfsys@defobject{currentmarker}{\pgfqpoint{0.000000in}{0.000000in}}{\pgfqpoint{0.069444in}{0.000000in}}{%
\pgfpathmoveto{\pgfqpoint{0.000000in}{0.000000in}}%
\pgfpathlineto{\pgfqpoint{0.069444in}{0.000000in}}%
\pgfusepath{stroke,fill}%
}%
\begin{pgfscope}%
\pgfsys@transformshift{0.366840in}{0.783012in}%
\pgfsys@useobject{currentmarker}{}%
\end{pgfscope}%
\end{pgfscope}%
\begin{pgfscope}%
\pgfsetbuttcap%
\pgfsetroundjoin%
\definecolor{currentfill}{rgb}{0.000000,0.000000,0.000000}%
\pgfsetfillcolor{currentfill}%
\pgfsetlinewidth{0.501875pt}%
\definecolor{currentstroke}{rgb}{0.000000,0.000000,0.000000}%
\pgfsetstrokecolor{currentstroke}%
\pgfsetdash{}{0pt}%
\pgfsys@defobject{currentmarker}{\pgfqpoint{-0.069444in}{0.000000in}}{\pgfqpoint{0.000000in}{0.000000in}}{%
\pgfpathmoveto{\pgfqpoint{0.000000in}{0.000000in}}%
\pgfpathlineto{\pgfqpoint{-0.069444in}{0.000000in}}%
\pgfusepath{stroke,fill}%
}%
\begin{pgfscope}%
\pgfsys@transformshift{2.561650in}{0.783012in}%
\pgfsys@useobject{currentmarker}{}%
\end{pgfscope}%
\end{pgfscope}%
\begin{pgfscope}%
\pgftext[x=0.297396in,y=0.783012in,right,]{\rmfamily\fontsize{8.000000}{9.600000}\selectfont 0.04}%
\end{pgfscope}%
\begin{pgfscope}%
\pgfsetbuttcap%
\pgfsetroundjoin%
\definecolor{currentfill}{rgb}{0.000000,0.000000,0.000000}%
\pgfsetfillcolor{currentfill}%
\pgfsetlinewidth{0.501875pt}%
\definecolor{currentstroke}{rgb}{0.000000,0.000000,0.000000}%
\pgfsetstrokecolor{currentstroke}%
\pgfsetdash{}{0pt}%
\pgfsys@defobject{currentmarker}{\pgfqpoint{0.000000in}{0.000000in}}{\pgfqpoint{0.069444in}{0.000000in}}{%
\pgfpathmoveto{\pgfqpoint{0.000000in}{0.000000in}}%
\pgfpathlineto{\pgfqpoint{0.069444in}{0.000000in}}%
\pgfusepath{stroke,fill}%
}%
\begin{pgfscope}%
\pgfsys@transformshift{0.366840in}{0.949526in}%
\pgfsys@useobject{currentmarker}{}%
\end{pgfscope}%
\end{pgfscope}%
\begin{pgfscope}%
\pgfsetbuttcap%
\pgfsetroundjoin%
\definecolor{currentfill}{rgb}{0.000000,0.000000,0.000000}%
\pgfsetfillcolor{currentfill}%
\pgfsetlinewidth{0.501875pt}%
\definecolor{currentstroke}{rgb}{0.000000,0.000000,0.000000}%
\pgfsetstrokecolor{currentstroke}%
\pgfsetdash{}{0pt}%
\pgfsys@defobject{currentmarker}{\pgfqpoint{-0.069444in}{0.000000in}}{\pgfqpoint{0.000000in}{0.000000in}}{%
\pgfpathmoveto{\pgfqpoint{0.000000in}{0.000000in}}%
\pgfpathlineto{\pgfqpoint{-0.069444in}{0.000000in}}%
\pgfusepath{stroke,fill}%
}%
\begin{pgfscope}%
\pgfsys@transformshift{2.561650in}{0.949526in}%
\pgfsys@useobject{currentmarker}{}%
\end{pgfscope}%
\end{pgfscope}%
\begin{pgfscope}%
\pgftext[x=0.297396in,y=0.949526in,right,]{\rmfamily\fontsize{8.000000}{9.600000}\selectfont 0.06}%
\end{pgfscope}%
\begin{pgfscope}%
\pgfsetbuttcap%
\pgfsetroundjoin%
\definecolor{currentfill}{rgb}{0.000000,0.000000,0.000000}%
\pgfsetfillcolor{currentfill}%
\pgfsetlinewidth{0.501875pt}%
\definecolor{currentstroke}{rgb}{0.000000,0.000000,0.000000}%
\pgfsetstrokecolor{currentstroke}%
\pgfsetdash{}{0pt}%
\pgfsys@defobject{currentmarker}{\pgfqpoint{0.000000in}{0.000000in}}{\pgfqpoint{0.069444in}{0.000000in}}{%
\pgfpathmoveto{\pgfqpoint{0.000000in}{0.000000in}}%
\pgfpathlineto{\pgfqpoint{0.069444in}{0.000000in}}%
\pgfusepath{stroke,fill}%
}%
\begin{pgfscope}%
\pgfsys@transformshift{0.366840in}{1.116041in}%
\pgfsys@useobject{currentmarker}{}%
\end{pgfscope}%
\end{pgfscope}%
\begin{pgfscope}%
\pgfsetbuttcap%
\pgfsetroundjoin%
\definecolor{currentfill}{rgb}{0.000000,0.000000,0.000000}%
\pgfsetfillcolor{currentfill}%
\pgfsetlinewidth{0.501875pt}%
\definecolor{currentstroke}{rgb}{0.000000,0.000000,0.000000}%
\pgfsetstrokecolor{currentstroke}%
\pgfsetdash{}{0pt}%
\pgfsys@defobject{currentmarker}{\pgfqpoint{-0.069444in}{0.000000in}}{\pgfqpoint{0.000000in}{0.000000in}}{%
\pgfpathmoveto{\pgfqpoint{0.000000in}{0.000000in}}%
\pgfpathlineto{\pgfqpoint{-0.069444in}{0.000000in}}%
\pgfusepath{stroke,fill}%
}%
\begin{pgfscope}%
\pgfsys@transformshift{2.561650in}{1.116041in}%
\pgfsys@useobject{currentmarker}{}%
\end{pgfscope}%
\end{pgfscope}%
\begin{pgfscope}%
\pgftext[x=0.297396in,y=1.116041in,right,]{\rmfamily\fontsize{8.000000}{9.600000}\selectfont 0.08}%
\end{pgfscope}%
\begin{pgfscope}%
\pgfsetbuttcap%
\pgfsetroundjoin%
\definecolor{currentfill}{rgb}{0.000000,0.000000,0.000000}%
\pgfsetfillcolor{currentfill}%
\pgfsetlinewidth{0.501875pt}%
\definecolor{currentstroke}{rgb}{0.000000,0.000000,0.000000}%
\pgfsetstrokecolor{currentstroke}%
\pgfsetdash{}{0pt}%
\pgfsys@defobject{currentmarker}{\pgfqpoint{0.000000in}{0.000000in}}{\pgfqpoint{0.069444in}{0.000000in}}{%
\pgfpathmoveto{\pgfqpoint{0.000000in}{0.000000in}}%
\pgfpathlineto{\pgfqpoint{0.069444in}{0.000000in}}%
\pgfusepath{stroke,fill}%
}%
\begin{pgfscope}%
\pgfsys@transformshift{0.366840in}{1.282555in}%
\pgfsys@useobject{currentmarker}{}%
\end{pgfscope}%
\end{pgfscope}%
\begin{pgfscope}%
\pgfsetbuttcap%
\pgfsetroundjoin%
\definecolor{currentfill}{rgb}{0.000000,0.000000,0.000000}%
\pgfsetfillcolor{currentfill}%
\pgfsetlinewidth{0.501875pt}%
\definecolor{currentstroke}{rgb}{0.000000,0.000000,0.000000}%
\pgfsetstrokecolor{currentstroke}%
\pgfsetdash{}{0pt}%
\pgfsys@defobject{currentmarker}{\pgfqpoint{-0.069444in}{0.000000in}}{\pgfqpoint{0.000000in}{0.000000in}}{%
\pgfpathmoveto{\pgfqpoint{0.000000in}{0.000000in}}%
\pgfpathlineto{\pgfqpoint{-0.069444in}{0.000000in}}%
\pgfusepath{stroke,fill}%
}%
\begin{pgfscope}%
\pgfsys@transformshift{2.561650in}{1.282555in}%
\pgfsys@useobject{currentmarker}{}%
\end{pgfscope}%
\end{pgfscope}%
\begin{pgfscope}%
\pgftext[x=0.297396in,y=1.282555in,right,]{\rmfamily\fontsize{8.000000}{9.600000}\selectfont 0.10}%
\end{pgfscope}%
\begin{pgfscope}%
\pgfsetbuttcap%
\pgfsetroundjoin%
\definecolor{currentfill}{rgb}{0.000000,0.000000,0.000000}%
\pgfsetfillcolor{currentfill}%
\pgfsetlinewidth{0.501875pt}%
\definecolor{currentstroke}{rgb}{0.000000,0.000000,0.000000}%
\pgfsetstrokecolor{currentstroke}%
\pgfsetdash{}{0pt}%
\pgfsys@defobject{currentmarker}{\pgfqpoint{0.000000in}{0.000000in}}{\pgfqpoint{0.069444in}{0.000000in}}{%
\pgfpathmoveto{\pgfqpoint{0.000000in}{0.000000in}}%
\pgfpathlineto{\pgfqpoint{0.069444in}{0.000000in}}%
\pgfusepath{stroke,fill}%
}%
\begin{pgfscope}%
\pgfsys@transformshift{0.366840in}{1.449069in}%
\pgfsys@useobject{currentmarker}{}%
\end{pgfscope}%
\end{pgfscope}%
\begin{pgfscope}%
\pgfsetbuttcap%
\pgfsetroundjoin%
\definecolor{currentfill}{rgb}{0.000000,0.000000,0.000000}%
\pgfsetfillcolor{currentfill}%
\pgfsetlinewidth{0.501875pt}%
\definecolor{currentstroke}{rgb}{0.000000,0.000000,0.000000}%
\pgfsetstrokecolor{currentstroke}%
\pgfsetdash{}{0pt}%
\pgfsys@defobject{currentmarker}{\pgfqpoint{-0.069444in}{0.000000in}}{\pgfqpoint{0.000000in}{0.000000in}}{%
\pgfpathmoveto{\pgfqpoint{0.000000in}{0.000000in}}%
\pgfpathlineto{\pgfqpoint{-0.069444in}{0.000000in}}%
\pgfusepath{stroke,fill}%
}%
\begin{pgfscope}%
\pgfsys@transformshift{2.561650in}{1.449069in}%
\pgfsys@useobject{currentmarker}{}%
\end{pgfscope}%
\end{pgfscope}%
\begin{pgfscope}%
\pgftext[x=0.297396in,y=1.449069in,right,]{\rmfamily\fontsize{8.000000}{9.600000}\selectfont 0.12}%
\end{pgfscope}%
\begin{pgfscope}%
\pgfsetbuttcap%
\pgfsetroundjoin%
\definecolor{currentfill}{rgb}{0.000000,0.000000,0.000000}%
\pgfsetfillcolor{currentfill}%
\pgfsetlinewidth{0.501875pt}%
\definecolor{currentstroke}{rgb}{0.000000,0.000000,0.000000}%
\pgfsetstrokecolor{currentstroke}%
\pgfsetdash{}{0pt}%
\pgfsys@defobject{currentmarker}{\pgfqpoint{0.000000in}{0.000000in}}{\pgfqpoint{0.069444in}{0.000000in}}{%
\pgfpathmoveto{\pgfqpoint{0.000000in}{0.000000in}}%
\pgfpathlineto{\pgfqpoint{0.069444in}{0.000000in}}%
\pgfusepath{stroke,fill}%
}%
\begin{pgfscope}%
\pgfsys@transformshift{0.366840in}{1.615583in}%
\pgfsys@useobject{currentmarker}{}%
\end{pgfscope}%
\end{pgfscope}%
\begin{pgfscope}%
\pgfsetbuttcap%
\pgfsetroundjoin%
\definecolor{currentfill}{rgb}{0.000000,0.000000,0.000000}%
\pgfsetfillcolor{currentfill}%
\pgfsetlinewidth{0.501875pt}%
\definecolor{currentstroke}{rgb}{0.000000,0.000000,0.000000}%
\pgfsetstrokecolor{currentstroke}%
\pgfsetdash{}{0pt}%
\pgfsys@defobject{currentmarker}{\pgfqpoint{-0.069444in}{0.000000in}}{\pgfqpoint{0.000000in}{0.000000in}}{%
\pgfpathmoveto{\pgfqpoint{0.000000in}{0.000000in}}%
\pgfpathlineto{\pgfqpoint{-0.069444in}{0.000000in}}%
\pgfusepath{stroke,fill}%
}%
\begin{pgfscope}%
\pgfsys@transformshift{2.561650in}{1.615583in}%
\pgfsys@useobject{currentmarker}{}%
\end{pgfscope}%
\end{pgfscope}%
\begin{pgfscope}%
\pgftext[x=0.297396in,y=1.615583in,right,]{\rmfamily\fontsize{8.000000}{9.600000}\selectfont 0.14}%
\end{pgfscope}%
\end{pgfpicture}%
\makeatother%
\endgroup%

	\end{subfigure}

	\begin{subfigure}[t]{0.49\textwidth}
		\centering
    %\includegraphics[width=\textwidth]{store/variables/SIG_BKG_piminus_PIDK.pdf}
    %% Creator: Matplotlib, PGF backend
%%
%% To include the figure in your LaTeX document, write
%%   \input{<filename>.pgf}
%%
%% Make sure the required packages are loaded in your preamble
%%   \usepackage{pgf}
%%
%% Figures using additional raster images can only be included by \input if
%% they are in the same directory as the main LaTeX file. For loading figures
%% from other directories you can use the `import` package
%%   \usepackage{import}
%% and then include the figures with
%%   \import{<path to file>}{<filename>.pgf}
%%
%% Matplotlib used the following preamble
%%   \usepackage{fontspec}
%%   \setmainfont{DejaVu Serif}
%%   \setsansfont{DejaVu Sans}
%%   \setmonofont{DejaVu Sans Mono}
%%
\begingroup%
\makeatletter%
\begin{pgfpicture}%
\pgfpathrectangle{\pgfpointorigin}{\pgfqpoint{2.679091in}{1.723197in}}%
\pgfusepath{use as bounding box, clip}%
\begin{pgfscope}%
\pgfsetbuttcap%
\pgfsetmiterjoin%
\definecolor{currentfill}{rgb}{1.000000,1.000000,1.000000}%
\pgfsetfillcolor{currentfill}%
\pgfsetlinewidth{0.000000pt}%
\definecolor{currentstroke}{rgb}{1.000000,1.000000,1.000000}%
\pgfsetstrokecolor{currentstroke}%
\pgfsetdash{}{0pt}%
\pgfpathmoveto{\pgfqpoint{0.000000in}{0.000000in}}%
\pgfpathlineto{\pgfqpoint{2.679091in}{0.000000in}}%
\pgfpathlineto{\pgfqpoint{2.679091in}{1.723197in}}%
\pgfpathlineto{\pgfqpoint{0.000000in}{1.723197in}}%
\pgfpathclose%
\pgfusepath{fill}%
\end{pgfscope}%
\begin{pgfscope}%
\pgfsetbuttcap%
\pgfsetmiterjoin%
\definecolor{currentfill}{rgb}{1.000000,1.000000,1.000000}%
\pgfsetfillcolor{currentfill}%
\pgfsetlinewidth{0.000000pt}%
\definecolor{currentstroke}{rgb}{0.000000,0.000000,0.000000}%
\pgfsetstrokecolor{currentstroke}%
\pgfsetstrokeopacity{0.000000}%
\pgfsetdash{}{0pt}%
\pgfpathmoveto{\pgfqpoint{0.437532in}{0.449983in}}%
\pgfpathlineto{\pgfqpoint{2.558398in}{0.449983in}}%
\pgfpathlineto{\pgfqpoint{2.558398in}{1.619432in}}%
\pgfpathlineto{\pgfqpoint{0.437532in}{1.619432in}}%
\pgfpathclose%
\pgfusepath{fill}%
\end{pgfscope}%
\begin{pgfscope}%
\pgfpathrectangle{\pgfqpoint{0.437532in}{0.449983in}}{\pgfqpoint{2.120866in}{1.169449in}} %
\pgfusepath{clip}%
\pgfsetbuttcap%
\pgfsetmiterjoin%
\definecolor{currentfill}{rgb}{0.215686,0.470588,0.749020}%
\pgfsetfillcolor{currentfill}%
\pgfsetlinewidth{0.000000pt}%
\definecolor{currentstroke}{rgb}{0.000000,0.000000,0.000000}%
\pgfsetstrokecolor{currentstroke}%
\pgfsetdash{}{0pt}%
\pgfpathmoveto{\pgfqpoint{0.555358in}{0.449983in}}%
\pgfpathlineto{\pgfqpoint{0.555358in}{0.449983in}}%
\pgfpathlineto{\pgfqpoint{0.593063in}{0.449983in}}%
\pgfpathlineto{\pgfqpoint{0.593063in}{0.449983in}}%
\pgfpathlineto{\pgfqpoint{0.630767in}{0.449983in}}%
\pgfpathlineto{\pgfqpoint{0.630767in}{0.450453in}}%
\pgfpathlineto{\pgfqpoint{0.668471in}{0.450453in}}%
\pgfpathlineto{\pgfqpoint{0.668471in}{0.450060in}}%
\pgfpathlineto{\pgfqpoint{0.706176in}{0.450060in}}%
\pgfpathlineto{\pgfqpoint{0.706176in}{0.449983in}}%
\pgfpathlineto{\pgfqpoint{0.743880in}{0.449983in}}%
\pgfpathlineto{\pgfqpoint{0.743880in}{0.450237in}}%
\pgfpathlineto{\pgfqpoint{0.781584in}{0.450237in}}%
\pgfpathlineto{\pgfqpoint{0.781584in}{0.450135in}}%
\pgfpathlineto{\pgfqpoint{0.819288in}{0.450135in}}%
\pgfpathlineto{\pgfqpoint{0.819288in}{0.451314in}}%
\pgfpathlineto{\pgfqpoint{0.856993in}{0.451314in}}%
\pgfpathlineto{\pgfqpoint{0.856993in}{0.451558in}}%
\pgfpathlineto{\pgfqpoint{0.894697in}{0.451558in}}%
\pgfpathlineto{\pgfqpoint{0.894697in}{0.451630in}}%
\pgfpathlineto{\pgfqpoint{0.932401in}{0.451630in}}%
\pgfpathlineto{\pgfqpoint{0.932401in}{0.451635in}}%
\pgfpathlineto{\pgfqpoint{0.970105in}{0.451635in}}%
\pgfpathlineto{\pgfqpoint{0.970105in}{0.452953in}}%
\pgfpathlineto{\pgfqpoint{1.007810in}{0.452953in}}%
\pgfpathlineto{\pgfqpoint{1.007810in}{0.452173in}}%
\pgfpathlineto{\pgfqpoint{1.045514in}{0.452173in}}%
\pgfpathlineto{\pgfqpoint{1.045514in}{0.455974in}}%
\pgfpathlineto{\pgfqpoint{1.083218in}{0.455974in}}%
\pgfpathlineto{\pgfqpoint{1.083218in}{0.456057in}}%
\pgfpathlineto{\pgfqpoint{1.120923in}{0.456057in}}%
\pgfpathlineto{\pgfqpoint{1.120923in}{0.463127in}}%
\pgfpathlineto{\pgfqpoint{1.158627in}{0.463127in}}%
\pgfpathlineto{\pgfqpoint{1.158627in}{0.463135in}}%
\pgfpathlineto{\pgfqpoint{1.196331in}{0.463135in}}%
\pgfpathlineto{\pgfqpoint{1.196331in}{0.475349in}}%
\pgfpathlineto{\pgfqpoint{1.234035in}{0.475349in}}%
\pgfpathlineto{\pgfqpoint{1.234035in}{0.478577in}}%
\pgfpathlineto{\pgfqpoint{1.271740in}{0.478577in}}%
\pgfpathlineto{\pgfqpoint{1.271740in}{0.488331in}}%
\pgfpathlineto{\pgfqpoint{1.309444in}{0.488331in}}%
\pgfpathlineto{\pgfqpoint{1.309444in}{0.507211in}}%
\pgfpathlineto{\pgfqpoint{1.347148in}{0.507211in}}%
\pgfpathlineto{\pgfqpoint{1.347148in}{0.512792in}}%
\pgfpathlineto{\pgfqpoint{1.384853in}{0.512792in}}%
\pgfpathlineto{\pgfqpoint{1.384853in}{0.532313in}}%
\pgfpathlineto{\pgfqpoint{1.422557in}{0.532313in}}%
\pgfpathlineto{\pgfqpoint{1.422557in}{0.555704in}}%
\pgfpathlineto{\pgfqpoint{1.460261in}{0.555704in}}%
\pgfpathlineto{\pgfqpoint{1.460261in}{0.583089in}}%
\pgfpathlineto{\pgfqpoint{1.497965in}{0.583089in}}%
\pgfpathlineto{\pgfqpoint{1.497965in}{0.621833in}}%
\pgfpathlineto{\pgfqpoint{1.535670in}{0.621833in}}%
\pgfpathlineto{\pgfqpoint{1.535670in}{0.643748in}}%
\pgfpathlineto{\pgfqpoint{1.573374in}{0.643748in}}%
\pgfpathlineto{\pgfqpoint{1.573374in}{0.693388in}}%
\pgfpathlineto{\pgfqpoint{1.611078in}{0.693388in}}%
\pgfpathlineto{\pgfqpoint{1.611078in}{0.742838in}}%
\pgfpathlineto{\pgfqpoint{1.648783in}{0.742838in}}%
\pgfpathlineto{\pgfqpoint{1.648783in}{0.776545in}}%
\pgfpathlineto{\pgfqpoint{1.686487in}{0.776545in}}%
\pgfpathlineto{\pgfqpoint{1.686487in}{0.806884in}}%
\pgfpathlineto{\pgfqpoint{1.724191in}{0.806884in}}%
\pgfpathlineto{\pgfqpoint{1.724191in}{0.857576in}}%
\pgfpathlineto{\pgfqpoint{1.761895in}{0.857576in}}%
\pgfpathlineto{\pgfqpoint{1.761895in}{0.871303in}}%
\pgfpathlineto{\pgfqpoint{1.799600in}{0.871303in}}%
\pgfpathlineto{\pgfqpoint{1.799600in}{0.927400in}}%
\pgfpathlineto{\pgfqpoint{1.837304in}{0.927400in}}%
\pgfpathlineto{\pgfqpoint{1.837304in}{0.994208in}}%
\pgfpathlineto{\pgfqpoint{1.875008in}{0.994208in}}%
\pgfpathlineto{\pgfqpoint{1.875008in}{1.069988in}}%
\pgfpathlineto{\pgfqpoint{1.912713in}{1.069988in}}%
\pgfpathlineto{\pgfqpoint{1.912713in}{1.141947in}}%
\pgfpathlineto{\pgfqpoint{1.950417in}{1.141947in}}%
\pgfpathlineto{\pgfqpoint{1.950417in}{1.265977in}}%
\pgfpathlineto{\pgfqpoint{1.988121in}{1.265977in}}%
\pgfpathlineto{\pgfqpoint{1.988121in}{1.339461in}}%
\pgfpathlineto{\pgfqpoint{2.025825in}{1.339461in}}%
\pgfpathlineto{\pgfqpoint{2.025825in}{1.464036in}}%
\pgfpathlineto{\pgfqpoint{2.063530in}{1.464036in}}%
\pgfpathlineto{\pgfqpoint{2.063530in}{1.461471in}}%
\pgfpathlineto{\pgfqpoint{2.101234in}{1.461471in}}%
\pgfpathlineto{\pgfqpoint{2.101234in}{1.106283in}}%
\pgfpathlineto{\pgfqpoint{2.138938in}{1.106283in}}%
\pgfpathlineto{\pgfqpoint{2.138938in}{0.799163in}}%
\pgfpathlineto{\pgfqpoint{2.176643in}{0.799163in}}%
\pgfpathlineto{\pgfqpoint{2.176643in}{0.662934in}}%
\pgfpathlineto{\pgfqpoint{2.214347in}{0.662934in}}%
\pgfpathlineto{\pgfqpoint{2.214347in}{0.540716in}}%
\pgfpathlineto{\pgfqpoint{2.252051in}{0.540716in}}%
\pgfpathlineto{\pgfqpoint{2.252051in}{0.483982in}}%
\pgfpathlineto{\pgfqpoint{2.289755in}{0.483982in}}%
\pgfpathlineto{\pgfqpoint{2.289755in}{0.470912in}}%
\pgfpathlineto{\pgfqpoint{2.327460in}{0.470912in}}%
\pgfpathlineto{\pgfqpoint{2.327460in}{0.460841in}}%
\pgfpathlineto{\pgfqpoint{2.365164in}{0.460841in}}%
\pgfpathlineto{\pgfqpoint{2.365164in}{0.452918in}}%
\pgfpathlineto{\pgfqpoint{2.402868in}{0.452918in}}%
\pgfpathlineto{\pgfqpoint{2.402868in}{0.450571in}}%
\pgfpathlineto{\pgfqpoint{2.440573in}{0.450571in}}%
\pgfpathlineto{\pgfqpoint{2.440573in}{0.449983in}}%
\pgfpathlineto{\pgfqpoint{2.402868in}{0.449983in}}%
\pgfpathlineto{\pgfqpoint{2.402868in}{0.449983in}}%
\pgfpathlineto{\pgfqpoint{2.365164in}{0.449983in}}%
\pgfpathlineto{\pgfqpoint{2.365164in}{0.449983in}}%
\pgfpathlineto{\pgfqpoint{2.327460in}{0.449983in}}%
\pgfpathlineto{\pgfqpoint{2.327460in}{0.449983in}}%
\pgfpathlineto{\pgfqpoint{2.289755in}{0.449983in}}%
\pgfpathlineto{\pgfqpoint{2.289755in}{0.449983in}}%
\pgfpathlineto{\pgfqpoint{2.252051in}{0.449983in}}%
\pgfpathlineto{\pgfqpoint{2.252051in}{0.449983in}}%
\pgfpathlineto{\pgfqpoint{2.214347in}{0.449983in}}%
\pgfpathlineto{\pgfqpoint{2.214347in}{0.449983in}}%
\pgfpathlineto{\pgfqpoint{2.176643in}{0.449983in}}%
\pgfpathlineto{\pgfqpoint{2.176643in}{0.449983in}}%
\pgfpathlineto{\pgfqpoint{2.138938in}{0.449983in}}%
\pgfpathlineto{\pgfqpoint{2.138938in}{0.449983in}}%
\pgfpathlineto{\pgfqpoint{2.101234in}{0.449983in}}%
\pgfpathlineto{\pgfqpoint{2.101234in}{0.449983in}}%
\pgfpathlineto{\pgfqpoint{2.063530in}{0.449983in}}%
\pgfpathlineto{\pgfqpoint{2.063530in}{0.449983in}}%
\pgfpathlineto{\pgfqpoint{2.025825in}{0.449983in}}%
\pgfpathlineto{\pgfqpoint{2.025825in}{0.449983in}}%
\pgfpathlineto{\pgfqpoint{1.988121in}{0.449983in}}%
\pgfpathlineto{\pgfqpoint{1.988121in}{0.449983in}}%
\pgfpathlineto{\pgfqpoint{1.950417in}{0.449983in}}%
\pgfpathlineto{\pgfqpoint{1.950417in}{0.449983in}}%
\pgfpathlineto{\pgfqpoint{1.912713in}{0.449983in}}%
\pgfpathlineto{\pgfqpoint{1.912713in}{0.449983in}}%
\pgfpathlineto{\pgfqpoint{1.875008in}{0.449983in}}%
\pgfpathlineto{\pgfqpoint{1.875008in}{0.449983in}}%
\pgfpathlineto{\pgfqpoint{1.837304in}{0.449983in}}%
\pgfpathlineto{\pgfqpoint{1.837304in}{0.449983in}}%
\pgfpathlineto{\pgfqpoint{1.799600in}{0.449983in}}%
\pgfpathlineto{\pgfqpoint{1.799600in}{0.449983in}}%
\pgfpathlineto{\pgfqpoint{1.761895in}{0.449983in}}%
\pgfpathlineto{\pgfqpoint{1.761895in}{0.449983in}}%
\pgfpathlineto{\pgfqpoint{1.724191in}{0.449983in}}%
\pgfpathlineto{\pgfqpoint{1.724191in}{0.449983in}}%
\pgfpathlineto{\pgfqpoint{1.686487in}{0.449983in}}%
\pgfpathlineto{\pgfqpoint{1.686487in}{0.449983in}}%
\pgfpathlineto{\pgfqpoint{1.648783in}{0.449983in}}%
\pgfpathlineto{\pgfqpoint{1.648783in}{0.449983in}}%
\pgfpathlineto{\pgfqpoint{1.611078in}{0.449983in}}%
\pgfpathlineto{\pgfqpoint{1.611078in}{0.449983in}}%
\pgfpathlineto{\pgfqpoint{1.573374in}{0.449983in}}%
\pgfpathlineto{\pgfqpoint{1.573374in}{0.449983in}}%
\pgfpathlineto{\pgfqpoint{1.535670in}{0.449983in}}%
\pgfpathlineto{\pgfqpoint{1.535670in}{0.449983in}}%
\pgfpathlineto{\pgfqpoint{1.497965in}{0.449983in}}%
\pgfpathlineto{\pgfqpoint{1.497965in}{0.449983in}}%
\pgfpathlineto{\pgfqpoint{1.460261in}{0.449983in}}%
\pgfpathlineto{\pgfqpoint{1.460261in}{0.449983in}}%
\pgfpathlineto{\pgfqpoint{1.422557in}{0.449983in}}%
\pgfpathlineto{\pgfqpoint{1.422557in}{0.449983in}}%
\pgfpathlineto{\pgfqpoint{1.384853in}{0.449983in}}%
\pgfpathlineto{\pgfqpoint{1.384853in}{0.449983in}}%
\pgfpathlineto{\pgfqpoint{1.347148in}{0.449983in}}%
\pgfpathlineto{\pgfqpoint{1.347148in}{0.449983in}}%
\pgfpathlineto{\pgfqpoint{1.309444in}{0.449983in}}%
\pgfpathlineto{\pgfqpoint{1.309444in}{0.449983in}}%
\pgfpathlineto{\pgfqpoint{1.271740in}{0.449983in}}%
\pgfpathlineto{\pgfqpoint{1.271740in}{0.449983in}}%
\pgfpathlineto{\pgfqpoint{1.234035in}{0.449983in}}%
\pgfpathlineto{\pgfqpoint{1.234035in}{0.449983in}}%
\pgfpathlineto{\pgfqpoint{1.196331in}{0.449983in}}%
\pgfpathlineto{\pgfqpoint{1.196331in}{0.449983in}}%
\pgfpathlineto{\pgfqpoint{1.158627in}{0.449983in}}%
\pgfpathlineto{\pgfqpoint{1.158627in}{0.449983in}}%
\pgfpathlineto{\pgfqpoint{1.120923in}{0.449983in}}%
\pgfpathlineto{\pgfqpoint{1.120923in}{0.449983in}}%
\pgfpathlineto{\pgfqpoint{1.083218in}{0.449983in}}%
\pgfpathlineto{\pgfqpoint{1.083218in}{0.449983in}}%
\pgfpathlineto{\pgfqpoint{1.045514in}{0.449983in}}%
\pgfpathlineto{\pgfqpoint{1.045514in}{0.449983in}}%
\pgfpathlineto{\pgfqpoint{1.007810in}{0.449983in}}%
\pgfpathlineto{\pgfqpoint{1.007810in}{0.449983in}}%
\pgfpathlineto{\pgfqpoint{0.970105in}{0.449983in}}%
\pgfpathlineto{\pgfqpoint{0.970105in}{0.449983in}}%
\pgfpathlineto{\pgfqpoint{0.932401in}{0.449983in}}%
\pgfpathlineto{\pgfqpoint{0.932401in}{0.449983in}}%
\pgfpathlineto{\pgfqpoint{0.894697in}{0.449983in}}%
\pgfpathlineto{\pgfqpoint{0.894697in}{0.449983in}}%
\pgfpathlineto{\pgfqpoint{0.856993in}{0.449983in}}%
\pgfpathlineto{\pgfqpoint{0.856993in}{0.449983in}}%
\pgfpathlineto{\pgfqpoint{0.819288in}{0.449983in}}%
\pgfpathlineto{\pgfqpoint{0.819288in}{0.449983in}}%
\pgfpathlineto{\pgfqpoint{0.781584in}{0.449983in}}%
\pgfpathlineto{\pgfqpoint{0.781584in}{0.449983in}}%
\pgfpathlineto{\pgfqpoint{0.743880in}{0.449983in}}%
\pgfpathlineto{\pgfqpoint{0.743880in}{0.449983in}}%
\pgfpathlineto{\pgfqpoint{0.706176in}{0.449983in}}%
\pgfpathlineto{\pgfqpoint{0.706176in}{0.449983in}}%
\pgfpathlineto{\pgfqpoint{0.668471in}{0.449983in}}%
\pgfpathlineto{\pgfqpoint{0.668471in}{0.449983in}}%
\pgfpathlineto{\pgfqpoint{0.630767in}{0.449983in}}%
\pgfpathlineto{\pgfqpoint{0.630767in}{0.449983in}}%
\pgfpathlineto{\pgfqpoint{0.593063in}{0.449983in}}%
\pgfpathlineto{\pgfqpoint{0.593063in}{0.449983in}}%
\pgfpathlineto{\pgfqpoint{0.555358in}{0.449983in}}%
\pgfusepath{fill}%
\end{pgfscope}%
\begin{pgfscope}%
\pgfpathrectangle{\pgfqpoint{0.437532in}{0.449983in}}{\pgfqpoint{2.120866in}{1.169449in}} %
\pgfusepath{clip}%
\pgfsetbuttcap%
\pgfsetmiterjoin%
\pgfsetlinewidth{0.501875pt}%
\definecolor{currentstroke}{rgb}{1.000000,0.000000,0.000000}%
\pgfsetstrokecolor{currentstroke}%
\pgfsetdash{}{0pt}%
\pgfpathmoveto{\pgfqpoint{0.555358in}{0.449983in}}%
\pgfpathlineto{\pgfqpoint{0.555358in}{0.450009in}}%
\pgfpathlineto{\pgfqpoint{0.593063in}{0.450009in}}%
\pgfpathlineto{\pgfqpoint{0.593063in}{0.450034in}}%
\pgfpathlineto{\pgfqpoint{0.630767in}{0.450034in}}%
\pgfpathlineto{\pgfqpoint{0.630767in}{0.450111in}}%
\pgfpathlineto{\pgfqpoint{0.668471in}{0.450111in}}%
\pgfpathlineto{\pgfqpoint{0.668471in}{0.450034in}}%
\pgfpathlineto{\pgfqpoint{0.706176in}{0.450034in}}%
\pgfpathlineto{\pgfqpoint{0.706176in}{0.450162in}}%
\pgfpathlineto{\pgfqpoint{0.743880in}{0.450162in}}%
\pgfpathlineto{\pgfqpoint{0.743880in}{0.450187in}}%
\pgfpathlineto{\pgfqpoint{0.781584in}{0.450187in}}%
\pgfpathlineto{\pgfqpoint{0.781584in}{0.450289in}}%
\pgfpathlineto{\pgfqpoint{0.819288in}{0.450289in}}%
\pgfpathlineto{\pgfqpoint{0.819288in}{0.450442in}}%
\pgfpathlineto{\pgfqpoint{0.856993in}{0.450442in}}%
\pgfpathlineto{\pgfqpoint{0.856993in}{0.450442in}}%
\pgfpathlineto{\pgfqpoint{0.894697in}{0.450442in}}%
\pgfpathlineto{\pgfqpoint{0.894697in}{0.450748in}}%
\pgfpathlineto{\pgfqpoint{0.932401in}{0.450748in}}%
\pgfpathlineto{\pgfqpoint{0.932401in}{0.450977in}}%
\pgfpathlineto{\pgfqpoint{0.970105in}{0.450977in}}%
\pgfpathlineto{\pgfqpoint{0.970105in}{0.451028in}}%
\pgfpathlineto{\pgfqpoint{1.007810in}{0.451028in}}%
\pgfpathlineto{\pgfqpoint{1.007810in}{0.451665in}}%
\pgfpathlineto{\pgfqpoint{1.045514in}{0.451665in}}%
\pgfpathlineto{\pgfqpoint{1.045514in}{0.452557in}}%
\pgfpathlineto{\pgfqpoint{1.083218in}{0.452557in}}%
\pgfpathlineto{\pgfqpoint{1.083218in}{0.452786in}}%
\pgfpathlineto{\pgfqpoint{1.120923in}{0.452786in}}%
\pgfpathlineto{\pgfqpoint{1.120923in}{0.453831in}}%
\pgfpathlineto{\pgfqpoint{1.158627in}{0.453831in}}%
\pgfpathlineto{\pgfqpoint{1.158627in}{0.454672in}}%
\pgfpathlineto{\pgfqpoint{1.196331in}{0.454672in}}%
\pgfpathlineto{\pgfqpoint{1.196331in}{0.456507in}}%
\pgfpathlineto{\pgfqpoint{1.234035in}{0.456507in}}%
\pgfpathlineto{\pgfqpoint{1.234035in}{0.459896in}}%
\pgfpathlineto{\pgfqpoint{1.271740in}{0.459896in}}%
\pgfpathlineto{\pgfqpoint{1.271740in}{0.462902in}}%
\pgfpathlineto{\pgfqpoint{1.309444in}{0.462902in}}%
\pgfpathlineto{\pgfqpoint{1.309444in}{0.465374in}}%
\pgfpathlineto{\pgfqpoint{1.347148in}{0.465374in}}%
\pgfpathlineto{\pgfqpoint{1.347148in}{0.471031in}}%
\pgfpathlineto{\pgfqpoint{1.384853in}{0.471031in}}%
\pgfpathlineto{\pgfqpoint{1.384853in}{0.477427in}}%
\pgfpathlineto{\pgfqpoint{1.422557in}{0.477427in}}%
\pgfpathlineto{\pgfqpoint{1.422557in}{0.484689in}}%
\pgfpathlineto{\pgfqpoint{1.460261in}{0.484689in}}%
\pgfpathlineto{\pgfqpoint{1.460261in}{0.497047in}}%
\pgfpathlineto{\pgfqpoint{1.497965in}{0.497047in}}%
\pgfpathlineto{\pgfqpoint{1.497965in}{0.508310in}}%
\pgfpathlineto{\pgfqpoint{1.535670in}{0.508310in}}%
\pgfpathlineto{\pgfqpoint{1.535670in}{0.527319in}}%
\pgfpathlineto{\pgfqpoint{1.573374in}{0.527319in}}%
\pgfpathlineto{\pgfqpoint{1.573374in}{0.543627in}}%
\pgfpathlineto{\pgfqpoint{1.611078in}{0.543627in}}%
\pgfpathlineto{\pgfqpoint{1.611078in}{0.567757in}}%
\pgfpathlineto{\pgfqpoint{1.648783in}{0.567757in}}%
\pgfpathlineto{\pgfqpoint{1.648783in}{0.599074in}}%
\pgfpathlineto{\pgfqpoint{1.686487in}{0.599074in}}%
\pgfpathlineto{\pgfqpoint{1.686487in}{0.627791in}}%
\pgfpathlineto{\pgfqpoint{1.724191in}{0.627791in}}%
\pgfpathlineto{\pgfqpoint{1.724191in}{0.664127in}}%
\pgfpathlineto{\pgfqpoint{1.761895in}{0.664127in}}%
\pgfpathlineto{\pgfqpoint{1.761895in}{0.703699in}}%
\pgfpathlineto{\pgfqpoint{1.799600in}{0.703699in}}%
\pgfpathlineto{\pgfqpoint{1.799600in}{0.754610in}}%
\pgfpathlineto{\pgfqpoint{1.837304in}{0.754610in}}%
\pgfpathlineto{\pgfqpoint{1.837304in}{0.796272in}}%
\pgfpathlineto{\pgfqpoint{1.875008in}{0.796272in}}%
\pgfpathlineto{\pgfqpoint{1.875008in}{0.848916in}}%
\pgfpathlineto{\pgfqpoint{1.912713in}{0.848916in}}%
\pgfpathlineto{\pgfqpoint{1.912713in}{0.919066in}}%
\pgfpathlineto{\pgfqpoint{1.950417in}{0.919066in}}%
\pgfpathlineto{\pgfqpoint{1.950417in}{1.009141in}}%
\pgfpathlineto{\pgfqpoint{1.988121in}{1.009141in}}%
\pgfpathlineto{\pgfqpoint{1.988121in}{1.112340in}}%
\pgfpathlineto{\pgfqpoint{2.025825in}{1.112340in}}%
\pgfpathlineto{\pgfqpoint{2.025825in}{1.245046in}}%
\pgfpathlineto{\pgfqpoint{2.063530in}{1.245046in}}%
\pgfpathlineto{\pgfqpoint{2.063530in}{1.412661in}}%
\pgfpathlineto{\pgfqpoint{2.101234in}{1.412661in}}%
\pgfpathlineto{\pgfqpoint{2.101234in}{1.394263in}}%
\pgfpathlineto{\pgfqpoint{2.138938in}{1.394263in}}%
\pgfpathlineto{\pgfqpoint{2.138938in}{1.226801in}}%
\pgfpathlineto{\pgfqpoint{2.176643in}{1.226801in}}%
\pgfpathlineto{\pgfqpoint{2.176643in}{1.080030in}}%
\pgfpathlineto{\pgfqpoint{2.214347in}{1.080030in}}%
\pgfpathlineto{\pgfqpoint{2.214347in}{0.996987in}}%
\pgfpathlineto{\pgfqpoint{2.252051in}{0.996987in}}%
\pgfpathlineto{\pgfqpoint{2.252051in}{0.922633in}}%
\pgfpathlineto{\pgfqpoint{2.289755in}{0.922633in}}%
\pgfpathlineto{\pgfqpoint{2.289755in}{0.859924in}}%
\pgfpathlineto{\pgfqpoint{2.327460in}{0.859924in}}%
\pgfpathlineto{\pgfqpoint{2.327460in}{0.786564in}}%
\pgfpathlineto{\pgfqpoint{2.365164in}{0.786564in}}%
\pgfpathlineto{\pgfqpoint{2.365164in}{0.722708in}}%
\pgfpathlineto{\pgfqpoint{2.402868in}{0.722708in}}%
\pgfpathlineto{\pgfqpoint{2.402868in}{0.666165in}}%
\pgfpathlineto{\pgfqpoint{2.440573in}{0.666165in}}%
\pgfpathlineto{\pgfqpoint{2.440573in}{0.449983in}}%
\pgfusepath{stroke}%
\end{pgfscope}%
\begin{pgfscope}%
\pgfsetrectcap%
\pgfsetmiterjoin%
\pgfsetlinewidth{1.003750pt}%
\definecolor{currentstroke}{rgb}{0.000000,0.000000,0.000000}%
\pgfsetstrokecolor{currentstroke}%
\pgfsetdash{}{0pt}%
\pgfpathmoveto{\pgfqpoint{0.437532in}{1.619432in}}%
\pgfpathlineto{\pgfqpoint{2.558398in}{1.619432in}}%
\pgfusepath{stroke}%
\end{pgfscope}%
\begin{pgfscope}%
\pgfsetrectcap%
\pgfsetmiterjoin%
\pgfsetlinewidth{1.003750pt}%
\definecolor{currentstroke}{rgb}{0.000000,0.000000,0.000000}%
\pgfsetstrokecolor{currentstroke}%
\pgfsetdash{}{0pt}%
\pgfpathmoveto{\pgfqpoint{2.558398in}{0.449983in}}%
\pgfpathlineto{\pgfqpoint{2.558398in}{1.619432in}}%
\pgfusepath{stroke}%
\end{pgfscope}%
\begin{pgfscope}%
\pgfsetrectcap%
\pgfsetmiterjoin%
\pgfsetlinewidth{1.003750pt}%
\definecolor{currentstroke}{rgb}{0.000000,0.000000,0.000000}%
\pgfsetstrokecolor{currentstroke}%
\pgfsetdash{}{0pt}%
\pgfpathmoveto{\pgfqpoint{0.437532in}{0.449983in}}%
\pgfpathlineto{\pgfqpoint{2.558398in}{0.449983in}}%
\pgfusepath{stroke}%
\end{pgfscope}%
\begin{pgfscope}%
\pgfsetrectcap%
\pgfsetmiterjoin%
\pgfsetlinewidth{1.003750pt}%
\definecolor{currentstroke}{rgb}{0.000000,0.000000,0.000000}%
\pgfsetstrokecolor{currentstroke}%
\pgfsetdash{}{0pt}%
\pgfpathmoveto{\pgfqpoint{0.437532in}{0.449983in}}%
\pgfpathlineto{\pgfqpoint{0.437532in}{1.619432in}}%
\pgfusepath{stroke}%
\end{pgfscope}%
\begin{pgfscope}%
\pgfsetbuttcap%
\pgfsetroundjoin%
\definecolor{currentfill}{rgb}{0.000000,0.000000,0.000000}%
\pgfsetfillcolor{currentfill}%
\pgfsetlinewidth{0.501875pt}%
\definecolor{currentstroke}{rgb}{0.000000,0.000000,0.000000}%
\pgfsetstrokecolor{currentstroke}%
\pgfsetdash{}{0pt}%
\pgfsys@defobject{currentmarker}{\pgfqpoint{0.000000in}{0.000000in}}{\pgfqpoint{0.000000in}{0.069444in}}{%
\pgfpathmoveto{\pgfqpoint{0.000000in}{0.000000in}}%
\pgfpathlineto{\pgfqpoint{0.000000in}{0.069444in}}%
\pgfusepath{stroke,fill}%
}%
\begin{pgfscope}%
\pgfsys@transformshift{0.437532in}{0.449983in}%
\pgfsys@useobject{currentmarker}{}%
\end{pgfscope}%
\end{pgfscope}%
\begin{pgfscope}%
\pgfsetbuttcap%
\pgfsetroundjoin%
\definecolor{currentfill}{rgb}{0.000000,0.000000,0.000000}%
\pgfsetfillcolor{currentfill}%
\pgfsetlinewidth{0.501875pt}%
\definecolor{currentstroke}{rgb}{0.000000,0.000000,0.000000}%
\pgfsetstrokecolor{currentstroke}%
\pgfsetdash{}{0pt}%
\pgfsys@defobject{currentmarker}{\pgfqpoint{0.000000in}{-0.069444in}}{\pgfqpoint{0.000000in}{0.000000in}}{%
\pgfpathmoveto{\pgfqpoint{0.000000in}{0.000000in}}%
\pgfpathlineto{\pgfqpoint{0.000000in}{-0.069444in}}%
\pgfusepath{stroke,fill}%
}%
\begin{pgfscope}%
\pgfsys@transformshift{0.437532in}{1.619432in}%
\pgfsys@useobject{currentmarker}{}%
\end{pgfscope}%
\end{pgfscope}%
\begin{pgfscope}%
\pgftext[x=0.437532in,y=0.380539in,,top]{\rmfamily\fontsize{8.000000}{9.600000}\selectfont −140}%
\end{pgfscope}%
\begin{pgfscope}%
\pgfsetbuttcap%
\pgfsetroundjoin%
\definecolor{currentfill}{rgb}{0.000000,0.000000,0.000000}%
\pgfsetfillcolor{currentfill}%
\pgfsetlinewidth{0.501875pt}%
\definecolor{currentstroke}{rgb}{0.000000,0.000000,0.000000}%
\pgfsetstrokecolor{currentstroke}%
\pgfsetdash{}{0pt}%
\pgfsys@defobject{currentmarker}{\pgfqpoint{0.000000in}{0.000000in}}{\pgfqpoint{0.000000in}{0.069444in}}{%
\pgfpathmoveto{\pgfqpoint{0.000000in}{0.000000in}}%
\pgfpathlineto{\pgfqpoint{0.000000in}{0.069444in}}%
\pgfusepath{stroke,fill}%
}%
\begin{pgfscope}%
\pgfsys@transformshift{0.673184in}{0.449983in}%
\pgfsys@useobject{currentmarker}{}%
\end{pgfscope}%
\end{pgfscope}%
\begin{pgfscope}%
\pgfsetbuttcap%
\pgfsetroundjoin%
\definecolor{currentfill}{rgb}{0.000000,0.000000,0.000000}%
\pgfsetfillcolor{currentfill}%
\pgfsetlinewidth{0.501875pt}%
\definecolor{currentstroke}{rgb}{0.000000,0.000000,0.000000}%
\pgfsetstrokecolor{currentstroke}%
\pgfsetdash{}{0pt}%
\pgfsys@defobject{currentmarker}{\pgfqpoint{0.000000in}{-0.069444in}}{\pgfqpoint{0.000000in}{0.000000in}}{%
\pgfpathmoveto{\pgfqpoint{0.000000in}{0.000000in}}%
\pgfpathlineto{\pgfqpoint{0.000000in}{-0.069444in}}%
\pgfusepath{stroke,fill}%
}%
\begin{pgfscope}%
\pgfsys@transformshift{0.673184in}{1.619432in}%
\pgfsys@useobject{currentmarker}{}%
\end{pgfscope}%
\end{pgfscope}%
\begin{pgfscope}%
\pgftext[x=0.673184in,y=0.380539in,,top]{\rmfamily\fontsize{8.000000}{9.600000}\selectfont −120}%
\end{pgfscope}%
\begin{pgfscope}%
\pgfsetbuttcap%
\pgfsetroundjoin%
\definecolor{currentfill}{rgb}{0.000000,0.000000,0.000000}%
\pgfsetfillcolor{currentfill}%
\pgfsetlinewidth{0.501875pt}%
\definecolor{currentstroke}{rgb}{0.000000,0.000000,0.000000}%
\pgfsetstrokecolor{currentstroke}%
\pgfsetdash{}{0pt}%
\pgfsys@defobject{currentmarker}{\pgfqpoint{0.000000in}{0.000000in}}{\pgfqpoint{0.000000in}{0.069444in}}{%
\pgfpathmoveto{\pgfqpoint{0.000000in}{0.000000in}}%
\pgfpathlineto{\pgfqpoint{0.000000in}{0.069444in}}%
\pgfusepath{stroke,fill}%
}%
\begin{pgfscope}%
\pgfsys@transformshift{0.908836in}{0.449983in}%
\pgfsys@useobject{currentmarker}{}%
\end{pgfscope}%
\end{pgfscope}%
\begin{pgfscope}%
\pgfsetbuttcap%
\pgfsetroundjoin%
\definecolor{currentfill}{rgb}{0.000000,0.000000,0.000000}%
\pgfsetfillcolor{currentfill}%
\pgfsetlinewidth{0.501875pt}%
\definecolor{currentstroke}{rgb}{0.000000,0.000000,0.000000}%
\pgfsetstrokecolor{currentstroke}%
\pgfsetdash{}{0pt}%
\pgfsys@defobject{currentmarker}{\pgfqpoint{0.000000in}{-0.069444in}}{\pgfqpoint{0.000000in}{0.000000in}}{%
\pgfpathmoveto{\pgfqpoint{0.000000in}{0.000000in}}%
\pgfpathlineto{\pgfqpoint{0.000000in}{-0.069444in}}%
\pgfusepath{stroke,fill}%
}%
\begin{pgfscope}%
\pgfsys@transformshift{0.908836in}{1.619432in}%
\pgfsys@useobject{currentmarker}{}%
\end{pgfscope}%
\end{pgfscope}%
\begin{pgfscope}%
\pgftext[x=0.908836in,y=0.380539in,,top]{\rmfamily\fontsize{8.000000}{9.600000}\selectfont −100}%
\end{pgfscope}%
\begin{pgfscope}%
\pgfsetbuttcap%
\pgfsetroundjoin%
\definecolor{currentfill}{rgb}{0.000000,0.000000,0.000000}%
\pgfsetfillcolor{currentfill}%
\pgfsetlinewidth{0.501875pt}%
\definecolor{currentstroke}{rgb}{0.000000,0.000000,0.000000}%
\pgfsetstrokecolor{currentstroke}%
\pgfsetdash{}{0pt}%
\pgfsys@defobject{currentmarker}{\pgfqpoint{0.000000in}{0.000000in}}{\pgfqpoint{0.000000in}{0.069444in}}{%
\pgfpathmoveto{\pgfqpoint{0.000000in}{0.000000in}}%
\pgfpathlineto{\pgfqpoint{0.000000in}{0.069444in}}%
\pgfusepath{stroke,fill}%
}%
\begin{pgfscope}%
\pgfsys@transformshift{1.144488in}{0.449983in}%
\pgfsys@useobject{currentmarker}{}%
\end{pgfscope}%
\end{pgfscope}%
\begin{pgfscope}%
\pgfsetbuttcap%
\pgfsetroundjoin%
\definecolor{currentfill}{rgb}{0.000000,0.000000,0.000000}%
\pgfsetfillcolor{currentfill}%
\pgfsetlinewidth{0.501875pt}%
\definecolor{currentstroke}{rgb}{0.000000,0.000000,0.000000}%
\pgfsetstrokecolor{currentstroke}%
\pgfsetdash{}{0pt}%
\pgfsys@defobject{currentmarker}{\pgfqpoint{0.000000in}{-0.069444in}}{\pgfqpoint{0.000000in}{0.000000in}}{%
\pgfpathmoveto{\pgfqpoint{0.000000in}{0.000000in}}%
\pgfpathlineto{\pgfqpoint{0.000000in}{-0.069444in}}%
\pgfusepath{stroke,fill}%
}%
\begin{pgfscope}%
\pgfsys@transformshift{1.144488in}{1.619432in}%
\pgfsys@useobject{currentmarker}{}%
\end{pgfscope}%
\end{pgfscope}%
\begin{pgfscope}%
\pgftext[x=1.144488in,y=0.380539in,,top]{\rmfamily\fontsize{8.000000}{9.600000}\selectfont −80}%
\end{pgfscope}%
\begin{pgfscope}%
\pgfsetbuttcap%
\pgfsetroundjoin%
\definecolor{currentfill}{rgb}{0.000000,0.000000,0.000000}%
\pgfsetfillcolor{currentfill}%
\pgfsetlinewidth{0.501875pt}%
\definecolor{currentstroke}{rgb}{0.000000,0.000000,0.000000}%
\pgfsetstrokecolor{currentstroke}%
\pgfsetdash{}{0pt}%
\pgfsys@defobject{currentmarker}{\pgfqpoint{0.000000in}{0.000000in}}{\pgfqpoint{0.000000in}{0.069444in}}{%
\pgfpathmoveto{\pgfqpoint{0.000000in}{0.000000in}}%
\pgfpathlineto{\pgfqpoint{0.000000in}{0.069444in}}%
\pgfusepath{stroke,fill}%
}%
\begin{pgfscope}%
\pgfsys@transformshift{1.380140in}{0.449983in}%
\pgfsys@useobject{currentmarker}{}%
\end{pgfscope}%
\end{pgfscope}%
\begin{pgfscope}%
\pgfsetbuttcap%
\pgfsetroundjoin%
\definecolor{currentfill}{rgb}{0.000000,0.000000,0.000000}%
\pgfsetfillcolor{currentfill}%
\pgfsetlinewidth{0.501875pt}%
\definecolor{currentstroke}{rgb}{0.000000,0.000000,0.000000}%
\pgfsetstrokecolor{currentstroke}%
\pgfsetdash{}{0pt}%
\pgfsys@defobject{currentmarker}{\pgfqpoint{0.000000in}{-0.069444in}}{\pgfqpoint{0.000000in}{0.000000in}}{%
\pgfpathmoveto{\pgfqpoint{0.000000in}{0.000000in}}%
\pgfpathlineto{\pgfqpoint{0.000000in}{-0.069444in}}%
\pgfusepath{stroke,fill}%
}%
\begin{pgfscope}%
\pgfsys@transformshift{1.380140in}{1.619432in}%
\pgfsys@useobject{currentmarker}{}%
\end{pgfscope}%
\end{pgfscope}%
\begin{pgfscope}%
\pgftext[x=1.380140in,y=0.380539in,,top]{\rmfamily\fontsize{8.000000}{9.600000}\selectfont −60}%
\end{pgfscope}%
\begin{pgfscope}%
\pgfsetbuttcap%
\pgfsetroundjoin%
\definecolor{currentfill}{rgb}{0.000000,0.000000,0.000000}%
\pgfsetfillcolor{currentfill}%
\pgfsetlinewidth{0.501875pt}%
\definecolor{currentstroke}{rgb}{0.000000,0.000000,0.000000}%
\pgfsetstrokecolor{currentstroke}%
\pgfsetdash{}{0pt}%
\pgfsys@defobject{currentmarker}{\pgfqpoint{0.000000in}{0.000000in}}{\pgfqpoint{0.000000in}{0.069444in}}{%
\pgfpathmoveto{\pgfqpoint{0.000000in}{0.000000in}}%
\pgfpathlineto{\pgfqpoint{0.000000in}{0.069444in}}%
\pgfusepath{stroke,fill}%
}%
\begin{pgfscope}%
\pgfsys@transformshift{1.615791in}{0.449983in}%
\pgfsys@useobject{currentmarker}{}%
\end{pgfscope}%
\end{pgfscope}%
\begin{pgfscope}%
\pgfsetbuttcap%
\pgfsetroundjoin%
\definecolor{currentfill}{rgb}{0.000000,0.000000,0.000000}%
\pgfsetfillcolor{currentfill}%
\pgfsetlinewidth{0.501875pt}%
\definecolor{currentstroke}{rgb}{0.000000,0.000000,0.000000}%
\pgfsetstrokecolor{currentstroke}%
\pgfsetdash{}{0pt}%
\pgfsys@defobject{currentmarker}{\pgfqpoint{0.000000in}{-0.069444in}}{\pgfqpoint{0.000000in}{0.000000in}}{%
\pgfpathmoveto{\pgfqpoint{0.000000in}{0.000000in}}%
\pgfpathlineto{\pgfqpoint{0.000000in}{-0.069444in}}%
\pgfusepath{stroke,fill}%
}%
\begin{pgfscope}%
\pgfsys@transformshift{1.615791in}{1.619432in}%
\pgfsys@useobject{currentmarker}{}%
\end{pgfscope}%
\end{pgfscope}%
\begin{pgfscope}%
\pgftext[x=1.615791in,y=0.380539in,,top]{\rmfamily\fontsize{8.000000}{9.600000}\selectfont −40}%
\end{pgfscope}%
\begin{pgfscope}%
\pgfsetbuttcap%
\pgfsetroundjoin%
\definecolor{currentfill}{rgb}{0.000000,0.000000,0.000000}%
\pgfsetfillcolor{currentfill}%
\pgfsetlinewidth{0.501875pt}%
\definecolor{currentstroke}{rgb}{0.000000,0.000000,0.000000}%
\pgfsetstrokecolor{currentstroke}%
\pgfsetdash{}{0pt}%
\pgfsys@defobject{currentmarker}{\pgfqpoint{0.000000in}{0.000000in}}{\pgfqpoint{0.000000in}{0.069444in}}{%
\pgfpathmoveto{\pgfqpoint{0.000000in}{0.000000in}}%
\pgfpathlineto{\pgfqpoint{0.000000in}{0.069444in}}%
\pgfusepath{stroke,fill}%
}%
\begin{pgfscope}%
\pgfsys@transformshift{1.851443in}{0.449983in}%
\pgfsys@useobject{currentmarker}{}%
\end{pgfscope}%
\end{pgfscope}%
\begin{pgfscope}%
\pgfsetbuttcap%
\pgfsetroundjoin%
\definecolor{currentfill}{rgb}{0.000000,0.000000,0.000000}%
\pgfsetfillcolor{currentfill}%
\pgfsetlinewidth{0.501875pt}%
\definecolor{currentstroke}{rgb}{0.000000,0.000000,0.000000}%
\pgfsetstrokecolor{currentstroke}%
\pgfsetdash{}{0pt}%
\pgfsys@defobject{currentmarker}{\pgfqpoint{0.000000in}{-0.069444in}}{\pgfqpoint{0.000000in}{0.000000in}}{%
\pgfpathmoveto{\pgfqpoint{0.000000in}{0.000000in}}%
\pgfpathlineto{\pgfqpoint{0.000000in}{-0.069444in}}%
\pgfusepath{stroke,fill}%
}%
\begin{pgfscope}%
\pgfsys@transformshift{1.851443in}{1.619432in}%
\pgfsys@useobject{currentmarker}{}%
\end{pgfscope}%
\end{pgfscope}%
\begin{pgfscope}%
\pgftext[x=1.851443in,y=0.380539in,,top]{\rmfamily\fontsize{8.000000}{9.600000}\selectfont −20}%
\end{pgfscope}%
\begin{pgfscope}%
\pgfsetbuttcap%
\pgfsetroundjoin%
\definecolor{currentfill}{rgb}{0.000000,0.000000,0.000000}%
\pgfsetfillcolor{currentfill}%
\pgfsetlinewidth{0.501875pt}%
\definecolor{currentstroke}{rgb}{0.000000,0.000000,0.000000}%
\pgfsetstrokecolor{currentstroke}%
\pgfsetdash{}{0pt}%
\pgfsys@defobject{currentmarker}{\pgfqpoint{0.000000in}{0.000000in}}{\pgfqpoint{0.000000in}{0.069444in}}{%
\pgfpathmoveto{\pgfqpoint{0.000000in}{0.000000in}}%
\pgfpathlineto{\pgfqpoint{0.000000in}{0.069444in}}%
\pgfusepath{stroke,fill}%
}%
\begin{pgfscope}%
\pgfsys@transformshift{2.087095in}{0.449983in}%
\pgfsys@useobject{currentmarker}{}%
\end{pgfscope}%
\end{pgfscope}%
\begin{pgfscope}%
\pgfsetbuttcap%
\pgfsetroundjoin%
\definecolor{currentfill}{rgb}{0.000000,0.000000,0.000000}%
\pgfsetfillcolor{currentfill}%
\pgfsetlinewidth{0.501875pt}%
\definecolor{currentstroke}{rgb}{0.000000,0.000000,0.000000}%
\pgfsetstrokecolor{currentstroke}%
\pgfsetdash{}{0pt}%
\pgfsys@defobject{currentmarker}{\pgfqpoint{0.000000in}{-0.069444in}}{\pgfqpoint{0.000000in}{0.000000in}}{%
\pgfpathmoveto{\pgfqpoint{0.000000in}{0.000000in}}%
\pgfpathlineto{\pgfqpoint{0.000000in}{-0.069444in}}%
\pgfusepath{stroke,fill}%
}%
\begin{pgfscope}%
\pgfsys@transformshift{2.087095in}{1.619432in}%
\pgfsys@useobject{currentmarker}{}%
\end{pgfscope}%
\end{pgfscope}%
\begin{pgfscope}%
\pgftext[x=2.087095in,y=0.380539in,,top]{\rmfamily\fontsize{8.000000}{9.600000}\selectfont 0}%
\end{pgfscope}%
\begin{pgfscope}%
\pgfsetbuttcap%
\pgfsetroundjoin%
\definecolor{currentfill}{rgb}{0.000000,0.000000,0.000000}%
\pgfsetfillcolor{currentfill}%
\pgfsetlinewidth{0.501875pt}%
\definecolor{currentstroke}{rgb}{0.000000,0.000000,0.000000}%
\pgfsetstrokecolor{currentstroke}%
\pgfsetdash{}{0pt}%
\pgfsys@defobject{currentmarker}{\pgfqpoint{0.000000in}{0.000000in}}{\pgfqpoint{0.000000in}{0.069444in}}{%
\pgfpathmoveto{\pgfqpoint{0.000000in}{0.000000in}}%
\pgfpathlineto{\pgfqpoint{0.000000in}{0.069444in}}%
\pgfusepath{stroke,fill}%
}%
\begin{pgfscope}%
\pgfsys@transformshift{2.322747in}{0.449983in}%
\pgfsys@useobject{currentmarker}{}%
\end{pgfscope}%
\end{pgfscope}%
\begin{pgfscope}%
\pgfsetbuttcap%
\pgfsetroundjoin%
\definecolor{currentfill}{rgb}{0.000000,0.000000,0.000000}%
\pgfsetfillcolor{currentfill}%
\pgfsetlinewidth{0.501875pt}%
\definecolor{currentstroke}{rgb}{0.000000,0.000000,0.000000}%
\pgfsetstrokecolor{currentstroke}%
\pgfsetdash{}{0pt}%
\pgfsys@defobject{currentmarker}{\pgfqpoint{0.000000in}{-0.069444in}}{\pgfqpoint{0.000000in}{0.000000in}}{%
\pgfpathmoveto{\pgfqpoint{0.000000in}{0.000000in}}%
\pgfpathlineto{\pgfqpoint{0.000000in}{-0.069444in}}%
\pgfusepath{stroke,fill}%
}%
\begin{pgfscope}%
\pgfsys@transformshift{2.322747in}{1.619432in}%
\pgfsys@useobject{currentmarker}{}%
\end{pgfscope}%
\end{pgfscope}%
\begin{pgfscope}%
\pgftext[x=2.322747in,y=0.380539in,,top]{\rmfamily\fontsize{8.000000}{9.600000}\selectfont 20}%
\end{pgfscope}%
\begin{pgfscope}%
\pgfsetbuttcap%
\pgfsetroundjoin%
\definecolor{currentfill}{rgb}{0.000000,0.000000,0.000000}%
\pgfsetfillcolor{currentfill}%
\pgfsetlinewidth{0.501875pt}%
\definecolor{currentstroke}{rgb}{0.000000,0.000000,0.000000}%
\pgfsetstrokecolor{currentstroke}%
\pgfsetdash{}{0pt}%
\pgfsys@defobject{currentmarker}{\pgfqpoint{0.000000in}{0.000000in}}{\pgfqpoint{0.000000in}{0.069444in}}{%
\pgfpathmoveto{\pgfqpoint{0.000000in}{0.000000in}}%
\pgfpathlineto{\pgfqpoint{0.000000in}{0.069444in}}%
\pgfusepath{stroke,fill}%
}%
\begin{pgfscope}%
\pgfsys@transformshift{2.558398in}{0.449983in}%
\pgfsys@useobject{currentmarker}{}%
\end{pgfscope}%
\end{pgfscope}%
\begin{pgfscope}%
\pgfsetbuttcap%
\pgfsetroundjoin%
\definecolor{currentfill}{rgb}{0.000000,0.000000,0.000000}%
\pgfsetfillcolor{currentfill}%
\pgfsetlinewidth{0.501875pt}%
\definecolor{currentstroke}{rgb}{0.000000,0.000000,0.000000}%
\pgfsetstrokecolor{currentstroke}%
\pgfsetdash{}{0pt}%
\pgfsys@defobject{currentmarker}{\pgfqpoint{0.000000in}{-0.069444in}}{\pgfqpoint{0.000000in}{0.000000in}}{%
\pgfpathmoveto{\pgfqpoint{0.000000in}{0.000000in}}%
\pgfpathlineto{\pgfqpoint{0.000000in}{-0.069444in}}%
\pgfusepath{stroke,fill}%
}%
\begin{pgfscope}%
\pgfsys@transformshift{2.558398in}{1.619432in}%
\pgfsys@useobject{currentmarker}{}%
\end{pgfscope}%
\end{pgfscope}%
\begin{pgfscope}%
\pgftext[x=2.558398in,y=0.380539in,,top]{\rmfamily\fontsize{8.000000}{9.600000}\selectfont 40}%
\end{pgfscope}%
\begin{pgfscope}%
\pgftext[x=1.497965in,y=0.203564in,,top]{\rmfamily\fontsize{9.000000}{10.800000}\selectfont \(\displaystyle \mathrm{DLL}_{K/\pi}(\pi^-)\)}%
\end{pgfscope}%
\begin{pgfscope}%
\pgfsetbuttcap%
\pgfsetroundjoin%
\definecolor{currentfill}{rgb}{0.000000,0.000000,0.000000}%
\pgfsetfillcolor{currentfill}%
\pgfsetlinewidth{0.501875pt}%
\definecolor{currentstroke}{rgb}{0.000000,0.000000,0.000000}%
\pgfsetstrokecolor{currentstroke}%
\pgfsetdash{}{0pt}%
\pgfsys@defobject{currentmarker}{\pgfqpoint{0.000000in}{0.000000in}}{\pgfqpoint{0.069444in}{0.000000in}}{%
\pgfpathmoveto{\pgfqpoint{0.000000in}{0.000000in}}%
\pgfpathlineto{\pgfqpoint{0.069444in}{0.000000in}}%
\pgfusepath{stroke,fill}%
}%
\begin{pgfscope}%
\pgfsys@transformshift{0.437532in}{0.449983in}%
\pgfsys@useobject{currentmarker}{}%
\end{pgfscope}%
\end{pgfscope}%
\begin{pgfscope}%
\pgfsetbuttcap%
\pgfsetroundjoin%
\definecolor{currentfill}{rgb}{0.000000,0.000000,0.000000}%
\pgfsetfillcolor{currentfill}%
\pgfsetlinewidth{0.501875pt}%
\definecolor{currentstroke}{rgb}{0.000000,0.000000,0.000000}%
\pgfsetstrokecolor{currentstroke}%
\pgfsetdash{}{0pt}%
\pgfsys@defobject{currentmarker}{\pgfqpoint{-0.069444in}{0.000000in}}{\pgfqpoint{0.000000in}{0.000000in}}{%
\pgfpathmoveto{\pgfqpoint{0.000000in}{0.000000in}}%
\pgfpathlineto{\pgfqpoint{-0.069444in}{0.000000in}}%
\pgfusepath{stroke,fill}%
}%
\begin{pgfscope}%
\pgfsys@transformshift{2.558398in}{0.449983in}%
\pgfsys@useobject{currentmarker}{}%
\end{pgfscope}%
\end{pgfscope}%
\begin{pgfscope}%
\pgftext[x=0.368088in,y=0.449983in,right,]{\rmfamily\fontsize{8.000000}{9.600000}\selectfont 0.000}%
\end{pgfscope}%
\begin{pgfscope}%
\pgfsetbuttcap%
\pgfsetroundjoin%
\definecolor{currentfill}{rgb}{0.000000,0.000000,0.000000}%
\pgfsetfillcolor{currentfill}%
\pgfsetlinewidth{0.501875pt}%
\definecolor{currentstroke}{rgb}{0.000000,0.000000,0.000000}%
\pgfsetstrokecolor{currentstroke}%
\pgfsetdash{}{0pt}%
\pgfsys@defobject{currentmarker}{\pgfqpoint{0.000000in}{0.000000in}}{\pgfqpoint{0.069444in}{0.000000in}}{%
\pgfpathmoveto{\pgfqpoint{0.000000in}{0.000000in}}%
\pgfpathlineto{\pgfqpoint{0.069444in}{0.000000in}}%
\pgfusepath{stroke,fill}%
}%
\begin{pgfscope}%
\pgfsys@transformshift{0.437532in}{0.617048in}%
\pgfsys@useobject{currentmarker}{}%
\end{pgfscope}%
\end{pgfscope}%
\begin{pgfscope}%
\pgfsetbuttcap%
\pgfsetroundjoin%
\definecolor{currentfill}{rgb}{0.000000,0.000000,0.000000}%
\pgfsetfillcolor{currentfill}%
\pgfsetlinewidth{0.501875pt}%
\definecolor{currentstroke}{rgb}{0.000000,0.000000,0.000000}%
\pgfsetstrokecolor{currentstroke}%
\pgfsetdash{}{0pt}%
\pgfsys@defobject{currentmarker}{\pgfqpoint{-0.069444in}{0.000000in}}{\pgfqpoint{0.000000in}{0.000000in}}{%
\pgfpathmoveto{\pgfqpoint{0.000000in}{0.000000in}}%
\pgfpathlineto{\pgfqpoint{-0.069444in}{0.000000in}}%
\pgfusepath{stroke,fill}%
}%
\begin{pgfscope}%
\pgfsys@transformshift{2.558398in}{0.617048in}%
\pgfsys@useobject{currentmarker}{}%
\end{pgfscope}%
\end{pgfscope}%
\begin{pgfscope}%
\pgftext[x=0.368088in,y=0.617048in,right,]{\rmfamily\fontsize{8.000000}{9.600000}\selectfont 0.005}%
\end{pgfscope}%
\begin{pgfscope}%
\pgfsetbuttcap%
\pgfsetroundjoin%
\definecolor{currentfill}{rgb}{0.000000,0.000000,0.000000}%
\pgfsetfillcolor{currentfill}%
\pgfsetlinewidth{0.501875pt}%
\definecolor{currentstroke}{rgb}{0.000000,0.000000,0.000000}%
\pgfsetstrokecolor{currentstroke}%
\pgfsetdash{}{0pt}%
\pgfsys@defobject{currentmarker}{\pgfqpoint{0.000000in}{0.000000in}}{\pgfqpoint{0.069444in}{0.000000in}}{%
\pgfpathmoveto{\pgfqpoint{0.000000in}{0.000000in}}%
\pgfpathlineto{\pgfqpoint{0.069444in}{0.000000in}}%
\pgfusepath{stroke,fill}%
}%
\begin{pgfscope}%
\pgfsys@transformshift{0.437532in}{0.784112in}%
\pgfsys@useobject{currentmarker}{}%
\end{pgfscope}%
\end{pgfscope}%
\begin{pgfscope}%
\pgfsetbuttcap%
\pgfsetroundjoin%
\definecolor{currentfill}{rgb}{0.000000,0.000000,0.000000}%
\pgfsetfillcolor{currentfill}%
\pgfsetlinewidth{0.501875pt}%
\definecolor{currentstroke}{rgb}{0.000000,0.000000,0.000000}%
\pgfsetstrokecolor{currentstroke}%
\pgfsetdash{}{0pt}%
\pgfsys@defobject{currentmarker}{\pgfqpoint{-0.069444in}{0.000000in}}{\pgfqpoint{0.000000in}{0.000000in}}{%
\pgfpathmoveto{\pgfqpoint{0.000000in}{0.000000in}}%
\pgfpathlineto{\pgfqpoint{-0.069444in}{0.000000in}}%
\pgfusepath{stroke,fill}%
}%
\begin{pgfscope}%
\pgfsys@transformshift{2.558398in}{0.784112in}%
\pgfsys@useobject{currentmarker}{}%
\end{pgfscope}%
\end{pgfscope}%
\begin{pgfscope}%
\pgftext[x=0.368088in,y=0.784112in,right,]{\rmfamily\fontsize{8.000000}{9.600000}\selectfont 0.010}%
\end{pgfscope}%
\begin{pgfscope}%
\pgfsetbuttcap%
\pgfsetroundjoin%
\definecolor{currentfill}{rgb}{0.000000,0.000000,0.000000}%
\pgfsetfillcolor{currentfill}%
\pgfsetlinewidth{0.501875pt}%
\definecolor{currentstroke}{rgb}{0.000000,0.000000,0.000000}%
\pgfsetstrokecolor{currentstroke}%
\pgfsetdash{}{0pt}%
\pgfsys@defobject{currentmarker}{\pgfqpoint{0.000000in}{0.000000in}}{\pgfqpoint{0.069444in}{0.000000in}}{%
\pgfpathmoveto{\pgfqpoint{0.000000in}{0.000000in}}%
\pgfpathlineto{\pgfqpoint{0.069444in}{0.000000in}}%
\pgfusepath{stroke,fill}%
}%
\begin{pgfscope}%
\pgfsys@transformshift{0.437532in}{0.951176in}%
\pgfsys@useobject{currentmarker}{}%
\end{pgfscope}%
\end{pgfscope}%
\begin{pgfscope}%
\pgfsetbuttcap%
\pgfsetroundjoin%
\definecolor{currentfill}{rgb}{0.000000,0.000000,0.000000}%
\pgfsetfillcolor{currentfill}%
\pgfsetlinewidth{0.501875pt}%
\definecolor{currentstroke}{rgb}{0.000000,0.000000,0.000000}%
\pgfsetstrokecolor{currentstroke}%
\pgfsetdash{}{0pt}%
\pgfsys@defobject{currentmarker}{\pgfqpoint{-0.069444in}{0.000000in}}{\pgfqpoint{0.000000in}{0.000000in}}{%
\pgfpathmoveto{\pgfqpoint{0.000000in}{0.000000in}}%
\pgfpathlineto{\pgfqpoint{-0.069444in}{0.000000in}}%
\pgfusepath{stroke,fill}%
}%
\begin{pgfscope}%
\pgfsys@transformshift{2.558398in}{0.951176in}%
\pgfsys@useobject{currentmarker}{}%
\end{pgfscope}%
\end{pgfscope}%
\begin{pgfscope}%
\pgftext[x=0.368088in,y=0.951176in,right,]{\rmfamily\fontsize{8.000000}{9.600000}\selectfont 0.015}%
\end{pgfscope}%
\begin{pgfscope}%
\pgfsetbuttcap%
\pgfsetroundjoin%
\definecolor{currentfill}{rgb}{0.000000,0.000000,0.000000}%
\pgfsetfillcolor{currentfill}%
\pgfsetlinewidth{0.501875pt}%
\definecolor{currentstroke}{rgb}{0.000000,0.000000,0.000000}%
\pgfsetstrokecolor{currentstroke}%
\pgfsetdash{}{0pt}%
\pgfsys@defobject{currentmarker}{\pgfqpoint{0.000000in}{0.000000in}}{\pgfqpoint{0.069444in}{0.000000in}}{%
\pgfpathmoveto{\pgfqpoint{0.000000in}{0.000000in}}%
\pgfpathlineto{\pgfqpoint{0.069444in}{0.000000in}}%
\pgfusepath{stroke,fill}%
}%
\begin{pgfscope}%
\pgfsys@transformshift{0.437532in}{1.118240in}%
\pgfsys@useobject{currentmarker}{}%
\end{pgfscope}%
\end{pgfscope}%
\begin{pgfscope}%
\pgfsetbuttcap%
\pgfsetroundjoin%
\definecolor{currentfill}{rgb}{0.000000,0.000000,0.000000}%
\pgfsetfillcolor{currentfill}%
\pgfsetlinewidth{0.501875pt}%
\definecolor{currentstroke}{rgb}{0.000000,0.000000,0.000000}%
\pgfsetstrokecolor{currentstroke}%
\pgfsetdash{}{0pt}%
\pgfsys@defobject{currentmarker}{\pgfqpoint{-0.069444in}{0.000000in}}{\pgfqpoint{0.000000in}{0.000000in}}{%
\pgfpathmoveto{\pgfqpoint{0.000000in}{0.000000in}}%
\pgfpathlineto{\pgfqpoint{-0.069444in}{0.000000in}}%
\pgfusepath{stroke,fill}%
}%
\begin{pgfscope}%
\pgfsys@transformshift{2.558398in}{1.118240in}%
\pgfsys@useobject{currentmarker}{}%
\end{pgfscope}%
\end{pgfscope}%
\begin{pgfscope}%
\pgftext[x=0.368088in,y=1.118240in,right,]{\rmfamily\fontsize{8.000000}{9.600000}\selectfont 0.020}%
\end{pgfscope}%
\begin{pgfscope}%
\pgfsetbuttcap%
\pgfsetroundjoin%
\definecolor{currentfill}{rgb}{0.000000,0.000000,0.000000}%
\pgfsetfillcolor{currentfill}%
\pgfsetlinewidth{0.501875pt}%
\definecolor{currentstroke}{rgb}{0.000000,0.000000,0.000000}%
\pgfsetstrokecolor{currentstroke}%
\pgfsetdash{}{0pt}%
\pgfsys@defobject{currentmarker}{\pgfqpoint{0.000000in}{0.000000in}}{\pgfqpoint{0.069444in}{0.000000in}}{%
\pgfpathmoveto{\pgfqpoint{0.000000in}{0.000000in}}%
\pgfpathlineto{\pgfqpoint{0.069444in}{0.000000in}}%
\pgfusepath{stroke,fill}%
}%
\begin{pgfscope}%
\pgfsys@transformshift{0.437532in}{1.285304in}%
\pgfsys@useobject{currentmarker}{}%
\end{pgfscope}%
\end{pgfscope}%
\begin{pgfscope}%
\pgfsetbuttcap%
\pgfsetroundjoin%
\definecolor{currentfill}{rgb}{0.000000,0.000000,0.000000}%
\pgfsetfillcolor{currentfill}%
\pgfsetlinewidth{0.501875pt}%
\definecolor{currentstroke}{rgb}{0.000000,0.000000,0.000000}%
\pgfsetstrokecolor{currentstroke}%
\pgfsetdash{}{0pt}%
\pgfsys@defobject{currentmarker}{\pgfqpoint{-0.069444in}{0.000000in}}{\pgfqpoint{0.000000in}{0.000000in}}{%
\pgfpathmoveto{\pgfqpoint{0.000000in}{0.000000in}}%
\pgfpathlineto{\pgfqpoint{-0.069444in}{0.000000in}}%
\pgfusepath{stroke,fill}%
}%
\begin{pgfscope}%
\pgfsys@transformshift{2.558398in}{1.285304in}%
\pgfsys@useobject{currentmarker}{}%
\end{pgfscope}%
\end{pgfscope}%
\begin{pgfscope}%
\pgftext[x=0.368088in,y=1.285304in,right,]{\rmfamily\fontsize{8.000000}{9.600000}\selectfont 0.025}%
\end{pgfscope}%
\begin{pgfscope}%
\pgfsetbuttcap%
\pgfsetroundjoin%
\definecolor{currentfill}{rgb}{0.000000,0.000000,0.000000}%
\pgfsetfillcolor{currentfill}%
\pgfsetlinewidth{0.501875pt}%
\definecolor{currentstroke}{rgb}{0.000000,0.000000,0.000000}%
\pgfsetstrokecolor{currentstroke}%
\pgfsetdash{}{0pt}%
\pgfsys@defobject{currentmarker}{\pgfqpoint{0.000000in}{0.000000in}}{\pgfqpoint{0.069444in}{0.000000in}}{%
\pgfpathmoveto{\pgfqpoint{0.000000in}{0.000000in}}%
\pgfpathlineto{\pgfqpoint{0.069444in}{0.000000in}}%
\pgfusepath{stroke,fill}%
}%
\begin{pgfscope}%
\pgfsys@transformshift{0.437532in}{1.452368in}%
\pgfsys@useobject{currentmarker}{}%
\end{pgfscope}%
\end{pgfscope}%
\begin{pgfscope}%
\pgfsetbuttcap%
\pgfsetroundjoin%
\definecolor{currentfill}{rgb}{0.000000,0.000000,0.000000}%
\pgfsetfillcolor{currentfill}%
\pgfsetlinewidth{0.501875pt}%
\definecolor{currentstroke}{rgb}{0.000000,0.000000,0.000000}%
\pgfsetstrokecolor{currentstroke}%
\pgfsetdash{}{0pt}%
\pgfsys@defobject{currentmarker}{\pgfqpoint{-0.069444in}{0.000000in}}{\pgfqpoint{0.000000in}{0.000000in}}{%
\pgfpathmoveto{\pgfqpoint{0.000000in}{0.000000in}}%
\pgfpathlineto{\pgfqpoint{-0.069444in}{0.000000in}}%
\pgfusepath{stroke,fill}%
}%
\begin{pgfscope}%
\pgfsys@transformshift{2.558398in}{1.452368in}%
\pgfsys@useobject{currentmarker}{}%
\end{pgfscope}%
\end{pgfscope}%
\begin{pgfscope}%
\pgftext[x=0.368088in,y=1.452368in,right,]{\rmfamily\fontsize{8.000000}{9.600000}\selectfont 0.030}%
\end{pgfscope}%
\begin{pgfscope}%
\pgfsetbuttcap%
\pgfsetroundjoin%
\definecolor{currentfill}{rgb}{0.000000,0.000000,0.000000}%
\pgfsetfillcolor{currentfill}%
\pgfsetlinewidth{0.501875pt}%
\definecolor{currentstroke}{rgb}{0.000000,0.000000,0.000000}%
\pgfsetstrokecolor{currentstroke}%
\pgfsetdash{}{0pt}%
\pgfsys@defobject{currentmarker}{\pgfqpoint{0.000000in}{0.000000in}}{\pgfqpoint{0.069444in}{0.000000in}}{%
\pgfpathmoveto{\pgfqpoint{0.000000in}{0.000000in}}%
\pgfpathlineto{\pgfqpoint{0.069444in}{0.000000in}}%
\pgfusepath{stroke,fill}%
}%
\begin{pgfscope}%
\pgfsys@transformshift{0.437532in}{1.619432in}%
\pgfsys@useobject{currentmarker}{}%
\end{pgfscope}%
\end{pgfscope}%
\begin{pgfscope}%
\pgfsetbuttcap%
\pgfsetroundjoin%
\definecolor{currentfill}{rgb}{0.000000,0.000000,0.000000}%
\pgfsetfillcolor{currentfill}%
\pgfsetlinewidth{0.501875pt}%
\definecolor{currentstroke}{rgb}{0.000000,0.000000,0.000000}%
\pgfsetstrokecolor{currentstroke}%
\pgfsetdash{}{0pt}%
\pgfsys@defobject{currentmarker}{\pgfqpoint{-0.069444in}{0.000000in}}{\pgfqpoint{0.000000in}{0.000000in}}{%
\pgfpathmoveto{\pgfqpoint{0.000000in}{0.000000in}}%
\pgfpathlineto{\pgfqpoint{-0.069444in}{0.000000in}}%
\pgfusepath{stroke,fill}%
}%
\begin{pgfscope}%
\pgfsys@transformshift{2.558398in}{1.619432in}%
\pgfsys@useobject{currentmarker}{}%
\end{pgfscope}%
\end{pgfscope}%
\begin{pgfscope}%
\pgftext[x=0.368088in,y=1.619432in,right,]{\rmfamily\fontsize{8.000000}{9.600000}\selectfont 0.035}%
\end{pgfscope}%
\end{pgfpicture}%
\makeatother%
\endgroup%

	\end{subfigure}
	\begin{subfigure}[t]{0.49\textwidth}
		\centering
    %\includegraphics[width=\textwidth]{store/variables/SIG_BKG_piminus_PIDmu.pdf}
    %% Creator: Matplotlib, PGF backend
%%
%% To include the figure in your LaTeX document, write
%%   \input{<filename>.pgf}
%%
%% Make sure the required packages are loaded in your preamble
%%   \usepackage{pgf}
%%
%% Figures using additional raster images can only be included by \input if
%% they are in the same directory as the main LaTeX file. For loading figures
%% from other directories you can use the `import` package
%%   \usepackage{import}
%% and then include the figures with
%%   \import{<path to file>}{<filename>.pgf}
%%
%% Matplotlib used the following preamble
%%   \usepackage{fontspec}
%%   \setmainfont{DejaVu Serif}
%%   \setsansfont{DejaVu Sans}
%%   \setmonofont{DejaVu Sans Mono}
%%
\begingroup%
\makeatletter%
\begin{pgfpicture}%
\pgfpathrectangle{\pgfpointorigin}{\pgfqpoint{2.682342in}{1.723197in}}%
\pgfusepath{use as bounding box, clip}%
\begin{pgfscope}%
\pgfsetbuttcap%
\pgfsetmiterjoin%
\definecolor{currentfill}{rgb}{1.000000,1.000000,1.000000}%
\pgfsetfillcolor{currentfill}%
\pgfsetlinewidth{0.000000pt}%
\definecolor{currentstroke}{rgb}{1.000000,1.000000,1.000000}%
\pgfsetstrokecolor{currentstroke}%
\pgfsetdash{}{0pt}%
\pgfpathmoveto{\pgfqpoint{0.000000in}{0.000000in}}%
\pgfpathlineto{\pgfqpoint{2.682342in}{0.000000in}}%
\pgfpathlineto{\pgfqpoint{2.682342in}{1.723197in}}%
\pgfpathlineto{\pgfqpoint{0.000000in}{1.723197in}}%
\pgfpathclose%
\pgfusepath{fill}%
\end{pgfscope}%
\begin{pgfscope}%
\pgfsetbuttcap%
\pgfsetmiterjoin%
\definecolor{currentfill}{rgb}{1.000000,1.000000,1.000000}%
\pgfsetfillcolor{currentfill}%
\pgfsetlinewidth{0.000000pt}%
\definecolor{currentstroke}{rgb}{0.000000,0.000000,0.000000}%
\pgfsetstrokecolor{currentstroke}%
\pgfsetstrokeopacity{0.000000}%
\pgfsetdash{}{0pt}%
\pgfpathmoveto{\pgfqpoint{0.366840in}{0.449983in}}%
\pgfpathlineto{\pgfqpoint{2.561650in}{0.449983in}}%
\pgfpathlineto{\pgfqpoint{2.561650in}{1.619432in}}%
\pgfpathlineto{\pgfqpoint{0.366840in}{1.619432in}}%
\pgfpathclose%
\pgfusepath{fill}%
\end{pgfscope}%
\begin{pgfscope}%
\pgfpathrectangle{\pgfqpoint{0.366840in}{0.449983in}}{\pgfqpoint{2.194810in}{1.169449in}} %
\pgfusepath{clip}%
\pgfsetbuttcap%
\pgfsetmiterjoin%
\definecolor{currentfill}{rgb}{0.215686,0.470588,0.749020}%
\pgfsetfillcolor{currentfill}%
\pgfsetlinewidth{0.000000pt}%
\definecolor{currentstroke}{rgb}{0.000000,0.000000,0.000000}%
\pgfsetstrokecolor{currentstroke}%
\pgfsetdash{}{0pt}%
\pgfpathmoveto{\pgfqpoint{0.366840in}{0.449983in}}%
\pgfpathlineto{\pgfqpoint{0.366840in}{0.449983in}}%
\pgfpathlineto{\pgfqpoint{0.410736in}{0.449983in}}%
\pgfpathlineto{\pgfqpoint{0.410736in}{0.449983in}}%
\pgfpathlineto{\pgfqpoint{0.454633in}{0.449983in}}%
\pgfpathlineto{\pgfqpoint{0.454633in}{0.449983in}}%
\pgfpathlineto{\pgfqpoint{0.498529in}{0.449983in}}%
\pgfpathlineto{\pgfqpoint{0.498529in}{0.468199in}}%
\pgfpathlineto{\pgfqpoint{0.542425in}{0.468199in}}%
\pgfpathlineto{\pgfqpoint{0.542425in}{0.544486in}}%
\pgfpathlineto{\pgfqpoint{0.586321in}{0.544486in}}%
\pgfpathlineto{\pgfqpoint{0.586321in}{0.632540in}}%
\pgfpathlineto{\pgfqpoint{0.630217in}{0.632540in}}%
\pgfpathlineto{\pgfqpoint{0.630217in}{0.668428in}}%
\pgfpathlineto{\pgfqpoint{0.674114in}{0.668428in}}%
\pgfpathlineto{\pgfqpoint{0.674114in}{0.671133in}}%
\pgfpathlineto{\pgfqpoint{0.718010in}{0.671133in}}%
\pgfpathlineto{\pgfqpoint{0.718010in}{0.729751in}}%
\pgfpathlineto{\pgfqpoint{0.761906in}{0.729751in}}%
\pgfpathlineto{\pgfqpoint{0.761906in}{0.832841in}}%
\pgfpathlineto{\pgfqpoint{0.805802in}{0.832841in}}%
\pgfpathlineto{\pgfqpoint{0.805802in}{1.013559in}}%
\pgfpathlineto{\pgfqpoint{0.849698in}{1.013559in}}%
\pgfpathlineto{\pgfqpoint{0.849698in}{1.054820in}}%
\pgfpathlineto{\pgfqpoint{0.893595in}{1.054820in}}%
\pgfpathlineto{\pgfqpoint{0.893595in}{1.158030in}}%
\pgfpathlineto{\pgfqpoint{0.937491in}{1.158030in}}%
\pgfpathlineto{\pgfqpoint{0.937491in}{1.517135in}}%
\pgfpathlineto{\pgfqpoint{0.981387in}{1.517135in}}%
\pgfpathlineto{\pgfqpoint{0.981387in}{1.416636in}}%
\pgfpathlineto{\pgfqpoint{1.025283in}{1.416636in}}%
\pgfpathlineto{\pgfqpoint{1.025283in}{1.343738in}}%
\pgfpathlineto{\pgfqpoint{1.069179in}{1.343738in}}%
\pgfpathlineto{\pgfqpoint{1.069179in}{1.457917in}}%
\pgfpathlineto{\pgfqpoint{1.113076in}{1.457917in}}%
\pgfpathlineto{\pgfqpoint{1.113076in}{1.464990in}}%
\pgfpathlineto{\pgfqpoint{1.156972in}{1.464990in}}%
\pgfpathlineto{\pgfqpoint{1.156972in}{1.468264in}}%
\pgfpathlineto{\pgfqpoint{1.200868in}{1.468264in}}%
\pgfpathlineto{\pgfqpoint{1.200868in}{1.501419in}}%
\pgfpathlineto{\pgfqpoint{1.244764in}{1.501419in}}%
\pgfpathlineto{\pgfqpoint{1.244764in}{1.496717in}}%
\pgfpathlineto{\pgfqpoint{1.288660in}{1.496717in}}%
\pgfpathlineto{\pgfqpoint{1.288660in}{1.360288in}}%
\pgfpathlineto{\pgfqpoint{1.332556in}{1.360288in}}%
\pgfpathlineto{\pgfqpoint{1.332556in}{1.305869in}}%
\pgfpathlineto{\pgfqpoint{1.376453in}{1.305869in}}%
\pgfpathlineto{\pgfqpoint{1.376453in}{1.077419in}}%
\pgfpathlineto{\pgfqpoint{1.420349in}{1.077419in}}%
\pgfpathlineto{\pgfqpoint{1.420349in}{1.123415in}}%
\pgfpathlineto{\pgfqpoint{1.464245in}{1.123415in}}%
\pgfpathlineto{\pgfqpoint{1.464245in}{0.985677in}}%
\pgfpathlineto{\pgfqpoint{1.508141in}{0.985677in}}%
\pgfpathlineto{\pgfqpoint{1.508141in}{0.776492in}}%
\pgfpathlineto{\pgfqpoint{1.552037in}{0.776492in}}%
\pgfpathlineto{\pgfqpoint{1.552037in}{0.685257in}}%
\pgfpathlineto{\pgfqpoint{1.595934in}{0.685257in}}%
\pgfpathlineto{\pgfqpoint{1.595934in}{0.629369in}}%
\pgfpathlineto{\pgfqpoint{1.639830in}{0.629369in}}%
\pgfpathlineto{\pgfqpoint{1.639830in}{0.600307in}}%
\pgfpathlineto{\pgfqpoint{1.683726in}{0.600307in}}%
\pgfpathlineto{\pgfqpoint{1.683726in}{0.537647in}}%
\pgfpathlineto{\pgfqpoint{1.727622in}{0.537647in}}%
\pgfpathlineto{\pgfqpoint{1.727622in}{0.525083in}}%
\pgfpathlineto{\pgfqpoint{1.771518in}{0.525083in}}%
\pgfpathlineto{\pgfqpoint{1.771518in}{0.515386in}}%
\pgfpathlineto{\pgfqpoint{1.815415in}{0.515386in}}%
\pgfpathlineto{\pgfqpoint{1.815415in}{0.489001in}}%
\pgfpathlineto{\pgfqpoint{1.859311in}{0.489001in}}%
\pgfpathlineto{\pgfqpoint{1.859311in}{0.488673in}}%
\pgfpathlineto{\pgfqpoint{1.903207in}{0.488673in}}%
\pgfpathlineto{\pgfqpoint{1.903207in}{0.479800in}}%
\pgfpathlineto{\pgfqpoint{1.947103in}{0.479800in}}%
\pgfpathlineto{\pgfqpoint{1.947103in}{0.478239in}}%
\pgfpathlineto{\pgfqpoint{1.990999in}{0.478239in}}%
\pgfpathlineto{\pgfqpoint{1.990999in}{0.462602in}}%
\pgfpathlineto{\pgfqpoint{2.034896in}{0.462602in}}%
\pgfpathlineto{\pgfqpoint{2.034896in}{0.460271in}}%
\pgfpathlineto{\pgfqpoint{2.078792in}{0.460271in}}%
\pgfpathlineto{\pgfqpoint{2.078792in}{0.454802in}}%
\pgfpathlineto{\pgfqpoint{2.122688in}{0.454802in}}%
\pgfpathlineto{\pgfqpoint{2.122688in}{0.456824in}}%
\pgfpathlineto{\pgfqpoint{2.166584in}{0.456824in}}%
\pgfpathlineto{\pgfqpoint{2.166584in}{0.452710in}}%
\pgfpathlineto{\pgfqpoint{2.210480in}{0.452710in}}%
\pgfpathlineto{\pgfqpoint{2.210480in}{0.451935in}}%
\pgfpathlineto{\pgfqpoint{2.254377in}{0.451935in}}%
\pgfpathlineto{\pgfqpoint{2.254377in}{0.452566in}}%
\pgfpathlineto{\pgfqpoint{2.298273in}{0.452566in}}%
\pgfpathlineto{\pgfqpoint{2.298273in}{0.451042in}}%
\pgfpathlineto{\pgfqpoint{2.342169in}{0.451042in}}%
\pgfpathlineto{\pgfqpoint{2.342169in}{0.450179in}}%
\pgfpathlineto{\pgfqpoint{2.386065in}{0.450179in}}%
\pgfpathlineto{\pgfqpoint{2.386065in}{0.449983in}}%
\pgfpathlineto{\pgfqpoint{2.429961in}{0.449983in}}%
\pgfpathlineto{\pgfqpoint{2.429961in}{0.450155in}}%
\pgfpathlineto{\pgfqpoint{2.473857in}{0.450155in}}%
\pgfpathlineto{\pgfqpoint{2.473857in}{0.449983in}}%
\pgfpathlineto{\pgfqpoint{2.517754in}{0.449983in}}%
\pgfpathlineto{\pgfqpoint{2.517754in}{0.449983in}}%
\pgfpathlineto{\pgfqpoint{2.561650in}{0.449983in}}%
\pgfpathlineto{\pgfqpoint{2.561650in}{0.449983in}}%
\pgfpathlineto{\pgfqpoint{2.517754in}{0.449983in}}%
\pgfpathlineto{\pgfqpoint{2.517754in}{0.449983in}}%
\pgfpathlineto{\pgfqpoint{2.473857in}{0.449983in}}%
\pgfpathlineto{\pgfqpoint{2.473857in}{0.449983in}}%
\pgfpathlineto{\pgfqpoint{2.429961in}{0.449983in}}%
\pgfpathlineto{\pgfqpoint{2.429961in}{0.449983in}}%
\pgfpathlineto{\pgfqpoint{2.386065in}{0.449983in}}%
\pgfpathlineto{\pgfqpoint{2.386065in}{0.449983in}}%
\pgfpathlineto{\pgfqpoint{2.342169in}{0.449983in}}%
\pgfpathlineto{\pgfqpoint{2.342169in}{0.449983in}}%
\pgfpathlineto{\pgfqpoint{2.298273in}{0.449983in}}%
\pgfpathlineto{\pgfqpoint{2.298273in}{0.449983in}}%
\pgfpathlineto{\pgfqpoint{2.254377in}{0.449983in}}%
\pgfpathlineto{\pgfqpoint{2.254377in}{0.449983in}}%
\pgfpathlineto{\pgfqpoint{2.210480in}{0.449983in}}%
\pgfpathlineto{\pgfqpoint{2.210480in}{0.449983in}}%
\pgfpathlineto{\pgfqpoint{2.166584in}{0.449983in}}%
\pgfpathlineto{\pgfqpoint{2.166584in}{0.449983in}}%
\pgfpathlineto{\pgfqpoint{2.122688in}{0.449983in}}%
\pgfpathlineto{\pgfqpoint{2.122688in}{0.449983in}}%
\pgfpathlineto{\pgfqpoint{2.078792in}{0.449983in}}%
\pgfpathlineto{\pgfqpoint{2.078792in}{0.449983in}}%
\pgfpathlineto{\pgfqpoint{2.034896in}{0.449983in}}%
\pgfpathlineto{\pgfqpoint{2.034896in}{0.449983in}}%
\pgfpathlineto{\pgfqpoint{1.990999in}{0.449983in}}%
\pgfpathlineto{\pgfqpoint{1.990999in}{0.449983in}}%
\pgfpathlineto{\pgfqpoint{1.947103in}{0.449983in}}%
\pgfpathlineto{\pgfqpoint{1.947103in}{0.449983in}}%
\pgfpathlineto{\pgfqpoint{1.903207in}{0.449983in}}%
\pgfpathlineto{\pgfqpoint{1.903207in}{0.449983in}}%
\pgfpathlineto{\pgfqpoint{1.859311in}{0.449983in}}%
\pgfpathlineto{\pgfqpoint{1.859311in}{0.449983in}}%
\pgfpathlineto{\pgfqpoint{1.815415in}{0.449983in}}%
\pgfpathlineto{\pgfqpoint{1.815415in}{0.449983in}}%
\pgfpathlineto{\pgfqpoint{1.771518in}{0.449983in}}%
\pgfpathlineto{\pgfqpoint{1.771518in}{0.449983in}}%
\pgfpathlineto{\pgfqpoint{1.727622in}{0.449983in}}%
\pgfpathlineto{\pgfqpoint{1.727622in}{0.449983in}}%
\pgfpathlineto{\pgfqpoint{1.683726in}{0.449983in}}%
\pgfpathlineto{\pgfqpoint{1.683726in}{0.449983in}}%
\pgfpathlineto{\pgfqpoint{1.639830in}{0.449983in}}%
\pgfpathlineto{\pgfqpoint{1.639830in}{0.449983in}}%
\pgfpathlineto{\pgfqpoint{1.595934in}{0.449983in}}%
\pgfpathlineto{\pgfqpoint{1.595934in}{0.449983in}}%
\pgfpathlineto{\pgfqpoint{1.552037in}{0.449983in}}%
\pgfpathlineto{\pgfqpoint{1.552037in}{0.449983in}}%
\pgfpathlineto{\pgfqpoint{1.508141in}{0.449983in}}%
\pgfpathlineto{\pgfqpoint{1.508141in}{0.449983in}}%
\pgfpathlineto{\pgfqpoint{1.464245in}{0.449983in}}%
\pgfpathlineto{\pgfqpoint{1.464245in}{0.449983in}}%
\pgfpathlineto{\pgfqpoint{1.420349in}{0.449983in}}%
\pgfpathlineto{\pgfqpoint{1.420349in}{0.449983in}}%
\pgfpathlineto{\pgfqpoint{1.376453in}{0.449983in}}%
\pgfpathlineto{\pgfqpoint{1.376453in}{0.449983in}}%
\pgfpathlineto{\pgfqpoint{1.332556in}{0.449983in}}%
\pgfpathlineto{\pgfqpoint{1.332556in}{0.449983in}}%
\pgfpathlineto{\pgfqpoint{1.288660in}{0.449983in}}%
\pgfpathlineto{\pgfqpoint{1.288660in}{0.449983in}}%
\pgfpathlineto{\pgfqpoint{1.244764in}{0.449983in}}%
\pgfpathlineto{\pgfqpoint{1.244764in}{0.449983in}}%
\pgfpathlineto{\pgfqpoint{1.200868in}{0.449983in}}%
\pgfpathlineto{\pgfqpoint{1.200868in}{0.449983in}}%
\pgfpathlineto{\pgfqpoint{1.156972in}{0.449983in}}%
\pgfpathlineto{\pgfqpoint{1.156972in}{0.449983in}}%
\pgfpathlineto{\pgfqpoint{1.113076in}{0.449983in}}%
\pgfpathlineto{\pgfqpoint{1.113076in}{0.449983in}}%
\pgfpathlineto{\pgfqpoint{1.069179in}{0.449983in}}%
\pgfpathlineto{\pgfqpoint{1.069179in}{0.449983in}}%
\pgfpathlineto{\pgfqpoint{1.025283in}{0.449983in}}%
\pgfpathlineto{\pgfqpoint{1.025283in}{0.449983in}}%
\pgfpathlineto{\pgfqpoint{0.981387in}{0.449983in}}%
\pgfpathlineto{\pgfqpoint{0.981387in}{0.449983in}}%
\pgfpathlineto{\pgfqpoint{0.937491in}{0.449983in}}%
\pgfpathlineto{\pgfqpoint{0.937491in}{0.449983in}}%
\pgfpathlineto{\pgfqpoint{0.893595in}{0.449983in}}%
\pgfpathlineto{\pgfqpoint{0.893595in}{0.449983in}}%
\pgfpathlineto{\pgfqpoint{0.849698in}{0.449983in}}%
\pgfpathlineto{\pgfqpoint{0.849698in}{0.449983in}}%
\pgfpathlineto{\pgfqpoint{0.805802in}{0.449983in}}%
\pgfpathlineto{\pgfqpoint{0.805802in}{0.449983in}}%
\pgfpathlineto{\pgfqpoint{0.761906in}{0.449983in}}%
\pgfpathlineto{\pgfqpoint{0.761906in}{0.449983in}}%
\pgfpathlineto{\pgfqpoint{0.718010in}{0.449983in}}%
\pgfpathlineto{\pgfqpoint{0.718010in}{0.449983in}}%
\pgfpathlineto{\pgfqpoint{0.674114in}{0.449983in}}%
\pgfpathlineto{\pgfqpoint{0.674114in}{0.449983in}}%
\pgfpathlineto{\pgfqpoint{0.630217in}{0.449983in}}%
\pgfpathlineto{\pgfqpoint{0.630217in}{0.449983in}}%
\pgfpathlineto{\pgfqpoint{0.586321in}{0.449983in}}%
\pgfpathlineto{\pgfqpoint{0.586321in}{0.449983in}}%
\pgfpathlineto{\pgfqpoint{0.542425in}{0.449983in}}%
\pgfpathlineto{\pgfqpoint{0.542425in}{0.449983in}}%
\pgfpathlineto{\pgfqpoint{0.498529in}{0.449983in}}%
\pgfpathlineto{\pgfqpoint{0.498529in}{0.449983in}}%
\pgfpathlineto{\pgfqpoint{0.454633in}{0.449983in}}%
\pgfpathlineto{\pgfqpoint{0.454633in}{0.449983in}}%
\pgfpathlineto{\pgfqpoint{0.410736in}{0.449983in}}%
\pgfpathlineto{\pgfqpoint{0.410736in}{0.449983in}}%
\pgfpathlineto{\pgfqpoint{0.366840in}{0.449983in}}%
\pgfusepath{fill}%
\end{pgfscope}%
\begin{pgfscope}%
\pgfpathrectangle{\pgfqpoint{0.366840in}{0.449983in}}{\pgfqpoint{2.194810in}{1.169449in}} %
\pgfusepath{clip}%
\pgfsetbuttcap%
\pgfsetmiterjoin%
\pgfsetlinewidth{0.501875pt}%
\definecolor{currentstroke}{rgb}{1.000000,0.000000,0.000000}%
\pgfsetstrokecolor{currentstroke}%
\pgfsetdash{}{0pt}%
\pgfpathmoveto{\pgfqpoint{0.366840in}{0.449983in}}%
\pgfpathlineto{\pgfqpoint{0.366840in}{0.449983in}}%
\pgfpathlineto{\pgfqpoint{0.410736in}{0.449983in}}%
\pgfpathlineto{\pgfqpoint{0.410736in}{0.450132in}}%
\pgfpathlineto{\pgfqpoint{0.454633in}{0.450132in}}%
\pgfpathlineto{\pgfqpoint{0.454633in}{0.450132in}}%
\pgfpathlineto{\pgfqpoint{0.498529in}{0.450132in}}%
\pgfpathlineto{\pgfqpoint{0.498529in}{0.453331in}}%
\pgfpathlineto{\pgfqpoint{0.542425in}{0.453331in}}%
\pgfpathlineto{\pgfqpoint{0.542425in}{0.492498in}}%
\pgfpathlineto{\pgfqpoint{0.586321in}{0.492498in}}%
\pgfpathlineto{\pgfqpoint{0.586321in}{0.564583in}}%
\pgfpathlineto{\pgfqpoint{0.630217in}{0.564583in}}%
\pgfpathlineto{\pgfqpoint{0.630217in}{0.590322in}}%
\pgfpathlineto{\pgfqpoint{0.674114in}{0.590322in}}%
\pgfpathlineto{\pgfqpoint{0.674114in}{0.599956in}}%
\pgfpathlineto{\pgfqpoint{0.718010in}{0.599956in}}%
\pgfpathlineto{\pgfqpoint{0.718010in}{0.633841in}}%
\pgfpathlineto{\pgfqpoint{0.761906in}{0.633841in}}%
\pgfpathlineto{\pgfqpoint{0.761906in}{0.714928in}}%
\pgfpathlineto{\pgfqpoint{0.805802in}{0.714928in}}%
\pgfpathlineto{\pgfqpoint{0.805802in}{0.887069in}}%
\pgfpathlineto{\pgfqpoint{0.849698in}{0.887069in}}%
\pgfpathlineto{\pgfqpoint{0.849698in}{0.920991in}}%
\pgfpathlineto{\pgfqpoint{0.893595in}{0.920991in}}%
\pgfpathlineto{\pgfqpoint{0.893595in}{0.990584in}}%
\pgfpathlineto{\pgfqpoint{0.937491in}{0.990584in}}%
\pgfpathlineto{\pgfqpoint{0.937491in}{1.275986in}}%
\pgfpathlineto{\pgfqpoint{0.981387in}{1.275986in}}%
\pgfpathlineto{\pgfqpoint{0.981387in}{1.233843in}}%
\pgfpathlineto{\pgfqpoint{1.025283in}{1.233843in}}%
\pgfpathlineto{\pgfqpoint{1.025283in}{1.297485in}}%
\pgfpathlineto{\pgfqpoint{1.069179in}{1.297485in}}%
\pgfpathlineto{\pgfqpoint{1.069179in}{1.324005in}}%
\pgfpathlineto{\pgfqpoint{1.113076in}{1.324005in}}%
\pgfpathlineto{\pgfqpoint{1.113076in}{1.368975in}}%
\pgfpathlineto{\pgfqpoint{1.156972in}{1.368975in}}%
\pgfpathlineto{\pgfqpoint{1.156972in}{1.395644in}}%
\pgfpathlineto{\pgfqpoint{1.200868in}{1.395644in}}%
\pgfpathlineto{\pgfqpoint{1.200868in}{1.397281in}}%
\pgfpathlineto{\pgfqpoint{1.244764in}{1.397281in}}%
\pgfpathlineto{\pgfqpoint{1.244764in}{1.391143in}}%
\pgfpathlineto{\pgfqpoint{1.288660in}{1.391143in}}%
\pgfpathlineto{\pgfqpoint{1.288660in}{1.349856in}}%
\pgfpathlineto{\pgfqpoint{1.332556in}{1.349856in}}%
\pgfpathlineto{\pgfqpoint{1.332556in}{1.292091in}}%
\pgfpathlineto{\pgfqpoint{1.376453in}{1.292091in}}%
\pgfpathlineto{\pgfqpoint{1.376453in}{1.266129in}}%
\pgfpathlineto{\pgfqpoint{1.420349in}{1.266129in}}%
\pgfpathlineto{\pgfqpoint{1.420349in}{1.405724in}}%
\pgfpathlineto{\pgfqpoint{1.464245in}{1.405724in}}%
\pgfpathlineto{\pgfqpoint{1.464245in}{1.170425in}}%
\pgfpathlineto{\pgfqpoint{1.508141in}{1.170425in}}%
\pgfpathlineto{\pgfqpoint{1.508141in}{0.970164in}}%
\pgfpathlineto{\pgfqpoint{1.552037in}{0.970164in}}%
\pgfpathlineto{\pgfqpoint{1.552037in}{0.859507in}}%
\pgfpathlineto{\pgfqpoint{1.595934in}{0.859507in}}%
\pgfpathlineto{\pgfqpoint{1.595934in}{0.778011in}}%
\pgfpathlineto{\pgfqpoint{1.639830in}{0.778011in}}%
\pgfpathlineto{\pgfqpoint{1.639830in}{0.710873in}}%
\pgfpathlineto{\pgfqpoint{1.683726in}{0.710873in}}%
\pgfpathlineto{\pgfqpoint{1.683726in}{0.662965in}}%
\pgfpathlineto{\pgfqpoint{1.727622in}{0.662965in}}%
\pgfpathlineto{\pgfqpoint{1.727622in}{0.623278in}}%
\pgfpathlineto{\pgfqpoint{1.771518in}{0.623278in}}%
\pgfpathlineto{\pgfqpoint{1.771518in}{0.593224in}}%
\pgfpathlineto{\pgfqpoint{1.815415in}{0.593224in}}%
\pgfpathlineto{\pgfqpoint{1.815415in}{0.564546in}}%
\pgfpathlineto{\pgfqpoint{1.859311in}{0.564546in}}%
\pgfpathlineto{\pgfqpoint{1.859311in}{0.534641in}}%
\pgfpathlineto{\pgfqpoint{1.903207in}{0.534641in}}%
\pgfpathlineto{\pgfqpoint{1.903207in}{0.520692in}}%
\pgfpathlineto{\pgfqpoint{1.947103in}{0.520692in}}%
\pgfpathlineto{\pgfqpoint{1.947103in}{0.512360in}}%
\pgfpathlineto{\pgfqpoint{1.990999in}{0.512360in}}%
\pgfpathlineto{\pgfqpoint{1.990999in}{0.494693in}}%
\pgfpathlineto{\pgfqpoint{2.034896in}{0.494693in}}%
\pgfpathlineto{\pgfqpoint{2.034896in}{0.484538in}}%
\pgfpathlineto{\pgfqpoint{2.078792in}{0.484538in}}%
\pgfpathlineto{\pgfqpoint{2.078792in}{0.476392in}}%
\pgfpathlineto{\pgfqpoint{2.122688in}{0.476392in}}%
\pgfpathlineto{\pgfqpoint{2.122688in}{0.470887in}}%
\pgfpathlineto{\pgfqpoint{2.166584in}{0.470887in}}%
\pgfpathlineto{\pgfqpoint{2.166584in}{0.468916in}}%
\pgfpathlineto{\pgfqpoint{2.210480in}{0.468916in}}%
\pgfpathlineto{\pgfqpoint{2.210480in}{0.466461in}}%
\pgfpathlineto{\pgfqpoint{2.254377in}{0.466461in}}%
\pgfpathlineto{\pgfqpoint{2.254377in}{0.463411in}}%
\pgfpathlineto{\pgfqpoint{2.298273in}{0.463411in}}%
\pgfpathlineto{\pgfqpoint{2.298273in}{0.460361in}}%
\pgfpathlineto{\pgfqpoint{2.342169in}{0.460361in}}%
\pgfpathlineto{\pgfqpoint{2.342169in}{0.455228in}}%
\pgfpathlineto{\pgfqpoint{2.386065in}{0.455228in}}%
\pgfpathlineto{\pgfqpoint{2.386065in}{0.452401in}}%
\pgfpathlineto{\pgfqpoint{2.429961in}{0.452401in}}%
\pgfpathlineto{\pgfqpoint{2.429961in}{0.450951in}}%
\pgfpathlineto{\pgfqpoint{2.473857in}{0.450951in}}%
\pgfpathlineto{\pgfqpoint{2.473857in}{0.450095in}}%
\pgfpathlineto{\pgfqpoint{2.517754in}{0.450095in}}%
\pgfpathlineto{\pgfqpoint{2.517754in}{0.449983in}}%
\pgfpathlineto{\pgfqpoint{2.561650in}{0.449983in}}%
\pgfpathlineto{\pgfqpoint{2.561650in}{0.449983in}}%
\pgfusepath{stroke}%
\end{pgfscope}%
\begin{pgfscope}%
\pgfsetrectcap%
\pgfsetmiterjoin%
\pgfsetlinewidth{1.003750pt}%
\definecolor{currentstroke}{rgb}{0.000000,0.000000,0.000000}%
\pgfsetstrokecolor{currentstroke}%
\pgfsetdash{}{0pt}%
\pgfpathmoveto{\pgfqpoint{0.366840in}{1.619432in}}%
\pgfpathlineto{\pgfqpoint{2.561650in}{1.619432in}}%
\pgfusepath{stroke}%
\end{pgfscope}%
\begin{pgfscope}%
\pgfsetrectcap%
\pgfsetmiterjoin%
\pgfsetlinewidth{1.003750pt}%
\definecolor{currentstroke}{rgb}{0.000000,0.000000,0.000000}%
\pgfsetstrokecolor{currentstroke}%
\pgfsetdash{}{0pt}%
\pgfpathmoveto{\pgfqpoint{2.561650in}{0.449983in}}%
\pgfpathlineto{\pgfqpoint{2.561650in}{1.619432in}}%
\pgfusepath{stroke}%
\end{pgfscope}%
\begin{pgfscope}%
\pgfsetrectcap%
\pgfsetmiterjoin%
\pgfsetlinewidth{1.003750pt}%
\definecolor{currentstroke}{rgb}{0.000000,0.000000,0.000000}%
\pgfsetstrokecolor{currentstroke}%
\pgfsetdash{}{0pt}%
\pgfpathmoveto{\pgfqpoint{0.366840in}{0.449983in}}%
\pgfpathlineto{\pgfqpoint{2.561650in}{0.449983in}}%
\pgfusepath{stroke}%
\end{pgfscope}%
\begin{pgfscope}%
\pgfsetrectcap%
\pgfsetmiterjoin%
\pgfsetlinewidth{1.003750pt}%
\definecolor{currentstroke}{rgb}{0.000000,0.000000,0.000000}%
\pgfsetstrokecolor{currentstroke}%
\pgfsetdash{}{0pt}%
\pgfpathmoveto{\pgfqpoint{0.366840in}{0.449983in}}%
\pgfpathlineto{\pgfqpoint{0.366840in}{1.619432in}}%
\pgfusepath{stroke}%
\end{pgfscope}%
\begin{pgfscope}%
\pgfsetbuttcap%
\pgfsetroundjoin%
\definecolor{currentfill}{rgb}{0.000000,0.000000,0.000000}%
\pgfsetfillcolor{currentfill}%
\pgfsetlinewidth{0.501875pt}%
\definecolor{currentstroke}{rgb}{0.000000,0.000000,0.000000}%
\pgfsetstrokecolor{currentstroke}%
\pgfsetdash{}{0pt}%
\pgfsys@defobject{currentmarker}{\pgfqpoint{0.000000in}{0.000000in}}{\pgfqpoint{0.000000in}{0.069444in}}{%
\pgfpathmoveto{\pgfqpoint{0.000000in}{0.000000in}}%
\pgfpathlineto{\pgfqpoint{0.000000in}{0.069444in}}%
\pgfusepath{stroke,fill}%
}%
\begin{pgfscope}%
\pgfsys@transformshift{0.366840in}{0.449983in}%
\pgfsys@useobject{currentmarker}{}%
\end{pgfscope}%
\end{pgfscope}%
\begin{pgfscope}%
\pgfsetbuttcap%
\pgfsetroundjoin%
\definecolor{currentfill}{rgb}{0.000000,0.000000,0.000000}%
\pgfsetfillcolor{currentfill}%
\pgfsetlinewidth{0.501875pt}%
\definecolor{currentstroke}{rgb}{0.000000,0.000000,0.000000}%
\pgfsetstrokecolor{currentstroke}%
\pgfsetdash{}{0pt}%
\pgfsys@defobject{currentmarker}{\pgfqpoint{0.000000in}{-0.069444in}}{\pgfqpoint{0.000000in}{0.000000in}}{%
\pgfpathmoveto{\pgfqpoint{0.000000in}{0.000000in}}%
\pgfpathlineto{\pgfqpoint{0.000000in}{-0.069444in}}%
\pgfusepath{stroke,fill}%
}%
\begin{pgfscope}%
\pgfsys@transformshift{0.366840in}{1.619432in}%
\pgfsys@useobject{currentmarker}{}%
\end{pgfscope}%
\end{pgfscope}%
\begin{pgfscope}%
\pgftext[x=0.366840in,y=0.380539in,,top]{\rmfamily\fontsize{8.000000}{9.600000}\selectfont −15}%
\end{pgfscope}%
\begin{pgfscope}%
\pgfsetbuttcap%
\pgfsetroundjoin%
\definecolor{currentfill}{rgb}{0.000000,0.000000,0.000000}%
\pgfsetfillcolor{currentfill}%
\pgfsetlinewidth{0.501875pt}%
\definecolor{currentstroke}{rgb}{0.000000,0.000000,0.000000}%
\pgfsetstrokecolor{currentstroke}%
\pgfsetdash{}{0pt}%
\pgfsys@defobject{currentmarker}{\pgfqpoint{0.000000in}{0.000000in}}{\pgfqpoint{0.000000in}{0.069444in}}{%
\pgfpathmoveto{\pgfqpoint{0.000000in}{0.000000in}}%
\pgfpathlineto{\pgfqpoint{0.000000in}{0.069444in}}%
\pgfusepath{stroke,fill}%
}%
\begin{pgfscope}%
\pgfsys@transformshift{0.732642in}{0.449983in}%
\pgfsys@useobject{currentmarker}{}%
\end{pgfscope}%
\end{pgfscope}%
\begin{pgfscope}%
\pgfsetbuttcap%
\pgfsetroundjoin%
\definecolor{currentfill}{rgb}{0.000000,0.000000,0.000000}%
\pgfsetfillcolor{currentfill}%
\pgfsetlinewidth{0.501875pt}%
\definecolor{currentstroke}{rgb}{0.000000,0.000000,0.000000}%
\pgfsetstrokecolor{currentstroke}%
\pgfsetdash{}{0pt}%
\pgfsys@defobject{currentmarker}{\pgfqpoint{0.000000in}{-0.069444in}}{\pgfqpoint{0.000000in}{0.000000in}}{%
\pgfpathmoveto{\pgfqpoint{0.000000in}{0.000000in}}%
\pgfpathlineto{\pgfqpoint{0.000000in}{-0.069444in}}%
\pgfusepath{stroke,fill}%
}%
\begin{pgfscope}%
\pgfsys@transformshift{0.732642in}{1.619432in}%
\pgfsys@useobject{currentmarker}{}%
\end{pgfscope}%
\end{pgfscope}%
\begin{pgfscope}%
\pgftext[x=0.732642in,y=0.380539in,,top]{\rmfamily\fontsize{8.000000}{9.600000}\selectfont −10}%
\end{pgfscope}%
\begin{pgfscope}%
\pgfsetbuttcap%
\pgfsetroundjoin%
\definecolor{currentfill}{rgb}{0.000000,0.000000,0.000000}%
\pgfsetfillcolor{currentfill}%
\pgfsetlinewidth{0.501875pt}%
\definecolor{currentstroke}{rgb}{0.000000,0.000000,0.000000}%
\pgfsetstrokecolor{currentstroke}%
\pgfsetdash{}{0pt}%
\pgfsys@defobject{currentmarker}{\pgfqpoint{0.000000in}{0.000000in}}{\pgfqpoint{0.000000in}{0.069444in}}{%
\pgfpathmoveto{\pgfqpoint{0.000000in}{0.000000in}}%
\pgfpathlineto{\pgfqpoint{0.000000in}{0.069444in}}%
\pgfusepath{stroke,fill}%
}%
\begin{pgfscope}%
\pgfsys@transformshift{1.098443in}{0.449983in}%
\pgfsys@useobject{currentmarker}{}%
\end{pgfscope}%
\end{pgfscope}%
\begin{pgfscope}%
\pgfsetbuttcap%
\pgfsetroundjoin%
\definecolor{currentfill}{rgb}{0.000000,0.000000,0.000000}%
\pgfsetfillcolor{currentfill}%
\pgfsetlinewidth{0.501875pt}%
\definecolor{currentstroke}{rgb}{0.000000,0.000000,0.000000}%
\pgfsetstrokecolor{currentstroke}%
\pgfsetdash{}{0pt}%
\pgfsys@defobject{currentmarker}{\pgfqpoint{0.000000in}{-0.069444in}}{\pgfqpoint{0.000000in}{0.000000in}}{%
\pgfpathmoveto{\pgfqpoint{0.000000in}{0.000000in}}%
\pgfpathlineto{\pgfqpoint{0.000000in}{-0.069444in}}%
\pgfusepath{stroke,fill}%
}%
\begin{pgfscope}%
\pgfsys@transformshift{1.098443in}{1.619432in}%
\pgfsys@useobject{currentmarker}{}%
\end{pgfscope}%
\end{pgfscope}%
\begin{pgfscope}%
\pgftext[x=1.098443in,y=0.380539in,,top]{\rmfamily\fontsize{8.000000}{9.600000}\selectfont −5}%
\end{pgfscope}%
\begin{pgfscope}%
\pgfsetbuttcap%
\pgfsetroundjoin%
\definecolor{currentfill}{rgb}{0.000000,0.000000,0.000000}%
\pgfsetfillcolor{currentfill}%
\pgfsetlinewidth{0.501875pt}%
\definecolor{currentstroke}{rgb}{0.000000,0.000000,0.000000}%
\pgfsetstrokecolor{currentstroke}%
\pgfsetdash{}{0pt}%
\pgfsys@defobject{currentmarker}{\pgfqpoint{0.000000in}{0.000000in}}{\pgfqpoint{0.000000in}{0.069444in}}{%
\pgfpathmoveto{\pgfqpoint{0.000000in}{0.000000in}}%
\pgfpathlineto{\pgfqpoint{0.000000in}{0.069444in}}%
\pgfusepath{stroke,fill}%
}%
\begin{pgfscope}%
\pgfsys@transformshift{1.464245in}{0.449983in}%
\pgfsys@useobject{currentmarker}{}%
\end{pgfscope}%
\end{pgfscope}%
\begin{pgfscope}%
\pgfsetbuttcap%
\pgfsetroundjoin%
\definecolor{currentfill}{rgb}{0.000000,0.000000,0.000000}%
\pgfsetfillcolor{currentfill}%
\pgfsetlinewidth{0.501875pt}%
\definecolor{currentstroke}{rgb}{0.000000,0.000000,0.000000}%
\pgfsetstrokecolor{currentstroke}%
\pgfsetdash{}{0pt}%
\pgfsys@defobject{currentmarker}{\pgfqpoint{0.000000in}{-0.069444in}}{\pgfqpoint{0.000000in}{0.000000in}}{%
\pgfpathmoveto{\pgfqpoint{0.000000in}{0.000000in}}%
\pgfpathlineto{\pgfqpoint{0.000000in}{-0.069444in}}%
\pgfusepath{stroke,fill}%
}%
\begin{pgfscope}%
\pgfsys@transformshift{1.464245in}{1.619432in}%
\pgfsys@useobject{currentmarker}{}%
\end{pgfscope}%
\end{pgfscope}%
\begin{pgfscope}%
\pgftext[x=1.464245in,y=0.380539in,,top]{\rmfamily\fontsize{8.000000}{9.600000}\selectfont 0}%
\end{pgfscope}%
\begin{pgfscope}%
\pgfsetbuttcap%
\pgfsetroundjoin%
\definecolor{currentfill}{rgb}{0.000000,0.000000,0.000000}%
\pgfsetfillcolor{currentfill}%
\pgfsetlinewidth{0.501875pt}%
\definecolor{currentstroke}{rgb}{0.000000,0.000000,0.000000}%
\pgfsetstrokecolor{currentstroke}%
\pgfsetdash{}{0pt}%
\pgfsys@defobject{currentmarker}{\pgfqpoint{0.000000in}{0.000000in}}{\pgfqpoint{0.000000in}{0.069444in}}{%
\pgfpathmoveto{\pgfqpoint{0.000000in}{0.000000in}}%
\pgfpathlineto{\pgfqpoint{0.000000in}{0.069444in}}%
\pgfusepath{stroke,fill}%
}%
\begin{pgfscope}%
\pgfsys@transformshift{1.830047in}{0.449983in}%
\pgfsys@useobject{currentmarker}{}%
\end{pgfscope}%
\end{pgfscope}%
\begin{pgfscope}%
\pgfsetbuttcap%
\pgfsetroundjoin%
\definecolor{currentfill}{rgb}{0.000000,0.000000,0.000000}%
\pgfsetfillcolor{currentfill}%
\pgfsetlinewidth{0.501875pt}%
\definecolor{currentstroke}{rgb}{0.000000,0.000000,0.000000}%
\pgfsetstrokecolor{currentstroke}%
\pgfsetdash{}{0pt}%
\pgfsys@defobject{currentmarker}{\pgfqpoint{0.000000in}{-0.069444in}}{\pgfqpoint{0.000000in}{0.000000in}}{%
\pgfpathmoveto{\pgfqpoint{0.000000in}{0.000000in}}%
\pgfpathlineto{\pgfqpoint{0.000000in}{-0.069444in}}%
\pgfusepath{stroke,fill}%
}%
\begin{pgfscope}%
\pgfsys@transformshift{1.830047in}{1.619432in}%
\pgfsys@useobject{currentmarker}{}%
\end{pgfscope}%
\end{pgfscope}%
\begin{pgfscope}%
\pgftext[x=1.830047in,y=0.380539in,,top]{\rmfamily\fontsize{8.000000}{9.600000}\selectfont 5}%
\end{pgfscope}%
\begin{pgfscope}%
\pgfsetbuttcap%
\pgfsetroundjoin%
\definecolor{currentfill}{rgb}{0.000000,0.000000,0.000000}%
\pgfsetfillcolor{currentfill}%
\pgfsetlinewidth{0.501875pt}%
\definecolor{currentstroke}{rgb}{0.000000,0.000000,0.000000}%
\pgfsetstrokecolor{currentstroke}%
\pgfsetdash{}{0pt}%
\pgfsys@defobject{currentmarker}{\pgfqpoint{0.000000in}{0.000000in}}{\pgfqpoint{0.000000in}{0.069444in}}{%
\pgfpathmoveto{\pgfqpoint{0.000000in}{0.000000in}}%
\pgfpathlineto{\pgfqpoint{0.000000in}{0.069444in}}%
\pgfusepath{stroke,fill}%
}%
\begin{pgfscope}%
\pgfsys@transformshift{2.195848in}{0.449983in}%
\pgfsys@useobject{currentmarker}{}%
\end{pgfscope}%
\end{pgfscope}%
\begin{pgfscope}%
\pgfsetbuttcap%
\pgfsetroundjoin%
\definecolor{currentfill}{rgb}{0.000000,0.000000,0.000000}%
\pgfsetfillcolor{currentfill}%
\pgfsetlinewidth{0.501875pt}%
\definecolor{currentstroke}{rgb}{0.000000,0.000000,0.000000}%
\pgfsetstrokecolor{currentstroke}%
\pgfsetdash{}{0pt}%
\pgfsys@defobject{currentmarker}{\pgfqpoint{0.000000in}{-0.069444in}}{\pgfqpoint{0.000000in}{0.000000in}}{%
\pgfpathmoveto{\pgfqpoint{0.000000in}{0.000000in}}%
\pgfpathlineto{\pgfqpoint{0.000000in}{-0.069444in}}%
\pgfusepath{stroke,fill}%
}%
\begin{pgfscope}%
\pgfsys@transformshift{2.195848in}{1.619432in}%
\pgfsys@useobject{currentmarker}{}%
\end{pgfscope}%
\end{pgfscope}%
\begin{pgfscope}%
\pgftext[x=2.195848in,y=0.380539in,,top]{\rmfamily\fontsize{8.000000}{9.600000}\selectfont 10}%
\end{pgfscope}%
\begin{pgfscope}%
\pgfsetbuttcap%
\pgfsetroundjoin%
\definecolor{currentfill}{rgb}{0.000000,0.000000,0.000000}%
\pgfsetfillcolor{currentfill}%
\pgfsetlinewidth{0.501875pt}%
\definecolor{currentstroke}{rgb}{0.000000,0.000000,0.000000}%
\pgfsetstrokecolor{currentstroke}%
\pgfsetdash{}{0pt}%
\pgfsys@defobject{currentmarker}{\pgfqpoint{0.000000in}{0.000000in}}{\pgfqpoint{0.000000in}{0.069444in}}{%
\pgfpathmoveto{\pgfqpoint{0.000000in}{0.000000in}}%
\pgfpathlineto{\pgfqpoint{0.000000in}{0.069444in}}%
\pgfusepath{stroke,fill}%
}%
\begin{pgfscope}%
\pgfsys@transformshift{2.561650in}{0.449983in}%
\pgfsys@useobject{currentmarker}{}%
\end{pgfscope}%
\end{pgfscope}%
\begin{pgfscope}%
\pgfsetbuttcap%
\pgfsetroundjoin%
\definecolor{currentfill}{rgb}{0.000000,0.000000,0.000000}%
\pgfsetfillcolor{currentfill}%
\pgfsetlinewidth{0.501875pt}%
\definecolor{currentstroke}{rgb}{0.000000,0.000000,0.000000}%
\pgfsetstrokecolor{currentstroke}%
\pgfsetdash{}{0pt}%
\pgfsys@defobject{currentmarker}{\pgfqpoint{0.000000in}{-0.069444in}}{\pgfqpoint{0.000000in}{0.000000in}}{%
\pgfpathmoveto{\pgfqpoint{0.000000in}{0.000000in}}%
\pgfpathlineto{\pgfqpoint{0.000000in}{-0.069444in}}%
\pgfusepath{stroke,fill}%
}%
\begin{pgfscope}%
\pgfsys@transformshift{2.561650in}{1.619432in}%
\pgfsys@useobject{currentmarker}{}%
\end{pgfscope}%
\end{pgfscope}%
\begin{pgfscope}%
\pgftext[x=2.561650in,y=0.380539in,,top]{\rmfamily\fontsize{8.000000}{9.600000}\selectfont 15}%
\end{pgfscope}%
\begin{pgfscope}%
\pgftext[x=1.464245in,y=0.203564in,,top]{\rmfamily\fontsize{9.000000}{10.800000}\selectfont \(\displaystyle \mathrm{DLL}_{\mu/\pi}(\pi^-)\)}%
\end{pgfscope}%
\begin{pgfscope}%
\pgfsetbuttcap%
\pgfsetroundjoin%
\definecolor{currentfill}{rgb}{0.000000,0.000000,0.000000}%
\pgfsetfillcolor{currentfill}%
\pgfsetlinewidth{0.501875pt}%
\definecolor{currentstroke}{rgb}{0.000000,0.000000,0.000000}%
\pgfsetstrokecolor{currentstroke}%
\pgfsetdash{}{0pt}%
\pgfsys@defobject{currentmarker}{\pgfqpoint{0.000000in}{0.000000in}}{\pgfqpoint{0.069444in}{0.000000in}}{%
\pgfpathmoveto{\pgfqpoint{0.000000in}{0.000000in}}%
\pgfpathlineto{\pgfqpoint{0.069444in}{0.000000in}}%
\pgfusepath{stroke,fill}%
}%
\begin{pgfscope}%
\pgfsys@transformshift{0.366840in}{0.449983in}%
\pgfsys@useobject{currentmarker}{}%
\end{pgfscope}%
\end{pgfscope}%
\begin{pgfscope}%
\pgfsetbuttcap%
\pgfsetroundjoin%
\definecolor{currentfill}{rgb}{0.000000,0.000000,0.000000}%
\pgfsetfillcolor{currentfill}%
\pgfsetlinewidth{0.501875pt}%
\definecolor{currentstroke}{rgb}{0.000000,0.000000,0.000000}%
\pgfsetstrokecolor{currentstroke}%
\pgfsetdash{}{0pt}%
\pgfsys@defobject{currentmarker}{\pgfqpoint{-0.069444in}{0.000000in}}{\pgfqpoint{0.000000in}{0.000000in}}{%
\pgfpathmoveto{\pgfqpoint{0.000000in}{0.000000in}}%
\pgfpathlineto{\pgfqpoint{-0.069444in}{0.000000in}}%
\pgfusepath{stroke,fill}%
}%
\begin{pgfscope}%
\pgfsys@transformshift{2.561650in}{0.449983in}%
\pgfsys@useobject{currentmarker}{}%
\end{pgfscope}%
\end{pgfscope}%
\begin{pgfscope}%
\pgftext[x=0.297396in,y=0.449983in,right,]{\rmfamily\fontsize{8.000000}{9.600000}\selectfont 0.00}%
\end{pgfscope}%
\begin{pgfscope}%
\pgfsetbuttcap%
\pgfsetroundjoin%
\definecolor{currentfill}{rgb}{0.000000,0.000000,0.000000}%
\pgfsetfillcolor{currentfill}%
\pgfsetlinewidth{0.501875pt}%
\definecolor{currentstroke}{rgb}{0.000000,0.000000,0.000000}%
\pgfsetstrokecolor{currentstroke}%
\pgfsetdash{}{0pt}%
\pgfsys@defobject{currentmarker}{\pgfqpoint{0.000000in}{0.000000in}}{\pgfqpoint{0.069444in}{0.000000in}}{%
\pgfpathmoveto{\pgfqpoint{0.000000in}{0.000000in}}%
\pgfpathlineto{\pgfqpoint{0.069444in}{0.000000in}}%
\pgfusepath{stroke,fill}%
}%
\begin{pgfscope}%
\pgfsys@transformshift{0.366840in}{0.644892in}%
\pgfsys@useobject{currentmarker}{}%
\end{pgfscope}%
\end{pgfscope}%
\begin{pgfscope}%
\pgfsetbuttcap%
\pgfsetroundjoin%
\definecolor{currentfill}{rgb}{0.000000,0.000000,0.000000}%
\pgfsetfillcolor{currentfill}%
\pgfsetlinewidth{0.501875pt}%
\definecolor{currentstroke}{rgb}{0.000000,0.000000,0.000000}%
\pgfsetstrokecolor{currentstroke}%
\pgfsetdash{}{0pt}%
\pgfsys@defobject{currentmarker}{\pgfqpoint{-0.069444in}{0.000000in}}{\pgfqpoint{0.000000in}{0.000000in}}{%
\pgfpathmoveto{\pgfqpoint{0.000000in}{0.000000in}}%
\pgfpathlineto{\pgfqpoint{-0.069444in}{0.000000in}}%
\pgfusepath{stroke,fill}%
}%
\begin{pgfscope}%
\pgfsys@transformshift{2.561650in}{0.644892in}%
\pgfsys@useobject{currentmarker}{}%
\end{pgfscope}%
\end{pgfscope}%
\begin{pgfscope}%
\pgftext[x=0.297396in,y=0.644892in,right,]{\rmfamily\fontsize{8.000000}{9.600000}\selectfont 0.02}%
\end{pgfscope}%
\begin{pgfscope}%
\pgfsetbuttcap%
\pgfsetroundjoin%
\definecolor{currentfill}{rgb}{0.000000,0.000000,0.000000}%
\pgfsetfillcolor{currentfill}%
\pgfsetlinewidth{0.501875pt}%
\definecolor{currentstroke}{rgb}{0.000000,0.000000,0.000000}%
\pgfsetstrokecolor{currentstroke}%
\pgfsetdash{}{0pt}%
\pgfsys@defobject{currentmarker}{\pgfqpoint{0.000000in}{0.000000in}}{\pgfqpoint{0.069444in}{0.000000in}}{%
\pgfpathmoveto{\pgfqpoint{0.000000in}{0.000000in}}%
\pgfpathlineto{\pgfqpoint{0.069444in}{0.000000in}}%
\pgfusepath{stroke,fill}%
}%
\begin{pgfscope}%
\pgfsys@transformshift{0.366840in}{0.839800in}%
\pgfsys@useobject{currentmarker}{}%
\end{pgfscope}%
\end{pgfscope}%
\begin{pgfscope}%
\pgfsetbuttcap%
\pgfsetroundjoin%
\definecolor{currentfill}{rgb}{0.000000,0.000000,0.000000}%
\pgfsetfillcolor{currentfill}%
\pgfsetlinewidth{0.501875pt}%
\definecolor{currentstroke}{rgb}{0.000000,0.000000,0.000000}%
\pgfsetstrokecolor{currentstroke}%
\pgfsetdash{}{0pt}%
\pgfsys@defobject{currentmarker}{\pgfqpoint{-0.069444in}{0.000000in}}{\pgfqpoint{0.000000in}{0.000000in}}{%
\pgfpathmoveto{\pgfqpoint{0.000000in}{0.000000in}}%
\pgfpathlineto{\pgfqpoint{-0.069444in}{0.000000in}}%
\pgfusepath{stroke,fill}%
}%
\begin{pgfscope}%
\pgfsys@transformshift{2.561650in}{0.839800in}%
\pgfsys@useobject{currentmarker}{}%
\end{pgfscope}%
\end{pgfscope}%
\begin{pgfscope}%
\pgftext[x=0.297396in,y=0.839800in,right,]{\rmfamily\fontsize{8.000000}{9.600000}\selectfont 0.04}%
\end{pgfscope}%
\begin{pgfscope}%
\pgfsetbuttcap%
\pgfsetroundjoin%
\definecolor{currentfill}{rgb}{0.000000,0.000000,0.000000}%
\pgfsetfillcolor{currentfill}%
\pgfsetlinewidth{0.501875pt}%
\definecolor{currentstroke}{rgb}{0.000000,0.000000,0.000000}%
\pgfsetstrokecolor{currentstroke}%
\pgfsetdash{}{0pt}%
\pgfsys@defobject{currentmarker}{\pgfqpoint{0.000000in}{0.000000in}}{\pgfqpoint{0.069444in}{0.000000in}}{%
\pgfpathmoveto{\pgfqpoint{0.000000in}{0.000000in}}%
\pgfpathlineto{\pgfqpoint{0.069444in}{0.000000in}}%
\pgfusepath{stroke,fill}%
}%
\begin{pgfscope}%
\pgfsys@transformshift{0.366840in}{1.034708in}%
\pgfsys@useobject{currentmarker}{}%
\end{pgfscope}%
\end{pgfscope}%
\begin{pgfscope}%
\pgfsetbuttcap%
\pgfsetroundjoin%
\definecolor{currentfill}{rgb}{0.000000,0.000000,0.000000}%
\pgfsetfillcolor{currentfill}%
\pgfsetlinewidth{0.501875pt}%
\definecolor{currentstroke}{rgb}{0.000000,0.000000,0.000000}%
\pgfsetstrokecolor{currentstroke}%
\pgfsetdash{}{0pt}%
\pgfsys@defobject{currentmarker}{\pgfqpoint{-0.069444in}{0.000000in}}{\pgfqpoint{0.000000in}{0.000000in}}{%
\pgfpathmoveto{\pgfqpoint{0.000000in}{0.000000in}}%
\pgfpathlineto{\pgfqpoint{-0.069444in}{0.000000in}}%
\pgfusepath{stroke,fill}%
}%
\begin{pgfscope}%
\pgfsys@transformshift{2.561650in}{1.034708in}%
\pgfsys@useobject{currentmarker}{}%
\end{pgfscope}%
\end{pgfscope}%
\begin{pgfscope}%
\pgftext[x=0.297396in,y=1.034708in,right,]{\rmfamily\fontsize{8.000000}{9.600000}\selectfont 0.06}%
\end{pgfscope}%
\begin{pgfscope}%
\pgfsetbuttcap%
\pgfsetroundjoin%
\definecolor{currentfill}{rgb}{0.000000,0.000000,0.000000}%
\pgfsetfillcolor{currentfill}%
\pgfsetlinewidth{0.501875pt}%
\definecolor{currentstroke}{rgb}{0.000000,0.000000,0.000000}%
\pgfsetstrokecolor{currentstroke}%
\pgfsetdash{}{0pt}%
\pgfsys@defobject{currentmarker}{\pgfqpoint{0.000000in}{0.000000in}}{\pgfqpoint{0.069444in}{0.000000in}}{%
\pgfpathmoveto{\pgfqpoint{0.000000in}{0.000000in}}%
\pgfpathlineto{\pgfqpoint{0.069444in}{0.000000in}}%
\pgfusepath{stroke,fill}%
}%
\begin{pgfscope}%
\pgfsys@transformshift{0.366840in}{1.229616in}%
\pgfsys@useobject{currentmarker}{}%
\end{pgfscope}%
\end{pgfscope}%
\begin{pgfscope}%
\pgfsetbuttcap%
\pgfsetroundjoin%
\definecolor{currentfill}{rgb}{0.000000,0.000000,0.000000}%
\pgfsetfillcolor{currentfill}%
\pgfsetlinewidth{0.501875pt}%
\definecolor{currentstroke}{rgb}{0.000000,0.000000,0.000000}%
\pgfsetstrokecolor{currentstroke}%
\pgfsetdash{}{0pt}%
\pgfsys@defobject{currentmarker}{\pgfqpoint{-0.069444in}{0.000000in}}{\pgfqpoint{0.000000in}{0.000000in}}{%
\pgfpathmoveto{\pgfqpoint{0.000000in}{0.000000in}}%
\pgfpathlineto{\pgfqpoint{-0.069444in}{0.000000in}}%
\pgfusepath{stroke,fill}%
}%
\begin{pgfscope}%
\pgfsys@transformshift{2.561650in}{1.229616in}%
\pgfsys@useobject{currentmarker}{}%
\end{pgfscope}%
\end{pgfscope}%
\begin{pgfscope}%
\pgftext[x=0.297396in,y=1.229616in,right,]{\rmfamily\fontsize{8.000000}{9.600000}\selectfont 0.08}%
\end{pgfscope}%
\begin{pgfscope}%
\pgfsetbuttcap%
\pgfsetroundjoin%
\definecolor{currentfill}{rgb}{0.000000,0.000000,0.000000}%
\pgfsetfillcolor{currentfill}%
\pgfsetlinewidth{0.501875pt}%
\definecolor{currentstroke}{rgb}{0.000000,0.000000,0.000000}%
\pgfsetstrokecolor{currentstroke}%
\pgfsetdash{}{0pt}%
\pgfsys@defobject{currentmarker}{\pgfqpoint{0.000000in}{0.000000in}}{\pgfqpoint{0.069444in}{0.000000in}}{%
\pgfpathmoveto{\pgfqpoint{0.000000in}{0.000000in}}%
\pgfpathlineto{\pgfqpoint{0.069444in}{0.000000in}}%
\pgfusepath{stroke,fill}%
}%
\begin{pgfscope}%
\pgfsys@transformshift{0.366840in}{1.424524in}%
\pgfsys@useobject{currentmarker}{}%
\end{pgfscope}%
\end{pgfscope}%
\begin{pgfscope}%
\pgfsetbuttcap%
\pgfsetroundjoin%
\definecolor{currentfill}{rgb}{0.000000,0.000000,0.000000}%
\pgfsetfillcolor{currentfill}%
\pgfsetlinewidth{0.501875pt}%
\definecolor{currentstroke}{rgb}{0.000000,0.000000,0.000000}%
\pgfsetstrokecolor{currentstroke}%
\pgfsetdash{}{0pt}%
\pgfsys@defobject{currentmarker}{\pgfqpoint{-0.069444in}{0.000000in}}{\pgfqpoint{0.000000in}{0.000000in}}{%
\pgfpathmoveto{\pgfqpoint{0.000000in}{0.000000in}}%
\pgfpathlineto{\pgfqpoint{-0.069444in}{0.000000in}}%
\pgfusepath{stroke,fill}%
}%
\begin{pgfscope}%
\pgfsys@transformshift{2.561650in}{1.424524in}%
\pgfsys@useobject{currentmarker}{}%
\end{pgfscope}%
\end{pgfscope}%
\begin{pgfscope}%
\pgftext[x=0.297396in,y=1.424524in,right,]{\rmfamily\fontsize{8.000000}{9.600000}\selectfont 0.10}%
\end{pgfscope}%
\begin{pgfscope}%
\pgfsetbuttcap%
\pgfsetroundjoin%
\definecolor{currentfill}{rgb}{0.000000,0.000000,0.000000}%
\pgfsetfillcolor{currentfill}%
\pgfsetlinewidth{0.501875pt}%
\definecolor{currentstroke}{rgb}{0.000000,0.000000,0.000000}%
\pgfsetstrokecolor{currentstroke}%
\pgfsetdash{}{0pt}%
\pgfsys@defobject{currentmarker}{\pgfqpoint{0.000000in}{0.000000in}}{\pgfqpoint{0.069444in}{0.000000in}}{%
\pgfpathmoveto{\pgfqpoint{0.000000in}{0.000000in}}%
\pgfpathlineto{\pgfqpoint{0.069444in}{0.000000in}}%
\pgfusepath{stroke,fill}%
}%
\begin{pgfscope}%
\pgfsys@transformshift{0.366840in}{1.619432in}%
\pgfsys@useobject{currentmarker}{}%
\end{pgfscope}%
\end{pgfscope}%
\begin{pgfscope}%
\pgfsetbuttcap%
\pgfsetroundjoin%
\definecolor{currentfill}{rgb}{0.000000,0.000000,0.000000}%
\pgfsetfillcolor{currentfill}%
\pgfsetlinewidth{0.501875pt}%
\definecolor{currentstroke}{rgb}{0.000000,0.000000,0.000000}%
\pgfsetstrokecolor{currentstroke}%
\pgfsetdash{}{0pt}%
\pgfsys@defobject{currentmarker}{\pgfqpoint{-0.069444in}{0.000000in}}{\pgfqpoint{0.000000in}{0.000000in}}{%
\pgfpathmoveto{\pgfqpoint{0.000000in}{0.000000in}}%
\pgfpathlineto{\pgfqpoint{-0.069444in}{0.000000in}}%
\pgfusepath{stroke,fill}%
}%
\begin{pgfscope}%
\pgfsys@transformshift{2.561650in}{1.619432in}%
\pgfsys@useobject{currentmarker}{}%
\end{pgfscope}%
\end{pgfscope}%
\begin{pgfscope}%
\pgftext[x=0.297396in,y=1.619432in,right,]{\rmfamily\fontsize{8.000000}{9.600000}\selectfont 0.12}%
\end{pgfscope}%
\end{pgfpicture}%
\makeatother%
\endgroup%

	\end{subfigure}

	\begin{subfigure}[t]{0.49\textwidth}
		\centering
    %\includegraphics[width=\textwidth]{store/variables/SIG_BKG_muminus_PIDmu.pdf}
    %% Creator: Matplotlib, PGF backend
%%
%% To include the figure in your LaTeX document, write
%%   \input{<filename>.pgf}
%%
%% Make sure the required packages are loaded in your preamble
%%   \usepackage{pgf}
%%
%% Figures using additional raster images can only be included by \input if
%% they are in the same directory as the main LaTeX file. For loading figures
%% from other directories you can use the `import` package
%%   \usepackage{import}
%% and then include the figures with
%%   \import{<path to file>}{<filename>.pgf}
%%
%% Matplotlib used the following preamble
%%   \usepackage{fontspec}
%%   \setmainfont{DejaVu Serif}
%%   \setsansfont{DejaVu Sans}
%%   \setmonofont{DejaVu Sans Mono}
%%
\begingroup%
\makeatletter%
\begin{pgfpicture}%
\pgfpathrectangle{\pgfpointorigin}{\pgfqpoint{2.682342in}{1.723197in}}%
\pgfusepath{use as bounding box, clip}%
\begin{pgfscope}%
\pgfsetbuttcap%
\pgfsetmiterjoin%
\definecolor{currentfill}{rgb}{1.000000,1.000000,1.000000}%
\pgfsetfillcolor{currentfill}%
\pgfsetlinewidth{0.000000pt}%
\definecolor{currentstroke}{rgb}{1.000000,1.000000,1.000000}%
\pgfsetstrokecolor{currentstroke}%
\pgfsetdash{}{0pt}%
\pgfpathmoveto{\pgfqpoint{0.000000in}{0.000000in}}%
\pgfpathlineto{\pgfqpoint{2.682342in}{0.000000in}}%
\pgfpathlineto{\pgfqpoint{2.682342in}{1.723197in}}%
\pgfpathlineto{\pgfqpoint{0.000000in}{1.723197in}}%
\pgfpathclose%
\pgfusepath{fill}%
\end{pgfscope}%
\begin{pgfscope}%
\pgfsetbuttcap%
\pgfsetmiterjoin%
\definecolor{currentfill}{rgb}{1.000000,1.000000,1.000000}%
\pgfsetfillcolor{currentfill}%
\pgfsetlinewidth{0.000000pt}%
\definecolor{currentstroke}{rgb}{0.000000,0.000000,0.000000}%
\pgfsetstrokecolor{currentstroke}%
\pgfsetstrokeopacity{0.000000}%
\pgfsetdash{}{0pt}%
\pgfpathmoveto{\pgfqpoint{0.366840in}{0.449983in}}%
\pgfpathlineto{\pgfqpoint{2.561650in}{0.449983in}}%
\pgfpathlineto{\pgfqpoint{2.561650in}{1.619432in}}%
\pgfpathlineto{\pgfqpoint{0.366840in}{1.619432in}}%
\pgfpathclose%
\pgfusepath{fill}%
\end{pgfscope}%
\begin{pgfscope}%
\pgfpathrectangle{\pgfqpoint{0.366840in}{0.449983in}}{\pgfqpoint{2.194810in}{1.169449in}} %
\pgfusepath{clip}%
\pgfsetbuttcap%
\pgfsetmiterjoin%
\definecolor{currentfill}{rgb}{0.215686,0.470588,0.749020}%
\pgfsetfillcolor{currentfill}%
\pgfsetlinewidth{0.000000pt}%
\definecolor{currentstroke}{rgb}{0.000000,0.000000,0.000000}%
\pgfsetstrokecolor{currentstroke}%
\pgfsetdash{}{0pt}%
\pgfpathmoveto{\pgfqpoint{0.586370in}{0.449983in}}%
\pgfpathlineto{\pgfqpoint{0.586370in}{0.469705in}}%
\pgfpathlineto{\pgfqpoint{0.625210in}{0.469705in}}%
\pgfpathlineto{\pgfqpoint{0.625210in}{0.465834in}}%
\pgfpathlineto{\pgfqpoint{0.664051in}{0.465834in}}%
\pgfpathlineto{\pgfqpoint{0.664051in}{0.488660in}}%
\pgfpathlineto{\pgfqpoint{0.702891in}{0.488660in}}%
\pgfpathlineto{\pgfqpoint{0.702891in}{0.491829in}}%
\pgfpathlineto{\pgfqpoint{0.741732in}{0.491829in}}%
\pgfpathlineto{\pgfqpoint{0.741732in}{0.491609in}}%
\pgfpathlineto{\pgfqpoint{0.780572in}{0.491609in}}%
\pgfpathlineto{\pgfqpoint{0.780572in}{0.525420in}}%
\pgfpathlineto{\pgfqpoint{0.819413in}{0.525420in}}%
\pgfpathlineto{\pgfqpoint{0.819413in}{0.513905in}}%
\pgfpathlineto{\pgfqpoint{0.858253in}{0.513905in}}%
\pgfpathlineto{\pgfqpoint{0.858253in}{0.528256in}}%
\pgfpathlineto{\pgfqpoint{0.897094in}{0.528256in}}%
\pgfpathlineto{\pgfqpoint{0.897094in}{0.567206in}}%
\pgfpathlineto{\pgfqpoint{0.935935in}{0.567206in}}%
\pgfpathlineto{\pgfqpoint{0.935935in}{0.577579in}}%
\pgfpathlineto{\pgfqpoint{0.974775in}{0.577579in}}%
\pgfpathlineto{\pgfqpoint{0.974775in}{0.569519in}}%
\pgfpathlineto{\pgfqpoint{1.013616in}{0.569519in}}%
\pgfpathlineto{\pgfqpoint{1.013616in}{0.645346in}}%
\pgfpathlineto{\pgfqpoint{1.052456in}{0.645346in}}%
\pgfpathlineto{\pgfqpoint{1.052456in}{0.653144in}}%
\pgfpathlineto{\pgfqpoint{1.091297in}{0.653144in}}%
\pgfpathlineto{\pgfqpoint{1.091297in}{0.690549in}}%
\pgfpathlineto{\pgfqpoint{1.130137in}{0.690549in}}%
\pgfpathlineto{\pgfqpoint{1.130137in}{0.756564in}}%
\pgfpathlineto{\pgfqpoint{1.168978in}{0.756564in}}%
\pgfpathlineto{\pgfqpoint{1.168978in}{0.853748in}}%
\pgfpathlineto{\pgfqpoint{1.207818in}{0.853748in}}%
\pgfpathlineto{\pgfqpoint{1.207818in}{0.871097in}}%
\pgfpathlineto{\pgfqpoint{1.246659in}{0.871097in}}%
\pgfpathlineto{\pgfqpoint{1.246659in}{1.000228in}}%
\pgfpathlineto{\pgfqpoint{1.285500in}{1.000228in}}%
\pgfpathlineto{\pgfqpoint{1.285500in}{1.016409in}}%
\pgfpathlineto{\pgfqpoint{1.324340in}{1.016409in}}%
\pgfpathlineto{\pgfqpoint{1.324340in}{1.127729in}}%
\pgfpathlineto{\pgfqpoint{1.363181in}{1.127729in}}%
\pgfpathlineto{\pgfqpoint{1.363181in}{1.196708in}}%
\pgfpathlineto{\pgfqpoint{1.402021in}{1.196708in}}%
\pgfpathlineto{\pgfqpoint{1.402021in}{1.262501in}}%
\pgfpathlineto{\pgfqpoint{1.440862in}{1.262501in}}%
\pgfpathlineto{\pgfqpoint{1.440862in}{1.263044in}}%
\pgfpathlineto{\pgfqpoint{1.479702in}{1.263044in}}%
\pgfpathlineto{\pgfqpoint{1.479702in}{1.326516in}}%
\pgfpathlineto{\pgfqpoint{1.518543in}{1.326516in}}%
\pgfpathlineto{\pgfqpoint{1.518543in}{1.439677in}}%
\pgfpathlineto{\pgfqpoint{1.557383in}{1.439677in}}%
\pgfpathlineto{\pgfqpoint{1.557383in}{1.441461in}}%
\pgfpathlineto{\pgfqpoint{1.596224in}{1.441461in}}%
\pgfpathlineto{\pgfqpoint{1.596224in}{1.514468in}}%
\pgfpathlineto{\pgfqpoint{1.635065in}{1.514468in}}%
\pgfpathlineto{\pgfqpoint{1.635065in}{1.551085in}}%
\pgfpathlineto{\pgfqpoint{1.673905in}{1.551085in}}%
\pgfpathlineto{\pgfqpoint{1.673905in}{1.517978in}}%
\pgfpathlineto{\pgfqpoint{1.712746in}{1.517978in}}%
\pgfpathlineto{\pgfqpoint{1.712746in}{1.446230in}}%
\pgfpathlineto{\pgfqpoint{1.751586in}{1.446230in}}%
\pgfpathlineto{\pgfqpoint{1.751586in}{1.434508in}}%
\pgfpathlineto{\pgfqpoint{1.790427in}{1.434508in}}%
\pgfpathlineto{\pgfqpoint{1.790427in}{1.278519in}}%
\pgfpathlineto{\pgfqpoint{1.829267in}{1.278519in}}%
\pgfpathlineto{\pgfqpoint{1.829267in}{1.140223in}}%
\pgfpathlineto{\pgfqpoint{1.868108in}{1.140223in}}%
\pgfpathlineto{\pgfqpoint{1.868108in}{1.040414in}}%
\pgfpathlineto{\pgfqpoint{1.906949in}{1.040414in}}%
\pgfpathlineto{\pgfqpoint{1.906949in}{0.967932in}}%
\pgfpathlineto{\pgfqpoint{1.945789in}{0.967932in}}%
\pgfpathlineto{\pgfqpoint{1.945789in}{0.896303in}}%
\pgfpathlineto{\pgfqpoint{1.984630in}{0.896303in}}%
\pgfpathlineto{\pgfqpoint{1.984630in}{0.882189in}}%
\pgfpathlineto{\pgfqpoint{2.023470in}{0.882189in}}%
\pgfpathlineto{\pgfqpoint{2.023470in}{0.887771in}}%
\pgfpathlineto{\pgfqpoint{2.062311in}{0.887771in}}%
\pgfpathlineto{\pgfqpoint{2.062311in}{0.876932in}}%
\pgfpathlineto{\pgfqpoint{2.101151in}{0.876932in}}%
\pgfpathlineto{\pgfqpoint{2.101151in}{0.847448in}}%
\pgfpathlineto{\pgfqpoint{2.139992in}{0.847448in}}%
\pgfpathlineto{\pgfqpoint{2.139992in}{0.800080in}}%
\pgfpathlineto{\pgfqpoint{2.178832in}{0.800080in}}%
\pgfpathlineto{\pgfqpoint{2.178832in}{0.688025in}}%
\pgfpathlineto{\pgfqpoint{2.217673in}{0.688025in}}%
\pgfpathlineto{\pgfqpoint{2.217673in}{0.657571in}}%
\pgfpathlineto{\pgfqpoint{2.256514in}{0.657571in}}%
\pgfpathlineto{\pgfqpoint{2.256514in}{0.597762in}}%
\pgfpathlineto{\pgfqpoint{2.295354in}{0.597762in}}%
\pgfpathlineto{\pgfqpoint{2.295354in}{0.528684in}}%
\pgfpathlineto{\pgfqpoint{2.334195in}{0.528684in}}%
\pgfpathlineto{\pgfqpoint{2.334195in}{0.498568in}}%
\pgfpathlineto{\pgfqpoint{2.373035in}{0.498568in}}%
\pgfpathlineto{\pgfqpoint{2.373035in}{0.489941in}}%
\pgfpathlineto{\pgfqpoint{2.411876in}{0.489941in}}%
\pgfpathlineto{\pgfqpoint{2.411876in}{0.463572in}}%
\pgfpathlineto{\pgfqpoint{2.450716in}{0.463572in}}%
\pgfpathlineto{\pgfqpoint{2.450716in}{0.455921in}}%
\pgfpathlineto{\pgfqpoint{2.489557in}{0.455921in}}%
\pgfpathlineto{\pgfqpoint{2.489557in}{0.453877in}}%
\pgfpathlineto{\pgfqpoint{2.528397in}{0.453877in}}%
\pgfpathlineto{\pgfqpoint{2.528397in}{0.449983in}}%
\pgfpathlineto{\pgfqpoint{2.489557in}{0.449983in}}%
\pgfpathlineto{\pgfqpoint{2.489557in}{0.449983in}}%
\pgfpathlineto{\pgfqpoint{2.450716in}{0.449983in}}%
\pgfpathlineto{\pgfqpoint{2.450716in}{0.449983in}}%
\pgfpathlineto{\pgfqpoint{2.411876in}{0.449983in}}%
\pgfpathlineto{\pgfqpoint{2.411876in}{0.449983in}}%
\pgfpathlineto{\pgfqpoint{2.373035in}{0.449983in}}%
\pgfpathlineto{\pgfqpoint{2.373035in}{0.449983in}}%
\pgfpathlineto{\pgfqpoint{2.334195in}{0.449983in}}%
\pgfpathlineto{\pgfqpoint{2.334195in}{0.449983in}}%
\pgfpathlineto{\pgfqpoint{2.295354in}{0.449983in}}%
\pgfpathlineto{\pgfqpoint{2.295354in}{0.449983in}}%
\pgfpathlineto{\pgfqpoint{2.256514in}{0.449983in}}%
\pgfpathlineto{\pgfqpoint{2.256514in}{0.449983in}}%
\pgfpathlineto{\pgfqpoint{2.217673in}{0.449983in}}%
\pgfpathlineto{\pgfqpoint{2.217673in}{0.449983in}}%
\pgfpathlineto{\pgfqpoint{2.178832in}{0.449983in}}%
\pgfpathlineto{\pgfqpoint{2.178832in}{0.449983in}}%
\pgfpathlineto{\pgfqpoint{2.139992in}{0.449983in}}%
\pgfpathlineto{\pgfqpoint{2.139992in}{0.449983in}}%
\pgfpathlineto{\pgfqpoint{2.101151in}{0.449983in}}%
\pgfpathlineto{\pgfqpoint{2.101151in}{0.449983in}}%
\pgfpathlineto{\pgfqpoint{2.062311in}{0.449983in}}%
\pgfpathlineto{\pgfqpoint{2.062311in}{0.449983in}}%
\pgfpathlineto{\pgfqpoint{2.023470in}{0.449983in}}%
\pgfpathlineto{\pgfqpoint{2.023470in}{0.449983in}}%
\pgfpathlineto{\pgfqpoint{1.984630in}{0.449983in}}%
\pgfpathlineto{\pgfqpoint{1.984630in}{0.449983in}}%
\pgfpathlineto{\pgfqpoint{1.945789in}{0.449983in}}%
\pgfpathlineto{\pgfqpoint{1.945789in}{0.449983in}}%
\pgfpathlineto{\pgfqpoint{1.906949in}{0.449983in}}%
\pgfpathlineto{\pgfqpoint{1.906949in}{0.449983in}}%
\pgfpathlineto{\pgfqpoint{1.868108in}{0.449983in}}%
\pgfpathlineto{\pgfqpoint{1.868108in}{0.449983in}}%
\pgfpathlineto{\pgfqpoint{1.829267in}{0.449983in}}%
\pgfpathlineto{\pgfqpoint{1.829267in}{0.449983in}}%
\pgfpathlineto{\pgfqpoint{1.790427in}{0.449983in}}%
\pgfpathlineto{\pgfqpoint{1.790427in}{0.449983in}}%
\pgfpathlineto{\pgfqpoint{1.751586in}{0.449983in}}%
\pgfpathlineto{\pgfqpoint{1.751586in}{0.449983in}}%
\pgfpathlineto{\pgfqpoint{1.712746in}{0.449983in}}%
\pgfpathlineto{\pgfqpoint{1.712746in}{0.449983in}}%
\pgfpathlineto{\pgfqpoint{1.673905in}{0.449983in}}%
\pgfpathlineto{\pgfqpoint{1.673905in}{0.449983in}}%
\pgfpathlineto{\pgfqpoint{1.635065in}{0.449983in}}%
\pgfpathlineto{\pgfqpoint{1.635065in}{0.449983in}}%
\pgfpathlineto{\pgfqpoint{1.596224in}{0.449983in}}%
\pgfpathlineto{\pgfqpoint{1.596224in}{0.449983in}}%
\pgfpathlineto{\pgfqpoint{1.557383in}{0.449983in}}%
\pgfpathlineto{\pgfqpoint{1.557383in}{0.449983in}}%
\pgfpathlineto{\pgfqpoint{1.518543in}{0.449983in}}%
\pgfpathlineto{\pgfqpoint{1.518543in}{0.449983in}}%
\pgfpathlineto{\pgfqpoint{1.479702in}{0.449983in}}%
\pgfpathlineto{\pgfqpoint{1.479702in}{0.449983in}}%
\pgfpathlineto{\pgfqpoint{1.440862in}{0.449983in}}%
\pgfpathlineto{\pgfqpoint{1.440862in}{0.449983in}}%
\pgfpathlineto{\pgfqpoint{1.402021in}{0.449983in}}%
\pgfpathlineto{\pgfqpoint{1.402021in}{0.449983in}}%
\pgfpathlineto{\pgfqpoint{1.363181in}{0.449983in}}%
\pgfpathlineto{\pgfqpoint{1.363181in}{0.449983in}}%
\pgfpathlineto{\pgfqpoint{1.324340in}{0.449983in}}%
\pgfpathlineto{\pgfqpoint{1.324340in}{0.449983in}}%
\pgfpathlineto{\pgfqpoint{1.285500in}{0.449983in}}%
\pgfpathlineto{\pgfqpoint{1.285500in}{0.449983in}}%
\pgfpathlineto{\pgfqpoint{1.246659in}{0.449983in}}%
\pgfpathlineto{\pgfqpoint{1.246659in}{0.449983in}}%
\pgfpathlineto{\pgfqpoint{1.207818in}{0.449983in}}%
\pgfpathlineto{\pgfqpoint{1.207818in}{0.449983in}}%
\pgfpathlineto{\pgfqpoint{1.168978in}{0.449983in}}%
\pgfpathlineto{\pgfqpoint{1.168978in}{0.449983in}}%
\pgfpathlineto{\pgfqpoint{1.130137in}{0.449983in}}%
\pgfpathlineto{\pgfqpoint{1.130137in}{0.449983in}}%
\pgfpathlineto{\pgfqpoint{1.091297in}{0.449983in}}%
\pgfpathlineto{\pgfqpoint{1.091297in}{0.449983in}}%
\pgfpathlineto{\pgfqpoint{1.052456in}{0.449983in}}%
\pgfpathlineto{\pgfqpoint{1.052456in}{0.449983in}}%
\pgfpathlineto{\pgfqpoint{1.013616in}{0.449983in}}%
\pgfpathlineto{\pgfqpoint{1.013616in}{0.449983in}}%
\pgfpathlineto{\pgfqpoint{0.974775in}{0.449983in}}%
\pgfpathlineto{\pgfqpoint{0.974775in}{0.449983in}}%
\pgfpathlineto{\pgfqpoint{0.935935in}{0.449983in}}%
\pgfpathlineto{\pgfqpoint{0.935935in}{0.449983in}}%
\pgfpathlineto{\pgfqpoint{0.897094in}{0.449983in}}%
\pgfpathlineto{\pgfqpoint{0.897094in}{0.449983in}}%
\pgfpathlineto{\pgfqpoint{0.858253in}{0.449983in}}%
\pgfpathlineto{\pgfqpoint{0.858253in}{0.449983in}}%
\pgfpathlineto{\pgfqpoint{0.819413in}{0.449983in}}%
\pgfpathlineto{\pgfqpoint{0.819413in}{0.449983in}}%
\pgfpathlineto{\pgfqpoint{0.780572in}{0.449983in}}%
\pgfpathlineto{\pgfqpoint{0.780572in}{0.449983in}}%
\pgfpathlineto{\pgfqpoint{0.741732in}{0.449983in}}%
\pgfpathlineto{\pgfqpoint{0.741732in}{0.449983in}}%
\pgfpathlineto{\pgfqpoint{0.702891in}{0.449983in}}%
\pgfpathlineto{\pgfqpoint{0.702891in}{0.449983in}}%
\pgfpathlineto{\pgfqpoint{0.664051in}{0.449983in}}%
\pgfpathlineto{\pgfqpoint{0.664051in}{0.449983in}}%
\pgfpathlineto{\pgfqpoint{0.625210in}{0.449983in}}%
\pgfpathlineto{\pgfqpoint{0.625210in}{0.449983in}}%
\pgfpathlineto{\pgfqpoint{0.586370in}{0.449983in}}%
\pgfusepath{fill}%
\end{pgfscope}%
\begin{pgfscope}%
\pgfpathrectangle{\pgfqpoint{0.366840in}{0.449983in}}{\pgfqpoint{2.194810in}{1.169449in}} %
\pgfusepath{clip}%
\pgfsetbuttcap%
\pgfsetmiterjoin%
\pgfsetlinewidth{0.501875pt}%
\definecolor{currentstroke}{rgb}{1.000000,0.000000,0.000000}%
\pgfsetstrokecolor{currentstroke}%
\pgfsetdash{}{0pt}%
\pgfpathmoveto{\pgfqpoint{0.586370in}{0.449983in}}%
\pgfpathlineto{\pgfqpoint{0.586370in}{0.675127in}}%
\pgfpathlineto{\pgfqpoint{0.625210in}{0.675127in}}%
\pgfpathlineto{\pgfqpoint{0.625210in}{0.680580in}}%
\pgfpathlineto{\pgfqpoint{0.664051in}{0.680580in}}%
\pgfpathlineto{\pgfqpoint{0.664051in}{0.684482in}}%
\pgfpathlineto{\pgfqpoint{0.702891in}{0.684482in}}%
\pgfpathlineto{\pgfqpoint{0.702891in}{0.688760in}}%
\pgfpathlineto{\pgfqpoint{0.741732in}{0.688760in}}%
\pgfpathlineto{\pgfqpoint{0.741732in}{0.687350in}}%
\pgfpathlineto{\pgfqpoint{0.780572in}{0.687350in}}%
\pgfpathlineto{\pgfqpoint{0.780572in}{0.694308in}}%
\pgfpathlineto{\pgfqpoint{0.819413in}{0.694308in}}%
\pgfpathlineto{\pgfqpoint{0.819413in}{0.708318in}}%
\pgfpathlineto{\pgfqpoint{0.858253in}{0.708318in}}%
\pgfpathlineto{\pgfqpoint{0.858253in}{0.715040in}}%
\pgfpathlineto{\pgfqpoint{0.897094in}{0.715040in}}%
\pgfpathlineto{\pgfqpoint{0.897094in}{0.734127in}}%
\pgfpathlineto{\pgfqpoint{0.935935in}{0.734127in}}%
\pgfpathlineto{\pgfqpoint{0.935935in}{0.746492in}}%
\pgfpathlineto{\pgfqpoint{0.974775in}{0.746492in}}%
\pgfpathlineto{\pgfqpoint{0.974775in}{0.777379in}}%
\pgfpathlineto{\pgfqpoint{1.013616in}{0.777379in}}%
\pgfpathlineto{\pgfqpoint{1.013616in}{0.793974in}}%
\pgfpathlineto{\pgfqpoint{1.052456in}{0.793974in}}%
\pgfpathlineto{\pgfqpoint{1.052456in}{0.821289in}}%
\pgfpathlineto{\pgfqpoint{1.091297in}{0.821289in}}%
\pgfpathlineto{\pgfqpoint{1.091297in}{0.863224in}}%
\pgfpathlineto{\pgfqpoint{1.130137in}{0.863224in}}%
\pgfpathlineto{\pgfqpoint{1.130137in}{0.899236in}}%
\pgfpathlineto{\pgfqpoint{1.168978in}{0.899236in}}%
\pgfpathlineto{\pgfqpoint{1.168978in}{0.939807in}}%
\pgfpathlineto{\pgfqpoint{1.207818in}{0.939807in}}%
\pgfpathlineto{\pgfqpoint{1.207818in}{0.981131in}}%
\pgfpathlineto{\pgfqpoint{1.246659in}{0.981131in}}%
\pgfpathlineto{\pgfqpoint{1.246659in}{1.037735in}}%
\pgfpathlineto{\pgfqpoint{1.285500in}{1.037735in}}%
\pgfpathlineto{\pgfqpoint{1.285500in}{1.105668in}}%
\pgfpathlineto{\pgfqpoint{1.324340in}{1.105668in}}%
\pgfpathlineto{\pgfqpoint{1.324340in}{1.219908in}}%
\pgfpathlineto{\pgfqpoint{1.363181in}{1.219908in}}%
\pgfpathlineto{\pgfqpoint{1.363181in}{1.321455in}}%
\pgfpathlineto{\pgfqpoint{1.402021in}{1.321455in}}%
\pgfpathlineto{\pgfqpoint{1.402021in}{1.195791in}}%
\pgfpathlineto{\pgfqpoint{1.440862in}{1.195791in}}%
\pgfpathlineto{\pgfqpoint{1.440862in}{1.225033in}}%
\pgfpathlineto{\pgfqpoint{1.479702in}{1.225033in}}%
\pgfpathlineto{\pgfqpoint{1.479702in}{1.261279in}}%
\pgfpathlineto{\pgfqpoint{1.518543in}{1.261279in}}%
\pgfpathlineto{\pgfqpoint{1.518543in}{1.269789in}}%
\pgfpathlineto{\pgfqpoint{1.557383in}{1.269789in}}%
\pgfpathlineto{\pgfqpoint{1.557383in}{1.266263in}}%
\pgfpathlineto{\pgfqpoint{1.596224in}{1.266263in}}%
\pgfpathlineto{\pgfqpoint{1.596224in}{1.273409in}}%
\pgfpathlineto{\pgfqpoint{1.635065in}{1.273409in}}%
\pgfpathlineto{\pgfqpoint{1.635065in}{1.277264in}}%
\pgfpathlineto{\pgfqpoint{1.673905in}{1.277264in}}%
\pgfpathlineto{\pgfqpoint{1.673905in}{1.258882in}}%
\pgfpathlineto{\pgfqpoint{1.712746in}{1.258882in}}%
\pgfpathlineto{\pgfqpoint{1.712746in}{1.226067in}}%
\pgfpathlineto{\pgfqpoint{1.751586in}{1.226067in}}%
\pgfpathlineto{\pgfqpoint{1.751586in}{1.160531in}}%
\pgfpathlineto{\pgfqpoint{1.790427in}{1.160531in}}%
\pgfpathlineto{\pgfqpoint{1.790427in}{1.061147in}}%
\pgfpathlineto{\pgfqpoint{1.829267in}{1.061147in}}%
\pgfpathlineto{\pgfqpoint{1.829267in}{0.970554in}}%
\pgfpathlineto{\pgfqpoint{1.868108in}{0.970554in}}%
\pgfpathlineto{\pgfqpoint{1.868108in}{0.893077in}}%
\pgfpathlineto{\pgfqpoint{1.906949in}{0.893077in}}%
\pgfpathlineto{\pgfqpoint{1.906949in}{0.850766in}}%
\pgfpathlineto{\pgfqpoint{1.945789in}{0.850766in}}%
\pgfpathlineto{\pgfqpoint{1.945789in}{0.823827in}}%
\pgfpathlineto{\pgfqpoint{1.984630in}{0.823827in}}%
\pgfpathlineto{\pgfqpoint{1.984630in}{0.800838in}}%
\pgfpathlineto{\pgfqpoint{2.023470in}{0.800838in}}%
\pgfpathlineto{\pgfqpoint{2.023470in}{0.790872in}}%
\pgfpathlineto{\pgfqpoint{2.062311in}{0.790872in}}%
\pgfpathlineto{\pgfqpoint{2.062311in}{0.765814in}}%
\pgfpathlineto{\pgfqpoint{2.101151in}{0.765814in}}%
\pgfpathlineto{\pgfqpoint{2.101151in}{0.748043in}}%
\pgfpathlineto{\pgfqpoint{2.139992in}{0.748043in}}%
\pgfpathlineto{\pgfqpoint{2.139992in}{0.693462in}}%
\pgfpathlineto{\pgfqpoint{2.178832in}{0.693462in}}%
\pgfpathlineto{\pgfqpoint{2.178832in}{0.632345in}}%
\pgfpathlineto{\pgfqpoint{2.217673in}{0.632345in}}%
\pgfpathlineto{\pgfqpoint{2.217673in}{0.585662in}}%
\pgfpathlineto{\pgfqpoint{2.256514in}{0.585662in}}%
\pgfpathlineto{\pgfqpoint{2.256514in}{0.534042in}}%
\pgfpathlineto{\pgfqpoint{2.295354in}{0.534042in}}%
\pgfpathlineto{\pgfqpoint{2.295354in}{0.501227in}}%
\pgfpathlineto{\pgfqpoint{2.334195in}{0.501227in}}%
\pgfpathlineto{\pgfqpoint{2.334195in}{0.479319in}}%
\pgfpathlineto{\pgfqpoint{2.373035in}{0.479319in}}%
\pgfpathlineto{\pgfqpoint{2.373035in}{0.467566in}}%
\pgfpathlineto{\pgfqpoint{2.411876in}{0.467566in}}%
\pgfpathlineto{\pgfqpoint{2.411876in}{0.457741in}}%
\pgfpathlineto{\pgfqpoint{2.450716in}{0.457741in}}%
\pgfpathlineto{\pgfqpoint{2.450716in}{0.453133in}}%
\pgfpathlineto{\pgfqpoint{2.489557in}{0.453133in}}%
\pgfpathlineto{\pgfqpoint{2.489557in}{0.451112in}}%
\pgfpathlineto{\pgfqpoint{2.528397in}{0.451112in}}%
\pgfpathlineto{\pgfqpoint{2.528397in}{0.449983in}}%
\pgfusepath{stroke}%
\end{pgfscope}%
\begin{pgfscope}%
\pgfsetrectcap%
\pgfsetmiterjoin%
\pgfsetlinewidth{1.003750pt}%
\definecolor{currentstroke}{rgb}{0.000000,0.000000,0.000000}%
\pgfsetstrokecolor{currentstroke}%
\pgfsetdash{}{0pt}%
\pgfpathmoveto{\pgfqpoint{0.366840in}{1.619432in}}%
\pgfpathlineto{\pgfqpoint{2.561650in}{1.619432in}}%
\pgfusepath{stroke}%
\end{pgfscope}%
\begin{pgfscope}%
\pgfsetrectcap%
\pgfsetmiterjoin%
\pgfsetlinewidth{1.003750pt}%
\definecolor{currentstroke}{rgb}{0.000000,0.000000,0.000000}%
\pgfsetstrokecolor{currentstroke}%
\pgfsetdash{}{0pt}%
\pgfpathmoveto{\pgfqpoint{2.561650in}{0.449983in}}%
\pgfpathlineto{\pgfqpoint{2.561650in}{1.619432in}}%
\pgfusepath{stroke}%
\end{pgfscope}%
\begin{pgfscope}%
\pgfsetrectcap%
\pgfsetmiterjoin%
\pgfsetlinewidth{1.003750pt}%
\definecolor{currentstroke}{rgb}{0.000000,0.000000,0.000000}%
\pgfsetstrokecolor{currentstroke}%
\pgfsetdash{}{0pt}%
\pgfpathmoveto{\pgfqpoint{0.366840in}{0.449983in}}%
\pgfpathlineto{\pgfqpoint{2.561650in}{0.449983in}}%
\pgfusepath{stroke}%
\end{pgfscope}%
\begin{pgfscope}%
\pgfsetrectcap%
\pgfsetmiterjoin%
\pgfsetlinewidth{1.003750pt}%
\definecolor{currentstroke}{rgb}{0.000000,0.000000,0.000000}%
\pgfsetstrokecolor{currentstroke}%
\pgfsetdash{}{0pt}%
\pgfpathmoveto{\pgfqpoint{0.366840in}{0.449983in}}%
\pgfpathlineto{\pgfqpoint{0.366840in}{1.619432in}}%
\pgfusepath{stroke}%
\end{pgfscope}%
\begin{pgfscope}%
\pgfsetbuttcap%
\pgfsetroundjoin%
\definecolor{currentfill}{rgb}{0.000000,0.000000,0.000000}%
\pgfsetfillcolor{currentfill}%
\pgfsetlinewidth{0.501875pt}%
\definecolor{currentstroke}{rgb}{0.000000,0.000000,0.000000}%
\pgfsetstrokecolor{currentstroke}%
\pgfsetdash{}{0pt}%
\pgfsys@defobject{currentmarker}{\pgfqpoint{0.000000in}{0.000000in}}{\pgfqpoint{0.000000in}{0.069444in}}{%
\pgfpathmoveto{\pgfqpoint{0.000000in}{0.000000in}}%
\pgfpathlineto{\pgfqpoint{0.000000in}{0.069444in}}%
\pgfusepath{stroke,fill}%
}%
\begin{pgfscope}%
\pgfsys@transformshift{0.366840in}{0.449983in}%
\pgfsys@useobject{currentmarker}{}%
\end{pgfscope}%
\end{pgfscope}%
\begin{pgfscope}%
\pgfsetbuttcap%
\pgfsetroundjoin%
\definecolor{currentfill}{rgb}{0.000000,0.000000,0.000000}%
\pgfsetfillcolor{currentfill}%
\pgfsetlinewidth{0.501875pt}%
\definecolor{currentstroke}{rgb}{0.000000,0.000000,0.000000}%
\pgfsetstrokecolor{currentstroke}%
\pgfsetdash{}{0pt}%
\pgfsys@defobject{currentmarker}{\pgfqpoint{0.000000in}{-0.069444in}}{\pgfqpoint{0.000000in}{0.000000in}}{%
\pgfpathmoveto{\pgfqpoint{0.000000in}{0.000000in}}%
\pgfpathlineto{\pgfqpoint{0.000000in}{-0.069444in}}%
\pgfusepath{stroke,fill}%
}%
\begin{pgfscope}%
\pgfsys@transformshift{0.366840in}{1.619432in}%
\pgfsys@useobject{currentmarker}{}%
\end{pgfscope}%
\end{pgfscope}%
\begin{pgfscope}%
\pgftext[x=0.366840in,y=0.380539in,,top]{\rmfamily\fontsize{8.000000}{9.600000}\selectfont −5}%
\end{pgfscope}%
\begin{pgfscope}%
\pgfsetbuttcap%
\pgfsetroundjoin%
\definecolor{currentfill}{rgb}{0.000000,0.000000,0.000000}%
\pgfsetfillcolor{currentfill}%
\pgfsetlinewidth{0.501875pt}%
\definecolor{currentstroke}{rgb}{0.000000,0.000000,0.000000}%
\pgfsetstrokecolor{currentstroke}%
\pgfsetdash{}{0pt}%
\pgfsys@defobject{currentmarker}{\pgfqpoint{0.000000in}{0.000000in}}{\pgfqpoint{0.000000in}{0.069444in}}{%
\pgfpathmoveto{\pgfqpoint{0.000000in}{0.000000in}}%
\pgfpathlineto{\pgfqpoint{0.000000in}{0.069444in}}%
\pgfusepath{stroke,fill}%
}%
\begin{pgfscope}%
\pgfsys@transformshift{0.915543in}{0.449983in}%
\pgfsys@useobject{currentmarker}{}%
\end{pgfscope}%
\end{pgfscope}%
\begin{pgfscope}%
\pgfsetbuttcap%
\pgfsetroundjoin%
\definecolor{currentfill}{rgb}{0.000000,0.000000,0.000000}%
\pgfsetfillcolor{currentfill}%
\pgfsetlinewidth{0.501875pt}%
\definecolor{currentstroke}{rgb}{0.000000,0.000000,0.000000}%
\pgfsetstrokecolor{currentstroke}%
\pgfsetdash{}{0pt}%
\pgfsys@defobject{currentmarker}{\pgfqpoint{0.000000in}{-0.069444in}}{\pgfqpoint{0.000000in}{0.000000in}}{%
\pgfpathmoveto{\pgfqpoint{0.000000in}{0.000000in}}%
\pgfpathlineto{\pgfqpoint{0.000000in}{-0.069444in}}%
\pgfusepath{stroke,fill}%
}%
\begin{pgfscope}%
\pgfsys@transformshift{0.915543in}{1.619432in}%
\pgfsys@useobject{currentmarker}{}%
\end{pgfscope}%
\end{pgfscope}%
\begin{pgfscope}%
\pgftext[x=0.915543in,y=0.380539in,,top]{\rmfamily\fontsize{8.000000}{9.600000}\selectfont 0}%
\end{pgfscope}%
\begin{pgfscope}%
\pgfsetbuttcap%
\pgfsetroundjoin%
\definecolor{currentfill}{rgb}{0.000000,0.000000,0.000000}%
\pgfsetfillcolor{currentfill}%
\pgfsetlinewidth{0.501875pt}%
\definecolor{currentstroke}{rgb}{0.000000,0.000000,0.000000}%
\pgfsetstrokecolor{currentstroke}%
\pgfsetdash{}{0pt}%
\pgfsys@defobject{currentmarker}{\pgfqpoint{0.000000in}{0.000000in}}{\pgfqpoint{0.000000in}{0.069444in}}{%
\pgfpathmoveto{\pgfqpoint{0.000000in}{0.000000in}}%
\pgfpathlineto{\pgfqpoint{0.000000in}{0.069444in}}%
\pgfusepath{stroke,fill}%
}%
\begin{pgfscope}%
\pgfsys@transformshift{1.464245in}{0.449983in}%
\pgfsys@useobject{currentmarker}{}%
\end{pgfscope}%
\end{pgfscope}%
\begin{pgfscope}%
\pgfsetbuttcap%
\pgfsetroundjoin%
\definecolor{currentfill}{rgb}{0.000000,0.000000,0.000000}%
\pgfsetfillcolor{currentfill}%
\pgfsetlinewidth{0.501875pt}%
\definecolor{currentstroke}{rgb}{0.000000,0.000000,0.000000}%
\pgfsetstrokecolor{currentstroke}%
\pgfsetdash{}{0pt}%
\pgfsys@defobject{currentmarker}{\pgfqpoint{0.000000in}{-0.069444in}}{\pgfqpoint{0.000000in}{0.000000in}}{%
\pgfpathmoveto{\pgfqpoint{0.000000in}{0.000000in}}%
\pgfpathlineto{\pgfqpoint{0.000000in}{-0.069444in}}%
\pgfusepath{stroke,fill}%
}%
\begin{pgfscope}%
\pgfsys@transformshift{1.464245in}{1.619432in}%
\pgfsys@useobject{currentmarker}{}%
\end{pgfscope}%
\end{pgfscope}%
\begin{pgfscope}%
\pgftext[x=1.464245in,y=0.380539in,,top]{\rmfamily\fontsize{8.000000}{9.600000}\selectfont 5}%
\end{pgfscope}%
\begin{pgfscope}%
\pgfsetbuttcap%
\pgfsetroundjoin%
\definecolor{currentfill}{rgb}{0.000000,0.000000,0.000000}%
\pgfsetfillcolor{currentfill}%
\pgfsetlinewidth{0.501875pt}%
\definecolor{currentstroke}{rgb}{0.000000,0.000000,0.000000}%
\pgfsetstrokecolor{currentstroke}%
\pgfsetdash{}{0pt}%
\pgfsys@defobject{currentmarker}{\pgfqpoint{0.000000in}{0.000000in}}{\pgfqpoint{0.000000in}{0.069444in}}{%
\pgfpathmoveto{\pgfqpoint{0.000000in}{0.000000in}}%
\pgfpathlineto{\pgfqpoint{0.000000in}{0.069444in}}%
\pgfusepath{stroke,fill}%
}%
\begin{pgfscope}%
\pgfsys@transformshift{2.012947in}{0.449983in}%
\pgfsys@useobject{currentmarker}{}%
\end{pgfscope}%
\end{pgfscope}%
\begin{pgfscope}%
\pgfsetbuttcap%
\pgfsetroundjoin%
\definecolor{currentfill}{rgb}{0.000000,0.000000,0.000000}%
\pgfsetfillcolor{currentfill}%
\pgfsetlinewidth{0.501875pt}%
\definecolor{currentstroke}{rgb}{0.000000,0.000000,0.000000}%
\pgfsetstrokecolor{currentstroke}%
\pgfsetdash{}{0pt}%
\pgfsys@defobject{currentmarker}{\pgfqpoint{0.000000in}{-0.069444in}}{\pgfqpoint{0.000000in}{0.000000in}}{%
\pgfpathmoveto{\pgfqpoint{0.000000in}{0.000000in}}%
\pgfpathlineto{\pgfqpoint{0.000000in}{-0.069444in}}%
\pgfusepath{stroke,fill}%
}%
\begin{pgfscope}%
\pgfsys@transformshift{2.012947in}{1.619432in}%
\pgfsys@useobject{currentmarker}{}%
\end{pgfscope}%
\end{pgfscope}%
\begin{pgfscope}%
\pgftext[x=2.012947in,y=0.380539in,,top]{\rmfamily\fontsize{8.000000}{9.600000}\selectfont 10}%
\end{pgfscope}%
\begin{pgfscope}%
\pgfsetbuttcap%
\pgfsetroundjoin%
\definecolor{currentfill}{rgb}{0.000000,0.000000,0.000000}%
\pgfsetfillcolor{currentfill}%
\pgfsetlinewidth{0.501875pt}%
\definecolor{currentstroke}{rgb}{0.000000,0.000000,0.000000}%
\pgfsetstrokecolor{currentstroke}%
\pgfsetdash{}{0pt}%
\pgfsys@defobject{currentmarker}{\pgfqpoint{0.000000in}{0.000000in}}{\pgfqpoint{0.000000in}{0.069444in}}{%
\pgfpathmoveto{\pgfqpoint{0.000000in}{0.000000in}}%
\pgfpathlineto{\pgfqpoint{0.000000in}{0.069444in}}%
\pgfusepath{stroke,fill}%
}%
\begin{pgfscope}%
\pgfsys@transformshift{2.561650in}{0.449983in}%
\pgfsys@useobject{currentmarker}{}%
\end{pgfscope}%
\end{pgfscope}%
\begin{pgfscope}%
\pgfsetbuttcap%
\pgfsetroundjoin%
\definecolor{currentfill}{rgb}{0.000000,0.000000,0.000000}%
\pgfsetfillcolor{currentfill}%
\pgfsetlinewidth{0.501875pt}%
\definecolor{currentstroke}{rgb}{0.000000,0.000000,0.000000}%
\pgfsetstrokecolor{currentstroke}%
\pgfsetdash{}{0pt}%
\pgfsys@defobject{currentmarker}{\pgfqpoint{0.000000in}{-0.069444in}}{\pgfqpoint{0.000000in}{0.000000in}}{%
\pgfpathmoveto{\pgfqpoint{0.000000in}{0.000000in}}%
\pgfpathlineto{\pgfqpoint{0.000000in}{-0.069444in}}%
\pgfusepath{stroke,fill}%
}%
\begin{pgfscope}%
\pgfsys@transformshift{2.561650in}{1.619432in}%
\pgfsys@useobject{currentmarker}{}%
\end{pgfscope}%
\end{pgfscope}%
\begin{pgfscope}%
\pgftext[x=2.561650in,y=0.380539in,,top]{\rmfamily\fontsize{8.000000}{9.600000}\selectfont 15}%
\end{pgfscope}%
\begin{pgfscope}%
\pgftext[x=1.464245in,y=0.203564in,,top]{\rmfamily\fontsize{9.000000}{10.800000}\selectfont \(\displaystyle \mathrm{DLL}_{\mu/\pi}(\mu^-)\)}%
\end{pgfscope}%
\begin{pgfscope}%
\pgfsetbuttcap%
\pgfsetroundjoin%
\definecolor{currentfill}{rgb}{0.000000,0.000000,0.000000}%
\pgfsetfillcolor{currentfill}%
\pgfsetlinewidth{0.501875pt}%
\definecolor{currentstroke}{rgb}{0.000000,0.000000,0.000000}%
\pgfsetstrokecolor{currentstroke}%
\pgfsetdash{}{0pt}%
\pgfsys@defobject{currentmarker}{\pgfqpoint{0.000000in}{0.000000in}}{\pgfqpoint{0.069444in}{0.000000in}}{%
\pgfpathmoveto{\pgfqpoint{0.000000in}{0.000000in}}%
\pgfpathlineto{\pgfqpoint{0.069444in}{0.000000in}}%
\pgfusepath{stroke,fill}%
}%
\begin{pgfscope}%
\pgfsys@transformshift{0.366840in}{0.449983in}%
\pgfsys@useobject{currentmarker}{}%
\end{pgfscope}%
\end{pgfscope}%
\begin{pgfscope}%
\pgfsetbuttcap%
\pgfsetroundjoin%
\definecolor{currentfill}{rgb}{0.000000,0.000000,0.000000}%
\pgfsetfillcolor{currentfill}%
\pgfsetlinewidth{0.501875pt}%
\definecolor{currentstroke}{rgb}{0.000000,0.000000,0.000000}%
\pgfsetstrokecolor{currentstroke}%
\pgfsetdash{}{0pt}%
\pgfsys@defobject{currentmarker}{\pgfqpoint{-0.069444in}{0.000000in}}{\pgfqpoint{0.000000in}{0.000000in}}{%
\pgfpathmoveto{\pgfqpoint{0.000000in}{0.000000in}}%
\pgfpathlineto{\pgfqpoint{-0.069444in}{0.000000in}}%
\pgfusepath{stroke,fill}%
}%
\begin{pgfscope}%
\pgfsys@transformshift{2.561650in}{0.449983in}%
\pgfsys@useobject{currentmarker}{}%
\end{pgfscope}%
\end{pgfscope}%
\begin{pgfscope}%
\pgftext[x=0.297396in,y=0.449983in,right,]{\rmfamily\fontsize{8.000000}{9.600000}\selectfont 0.00}%
\end{pgfscope}%
\begin{pgfscope}%
\pgfsetbuttcap%
\pgfsetroundjoin%
\definecolor{currentfill}{rgb}{0.000000,0.000000,0.000000}%
\pgfsetfillcolor{currentfill}%
\pgfsetlinewidth{0.501875pt}%
\definecolor{currentstroke}{rgb}{0.000000,0.000000,0.000000}%
\pgfsetstrokecolor{currentstroke}%
\pgfsetdash{}{0pt}%
\pgfsys@defobject{currentmarker}{\pgfqpoint{0.000000in}{0.000000in}}{\pgfqpoint{0.069444in}{0.000000in}}{%
\pgfpathmoveto{\pgfqpoint{0.000000in}{0.000000in}}%
\pgfpathlineto{\pgfqpoint{0.069444in}{0.000000in}}%
\pgfusepath{stroke,fill}%
}%
\begin{pgfscope}%
\pgfsys@transformshift{0.366840in}{0.596165in}%
\pgfsys@useobject{currentmarker}{}%
\end{pgfscope}%
\end{pgfscope}%
\begin{pgfscope}%
\pgfsetbuttcap%
\pgfsetroundjoin%
\definecolor{currentfill}{rgb}{0.000000,0.000000,0.000000}%
\pgfsetfillcolor{currentfill}%
\pgfsetlinewidth{0.501875pt}%
\definecolor{currentstroke}{rgb}{0.000000,0.000000,0.000000}%
\pgfsetstrokecolor{currentstroke}%
\pgfsetdash{}{0pt}%
\pgfsys@defobject{currentmarker}{\pgfqpoint{-0.069444in}{0.000000in}}{\pgfqpoint{0.000000in}{0.000000in}}{%
\pgfpathmoveto{\pgfqpoint{0.000000in}{0.000000in}}%
\pgfpathlineto{\pgfqpoint{-0.069444in}{0.000000in}}%
\pgfusepath{stroke,fill}%
}%
\begin{pgfscope}%
\pgfsys@transformshift{2.561650in}{0.596165in}%
\pgfsys@useobject{currentmarker}{}%
\end{pgfscope}%
\end{pgfscope}%
\begin{pgfscope}%
\pgftext[x=0.297396in,y=0.596165in,right,]{\rmfamily\fontsize{8.000000}{9.600000}\selectfont 0.02}%
\end{pgfscope}%
\begin{pgfscope}%
\pgfsetbuttcap%
\pgfsetroundjoin%
\definecolor{currentfill}{rgb}{0.000000,0.000000,0.000000}%
\pgfsetfillcolor{currentfill}%
\pgfsetlinewidth{0.501875pt}%
\definecolor{currentstroke}{rgb}{0.000000,0.000000,0.000000}%
\pgfsetstrokecolor{currentstroke}%
\pgfsetdash{}{0pt}%
\pgfsys@defobject{currentmarker}{\pgfqpoint{0.000000in}{0.000000in}}{\pgfqpoint{0.069444in}{0.000000in}}{%
\pgfpathmoveto{\pgfqpoint{0.000000in}{0.000000in}}%
\pgfpathlineto{\pgfqpoint{0.069444in}{0.000000in}}%
\pgfusepath{stroke,fill}%
}%
\begin{pgfscope}%
\pgfsys@transformshift{0.366840in}{0.742346in}%
\pgfsys@useobject{currentmarker}{}%
\end{pgfscope}%
\end{pgfscope}%
\begin{pgfscope}%
\pgfsetbuttcap%
\pgfsetroundjoin%
\definecolor{currentfill}{rgb}{0.000000,0.000000,0.000000}%
\pgfsetfillcolor{currentfill}%
\pgfsetlinewidth{0.501875pt}%
\definecolor{currentstroke}{rgb}{0.000000,0.000000,0.000000}%
\pgfsetstrokecolor{currentstroke}%
\pgfsetdash{}{0pt}%
\pgfsys@defobject{currentmarker}{\pgfqpoint{-0.069444in}{0.000000in}}{\pgfqpoint{0.000000in}{0.000000in}}{%
\pgfpathmoveto{\pgfqpoint{0.000000in}{0.000000in}}%
\pgfpathlineto{\pgfqpoint{-0.069444in}{0.000000in}}%
\pgfusepath{stroke,fill}%
}%
\begin{pgfscope}%
\pgfsys@transformshift{2.561650in}{0.742346in}%
\pgfsys@useobject{currentmarker}{}%
\end{pgfscope}%
\end{pgfscope}%
\begin{pgfscope}%
\pgftext[x=0.297396in,y=0.742346in,right,]{\rmfamily\fontsize{8.000000}{9.600000}\selectfont 0.04}%
\end{pgfscope}%
\begin{pgfscope}%
\pgfsetbuttcap%
\pgfsetroundjoin%
\definecolor{currentfill}{rgb}{0.000000,0.000000,0.000000}%
\pgfsetfillcolor{currentfill}%
\pgfsetlinewidth{0.501875pt}%
\definecolor{currentstroke}{rgb}{0.000000,0.000000,0.000000}%
\pgfsetstrokecolor{currentstroke}%
\pgfsetdash{}{0pt}%
\pgfsys@defobject{currentmarker}{\pgfqpoint{0.000000in}{0.000000in}}{\pgfqpoint{0.069444in}{0.000000in}}{%
\pgfpathmoveto{\pgfqpoint{0.000000in}{0.000000in}}%
\pgfpathlineto{\pgfqpoint{0.069444in}{0.000000in}}%
\pgfusepath{stroke,fill}%
}%
\begin{pgfscope}%
\pgfsys@transformshift{0.366840in}{0.888527in}%
\pgfsys@useobject{currentmarker}{}%
\end{pgfscope}%
\end{pgfscope}%
\begin{pgfscope}%
\pgfsetbuttcap%
\pgfsetroundjoin%
\definecolor{currentfill}{rgb}{0.000000,0.000000,0.000000}%
\pgfsetfillcolor{currentfill}%
\pgfsetlinewidth{0.501875pt}%
\definecolor{currentstroke}{rgb}{0.000000,0.000000,0.000000}%
\pgfsetstrokecolor{currentstroke}%
\pgfsetdash{}{0pt}%
\pgfsys@defobject{currentmarker}{\pgfqpoint{-0.069444in}{0.000000in}}{\pgfqpoint{0.000000in}{0.000000in}}{%
\pgfpathmoveto{\pgfqpoint{0.000000in}{0.000000in}}%
\pgfpathlineto{\pgfqpoint{-0.069444in}{0.000000in}}%
\pgfusepath{stroke,fill}%
}%
\begin{pgfscope}%
\pgfsys@transformshift{2.561650in}{0.888527in}%
\pgfsys@useobject{currentmarker}{}%
\end{pgfscope}%
\end{pgfscope}%
\begin{pgfscope}%
\pgftext[x=0.297396in,y=0.888527in,right,]{\rmfamily\fontsize{8.000000}{9.600000}\selectfont 0.06}%
\end{pgfscope}%
\begin{pgfscope}%
\pgfsetbuttcap%
\pgfsetroundjoin%
\definecolor{currentfill}{rgb}{0.000000,0.000000,0.000000}%
\pgfsetfillcolor{currentfill}%
\pgfsetlinewidth{0.501875pt}%
\definecolor{currentstroke}{rgb}{0.000000,0.000000,0.000000}%
\pgfsetstrokecolor{currentstroke}%
\pgfsetdash{}{0pt}%
\pgfsys@defobject{currentmarker}{\pgfqpoint{0.000000in}{0.000000in}}{\pgfqpoint{0.069444in}{0.000000in}}{%
\pgfpathmoveto{\pgfqpoint{0.000000in}{0.000000in}}%
\pgfpathlineto{\pgfqpoint{0.069444in}{0.000000in}}%
\pgfusepath{stroke,fill}%
}%
\begin{pgfscope}%
\pgfsys@transformshift{0.366840in}{1.034708in}%
\pgfsys@useobject{currentmarker}{}%
\end{pgfscope}%
\end{pgfscope}%
\begin{pgfscope}%
\pgfsetbuttcap%
\pgfsetroundjoin%
\definecolor{currentfill}{rgb}{0.000000,0.000000,0.000000}%
\pgfsetfillcolor{currentfill}%
\pgfsetlinewidth{0.501875pt}%
\definecolor{currentstroke}{rgb}{0.000000,0.000000,0.000000}%
\pgfsetstrokecolor{currentstroke}%
\pgfsetdash{}{0pt}%
\pgfsys@defobject{currentmarker}{\pgfqpoint{-0.069444in}{0.000000in}}{\pgfqpoint{0.000000in}{0.000000in}}{%
\pgfpathmoveto{\pgfqpoint{0.000000in}{0.000000in}}%
\pgfpathlineto{\pgfqpoint{-0.069444in}{0.000000in}}%
\pgfusepath{stroke,fill}%
}%
\begin{pgfscope}%
\pgfsys@transformshift{2.561650in}{1.034708in}%
\pgfsys@useobject{currentmarker}{}%
\end{pgfscope}%
\end{pgfscope}%
\begin{pgfscope}%
\pgftext[x=0.297396in,y=1.034708in,right,]{\rmfamily\fontsize{8.000000}{9.600000}\selectfont 0.08}%
\end{pgfscope}%
\begin{pgfscope}%
\pgfsetbuttcap%
\pgfsetroundjoin%
\definecolor{currentfill}{rgb}{0.000000,0.000000,0.000000}%
\pgfsetfillcolor{currentfill}%
\pgfsetlinewidth{0.501875pt}%
\definecolor{currentstroke}{rgb}{0.000000,0.000000,0.000000}%
\pgfsetstrokecolor{currentstroke}%
\pgfsetdash{}{0pt}%
\pgfsys@defobject{currentmarker}{\pgfqpoint{0.000000in}{0.000000in}}{\pgfqpoint{0.069444in}{0.000000in}}{%
\pgfpathmoveto{\pgfqpoint{0.000000in}{0.000000in}}%
\pgfpathlineto{\pgfqpoint{0.069444in}{0.000000in}}%
\pgfusepath{stroke,fill}%
}%
\begin{pgfscope}%
\pgfsys@transformshift{0.366840in}{1.180889in}%
\pgfsys@useobject{currentmarker}{}%
\end{pgfscope}%
\end{pgfscope}%
\begin{pgfscope}%
\pgfsetbuttcap%
\pgfsetroundjoin%
\definecolor{currentfill}{rgb}{0.000000,0.000000,0.000000}%
\pgfsetfillcolor{currentfill}%
\pgfsetlinewidth{0.501875pt}%
\definecolor{currentstroke}{rgb}{0.000000,0.000000,0.000000}%
\pgfsetstrokecolor{currentstroke}%
\pgfsetdash{}{0pt}%
\pgfsys@defobject{currentmarker}{\pgfqpoint{-0.069444in}{0.000000in}}{\pgfqpoint{0.000000in}{0.000000in}}{%
\pgfpathmoveto{\pgfqpoint{0.000000in}{0.000000in}}%
\pgfpathlineto{\pgfqpoint{-0.069444in}{0.000000in}}%
\pgfusepath{stroke,fill}%
}%
\begin{pgfscope}%
\pgfsys@transformshift{2.561650in}{1.180889in}%
\pgfsys@useobject{currentmarker}{}%
\end{pgfscope}%
\end{pgfscope}%
\begin{pgfscope}%
\pgftext[x=0.297396in,y=1.180889in,right,]{\rmfamily\fontsize{8.000000}{9.600000}\selectfont 0.10}%
\end{pgfscope}%
\begin{pgfscope}%
\pgfsetbuttcap%
\pgfsetroundjoin%
\definecolor{currentfill}{rgb}{0.000000,0.000000,0.000000}%
\pgfsetfillcolor{currentfill}%
\pgfsetlinewidth{0.501875pt}%
\definecolor{currentstroke}{rgb}{0.000000,0.000000,0.000000}%
\pgfsetstrokecolor{currentstroke}%
\pgfsetdash{}{0pt}%
\pgfsys@defobject{currentmarker}{\pgfqpoint{0.000000in}{0.000000in}}{\pgfqpoint{0.069444in}{0.000000in}}{%
\pgfpathmoveto{\pgfqpoint{0.000000in}{0.000000in}}%
\pgfpathlineto{\pgfqpoint{0.069444in}{0.000000in}}%
\pgfusepath{stroke,fill}%
}%
\begin{pgfscope}%
\pgfsys@transformshift{0.366840in}{1.327070in}%
\pgfsys@useobject{currentmarker}{}%
\end{pgfscope}%
\end{pgfscope}%
\begin{pgfscope}%
\pgfsetbuttcap%
\pgfsetroundjoin%
\definecolor{currentfill}{rgb}{0.000000,0.000000,0.000000}%
\pgfsetfillcolor{currentfill}%
\pgfsetlinewidth{0.501875pt}%
\definecolor{currentstroke}{rgb}{0.000000,0.000000,0.000000}%
\pgfsetstrokecolor{currentstroke}%
\pgfsetdash{}{0pt}%
\pgfsys@defobject{currentmarker}{\pgfqpoint{-0.069444in}{0.000000in}}{\pgfqpoint{0.000000in}{0.000000in}}{%
\pgfpathmoveto{\pgfqpoint{0.000000in}{0.000000in}}%
\pgfpathlineto{\pgfqpoint{-0.069444in}{0.000000in}}%
\pgfusepath{stroke,fill}%
}%
\begin{pgfscope}%
\pgfsys@transformshift{2.561650in}{1.327070in}%
\pgfsys@useobject{currentmarker}{}%
\end{pgfscope}%
\end{pgfscope}%
\begin{pgfscope}%
\pgftext[x=0.297396in,y=1.327070in,right,]{\rmfamily\fontsize{8.000000}{9.600000}\selectfont 0.12}%
\end{pgfscope}%
\begin{pgfscope}%
\pgfsetbuttcap%
\pgfsetroundjoin%
\definecolor{currentfill}{rgb}{0.000000,0.000000,0.000000}%
\pgfsetfillcolor{currentfill}%
\pgfsetlinewidth{0.501875pt}%
\definecolor{currentstroke}{rgb}{0.000000,0.000000,0.000000}%
\pgfsetstrokecolor{currentstroke}%
\pgfsetdash{}{0pt}%
\pgfsys@defobject{currentmarker}{\pgfqpoint{0.000000in}{0.000000in}}{\pgfqpoint{0.069444in}{0.000000in}}{%
\pgfpathmoveto{\pgfqpoint{0.000000in}{0.000000in}}%
\pgfpathlineto{\pgfqpoint{0.069444in}{0.000000in}}%
\pgfusepath{stroke,fill}%
}%
\begin{pgfscope}%
\pgfsys@transformshift{0.366840in}{1.473251in}%
\pgfsys@useobject{currentmarker}{}%
\end{pgfscope}%
\end{pgfscope}%
\begin{pgfscope}%
\pgfsetbuttcap%
\pgfsetroundjoin%
\definecolor{currentfill}{rgb}{0.000000,0.000000,0.000000}%
\pgfsetfillcolor{currentfill}%
\pgfsetlinewidth{0.501875pt}%
\definecolor{currentstroke}{rgb}{0.000000,0.000000,0.000000}%
\pgfsetstrokecolor{currentstroke}%
\pgfsetdash{}{0pt}%
\pgfsys@defobject{currentmarker}{\pgfqpoint{-0.069444in}{0.000000in}}{\pgfqpoint{0.000000in}{0.000000in}}{%
\pgfpathmoveto{\pgfqpoint{0.000000in}{0.000000in}}%
\pgfpathlineto{\pgfqpoint{-0.069444in}{0.000000in}}%
\pgfusepath{stroke,fill}%
}%
\begin{pgfscope}%
\pgfsys@transformshift{2.561650in}{1.473251in}%
\pgfsys@useobject{currentmarker}{}%
\end{pgfscope}%
\end{pgfscope}%
\begin{pgfscope}%
\pgftext[x=0.297396in,y=1.473251in,right,]{\rmfamily\fontsize{8.000000}{9.600000}\selectfont 0.14}%
\end{pgfscope}%
\begin{pgfscope}%
\pgfsetbuttcap%
\pgfsetroundjoin%
\definecolor{currentfill}{rgb}{0.000000,0.000000,0.000000}%
\pgfsetfillcolor{currentfill}%
\pgfsetlinewidth{0.501875pt}%
\definecolor{currentstroke}{rgb}{0.000000,0.000000,0.000000}%
\pgfsetstrokecolor{currentstroke}%
\pgfsetdash{}{0pt}%
\pgfsys@defobject{currentmarker}{\pgfqpoint{0.000000in}{0.000000in}}{\pgfqpoint{0.069444in}{0.000000in}}{%
\pgfpathmoveto{\pgfqpoint{0.000000in}{0.000000in}}%
\pgfpathlineto{\pgfqpoint{0.069444in}{0.000000in}}%
\pgfusepath{stroke,fill}%
}%
\begin{pgfscope}%
\pgfsys@transformshift{0.366840in}{1.619432in}%
\pgfsys@useobject{currentmarker}{}%
\end{pgfscope}%
\end{pgfscope}%
\begin{pgfscope}%
\pgfsetbuttcap%
\pgfsetroundjoin%
\definecolor{currentfill}{rgb}{0.000000,0.000000,0.000000}%
\pgfsetfillcolor{currentfill}%
\pgfsetlinewidth{0.501875pt}%
\definecolor{currentstroke}{rgb}{0.000000,0.000000,0.000000}%
\pgfsetstrokecolor{currentstroke}%
\pgfsetdash{}{0pt}%
\pgfsys@defobject{currentmarker}{\pgfqpoint{-0.069444in}{0.000000in}}{\pgfqpoint{0.000000in}{0.000000in}}{%
\pgfpathmoveto{\pgfqpoint{0.000000in}{0.000000in}}%
\pgfpathlineto{\pgfqpoint{-0.069444in}{0.000000in}}%
\pgfusepath{stroke,fill}%
}%
\begin{pgfscope}%
\pgfsys@transformshift{2.561650in}{1.619432in}%
\pgfsys@useobject{currentmarker}{}%
\end{pgfscope}%
\end{pgfscope}%
\begin{pgfscope}%
\pgftext[x=0.297396in,y=1.619432in,right,]{\rmfamily\fontsize{8.000000}{9.600000}\selectfont 0.16}%
\end{pgfscope}%
\end{pgfpicture}%
\makeatother%
\endgroup%

	\end{subfigure}
	\begin{subfigure}[t]{0.49\textwidth}
		\centering
    %\includegraphics[width=\textwidth]{store/variables/SIG_BKG_muplus_PIDmu.pdf}
    %% Creator: Matplotlib, PGF backend
%%
%% To include the figure in your LaTeX document, write
%%   \input{<filename>.pgf}
%%
%% Make sure the required packages are loaded in your preamble
%%   \usepackage{pgf}
%%
%% Figures using additional raster images can only be included by \input if
%% they are in the same directory as the main LaTeX file. For loading figures
%% from other directories you can use the `import` package
%%   \usepackage{import}
%% and then include the figures with
%%   \import{<path to file>}{<filename>.pgf}
%%
%% Matplotlib used the following preamble
%%   \usepackage{fontspec}
%%   \setmainfont{DejaVu Serif}
%%   \setsansfont{DejaVu Sans}
%%   \setmonofont{DejaVu Sans Mono}
%%
\begingroup%
\makeatletter%
\begin{pgfpicture}%
\pgfpathrectangle{\pgfpointorigin}{\pgfqpoint{2.682342in}{1.719349in}}%
\pgfusepath{use as bounding box, clip}%
\begin{pgfscope}%
\pgfsetbuttcap%
\pgfsetmiterjoin%
\definecolor{currentfill}{rgb}{1.000000,1.000000,1.000000}%
\pgfsetfillcolor{currentfill}%
\pgfsetlinewidth{0.000000pt}%
\definecolor{currentstroke}{rgb}{1.000000,1.000000,1.000000}%
\pgfsetstrokecolor{currentstroke}%
\pgfsetdash{}{0pt}%
\pgfpathmoveto{\pgfqpoint{0.000000in}{0.000000in}}%
\pgfpathlineto{\pgfqpoint{2.682342in}{0.000000in}}%
\pgfpathlineto{\pgfqpoint{2.682342in}{1.719349in}}%
\pgfpathlineto{\pgfqpoint{0.000000in}{1.719349in}}%
\pgfpathclose%
\pgfusepath{fill}%
\end{pgfscope}%
\begin{pgfscope}%
\pgfsetbuttcap%
\pgfsetmiterjoin%
\definecolor{currentfill}{rgb}{1.000000,1.000000,1.000000}%
\pgfsetfillcolor{currentfill}%
\pgfsetlinewidth{0.000000pt}%
\definecolor{currentstroke}{rgb}{0.000000,0.000000,0.000000}%
\pgfsetstrokecolor{currentstroke}%
\pgfsetstrokeopacity{0.000000}%
\pgfsetdash{}{0pt}%
\pgfpathmoveto{\pgfqpoint{0.366840in}{0.449983in}}%
\pgfpathlineto{\pgfqpoint{2.561650in}{0.449983in}}%
\pgfpathlineto{\pgfqpoint{2.561650in}{1.615583in}}%
\pgfpathlineto{\pgfqpoint{0.366840in}{1.615583in}}%
\pgfpathclose%
\pgfusepath{fill}%
\end{pgfscope}%
\begin{pgfscope}%
\pgfpathrectangle{\pgfqpoint{0.366840in}{0.449983in}}{\pgfqpoint{2.194810in}{1.165600in}} %
\pgfusepath{clip}%
\pgfsetbuttcap%
\pgfsetmiterjoin%
\definecolor{currentfill}{rgb}{0.215686,0.470588,0.749020}%
\pgfsetfillcolor{currentfill}%
\pgfsetlinewidth{0.000000pt}%
\definecolor{currentstroke}{rgb}{0.000000,0.000000,0.000000}%
\pgfsetstrokecolor{currentstroke}%
\pgfsetdash{}{0pt}%
\pgfpathmoveto{\pgfqpoint{0.586773in}{0.449983in}}%
\pgfpathlineto{\pgfqpoint{0.586773in}{0.467849in}}%
\pgfpathlineto{\pgfqpoint{0.625341in}{0.467849in}}%
\pgfpathlineto{\pgfqpoint{0.625341in}{0.483479in}}%
\pgfpathlineto{\pgfqpoint{0.663910in}{0.483479in}}%
\pgfpathlineto{\pgfqpoint{0.663910in}{0.488206in}}%
\pgfpathlineto{\pgfqpoint{0.702478in}{0.488206in}}%
\pgfpathlineto{\pgfqpoint{0.702478in}{0.495732in}}%
\pgfpathlineto{\pgfqpoint{0.741047in}{0.495732in}}%
\pgfpathlineto{\pgfqpoint{0.741047in}{0.515074in}}%
\pgfpathlineto{\pgfqpoint{0.779615in}{0.515074in}}%
\pgfpathlineto{\pgfqpoint{0.779615in}{0.508332in}}%
\pgfpathlineto{\pgfqpoint{0.818184in}{0.508332in}}%
\pgfpathlineto{\pgfqpoint{0.818184in}{0.533289in}}%
\pgfpathlineto{\pgfqpoint{0.856752in}{0.533289in}}%
\pgfpathlineto{\pgfqpoint{0.856752in}{0.533997in}}%
\pgfpathlineto{\pgfqpoint{0.895320in}{0.533997in}}%
\pgfpathlineto{\pgfqpoint{0.895320in}{0.550207in}}%
\pgfpathlineto{\pgfqpoint{0.933889in}{0.550207in}}%
\pgfpathlineto{\pgfqpoint{0.933889in}{0.581855in}}%
\pgfpathlineto{\pgfqpoint{0.972457in}{0.581855in}}%
\pgfpathlineto{\pgfqpoint{0.972457in}{0.604574in}}%
\pgfpathlineto{\pgfqpoint{1.011026in}{0.604574in}}%
\pgfpathlineto{\pgfqpoint{1.011026in}{0.638151in}}%
\pgfpathlineto{\pgfqpoint{1.049594in}{0.638151in}}%
\pgfpathlineto{\pgfqpoint{1.049594in}{0.672445in}}%
\pgfpathlineto{\pgfqpoint{1.088163in}{0.672445in}}%
\pgfpathlineto{\pgfqpoint{1.088163in}{0.672209in}}%
\pgfpathlineto{\pgfqpoint{1.126731in}{0.672209in}}%
\pgfpathlineto{\pgfqpoint{1.126731in}{0.742315in}}%
\pgfpathlineto{\pgfqpoint{1.165300in}{0.742315in}}%
\pgfpathlineto{\pgfqpoint{1.165300in}{0.751616in}}%
\pgfpathlineto{\pgfqpoint{1.203868in}{0.751616in}}%
\pgfpathlineto{\pgfqpoint{1.203868in}{0.882737in}}%
\pgfpathlineto{\pgfqpoint{1.242436in}{0.882737in}}%
\pgfpathlineto{\pgfqpoint{1.242436in}{0.945471in}}%
\pgfpathlineto{\pgfqpoint{1.281005in}{0.945471in}}%
\pgfpathlineto{\pgfqpoint{1.281005in}{1.027583in}}%
\pgfpathlineto{\pgfqpoint{1.319573in}{1.027583in}}%
\pgfpathlineto{\pgfqpoint{1.319573in}{1.145399in}}%
\pgfpathlineto{\pgfqpoint{1.358142in}{1.145399in}}%
\pgfpathlineto{\pgfqpoint{1.358142in}{1.233604in}}%
\pgfpathlineto{\pgfqpoint{1.396710in}{1.233604in}}%
\pgfpathlineto{\pgfqpoint{1.396710in}{1.244321in}}%
\pgfpathlineto{\pgfqpoint{1.435279in}{1.244321in}}%
\pgfpathlineto{\pgfqpoint{1.435279in}{1.237121in}}%
\pgfpathlineto{\pgfqpoint{1.473847in}{1.237121in}}%
\pgfpathlineto{\pgfqpoint{1.473847in}{1.296039in}}%
\pgfpathlineto{\pgfqpoint{1.512415in}{1.296039in}}%
\pgfpathlineto{\pgfqpoint{1.512415in}{1.357505in}}%
\pgfpathlineto{\pgfqpoint{1.550984in}{1.357505in}}%
\pgfpathlineto{\pgfqpoint{1.550984in}{1.371317in}}%
\pgfpathlineto{\pgfqpoint{1.589552in}{1.371317in}}%
\pgfpathlineto{\pgfqpoint{1.589552in}{1.470079in}}%
\pgfpathlineto{\pgfqpoint{1.628121in}{1.470079in}}%
\pgfpathlineto{\pgfqpoint{1.628121in}{1.527030in}}%
\pgfpathlineto{\pgfqpoint{1.666689in}{1.527030in}}%
\pgfpathlineto{\pgfqpoint{1.666689in}{1.449413in}}%
\pgfpathlineto{\pgfqpoint{1.705258in}{1.449413in}}%
\pgfpathlineto{\pgfqpoint{1.705258in}{1.497875in}}%
\pgfpathlineto{\pgfqpoint{1.743826in}{1.497875in}}%
\pgfpathlineto{\pgfqpoint{1.743826in}{1.430967in}}%
\pgfpathlineto{\pgfqpoint{1.782395in}{1.430967in}}%
\pgfpathlineto{\pgfqpoint{1.782395in}{1.210238in}}%
\pgfpathlineto{\pgfqpoint{1.820963in}{1.210238in}}%
\pgfpathlineto{\pgfqpoint{1.820963in}{1.173100in}}%
\pgfpathlineto{\pgfqpoint{1.859531in}{1.173100in}}%
\pgfpathlineto{\pgfqpoint{1.859531in}{1.110428in}}%
\pgfpathlineto{\pgfqpoint{1.898100in}{1.110428in}}%
\pgfpathlineto{\pgfqpoint{1.898100in}{0.976481in}}%
\pgfpathlineto{\pgfqpoint{1.936668in}{0.976481in}}%
\pgfpathlineto{\pgfqpoint{1.936668in}{0.945554in}}%
\pgfpathlineto{\pgfqpoint{1.975237in}{0.945554in}}%
\pgfpathlineto{\pgfqpoint{1.975237in}{0.952508in}}%
\pgfpathlineto{\pgfqpoint{2.013805in}{0.952508in}}%
\pgfpathlineto{\pgfqpoint{2.013805in}{0.932366in}}%
\pgfpathlineto{\pgfqpoint{2.052374in}{0.932366in}}%
\pgfpathlineto{\pgfqpoint{2.052374in}{0.876080in}}%
\pgfpathlineto{\pgfqpoint{2.090942in}{0.876080in}}%
\pgfpathlineto{\pgfqpoint{2.090942in}{0.845702in}}%
\pgfpathlineto{\pgfqpoint{2.129510in}{0.845702in}}%
\pgfpathlineto{\pgfqpoint{2.129510in}{0.845629in}}%
\pgfpathlineto{\pgfqpoint{2.168079in}{0.845629in}}%
\pgfpathlineto{\pgfqpoint{2.168079in}{0.764910in}}%
\pgfpathlineto{\pgfqpoint{2.206647in}{0.764910in}}%
\pgfpathlineto{\pgfqpoint{2.206647in}{0.685622in}}%
\pgfpathlineto{\pgfqpoint{2.245216in}{0.685622in}}%
\pgfpathlineto{\pgfqpoint{2.245216in}{0.592452in}}%
\pgfpathlineto{\pgfqpoint{2.283784in}{0.592452in}}%
\pgfpathlineto{\pgfqpoint{2.283784in}{0.558194in}}%
\pgfpathlineto{\pgfqpoint{2.322353in}{0.558194in}}%
\pgfpathlineto{\pgfqpoint{2.322353in}{0.523237in}}%
\pgfpathlineto{\pgfqpoint{2.360921in}{0.523237in}}%
\pgfpathlineto{\pgfqpoint{2.360921in}{0.485321in}}%
\pgfpathlineto{\pgfqpoint{2.399490in}{0.485321in}}%
\pgfpathlineto{\pgfqpoint{2.399490in}{0.479680in}}%
\pgfpathlineto{\pgfqpoint{2.438058in}{0.479680in}}%
\pgfpathlineto{\pgfqpoint{2.438058in}{0.462325in}}%
\pgfpathlineto{\pgfqpoint{2.476626in}{0.462325in}}%
\pgfpathlineto{\pgfqpoint{2.476626in}{0.451885in}}%
\pgfpathlineto{\pgfqpoint{2.515195in}{0.451885in}}%
\pgfpathlineto{\pgfqpoint{2.515195in}{0.449983in}}%
\pgfpathlineto{\pgfqpoint{2.476626in}{0.449983in}}%
\pgfpathlineto{\pgfqpoint{2.476626in}{0.449983in}}%
\pgfpathlineto{\pgfqpoint{2.438058in}{0.449983in}}%
\pgfpathlineto{\pgfqpoint{2.438058in}{0.449983in}}%
\pgfpathlineto{\pgfqpoint{2.399490in}{0.449983in}}%
\pgfpathlineto{\pgfqpoint{2.399490in}{0.449983in}}%
\pgfpathlineto{\pgfqpoint{2.360921in}{0.449983in}}%
\pgfpathlineto{\pgfqpoint{2.360921in}{0.449983in}}%
\pgfpathlineto{\pgfqpoint{2.322353in}{0.449983in}}%
\pgfpathlineto{\pgfqpoint{2.322353in}{0.449983in}}%
\pgfpathlineto{\pgfqpoint{2.283784in}{0.449983in}}%
\pgfpathlineto{\pgfqpoint{2.283784in}{0.449983in}}%
\pgfpathlineto{\pgfqpoint{2.245216in}{0.449983in}}%
\pgfpathlineto{\pgfqpoint{2.245216in}{0.449983in}}%
\pgfpathlineto{\pgfqpoint{2.206647in}{0.449983in}}%
\pgfpathlineto{\pgfqpoint{2.206647in}{0.449983in}}%
\pgfpathlineto{\pgfqpoint{2.168079in}{0.449983in}}%
\pgfpathlineto{\pgfqpoint{2.168079in}{0.449983in}}%
\pgfpathlineto{\pgfqpoint{2.129510in}{0.449983in}}%
\pgfpathlineto{\pgfqpoint{2.129510in}{0.449983in}}%
\pgfpathlineto{\pgfqpoint{2.090942in}{0.449983in}}%
\pgfpathlineto{\pgfqpoint{2.090942in}{0.449983in}}%
\pgfpathlineto{\pgfqpoint{2.052374in}{0.449983in}}%
\pgfpathlineto{\pgfqpoint{2.052374in}{0.449983in}}%
\pgfpathlineto{\pgfqpoint{2.013805in}{0.449983in}}%
\pgfpathlineto{\pgfqpoint{2.013805in}{0.449983in}}%
\pgfpathlineto{\pgfqpoint{1.975237in}{0.449983in}}%
\pgfpathlineto{\pgfqpoint{1.975237in}{0.449983in}}%
\pgfpathlineto{\pgfqpoint{1.936668in}{0.449983in}}%
\pgfpathlineto{\pgfqpoint{1.936668in}{0.449983in}}%
\pgfpathlineto{\pgfqpoint{1.898100in}{0.449983in}}%
\pgfpathlineto{\pgfqpoint{1.898100in}{0.449983in}}%
\pgfpathlineto{\pgfqpoint{1.859531in}{0.449983in}}%
\pgfpathlineto{\pgfqpoint{1.859531in}{0.449983in}}%
\pgfpathlineto{\pgfqpoint{1.820963in}{0.449983in}}%
\pgfpathlineto{\pgfqpoint{1.820963in}{0.449983in}}%
\pgfpathlineto{\pgfqpoint{1.782395in}{0.449983in}}%
\pgfpathlineto{\pgfqpoint{1.782395in}{0.449983in}}%
\pgfpathlineto{\pgfqpoint{1.743826in}{0.449983in}}%
\pgfpathlineto{\pgfqpoint{1.743826in}{0.449983in}}%
\pgfpathlineto{\pgfqpoint{1.705258in}{0.449983in}}%
\pgfpathlineto{\pgfqpoint{1.705258in}{0.449983in}}%
\pgfpathlineto{\pgfqpoint{1.666689in}{0.449983in}}%
\pgfpathlineto{\pgfqpoint{1.666689in}{0.449983in}}%
\pgfpathlineto{\pgfqpoint{1.628121in}{0.449983in}}%
\pgfpathlineto{\pgfqpoint{1.628121in}{0.449983in}}%
\pgfpathlineto{\pgfqpoint{1.589552in}{0.449983in}}%
\pgfpathlineto{\pgfqpoint{1.589552in}{0.449983in}}%
\pgfpathlineto{\pgfqpoint{1.550984in}{0.449983in}}%
\pgfpathlineto{\pgfqpoint{1.550984in}{0.449983in}}%
\pgfpathlineto{\pgfqpoint{1.512415in}{0.449983in}}%
\pgfpathlineto{\pgfqpoint{1.512415in}{0.449983in}}%
\pgfpathlineto{\pgfqpoint{1.473847in}{0.449983in}}%
\pgfpathlineto{\pgfqpoint{1.473847in}{0.449983in}}%
\pgfpathlineto{\pgfqpoint{1.435279in}{0.449983in}}%
\pgfpathlineto{\pgfqpoint{1.435279in}{0.449983in}}%
\pgfpathlineto{\pgfqpoint{1.396710in}{0.449983in}}%
\pgfpathlineto{\pgfqpoint{1.396710in}{0.449983in}}%
\pgfpathlineto{\pgfqpoint{1.358142in}{0.449983in}}%
\pgfpathlineto{\pgfqpoint{1.358142in}{0.449983in}}%
\pgfpathlineto{\pgfqpoint{1.319573in}{0.449983in}}%
\pgfpathlineto{\pgfqpoint{1.319573in}{0.449983in}}%
\pgfpathlineto{\pgfqpoint{1.281005in}{0.449983in}}%
\pgfpathlineto{\pgfqpoint{1.281005in}{0.449983in}}%
\pgfpathlineto{\pgfqpoint{1.242436in}{0.449983in}}%
\pgfpathlineto{\pgfqpoint{1.242436in}{0.449983in}}%
\pgfpathlineto{\pgfqpoint{1.203868in}{0.449983in}}%
\pgfpathlineto{\pgfqpoint{1.203868in}{0.449983in}}%
\pgfpathlineto{\pgfqpoint{1.165300in}{0.449983in}}%
\pgfpathlineto{\pgfqpoint{1.165300in}{0.449983in}}%
\pgfpathlineto{\pgfqpoint{1.126731in}{0.449983in}}%
\pgfpathlineto{\pgfqpoint{1.126731in}{0.449983in}}%
\pgfpathlineto{\pgfqpoint{1.088163in}{0.449983in}}%
\pgfpathlineto{\pgfqpoint{1.088163in}{0.449983in}}%
\pgfpathlineto{\pgfqpoint{1.049594in}{0.449983in}}%
\pgfpathlineto{\pgfqpoint{1.049594in}{0.449983in}}%
\pgfpathlineto{\pgfqpoint{1.011026in}{0.449983in}}%
\pgfpathlineto{\pgfqpoint{1.011026in}{0.449983in}}%
\pgfpathlineto{\pgfqpoint{0.972457in}{0.449983in}}%
\pgfpathlineto{\pgfqpoint{0.972457in}{0.449983in}}%
\pgfpathlineto{\pgfqpoint{0.933889in}{0.449983in}}%
\pgfpathlineto{\pgfqpoint{0.933889in}{0.449983in}}%
\pgfpathlineto{\pgfqpoint{0.895320in}{0.449983in}}%
\pgfpathlineto{\pgfqpoint{0.895320in}{0.449983in}}%
\pgfpathlineto{\pgfqpoint{0.856752in}{0.449983in}}%
\pgfpathlineto{\pgfqpoint{0.856752in}{0.449983in}}%
\pgfpathlineto{\pgfqpoint{0.818184in}{0.449983in}}%
\pgfpathlineto{\pgfqpoint{0.818184in}{0.449983in}}%
\pgfpathlineto{\pgfqpoint{0.779615in}{0.449983in}}%
\pgfpathlineto{\pgfqpoint{0.779615in}{0.449983in}}%
\pgfpathlineto{\pgfqpoint{0.741047in}{0.449983in}}%
\pgfpathlineto{\pgfqpoint{0.741047in}{0.449983in}}%
\pgfpathlineto{\pgfqpoint{0.702478in}{0.449983in}}%
\pgfpathlineto{\pgfqpoint{0.702478in}{0.449983in}}%
\pgfpathlineto{\pgfqpoint{0.663910in}{0.449983in}}%
\pgfpathlineto{\pgfqpoint{0.663910in}{0.449983in}}%
\pgfpathlineto{\pgfqpoint{0.625341in}{0.449983in}}%
\pgfpathlineto{\pgfqpoint{0.625341in}{0.449983in}}%
\pgfpathlineto{\pgfqpoint{0.586773in}{0.449983in}}%
\pgfusepath{fill}%
\end{pgfscope}%
\begin{pgfscope}%
\pgfpathrectangle{\pgfqpoint{0.366840in}{0.449983in}}{\pgfqpoint{2.194810in}{1.165600in}} %
\pgfusepath{clip}%
\pgfsetbuttcap%
\pgfsetmiterjoin%
\pgfsetlinewidth{0.501875pt}%
\definecolor{currentstroke}{rgb}{1.000000,0.000000,0.000000}%
\pgfsetstrokecolor{currentstroke}%
\pgfsetdash{}{0pt}%
\pgfpathmoveto{\pgfqpoint{0.586773in}{0.449983in}}%
\pgfpathlineto{\pgfqpoint{0.586773in}{0.675146in}}%
\pgfpathlineto{\pgfqpoint{0.625341in}{0.675146in}}%
\pgfpathlineto{\pgfqpoint{0.625341in}{0.685906in}}%
\pgfpathlineto{\pgfqpoint{0.663910in}{0.685906in}}%
\pgfpathlineto{\pgfqpoint{0.663910in}{0.688124in}}%
\pgfpathlineto{\pgfqpoint{0.702478in}{0.688124in}}%
\pgfpathlineto{\pgfqpoint{0.702478in}{0.685056in}}%
\pgfpathlineto{\pgfqpoint{0.741047in}{0.685056in}}%
\pgfpathlineto{\pgfqpoint{0.741047in}{0.691522in}}%
\pgfpathlineto{\pgfqpoint{0.779615in}{0.691522in}}%
\pgfpathlineto{\pgfqpoint{0.779615in}{0.693174in}}%
\pgfpathlineto{\pgfqpoint{0.818184in}{0.693174in}}%
\pgfpathlineto{\pgfqpoint{0.818184in}{0.704548in}}%
\pgfpathlineto{\pgfqpoint{0.856752in}{0.704548in}}%
\pgfpathlineto{\pgfqpoint{0.856752in}{0.710777in}}%
\pgfpathlineto{\pgfqpoint{0.895320in}{0.710777in}}%
\pgfpathlineto{\pgfqpoint{0.895320in}{0.733902in}}%
\pgfpathlineto{\pgfqpoint{0.933889in}{0.733902in}}%
\pgfpathlineto{\pgfqpoint{0.933889in}{0.753865in}}%
\pgfpathlineto{\pgfqpoint{0.972457in}{0.753865in}}%
\pgfpathlineto{\pgfqpoint{0.972457in}{0.772696in}}%
\pgfpathlineto{\pgfqpoint{1.011026in}{0.772696in}}%
\pgfpathlineto{\pgfqpoint{1.011026in}{0.788789in}}%
\pgfpathlineto{\pgfqpoint{1.049594in}{0.788789in}}%
\pgfpathlineto{\pgfqpoint{1.049594in}{0.826025in}}%
\pgfpathlineto{\pgfqpoint{1.088163in}{0.826025in}}%
\pgfpathlineto{\pgfqpoint{1.088163in}{0.859296in}}%
\pgfpathlineto{\pgfqpoint{1.126731in}{0.859296in}}%
\pgfpathlineto{\pgfqpoint{1.126731in}{0.889548in}}%
\pgfpathlineto{\pgfqpoint{1.165300in}{0.889548in}}%
\pgfpathlineto{\pgfqpoint{1.165300in}{0.925510in}}%
\pgfpathlineto{\pgfqpoint{1.203868in}{0.925510in}}%
\pgfpathlineto{\pgfqpoint{1.203868in}{0.977706in}}%
\pgfpathlineto{\pgfqpoint{1.242436in}{0.977706in}}%
\pgfpathlineto{\pgfqpoint{1.242436in}{1.017679in}}%
\pgfpathlineto{\pgfqpoint{1.281005in}{1.017679in}}%
\pgfpathlineto{\pgfqpoint{1.281005in}{1.084978in}}%
\pgfpathlineto{\pgfqpoint{1.319573in}{1.084978in}}%
\pgfpathlineto{\pgfqpoint{1.319573in}{1.186870in}}%
\pgfpathlineto{\pgfqpoint{1.358142in}{1.186870in}}%
\pgfpathlineto{\pgfqpoint{1.358142in}{1.317172in}}%
\pgfpathlineto{\pgfqpoint{1.396710in}{1.317172in}}%
\pgfpathlineto{\pgfqpoint{1.396710in}{1.187294in}}%
\pgfpathlineto{\pgfqpoint{1.435279in}{1.187294in}}%
\pgfpathlineto{\pgfqpoint{1.435279in}{1.211269in}}%
\pgfpathlineto{\pgfqpoint{1.473847in}{1.211269in}}%
\pgfpathlineto{\pgfqpoint{1.473847in}{1.243030in}}%
\pgfpathlineto{\pgfqpoint{1.512415in}{1.243030in}}%
\pgfpathlineto{\pgfqpoint{1.512415in}{1.257424in}}%
\pgfpathlineto{\pgfqpoint{1.550984in}{1.257424in}}%
\pgfpathlineto{\pgfqpoint{1.550984in}{1.254451in}}%
\pgfpathlineto{\pgfqpoint{1.589552in}{1.254451in}}%
\pgfpathlineto{\pgfqpoint{1.589552in}{1.267902in}}%
\pgfpathlineto{\pgfqpoint{1.628121in}{1.267902in}}%
\pgfpathlineto{\pgfqpoint{1.628121in}{1.263607in}}%
\pgfpathlineto{\pgfqpoint{1.666689in}{1.263607in}}%
\pgfpathlineto{\pgfqpoint{1.666689in}{1.253319in}}%
\pgfpathlineto{\pgfqpoint{1.705258in}{1.253319in}}%
\pgfpathlineto{\pgfqpoint{1.705258in}{1.251006in}}%
\pgfpathlineto{\pgfqpoint{1.743826in}{1.251006in}}%
\pgfpathlineto{\pgfqpoint{1.743826in}{1.171437in}}%
\pgfpathlineto{\pgfqpoint{1.782395in}{1.171437in}}%
\pgfpathlineto{\pgfqpoint{1.782395in}{1.089603in}}%
\pgfpathlineto{\pgfqpoint{1.820963in}{1.089603in}}%
\pgfpathlineto{\pgfqpoint{1.820963in}{0.993610in}}%
\pgfpathlineto{\pgfqpoint{1.859531in}{0.993610in}}%
\pgfpathlineto{\pgfqpoint{1.859531in}{0.910927in}}%
\pgfpathlineto{\pgfqpoint{1.898100in}{0.910927in}}%
\pgfpathlineto{\pgfqpoint{1.898100in}{0.864960in}}%
\pgfpathlineto{\pgfqpoint{1.936668in}{0.864960in}}%
\pgfpathlineto{\pgfqpoint{1.936668in}{0.838861in}}%
\pgfpathlineto{\pgfqpoint{1.975237in}{0.838861in}}%
\pgfpathlineto{\pgfqpoint{1.975237in}{0.808799in}}%
\pgfpathlineto{\pgfqpoint{2.013805in}{0.808799in}}%
\pgfpathlineto{\pgfqpoint{2.013805in}{0.793036in}}%
\pgfpathlineto{\pgfqpoint{2.052374in}{0.793036in}}%
\pgfpathlineto{\pgfqpoint{2.052374in}{0.781002in}}%
\pgfpathlineto{\pgfqpoint{2.090942in}{0.781002in}}%
\pgfpathlineto{\pgfqpoint{2.090942in}{0.744710in}}%
\pgfpathlineto{\pgfqpoint{2.129510in}{0.744710in}}%
\pgfpathlineto{\pgfqpoint{2.129510in}{0.719602in}}%
\pgfpathlineto{\pgfqpoint{2.168079in}{0.719602in}}%
\pgfpathlineto{\pgfqpoint{2.168079in}{0.657967in}}%
\pgfpathlineto{\pgfqpoint{2.206647in}{0.657967in}}%
\pgfpathlineto{\pgfqpoint{2.206647in}{0.606243in}}%
\pgfpathlineto{\pgfqpoint{2.245216in}{0.606243in}}%
\pgfpathlineto{\pgfqpoint{2.245216in}{0.552064in}}%
\pgfpathlineto{\pgfqpoint{2.283784in}{0.552064in}}%
\pgfpathlineto{\pgfqpoint{2.283784in}{0.510392in}}%
\pgfpathlineto{\pgfqpoint{2.322353in}{0.510392in}}%
\pgfpathlineto{\pgfqpoint{2.322353in}{0.485568in}}%
\pgfpathlineto{\pgfqpoint{2.360921in}{0.485568in}}%
\pgfpathlineto{\pgfqpoint{2.360921in}{0.472117in}}%
\pgfpathlineto{\pgfqpoint{2.399490in}{0.472117in}}%
\pgfpathlineto{\pgfqpoint{2.399490in}{0.462301in}}%
\pgfpathlineto{\pgfqpoint{2.438058in}{0.462301in}}%
\pgfpathlineto{\pgfqpoint{2.438058in}{0.454939in}}%
\pgfpathlineto{\pgfqpoint{2.476626in}{0.454939in}}%
\pgfpathlineto{\pgfqpoint{2.476626in}{0.451777in}}%
\pgfpathlineto{\pgfqpoint{2.515195in}{0.451777in}}%
\pgfpathlineto{\pgfqpoint{2.515195in}{0.449983in}}%
\pgfusepath{stroke}%
\end{pgfscope}%
\begin{pgfscope}%
\pgfsetrectcap%
\pgfsetmiterjoin%
\pgfsetlinewidth{1.003750pt}%
\definecolor{currentstroke}{rgb}{0.000000,0.000000,0.000000}%
\pgfsetstrokecolor{currentstroke}%
\pgfsetdash{}{0pt}%
\pgfpathmoveto{\pgfqpoint{0.366840in}{1.615583in}}%
\pgfpathlineto{\pgfqpoint{2.561650in}{1.615583in}}%
\pgfusepath{stroke}%
\end{pgfscope}%
\begin{pgfscope}%
\pgfsetrectcap%
\pgfsetmiterjoin%
\pgfsetlinewidth{1.003750pt}%
\definecolor{currentstroke}{rgb}{0.000000,0.000000,0.000000}%
\pgfsetstrokecolor{currentstroke}%
\pgfsetdash{}{0pt}%
\pgfpathmoveto{\pgfqpoint{2.561650in}{0.449983in}}%
\pgfpathlineto{\pgfqpoint{2.561650in}{1.615583in}}%
\pgfusepath{stroke}%
\end{pgfscope}%
\begin{pgfscope}%
\pgfsetrectcap%
\pgfsetmiterjoin%
\pgfsetlinewidth{1.003750pt}%
\definecolor{currentstroke}{rgb}{0.000000,0.000000,0.000000}%
\pgfsetstrokecolor{currentstroke}%
\pgfsetdash{}{0pt}%
\pgfpathmoveto{\pgfqpoint{0.366840in}{0.449983in}}%
\pgfpathlineto{\pgfqpoint{2.561650in}{0.449983in}}%
\pgfusepath{stroke}%
\end{pgfscope}%
\begin{pgfscope}%
\pgfsetrectcap%
\pgfsetmiterjoin%
\pgfsetlinewidth{1.003750pt}%
\definecolor{currentstroke}{rgb}{0.000000,0.000000,0.000000}%
\pgfsetstrokecolor{currentstroke}%
\pgfsetdash{}{0pt}%
\pgfpathmoveto{\pgfqpoint{0.366840in}{0.449983in}}%
\pgfpathlineto{\pgfqpoint{0.366840in}{1.615583in}}%
\pgfusepath{stroke}%
\end{pgfscope}%
\begin{pgfscope}%
\pgfsetbuttcap%
\pgfsetroundjoin%
\definecolor{currentfill}{rgb}{0.000000,0.000000,0.000000}%
\pgfsetfillcolor{currentfill}%
\pgfsetlinewidth{0.501875pt}%
\definecolor{currentstroke}{rgb}{0.000000,0.000000,0.000000}%
\pgfsetstrokecolor{currentstroke}%
\pgfsetdash{}{0pt}%
\pgfsys@defobject{currentmarker}{\pgfqpoint{0.000000in}{0.000000in}}{\pgfqpoint{0.000000in}{0.069444in}}{%
\pgfpathmoveto{\pgfqpoint{0.000000in}{0.000000in}}%
\pgfpathlineto{\pgfqpoint{0.000000in}{0.069444in}}%
\pgfusepath{stroke,fill}%
}%
\begin{pgfscope}%
\pgfsys@transformshift{0.366840in}{0.449983in}%
\pgfsys@useobject{currentmarker}{}%
\end{pgfscope}%
\end{pgfscope}%
\begin{pgfscope}%
\pgfsetbuttcap%
\pgfsetroundjoin%
\definecolor{currentfill}{rgb}{0.000000,0.000000,0.000000}%
\pgfsetfillcolor{currentfill}%
\pgfsetlinewidth{0.501875pt}%
\definecolor{currentstroke}{rgb}{0.000000,0.000000,0.000000}%
\pgfsetstrokecolor{currentstroke}%
\pgfsetdash{}{0pt}%
\pgfsys@defobject{currentmarker}{\pgfqpoint{0.000000in}{-0.069444in}}{\pgfqpoint{0.000000in}{0.000000in}}{%
\pgfpathmoveto{\pgfqpoint{0.000000in}{0.000000in}}%
\pgfpathlineto{\pgfqpoint{0.000000in}{-0.069444in}}%
\pgfusepath{stroke,fill}%
}%
\begin{pgfscope}%
\pgfsys@transformshift{0.366840in}{1.615583in}%
\pgfsys@useobject{currentmarker}{}%
\end{pgfscope}%
\end{pgfscope}%
\begin{pgfscope}%
\pgftext[x=0.366840in,y=0.380539in,,top]{\rmfamily\fontsize{8.000000}{9.600000}\selectfont −5}%
\end{pgfscope}%
\begin{pgfscope}%
\pgfsetbuttcap%
\pgfsetroundjoin%
\definecolor{currentfill}{rgb}{0.000000,0.000000,0.000000}%
\pgfsetfillcolor{currentfill}%
\pgfsetlinewidth{0.501875pt}%
\definecolor{currentstroke}{rgb}{0.000000,0.000000,0.000000}%
\pgfsetstrokecolor{currentstroke}%
\pgfsetdash{}{0pt}%
\pgfsys@defobject{currentmarker}{\pgfqpoint{0.000000in}{0.000000in}}{\pgfqpoint{0.000000in}{0.069444in}}{%
\pgfpathmoveto{\pgfqpoint{0.000000in}{0.000000in}}%
\pgfpathlineto{\pgfqpoint{0.000000in}{0.069444in}}%
\pgfusepath{stroke,fill}%
}%
\begin{pgfscope}%
\pgfsys@transformshift{0.915543in}{0.449983in}%
\pgfsys@useobject{currentmarker}{}%
\end{pgfscope}%
\end{pgfscope}%
\begin{pgfscope}%
\pgfsetbuttcap%
\pgfsetroundjoin%
\definecolor{currentfill}{rgb}{0.000000,0.000000,0.000000}%
\pgfsetfillcolor{currentfill}%
\pgfsetlinewidth{0.501875pt}%
\definecolor{currentstroke}{rgb}{0.000000,0.000000,0.000000}%
\pgfsetstrokecolor{currentstroke}%
\pgfsetdash{}{0pt}%
\pgfsys@defobject{currentmarker}{\pgfqpoint{0.000000in}{-0.069444in}}{\pgfqpoint{0.000000in}{0.000000in}}{%
\pgfpathmoveto{\pgfqpoint{0.000000in}{0.000000in}}%
\pgfpathlineto{\pgfqpoint{0.000000in}{-0.069444in}}%
\pgfusepath{stroke,fill}%
}%
\begin{pgfscope}%
\pgfsys@transformshift{0.915543in}{1.615583in}%
\pgfsys@useobject{currentmarker}{}%
\end{pgfscope}%
\end{pgfscope}%
\begin{pgfscope}%
\pgftext[x=0.915543in,y=0.380539in,,top]{\rmfamily\fontsize{8.000000}{9.600000}\selectfont 0}%
\end{pgfscope}%
\begin{pgfscope}%
\pgfsetbuttcap%
\pgfsetroundjoin%
\definecolor{currentfill}{rgb}{0.000000,0.000000,0.000000}%
\pgfsetfillcolor{currentfill}%
\pgfsetlinewidth{0.501875pt}%
\definecolor{currentstroke}{rgb}{0.000000,0.000000,0.000000}%
\pgfsetstrokecolor{currentstroke}%
\pgfsetdash{}{0pt}%
\pgfsys@defobject{currentmarker}{\pgfqpoint{0.000000in}{0.000000in}}{\pgfqpoint{0.000000in}{0.069444in}}{%
\pgfpathmoveto{\pgfqpoint{0.000000in}{0.000000in}}%
\pgfpathlineto{\pgfqpoint{0.000000in}{0.069444in}}%
\pgfusepath{stroke,fill}%
}%
\begin{pgfscope}%
\pgfsys@transformshift{1.464245in}{0.449983in}%
\pgfsys@useobject{currentmarker}{}%
\end{pgfscope}%
\end{pgfscope}%
\begin{pgfscope}%
\pgfsetbuttcap%
\pgfsetroundjoin%
\definecolor{currentfill}{rgb}{0.000000,0.000000,0.000000}%
\pgfsetfillcolor{currentfill}%
\pgfsetlinewidth{0.501875pt}%
\definecolor{currentstroke}{rgb}{0.000000,0.000000,0.000000}%
\pgfsetstrokecolor{currentstroke}%
\pgfsetdash{}{0pt}%
\pgfsys@defobject{currentmarker}{\pgfqpoint{0.000000in}{-0.069444in}}{\pgfqpoint{0.000000in}{0.000000in}}{%
\pgfpathmoveto{\pgfqpoint{0.000000in}{0.000000in}}%
\pgfpathlineto{\pgfqpoint{0.000000in}{-0.069444in}}%
\pgfusepath{stroke,fill}%
}%
\begin{pgfscope}%
\pgfsys@transformshift{1.464245in}{1.615583in}%
\pgfsys@useobject{currentmarker}{}%
\end{pgfscope}%
\end{pgfscope}%
\begin{pgfscope}%
\pgftext[x=1.464245in,y=0.380539in,,top]{\rmfamily\fontsize{8.000000}{9.600000}\selectfont 5}%
\end{pgfscope}%
\begin{pgfscope}%
\pgfsetbuttcap%
\pgfsetroundjoin%
\definecolor{currentfill}{rgb}{0.000000,0.000000,0.000000}%
\pgfsetfillcolor{currentfill}%
\pgfsetlinewidth{0.501875pt}%
\definecolor{currentstroke}{rgb}{0.000000,0.000000,0.000000}%
\pgfsetstrokecolor{currentstroke}%
\pgfsetdash{}{0pt}%
\pgfsys@defobject{currentmarker}{\pgfqpoint{0.000000in}{0.000000in}}{\pgfqpoint{0.000000in}{0.069444in}}{%
\pgfpathmoveto{\pgfqpoint{0.000000in}{0.000000in}}%
\pgfpathlineto{\pgfqpoint{0.000000in}{0.069444in}}%
\pgfusepath{stroke,fill}%
}%
\begin{pgfscope}%
\pgfsys@transformshift{2.012947in}{0.449983in}%
\pgfsys@useobject{currentmarker}{}%
\end{pgfscope}%
\end{pgfscope}%
\begin{pgfscope}%
\pgfsetbuttcap%
\pgfsetroundjoin%
\definecolor{currentfill}{rgb}{0.000000,0.000000,0.000000}%
\pgfsetfillcolor{currentfill}%
\pgfsetlinewidth{0.501875pt}%
\definecolor{currentstroke}{rgb}{0.000000,0.000000,0.000000}%
\pgfsetstrokecolor{currentstroke}%
\pgfsetdash{}{0pt}%
\pgfsys@defobject{currentmarker}{\pgfqpoint{0.000000in}{-0.069444in}}{\pgfqpoint{0.000000in}{0.000000in}}{%
\pgfpathmoveto{\pgfqpoint{0.000000in}{0.000000in}}%
\pgfpathlineto{\pgfqpoint{0.000000in}{-0.069444in}}%
\pgfusepath{stroke,fill}%
}%
\begin{pgfscope}%
\pgfsys@transformshift{2.012947in}{1.615583in}%
\pgfsys@useobject{currentmarker}{}%
\end{pgfscope}%
\end{pgfscope}%
\begin{pgfscope}%
\pgftext[x=2.012947in,y=0.380539in,,top]{\rmfamily\fontsize{8.000000}{9.600000}\selectfont 10}%
\end{pgfscope}%
\begin{pgfscope}%
\pgfsetbuttcap%
\pgfsetroundjoin%
\definecolor{currentfill}{rgb}{0.000000,0.000000,0.000000}%
\pgfsetfillcolor{currentfill}%
\pgfsetlinewidth{0.501875pt}%
\definecolor{currentstroke}{rgb}{0.000000,0.000000,0.000000}%
\pgfsetstrokecolor{currentstroke}%
\pgfsetdash{}{0pt}%
\pgfsys@defobject{currentmarker}{\pgfqpoint{0.000000in}{0.000000in}}{\pgfqpoint{0.000000in}{0.069444in}}{%
\pgfpathmoveto{\pgfqpoint{0.000000in}{0.000000in}}%
\pgfpathlineto{\pgfqpoint{0.000000in}{0.069444in}}%
\pgfusepath{stroke,fill}%
}%
\begin{pgfscope}%
\pgfsys@transformshift{2.561650in}{0.449983in}%
\pgfsys@useobject{currentmarker}{}%
\end{pgfscope}%
\end{pgfscope}%
\begin{pgfscope}%
\pgfsetbuttcap%
\pgfsetroundjoin%
\definecolor{currentfill}{rgb}{0.000000,0.000000,0.000000}%
\pgfsetfillcolor{currentfill}%
\pgfsetlinewidth{0.501875pt}%
\definecolor{currentstroke}{rgb}{0.000000,0.000000,0.000000}%
\pgfsetstrokecolor{currentstroke}%
\pgfsetdash{}{0pt}%
\pgfsys@defobject{currentmarker}{\pgfqpoint{0.000000in}{-0.069444in}}{\pgfqpoint{0.000000in}{0.000000in}}{%
\pgfpathmoveto{\pgfqpoint{0.000000in}{0.000000in}}%
\pgfpathlineto{\pgfqpoint{0.000000in}{-0.069444in}}%
\pgfusepath{stroke,fill}%
}%
\begin{pgfscope}%
\pgfsys@transformshift{2.561650in}{1.615583in}%
\pgfsys@useobject{currentmarker}{}%
\end{pgfscope}%
\end{pgfscope}%
\begin{pgfscope}%
\pgftext[x=2.561650in,y=0.380539in,,top]{\rmfamily\fontsize{8.000000}{9.600000}\selectfont 15}%
\end{pgfscope}%
\begin{pgfscope}%
\pgftext[x=1.464245in,y=0.203564in,,top]{\rmfamily\fontsize{9.000000}{10.800000}\selectfont \(\displaystyle \mathrm{DLL}_{\mu/\pi}(\mu^+)\)}%
\end{pgfscope}%
\begin{pgfscope}%
\pgfsetbuttcap%
\pgfsetroundjoin%
\definecolor{currentfill}{rgb}{0.000000,0.000000,0.000000}%
\pgfsetfillcolor{currentfill}%
\pgfsetlinewidth{0.501875pt}%
\definecolor{currentstroke}{rgb}{0.000000,0.000000,0.000000}%
\pgfsetstrokecolor{currentstroke}%
\pgfsetdash{}{0pt}%
\pgfsys@defobject{currentmarker}{\pgfqpoint{0.000000in}{0.000000in}}{\pgfqpoint{0.069444in}{0.000000in}}{%
\pgfpathmoveto{\pgfqpoint{0.000000in}{0.000000in}}%
\pgfpathlineto{\pgfqpoint{0.069444in}{0.000000in}}%
\pgfusepath{stroke,fill}%
}%
\begin{pgfscope}%
\pgfsys@transformshift{0.366840in}{0.449983in}%
\pgfsys@useobject{currentmarker}{}%
\end{pgfscope}%
\end{pgfscope}%
\begin{pgfscope}%
\pgfsetbuttcap%
\pgfsetroundjoin%
\definecolor{currentfill}{rgb}{0.000000,0.000000,0.000000}%
\pgfsetfillcolor{currentfill}%
\pgfsetlinewidth{0.501875pt}%
\definecolor{currentstroke}{rgb}{0.000000,0.000000,0.000000}%
\pgfsetstrokecolor{currentstroke}%
\pgfsetdash{}{0pt}%
\pgfsys@defobject{currentmarker}{\pgfqpoint{-0.069444in}{0.000000in}}{\pgfqpoint{0.000000in}{0.000000in}}{%
\pgfpathmoveto{\pgfqpoint{0.000000in}{0.000000in}}%
\pgfpathlineto{\pgfqpoint{-0.069444in}{0.000000in}}%
\pgfusepath{stroke,fill}%
}%
\begin{pgfscope}%
\pgfsys@transformshift{2.561650in}{0.449983in}%
\pgfsys@useobject{currentmarker}{}%
\end{pgfscope}%
\end{pgfscope}%
\begin{pgfscope}%
\pgftext[x=0.297396in,y=0.449983in,right,]{\rmfamily\fontsize{8.000000}{9.600000}\selectfont 0.00}%
\end{pgfscope}%
\begin{pgfscope}%
\pgfsetbuttcap%
\pgfsetroundjoin%
\definecolor{currentfill}{rgb}{0.000000,0.000000,0.000000}%
\pgfsetfillcolor{currentfill}%
\pgfsetlinewidth{0.501875pt}%
\definecolor{currentstroke}{rgb}{0.000000,0.000000,0.000000}%
\pgfsetstrokecolor{currentstroke}%
\pgfsetdash{}{0pt}%
\pgfsys@defobject{currentmarker}{\pgfqpoint{0.000000in}{0.000000in}}{\pgfqpoint{0.069444in}{0.000000in}}{%
\pgfpathmoveto{\pgfqpoint{0.000000in}{0.000000in}}%
\pgfpathlineto{\pgfqpoint{0.069444in}{0.000000in}}%
\pgfusepath{stroke,fill}%
}%
\begin{pgfscope}%
\pgfsys@transformshift{0.366840in}{0.595683in}%
\pgfsys@useobject{currentmarker}{}%
\end{pgfscope}%
\end{pgfscope}%
\begin{pgfscope}%
\pgfsetbuttcap%
\pgfsetroundjoin%
\definecolor{currentfill}{rgb}{0.000000,0.000000,0.000000}%
\pgfsetfillcolor{currentfill}%
\pgfsetlinewidth{0.501875pt}%
\definecolor{currentstroke}{rgb}{0.000000,0.000000,0.000000}%
\pgfsetstrokecolor{currentstroke}%
\pgfsetdash{}{0pt}%
\pgfsys@defobject{currentmarker}{\pgfqpoint{-0.069444in}{0.000000in}}{\pgfqpoint{0.000000in}{0.000000in}}{%
\pgfpathmoveto{\pgfqpoint{0.000000in}{0.000000in}}%
\pgfpathlineto{\pgfqpoint{-0.069444in}{0.000000in}}%
\pgfusepath{stroke,fill}%
}%
\begin{pgfscope}%
\pgfsys@transformshift{2.561650in}{0.595683in}%
\pgfsys@useobject{currentmarker}{}%
\end{pgfscope}%
\end{pgfscope}%
\begin{pgfscope}%
\pgftext[x=0.297396in,y=0.595683in,right,]{\rmfamily\fontsize{8.000000}{9.600000}\selectfont 0.02}%
\end{pgfscope}%
\begin{pgfscope}%
\pgfsetbuttcap%
\pgfsetroundjoin%
\definecolor{currentfill}{rgb}{0.000000,0.000000,0.000000}%
\pgfsetfillcolor{currentfill}%
\pgfsetlinewidth{0.501875pt}%
\definecolor{currentstroke}{rgb}{0.000000,0.000000,0.000000}%
\pgfsetstrokecolor{currentstroke}%
\pgfsetdash{}{0pt}%
\pgfsys@defobject{currentmarker}{\pgfqpoint{0.000000in}{0.000000in}}{\pgfqpoint{0.069444in}{0.000000in}}{%
\pgfpathmoveto{\pgfqpoint{0.000000in}{0.000000in}}%
\pgfpathlineto{\pgfqpoint{0.069444in}{0.000000in}}%
\pgfusepath{stroke,fill}%
}%
\begin{pgfscope}%
\pgfsys@transformshift{0.366840in}{0.741383in}%
\pgfsys@useobject{currentmarker}{}%
\end{pgfscope}%
\end{pgfscope}%
\begin{pgfscope}%
\pgfsetbuttcap%
\pgfsetroundjoin%
\definecolor{currentfill}{rgb}{0.000000,0.000000,0.000000}%
\pgfsetfillcolor{currentfill}%
\pgfsetlinewidth{0.501875pt}%
\definecolor{currentstroke}{rgb}{0.000000,0.000000,0.000000}%
\pgfsetstrokecolor{currentstroke}%
\pgfsetdash{}{0pt}%
\pgfsys@defobject{currentmarker}{\pgfqpoint{-0.069444in}{0.000000in}}{\pgfqpoint{0.000000in}{0.000000in}}{%
\pgfpathmoveto{\pgfqpoint{0.000000in}{0.000000in}}%
\pgfpathlineto{\pgfqpoint{-0.069444in}{0.000000in}}%
\pgfusepath{stroke,fill}%
}%
\begin{pgfscope}%
\pgfsys@transformshift{2.561650in}{0.741383in}%
\pgfsys@useobject{currentmarker}{}%
\end{pgfscope}%
\end{pgfscope}%
\begin{pgfscope}%
\pgftext[x=0.297396in,y=0.741383in,right,]{\rmfamily\fontsize{8.000000}{9.600000}\selectfont 0.04}%
\end{pgfscope}%
\begin{pgfscope}%
\pgfsetbuttcap%
\pgfsetroundjoin%
\definecolor{currentfill}{rgb}{0.000000,0.000000,0.000000}%
\pgfsetfillcolor{currentfill}%
\pgfsetlinewidth{0.501875pt}%
\definecolor{currentstroke}{rgb}{0.000000,0.000000,0.000000}%
\pgfsetstrokecolor{currentstroke}%
\pgfsetdash{}{0pt}%
\pgfsys@defobject{currentmarker}{\pgfqpoint{0.000000in}{0.000000in}}{\pgfqpoint{0.069444in}{0.000000in}}{%
\pgfpathmoveto{\pgfqpoint{0.000000in}{0.000000in}}%
\pgfpathlineto{\pgfqpoint{0.069444in}{0.000000in}}%
\pgfusepath{stroke,fill}%
}%
\begin{pgfscope}%
\pgfsys@transformshift{0.366840in}{0.887083in}%
\pgfsys@useobject{currentmarker}{}%
\end{pgfscope}%
\end{pgfscope}%
\begin{pgfscope}%
\pgfsetbuttcap%
\pgfsetroundjoin%
\definecolor{currentfill}{rgb}{0.000000,0.000000,0.000000}%
\pgfsetfillcolor{currentfill}%
\pgfsetlinewidth{0.501875pt}%
\definecolor{currentstroke}{rgb}{0.000000,0.000000,0.000000}%
\pgfsetstrokecolor{currentstroke}%
\pgfsetdash{}{0pt}%
\pgfsys@defobject{currentmarker}{\pgfqpoint{-0.069444in}{0.000000in}}{\pgfqpoint{0.000000in}{0.000000in}}{%
\pgfpathmoveto{\pgfqpoint{0.000000in}{0.000000in}}%
\pgfpathlineto{\pgfqpoint{-0.069444in}{0.000000in}}%
\pgfusepath{stroke,fill}%
}%
\begin{pgfscope}%
\pgfsys@transformshift{2.561650in}{0.887083in}%
\pgfsys@useobject{currentmarker}{}%
\end{pgfscope}%
\end{pgfscope}%
\begin{pgfscope}%
\pgftext[x=0.297396in,y=0.887083in,right,]{\rmfamily\fontsize{8.000000}{9.600000}\selectfont 0.06}%
\end{pgfscope}%
\begin{pgfscope}%
\pgfsetbuttcap%
\pgfsetroundjoin%
\definecolor{currentfill}{rgb}{0.000000,0.000000,0.000000}%
\pgfsetfillcolor{currentfill}%
\pgfsetlinewidth{0.501875pt}%
\definecolor{currentstroke}{rgb}{0.000000,0.000000,0.000000}%
\pgfsetstrokecolor{currentstroke}%
\pgfsetdash{}{0pt}%
\pgfsys@defobject{currentmarker}{\pgfqpoint{0.000000in}{0.000000in}}{\pgfqpoint{0.069444in}{0.000000in}}{%
\pgfpathmoveto{\pgfqpoint{0.000000in}{0.000000in}}%
\pgfpathlineto{\pgfqpoint{0.069444in}{0.000000in}}%
\pgfusepath{stroke,fill}%
}%
\begin{pgfscope}%
\pgfsys@transformshift{0.366840in}{1.032783in}%
\pgfsys@useobject{currentmarker}{}%
\end{pgfscope}%
\end{pgfscope}%
\begin{pgfscope}%
\pgfsetbuttcap%
\pgfsetroundjoin%
\definecolor{currentfill}{rgb}{0.000000,0.000000,0.000000}%
\pgfsetfillcolor{currentfill}%
\pgfsetlinewidth{0.501875pt}%
\definecolor{currentstroke}{rgb}{0.000000,0.000000,0.000000}%
\pgfsetstrokecolor{currentstroke}%
\pgfsetdash{}{0pt}%
\pgfsys@defobject{currentmarker}{\pgfqpoint{-0.069444in}{0.000000in}}{\pgfqpoint{0.000000in}{0.000000in}}{%
\pgfpathmoveto{\pgfqpoint{0.000000in}{0.000000in}}%
\pgfpathlineto{\pgfqpoint{-0.069444in}{0.000000in}}%
\pgfusepath{stroke,fill}%
}%
\begin{pgfscope}%
\pgfsys@transformshift{2.561650in}{1.032783in}%
\pgfsys@useobject{currentmarker}{}%
\end{pgfscope}%
\end{pgfscope}%
\begin{pgfscope}%
\pgftext[x=0.297396in,y=1.032783in,right,]{\rmfamily\fontsize{8.000000}{9.600000}\selectfont 0.08}%
\end{pgfscope}%
\begin{pgfscope}%
\pgfsetbuttcap%
\pgfsetroundjoin%
\definecolor{currentfill}{rgb}{0.000000,0.000000,0.000000}%
\pgfsetfillcolor{currentfill}%
\pgfsetlinewidth{0.501875pt}%
\definecolor{currentstroke}{rgb}{0.000000,0.000000,0.000000}%
\pgfsetstrokecolor{currentstroke}%
\pgfsetdash{}{0pt}%
\pgfsys@defobject{currentmarker}{\pgfqpoint{0.000000in}{0.000000in}}{\pgfqpoint{0.069444in}{0.000000in}}{%
\pgfpathmoveto{\pgfqpoint{0.000000in}{0.000000in}}%
\pgfpathlineto{\pgfqpoint{0.069444in}{0.000000in}}%
\pgfusepath{stroke,fill}%
}%
\begin{pgfscope}%
\pgfsys@transformshift{0.366840in}{1.178483in}%
\pgfsys@useobject{currentmarker}{}%
\end{pgfscope}%
\end{pgfscope}%
\begin{pgfscope}%
\pgfsetbuttcap%
\pgfsetroundjoin%
\definecolor{currentfill}{rgb}{0.000000,0.000000,0.000000}%
\pgfsetfillcolor{currentfill}%
\pgfsetlinewidth{0.501875pt}%
\definecolor{currentstroke}{rgb}{0.000000,0.000000,0.000000}%
\pgfsetstrokecolor{currentstroke}%
\pgfsetdash{}{0pt}%
\pgfsys@defobject{currentmarker}{\pgfqpoint{-0.069444in}{0.000000in}}{\pgfqpoint{0.000000in}{0.000000in}}{%
\pgfpathmoveto{\pgfqpoint{0.000000in}{0.000000in}}%
\pgfpathlineto{\pgfqpoint{-0.069444in}{0.000000in}}%
\pgfusepath{stroke,fill}%
}%
\begin{pgfscope}%
\pgfsys@transformshift{2.561650in}{1.178483in}%
\pgfsys@useobject{currentmarker}{}%
\end{pgfscope}%
\end{pgfscope}%
\begin{pgfscope}%
\pgftext[x=0.297396in,y=1.178483in,right,]{\rmfamily\fontsize{8.000000}{9.600000}\selectfont 0.10}%
\end{pgfscope}%
\begin{pgfscope}%
\pgfsetbuttcap%
\pgfsetroundjoin%
\definecolor{currentfill}{rgb}{0.000000,0.000000,0.000000}%
\pgfsetfillcolor{currentfill}%
\pgfsetlinewidth{0.501875pt}%
\definecolor{currentstroke}{rgb}{0.000000,0.000000,0.000000}%
\pgfsetstrokecolor{currentstroke}%
\pgfsetdash{}{0pt}%
\pgfsys@defobject{currentmarker}{\pgfqpoint{0.000000in}{0.000000in}}{\pgfqpoint{0.069444in}{0.000000in}}{%
\pgfpathmoveto{\pgfqpoint{0.000000in}{0.000000in}}%
\pgfpathlineto{\pgfqpoint{0.069444in}{0.000000in}}%
\pgfusepath{stroke,fill}%
}%
\begin{pgfscope}%
\pgfsys@transformshift{0.366840in}{1.324183in}%
\pgfsys@useobject{currentmarker}{}%
\end{pgfscope}%
\end{pgfscope}%
\begin{pgfscope}%
\pgfsetbuttcap%
\pgfsetroundjoin%
\definecolor{currentfill}{rgb}{0.000000,0.000000,0.000000}%
\pgfsetfillcolor{currentfill}%
\pgfsetlinewidth{0.501875pt}%
\definecolor{currentstroke}{rgb}{0.000000,0.000000,0.000000}%
\pgfsetstrokecolor{currentstroke}%
\pgfsetdash{}{0pt}%
\pgfsys@defobject{currentmarker}{\pgfqpoint{-0.069444in}{0.000000in}}{\pgfqpoint{0.000000in}{0.000000in}}{%
\pgfpathmoveto{\pgfqpoint{0.000000in}{0.000000in}}%
\pgfpathlineto{\pgfqpoint{-0.069444in}{0.000000in}}%
\pgfusepath{stroke,fill}%
}%
\begin{pgfscope}%
\pgfsys@transformshift{2.561650in}{1.324183in}%
\pgfsys@useobject{currentmarker}{}%
\end{pgfscope}%
\end{pgfscope}%
\begin{pgfscope}%
\pgftext[x=0.297396in,y=1.324183in,right,]{\rmfamily\fontsize{8.000000}{9.600000}\selectfont 0.12}%
\end{pgfscope}%
\begin{pgfscope}%
\pgfsetbuttcap%
\pgfsetroundjoin%
\definecolor{currentfill}{rgb}{0.000000,0.000000,0.000000}%
\pgfsetfillcolor{currentfill}%
\pgfsetlinewidth{0.501875pt}%
\definecolor{currentstroke}{rgb}{0.000000,0.000000,0.000000}%
\pgfsetstrokecolor{currentstroke}%
\pgfsetdash{}{0pt}%
\pgfsys@defobject{currentmarker}{\pgfqpoint{0.000000in}{0.000000in}}{\pgfqpoint{0.069444in}{0.000000in}}{%
\pgfpathmoveto{\pgfqpoint{0.000000in}{0.000000in}}%
\pgfpathlineto{\pgfqpoint{0.069444in}{0.000000in}}%
\pgfusepath{stroke,fill}%
}%
\begin{pgfscope}%
\pgfsys@transformshift{0.366840in}{1.469883in}%
\pgfsys@useobject{currentmarker}{}%
\end{pgfscope}%
\end{pgfscope}%
\begin{pgfscope}%
\pgfsetbuttcap%
\pgfsetroundjoin%
\definecolor{currentfill}{rgb}{0.000000,0.000000,0.000000}%
\pgfsetfillcolor{currentfill}%
\pgfsetlinewidth{0.501875pt}%
\definecolor{currentstroke}{rgb}{0.000000,0.000000,0.000000}%
\pgfsetstrokecolor{currentstroke}%
\pgfsetdash{}{0pt}%
\pgfsys@defobject{currentmarker}{\pgfqpoint{-0.069444in}{0.000000in}}{\pgfqpoint{0.000000in}{0.000000in}}{%
\pgfpathmoveto{\pgfqpoint{0.000000in}{0.000000in}}%
\pgfpathlineto{\pgfqpoint{-0.069444in}{0.000000in}}%
\pgfusepath{stroke,fill}%
}%
\begin{pgfscope}%
\pgfsys@transformshift{2.561650in}{1.469883in}%
\pgfsys@useobject{currentmarker}{}%
\end{pgfscope}%
\end{pgfscope}%
\begin{pgfscope}%
\pgftext[x=0.297396in,y=1.469883in,right,]{\rmfamily\fontsize{8.000000}{9.600000}\selectfont 0.14}%
\end{pgfscope}%
\begin{pgfscope}%
\pgfsetbuttcap%
\pgfsetroundjoin%
\definecolor{currentfill}{rgb}{0.000000,0.000000,0.000000}%
\pgfsetfillcolor{currentfill}%
\pgfsetlinewidth{0.501875pt}%
\definecolor{currentstroke}{rgb}{0.000000,0.000000,0.000000}%
\pgfsetstrokecolor{currentstroke}%
\pgfsetdash{}{0pt}%
\pgfsys@defobject{currentmarker}{\pgfqpoint{0.000000in}{0.000000in}}{\pgfqpoint{0.069444in}{0.000000in}}{%
\pgfpathmoveto{\pgfqpoint{0.000000in}{0.000000in}}%
\pgfpathlineto{\pgfqpoint{0.069444in}{0.000000in}}%
\pgfusepath{stroke,fill}%
}%
\begin{pgfscope}%
\pgfsys@transformshift{0.366840in}{1.615583in}%
\pgfsys@useobject{currentmarker}{}%
\end{pgfscope}%
\end{pgfscope}%
\begin{pgfscope}%
\pgfsetbuttcap%
\pgfsetroundjoin%
\definecolor{currentfill}{rgb}{0.000000,0.000000,0.000000}%
\pgfsetfillcolor{currentfill}%
\pgfsetlinewidth{0.501875pt}%
\definecolor{currentstroke}{rgb}{0.000000,0.000000,0.000000}%
\pgfsetstrokecolor{currentstroke}%
\pgfsetdash{}{0pt}%
\pgfsys@defobject{currentmarker}{\pgfqpoint{-0.069444in}{0.000000in}}{\pgfqpoint{0.000000in}{0.000000in}}{%
\pgfpathmoveto{\pgfqpoint{0.000000in}{0.000000in}}%
\pgfpathlineto{\pgfqpoint{-0.069444in}{0.000000in}}%
\pgfusepath{stroke,fill}%
}%
\begin{pgfscope}%
\pgfsys@transformshift{2.561650in}{1.615583in}%
\pgfsys@useobject{currentmarker}{}%
\end{pgfscope}%
\end{pgfscope}%
\begin{pgfscope}%
\pgftext[x=0.297396in,y=1.615583in,right,]{\rmfamily\fontsize{8.000000}{9.600000}\selectfont 0.16}%
\end{pgfscope}%
\end{pgfpicture}%
\makeatother%
\endgroup%

	\end{subfigure}

  \caption{
    Comparison of signal distributions (blue) and background distributions (red) for all variables used in the multivariate selection.
  }
  \thisfloatpagestyle{empty}
  \label{fig:features}
\end{figure}

\section{Controlling differences between data and simulation}
\label{datamc}

An important systematic uncertainty on the multivariate selection procedure is the influence of differences between simulated and real decays.
Multivariate classifiers are able to  exploit even small differences in the distributions of the input variables, which can lead to large differences between the expected signal efficiency as determined using simulated samples and real decays.

In order to investigate this effect, an sweighted $\PBzero\to\PJpsi\PKstar$ data sample (obtained in section \ref{normfit}) has been classified together with the simulated $\PBzero\to\PJpsi\PKstar$ data sample using the classifier trained in the previous section.
As can be seen in figure \ref{fig:datamcsystematic}, the relative efficiency difference reaches around \SI{30}{\percent} at the chosen threshold of $3.905$.
A possible way to avoid such a large discrepancy is to refrain from using input variables that are known to show data-simulation differences (in particular, this is the case for the PID variables).
But this reduces the achievable signal efficiency, as the available data is only partially used.

In the following two sections, two alternative methods to reduce data-simulation differences are described and compared.
The first is a partial, bin-based reweighting of the dataset and a resampling of the PID variables from data.
This is a standard approach that is also currently used for this analysis.
The alternative is a full reweighting of the input data using a classifier that is trained on data-simulation differences.

\begin{figure}
  \centering
  %% Creator: Matplotlib, PGF backend
%%
%% To include the figure in your LaTeX document, write
%%   \input{<filename>.pgf}
%%
%% Make sure the required packages are loaded in your preamble
%%   \usepackage{pgf}
%%
%% Figures using additional raster images can only be included by \input if
%% they are in the same directory as the main LaTeX file. For loading figures
%% from other directories you can use the `import` package
%%   \usepackage{import}
%% and then include the figures with
%%   \import{<path to file>}{<filename>.pgf}
%%
%% Matplotlib used the following preamble
%%   \usepackage{fontspec}
%%   \setmainfont{DejaVu Serif}
%%   \setsansfont{DejaVu Sans}
%%   \setmonofont{DejaVu Sans Mono}
%%
\begingroup%
\makeatletter%
\begin{pgfpicture}%
\pgfpathrectangle{\pgfpointorigin}{\pgfqpoint{3.760382in}{2.576728in}}%
\pgfusepath{use as bounding box, clip}%
\begin{pgfscope}%
\pgfsetbuttcap%
\pgfsetmiterjoin%
\definecolor{currentfill}{rgb}{1.000000,1.000000,1.000000}%
\pgfsetfillcolor{currentfill}%
\pgfsetlinewidth{0.000000pt}%
\definecolor{currentstroke}{rgb}{1.000000,1.000000,1.000000}%
\pgfsetstrokecolor{currentstroke}%
\pgfsetdash{}{0pt}%
\pgfpathmoveto{\pgfqpoint{0.000000in}{0.000000in}}%
\pgfpathlineto{\pgfqpoint{3.760382in}{0.000000in}}%
\pgfpathlineto{\pgfqpoint{3.760382in}{2.576728in}}%
\pgfpathlineto{\pgfqpoint{0.000000in}{2.576728in}}%
\pgfpathclose%
\pgfusepath{fill}%
\end{pgfscope}%
\begin{pgfscope}%
\pgfsetbuttcap%
\pgfsetmiterjoin%
\definecolor{currentfill}{rgb}{1.000000,1.000000,1.000000}%
\pgfsetfillcolor{currentfill}%
\pgfsetlinewidth{0.000000pt}%
\definecolor{currentstroke}{rgb}{0.000000,0.000000,0.000000}%
\pgfsetstrokecolor{currentstroke}%
\pgfsetstrokeopacity{0.000000}%
\pgfsetdash{}{0pt}%
\pgfpathmoveto{\pgfqpoint{0.488334in}{0.417391in}}%
\pgfpathlineto{\pgfqpoint{3.710382in}{0.417391in}}%
\pgfpathlineto{\pgfqpoint{3.710382in}{2.472963in}}%
\pgfpathlineto{\pgfqpoint{0.488334in}{2.472963in}}%
\pgfpathclose%
\pgfusepath{fill}%
\end{pgfscope}%
\begin{pgfscope}%
\pgfpathrectangle{\pgfqpoint{0.488334in}{0.417391in}}{\pgfqpoint{3.222048in}{2.055572in}} %
\pgfusepath{clip}%
\pgfsetrectcap%
\pgfsetroundjoin%
\pgfsetlinewidth{1.003750pt}%
\definecolor{currentstroke}{rgb}{0.000000,0.000000,0.000000}%
\pgfsetstrokecolor{currentstroke}%
\pgfsetdash{}{0pt}%
\pgfpathmoveto{\pgfqpoint{0.488334in}{0.645788in}}%
\pgfpathlineto{\pgfqpoint{3.710382in}{0.645788in}}%
\pgfusepath{stroke}%
\end{pgfscope}%
\begin{pgfscope}%
\pgfpathrectangle{\pgfqpoint{0.488334in}{0.417391in}}{\pgfqpoint{3.222048in}{2.055572in}} %
\pgfusepath{clip}%
\pgfsetrectcap%
\pgfsetroundjoin%
\pgfsetlinewidth{1.003750pt}%
\definecolor{currentstroke}{rgb}{0.000000,0.000000,1.000000}%
\pgfsetstrokecolor{currentstroke}%
\pgfsetdash{}{0pt}%
\pgfpathmoveto{\pgfqpoint{0.488334in}{0.715760in}}%
\pgfpathlineto{\pgfqpoint{0.542215in}{0.721750in}}%
\pgfpathlineto{\pgfqpoint{0.725408in}{0.743907in}}%
\pgfpathlineto{\pgfqpoint{0.790064in}{0.753531in}}%
\pgfpathlineto{\pgfqpoint{0.843945in}{0.760932in}}%
\pgfpathlineto{\pgfqpoint{0.897825in}{0.768586in}}%
\pgfpathlineto{\pgfqpoint{0.940929in}{0.775650in}}%
\pgfpathlineto{\pgfqpoint{0.984034in}{0.783096in}}%
\pgfpathlineto{\pgfqpoint{1.124123in}{0.805233in}}%
\pgfpathlineto{\pgfqpoint{1.145675in}{0.807999in}}%
\pgfpathlineto{\pgfqpoint{1.274988in}{0.828187in}}%
\pgfpathlineto{\pgfqpoint{1.393525in}{0.851166in}}%
\pgfpathlineto{\pgfqpoint{1.415077in}{0.855487in}}%
\pgfpathlineto{\pgfqpoint{1.468957in}{0.864375in}}%
\pgfpathlineto{\pgfqpoint{1.490510in}{0.867530in}}%
\pgfpathlineto{\pgfqpoint{1.565942in}{0.882578in}}%
\pgfpathlineto{\pgfqpoint{1.576718in}{0.884502in}}%
\pgfpathlineto{\pgfqpoint{1.587494in}{0.888109in}}%
\pgfpathlineto{\pgfqpoint{1.598270in}{0.889594in}}%
\pgfpathlineto{\pgfqpoint{1.641375in}{0.900400in}}%
\pgfpathlineto{\pgfqpoint{1.673703in}{0.906659in}}%
\pgfpathlineto{\pgfqpoint{1.684479in}{0.908353in}}%
\pgfpathlineto{\pgfqpoint{1.695255in}{0.912015in}}%
\pgfpathlineto{\pgfqpoint{1.716807in}{0.916927in}}%
\pgfpathlineto{\pgfqpoint{1.749135in}{0.925802in}}%
\pgfpathlineto{\pgfqpoint{1.781464in}{0.934452in}}%
\pgfpathlineto{\pgfqpoint{1.792240in}{0.938936in}}%
\pgfpathlineto{\pgfqpoint{1.921553in}{0.974945in}}%
\pgfpathlineto{\pgfqpoint{1.943105in}{0.980721in}}%
\pgfpathlineto{\pgfqpoint{2.018537in}{1.001437in}}%
\pgfpathlineto{\pgfqpoint{2.050866in}{1.013317in}}%
\pgfpathlineto{\pgfqpoint{2.083194in}{1.024949in}}%
\pgfpathlineto{\pgfqpoint{2.126298in}{1.039456in}}%
\pgfpathlineto{\pgfqpoint{2.137074in}{1.041955in}}%
\pgfpathlineto{\pgfqpoint{2.158626in}{1.049539in}}%
\pgfpathlineto{\pgfqpoint{2.201731in}{1.062452in}}%
\pgfpathlineto{\pgfqpoint{2.223283in}{1.069699in}}%
\pgfpathlineto{\pgfqpoint{2.244835in}{1.077734in}}%
\pgfpathlineto{\pgfqpoint{2.266387in}{1.087354in}}%
\pgfpathlineto{\pgfqpoint{2.277163in}{1.090304in}}%
\pgfpathlineto{\pgfqpoint{2.287939in}{1.095224in}}%
\pgfpathlineto{\pgfqpoint{2.309492in}{1.102563in}}%
\pgfpathlineto{\pgfqpoint{2.320268in}{1.107168in}}%
\pgfpathlineto{\pgfqpoint{2.352596in}{1.118168in}}%
\pgfpathlineto{\pgfqpoint{2.374148in}{1.126710in}}%
\pgfpathlineto{\pgfqpoint{2.384924in}{1.130193in}}%
\pgfpathlineto{\pgfqpoint{2.438805in}{1.154227in}}%
\pgfpathlineto{\pgfqpoint{2.471133in}{1.169982in}}%
\pgfpathlineto{\pgfqpoint{2.481909in}{1.173893in}}%
\pgfpathlineto{\pgfqpoint{2.525013in}{1.197450in}}%
\pgfpathlineto{\pgfqpoint{2.546565in}{1.206556in}}%
\pgfpathlineto{\pgfqpoint{2.557341in}{1.209906in}}%
\pgfpathlineto{\pgfqpoint{2.611222in}{1.238449in}}%
\pgfpathlineto{\pgfqpoint{2.632774in}{1.248646in}}%
\pgfpathlineto{\pgfqpoint{2.643550in}{1.256094in}}%
\pgfpathlineto{\pgfqpoint{2.665102in}{1.268569in}}%
\pgfpathlineto{\pgfqpoint{2.718983in}{1.302873in}}%
\pgfpathlineto{\pgfqpoint{2.762087in}{1.327521in}}%
\pgfpathlineto{\pgfqpoint{2.783639in}{1.343094in}}%
\pgfpathlineto{\pgfqpoint{2.794415in}{1.349170in}}%
\pgfpathlineto{\pgfqpoint{2.805191in}{1.357362in}}%
\pgfpathlineto{\pgfqpoint{2.815967in}{1.363345in}}%
\pgfpathlineto{\pgfqpoint{2.826743in}{1.373031in}}%
\pgfpathlineto{\pgfqpoint{2.837519in}{1.379102in}}%
\pgfpathlineto{\pgfqpoint{2.848296in}{1.390120in}}%
\pgfpathlineto{\pgfqpoint{2.859072in}{1.395930in}}%
\pgfpathlineto{\pgfqpoint{2.880624in}{1.411162in}}%
\pgfpathlineto{\pgfqpoint{2.945280in}{1.463611in}}%
\pgfpathlineto{\pgfqpoint{2.956056in}{1.474362in}}%
\pgfpathlineto{\pgfqpoint{2.966832in}{1.482581in}}%
\pgfpathlineto{\pgfqpoint{2.977608in}{1.489342in}}%
\pgfpathlineto{\pgfqpoint{2.988385in}{1.498911in}}%
\pgfpathlineto{\pgfqpoint{3.009937in}{1.514614in}}%
\pgfpathlineto{\pgfqpoint{3.020713in}{1.525184in}}%
\pgfpathlineto{\pgfqpoint{3.031489in}{1.532255in}}%
\pgfpathlineto{\pgfqpoint{3.053041in}{1.555257in}}%
\pgfpathlineto{\pgfqpoint{3.063817in}{1.566176in}}%
\pgfpathlineto{\pgfqpoint{3.074593in}{1.575436in}}%
\pgfpathlineto{\pgfqpoint{3.085369in}{1.586385in}}%
\pgfpathlineto{\pgfqpoint{3.096145in}{1.594261in}}%
\pgfpathlineto{\pgfqpoint{3.106921in}{1.600652in}}%
\pgfpathlineto{\pgfqpoint{3.117697in}{1.615144in}}%
\pgfpathlineto{\pgfqpoint{3.150026in}{1.643345in}}%
\pgfpathlineto{\pgfqpoint{3.160802in}{1.655682in}}%
\pgfpathlineto{\pgfqpoint{3.171578in}{1.666232in}}%
\pgfpathlineto{\pgfqpoint{3.182354in}{1.675198in}}%
\pgfpathlineto{\pgfqpoint{3.193130in}{1.688176in}}%
\pgfpathlineto{\pgfqpoint{3.203906in}{1.695201in}}%
\pgfpathlineto{\pgfqpoint{3.225458in}{1.718566in}}%
\pgfpathlineto{\pgfqpoint{3.236234in}{1.730801in}}%
\pgfpathlineto{\pgfqpoint{3.247010in}{1.738712in}}%
\pgfpathlineto{\pgfqpoint{3.268563in}{1.764434in}}%
\pgfpathlineto{\pgfqpoint{3.279339in}{1.776654in}}%
\pgfpathlineto{\pgfqpoint{3.300891in}{1.807838in}}%
\pgfpathlineto{\pgfqpoint{3.311667in}{1.817976in}}%
\pgfpathlineto{\pgfqpoint{3.333219in}{1.847678in}}%
\pgfpathlineto{\pgfqpoint{3.343995in}{1.856028in}}%
\pgfpathlineto{\pgfqpoint{3.354771in}{1.876301in}}%
\pgfpathlineto{\pgfqpoint{3.365547in}{1.899509in}}%
\pgfpathlineto{\pgfqpoint{3.376323in}{1.912654in}}%
\pgfpathlineto{\pgfqpoint{3.387099in}{1.937934in}}%
\pgfpathlineto{\pgfqpoint{3.397876in}{1.946622in}}%
\pgfpathlineto{\pgfqpoint{3.408652in}{1.951827in}}%
\pgfpathlineto{\pgfqpoint{3.430204in}{1.978455in}}%
\pgfpathlineto{\pgfqpoint{3.440980in}{1.987391in}}%
\pgfpathlineto{\pgfqpoint{3.473308in}{2.039726in}}%
\pgfpathlineto{\pgfqpoint{3.484084in}{2.049939in}}%
\pgfpathlineto{\pgfqpoint{3.516412in}{2.106132in}}%
\pgfpathlineto{\pgfqpoint{3.527189in}{2.117896in}}%
\pgfpathlineto{\pgfqpoint{3.537965in}{2.138003in}}%
\pgfpathlineto{\pgfqpoint{3.548741in}{2.166883in}}%
\pgfpathlineto{\pgfqpoint{3.559517in}{2.180103in}}%
\pgfpathlineto{\pgfqpoint{3.570293in}{2.215010in}}%
\pgfpathlineto{\pgfqpoint{3.581069in}{2.231767in}}%
\pgfpathlineto{\pgfqpoint{3.591845in}{2.262343in}}%
\pgfpathlineto{\pgfqpoint{3.602621in}{2.284294in}}%
\pgfpathlineto{\pgfqpoint{3.624173in}{2.345175in}}%
\pgfpathlineto{\pgfqpoint{3.634949in}{2.384612in}}%
\pgfpathlineto{\pgfqpoint{3.645725in}{2.398858in}}%
\pgfpathlineto{\pgfqpoint{3.656501in}{2.436348in}}%
\pgfpathlineto{\pgfqpoint{3.667278in}{2.467500in}}%
\pgfpathlineto{\pgfqpoint{3.676866in}{2.482963in}}%
\pgfpathlineto{\pgfqpoint{3.676866in}{2.482963in}}%
\pgfusepath{stroke}%
\end{pgfscope}%
\begin{pgfscope}%
\pgfpathrectangle{\pgfqpoint{0.488334in}{0.417391in}}{\pgfqpoint{3.222048in}{2.055572in}} %
\pgfusepath{clip}%
\pgfsetrectcap%
\pgfsetroundjoin%
\pgfsetlinewidth{1.003750pt}%
\definecolor{currentstroke}{rgb}{1.000000,0.000000,0.000000}%
\pgfsetstrokecolor{currentstroke}%
\pgfsetdash{}{0pt}%
\pgfpathmoveto{\pgfqpoint{0.488334in}{0.708209in}}%
\pgfpathlineto{\pgfqpoint{0.552991in}{0.713667in}}%
\pgfpathlineto{\pgfqpoint{0.585319in}{0.714213in}}%
\pgfpathlineto{\pgfqpoint{0.606871in}{0.716278in}}%
\pgfpathlineto{\pgfqpoint{0.617647in}{0.718137in}}%
\pgfpathlineto{\pgfqpoint{0.746960in}{0.726647in}}%
\pgfpathlineto{\pgfqpoint{0.843945in}{0.736775in}}%
\pgfpathlineto{\pgfqpoint{0.919377in}{0.743811in}}%
\pgfpathlineto{\pgfqpoint{0.951706in}{0.747606in}}%
\pgfpathlineto{\pgfqpoint{0.962482in}{0.747310in}}%
\pgfpathlineto{\pgfqpoint{0.994810in}{0.750989in}}%
\pgfpathlineto{\pgfqpoint{1.102571in}{0.760849in}}%
\pgfpathlineto{\pgfqpoint{1.124123in}{0.760903in}}%
\pgfpathlineto{\pgfqpoint{1.231884in}{0.764603in}}%
\pgfpathlineto{\pgfqpoint{1.253436in}{0.766280in}}%
\pgfpathlineto{\pgfqpoint{1.264212in}{0.767835in}}%
\pgfpathlineto{\pgfqpoint{1.285764in}{0.767551in}}%
\pgfpathlineto{\pgfqpoint{1.318092in}{0.770589in}}%
\pgfpathlineto{\pgfqpoint{1.339644in}{0.774085in}}%
\pgfpathlineto{\pgfqpoint{1.350421in}{0.774014in}}%
\pgfpathlineto{\pgfqpoint{1.361197in}{0.775526in}}%
\pgfpathlineto{\pgfqpoint{1.382749in}{0.775310in}}%
\pgfpathlineto{\pgfqpoint{1.404301in}{0.778353in}}%
\pgfpathlineto{\pgfqpoint{1.425853in}{0.778924in}}%
\pgfpathlineto{\pgfqpoint{1.436629in}{0.780484in}}%
\pgfpathlineto{\pgfqpoint{1.490510in}{0.780070in}}%
\pgfpathlineto{\pgfqpoint{1.501286in}{0.781715in}}%
\pgfpathlineto{\pgfqpoint{1.555166in}{0.783962in}}%
\pgfpathlineto{\pgfqpoint{1.598270in}{0.787599in}}%
\pgfpathlineto{\pgfqpoint{1.609046in}{0.790056in}}%
\pgfpathlineto{\pgfqpoint{1.652151in}{0.794091in}}%
\pgfpathlineto{\pgfqpoint{1.695255in}{0.796632in}}%
\pgfpathlineto{\pgfqpoint{1.706031in}{0.798622in}}%
\pgfpathlineto{\pgfqpoint{1.716807in}{0.798847in}}%
\pgfpathlineto{\pgfqpoint{1.727583in}{0.801134in}}%
\pgfpathlineto{\pgfqpoint{1.738359in}{0.800872in}}%
\pgfpathlineto{\pgfqpoint{1.781464in}{0.805924in}}%
\pgfpathlineto{\pgfqpoint{1.803016in}{0.810744in}}%
\pgfpathlineto{\pgfqpoint{1.813792in}{0.811888in}}%
\pgfpathlineto{\pgfqpoint{1.824568in}{0.814581in}}%
\pgfpathlineto{\pgfqpoint{1.856896in}{0.816209in}}%
\pgfpathlineto{\pgfqpoint{1.878448in}{0.820848in}}%
\pgfpathlineto{\pgfqpoint{1.900001in}{0.823294in}}%
\pgfpathlineto{\pgfqpoint{1.932329in}{0.827732in}}%
\pgfpathlineto{\pgfqpoint{1.943105in}{0.829559in}}%
\pgfpathlineto{\pgfqpoint{1.964657in}{0.829768in}}%
\pgfpathlineto{\pgfqpoint{1.986209in}{0.833382in}}%
\pgfpathlineto{\pgfqpoint{1.996985in}{0.834978in}}%
\pgfpathlineto{\pgfqpoint{2.007761in}{0.834941in}}%
\pgfpathlineto{\pgfqpoint{2.018537in}{0.833492in}}%
\pgfpathlineto{\pgfqpoint{2.029313in}{0.833624in}}%
\pgfpathlineto{\pgfqpoint{2.040090in}{0.834924in}}%
\pgfpathlineto{\pgfqpoint{2.061642in}{0.839785in}}%
\pgfpathlineto{\pgfqpoint{2.072418in}{0.840907in}}%
\pgfpathlineto{\pgfqpoint{2.083194in}{0.843804in}}%
\pgfpathlineto{\pgfqpoint{2.104746in}{0.842692in}}%
\pgfpathlineto{\pgfqpoint{2.126298in}{0.846214in}}%
\pgfpathlineto{\pgfqpoint{2.137074in}{0.846551in}}%
\pgfpathlineto{\pgfqpoint{2.158626in}{0.850763in}}%
\pgfpathlineto{\pgfqpoint{2.169403in}{0.848137in}}%
\pgfpathlineto{\pgfqpoint{2.180179in}{0.849373in}}%
\pgfpathlineto{\pgfqpoint{2.201731in}{0.854720in}}%
\pgfpathlineto{\pgfqpoint{2.212507in}{0.853978in}}%
\pgfpathlineto{\pgfqpoint{2.234059in}{0.858593in}}%
\pgfpathlineto{\pgfqpoint{2.255611in}{0.863278in}}%
\pgfpathlineto{\pgfqpoint{2.266387in}{0.867366in}}%
\pgfpathlineto{\pgfqpoint{2.277163in}{0.867316in}}%
\pgfpathlineto{\pgfqpoint{2.287939in}{0.869663in}}%
\pgfpathlineto{\pgfqpoint{2.298715in}{0.870066in}}%
\pgfpathlineto{\pgfqpoint{2.320268in}{0.875397in}}%
\pgfpathlineto{\pgfqpoint{2.363372in}{0.879895in}}%
\pgfpathlineto{\pgfqpoint{2.374148in}{0.883232in}}%
\pgfpathlineto{\pgfqpoint{2.384924in}{0.883431in}}%
\pgfpathlineto{\pgfqpoint{2.406476in}{0.887375in}}%
\pgfpathlineto{\pgfqpoint{2.417252in}{0.886063in}}%
\pgfpathlineto{\pgfqpoint{2.438805in}{0.894573in}}%
\pgfpathlineto{\pgfqpoint{2.471133in}{0.902837in}}%
\pgfpathlineto{\pgfqpoint{2.481909in}{0.903367in}}%
\pgfpathlineto{\pgfqpoint{2.492685in}{0.906807in}}%
\pgfpathlineto{\pgfqpoint{2.514237in}{0.911415in}}%
\pgfpathlineto{\pgfqpoint{2.535789in}{0.918081in}}%
\pgfpathlineto{\pgfqpoint{2.578894in}{0.920804in}}%
\pgfpathlineto{\pgfqpoint{2.600446in}{0.924825in}}%
\pgfpathlineto{\pgfqpoint{2.611222in}{0.923940in}}%
\pgfpathlineto{\pgfqpoint{2.621998in}{0.924734in}}%
\pgfpathlineto{\pgfqpoint{2.643550in}{0.933076in}}%
\pgfpathlineto{\pgfqpoint{2.665102in}{0.936456in}}%
\pgfpathlineto{\pgfqpoint{2.708206in}{0.951505in}}%
\pgfpathlineto{\pgfqpoint{2.718983in}{0.954560in}}%
\pgfpathlineto{\pgfqpoint{2.751311in}{0.957839in}}%
\pgfpathlineto{\pgfqpoint{2.762087in}{0.958957in}}%
\pgfpathlineto{\pgfqpoint{2.772863in}{0.962298in}}%
\pgfpathlineto{\pgfqpoint{2.783639in}{0.968392in}}%
\pgfpathlineto{\pgfqpoint{2.815967in}{0.971976in}}%
\pgfpathlineto{\pgfqpoint{2.826743in}{0.978034in}}%
\pgfpathlineto{\pgfqpoint{2.837519in}{0.982129in}}%
\pgfpathlineto{\pgfqpoint{2.848296in}{0.992437in}}%
\pgfpathlineto{\pgfqpoint{2.859072in}{0.993557in}}%
\pgfpathlineto{\pgfqpoint{2.869848in}{0.993386in}}%
\pgfpathlineto{\pgfqpoint{2.880624in}{0.996958in}}%
\pgfpathlineto{\pgfqpoint{2.891400in}{0.998609in}}%
\pgfpathlineto{\pgfqpoint{2.902176in}{1.002087in}}%
\pgfpathlineto{\pgfqpoint{2.912952in}{1.007982in}}%
\pgfpathlineto{\pgfqpoint{2.923728in}{1.009998in}}%
\pgfpathlineto{\pgfqpoint{2.934504in}{1.010274in}}%
\pgfpathlineto{\pgfqpoint{2.956056in}{1.023636in}}%
\pgfpathlineto{\pgfqpoint{2.966832in}{1.026210in}}%
\pgfpathlineto{\pgfqpoint{2.977608in}{1.024199in}}%
\pgfpathlineto{\pgfqpoint{2.999161in}{1.035086in}}%
\pgfpathlineto{\pgfqpoint{3.031489in}{1.034480in}}%
\pgfpathlineto{\pgfqpoint{3.042265in}{1.037960in}}%
\pgfpathlineto{\pgfqpoint{3.053041in}{1.043183in}}%
\pgfpathlineto{\pgfqpoint{3.063817in}{1.046678in}}%
\pgfpathlineto{\pgfqpoint{3.074593in}{1.043275in}}%
\pgfpathlineto{\pgfqpoint{3.085369in}{1.046106in}}%
\pgfpathlineto{\pgfqpoint{3.096145in}{1.053960in}}%
\pgfpathlineto{\pgfqpoint{3.106921in}{1.052297in}}%
\pgfpathlineto{\pgfqpoint{3.128474in}{1.058325in}}%
\pgfpathlineto{\pgfqpoint{3.139250in}{1.064254in}}%
\pgfpathlineto{\pgfqpoint{3.150026in}{1.063106in}}%
\pgfpathlineto{\pgfqpoint{3.160802in}{1.071593in}}%
\pgfpathlineto{\pgfqpoint{3.171578in}{1.069231in}}%
\pgfpathlineto{\pgfqpoint{3.182354in}{1.074003in}}%
\pgfpathlineto{\pgfqpoint{3.193130in}{1.074507in}}%
\pgfpathlineto{\pgfqpoint{3.203906in}{1.069946in}}%
\pgfpathlineto{\pgfqpoint{3.214682in}{1.070132in}}%
\pgfpathlineto{\pgfqpoint{3.225458in}{1.067481in}}%
\pgfpathlineto{\pgfqpoint{3.236234in}{1.066681in}}%
\pgfpathlineto{\pgfqpoint{3.247010in}{1.064650in}}%
\pgfpathlineto{\pgfqpoint{3.268563in}{1.071212in}}%
\pgfpathlineto{\pgfqpoint{3.279339in}{1.082503in}}%
\pgfpathlineto{\pgfqpoint{3.300891in}{1.078946in}}%
\pgfpathlineto{\pgfqpoint{3.311667in}{1.077721in}}%
\pgfpathlineto{\pgfqpoint{3.322443in}{1.089736in}}%
\pgfpathlineto{\pgfqpoint{3.333219in}{1.094168in}}%
\pgfpathlineto{\pgfqpoint{3.343995in}{1.094977in}}%
\pgfpathlineto{\pgfqpoint{3.354771in}{1.097854in}}%
\pgfpathlineto{\pgfqpoint{3.365547in}{1.109113in}}%
\pgfpathlineto{\pgfqpoint{3.376323in}{1.107801in}}%
\pgfpathlineto{\pgfqpoint{3.387099in}{1.121741in}}%
\pgfpathlineto{\pgfqpoint{3.397876in}{1.116332in}}%
\pgfpathlineto{\pgfqpoint{3.408652in}{1.116221in}}%
\pgfpathlineto{\pgfqpoint{3.419428in}{1.113226in}}%
\pgfpathlineto{\pgfqpoint{3.430204in}{1.127998in}}%
\pgfpathlineto{\pgfqpoint{3.440980in}{1.125098in}}%
\pgfpathlineto{\pgfqpoint{3.451756in}{1.127794in}}%
\pgfpathlineto{\pgfqpoint{3.462532in}{1.133785in}}%
\pgfpathlineto{\pgfqpoint{3.473308in}{1.136142in}}%
\pgfpathlineto{\pgfqpoint{3.484084in}{1.130310in}}%
\pgfpathlineto{\pgfqpoint{3.505636in}{1.142313in}}%
\pgfpathlineto{\pgfqpoint{3.516412in}{1.165141in}}%
\pgfpathlineto{\pgfqpoint{3.527189in}{1.173390in}}%
\pgfpathlineto{\pgfqpoint{3.537965in}{1.187266in}}%
\pgfpathlineto{\pgfqpoint{3.559517in}{1.180307in}}%
\pgfpathlineto{\pgfqpoint{3.570293in}{1.192925in}}%
\pgfpathlineto{\pgfqpoint{3.581069in}{1.191918in}}%
\pgfpathlineto{\pgfqpoint{3.591845in}{1.205247in}}%
\pgfpathlineto{\pgfqpoint{3.602621in}{1.224777in}}%
\pgfpathlineto{\pgfqpoint{3.613397in}{1.224237in}}%
\pgfpathlineto{\pgfqpoint{3.624173in}{1.228807in}}%
\pgfpathlineto{\pgfqpoint{3.634949in}{1.250802in}}%
\pgfpathlineto{\pgfqpoint{3.645725in}{1.256105in}}%
\pgfpathlineto{\pgfqpoint{3.656501in}{1.273151in}}%
\pgfpathlineto{\pgfqpoint{3.667278in}{1.287512in}}%
\pgfpathlineto{\pgfqpoint{3.678054in}{1.275054in}}%
\pgfpathlineto{\pgfqpoint{3.688830in}{1.287907in}}%
\pgfpathlineto{\pgfqpoint{3.699606in}{1.331745in}}%
\pgfpathlineto{\pgfqpoint{3.710382in}{1.346937in}}%
\pgfpathlineto{\pgfqpoint{3.710382in}{1.346937in}}%
\pgfusepath{stroke}%
\end{pgfscope}%
\begin{pgfscope}%
\pgfpathrectangle{\pgfqpoint{0.488334in}{0.417391in}}{\pgfqpoint{3.222048in}{2.055572in}} %
\pgfusepath{clip}%
\pgfsetrectcap%
\pgfsetroundjoin%
\pgfsetlinewidth{1.003750pt}%
\definecolor{currentstroke}{rgb}{0.000000,0.500000,0.000000}%
\pgfsetstrokecolor{currentstroke}%
\pgfsetdash{}{0pt}%
\pgfpathmoveto{\pgfqpoint{0.488334in}{0.656353in}}%
\pgfpathlineto{\pgfqpoint{0.585319in}{0.657243in}}%
\pgfpathlineto{\pgfqpoint{0.682304in}{0.659890in}}%
\pgfpathlineto{\pgfqpoint{0.714632in}{0.659286in}}%
\pgfpathlineto{\pgfqpoint{0.800840in}{0.662485in}}%
\pgfpathlineto{\pgfqpoint{0.833169in}{0.662103in}}%
\pgfpathlineto{\pgfqpoint{0.887049in}{0.663391in}}%
\pgfpathlineto{\pgfqpoint{0.951706in}{0.664753in}}%
\pgfpathlineto{\pgfqpoint{0.973258in}{0.665038in}}%
\pgfpathlineto{\pgfqpoint{0.994810in}{0.666490in}}%
\pgfpathlineto{\pgfqpoint{1.037914in}{0.667851in}}%
\pgfpathlineto{\pgfqpoint{1.081019in}{0.668373in}}%
\pgfpathlineto{\pgfqpoint{1.264212in}{0.668287in}}%
\pgfpathlineto{\pgfqpoint{1.339644in}{0.669210in}}%
\pgfpathlineto{\pgfqpoint{1.371973in}{0.668174in}}%
\pgfpathlineto{\pgfqpoint{1.415077in}{0.669802in}}%
\pgfpathlineto{\pgfqpoint{1.447405in}{0.670043in}}%
\pgfpathlineto{\pgfqpoint{1.479733in}{0.667151in}}%
\pgfpathlineto{\pgfqpoint{1.576718in}{0.666422in}}%
\pgfpathlineto{\pgfqpoint{1.587494in}{0.667820in}}%
\pgfpathlineto{\pgfqpoint{1.598270in}{0.666673in}}%
\pgfpathlineto{\pgfqpoint{1.609046in}{0.667792in}}%
\pgfpathlineto{\pgfqpoint{1.630599in}{0.668188in}}%
\pgfpathlineto{\pgfqpoint{1.662927in}{0.667604in}}%
\pgfpathlineto{\pgfqpoint{1.684479in}{0.666711in}}%
\pgfpathlineto{\pgfqpoint{1.695255in}{0.667858in}}%
\pgfpathlineto{\pgfqpoint{1.716807in}{0.667723in}}%
\pgfpathlineto{\pgfqpoint{1.749135in}{0.668836in}}%
\pgfpathlineto{\pgfqpoint{1.781464in}{0.669598in}}%
\pgfpathlineto{\pgfqpoint{1.792240in}{0.671478in}}%
\pgfpathlineto{\pgfqpoint{1.921553in}{0.675050in}}%
\pgfpathlineto{\pgfqpoint{1.943105in}{0.675167in}}%
\pgfpathlineto{\pgfqpoint{1.986209in}{0.675057in}}%
\pgfpathlineto{\pgfqpoint{2.007761in}{0.675352in}}%
\pgfpathlineto{\pgfqpoint{2.018537in}{0.674639in}}%
\pgfpathlineto{\pgfqpoint{2.040090in}{0.677036in}}%
\pgfpathlineto{\pgfqpoint{2.072418in}{0.678095in}}%
\pgfpathlineto{\pgfqpoint{2.083194in}{0.679625in}}%
\pgfpathlineto{\pgfqpoint{2.190955in}{0.678417in}}%
\pgfpathlineto{\pgfqpoint{2.234059in}{0.678083in}}%
\pgfpathlineto{\pgfqpoint{2.287939in}{0.678761in}}%
\pgfpathlineto{\pgfqpoint{2.395700in}{0.675892in}}%
\pgfpathlineto{\pgfqpoint{2.406476in}{0.677242in}}%
\pgfpathlineto{\pgfqpoint{2.417252in}{0.676408in}}%
\pgfpathlineto{\pgfqpoint{2.428028in}{0.678001in}}%
\pgfpathlineto{\pgfqpoint{2.438805in}{0.676879in}}%
\pgfpathlineto{\pgfqpoint{2.449581in}{0.677826in}}%
\pgfpathlineto{\pgfqpoint{2.481909in}{0.676872in}}%
\pgfpathlineto{\pgfqpoint{2.514237in}{0.678791in}}%
\pgfpathlineto{\pgfqpoint{2.525013in}{0.679149in}}%
\pgfpathlineto{\pgfqpoint{2.546565in}{0.677563in}}%
\pgfpathlineto{\pgfqpoint{2.557341in}{0.675285in}}%
\pgfpathlineto{\pgfqpoint{2.600446in}{0.676210in}}%
\pgfpathlineto{\pgfqpoint{2.611222in}{0.676155in}}%
\pgfpathlineto{\pgfqpoint{2.621998in}{0.674463in}}%
\pgfpathlineto{\pgfqpoint{2.632774in}{0.674042in}}%
\pgfpathlineto{\pgfqpoint{2.643550in}{0.675505in}}%
\pgfpathlineto{\pgfqpoint{2.729759in}{0.677038in}}%
\pgfpathlineto{\pgfqpoint{2.740535in}{0.675747in}}%
\pgfpathlineto{\pgfqpoint{2.751311in}{0.676859in}}%
\pgfpathlineto{\pgfqpoint{2.762087in}{0.675046in}}%
\pgfpathlineto{\pgfqpoint{2.772863in}{0.676647in}}%
\pgfpathlineto{\pgfqpoint{2.783639in}{0.676323in}}%
\pgfpathlineto{\pgfqpoint{2.794415in}{0.674649in}}%
\pgfpathlineto{\pgfqpoint{2.805191in}{0.676095in}}%
\pgfpathlineto{\pgfqpoint{2.815967in}{0.674879in}}%
\pgfpathlineto{\pgfqpoint{2.826743in}{0.676598in}}%
\pgfpathlineto{\pgfqpoint{2.837519in}{0.675062in}}%
\pgfpathlineto{\pgfqpoint{2.848296in}{0.678368in}}%
\pgfpathlineto{\pgfqpoint{2.859072in}{0.675849in}}%
\pgfpathlineto{\pgfqpoint{2.880624in}{0.675699in}}%
\pgfpathlineto{\pgfqpoint{2.891400in}{0.677183in}}%
\pgfpathlineto{\pgfqpoint{2.912952in}{0.677054in}}%
\pgfpathlineto{\pgfqpoint{2.923728in}{0.678627in}}%
\pgfpathlineto{\pgfqpoint{2.934504in}{0.677246in}}%
\pgfpathlineto{\pgfqpoint{2.956056in}{0.679852in}}%
\pgfpathlineto{\pgfqpoint{2.966832in}{0.678460in}}%
\pgfpathlineto{\pgfqpoint{2.977608in}{0.675596in}}%
\pgfpathlineto{\pgfqpoint{2.988385in}{0.676838in}}%
\pgfpathlineto{\pgfqpoint{3.009937in}{0.673391in}}%
\pgfpathlineto{\pgfqpoint{3.020713in}{0.673526in}}%
\pgfpathlineto{\pgfqpoint{3.031489in}{0.670992in}}%
\pgfpathlineto{\pgfqpoint{3.063817in}{0.674272in}}%
\pgfpathlineto{\pgfqpoint{3.074593in}{0.672769in}}%
\pgfpathlineto{\pgfqpoint{3.085369in}{0.672851in}}%
\pgfpathlineto{\pgfqpoint{3.106921in}{0.666703in}}%
\pgfpathlineto{\pgfqpoint{3.117697in}{0.669031in}}%
\pgfpathlineto{\pgfqpoint{3.139250in}{0.665055in}}%
\pgfpathlineto{\pgfqpoint{3.150026in}{0.661653in}}%
\pgfpathlineto{\pgfqpoint{3.160802in}{0.661742in}}%
\pgfpathlineto{\pgfqpoint{3.182354in}{0.659256in}}%
\pgfpathlineto{\pgfqpoint{3.193130in}{0.660203in}}%
\pgfpathlineto{\pgfqpoint{3.203906in}{0.656392in}}%
\pgfpathlineto{\pgfqpoint{3.214682in}{0.656695in}}%
\pgfpathlineto{\pgfqpoint{3.225458in}{0.654667in}}%
\pgfpathlineto{\pgfqpoint{3.236234in}{0.656125in}}%
\pgfpathlineto{\pgfqpoint{3.247010in}{0.650856in}}%
\pgfpathlineto{\pgfqpoint{3.268563in}{0.649369in}}%
\pgfpathlineto{\pgfqpoint{3.279339in}{0.648048in}}%
\pgfpathlineto{\pgfqpoint{3.300891in}{0.650411in}}%
\pgfpathlineto{\pgfqpoint{3.311667in}{0.648311in}}%
\pgfpathlineto{\pgfqpoint{3.322443in}{0.647891in}}%
\pgfpathlineto{\pgfqpoint{3.333219in}{0.648601in}}%
\pgfpathlineto{\pgfqpoint{3.343995in}{0.642165in}}%
\pgfpathlineto{\pgfqpoint{3.354771in}{0.646424in}}%
\pgfpathlineto{\pgfqpoint{3.365547in}{0.654436in}}%
\pgfpathlineto{\pgfqpoint{3.376323in}{0.653336in}}%
\pgfpathlineto{\pgfqpoint{3.387099in}{0.661657in}}%
\pgfpathlineto{\pgfqpoint{3.397876in}{0.657203in}}%
\pgfpathlineto{\pgfqpoint{3.408652in}{0.648337in}}%
\pgfpathlineto{\pgfqpoint{3.430204in}{0.641159in}}%
\pgfpathlineto{\pgfqpoint{3.440980in}{0.635159in}}%
\pgfpathlineto{\pgfqpoint{3.462532in}{0.638302in}}%
\pgfpathlineto{\pgfqpoint{3.473308in}{0.636691in}}%
\pgfpathlineto{\pgfqpoint{3.484084in}{0.631835in}}%
\pgfpathlineto{\pgfqpoint{3.516412in}{0.627906in}}%
\pgfpathlineto{\pgfqpoint{3.527189in}{0.622025in}}%
\pgfpathlineto{\pgfqpoint{3.537965in}{0.622598in}}%
\pgfpathlineto{\pgfqpoint{3.548741in}{0.629971in}}%
\pgfpathlineto{\pgfqpoint{3.559517in}{0.623576in}}%
\pgfpathlineto{\pgfqpoint{3.570293in}{0.631674in}}%
\pgfpathlineto{\pgfqpoint{3.581069in}{0.627723in}}%
\pgfpathlineto{\pgfqpoint{3.591845in}{0.635090in}}%
\pgfpathlineto{\pgfqpoint{3.602621in}{0.634956in}}%
\pgfpathlineto{\pgfqpoint{3.613397in}{0.639109in}}%
\pgfpathlineto{\pgfqpoint{3.624173in}{0.645226in}}%
\pgfpathlineto{\pgfqpoint{3.634949in}{0.659147in}}%
\pgfpathlineto{\pgfqpoint{3.645725in}{0.654670in}}%
\pgfpathlineto{\pgfqpoint{3.656501in}{0.662378in}}%
\pgfpathlineto{\pgfqpoint{3.667278in}{0.664149in}}%
\pgfpathlineto{\pgfqpoint{3.678054in}{0.658786in}}%
\pgfpathlineto{\pgfqpoint{3.688830in}{0.639463in}}%
\pgfpathlineto{\pgfqpoint{3.699606in}{0.658009in}}%
\pgfpathlineto{\pgfqpoint{3.710382in}{0.651155in}}%
\pgfpathlineto{\pgfqpoint{3.710382in}{0.651155in}}%
\pgfusepath{stroke}%
\end{pgfscope}%
\begin{pgfscope}%
\pgfsetrectcap%
\pgfsetmiterjoin%
\pgfsetlinewidth{1.003750pt}%
\definecolor{currentstroke}{rgb}{0.000000,0.000000,0.000000}%
\pgfsetstrokecolor{currentstroke}%
\pgfsetdash{}{0pt}%
\pgfpathmoveto{\pgfqpoint{0.488334in}{2.472963in}}%
\pgfpathlineto{\pgfqpoint{3.710382in}{2.472963in}}%
\pgfusepath{stroke}%
\end{pgfscope}%
\begin{pgfscope}%
\pgfsetrectcap%
\pgfsetmiterjoin%
\pgfsetlinewidth{1.003750pt}%
\definecolor{currentstroke}{rgb}{0.000000,0.000000,0.000000}%
\pgfsetstrokecolor{currentstroke}%
\pgfsetdash{}{0pt}%
\pgfpathmoveto{\pgfqpoint{3.710382in}{0.417391in}}%
\pgfpathlineto{\pgfqpoint{3.710382in}{2.472963in}}%
\pgfusepath{stroke}%
\end{pgfscope}%
\begin{pgfscope}%
\pgfsetrectcap%
\pgfsetmiterjoin%
\pgfsetlinewidth{1.003750pt}%
\definecolor{currentstroke}{rgb}{0.000000,0.000000,0.000000}%
\pgfsetstrokecolor{currentstroke}%
\pgfsetdash{}{0pt}%
\pgfpathmoveto{\pgfqpoint{0.488334in}{0.417391in}}%
\pgfpathlineto{\pgfqpoint{3.710382in}{0.417391in}}%
\pgfusepath{stroke}%
\end{pgfscope}%
\begin{pgfscope}%
\pgfsetrectcap%
\pgfsetmiterjoin%
\pgfsetlinewidth{1.003750pt}%
\definecolor{currentstroke}{rgb}{0.000000,0.000000,0.000000}%
\pgfsetstrokecolor{currentstroke}%
\pgfsetdash{}{0pt}%
\pgfpathmoveto{\pgfqpoint{0.488334in}{0.417391in}}%
\pgfpathlineto{\pgfqpoint{0.488334in}{2.472963in}}%
\pgfusepath{stroke}%
\end{pgfscope}%
\begin{pgfscope}%
\pgfsetbuttcap%
\pgfsetroundjoin%
\definecolor{currentfill}{rgb}{0.000000,0.000000,0.000000}%
\pgfsetfillcolor{currentfill}%
\pgfsetlinewidth{0.501875pt}%
\definecolor{currentstroke}{rgb}{0.000000,0.000000,0.000000}%
\pgfsetstrokecolor{currentstroke}%
\pgfsetdash{}{0pt}%
\pgfsys@defobject{currentmarker}{\pgfqpoint{0.000000in}{0.000000in}}{\pgfqpoint{0.000000in}{0.069444in}}{%
\pgfpathmoveto{\pgfqpoint{0.000000in}{0.000000in}}%
\pgfpathlineto{\pgfqpoint{0.000000in}{0.069444in}}%
\pgfusepath{stroke,fill}%
}%
\begin{pgfscope}%
\pgfsys@transformshift{0.488334in}{0.417391in}%
\pgfsys@useobject{currentmarker}{}%
\end{pgfscope}%
\end{pgfscope}%
\begin{pgfscope}%
\pgfsetbuttcap%
\pgfsetroundjoin%
\definecolor{currentfill}{rgb}{0.000000,0.000000,0.000000}%
\pgfsetfillcolor{currentfill}%
\pgfsetlinewidth{0.501875pt}%
\definecolor{currentstroke}{rgb}{0.000000,0.000000,0.000000}%
\pgfsetstrokecolor{currentstroke}%
\pgfsetdash{}{0pt}%
\pgfsys@defobject{currentmarker}{\pgfqpoint{0.000000in}{-0.069444in}}{\pgfqpoint{0.000000in}{0.000000in}}{%
\pgfpathmoveto{\pgfqpoint{0.000000in}{0.000000in}}%
\pgfpathlineto{\pgfqpoint{0.000000in}{-0.069444in}}%
\pgfusepath{stroke,fill}%
}%
\begin{pgfscope}%
\pgfsys@transformshift{0.488334in}{2.472963in}%
\pgfsys@useobject{currentmarker}{}%
\end{pgfscope}%
\end{pgfscope}%
\begin{pgfscope}%
\pgftext[x=0.488334in,y=0.347947in,,top]{\rmfamily\fontsize{8.000000}{9.600000}\selectfont −2}%
\end{pgfscope}%
\begin{pgfscope}%
\pgfsetbuttcap%
\pgfsetroundjoin%
\definecolor{currentfill}{rgb}{0.000000,0.000000,0.000000}%
\pgfsetfillcolor{currentfill}%
\pgfsetlinewidth{0.501875pt}%
\definecolor{currentstroke}{rgb}{0.000000,0.000000,0.000000}%
\pgfsetstrokecolor{currentstroke}%
\pgfsetdash{}{0pt}%
\pgfsys@defobject{currentmarker}{\pgfqpoint{0.000000in}{0.000000in}}{\pgfqpoint{0.000000in}{0.069444in}}{%
\pgfpathmoveto{\pgfqpoint{0.000000in}{0.000000in}}%
\pgfpathlineto{\pgfqpoint{0.000000in}{0.069444in}}%
\pgfusepath{stroke,fill}%
}%
\begin{pgfscope}%
\pgfsys@transformshift{0.984034in}{0.417391in}%
\pgfsys@useobject{currentmarker}{}%
\end{pgfscope}%
\end{pgfscope}%
\begin{pgfscope}%
\pgfsetbuttcap%
\pgfsetroundjoin%
\definecolor{currentfill}{rgb}{0.000000,0.000000,0.000000}%
\pgfsetfillcolor{currentfill}%
\pgfsetlinewidth{0.501875pt}%
\definecolor{currentstroke}{rgb}{0.000000,0.000000,0.000000}%
\pgfsetstrokecolor{currentstroke}%
\pgfsetdash{}{0pt}%
\pgfsys@defobject{currentmarker}{\pgfqpoint{0.000000in}{-0.069444in}}{\pgfqpoint{0.000000in}{0.000000in}}{%
\pgfpathmoveto{\pgfqpoint{0.000000in}{0.000000in}}%
\pgfpathlineto{\pgfqpoint{0.000000in}{-0.069444in}}%
\pgfusepath{stroke,fill}%
}%
\begin{pgfscope}%
\pgfsys@transformshift{0.984034in}{2.472963in}%
\pgfsys@useobject{currentmarker}{}%
\end{pgfscope}%
\end{pgfscope}%
\begin{pgfscope}%
\pgftext[x=0.984034in,y=0.347947in,,top]{\rmfamily\fontsize{8.000000}{9.600000}\selectfont −1}%
\end{pgfscope}%
\begin{pgfscope}%
\pgfsetbuttcap%
\pgfsetroundjoin%
\definecolor{currentfill}{rgb}{0.000000,0.000000,0.000000}%
\pgfsetfillcolor{currentfill}%
\pgfsetlinewidth{0.501875pt}%
\definecolor{currentstroke}{rgb}{0.000000,0.000000,0.000000}%
\pgfsetstrokecolor{currentstroke}%
\pgfsetdash{}{0pt}%
\pgfsys@defobject{currentmarker}{\pgfqpoint{0.000000in}{0.000000in}}{\pgfqpoint{0.000000in}{0.069444in}}{%
\pgfpathmoveto{\pgfqpoint{0.000000in}{0.000000in}}%
\pgfpathlineto{\pgfqpoint{0.000000in}{0.069444in}}%
\pgfusepath{stroke,fill}%
}%
\begin{pgfscope}%
\pgfsys@transformshift{1.479733in}{0.417391in}%
\pgfsys@useobject{currentmarker}{}%
\end{pgfscope}%
\end{pgfscope}%
\begin{pgfscope}%
\pgfsetbuttcap%
\pgfsetroundjoin%
\definecolor{currentfill}{rgb}{0.000000,0.000000,0.000000}%
\pgfsetfillcolor{currentfill}%
\pgfsetlinewidth{0.501875pt}%
\definecolor{currentstroke}{rgb}{0.000000,0.000000,0.000000}%
\pgfsetstrokecolor{currentstroke}%
\pgfsetdash{}{0pt}%
\pgfsys@defobject{currentmarker}{\pgfqpoint{0.000000in}{-0.069444in}}{\pgfqpoint{0.000000in}{0.000000in}}{%
\pgfpathmoveto{\pgfqpoint{0.000000in}{0.000000in}}%
\pgfpathlineto{\pgfqpoint{0.000000in}{-0.069444in}}%
\pgfusepath{stroke,fill}%
}%
\begin{pgfscope}%
\pgfsys@transformshift{1.479733in}{2.472963in}%
\pgfsys@useobject{currentmarker}{}%
\end{pgfscope}%
\end{pgfscope}%
\begin{pgfscope}%
\pgftext[x=1.479733in,y=0.347947in,,top]{\rmfamily\fontsize{8.000000}{9.600000}\selectfont 0}%
\end{pgfscope}%
\begin{pgfscope}%
\pgfsetbuttcap%
\pgfsetroundjoin%
\definecolor{currentfill}{rgb}{0.000000,0.000000,0.000000}%
\pgfsetfillcolor{currentfill}%
\pgfsetlinewidth{0.501875pt}%
\definecolor{currentstroke}{rgb}{0.000000,0.000000,0.000000}%
\pgfsetstrokecolor{currentstroke}%
\pgfsetdash{}{0pt}%
\pgfsys@defobject{currentmarker}{\pgfqpoint{0.000000in}{0.000000in}}{\pgfqpoint{0.000000in}{0.069444in}}{%
\pgfpathmoveto{\pgfqpoint{0.000000in}{0.000000in}}%
\pgfpathlineto{\pgfqpoint{0.000000in}{0.069444in}}%
\pgfusepath{stroke,fill}%
}%
\begin{pgfscope}%
\pgfsys@transformshift{1.975433in}{0.417391in}%
\pgfsys@useobject{currentmarker}{}%
\end{pgfscope}%
\end{pgfscope}%
\begin{pgfscope}%
\pgfsetbuttcap%
\pgfsetroundjoin%
\definecolor{currentfill}{rgb}{0.000000,0.000000,0.000000}%
\pgfsetfillcolor{currentfill}%
\pgfsetlinewidth{0.501875pt}%
\definecolor{currentstroke}{rgb}{0.000000,0.000000,0.000000}%
\pgfsetstrokecolor{currentstroke}%
\pgfsetdash{}{0pt}%
\pgfsys@defobject{currentmarker}{\pgfqpoint{0.000000in}{-0.069444in}}{\pgfqpoint{0.000000in}{0.000000in}}{%
\pgfpathmoveto{\pgfqpoint{0.000000in}{0.000000in}}%
\pgfpathlineto{\pgfqpoint{0.000000in}{-0.069444in}}%
\pgfusepath{stroke,fill}%
}%
\begin{pgfscope}%
\pgfsys@transformshift{1.975433in}{2.472963in}%
\pgfsys@useobject{currentmarker}{}%
\end{pgfscope}%
\end{pgfscope}%
\begin{pgfscope}%
\pgftext[x=1.975433in,y=0.347947in,,top]{\rmfamily\fontsize{8.000000}{9.600000}\selectfont 1}%
\end{pgfscope}%
\begin{pgfscope}%
\pgfsetbuttcap%
\pgfsetroundjoin%
\definecolor{currentfill}{rgb}{0.000000,0.000000,0.000000}%
\pgfsetfillcolor{currentfill}%
\pgfsetlinewidth{0.501875pt}%
\definecolor{currentstroke}{rgb}{0.000000,0.000000,0.000000}%
\pgfsetstrokecolor{currentstroke}%
\pgfsetdash{}{0pt}%
\pgfsys@defobject{currentmarker}{\pgfqpoint{0.000000in}{0.000000in}}{\pgfqpoint{0.000000in}{0.069444in}}{%
\pgfpathmoveto{\pgfqpoint{0.000000in}{0.000000in}}%
\pgfpathlineto{\pgfqpoint{0.000000in}{0.069444in}}%
\pgfusepath{stroke,fill}%
}%
\begin{pgfscope}%
\pgfsys@transformshift{2.471133in}{0.417391in}%
\pgfsys@useobject{currentmarker}{}%
\end{pgfscope}%
\end{pgfscope}%
\begin{pgfscope}%
\pgfsetbuttcap%
\pgfsetroundjoin%
\definecolor{currentfill}{rgb}{0.000000,0.000000,0.000000}%
\pgfsetfillcolor{currentfill}%
\pgfsetlinewidth{0.501875pt}%
\definecolor{currentstroke}{rgb}{0.000000,0.000000,0.000000}%
\pgfsetstrokecolor{currentstroke}%
\pgfsetdash{}{0pt}%
\pgfsys@defobject{currentmarker}{\pgfqpoint{0.000000in}{-0.069444in}}{\pgfqpoint{0.000000in}{0.000000in}}{%
\pgfpathmoveto{\pgfqpoint{0.000000in}{0.000000in}}%
\pgfpathlineto{\pgfqpoint{0.000000in}{-0.069444in}}%
\pgfusepath{stroke,fill}%
}%
\begin{pgfscope}%
\pgfsys@transformshift{2.471133in}{2.472963in}%
\pgfsys@useobject{currentmarker}{}%
\end{pgfscope}%
\end{pgfscope}%
\begin{pgfscope}%
\pgftext[x=2.471133in,y=0.347947in,,top]{\rmfamily\fontsize{8.000000}{9.600000}\selectfont 2}%
\end{pgfscope}%
\begin{pgfscope}%
\pgfsetbuttcap%
\pgfsetroundjoin%
\definecolor{currentfill}{rgb}{0.000000,0.000000,0.000000}%
\pgfsetfillcolor{currentfill}%
\pgfsetlinewidth{0.501875pt}%
\definecolor{currentstroke}{rgb}{0.000000,0.000000,0.000000}%
\pgfsetstrokecolor{currentstroke}%
\pgfsetdash{}{0pt}%
\pgfsys@defobject{currentmarker}{\pgfqpoint{0.000000in}{0.000000in}}{\pgfqpoint{0.000000in}{0.069444in}}{%
\pgfpathmoveto{\pgfqpoint{0.000000in}{0.000000in}}%
\pgfpathlineto{\pgfqpoint{0.000000in}{0.069444in}}%
\pgfusepath{stroke,fill}%
}%
\begin{pgfscope}%
\pgfsys@transformshift{2.966832in}{0.417391in}%
\pgfsys@useobject{currentmarker}{}%
\end{pgfscope}%
\end{pgfscope}%
\begin{pgfscope}%
\pgfsetbuttcap%
\pgfsetroundjoin%
\definecolor{currentfill}{rgb}{0.000000,0.000000,0.000000}%
\pgfsetfillcolor{currentfill}%
\pgfsetlinewidth{0.501875pt}%
\definecolor{currentstroke}{rgb}{0.000000,0.000000,0.000000}%
\pgfsetstrokecolor{currentstroke}%
\pgfsetdash{}{0pt}%
\pgfsys@defobject{currentmarker}{\pgfqpoint{0.000000in}{-0.069444in}}{\pgfqpoint{0.000000in}{0.000000in}}{%
\pgfpathmoveto{\pgfqpoint{0.000000in}{0.000000in}}%
\pgfpathlineto{\pgfqpoint{0.000000in}{-0.069444in}}%
\pgfusepath{stroke,fill}%
}%
\begin{pgfscope}%
\pgfsys@transformshift{2.966832in}{2.472963in}%
\pgfsys@useobject{currentmarker}{}%
\end{pgfscope}%
\end{pgfscope}%
\begin{pgfscope}%
\pgftext[x=2.966832in,y=0.347947in,,top]{\rmfamily\fontsize{8.000000}{9.600000}\selectfont 3}%
\end{pgfscope}%
\begin{pgfscope}%
\pgfsetbuttcap%
\pgfsetroundjoin%
\definecolor{currentfill}{rgb}{0.000000,0.000000,0.000000}%
\pgfsetfillcolor{currentfill}%
\pgfsetlinewidth{0.501875pt}%
\definecolor{currentstroke}{rgb}{0.000000,0.000000,0.000000}%
\pgfsetstrokecolor{currentstroke}%
\pgfsetdash{}{0pt}%
\pgfsys@defobject{currentmarker}{\pgfqpoint{0.000000in}{0.000000in}}{\pgfqpoint{0.000000in}{0.069444in}}{%
\pgfpathmoveto{\pgfqpoint{0.000000in}{0.000000in}}%
\pgfpathlineto{\pgfqpoint{0.000000in}{0.069444in}}%
\pgfusepath{stroke,fill}%
}%
\begin{pgfscope}%
\pgfsys@transformshift{3.462532in}{0.417391in}%
\pgfsys@useobject{currentmarker}{}%
\end{pgfscope}%
\end{pgfscope}%
\begin{pgfscope}%
\pgfsetbuttcap%
\pgfsetroundjoin%
\definecolor{currentfill}{rgb}{0.000000,0.000000,0.000000}%
\pgfsetfillcolor{currentfill}%
\pgfsetlinewidth{0.501875pt}%
\definecolor{currentstroke}{rgb}{0.000000,0.000000,0.000000}%
\pgfsetstrokecolor{currentstroke}%
\pgfsetdash{}{0pt}%
\pgfsys@defobject{currentmarker}{\pgfqpoint{0.000000in}{-0.069444in}}{\pgfqpoint{0.000000in}{0.000000in}}{%
\pgfpathmoveto{\pgfqpoint{0.000000in}{0.000000in}}%
\pgfpathlineto{\pgfqpoint{0.000000in}{-0.069444in}}%
\pgfusepath{stroke,fill}%
}%
\begin{pgfscope}%
\pgfsys@transformshift{3.462532in}{2.472963in}%
\pgfsys@useobject{currentmarker}{}%
\end{pgfscope}%
\end{pgfscope}%
\begin{pgfscope}%
\pgftext[x=3.462532in,y=0.347947in,,top]{\rmfamily\fontsize{8.000000}{9.600000}\selectfont 4}%
\end{pgfscope}%
\begin{pgfscope}%
\pgftext[x=2.099358in,y=0.170972in,,top]{\rmfamily\fontsize{9.000000}{10.800000}\selectfont Classifier cut}%
\end{pgfscope}%
\begin{pgfscope}%
\pgfsetbuttcap%
\pgfsetroundjoin%
\definecolor{currentfill}{rgb}{0.000000,0.000000,0.000000}%
\pgfsetfillcolor{currentfill}%
\pgfsetlinewidth{0.501875pt}%
\definecolor{currentstroke}{rgb}{0.000000,0.000000,0.000000}%
\pgfsetstrokecolor{currentstroke}%
\pgfsetdash{}{0pt}%
\pgfsys@defobject{currentmarker}{\pgfqpoint{0.000000in}{0.000000in}}{\pgfqpoint{0.069444in}{0.000000in}}{%
\pgfpathmoveto{\pgfqpoint{0.000000in}{0.000000in}}%
\pgfpathlineto{\pgfqpoint{0.069444in}{0.000000in}}%
\pgfusepath{stroke,fill}%
}%
\begin{pgfscope}%
\pgfsys@transformshift{0.488334in}{0.645788in}%
\pgfsys@useobject{currentmarker}{}%
\end{pgfscope}%
\end{pgfscope}%
\begin{pgfscope}%
\pgfsetbuttcap%
\pgfsetroundjoin%
\definecolor{currentfill}{rgb}{0.000000,0.000000,0.000000}%
\pgfsetfillcolor{currentfill}%
\pgfsetlinewidth{0.501875pt}%
\definecolor{currentstroke}{rgb}{0.000000,0.000000,0.000000}%
\pgfsetstrokecolor{currentstroke}%
\pgfsetdash{}{0pt}%
\pgfsys@defobject{currentmarker}{\pgfqpoint{-0.069444in}{0.000000in}}{\pgfqpoint{0.000000in}{0.000000in}}{%
\pgfpathmoveto{\pgfqpoint{0.000000in}{0.000000in}}%
\pgfpathlineto{\pgfqpoint{-0.069444in}{0.000000in}}%
\pgfusepath{stroke,fill}%
}%
\begin{pgfscope}%
\pgfsys@transformshift{3.710382in}{0.645788in}%
\pgfsys@useobject{currentmarker}{}%
\end{pgfscope}%
\end{pgfscope}%
\begin{pgfscope}%
\pgftext[x=0.418890in,y=0.645788in,right,]{\rmfamily\fontsize{8.000000}{9.600000}\selectfont 1.0}%
\end{pgfscope}%
\begin{pgfscope}%
\pgfsetbuttcap%
\pgfsetroundjoin%
\definecolor{currentfill}{rgb}{0.000000,0.000000,0.000000}%
\pgfsetfillcolor{currentfill}%
\pgfsetlinewidth{0.501875pt}%
\definecolor{currentstroke}{rgb}{0.000000,0.000000,0.000000}%
\pgfsetstrokecolor{currentstroke}%
\pgfsetdash{}{0pt}%
\pgfsys@defobject{currentmarker}{\pgfqpoint{0.000000in}{0.000000in}}{\pgfqpoint{0.069444in}{0.000000in}}{%
\pgfpathmoveto{\pgfqpoint{0.000000in}{0.000000in}}%
\pgfpathlineto{\pgfqpoint{0.069444in}{0.000000in}}%
\pgfusepath{stroke,fill}%
}%
\begin{pgfscope}%
\pgfsys@transformshift{0.488334in}{1.102582in}%
\pgfsys@useobject{currentmarker}{}%
\end{pgfscope}%
\end{pgfscope}%
\begin{pgfscope}%
\pgfsetbuttcap%
\pgfsetroundjoin%
\definecolor{currentfill}{rgb}{0.000000,0.000000,0.000000}%
\pgfsetfillcolor{currentfill}%
\pgfsetlinewidth{0.501875pt}%
\definecolor{currentstroke}{rgb}{0.000000,0.000000,0.000000}%
\pgfsetstrokecolor{currentstroke}%
\pgfsetdash{}{0pt}%
\pgfsys@defobject{currentmarker}{\pgfqpoint{-0.069444in}{0.000000in}}{\pgfqpoint{0.000000in}{0.000000in}}{%
\pgfpathmoveto{\pgfqpoint{0.000000in}{0.000000in}}%
\pgfpathlineto{\pgfqpoint{-0.069444in}{0.000000in}}%
\pgfusepath{stroke,fill}%
}%
\begin{pgfscope}%
\pgfsys@transformshift{3.710382in}{1.102582in}%
\pgfsys@useobject{currentmarker}{}%
\end{pgfscope}%
\end{pgfscope}%
\begin{pgfscope}%
\pgftext[x=0.418890in,y=1.102582in,right,]{\rmfamily\fontsize{8.000000}{9.600000}\selectfont 1.1}%
\end{pgfscope}%
\begin{pgfscope}%
\pgfsetbuttcap%
\pgfsetroundjoin%
\definecolor{currentfill}{rgb}{0.000000,0.000000,0.000000}%
\pgfsetfillcolor{currentfill}%
\pgfsetlinewidth{0.501875pt}%
\definecolor{currentstroke}{rgb}{0.000000,0.000000,0.000000}%
\pgfsetstrokecolor{currentstroke}%
\pgfsetdash{}{0pt}%
\pgfsys@defobject{currentmarker}{\pgfqpoint{0.000000in}{0.000000in}}{\pgfqpoint{0.069444in}{0.000000in}}{%
\pgfpathmoveto{\pgfqpoint{0.000000in}{0.000000in}}%
\pgfpathlineto{\pgfqpoint{0.069444in}{0.000000in}}%
\pgfusepath{stroke,fill}%
}%
\begin{pgfscope}%
\pgfsys@transformshift{0.488334in}{1.559375in}%
\pgfsys@useobject{currentmarker}{}%
\end{pgfscope}%
\end{pgfscope}%
\begin{pgfscope}%
\pgfsetbuttcap%
\pgfsetroundjoin%
\definecolor{currentfill}{rgb}{0.000000,0.000000,0.000000}%
\pgfsetfillcolor{currentfill}%
\pgfsetlinewidth{0.501875pt}%
\definecolor{currentstroke}{rgb}{0.000000,0.000000,0.000000}%
\pgfsetstrokecolor{currentstroke}%
\pgfsetdash{}{0pt}%
\pgfsys@defobject{currentmarker}{\pgfqpoint{-0.069444in}{0.000000in}}{\pgfqpoint{0.000000in}{0.000000in}}{%
\pgfpathmoveto{\pgfqpoint{0.000000in}{0.000000in}}%
\pgfpathlineto{\pgfqpoint{-0.069444in}{0.000000in}}%
\pgfusepath{stroke,fill}%
}%
\begin{pgfscope}%
\pgfsys@transformshift{3.710382in}{1.559375in}%
\pgfsys@useobject{currentmarker}{}%
\end{pgfscope}%
\end{pgfscope}%
\begin{pgfscope}%
\pgftext[x=0.418890in,y=1.559375in,right,]{\rmfamily\fontsize{8.000000}{9.600000}\selectfont 1.2}%
\end{pgfscope}%
\begin{pgfscope}%
\pgfsetbuttcap%
\pgfsetroundjoin%
\definecolor{currentfill}{rgb}{0.000000,0.000000,0.000000}%
\pgfsetfillcolor{currentfill}%
\pgfsetlinewidth{0.501875pt}%
\definecolor{currentstroke}{rgb}{0.000000,0.000000,0.000000}%
\pgfsetstrokecolor{currentstroke}%
\pgfsetdash{}{0pt}%
\pgfsys@defobject{currentmarker}{\pgfqpoint{0.000000in}{0.000000in}}{\pgfqpoint{0.069444in}{0.000000in}}{%
\pgfpathmoveto{\pgfqpoint{0.000000in}{0.000000in}}%
\pgfpathlineto{\pgfqpoint{0.069444in}{0.000000in}}%
\pgfusepath{stroke,fill}%
}%
\begin{pgfscope}%
\pgfsys@transformshift{0.488334in}{2.016169in}%
\pgfsys@useobject{currentmarker}{}%
\end{pgfscope}%
\end{pgfscope}%
\begin{pgfscope}%
\pgfsetbuttcap%
\pgfsetroundjoin%
\definecolor{currentfill}{rgb}{0.000000,0.000000,0.000000}%
\pgfsetfillcolor{currentfill}%
\pgfsetlinewidth{0.501875pt}%
\definecolor{currentstroke}{rgb}{0.000000,0.000000,0.000000}%
\pgfsetstrokecolor{currentstroke}%
\pgfsetdash{}{0pt}%
\pgfsys@defobject{currentmarker}{\pgfqpoint{-0.069444in}{0.000000in}}{\pgfqpoint{0.000000in}{0.000000in}}{%
\pgfpathmoveto{\pgfqpoint{0.000000in}{0.000000in}}%
\pgfpathlineto{\pgfqpoint{-0.069444in}{0.000000in}}%
\pgfusepath{stroke,fill}%
}%
\begin{pgfscope}%
\pgfsys@transformshift{3.710382in}{2.016169in}%
\pgfsys@useobject{currentmarker}{}%
\end{pgfscope}%
\end{pgfscope}%
\begin{pgfscope}%
\pgftext[x=0.418890in,y=2.016169in,right,]{\rmfamily\fontsize{8.000000}{9.600000}\selectfont 1.3}%
\end{pgfscope}%
\begin{pgfscope}%
\pgfsetbuttcap%
\pgfsetroundjoin%
\definecolor{currentfill}{rgb}{0.000000,0.000000,0.000000}%
\pgfsetfillcolor{currentfill}%
\pgfsetlinewidth{0.501875pt}%
\definecolor{currentstroke}{rgb}{0.000000,0.000000,0.000000}%
\pgfsetstrokecolor{currentstroke}%
\pgfsetdash{}{0pt}%
\pgfsys@defobject{currentmarker}{\pgfqpoint{0.000000in}{0.000000in}}{\pgfqpoint{0.069444in}{0.000000in}}{%
\pgfpathmoveto{\pgfqpoint{0.000000in}{0.000000in}}%
\pgfpathlineto{\pgfqpoint{0.069444in}{0.000000in}}%
\pgfusepath{stroke,fill}%
}%
\begin{pgfscope}%
\pgfsys@transformshift{0.488334in}{2.472963in}%
\pgfsys@useobject{currentmarker}{}%
\end{pgfscope}%
\end{pgfscope}%
\begin{pgfscope}%
\pgfsetbuttcap%
\pgfsetroundjoin%
\definecolor{currentfill}{rgb}{0.000000,0.000000,0.000000}%
\pgfsetfillcolor{currentfill}%
\pgfsetlinewidth{0.501875pt}%
\definecolor{currentstroke}{rgb}{0.000000,0.000000,0.000000}%
\pgfsetstrokecolor{currentstroke}%
\pgfsetdash{}{0pt}%
\pgfsys@defobject{currentmarker}{\pgfqpoint{-0.069444in}{0.000000in}}{\pgfqpoint{0.000000in}{0.000000in}}{%
\pgfpathmoveto{\pgfqpoint{0.000000in}{0.000000in}}%
\pgfpathlineto{\pgfqpoint{-0.069444in}{0.000000in}}%
\pgfusepath{stroke,fill}%
}%
\begin{pgfscope}%
\pgfsys@transformshift{3.710382in}{2.472963in}%
\pgfsys@useobject{currentmarker}{}%
\end{pgfscope}%
\end{pgfscope}%
\begin{pgfscope}%
\pgftext[x=0.418890in,y=2.472963in,right,]{\rmfamily\fontsize{8.000000}{9.600000}\selectfont 1.4}%
\end{pgfscope}%
\begin{pgfscope}%
\pgftext[x=0.172742in,y=1.445177in,,bottom,rotate=90.000000]{\rmfamily\fontsize{9.000000}{10.800000}\selectfont Relative signal efficiency}%
\end{pgfscope}%
\end{pgfpicture}%
\makeatother%
\endgroup%

  \caption{
    Ratios of classifier signal efficiencies of $\PBzero\to\PJpsi\PKstar$ simulation (blue), resampled simulation (red, see section \ref{resampling}) and fully reweighted simulation (green, see section \ref{reweighting}) to sweighted $\PBzero\to\PJpsi\PKstar$ data sample.
  }
  \label{fig:datamcsystematic}
\end{figure}

%0.242537719914
%0.312235905464
%0.26723365517
%0.242492112505

\subsection{Resampling technique}
\label{resampling}

The standard method of dealing with data-simulation differences in LHCb analyses is to perform a bin-based reweighting based on a few of the variables in the dataset, and optionally a resampling of PID variables.

For the bin-based reweighting, real and simulated data of a control channel is binned.
The ratio of bin heights is a heuristic approximation to the probability ratio $p / (1 - p)$ that a given entry in the dataset belongs to a real data sample, compared to a simulated one.
By weighting the signal simulation sample using the determined ratios of bin heights, data-simulation differences in the sample are reduced.

This approach suffers from the curse of dimensionality when applying it to more than a few variables.

In the case of PID variables, a resampling approach is used.
For any final state particle, a multi-dimensional histogram of $N_\text{tracks}$, $p$ and $η$ and a PID variable can be filled from a reference decay channel.
For a given simulated final state particle, a new PID value can be sampled from
\begin{equation}
  p(\text{PID}|N_\text{tracks},p,η)\:.
\end{equation}

The two variables $N_\text{tracks}$ and $N_\text{SPD hits}$ have been reweighted using the bin-based approach and all PID variables have been resampled from data.
The resulting variables can be seen in figure \ref{fig:mcfeaturesresampled} and the classifier response on the modified dataset is given in \ref{fig:resampledresponse}.

\begin{figure}
  \centering
  %% Creator: Matplotlib, PGF backend
%%
%% To include the figure in your LaTeX document, write
%%   \input{<filename>.pgf}
%%
%% Make sure the required packages are loaded in your preamble
%%   \usepackage{pgf}
%%
%% Figures using additional raster images can only be included by \input if
%% they are in the same directory as the main LaTeX file. For loading figures
%% from other directories you can use the `import` package
%%   \usepackage{import}
%% and then include the figures with
%%   \import{<path to file>}{<filename>.pgf}
%%
%% Matplotlib used the following preamble
%%   \usepackage{fontspec}
%%   \setmainfont{DejaVu Serif}
%%   \setsansfont{DejaVu Sans}
%%   \setmonofont{DejaVu Sans Mono}
%%
\begingroup%
\makeatletter%
\begin{pgfpicture}%
\pgfpathrectangle{\pgfpointorigin}{\pgfqpoint{3.979659in}{2.696729in}}%
\pgfusepath{use as bounding box, clip}%
\begin{pgfscope}%
\pgfsetbuttcap%
\pgfsetmiterjoin%
\definecolor{currentfill}{rgb}{1.000000,1.000000,1.000000}%
\pgfsetfillcolor{currentfill}%
\pgfsetlinewidth{0.000000pt}%
\definecolor{currentstroke}{rgb}{1.000000,1.000000,1.000000}%
\pgfsetstrokecolor{currentstroke}%
\pgfsetdash{}{0pt}%
\pgfpathmoveto{\pgfqpoint{0.000000in}{0.000000in}}%
\pgfpathlineto{\pgfqpoint{3.979659in}{0.000000in}}%
\pgfpathlineto{\pgfqpoint{3.979659in}{2.696729in}}%
\pgfpathlineto{\pgfqpoint{0.000000in}{2.696729in}}%
\pgfpathclose%
\pgfusepath{fill}%
\end{pgfscope}%
\begin{pgfscope}%
\pgfsetbuttcap%
\pgfsetmiterjoin%
\definecolor{currentfill}{rgb}{1.000000,1.000000,1.000000}%
\pgfsetfillcolor{currentfill}%
\pgfsetlinewidth{0.000000pt}%
\definecolor{currentstroke}{rgb}{0.000000,0.000000,0.000000}%
\pgfsetstrokecolor{currentstroke}%
\pgfsetstrokeopacity{0.000000}%
\pgfsetdash{}{0pt}%
\pgfpathmoveto{\pgfqpoint{0.366840in}{0.417391in}}%
\pgfpathlineto{\pgfqpoint{3.894313in}{0.417391in}}%
\pgfpathlineto{\pgfqpoint{3.894313in}{2.592964in}}%
\pgfpathlineto{\pgfqpoint{0.366840in}{2.592964in}}%
\pgfpathclose%
\pgfusepath{fill}%
\end{pgfscope}%
\begin{pgfscope}%
\pgfpathrectangle{\pgfqpoint{0.366840in}{0.417391in}}{\pgfqpoint{3.527473in}{2.175573in}} %
\pgfusepath{clip}%
\pgfsetbuttcap%
\pgfsetmiterjoin%
\definecolor{currentfill}{rgb}{0.215686,0.470588,0.749020}%
\pgfsetfillcolor{currentfill}%
\pgfsetlinewidth{0.000000pt}%
\definecolor{currentstroke}{rgb}{0.000000,0.000000,0.000000}%
\pgfsetstrokecolor{currentstroke}%
\pgfsetdash{}{0pt}%
\pgfpathmoveto{\pgfqpoint{0.734094in}{0.417391in}}%
\pgfpathlineto{\pgfqpoint{0.734094in}{0.417167in}}%
\pgfpathlineto{\pgfqpoint{0.793624in}{0.417167in}}%
\pgfpathlineto{\pgfqpoint{0.793624in}{0.417097in}}%
\pgfpathlineto{\pgfqpoint{0.853154in}{0.417097in}}%
\pgfpathlineto{\pgfqpoint{0.853154in}{0.421050in}}%
\pgfpathlineto{\pgfqpoint{0.912684in}{0.421050in}}%
\pgfpathlineto{\pgfqpoint{0.912684in}{0.420034in}}%
\pgfpathlineto{\pgfqpoint{0.972213in}{0.420034in}}%
\pgfpathlineto{\pgfqpoint{0.972213in}{0.421352in}}%
\pgfpathlineto{\pgfqpoint{1.031743in}{0.421352in}}%
\pgfpathlineto{\pgfqpoint{1.031743in}{0.424008in}}%
\pgfpathlineto{\pgfqpoint{1.091273in}{0.424008in}}%
\pgfpathlineto{\pgfqpoint{1.091273in}{0.426924in}}%
\pgfpathlineto{\pgfqpoint{1.150803in}{0.426924in}}%
\pgfpathlineto{\pgfqpoint{1.150803in}{0.430325in}}%
\pgfpathlineto{\pgfqpoint{1.210332in}{0.430325in}}%
\pgfpathlineto{\pgfqpoint{1.210332in}{0.431630in}}%
\pgfpathlineto{\pgfqpoint{1.269862in}{0.431630in}}%
\pgfpathlineto{\pgfqpoint{1.269862in}{0.442121in}}%
\pgfpathlineto{\pgfqpoint{1.329392in}{0.442121in}}%
\pgfpathlineto{\pgfqpoint{1.329392in}{0.446047in}}%
\pgfpathlineto{\pgfqpoint{1.388921in}{0.446047in}}%
\pgfpathlineto{\pgfqpoint{1.388921in}{0.458721in}}%
\pgfpathlineto{\pgfqpoint{1.448451in}{0.458721in}}%
\pgfpathlineto{\pgfqpoint{1.448451in}{0.473568in}}%
\pgfpathlineto{\pgfqpoint{1.507981in}{0.473568in}}%
\pgfpathlineto{\pgfqpoint{1.507981in}{0.491823in}}%
\pgfpathlineto{\pgfqpoint{1.567511in}{0.491823in}}%
\pgfpathlineto{\pgfqpoint{1.567511in}{0.500018in}}%
\pgfpathlineto{\pgfqpoint{1.627040in}{0.500018in}}%
\pgfpathlineto{\pgfqpoint{1.627040in}{0.517873in}}%
\pgfpathlineto{\pgfqpoint{1.686570in}{0.517873in}}%
\pgfpathlineto{\pgfqpoint{1.686570in}{0.532887in}}%
\pgfpathlineto{\pgfqpoint{1.746100in}{0.532887in}}%
\pgfpathlineto{\pgfqpoint{1.746100in}{0.546269in}}%
\pgfpathlineto{\pgfqpoint{1.805630in}{0.546269in}}%
\pgfpathlineto{\pgfqpoint{1.805630in}{0.571468in}}%
\pgfpathlineto{\pgfqpoint{1.865159in}{0.571468in}}%
\pgfpathlineto{\pgfqpoint{1.865159in}{0.588009in}}%
\pgfpathlineto{\pgfqpoint{1.924689in}{0.588009in}}%
\pgfpathlineto{\pgfqpoint{1.924689in}{0.611870in}}%
\pgfpathlineto{\pgfqpoint{1.984219in}{0.611870in}}%
\pgfpathlineto{\pgfqpoint{1.984219in}{0.638974in}}%
\pgfpathlineto{\pgfqpoint{2.043749in}{0.638974in}}%
\pgfpathlineto{\pgfqpoint{2.043749in}{0.655907in}}%
\pgfpathlineto{\pgfqpoint{2.103278in}{0.655907in}}%
\pgfpathlineto{\pgfqpoint{2.103278in}{0.686082in}}%
\pgfpathlineto{\pgfqpoint{2.162808in}{0.686082in}}%
\pgfpathlineto{\pgfqpoint{2.162808in}{0.690005in}}%
\pgfpathlineto{\pgfqpoint{2.222338in}{0.690005in}}%
\pgfpathlineto{\pgfqpoint{2.222338in}{0.725347in}}%
\pgfpathlineto{\pgfqpoint{2.281867in}{0.725347in}}%
\pgfpathlineto{\pgfqpoint{2.281867in}{0.773429in}}%
\pgfpathlineto{\pgfqpoint{2.341397in}{0.773429in}}%
\pgfpathlineto{\pgfqpoint{2.341397in}{0.788800in}}%
\pgfpathlineto{\pgfqpoint{2.400927in}{0.788800in}}%
\pgfpathlineto{\pgfqpoint{2.400927in}{0.840406in}}%
\pgfpathlineto{\pgfqpoint{2.460457in}{0.840406in}}%
\pgfpathlineto{\pgfqpoint{2.460457in}{0.880563in}}%
\pgfpathlineto{\pgfqpoint{2.519986in}{0.880563in}}%
\pgfpathlineto{\pgfqpoint{2.519986in}{0.938268in}}%
\pgfpathlineto{\pgfqpoint{2.579516in}{0.938268in}}%
\pgfpathlineto{\pgfqpoint{2.579516in}{0.982206in}}%
\pgfpathlineto{\pgfqpoint{2.639046in}{0.982206in}}%
\pgfpathlineto{\pgfqpoint{2.639046in}{1.044667in}}%
\pgfpathlineto{\pgfqpoint{2.698576in}{1.044667in}}%
\pgfpathlineto{\pgfqpoint{2.698576in}{1.098275in}}%
\pgfpathlineto{\pgfqpoint{2.758105in}{1.098275in}}%
\pgfpathlineto{\pgfqpoint{2.758105in}{1.184058in}}%
\pgfpathlineto{\pgfqpoint{2.817635in}{1.184058in}}%
\pgfpathlineto{\pgfqpoint{2.817635in}{1.280529in}}%
\pgfpathlineto{\pgfqpoint{2.877165in}{1.280529in}}%
\pgfpathlineto{\pgfqpoint{2.877165in}{1.379048in}}%
\pgfpathlineto{\pgfqpoint{2.936695in}{1.379048in}}%
\pgfpathlineto{\pgfqpoint{2.936695in}{1.500812in}}%
\pgfpathlineto{\pgfqpoint{2.996224in}{1.500812in}}%
\pgfpathlineto{\pgfqpoint{2.996224in}{1.642029in}}%
\pgfpathlineto{\pgfqpoint{3.055754in}{1.642029in}}%
\pgfpathlineto{\pgfqpoint{3.055754in}{1.791425in}}%
\pgfpathlineto{\pgfqpoint{3.115284in}{1.791425in}}%
\pgfpathlineto{\pgfqpoint{3.115284in}{1.939485in}}%
\pgfpathlineto{\pgfqpoint{3.174813in}{1.939485in}}%
\pgfpathlineto{\pgfqpoint{3.174813in}{2.078439in}}%
\pgfpathlineto{\pgfqpoint{3.234343in}{2.078439in}}%
\pgfpathlineto{\pgfqpoint{3.234343in}{2.155896in}}%
\pgfpathlineto{\pgfqpoint{3.293873in}{2.155896in}}%
\pgfpathlineto{\pgfqpoint{3.293873in}{2.231194in}}%
\pgfpathlineto{\pgfqpoint{3.353403in}{2.231194in}}%
\pgfpathlineto{\pgfqpoint{3.353403in}{2.162929in}}%
\pgfpathlineto{\pgfqpoint{3.412932in}{2.162929in}}%
\pgfpathlineto{\pgfqpoint{3.412932in}{2.060466in}}%
\pgfpathlineto{\pgfqpoint{3.472462in}{2.060466in}}%
\pgfpathlineto{\pgfqpoint{3.472462in}{1.809326in}}%
\pgfpathlineto{\pgfqpoint{3.531992in}{1.809326in}}%
\pgfpathlineto{\pgfqpoint{3.531992in}{1.491668in}}%
\pgfpathlineto{\pgfqpoint{3.591522in}{1.491668in}}%
\pgfpathlineto{\pgfqpoint{3.591522in}{1.106731in}}%
\pgfpathlineto{\pgfqpoint{3.651051in}{1.106731in}}%
\pgfpathlineto{\pgfqpoint{3.651051in}{0.631507in}}%
\pgfpathlineto{\pgfqpoint{3.710581in}{0.631507in}}%
\pgfpathlineto{\pgfqpoint{3.710581in}{0.417391in}}%
\pgfpathlineto{\pgfqpoint{3.651051in}{0.417391in}}%
\pgfpathlineto{\pgfqpoint{3.651051in}{0.417391in}}%
\pgfpathlineto{\pgfqpoint{3.591522in}{0.417391in}}%
\pgfpathlineto{\pgfqpoint{3.591522in}{0.417391in}}%
\pgfpathlineto{\pgfqpoint{3.531992in}{0.417391in}}%
\pgfpathlineto{\pgfqpoint{3.531992in}{0.417391in}}%
\pgfpathlineto{\pgfqpoint{3.472462in}{0.417391in}}%
\pgfpathlineto{\pgfqpoint{3.472462in}{0.417391in}}%
\pgfpathlineto{\pgfqpoint{3.412932in}{0.417391in}}%
\pgfpathlineto{\pgfqpoint{3.412932in}{0.417391in}}%
\pgfpathlineto{\pgfqpoint{3.353403in}{0.417391in}}%
\pgfpathlineto{\pgfqpoint{3.353403in}{0.417391in}}%
\pgfpathlineto{\pgfqpoint{3.293873in}{0.417391in}}%
\pgfpathlineto{\pgfqpoint{3.293873in}{0.417391in}}%
\pgfpathlineto{\pgfqpoint{3.234343in}{0.417391in}}%
\pgfpathlineto{\pgfqpoint{3.234343in}{0.417391in}}%
\pgfpathlineto{\pgfqpoint{3.174813in}{0.417391in}}%
\pgfpathlineto{\pgfqpoint{3.174813in}{0.417391in}}%
\pgfpathlineto{\pgfqpoint{3.115284in}{0.417391in}}%
\pgfpathlineto{\pgfqpoint{3.115284in}{0.417391in}}%
\pgfpathlineto{\pgfqpoint{3.055754in}{0.417391in}}%
\pgfpathlineto{\pgfqpoint{3.055754in}{0.417391in}}%
\pgfpathlineto{\pgfqpoint{2.996224in}{0.417391in}}%
\pgfpathlineto{\pgfqpoint{2.996224in}{0.417391in}}%
\pgfpathlineto{\pgfqpoint{2.936695in}{0.417391in}}%
\pgfpathlineto{\pgfqpoint{2.936695in}{0.417391in}}%
\pgfpathlineto{\pgfqpoint{2.877165in}{0.417391in}}%
\pgfpathlineto{\pgfqpoint{2.877165in}{0.417391in}}%
\pgfpathlineto{\pgfqpoint{2.817635in}{0.417391in}}%
\pgfpathlineto{\pgfqpoint{2.817635in}{0.417391in}}%
\pgfpathlineto{\pgfqpoint{2.758105in}{0.417391in}}%
\pgfpathlineto{\pgfqpoint{2.758105in}{0.417391in}}%
\pgfpathlineto{\pgfqpoint{2.698576in}{0.417391in}}%
\pgfpathlineto{\pgfqpoint{2.698576in}{0.417391in}}%
\pgfpathlineto{\pgfqpoint{2.639046in}{0.417391in}}%
\pgfpathlineto{\pgfqpoint{2.639046in}{0.417391in}}%
\pgfpathlineto{\pgfqpoint{2.579516in}{0.417391in}}%
\pgfpathlineto{\pgfqpoint{2.579516in}{0.417391in}}%
\pgfpathlineto{\pgfqpoint{2.519986in}{0.417391in}}%
\pgfpathlineto{\pgfqpoint{2.519986in}{0.417391in}}%
\pgfpathlineto{\pgfqpoint{2.460457in}{0.417391in}}%
\pgfpathlineto{\pgfqpoint{2.460457in}{0.417391in}}%
\pgfpathlineto{\pgfqpoint{2.400927in}{0.417391in}}%
\pgfpathlineto{\pgfqpoint{2.400927in}{0.417391in}}%
\pgfpathlineto{\pgfqpoint{2.341397in}{0.417391in}}%
\pgfpathlineto{\pgfqpoint{2.341397in}{0.417391in}}%
\pgfpathlineto{\pgfqpoint{2.281867in}{0.417391in}}%
\pgfpathlineto{\pgfqpoint{2.281867in}{0.417391in}}%
\pgfpathlineto{\pgfqpoint{2.222338in}{0.417391in}}%
\pgfpathlineto{\pgfqpoint{2.222338in}{0.417391in}}%
\pgfpathlineto{\pgfqpoint{2.162808in}{0.417391in}}%
\pgfpathlineto{\pgfqpoint{2.162808in}{0.417391in}}%
\pgfpathlineto{\pgfqpoint{2.103278in}{0.417391in}}%
\pgfpathlineto{\pgfqpoint{2.103278in}{0.417391in}}%
\pgfpathlineto{\pgfqpoint{2.043749in}{0.417391in}}%
\pgfpathlineto{\pgfqpoint{2.043749in}{0.417391in}}%
\pgfpathlineto{\pgfqpoint{1.984219in}{0.417391in}}%
\pgfpathlineto{\pgfqpoint{1.984219in}{0.417391in}}%
\pgfpathlineto{\pgfqpoint{1.924689in}{0.417391in}}%
\pgfpathlineto{\pgfqpoint{1.924689in}{0.417391in}}%
\pgfpathlineto{\pgfqpoint{1.865159in}{0.417391in}}%
\pgfpathlineto{\pgfqpoint{1.865159in}{0.417391in}}%
\pgfpathlineto{\pgfqpoint{1.805630in}{0.417391in}}%
\pgfpathlineto{\pgfqpoint{1.805630in}{0.417391in}}%
\pgfpathlineto{\pgfqpoint{1.746100in}{0.417391in}}%
\pgfpathlineto{\pgfqpoint{1.746100in}{0.417391in}}%
\pgfpathlineto{\pgfqpoint{1.686570in}{0.417391in}}%
\pgfpathlineto{\pgfqpoint{1.686570in}{0.417391in}}%
\pgfpathlineto{\pgfqpoint{1.627040in}{0.417391in}}%
\pgfpathlineto{\pgfqpoint{1.627040in}{0.417391in}}%
\pgfpathlineto{\pgfqpoint{1.567511in}{0.417391in}}%
\pgfpathlineto{\pgfqpoint{1.567511in}{0.417391in}}%
\pgfpathlineto{\pgfqpoint{1.507981in}{0.417391in}}%
\pgfpathlineto{\pgfqpoint{1.507981in}{0.417391in}}%
\pgfpathlineto{\pgfqpoint{1.448451in}{0.417391in}}%
\pgfpathlineto{\pgfqpoint{1.448451in}{0.417391in}}%
\pgfpathlineto{\pgfqpoint{1.388921in}{0.417391in}}%
\pgfpathlineto{\pgfqpoint{1.388921in}{0.417391in}}%
\pgfpathlineto{\pgfqpoint{1.329392in}{0.417391in}}%
\pgfpathlineto{\pgfqpoint{1.329392in}{0.417391in}}%
\pgfpathlineto{\pgfqpoint{1.269862in}{0.417391in}}%
\pgfpathlineto{\pgfqpoint{1.269862in}{0.417391in}}%
\pgfpathlineto{\pgfqpoint{1.210332in}{0.417391in}}%
\pgfpathlineto{\pgfqpoint{1.210332in}{0.417391in}}%
\pgfpathlineto{\pgfqpoint{1.150803in}{0.417391in}}%
\pgfpathlineto{\pgfqpoint{1.150803in}{0.417391in}}%
\pgfpathlineto{\pgfqpoint{1.091273in}{0.417391in}}%
\pgfpathlineto{\pgfqpoint{1.091273in}{0.417391in}}%
\pgfpathlineto{\pgfqpoint{1.031743in}{0.417391in}}%
\pgfpathlineto{\pgfqpoint{1.031743in}{0.417391in}}%
\pgfpathlineto{\pgfqpoint{0.972213in}{0.417391in}}%
\pgfpathlineto{\pgfqpoint{0.972213in}{0.417391in}}%
\pgfpathlineto{\pgfqpoint{0.912684in}{0.417391in}}%
\pgfpathlineto{\pgfqpoint{0.912684in}{0.417391in}}%
\pgfpathlineto{\pgfqpoint{0.853154in}{0.417391in}}%
\pgfpathlineto{\pgfqpoint{0.853154in}{0.417391in}}%
\pgfpathlineto{\pgfqpoint{0.793624in}{0.417391in}}%
\pgfpathlineto{\pgfqpoint{0.793624in}{0.417391in}}%
\pgfpathlineto{\pgfqpoint{0.734094in}{0.417391in}}%
\pgfusepath{fill}%
\end{pgfscope}%
\begin{pgfscope}%
\pgfpathrectangle{\pgfqpoint{0.366840in}{0.417391in}}{\pgfqpoint{3.527473in}{2.175573in}} %
\pgfusepath{clip}%
\pgfsetbuttcap%
\pgfsetmiterjoin%
\pgfsetlinewidth{0.501875pt}%
\definecolor{currentstroke}{rgb}{1.000000,0.000000,0.000000}%
\pgfsetstrokecolor{currentstroke}%
\pgfsetdash{}{0pt}%
\pgfpathmoveto{\pgfqpoint{0.734094in}{0.417391in}}%
\pgfpathlineto{\pgfqpoint{0.734094in}{0.417457in}}%
\pgfpathlineto{\pgfqpoint{0.793624in}{0.417457in}}%
\pgfpathlineto{\pgfqpoint{0.793624in}{0.417589in}}%
\pgfpathlineto{\pgfqpoint{0.853154in}{0.417589in}}%
\pgfpathlineto{\pgfqpoint{0.853154in}{0.418315in}}%
\pgfpathlineto{\pgfqpoint{0.912684in}{0.418315in}}%
\pgfpathlineto{\pgfqpoint{0.912684in}{0.419107in}}%
\pgfpathlineto{\pgfqpoint{0.972213in}{0.419107in}}%
\pgfpathlineto{\pgfqpoint{0.972213in}{0.419239in}}%
\pgfpathlineto{\pgfqpoint{1.031743in}{0.419239in}}%
\pgfpathlineto{\pgfqpoint{1.031743in}{0.420163in}}%
\pgfpathlineto{\pgfqpoint{1.091273in}{0.420163in}}%
\pgfpathlineto{\pgfqpoint{1.091273in}{0.421417in}}%
\pgfpathlineto{\pgfqpoint{1.150803in}{0.421417in}}%
\pgfpathlineto{\pgfqpoint{1.150803in}{0.423265in}}%
\pgfpathlineto{\pgfqpoint{1.210332in}{0.423265in}}%
\pgfpathlineto{\pgfqpoint{1.210332in}{0.425245in}}%
\pgfpathlineto{\pgfqpoint{1.269862in}{0.425245in}}%
\pgfpathlineto{\pgfqpoint{1.269862in}{0.429535in}}%
\pgfpathlineto{\pgfqpoint{1.329392in}{0.429535in}}%
\pgfpathlineto{\pgfqpoint{1.329392in}{0.435343in}}%
\pgfpathlineto{\pgfqpoint{1.388921in}{0.435343in}}%
\pgfpathlineto{\pgfqpoint{1.388921in}{0.439370in}}%
\pgfpathlineto{\pgfqpoint{1.448451in}{0.439370in}}%
\pgfpathlineto{\pgfqpoint{1.448451in}{0.447026in}}%
\pgfpathlineto{\pgfqpoint{1.507981in}{0.447026in}}%
\pgfpathlineto{\pgfqpoint{1.507981in}{0.455540in}}%
\pgfpathlineto{\pgfqpoint{1.567511in}{0.455540in}}%
\pgfpathlineto{\pgfqpoint{1.567511in}{0.464780in}}%
\pgfpathlineto{\pgfqpoint{1.627040in}{0.464780in}}%
\pgfpathlineto{\pgfqpoint{1.627040in}{0.472965in}}%
\pgfpathlineto{\pgfqpoint{1.686570in}{0.472965in}}%
\pgfpathlineto{\pgfqpoint{1.686570in}{0.481017in}}%
\pgfpathlineto{\pgfqpoint{1.746100in}{0.481017in}}%
\pgfpathlineto{\pgfqpoint{1.746100in}{0.499035in}}%
\pgfpathlineto{\pgfqpoint{1.805630in}{0.499035in}}%
\pgfpathlineto{\pgfqpoint{1.805630in}{0.507945in}}%
\pgfpathlineto{\pgfqpoint{1.865159in}{0.507945in}}%
\pgfpathlineto{\pgfqpoint{1.865159in}{0.525172in}}%
\pgfpathlineto{\pgfqpoint{1.924689in}{0.525172in}}%
\pgfpathlineto{\pgfqpoint{1.924689in}{0.534280in}}%
\pgfpathlineto{\pgfqpoint{1.984219in}{0.534280in}}%
\pgfpathlineto{\pgfqpoint{1.984219in}{0.552959in}}%
\pgfpathlineto{\pgfqpoint{2.043749in}{0.552959in}}%
\pgfpathlineto{\pgfqpoint{2.043749in}{0.568205in}}%
\pgfpathlineto{\pgfqpoint{2.103278in}{0.568205in}}%
\pgfpathlineto{\pgfqpoint{2.103278in}{0.587082in}}%
\pgfpathlineto{\pgfqpoint{2.162808in}{0.587082in}}%
\pgfpathlineto{\pgfqpoint{2.162808in}{0.611634in}}%
\pgfpathlineto{\pgfqpoint{2.222338in}{0.611634in}}%
\pgfpathlineto{\pgfqpoint{2.222338in}{0.637111in}}%
\pgfpathlineto{\pgfqpoint{2.281867in}{0.637111in}}%
\pgfpathlineto{\pgfqpoint{2.281867in}{0.669056in}}%
\pgfpathlineto{\pgfqpoint{2.341397in}{0.669056in}}%
\pgfpathlineto{\pgfqpoint{2.341397in}{0.704367in}}%
\pgfpathlineto{\pgfqpoint{2.400927in}{0.704367in}}%
\pgfpathlineto{\pgfqpoint{2.400927in}{0.731824in}}%
\pgfpathlineto{\pgfqpoint{2.460457in}{0.731824in}}%
\pgfpathlineto{\pgfqpoint{2.460457in}{0.774659in}}%
\pgfpathlineto{\pgfqpoint{2.519986in}{0.774659in}}%
\pgfpathlineto{\pgfqpoint{2.519986in}{0.809376in}}%
\pgfpathlineto{\pgfqpoint{2.579516in}{0.809376in}}%
\pgfpathlineto{\pgfqpoint{2.579516in}{0.873662in}}%
\pgfpathlineto{\pgfqpoint{2.639046in}{0.873662in}}%
\pgfpathlineto{\pgfqpoint{2.639046in}{0.919599in}}%
\pgfpathlineto{\pgfqpoint{2.698576in}{0.919599in}}%
\pgfpathlineto{\pgfqpoint{2.698576in}{0.995237in}}%
\pgfpathlineto{\pgfqpoint{2.758105in}{0.995237in}}%
\pgfpathlineto{\pgfqpoint{2.758105in}{1.070347in}}%
\pgfpathlineto{\pgfqpoint{2.817635in}{1.070347in}}%
\pgfpathlineto{\pgfqpoint{2.817635in}{1.172254in}}%
\pgfpathlineto{\pgfqpoint{2.877165in}{1.172254in}}%
\pgfpathlineto{\pgfqpoint{2.877165in}{1.275349in}}%
\pgfpathlineto{\pgfqpoint{2.936695in}{1.275349in}}%
\pgfpathlineto{\pgfqpoint{2.936695in}{1.410388in}}%
\pgfpathlineto{\pgfqpoint{2.996224in}{1.410388in}}%
\pgfpathlineto{\pgfqpoint{2.996224in}{1.569849in}}%
\pgfpathlineto{\pgfqpoint{3.055754in}{1.569849in}}%
\pgfpathlineto{\pgfqpoint{3.055754in}{1.749770in}}%
\pgfpathlineto{\pgfqpoint{3.115284in}{1.749770in}}%
\pgfpathlineto{\pgfqpoint{3.115284in}{1.952263in}}%
\pgfpathlineto{\pgfqpoint{3.174813in}{1.952263in}}%
\pgfpathlineto{\pgfqpoint{3.174813in}{2.154955in}}%
\pgfpathlineto{\pgfqpoint{3.234343in}{2.154955in}}%
\pgfpathlineto{\pgfqpoint{3.234343in}{2.308871in}}%
\pgfpathlineto{\pgfqpoint{3.293873in}{2.308871in}}%
\pgfpathlineto{\pgfqpoint{3.293873in}{2.426816in}}%
\pgfpathlineto{\pgfqpoint{3.353403in}{2.426816in}}%
\pgfpathlineto{\pgfqpoint{3.353403in}{2.496052in}}%
\pgfpathlineto{\pgfqpoint{3.412932in}{2.496052in}}%
\pgfpathlineto{\pgfqpoint{3.412932in}{2.391307in}}%
\pgfpathlineto{\pgfqpoint{3.472462in}{2.391307in}}%
\pgfpathlineto{\pgfqpoint{3.472462in}{2.174623in}}%
\pgfpathlineto{\pgfqpoint{3.531992in}{2.174623in}}%
\pgfpathlineto{\pgfqpoint{3.531992in}{1.867979in}}%
\pgfpathlineto{\pgfqpoint{3.591522in}{1.867979in}}%
\pgfpathlineto{\pgfqpoint{3.591522in}{1.452035in}}%
\pgfpathlineto{\pgfqpoint{3.651051in}{1.452035in}}%
\pgfpathlineto{\pgfqpoint{3.651051in}{0.797297in}}%
\pgfpathlineto{\pgfqpoint{3.710581in}{0.797297in}}%
\pgfpathlineto{\pgfqpoint{3.710581in}{0.417391in}}%
\pgfusepath{stroke}%
\end{pgfscope}%
\begin{pgfscope}%
\pgfpathrectangle{\pgfqpoint{0.366840in}{0.417391in}}{\pgfqpoint{3.527473in}{2.175573in}} %
\pgfusepath{clip}%
\pgfsetbuttcap%
\pgfsetmiterjoin%
\pgfsetlinewidth{0.501875pt}%
\definecolor{currentstroke}{rgb}{1.000000,0.647059,0.000000}%
\pgfsetstrokecolor{currentstroke}%
\pgfsetdash{}{0pt}%
\pgfpathmoveto{\pgfqpoint{0.734094in}{0.417391in}}%
\pgfpathlineto{\pgfqpoint{0.734094in}{0.417391in}}%
\pgfpathlineto{\pgfqpoint{0.793624in}{0.417391in}}%
\pgfpathlineto{\pgfqpoint{0.793624in}{0.417391in}}%
\pgfpathlineto{\pgfqpoint{0.853154in}{0.417391in}}%
\pgfpathlineto{\pgfqpoint{0.853154in}{0.417391in}}%
\pgfpathlineto{\pgfqpoint{0.912684in}{0.417391in}}%
\pgfpathlineto{\pgfqpoint{0.912684in}{0.417676in}}%
\pgfpathlineto{\pgfqpoint{0.972213in}{0.417676in}}%
\pgfpathlineto{\pgfqpoint{0.972213in}{0.417862in}}%
\pgfpathlineto{\pgfqpoint{1.031743in}{0.417862in}}%
\pgfpathlineto{\pgfqpoint{1.031743in}{0.420247in}}%
\pgfpathlineto{\pgfqpoint{1.091273in}{0.420247in}}%
\pgfpathlineto{\pgfqpoint{1.091273in}{0.418851in}}%
\pgfpathlineto{\pgfqpoint{1.150803in}{0.418851in}}%
\pgfpathlineto{\pgfqpoint{1.150803in}{0.420058in}}%
\pgfpathlineto{\pgfqpoint{1.210332in}{0.420058in}}%
\pgfpathlineto{\pgfqpoint{1.210332in}{0.421478in}}%
\pgfpathlineto{\pgfqpoint{1.269862in}{0.421478in}}%
\pgfpathlineto{\pgfqpoint{1.269862in}{0.426198in}}%
\pgfpathlineto{\pgfqpoint{1.329392in}{0.426198in}}%
\pgfpathlineto{\pgfqpoint{1.329392in}{0.430986in}}%
\pgfpathlineto{\pgfqpoint{1.388921in}{0.430986in}}%
\pgfpathlineto{\pgfqpoint{1.388921in}{0.436428in}}%
\pgfpathlineto{\pgfqpoint{1.448451in}{0.436428in}}%
\pgfpathlineto{\pgfqpoint{1.448451in}{0.447270in}}%
\pgfpathlineto{\pgfqpoint{1.507981in}{0.447270in}}%
\pgfpathlineto{\pgfqpoint{1.507981in}{0.458261in}}%
\pgfpathlineto{\pgfqpoint{1.567511in}{0.458261in}}%
\pgfpathlineto{\pgfqpoint{1.567511in}{0.467871in}}%
\pgfpathlineto{\pgfqpoint{1.627040in}{0.467871in}}%
\pgfpathlineto{\pgfqpoint{1.627040in}{0.488582in}}%
\pgfpathlineto{\pgfqpoint{1.686570in}{0.488582in}}%
\pgfpathlineto{\pgfqpoint{1.686570in}{0.490104in}}%
\pgfpathlineto{\pgfqpoint{1.746100in}{0.490104in}}%
\pgfpathlineto{\pgfqpoint{1.746100in}{0.511059in}}%
\pgfpathlineto{\pgfqpoint{1.805630in}{0.511059in}}%
\pgfpathlineto{\pgfqpoint{1.805630in}{0.524739in}}%
\pgfpathlineto{\pgfqpoint{1.865159in}{0.524739in}}%
\pgfpathlineto{\pgfqpoint{1.865159in}{0.548726in}}%
\pgfpathlineto{\pgfqpoint{1.924689in}{0.548726in}}%
\pgfpathlineto{\pgfqpoint{1.924689in}{0.569971in}}%
\pgfpathlineto{\pgfqpoint{1.984219in}{0.569971in}}%
\pgfpathlineto{\pgfqpoint{1.984219in}{0.582982in}}%
\pgfpathlineto{\pgfqpoint{2.043749in}{0.582982in}}%
\pgfpathlineto{\pgfqpoint{2.043749in}{0.596691in}}%
\pgfpathlineto{\pgfqpoint{2.103278in}{0.596691in}}%
\pgfpathlineto{\pgfqpoint{2.103278in}{0.638086in}}%
\pgfpathlineto{\pgfqpoint{2.162808in}{0.638086in}}%
\pgfpathlineto{\pgfqpoint{2.162808in}{0.663349in}}%
\pgfpathlineto{\pgfqpoint{2.222338in}{0.663349in}}%
\pgfpathlineto{\pgfqpoint{2.222338in}{0.705540in}}%
\pgfpathlineto{\pgfqpoint{2.281867in}{0.705540in}}%
\pgfpathlineto{\pgfqpoint{2.281867in}{0.729346in}}%
\pgfpathlineto{\pgfqpoint{2.341397in}{0.729346in}}%
\pgfpathlineto{\pgfqpoint{2.341397in}{0.778949in}}%
\pgfpathlineto{\pgfqpoint{2.400927in}{0.778949in}}%
\pgfpathlineto{\pgfqpoint{2.400927in}{0.800197in}}%
\pgfpathlineto{\pgfqpoint{2.460457in}{0.800197in}}%
\pgfpathlineto{\pgfqpoint{2.460457in}{0.846882in}}%
\pgfpathlineto{\pgfqpoint{2.519986in}{0.846882in}}%
\pgfpathlineto{\pgfqpoint{2.519986in}{0.878310in}}%
\pgfpathlineto{\pgfqpoint{2.579516in}{0.878310in}}%
\pgfpathlineto{\pgfqpoint{2.579516in}{0.938402in}}%
\pgfpathlineto{\pgfqpoint{2.639046in}{0.938402in}}%
\pgfpathlineto{\pgfqpoint{2.639046in}{1.023049in}}%
\pgfpathlineto{\pgfqpoint{2.698576in}{1.023049in}}%
\pgfpathlineto{\pgfqpoint{2.698576in}{1.080057in}}%
\pgfpathlineto{\pgfqpoint{2.758105in}{1.080057in}}%
\pgfpathlineto{\pgfqpoint{2.758105in}{1.122434in}}%
\pgfpathlineto{\pgfqpoint{2.817635in}{1.122434in}}%
\pgfpathlineto{\pgfqpoint{2.817635in}{1.239654in}}%
\pgfpathlineto{\pgfqpoint{2.877165in}{1.239654in}}%
\pgfpathlineto{\pgfqpoint{2.877165in}{1.348510in}}%
\pgfpathlineto{\pgfqpoint{2.936695in}{1.348510in}}%
\pgfpathlineto{\pgfqpoint{2.936695in}{1.486220in}}%
\pgfpathlineto{\pgfqpoint{2.996224in}{1.486220in}}%
\pgfpathlineto{\pgfqpoint{2.996224in}{1.634740in}}%
\pgfpathlineto{\pgfqpoint{3.055754in}{1.634740in}}%
\pgfpathlineto{\pgfqpoint{3.055754in}{1.778182in}}%
\pgfpathlineto{\pgfqpoint{3.115284in}{1.778182in}}%
\pgfpathlineto{\pgfqpoint{3.115284in}{1.988644in}}%
\pgfpathlineto{\pgfqpoint{3.174813in}{1.988644in}}%
\pgfpathlineto{\pgfqpoint{3.174813in}{2.148944in}}%
\pgfpathlineto{\pgfqpoint{3.234343in}{2.148944in}}%
\pgfpathlineto{\pgfqpoint{3.234343in}{2.299741in}}%
\pgfpathlineto{\pgfqpoint{3.293873in}{2.299741in}}%
\pgfpathlineto{\pgfqpoint{3.293873in}{2.333422in}}%
\pgfpathlineto{\pgfqpoint{3.353403in}{2.333422in}}%
\pgfpathlineto{\pgfqpoint{3.353403in}{2.322084in}}%
\pgfpathlineto{\pgfqpoint{3.412932in}{2.322084in}}%
\pgfpathlineto{\pgfqpoint{3.412932in}{2.177073in}}%
\pgfpathlineto{\pgfqpoint{3.472462in}{2.177073in}}%
\pgfpathlineto{\pgfqpoint{3.472462in}{1.943272in}}%
\pgfpathlineto{\pgfqpoint{3.531992in}{1.943272in}}%
\pgfpathlineto{\pgfqpoint{3.531992in}{1.641571in}}%
\pgfpathlineto{\pgfqpoint{3.591522in}{1.641571in}}%
\pgfpathlineto{\pgfqpoint{3.591522in}{1.227278in}}%
\pgfpathlineto{\pgfqpoint{3.651051in}{1.227278in}}%
\pgfpathlineto{\pgfqpoint{3.651051in}{0.656580in}}%
\pgfpathlineto{\pgfqpoint{3.710581in}{0.656580in}}%
\pgfpathlineto{\pgfqpoint{3.710581in}{0.417391in}}%
\pgfusepath{stroke}%
\end{pgfscope}%
\begin{pgfscope}%
\pgfsetrectcap%
\pgfsetmiterjoin%
\pgfsetlinewidth{1.003750pt}%
\definecolor{currentstroke}{rgb}{0.000000,0.000000,0.000000}%
\pgfsetstrokecolor{currentstroke}%
\pgfsetdash{}{0pt}%
\pgfpathmoveto{\pgfqpoint{0.366840in}{2.592964in}}%
\pgfpathlineto{\pgfqpoint{3.894313in}{2.592964in}}%
\pgfusepath{stroke}%
\end{pgfscope}%
\begin{pgfscope}%
\pgfsetrectcap%
\pgfsetmiterjoin%
\pgfsetlinewidth{1.003750pt}%
\definecolor{currentstroke}{rgb}{0.000000,0.000000,0.000000}%
\pgfsetstrokecolor{currentstroke}%
\pgfsetdash{}{0pt}%
\pgfpathmoveto{\pgfqpoint{3.894313in}{0.417391in}}%
\pgfpathlineto{\pgfqpoint{3.894313in}{2.592964in}}%
\pgfusepath{stroke}%
\end{pgfscope}%
\begin{pgfscope}%
\pgfsetrectcap%
\pgfsetmiterjoin%
\pgfsetlinewidth{1.003750pt}%
\definecolor{currentstroke}{rgb}{0.000000,0.000000,0.000000}%
\pgfsetstrokecolor{currentstroke}%
\pgfsetdash{}{0pt}%
\pgfpathmoveto{\pgfqpoint{0.366840in}{0.417391in}}%
\pgfpathlineto{\pgfqpoint{3.894313in}{0.417391in}}%
\pgfusepath{stroke}%
\end{pgfscope}%
\begin{pgfscope}%
\pgfsetrectcap%
\pgfsetmiterjoin%
\pgfsetlinewidth{1.003750pt}%
\definecolor{currentstroke}{rgb}{0.000000,0.000000,0.000000}%
\pgfsetstrokecolor{currentstroke}%
\pgfsetdash{}{0pt}%
\pgfpathmoveto{\pgfqpoint{0.366840in}{0.417391in}}%
\pgfpathlineto{\pgfqpoint{0.366840in}{2.592964in}}%
\pgfusepath{stroke}%
\end{pgfscope}%
\begin{pgfscope}%
\pgfsetbuttcap%
\pgfsetroundjoin%
\definecolor{currentfill}{rgb}{0.000000,0.000000,0.000000}%
\pgfsetfillcolor{currentfill}%
\pgfsetlinewidth{0.501875pt}%
\definecolor{currentstroke}{rgb}{0.000000,0.000000,0.000000}%
\pgfsetstrokecolor{currentstroke}%
\pgfsetdash{}{0pt}%
\pgfsys@defobject{currentmarker}{\pgfqpoint{0.000000in}{0.000000in}}{\pgfqpoint{0.000000in}{0.069444in}}{%
\pgfpathmoveto{\pgfqpoint{0.000000in}{0.000000in}}%
\pgfpathlineto{\pgfqpoint{0.000000in}{0.069444in}}%
\pgfusepath{stroke,fill}%
}%
\begin{pgfscope}%
\pgfsys@transformshift{0.366840in}{0.417391in}%
\pgfsys@useobject{currentmarker}{}%
\end{pgfscope}%
\end{pgfscope}%
\begin{pgfscope}%
\pgfsetbuttcap%
\pgfsetroundjoin%
\definecolor{currentfill}{rgb}{0.000000,0.000000,0.000000}%
\pgfsetfillcolor{currentfill}%
\pgfsetlinewidth{0.501875pt}%
\definecolor{currentstroke}{rgb}{0.000000,0.000000,0.000000}%
\pgfsetstrokecolor{currentstroke}%
\pgfsetdash{}{0pt}%
\pgfsys@defobject{currentmarker}{\pgfqpoint{0.000000in}{-0.069444in}}{\pgfqpoint{0.000000in}{0.000000in}}{%
\pgfpathmoveto{\pgfqpoint{0.000000in}{0.000000in}}%
\pgfpathlineto{\pgfqpoint{0.000000in}{-0.069444in}}%
\pgfusepath{stroke,fill}%
}%
\begin{pgfscope}%
\pgfsys@transformshift{0.366840in}{2.592964in}%
\pgfsys@useobject{currentmarker}{}%
\end{pgfscope}%
\end{pgfscope}%
\begin{pgfscope}%
\pgftext[x=0.366840in,y=0.347947in,,top]{\rmfamily\fontsize{8.000000}{9.600000}\selectfont −8}%
\end{pgfscope}%
\begin{pgfscope}%
\pgfsetbuttcap%
\pgfsetroundjoin%
\definecolor{currentfill}{rgb}{0.000000,0.000000,0.000000}%
\pgfsetfillcolor{currentfill}%
\pgfsetlinewidth{0.501875pt}%
\definecolor{currentstroke}{rgb}{0.000000,0.000000,0.000000}%
\pgfsetstrokecolor{currentstroke}%
\pgfsetdash{}{0pt}%
\pgfsys@defobject{currentmarker}{\pgfqpoint{0.000000in}{0.000000in}}{\pgfqpoint{0.000000in}{0.069444in}}{%
\pgfpathmoveto{\pgfqpoint{0.000000in}{0.000000in}}%
\pgfpathlineto{\pgfqpoint{0.000000in}{0.069444in}}%
\pgfusepath{stroke,fill}%
}%
\begin{pgfscope}%
\pgfsys@transformshift{0.870765in}{0.417391in}%
\pgfsys@useobject{currentmarker}{}%
\end{pgfscope}%
\end{pgfscope}%
\begin{pgfscope}%
\pgfsetbuttcap%
\pgfsetroundjoin%
\definecolor{currentfill}{rgb}{0.000000,0.000000,0.000000}%
\pgfsetfillcolor{currentfill}%
\pgfsetlinewidth{0.501875pt}%
\definecolor{currentstroke}{rgb}{0.000000,0.000000,0.000000}%
\pgfsetstrokecolor{currentstroke}%
\pgfsetdash{}{0pt}%
\pgfsys@defobject{currentmarker}{\pgfqpoint{0.000000in}{-0.069444in}}{\pgfqpoint{0.000000in}{0.000000in}}{%
\pgfpathmoveto{\pgfqpoint{0.000000in}{0.000000in}}%
\pgfpathlineto{\pgfqpoint{0.000000in}{-0.069444in}}%
\pgfusepath{stroke,fill}%
}%
\begin{pgfscope}%
\pgfsys@transformshift{0.870765in}{2.592964in}%
\pgfsys@useobject{currentmarker}{}%
\end{pgfscope}%
\end{pgfscope}%
\begin{pgfscope}%
\pgftext[x=0.870765in,y=0.347947in,,top]{\rmfamily\fontsize{8.000000}{9.600000}\selectfont −6}%
\end{pgfscope}%
\begin{pgfscope}%
\pgfsetbuttcap%
\pgfsetroundjoin%
\definecolor{currentfill}{rgb}{0.000000,0.000000,0.000000}%
\pgfsetfillcolor{currentfill}%
\pgfsetlinewidth{0.501875pt}%
\definecolor{currentstroke}{rgb}{0.000000,0.000000,0.000000}%
\pgfsetstrokecolor{currentstroke}%
\pgfsetdash{}{0pt}%
\pgfsys@defobject{currentmarker}{\pgfqpoint{0.000000in}{0.000000in}}{\pgfqpoint{0.000000in}{0.069444in}}{%
\pgfpathmoveto{\pgfqpoint{0.000000in}{0.000000in}}%
\pgfpathlineto{\pgfqpoint{0.000000in}{0.069444in}}%
\pgfusepath{stroke,fill}%
}%
\begin{pgfscope}%
\pgfsys@transformshift{1.374690in}{0.417391in}%
\pgfsys@useobject{currentmarker}{}%
\end{pgfscope}%
\end{pgfscope}%
\begin{pgfscope}%
\pgfsetbuttcap%
\pgfsetroundjoin%
\definecolor{currentfill}{rgb}{0.000000,0.000000,0.000000}%
\pgfsetfillcolor{currentfill}%
\pgfsetlinewidth{0.501875pt}%
\definecolor{currentstroke}{rgb}{0.000000,0.000000,0.000000}%
\pgfsetstrokecolor{currentstroke}%
\pgfsetdash{}{0pt}%
\pgfsys@defobject{currentmarker}{\pgfqpoint{0.000000in}{-0.069444in}}{\pgfqpoint{0.000000in}{0.000000in}}{%
\pgfpathmoveto{\pgfqpoint{0.000000in}{0.000000in}}%
\pgfpathlineto{\pgfqpoint{0.000000in}{-0.069444in}}%
\pgfusepath{stroke,fill}%
}%
\begin{pgfscope}%
\pgfsys@transformshift{1.374690in}{2.592964in}%
\pgfsys@useobject{currentmarker}{}%
\end{pgfscope}%
\end{pgfscope}%
\begin{pgfscope}%
\pgftext[x=1.374690in,y=0.347947in,,top]{\rmfamily\fontsize{8.000000}{9.600000}\selectfont −4}%
\end{pgfscope}%
\begin{pgfscope}%
\pgfsetbuttcap%
\pgfsetroundjoin%
\definecolor{currentfill}{rgb}{0.000000,0.000000,0.000000}%
\pgfsetfillcolor{currentfill}%
\pgfsetlinewidth{0.501875pt}%
\definecolor{currentstroke}{rgb}{0.000000,0.000000,0.000000}%
\pgfsetstrokecolor{currentstroke}%
\pgfsetdash{}{0pt}%
\pgfsys@defobject{currentmarker}{\pgfqpoint{0.000000in}{0.000000in}}{\pgfqpoint{0.000000in}{0.069444in}}{%
\pgfpathmoveto{\pgfqpoint{0.000000in}{0.000000in}}%
\pgfpathlineto{\pgfqpoint{0.000000in}{0.069444in}}%
\pgfusepath{stroke,fill}%
}%
\begin{pgfscope}%
\pgfsys@transformshift{1.878614in}{0.417391in}%
\pgfsys@useobject{currentmarker}{}%
\end{pgfscope}%
\end{pgfscope}%
\begin{pgfscope}%
\pgfsetbuttcap%
\pgfsetroundjoin%
\definecolor{currentfill}{rgb}{0.000000,0.000000,0.000000}%
\pgfsetfillcolor{currentfill}%
\pgfsetlinewidth{0.501875pt}%
\definecolor{currentstroke}{rgb}{0.000000,0.000000,0.000000}%
\pgfsetstrokecolor{currentstroke}%
\pgfsetdash{}{0pt}%
\pgfsys@defobject{currentmarker}{\pgfqpoint{0.000000in}{-0.069444in}}{\pgfqpoint{0.000000in}{0.000000in}}{%
\pgfpathmoveto{\pgfqpoint{0.000000in}{0.000000in}}%
\pgfpathlineto{\pgfqpoint{0.000000in}{-0.069444in}}%
\pgfusepath{stroke,fill}%
}%
\begin{pgfscope}%
\pgfsys@transformshift{1.878614in}{2.592964in}%
\pgfsys@useobject{currentmarker}{}%
\end{pgfscope}%
\end{pgfscope}%
\begin{pgfscope}%
\pgftext[x=1.878614in,y=0.347947in,,top]{\rmfamily\fontsize{8.000000}{9.600000}\selectfont −2}%
\end{pgfscope}%
\begin{pgfscope}%
\pgfsetbuttcap%
\pgfsetroundjoin%
\definecolor{currentfill}{rgb}{0.000000,0.000000,0.000000}%
\pgfsetfillcolor{currentfill}%
\pgfsetlinewidth{0.501875pt}%
\definecolor{currentstroke}{rgb}{0.000000,0.000000,0.000000}%
\pgfsetstrokecolor{currentstroke}%
\pgfsetdash{}{0pt}%
\pgfsys@defobject{currentmarker}{\pgfqpoint{0.000000in}{0.000000in}}{\pgfqpoint{0.000000in}{0.069444in}}{%
\pgfpathmoveto{\pgfqpoint{0.000000in}{0.000000in}}%
\pgfpathlineto{\pgfqpoint{0.000000in}{0.069444in}}%
\pgfusepath{stroke,fill}%
}%
\begin{pgfscope}%
\pgfsys@transformshift{2.382539in}{0.417391in}%
\pgfsys@useobject{currentmarker}{}%
\end{pgfscope}%
\end{pgfscope}%
\begin{pgfscope}%
\pgfsetbuttcap%
\pgfsetroundjoin%
\definecolor{currentfill}{rgb}{0.000000,0.000000,0.000000}%
\pgfsetfillcolor{currentfill}%
\pgfsetlinewidth{0.501875pt}%
\definecolor{currentstroke}{rgb}{0.000000,0.000000,0.000000}%
\pgfsetstrokecolor{currentstroke}%
\pgfsetdash{}{0pt}%
\pgfsys@defobject{currentmarker}{\pgfqpoint{0.000000in}{-0.069444in}}{\pgfqpoint{0.000000in}{0.000000in}}{%
\pgfpathmoveto{\pgfqpoint{0.000000in}{0.000000in}}%
\pgfpathlineto{\pgfqpoint{0.000000in}{-0.069444in}}%
\pgfusepath{stroke,fill}%
}%
\begin{pgfscope}%
\pgfsys@transformshift{2.382539in}{2.592964in}%
\pgfsys@useobject{currentmarker}{}%
\end{pgfscope}%
\end{pgfscope}%
\begin{pgfscope}%
\pgftext[x=2.382539in,y=0.347947in,,top]{\rmfamily\fontsize{8.000000}{9.600000}\selectfont 0}%
\end{pgfscope}%
\begin{pgfscope}%
\pgfsetbuttcap%
\pgfsetroundjoin%
\definecolor{currentfill}{rgb}{0.000000,0.000000,0.000000}%
\pgfsetfillcolor{currentfill}%
\pgfsetlinewidth{0.501875pt}%
\definecolor{currentstroke}{rgb}{0.000000,0.000000,0.000000}%
\pgfsetstrokecolor{currentstroke}%
\pgfsetdash{}{0pt}%
\pgfsys@defobject{currentmarker}{\pgfqpoint{0.000000in}{0.000000in}}{\pgfqpoint{0.000000in}{0.069444in}}{%
\pgfpathmoveto{\pgfqpoint{0.000000in}{0.000000in}}%
\pgfpathlineto{\pgfqpoint{0.000000in}{0.069444in}}%
\pgfusepath{stroke,fill}%
}%
\begin{pgfscope}%
\pgfsys@transformshift{2.886464in}{0.417391in}%
\pgfsys@useobject{currentmarker}{}%
\end{pgfscope}%
\end{pgfscope}%
\begin{pgfscope}%
\pgfsetbuttcap%
\pgfsetroundjoin%
\definecolor{currentfill}{rgb}{0.000000,0.000000,0.000000}%
\pgfsetfillcolor{currentfill}%
\pgfsetlinewidth{0.501875pt}%
\definecolor{currentstroke}{rgb}{0.000000,0.000000,0.000000}%
\pgfsetstrokecolor{currentstroke}%
\pgfsetdash{}{0pt}%
\pgfsys@defobject{currentmarker}{\pgfqpoint{0.000000in}{-0.069444in}}{\pgfqpoint{0.000000in}{0.000000in}}{%
\pgfpathmoveto{\pgfqpoint{0.000000in}{0.000000in}}%
\pgfpathlineto{\pgfqpoint{0.000000in}{-0.069444in}}%
\pgfusepath{stroke,fill}%
}%
\begin{pgfscope}%
\pgfsys@transformshift{2.886464in}{2.592964in}%
\pgfsys@useobject{currentmarker}{}%
\end{pgfscope}%
\end{pgfscope}%
\begin{pgfscope}%
\pgftext[x=2.886464in,y=0.347947in,,top]{\rmfamily\fontsize{8.000000}{9.600000}\selectfont 2}%
\end{pgfscope}%
\begin{pgfscope}%
\pgfsetbuttcap%
\pgfsetroundjoin%
\definecolor{currentfill}{rgb}{0.000000,0.000000,0.000000}%
\pgfsetfillcolor{currentfill}%
\pgfsetlinewidth{0.501875pt}%
\definecolor{currentstroke}{rgb}{0.000000,0.000000,0.000000}%
\pgfsetstrokecolor{currentstroke}%
\pgfsetdash{}{0pt}%
\pgfsys@defobject{currentmarker}{\pgfqpoint{0.000000in}{0.000000in}}{\pgfqpoint{0.000000in}{0.069444in}}{%
\pgfpathmoveto{\pgfqpoint{0.000000in}{0.000000in}}%
\pgfpathlineto{\pgfqpoint{0.000000in}{0.069444in}}%
\pgfusepath{stroke,fill}%
}%
\begin{pgfscope}%
\pgfsys@transformshift{3.390389in}{0.417391in}%
\pgfsys@useobject{currentmarker}{}%
\end{pgfscope}%
\end{pgfscope}%
\begin{pgfscope}%
\pgfsetbuttcap%
\pgfsetroundjoin%
\definecolor{currentfill}{rgb}{0.000000,0.000000,0.000000}%
\pgfsetfillcolor{currentfill}%
\pgfsetlinewidth{0.501875pt}%
\definecolor{currentstroke}{rgb}{0.000000,0.000000,0.000000}%
\pgfsetstrokecolor{currentstroke}%
\pgfsetdash{}{0pt}%
\pgfsys@defobject{currentmarker}{\pgfqpoint{0.000000in}{-0.069444in}}{\pgfqpoint{0.000000in}{0.000000in}}{%
\pgfpathmoveto{\pgfqpoint{0.000000in}{0.000000in}}%
\pgfpathlineto{\pgfqpoint{0.000000in}{-0.069444in}}%
\pgfusepath{stroke,fill}%
}%
\begin{pgfscope}%
\pgfsys@transformshift{3.390389in}{2.592964in}%
\pgfsys@useobject{currentmarker}{}%
\end{pgfscope}%
\end{pgfscope}%
\begin{pgfscope}%
\pgftext[x=3.390389in,y=0.347947in,,top]{\rmfamily\fontsize{8.000000}{9.600000}\selectfont 4}%
\end{pgfscope}%
\begin{pgfscope}%
\pgfsetbuttcap%
\pgfsetroundjoin%
\definecolor{currentfill}{rgb}{0.000000,0.000000,0.000000}%
\pgfsetfillcolor{currentfill}%
\pgfsetlinewidth{0.501875pt}%
\definecolor{currentstroke}{rgb}{0.000000,0.000000,0.000000}%
\pgfsetstrokecolor{currentstroke}%
\pgfsetdash{}{0pt}%
\pgfsys@defobject{currentmarker}{\pgfqpoint{0.000000in}{0.000000in}}{\pgfqpoint{0.000000in}{0.069444in}}{%
\pgfpathmoveto{\pgfqpoint{0.000000in}{0.000000in}}%
\pgfpathlineto{\pgfqpoint{0.000000in}{0.069444in}}%
\pgfusepath{stroke,fill}%
}%
\begin{pgfscope}%
\pgfsys@transformshift{3.894313in}{0.417391in}%
\pgfsys@useobject{currentmarker}{}%
\end{pgfscope}%
\end{pgfscope}%
\begin{pgfscope}%
\pgfsetbuttcap%
\pgfsetroundjoin%
\definecolor{currentfill}{rgb}{0.000000,0.000000,0.000000}%
\pgfsetfillcolor{currentfill}%
\pgfsetlinewidth{0.501875pt}%
\definecolor{currentstroke}{rgb}{0.000000,0.000000,0.000000}%
\pgfsetstrokecolor{currentstroke}%
\pgfsetdash{}{0pt}%
\pgfsys@defobject{currentmarker}{\pgfqpoint{0.000000in}{-0.069444in}}{\pgfqpoint{0.000000in}{0.000000in}}{%
\pgfpathmoveto{\pgfqpoint{0.000000in}{0.000000in}}%
\pgfpathlineto{\pgfqpoint{0.000000in}{-0.069444in}}%
\pgfusepath{stroke,fill}%
}%
\begin{pgfscope}%
\pgfsys@transformshift{3.894313in}{2.592964in}%
\pgfsys@useobject{currentmarker}{}%
\end{pgfscope}%
\end{pgfscope}%
\begin{pgfscope}%
\pgftext[x=3.894313in,y=0.347947in,,top]{\rmfamily\fontsize{8.000000}{9.600000}\selectfont 6}%
\end{pgfscope}%
\begin{pgfscope}%
\pgftext[x=2.130577in,y=0.170972in,,top]{\rmfamily\fontsize{9.000000}{10.800000}\selectfont classifier response}%
\end{pgfscope}%
\begin{pgfscope}%
\pgfsetbuttcap%
\pgfsetroundjoin%
\definecolor{currentfill}{rgb}{0.000000,0.000000,0.000000}%
\pgfsetfillcolor{currentfill}%
\pgfsetlinewidth{0.501875pt}%
\definecolor{currentstroke}{rgb}{0.000000,0.000000,0.000000}%
\pgfsetstrokecolor{currentstroke}%
\pgfsetdash{}{0pt}%
\pgfsys@defobject{currentmarker}{\pgfqpoint{0.000000in}{0.000000in}}{\pgfqpoint{0.069444in}{0.000000in}}{%
\pgfpathmoveto{\pgfqpoint{0.000000in}{0.000000in}}%
\pgfpathlineto{\pgfqpoint{0.069444in}{0.000000in}}%
\pgfusepath{stroke,fill}%
}%
\begin{pgfscope}%
\pgfsys@transformshift{0.366840in}{0.417391in}%
\pgfsys@useobject{currentmarker}{}%
\end{pgfscope}%
\end{pgfscope}%
\begin{pgfscope}%
\pgfsetbuttcap%
\pgfsetroundjoin%
\definecolor{currentfill}{rgb}{0.000000,0.000000,0.000000}%
\pgfsetfillcolor{currentfill}%
\pgfsetlinewidth{0.501875pt}%
\definecolor{currentstroke}{rgb}{0.000000,0.000000,0.000000}%
\pgfsetstrokecolor{currentstroke}%
\pgfsetdash{}{0pt}%
\pgfsys@defobject{currentmarker}{\pgfqpoint{-0.069444in}{0.000000in}}{\pgfqpoint{0.000000in}{0.000000in}}{%
\pgfpathmoveto{\pgfqpoint{0.000000in}{0.000000in}}%
\pgfpathlineto{\pgfqpoint{-0.069444in}{0.000000in}}%
\pgfusepath{stroke,fill}%
}%
\begin{pgfscope}%
\pgfsys@transformshift{3.894313in}{0.417391in}%
\pgfsys@useobject{currentmarker}{}%
\end{pgfscope}%
\end{pgfscope}%
\begin{pgfscope}%
\pgftext[x=0.297396in,y=0.417391in,right,]{\rmfamily\fontsize{8.000000}{9.600000}\selectfont 0.00}%
\end{pgfscope}%
\begin{pgfscope}%
\pgfsetbuttcap%
\pgfsetroundjoin%
\definecolor{currentfill}{rgb}{0.000000,0.000000,0.000000}%
\pgfsetfillcolor{currentfill}%
\pgfsetlinewidth{0.501875pt}%
\definecolor{currentstroke}{rgb}{0.000000,0.000000,0.000000}%
\pgfsetstrokecolor{currentstroke}%
\pgfsetdash{}{0pt}%
\pgfsys@defobject{currentmarker}{\pgfqpoint{0.000000in}{0.000000in}}{\pgfqpoint{0.069444in}{0.000000in}}{%
\pgfpathmoveto{\pgfqpoint{0.000000in}{0.000000in}}%
\pgfpathlineto{\pgfqpoint{0.069444in}{0.000000in}}%
\pgfusepath{stroke,fill}%
}%
\begin{pgfscope}%
\pgfsys@transformshift{0.366840in}{0.728187in}%
\pgfsys@useobject{currentmarker}{}%
\end{pgfscope}%
\end{pgfscope}%
\begin{pgfscope}%
\pgfsetbuttcap%
\pgfsetroundjoin%
\definecolor{currentfill}{rgb}{0.000000,0.000000,0.000000}%
\pgfsetfillcolor{currentfill}%
\pgfsetlinewidth{0.501875pt}%
\definecolor{currentstroke}{rgb}{0.000000,0.000000,0.000000}%
\pgfsetstrokecolor{currentstroke}%
\pgfsetdash{}{0pt}%
\pgfsys@defobject{currentmarker}{\pgfqpoint{-0.069444in}{0.000000in}}{\pgfqpoint{0.000000in}{0.000000in}}{%
\pgfpathmoveto{\pgfqpoint{0.000000in}{0.000000in}}%
\pgfpathlineto{\pgfqpoint{-0.069444in}{0.000000in}}%
\pgfusepath{stroke,fill}%
}%
\begin{pgfscope}%
\pgfsys@transformshift{3.894313in}{0.728187in}%
\pgfsys@useobject{currentmarker}{}%
\end{pgfscope}%
\end{pgfscope}%
\begin{pgfscope}%
\pgftext[x=0.297396in,y=0.728187in,right,]{\rmfamily\fontsize{8.000000}{9.600000}\selectfont 0.05}%
\end{pgfscope}%
\begin{pgfscope}%
\pgfsetbuttcap%
\pgfsetroundjoin%
\definecolor{currentfill}{rgb}{0.000000,0.000000,0.000000}%
\pgfsetfillcolor{currentfill}%
\pgfsetlinewidth{0.501875pt}%
\definecolor{currentstroke}{rgb}{0.000000,0.000000,0.000000}%
\pgfsetstrokecolor{currentstroke}%
\pgfsetdash{}{0pt}%
\pgfsys@defobject{currentmarker}{\pgfqpoint{0.000000in}{0.000000in}}{\pgfqpoint{0.069444in}{0.000000in}}{%
\pgfpathmoveto{\pgfqpoint{0.000000in}{0.000000in}}%
\pgfpathlineto{\pgfqpoint{0.069444in}{0.000000in}}%
\pgfusepath{stroke,fill}%
}%
\begin{pgfscope}%
\pgfsys@transformshift{0.366840in}{1.038983in}%
\pgfsys@useobject{currentmarker}{}%
\end{pgfscope}%
\end{pgfscope}%
\begin{pgfscope}%
\pgfsetbuttcap%
\pgfsetroundjoin%
\definecolor{currentfill}{rgb}{0.000000,0.000000,0.000000}%
\pgfsetfillcolor{currentfill}%
\pgfsetlinewidth{0.501875pt}%
\definecolor{currentstroke}{rgb}{0.000000,0.000000,0.000000}%
\pgfsetstrokecolor{currentstroke}%
\pgfsetdash{}{0pt}%
\pgfsys@defobject{currentmarker}{\pgfqpoint{-0.069444in}{0.000000in}}{\pgfqpoint{0.000000in}{0.000000in}}{%
\pgfpathmoveto{\pgfqpoint{0.000000in}{0.000000in}}%
\pgfpathlineto{\pgfqpoint{-0.069444in}{0.000000in}}%
\pgfusepath{stroke,fill}%
}%
\begin{pgfscope}%
\pgfsys@transformshift{3.894313in}{1.038983in}%
\pgfsys@useobject{currentmarker}{}%
\end{pgfscope}%
\end{pgfscope}%
\begin{pgfscope}%
\pgftext[x=0.297396in,y=1.038983in,right,]{\rmfamily\fontsize{8.000000}{9.600000}\selectfont 0.10}%
\end{pgfscope}%
\begin{pgfscope}%
\pgfsetbuttcap%
\pgfsetroundjoin%
\definecolor{currentfill}{rgb}{0.000000,0.000000,0.000000}%
\pgfsetfillcolor{currentfill}%
\pgfsetlinewidth{0.501875pt}%
\definecolor{currentstroke}{rgb}{0.000000,0.000000,0.000000}%
\pgfsetstrokecolor{currentstroke}%
\pgfsetdash{}{0pt}%
\pgfsys@defobject{currentmarker}{\pgfqpoint{0.000000in}{0.000000in}}{\pgfqpoint{0.069444in}{0.000000in}}{%
\pgfpathmoveto{\pgfqpoint{0.000000in}{0.000000in}}%
\pgfpathlineto{\pgfqpoint{0.069444in}{0.000000in}}%
\pgfusepath{stroke,fill}%
}%
\begin{pgfscope}%
\pgfsys@transformshift{0.366840in}{1.349779in}%
\pgfsys@useobject{currentmarker}{}%
\end{pgfscope}%
\end{pgfscope}%
\begin{pgfscope}%
\pgfsetbuttcap%
\pgfsetroundjoin%
\definecolor{currentfill}{rgb}{0.000000,0.000000,0.000000}%
\pgfsetfillcolor{currentfill}%
\pgfsetlinewidth{0.501875pt}%
\definecolor{currentstroke}{rgb}{0.000000,0.000000,0.000000}%
\pgfsetstrokecolor{currentstroke}%
\pgfsetdash{}{0pt}%
\pgfsys@defobject{currentmarker}{\pgfqpoint{-0.069444in}{0.000000in}}{\pgfqpoint{0.000000in}{0.000000in}}{%
\pgfpathmoveto{\pgfqpoint{0.000000in}{0.000000in}}%
\pgfpathlineto{\pgfqpoint{-0.069444in}{0.000000in}}%
\pgfusepath{stroke,fill}%
}%
\begin{pgfscope}%
\pgfsys@transformshift{3.894313in}{1.349779in}%
\pgfsys@useobject{currentmarker}{}%
\end{pgfscope}%
\end{pgfscope}%
\begin{pgfscope}%
\pgftext[x=0.297396in,y=1.349779in,right,]{\rmfamily\fontsize{8.000000}{9.600000}\selectfont 0.15}%
\end{pgfscope}%
\begin{pgfscope}%
\pgfsetbuttcap%
\pgfsetroundjoin%
\definecolor{currentfill}{rgb}{0.000000,0.000000,0.000000}%
\pgfsetfillcolor{currentfill}%
\pgfsetlinewidth{0.501875pt}%
\definecolor{currentstroke}{rgb}{0.000000,0.000000,0.000000}%
\pgfsetstrokecolor{currentstroke}%
\pgfsetdash{}{0pt}%
\pgfsys@defobject{currentmarker}{\pgfqpoint{0.000000in}{0.000000in}}{\pgfqpoint{0.069444in}{0.000000in}}{%
\pgfpathmoveto{\pgfqpoint{0.000000in}{0.000000in}}%
\pgfpathlineto{\pgfqpoint{0.069444in}{0.000000in}}%
\pgfusepath{stroke,fill}%
}%
\begin{pgfscope}%
\pgfsys@transformshift{0.366840in}{1.660576in}%
\pgfsys@useobject{currentmarker}{}%
\end{pgfscope}%
\end{pgfscope}%
\begin{pgfscope}%
\pgfsetbuttcap%
\pgfsetroundjoin%
\definecolor{currentfill}{rgb}{0.000000,0.000000,0.000000}%
\pgfsetfillcolor{currentfill}%
\pgfsetlinewidth{0.501875pt}%
\definecolor{currentstroke}{rgb}{0.000000,0.000000,0.000000}%
\pgfsetstrokecolor{currentstroke}%
\pgfsetdash{}{0pt}%
\pgfsys@defobject{currentmarker}{\pgfqpoint{-0.069444in}{0.000000in}}{\pgfqpoint{0.000000in}{0.000000in}}{%
\pgfpathmoveto{\pgfqpoint{0.000000in}{0.000000in}}%
\pgfpathlineto{\pgfqpoint{-0.069444in}{0.000000in}}%
\pgfusepath{stroke,fill}%
}%
\begin{pgfscope}%
\pgfsys@transformshift{3.894313in}{1.660576in}%
\pgfsys@useobject{currentmarker}{}%
\end{pgfscope}%
\end{pgfscope}%
\begin{pgfscope}%
\pgftext[x=0.297396in,y=1.660576in,right,]{\rmfamily\fontsize{8.000000}{9.600000}\selectfont 0.20}%
\end{pgfscope}%
\begin{pgfscope}%
\pgfsetbuttcap%
\pgfsetroundjoin%
\definecolor{currentfill}{rgb}{0.000000,0.000000,0.000000}%
\pgfsetfillcolor{currentfill}%
\pgfsetlinewidth{0.501875pt}%
\definecolor{currentstroke}{rgb}{0.000000,0.000000,0.000000}%
\pgfsetstrokecolor{currentstroke}%
\pgfsetdash{}{0pt}%
\pgfsys@defobject{currentmarker}{\pgfqpoint{0.000000in}{0.000000in}}{\pgfqpoint{0.069444in}{0.000000in}}{%
\pgfpathmoveto{\pgfqpoint{0.000000in}{0.000000in}}%
\pgfpathlineto{\pgfqpoint{0.069444in}{0.000000in}}%
\pgfusepath{stroke,fill}%
}%
\begin{pgfscope}%
\pgfsys@transformshift{0.366840in}{1.971372in}%
\pgfsys@useobject{currentmarker}{}%
\end{pgfscope}%
\end{pgfscope}%
\begin{pgfscope}%
\pgfsetbuttcap%
\pgfsetroundjoin%
\definecolor{currentfill}{rgb}{0.000000,0.000000,0.000000}%
\pgfsetfillcolor{currentfill}%
\pgfsetlinewidth{0.501875pt}%
\definecolor{currentstroke}{rgb}{0.000000,0.000000,0.000000}%
\pgfsetstrokecolor{currentstroke}%
\pgfsetdash{}{0pt}%
\pgfsys@defobject{currentmarker}{\pgfqpoint{-0.069444in}{0.000000in}}{\pgfqpoint{0.000000in}{0.000000in}}{%
\pgfpathmoveto{\pgfqpoint{0.000000in}{0.000000in}}%
\pgfpathlineto{\pgfqpoint{-0.069444in}{0.000000in}}%
\pgfusepath{stroke,fill}%
}%
\begin{pgfscope}%
\pgfsys@transformshift{3.894313in}{1.971372in}%
\pgfsys@useobject{currentmarker}{}%
\end{pgfscope}%
\end{pgfscope}%
\begin{pgfscope}%
\pgftext[x=0.297396in,y=1.971372in,right,]{\rmfamily\fontsize{8.000000}{9.600000}\selectfont 0.25}%
\end{pgfscope}%
\begin{pgfscope}%
\pgfsetbuttcap%
\pgfsetroundjoin%
\definecolor{currentfill}{rgb}{0.000000,0.000000,0.000000}%
\pgfsetfillcolor{currentfill}%
\pgfsetlinewidth{0.501875pt}%
\definecolor{currentstroke}{rgb}{0.000000,0.000000,0.000000}%
\pgfsetstrokecolor{currentstroke}%
\pgfsetdash{}{0pt}%
\pgfsys@defobject{currentmarker}{\pgfqpoint{0.000000in}{0.000000in}}{\pgfqpoint{0.069444in}{0.000000in}}{%
\pgfpathmoveto{\pgfqpoint{0.000000in}{0.000000in}}%
\pgfpathlineto{\pgfqpoint{0.069444in}{0.000000in}}%
\pgfusepath{stroke,fill}%
}%
\begin{pgfscope}%
\pgfsys@transformshift{0.366840in}{2.282168in}%
\pgfsys@useobject{currentmarker}{}%
\end{pgfscope}%
\end{pgfscope}%
\begin{pgfscope}%
\pgfsetbuttcap%
\pgfsetroundjoin%
\definecolor{currentfill}{rgb}{0.000000,0.000000,0.000000}%
\pgfsetfillcolor{currentfill}%
\pgfsetlinewidth{0.501875pt}%
\definecolor{currentstroke}{rgb}{0.000000,0.000000,0.000000}%
\pgfsetstrokecolor{currentstroke}%
\pgfsetdash{}{0pt}%
\pgfsys@defobject{currentmarker}{\pgfqpoint{-0.069444in}{0.000000in}}{\pgfqpoint{0.000000in}{0.000000in}}{%
\pgfpathmoveto{\pgfqpoint{0.000000in}{0.000000in}}%
\pgfpathlineto{\pgfqpoint{-0.069444in}{0.000000in}}%
\pgfusepath{stroke,fill}%
}%
\begin{pgfscope}%
\pgfsys@transformshift{3.894313in}{2.282168in}%
\pgfsys@useobject{currentmarker}{}%
\end{pgfscope}%
\end{pgfscope}%
\begin{pgfscope}%
\pgftext[x=0.297396in,y=2.282168in,right,]{\rmfamily\fontsize{8.000000}{9.600000}\selectfont 0.30}%
\end{pgfscope}%
\begin{pgfscope}%
\pgfsetbuttcap%
\pgfsetroundjoin%
\definecolor{currentfill}{rgb}{0.000000,0.000000,0.000000}%
\pgfsetfillcolor{currentfill}%
\pgfsetlinewidth{0.501875pt}%
\definecolor{currentstroke}{rgb}{0.000000,0.000000,0.000000}%
\pgfsetstrokecolor{currentstroke}%
\pgfsetdash{}{0pt}%
\pgfsys@defobject{currentmarker}{\pgfqpoint{0.000000in}{0.000000in}}{\pgfqpoint{0.069444in}{0.000000in}}{%
\pgfpathmoveto{\pgfqpoint{0.000000in}{0.000000in}}%
\pgfpathlineto{\pgfqpoint{0.069444in}{0.000000in}}%
\pgfusepath{stroke,fill}%
}%
\begin{pgfscope}%
\pgfsys@transformshift{0.366840in}{2.592964in}%
\pgfsys@useobject{currentmarker}{}%
\end{pgfscope}%
\end{pgfscope}%
\begin{pgfscope}%
\pgfsetbuttcap%
\pgfsetroundjoin%
\definecolor{currentfill}{rgb}{0.000000,0.000000,0.000000}%
\pgfsetfillcolor{currentfill}%
\pgfsetlinewidth{0.501875pt}%
\definecolor{currentstroke}{rgb}{0.000000,0.000000,0.000000}%
\pgfsetstrokecolor{currentstroke}%
\pgfsetdash{}{0pt}%
\pgfsys@defobject{currentmarker}{\pgfqpoint{-0.069444in}{0.000000in}}{\pgfqpoint{0.000000in}{0.000000in}}{%
\pgfpathmoveto{\pgfqpoint{0.000000in}{0.000000in}}%
\pgfpathlineto{\pgfqpoint{-0.069444in}{0.000000in}}%
\pgfusepath{stroke,fill}%
}%
\begin{pgfscope}%
\pgfsys@transformshift{3.894313in}{2.592964in}%
\pgfsys@useobject{currentmarker}{}%
\end{pgfscope}%
\end{pgfscope}%
\begin{pgfscope}%
\pgftext[x=0.297396in,y=2.592964in,right,]{\rmfamily\fontsize{8.000000}{9.600000}\selectfont 0.35}%
\end{pgfscope}%
\end{pgfpicture}%
\makeatother%
\endgroup%

  \caption{
    Classifier response of the resampled simulated sample.
    The original simulated dataset (red) and data distribution (blue) are  given as a reference.
  }
  \label{fig:resampledresponse}
\end{figure}

\begin{figure}	
	\centering
	\begin{subfigure}[t]{0.49\textwidth}
		\centering
    %\includegraphics[width=\textwidth]{store/variables/DATA_MC_B_DiraAngle.pdf}
    %% Creator: Matplotlib, PGF backend
%%
%% To include the figure in your LaTeX document, write
%%   \input{<filename>.pgf}
%%
%% Make sure the required packages are loaded in your preamble
%%   \usepackage{pgf}
%%
%% Figures using additional raster images can only be included by \input if
%% they are in the same directory as the main LaTeX file. For loading figures
%% from other directories you can use the `import` package
%%   \usepackage{import}
%% and then include the figures with
%%   \import{<path to file>}{<filename>.pgf}
%%
%% Matplotlib used the following preamble
%%   \usepackage{fontspec}
%%   \setmainfont{DejaVu Serif}
%%   \setsansfont{DejaVu Sans}
%%   \setmonofont{DejaVu Sans Mono}
%%
\begingroup%
\makeatletter%
\begin{pgfpicture}%
\pgfpathrectangle{\pgfpointorigin}{\pgfqpoint{2.676066in}{1.736191in}}%
\pgfusepath{use as bounding box, clip}%
\begin{pgfscope}%
\pgfsetbuttcap%
\pgfsetmiterjoin%
\definecolor{currentfill}{rgb}{1.000000,1.000000,1.000000}%
\pgfsetfillcolor{currentfill}%
\pgfsetlinewidth{0.000000pt}%
\definecolor{currentstroke}{rgb}{1.000000,1.000000,1.000000}%
\pgfsetstrokecolor{currentstroke}%
\pgfsetdash{}{0pt}%
\pgfpathmoveto{\pgfqpoint{0.000000in}{0.000000in}}%
\pgfpathlineto{\pgfqpoint{2.676066in}{0.000000in}}%
\pgfpathlineto{\pgfqpoint{2.676066in}{1.736191in}}%
\pgfpathlineto{\pgfqpoint{0.000000in}{1.736191in}}%
\pgfpathclose%
\pgfusepath{fill}%
\end{pgfscope}%
\begin{pgfscope}%
\pgfsetbuttcap%
\pgfsetmiterjoin%
\definecolor{currentfill}{rgb}{1.000000,1.000000,1.000000}%
\pgfsetfillcolor{currentfill}%
\pgfsetlinewidth{0.000000pt}%
\definecolor{currentstroke}{rgb}{0.000000,0.000000,0.000000}%
\pgfsetstrokecolor{currentstroke}%
\pgfsetstrokeopacity{0.000000}%
\pgfsetdash{}{0pt}%
\pgfpathmoveto{\pgfqpoint{0.331521in}{0.422640in}}%
\pgfpathlineto{\pgfqpoint{2.467022in}{0.422640in}}%
\pgfpathlineto{\pgfqpoint{2.467022in}{1.632426in}}%
\pgfpathlineto{\pgfqpoint{0.331521in}{1.632426in}}%
\pgfpathclose%
\pgfusepath{fill}%
\end{pgfscope}%
\begin{pgfscope}%
\pgfpathrectangle{\pgfqpoint{0.331521in}{0.422640in}}{\pgfqpoint{2.135501in}{1.209786in}} %
\pgfusepath{clip}%
\pgfsetbuttcap%
\pgfsetmiterjoin%
\definecolor{currentfill}{rgb}{0.215686,0.470588,0.749020}%
\pgfsetfillcolor{currentfill}%
\pgfsetlinewidth{0.000000pt}%
\definecolor{currentstroke}{rgb}{0.000000,0.000000,0.000000}%
\pgfsetstrokecolor{currentstroke}%
\pgfsetdash{}{0pt}%
\pgfpathmoveto{\pgfqpoint{0.331521in}{0.422640in}}%
\pgfpathlineto{\pgfqpoint{0.331521in}{0.992450in}}%
\pgfpathlineto{\pgfqpoint{0.374231in}{0.992450in}}%
\pgfpathlineto{\pgfqpoint{0.374231in}{1.450049in}}%
\pgfpathlineto{\pgfqpoint{0.416941in}{1.450049in}}%
\pgfpathlineto{\pgfqpoint{0.416941in}{1.477621in}}%
\pgfpathlineto{\pgfqpoint{0.459651in}{1.477621in}}%
\pgfpathlineto{\pgfqpoint{0.459651in}{1.373199in}}%
\pgfpathlineto{\pgfqpoint{0.502361in}{1.373199in}}%
\pgfpathlineto{\pgfqpoint{0.502361in}{1.266465in}}%
\pgfpathlineto{\pgfqpoint{0.545071in}{1.266465in}}%
\pgfpathlineto{\pgfqpoint{0.545071in}{1.157452in}}%
\pgfpathlineto{\pgfqpoint{0.587781in}{1.157452in}}%
\pgfpathlineto{\pgfqpoint{0.587781in}{1.060593in}}%
\pgfpathlineto{\pgfqpoint{0.630491in}{1.060593in}}%
\pgfpathlineto{\pgfqpoint{0.630491in}{0.964404in}}%
\pgfpathlineto{\pgfqpoint{0.673201in}{0.964404in}}%
\pgfpathlineto{\pgfqpoint{0.673201in}{0.904693in}}%
\pgfpathlineto{\pgfqpoint{0.715911in}{0.904693in}}%
\pgfpathlineto{\pgfqpoint{0.715911in}{0.841032in}}%
\pgfpathlineto{\pgfqpoint{0.758621in}{0.841032in}}%
\pgfpathlineto{\pgfqpoint{0.758621in}{0.783470in}}%
\pgfpathlineto{\pgfqpoint{0.801331in}{0.783470in}}%
\pgfpathlineto{\pgfqpoint{0.801331in}{0.741723in}}%
\pgfpathlineto{\pgfqpoint{0.844041in}{0.741723in}}%
\pgfpathlineto{\pgfqpoint{0.844041in}{0.714450in}}%
\pgfpathlineto{\pgfqpoint{0.886751in}{0.714450in}}%
\pgfpathlineto{\pgfqpoint{0.886751in}{0.675116in}}%
\pgfpathlineto{\pgfqpoint{0.929461in}{0.675116in}}%
\pgfpathlineto{\pgfqpoint{0.929461in}{0.648870in}}%
\pgfpathlineto{\pgfqpoint{0.972171in}{0.648870in}}%
\pgfpathlineto{\pgfqpoint{0.972171in}{0.618525in}}%
\pgfpathlineto{\pgfqpoint{1.014881in}{0.618525in}}%
\pgfpathlineto{\pgfqpoint{1.014881in}{0.598956in}}%
\pgfpathlineto{\pgfqpoint{1.057591in}{0.598956in}}%
\pgfpathlineto{\pgfqpoint{1.057591in}{0.581992in}}%
\pgfpathlineto{\pgfqpoint{1.100301in}{0.581992in}}%
\pgfpathlineto{\pgfqpoint{1.100301in}{0.565382in}}%
\pgfpathlineto{\pgfqpoint{1.143011in}{0.565382in}}%
\pgfpathlineto{\pgfqpoint{1.143011in}{0.552918in}}%
\pgfpathlineto{\pgfqpoint{1.185721in}{0.552918in}}%
\pgfpathlineto{\pgfqpoint{1.185721in}{0.542289in}}%
\pgfpathlineto{\pgfqpoint{1.228432in}{0.542289in}}%
\pgfpathlineto{\pgfqpoint{1.228432in}{0.527567in}}%
\pgfpathlineto{\pgfqpoint{1.271142in}{0.527567in}}%
\pgfpathlineto{\pgfqpoint{1.271142in}{0.513870in}}%
\pgfpathlineto{\pgfqpoint{1.313852in}{0.513870in}}%
\pgfpathlineto{\pgfqpoint{1.313852in}{0.513412in}}%
\pgfpathlineto{\pgfqpoint{1.356562in}{0.513412in}}%
\pgfpathlineto{\pgfqpoint{1.356562in}{0.502206in}}%
\pgfpathlineto{\pgfqpoint{1.399272in}{0.502206in}}%
\pgfpathlineto{\pgfqpoint{1.399272in}{0.494527in}}%
\pgfpathlineto{\pgfqpoint{1.441982in}{0.494527in}}%
\pgfpathlineto{\pgfqpoint{1.441982in}{0.488181in}}%
\pgfpathlineto{\pgfqpoint{1.484692in}{0.488181in}}%
\pgfpathlineto{\pgfqpoint{1.484692in}{0.483718in}}%
\pgfpathlineto{\pgfqpoint{1.527402in}{0.483718in}}%
\pgfpathlineto{\pgfqpoint{1.527402in}{0.478355in}}%
\pgfpathlineto{\pgfqpoint{1.570112in}{0.478355in}}%
\pgfpathlineto{\pgfqpoint{1.570112in}{0.472356in}}%
\pgfpathlineto{\pgfqpoint{1.612822in}{0.472356in}}%
\pgfpathlineto{\pgfqpoint{1.612822in}{0.470254in}}%
\pgfpathlineto{\pgfqpoint{1.655532in}{0.470254in}}%
\pgfpathlineto{\pgfqpoint{1.655532in}{0.463403in}}%
\pgfpathlineto{\pgfqpoint{1.698242in}{0.463403in}}%
\pgfpathlineto{\pgfqpoint{1.698242in}{0.464247in}}%
\pgfpathlineto{\pgfqpoint{1.740952in}{0.464247in}}%
\pgfpathlineto{\pgfqpoint{1.740952in}{0.460617in}}%
\pgfpathlineto{\pgfqpoint{1.783662in}{0.460617in}}%
\pgfpathlineto{\pgfqpoint{1.783662in}{0.455716in}}%
\pgfpathlineto{\pgfqpoint{1.826372in}{0.455716in}}%
\pgfpathlineto{\pgfqpoint{1.826372in}{0.453128in}}%
\pgfpathlineto{\pgfqpoint{1.869082in}{0.453128in}}%
\pgfpathlineto{\pgfqpoint{1.869082in}{0.453023in}}%
\pgfpathlineto{\pgfqpoint{1.911792in}{0.453023in}}%
\pgfpathlineto{\pgfqpoint{1.911792in}{0.449010in}}%
\pgfpathlineto{\pgfqpoint{1.954502in}{0.449010in}}%
\pgfpathlineto{\pgfqpoint{1.954502in}{0.448258in}}%
\pgfpathlineto{\pgfqpoint{1.997212in}{0.448258in}}%
\pgfpathlineto{\pgfqpoint{1.997212in}{0.447435in}}%
\pgfpathlineto{\pgfqpoint{2.039922in}{0.447435in}}%
\pgfpathlineto{\pgfqpoint{2.039922in}{0.443599in}}%
\pgfpathlineto{\pgfqpoint{2.082632in}{0.443599in}}%
\pgfpathlineto{\pgfqpoint{2.082632in}{0.442377in}}%
\pgfpathlineto{\pgfqpoint{2.125342in}{0.442377in}}%
\pgfpathlineto{\pgfqpoint{2.125342in}{0.441574in}}%
\pgfpathlineto{\pgfqpoint{2.168052in}{0.441574in}}%
\pgfpathlineto{\pgfqpoint{2.168052in}{0.439905in}}%
\pgfpathlineto{\pgfqpoint{2.210762in}{0.439905in}}%
\pgfpathlineto{\pgfqpoint{2.210762in}{0.439445in}}%
\pgfpathlineto{\pgfqpoint{2.253472in}{0.439445in}}%
\pgfpathlineto{\pgfqpoint{2.253472in}{0.437549in}}%
\pgfpathlineto{\pgfqpoint{2.296182in}{0.437549in}}%
\pgfpathlineto{\pgfqpoint{2.296182in}{0.435060in}}%
\pgfpathlineto{\pgfqpoint{2.338892in}{0.435060in}}%
\pgfpathlineto{\pgfqpoint{2.338892in}{0.435245in}}%
\pgfpathlineto{\pgfqpoint{2.381602in}{0.435245in}}%
\pgfpathlineto{\pgfqpoint{2.381602in}{0.434152in}}%
\pgfpathlineto{\pgfqpoint{2.424312in}{0.434152in}}%
\pgfpathlineto{\pgfqpoint{2.424312in}{0.433825in}}%
\pgfpathlineto{\pgfqpoint{2.467022in}{0.433825in}}%
\pgfpathlineto{\pgfqpoint{2.467022in}{0.422640in}}%
\pgfpathlineto{\pgfqpoint{2.424312in}{0.422640in}}%
\pgfpathlineto{\pgfqpoint{2.424312in}{0.422640in}}%
\pgfpathlineto{\pgfqpoint{2.381602in}{0.422640in}}%
\pgfpathlineto{\pgfqpoint{2.381602in}{0.422640in}}%
\pgfpathlineto{\pgfqpoint{2.338892in}{0.422640in}}%
\pgfpathlineto{\pgfqpoint{2.338892in}{0.422640in}}%
\pgfpathlineto{\pgfqpoint{2.296182in}{0.422640in}}%
\pgfpathlineto{\pgfqpoint{2.296182in}{0.422640in}}%
\pgfpathlineto{\pgfqpoint{2.253472in}{0.422640in}}%
\pgfpathlineto{\pgfqpoint{2.253472in}{0.422640in}}%
\pgfpathlineto{\pgfqpoint{2.210762in}{0.422640in}}%
\pgfpathlineto{\pgfqpoint{2.210762in}{0.422640in}}%
\pgfpathlineto{\pgfqpoint{2.168052in}{0.422640in}}%
\pgfpathlineto{\pgfqpoint{2.168052in}{0.422640in}}%
\pgfpathlineto{\pgfqpoint{2.125342in}{0.422640in}}%
\pgfpathlineto{\pgfqpoint{2.125342in}{0.422640in}}%
\pgfpathlineto{\pgfqpoint{2.082632in}{0.422640in}}%
\pgfpathlineto{\pgfqpoint{2.082632in}{0.422640in}}%
\pgfpathlineto{\pgfqpoint{2.039922in}{0.422640in}}%
\pgfpathlineto{\pgfqpoint{2.039922in}{0.422640in}}%
\pgfpathlineto{\pgfqpoint{1.997212in}{0.422640in}}%
\pgfpathlineto{\pgfqpoint{1.997212in}{0.422640in}}%
\pgfpathlineto{\pgfqpoint{1.954502in}{0.422640in}}%
\pgfpathlineto{\pgfqpoint{1.954502in}{0.422640in}}%
\pgfpathlineto{\pgfqpoint{1.911792in}{0.422640in}}%
\pgfpathlineto{\pgfqpoint{1.911792in}{0.422640in}}%
\pgfpathlineto{\pgfqpoint{1.869082in}{0.422640in}}%
\pgfpathlineto{\pgfqpoint{1.869082in}{0.422640in}}%
\pgfpathlineto{\pgfqpoint{1.826372in}{0.422640in}}%
\pgfpathlineto{\pgfqpoint{1.826372in}{0.422640in}}%
\pgfpathlineto{\pgfqpoint{1.783662in}{0.422640in}}%
\pgfpathlineto{\pgfqpoint{1.783662in}{0.422640in}}%
\pgfpathlineto{\pgfqpoint{1.740952in}{0.422640in}}%
\pgfpathlineto{\pgfqpoint{1.740952in}{0.422640in}}%
\pgfpathlineto{\pgfqpoint{1.698242in}{0.422640in}}%
\pgfpathlineto{\pgfqpoint{1.698242in}{0.422640in}}%
\pgfpathlineto{\pgfqpoint{1.655532in}{0.422640in}}%
\pgfpathlineto{\pgfqpoint{1.655532in}{0.422640in}}%
\pgfpathlineto{\pgfqpoint{1.612822in}{0.422640in}}%
\pgfpathlineto{\pgfqpoint{1.612822in}{0.422640in}}%
\pgfpathlineto{\pgfqpoint{1.570112in}{0.422640in}}%
\pgfpathlineto{\pgfqpoint{1.570112in}{0.422640in}}%
\pgfpathlineto{\pgfqpoint{1.527402in}{0.422640in}}%
\pgfpathlineto{\pgfqpoint{1.527402in}{0.422640in}}%
\pgfpathlineto{\pgfqpoint{1.484692in}{0.422640in}}%
\pgfpathlineto{\pgfqpoint{1.484692in}{0.422640in}}%
\pgfpathlineto{\pgfqpoint{1.441982in}{0.422640in}}%
\pgfpathlineto{\pgfqpoint{1.441982in}{0.422640in}}%
\pgfpathlineto{\pgfqpoint{1.399272in}{0.422640in}}%
\pgfpathlineto{\pgfqpoint{1.399272in}{0.422640in}}%
\pgfpathlineto{\pgfqpoint{1.356562in}{0.422640in}}%
\pgfpathlineto{\pgfqpoint{1.356562in}{0.422640in}}%
\pgfpathlineto{\pgfqpoint{1.313852in}{0.422640in}}%
\pgfpathlineto{\pgfqpoint{1.313852in}{0.422640in}}%
\pgfpathlineto{\pgfqpoint{1.271142in}{0.422640in}}%
\pgfpathlineto{\pgfqpoint{1.271142in}{0.422640in}}%
\pgfpathlineto{\pgfqpoint{1.228432in}{0.422640in}}%
\pgfpathlineto{\pgfqpoint{1.228432in}{0.422640in}}%
\pgfpathlineto{\pgfqpoint{1.185721in}{0.422640in}}%
\pgfpathlineto{\pgfqpoint{1.185721in}{0.422640in}}%
\pgfpathlineto{\pgfqpoint{1.143011in}{0.422640in}}%
\pgfpathlineto{\pgfqpoint{1.143011in}{0.422640in}}%
\pgfpathlineto{\pgfqpoint{1.100301in}{0.422640in}}%
\pgfpathlineto{\pgfqpoint{1.100301in}{0.422640in}}%
\pgfpathlineto{\pgfqpoint{1.057591in}{0.422640in}}%
\pgfpathlineto{\pgfqpoint{1.057591in}{0.422640in}}%
\pgfpathlineto{\pgfqpoint{1.014881in}{0.422640in}}%
\pgfpathlineto{\pgfqpoint{1.014881in}{0.422640in}}%
\pgfpathlineto{\pgfqpoint{0.972171in}{0.422640in}}%
\pgfpathlineto{\pgfqpoint{0.972171in}{0.422640in}}%
\pgfpathlineto{\pgfqpoint{0.929461in}{0.422640in}}%
\pgfpathlineto{\pgfqpoint{0.929461in}{0.422640in}}%
\pgfpathlineto{\pgfqpoint{0.886751in}{0.422640in}}%
\pgfpathlineto{\pgfqpoint{0.886751in}{0.422640in}}%
\pgfpathlineto{\pgfqpoint{0.844041in}{0.422640in}}%
\pgfpathlineto{\pgfqpoint{0.844041in}{0.422640in}}%
\pgfpathlineto{\pgfqpoint{0.801331in}{0.422640in}}%
\pgfpathlineto{\pgfqpoint{0.801331in}{0.422640in}}%
\pgfpathlineto{\pgfqpoint{0.758621in}{0.422640in}}%
\pgfpathlineto{\pgfqpoint{0.758621in}{0.422640in}}%
\pgfpathlineto{\pgfqpoint{0.715911in}{0.422640in}}%
\pgfpathlineto{\pgfqpoint{0.715911in}{0.422640in}}%
\pgfpathlineto{\pgfqpoint{0.673201in}{0.422640in}}%
\pgfpathlineto{\pgfqpoint{0.673201in}{0.422640in}}%
\pgfpathlineto{\pgfqpoint{0.630491in}{0.422640in}}%
\pgfpathlineto{\pgfqpoint{0.630491in}{0.422640in}}%
\pgfpathlineto{\pgfqpoint{0.587781in}{0.422640in}}%
\pgfpathlineto{\pgfqpoint{0.587781in}{0.422640in}}%
\pgfpathlineto{\pgfqpoint{0.545071in}{0.422640in}}%
\pgfpathlineto{\pgfqpoint{0.545071in}{0.422640in}}%
\pgfpathlineto{\pgfqpoint{0.502361in}{0.422640in}}%
\pgfpathlineto{\pgfqpoint{0.502361in}{0.422640in}}%
\pgfpathlineto{\pgfqpoint{0.459651in}{0.422640in}}%
\pgfpathlineto{\pgfqpoint{0.459651in}{0.422640in}}%
\pgfpathlineto{\pgfqpoint{0.416941in}{0.422640in}}%
\pgfpathlineto{\pgfqpoint{0.416941in}{0.422640in}}%
\pgfpathlineto{\pgfqpoint{0.374231in}{0.422640in}}%
\pgfpathlineto{\pgfqpoint{0.374231in}{0.422640in}}%
\pgfpathlineto{\pgfqpoint{0.331521in}{0.422640in}}%
\pgfusepath{fill}%
\end{pgfscope}%
\begin{pgfscope}%
\pgfpathrectangle{\pgfqpoint{0.331521in}{0.422640in}}{\pgfqpoint{2.135501in}{1.209786in}} %
\pgfusepath{clip}%
\pgfsetbuttcap%
\pgfsetmiterjoin%
\pgfsetlinewidth{0.501875pt}%
\definecolor{currentstroke}{rgb}{1.000000,0.000000,0.000000}%
\pgfsetstrokecolor{currentstroke}%
\pgfsetdash{}{0pt}%
\pgfpathmoveto{\pgfqpoint{0.331521in}{0.422640in}}%
\pgfpathlineto{\pgfqpoint{0.331521in}{1.071669in}}%
\pgfpathlineto{\pgfqpoint{0.374231in}{1.071669in}}%
\pgfpathlineto{\pgfqpoint{0.374231in}{1.554373in}}%
\pgfpathlineto{\pgfqpoint{0.416941in}{1.554373in}}%
\pgfpathlineto{\pgfqpoint{0.416941in}{1.529160in}}%
\pgfpathlineto{\pgfqpoint{0.459651in}{1.529160in}}%
\pgfpathlineto{\pgfqpoint{0.459651in}{1.400656in}}%
\pgfpathlineto{\pgfqpoint{0.502361in}{1.400656in}}%
\pgfpathlineto{\pgfqpoint{0.502361in}{1.276625in}}%
\pgfpathlineto{\pgfqpoint{0.545071in}{1.276625in}}%
\pgfpathlineto{\pgfqpoint{0.545071in}{1.158477in}}%
\pgfpathlineto{\pgfqpoint{0.587781in}{1.158477in}}%
\pgfpathlineto{\pgfqpoint{0.587781in}{1.045127in}}%
\pgfpathlineto{\pgfqpoint{0.630491in}{1.045127in}}%
\pgfpathlineto{\pgfqpoint{0.630491in}{0.962521in}}%
\pgfpathlineto{\pgfqpoint{0.673201in}{0.962521in}}%
\pgfpathlineto{\pgfqpoint{0.673201in}{0.885690in}}%
\pgfpathlineto{\pgfqpoint{0.715911in}{0.885690in}}%
\pgfpathlineto{\pgfqpoint{0.715911in}{0.824420in}}%
\pgfpathlineto{\pgfqpoint{0.758621in}{0.824420in}}%
\pgfpathlineto{\pgfqpoint{0.758621in}{0.776678in}}%
\pgfpathlineto{\pgfqpoint{0.801331in}{0.776678in}}%
\pgfpathlineto{\pgfqpoint{0.801331in}{0.734331in}}%
\pgfpathlineto{\pgfqpoint{0.844041in}{0.734331in}}%
\pgfpathlineto{\pgfqpoint{0.844041in}{0.692174in}}%
\pgfpathlineto{\pgfqpoint{0.886751in}{0.692174in}}%
\pgfpathlineto{\pgfqpoint{0.886751in}{0.659994in}}%
\pgfpathlineto{\pgfqpoint{0.929461in}{0.659994in}}%
\pgfpathlineto{\pgfqpoint{0.929461in}{0.632097in}}%
\pgfpathlineto{\pgfqpoint{0.972171in}{0.632097in}}%
\pgfpathlineto{\pgfqpoint{0.972171in}{0.611086in}}%
\pgfpathlineto{\pgfqpoint{1.014881in}{0.611086in}}%
\pgfpathlineto{\pgfqpoint{1.014881in}{0.592841in}}%
\pgfpathlineto{\pgfqpoint{1.057591in}{0.592841in}}%
\pgfpathlineto{\pgfqpoint{1.057591in}{0.570474in}}%
\pgfpathlineto{\pgfqpoint{1.100301in}{0.570474in}}%
\pgfpathlineto{\pgfqpoint{1.100301in}{0.558058in}}%
\pgfpathlineto{\pgfqpoint{1.143011in}{0.558058in}}%
\pgfpathlineto{\pgfqpoint{1.143011in}{0.543038in}}%
\pgfpathlineto{\pgfqpoint{1.185721in}{0.543038in}}%
\pgfpathlineto{\pgfqpoint{1.185721in}{0.532248in}}%
\pgfpathlineto{\pgfqpoint{1.228432in}{0.532248in}}%
\pgfpathlineto{\pgfqpoint{1.228432in}{0.518313in}}%
\pgfpathlineto{\pgfqpoint{1.271142in}{0.518313in}}%
\pgfpathlineto{\pgfqpoint{1.271142in}{0.512810in}}%
\pgfpathlineto{\pgfqpoint{1.313852in}{0.512810in}}%
\pgfpathlineto{\pgfqpoint{1.313852in}{0.505056in}}%
\pgfpathlineto{\pgfqpoint{1.356562in}{0.505056in}}%
\pgfpathlineto{\pgfqpoint{1.356562in}{0.494700in}}%
\pgfpathlineto{\pgfqpoint{1.399272in}{0.494700in}}%
\pgfpathlineto{\pgfqpoint{1.399272in}{0.488654in}}%
\pgfpathlineto{\pgfqpoint{1.441982in}{0.488654in}}%
\pgfpathlineto{\pgfqpoint{1.441982in}{0.481362in}}%
\pgfpathlineto{\pgfqpoint{1.484692in}{0.481362in}}%
\pgfpathlineto{\pgfqpoint{1.484692in}{0.478217in}}%
\pgfpathlineto{\pgfqpoint{1.527402in}{0.478217in}}%
\pgfpathlineto{\pgfqpoint{1.527402in}{0.474340in}}%
\pgfpathlineto{\pgfqpoint{1.570112in}{0.474340in}}%
\pgfpathlineto{\pgfqpoint{1.570112in}{0.467861in}}%
\pgfpathlineto{\pgfqpoint{1.612822in}{0.467861in}}%
\pgfpathlineto{\pgfqpoint{1.612822in}{0.465610in}}%
\pgfpathlineto{\pgfqpoint{1.655532in}{0.465610in}}%
\pgfpathlineto{\pgfqpoint{1.655532in}{0.462601in}}%
\pgfpathlineto{\pgfqpoint{1.698242in}{0.462601in}}%
\pgfpathlineto{\pgfqpoint{1.698242in}{0.460242in}}%
\pgfpathlineto{\pgfqpoint{1.740952in}{0.460242in}}%
\pgfpathlineto{\pgfqpoint{1.740952in}{0.456203in}}%
\pgfpathlineto{\pgfqpoint{1.783662in}{0.456203in}}%
\pgfpathlineto{\pgfqpoint{1.783662in}{0.451974in}}%
\pgfpathlineto{\pgfqpoint{1.826372in}{0.451974in}}%
\pgfpathlineto{\pgfqpoint{1.826372in}{0.451513in}}%
\pgfpathlineto{\pgfqpoint{1.869082in}{0.451513in}}%
\pgfpathlineto{\pgfqpoint{1.869082in}{0.449588in}}%
\pgfpathlineto{\pgfqpoint{1.911792in}{0.449588in}}%
\pgfpathlineto{\pgfqpoint{1.911792in}{0.446687in}}%
\pgfpathlineto{\pgfqpoint{1.954502in}{0.446687in}}%
\pgfpathlineto{\pgfqpoint{1.954502in}{0.445006in}}%
\pgfpathlineto{\pgfqpoint{1.997212in}{0.445006in}}%
\pgfpathlineto{\pgfqpoint{1.997212in}{0.442946in}}%
\pgfpathlineto{\pgfqpoint{2.039922in}{0.442946in}}%
\pgfpathlineto{\pgfqpoint{2.039922in}{0.442268in}}%
\pgfpathlineto{\pgfqpoint{2.082632in}{0.442268in}}%
\pgfpathlineto{\pgfqpoint{2.082632in}{0.440072in}}%
\pgfpathlineto{\pgfqpoint{2.125342in}{0.440072in}}%
\pgfpathlineto{\pgfqpoint{2.125342in}{0.438391in}}%
\pgfpathlineto{\pgfqpoint{2.168052in}{0.438391in}}%
\pgfpathlineto{\pgfqpoint{2.168052in}{0.439150in}}%
\pgfpathlineto{\pgfqpoint{2.210762in}{0.439150in}}%
\pgfpathlineto{\pgfqpoint{2.210762in}{0.436656in}}%
\pgfpathlineto{\pgfqpoint{2.253472in}{0.436656in}}%
\pgfpathlineto{\pgfqpoint{2.253472in}{0.435897in}}%
\pgfpathlineto{\pgfqpoint{2.296182in}{0.435897in}}%
\pgfpathlineto{\pgfqpoint{2.296182in}{0.435762in}}%
\pgfpathlineto{\pgfqpoint{2.338892in}{0.435762in}}%
\pgfpathlineto{\pgfqpoint{2.338892in}{0.433864in}}%
\pgfpathlineto{\pgfqpoint{2.381602in}{0.433864in}}%
\pgfpathlineto{\pgfqpoint{2.381602in}{0.433837in}}%
\pgfpathlineto{\pgfqpoint{2.424312in}{0.433837in}}%
\pgfpathlineto{\pgfqpoint{2.424312in}{0.432346in}}%
\pgfpathlineto{\pgfqpoint{2.467022in}{0.432346in}}%
\pgfpathlineto{\pgfqpoint{2.467022in}{0.422640in}}%
\pgfusepath{stroke}%
\end{pgfscope}%
\begin{pgfscope}%
\pgfpathrectangle{\pgfqpoint{0.331521in}{0.422640in}}{\pgfqpoint{2.135501in}{1.209786in}} %
\pgfusepath{clip}%
\pgfsetbuttcap%
\pgfsetmiterjoin%
\pgfsetlinewidth{0.501875pt}%
\definecolor{currentstroke}{rgb}{1.000000,0.647059,0.000000}%
\pgfsetstrokecolor{currentstroke}%
\pgfsetdash{}{0pt}%
\pgfpathmoveto{\pgfqpoint{0.331521in}{0.422640in}}%
\pgfpathlineto{\pgfqpoint{0.331521in}{1.096304in}}%
\pgfpathlineto{\pgfqpoint{0.374231in}{1.096304in}}%
\pgfpathlineto{\pgfqpoint{0.374231in}{1.591619in}}%
\pgfpathlineto{\pgfqpoint{0.416941in}{1.591619in}}%
\pgfpathlineto{\pgfqpoint{0.416941in}{1.551414in}}%
\pgfpathlineto{\pgfqpoint{0.459651in}{1.551414in}}%
\pgfpathlineto{\pgfqpoint{0.459651in}{1.392299in}}%
\pgfpathlineto{\pgfqpoint{0.502361in}{1.392299in}}%
\pgfpathlineto{\pgfqpoint{0.502361in}{1.284187in}}%
\pgfpathlineto{\pgfqpoint{0.545071in}{1.284187in}}%
\pgfpathlineto{\pgfqpoint{0.545071in}{1.155510in}}%
\pgfpathlineto{\pgfqpoint{0.587781in}{1.155510in}}%
\pgfpathlineto{\pgfqpoint{0.587781in}{1.032930in}}%
\pgfpathlineto{\pgfqpoint{0.630491in}{1.032930in}}%
\pgfpathlineto{\pgfqpoint{0.630491in}{0.948359in}}%
\pgfpathlineto{\pgfqpoint{0.673201in}{0.948359in}}%
\pgfpathlineto{\pgfqpoint{0.673201in}{0.884376in}}%
\pgfpathlineto{\pgfqpoint{0.715911in}{0.884376in}}%
\pgfpathlineto{\pgfqpoint{0.715911in}{0.829055in}}%
\pgfpathlineto{\pgfqpoint{0.758621in}{0.829055in}}%
\pgfpathlineto{\pgfqpoint{0.758621in}{0.788697in}}%
\pgfpathlineto{\pgfqpoint{0.801331in}{0.788697in}}%
\pgfpathlineto{\pgfqpoint{0.801331in}{0.724468in}}%
\pgfpathlineto{\pgfqpoint{0.844041in}{0.724468in}}%
\pgfpathlineto{\pgfqpoint{0.844041in}{0.680965in}}%
\pgfpathlineto{\pgfqpoint{0.886751in}{0.680965in}}%
\pgfpathlineto{\pgfqpoint{0.886751in}{0.655774in}}%
\pgfpathlineto{\pgfqpoint{0.929461in}{0.655774in}}%
\pgfpathlineto{\pgfqpoint{0.929461in}{0.621767in}}%
\pgfpathlineto{\pgfqpoint{0.972171in}{0.621767in}}%
\pgfpathlineto{\pgfqpoint{0.972171in}{0.613370in}}%
\pgfpathlineto{\pgfqpoint{1.014881in}{0.613370in}}%
\pgfpathlineto{\pgfqpoint{1.014881in}{0.587942in}}%
\pgfpathlineto{\pgfqpoint{1.057591in}{0.587942in}}%
\pgfpathlineto{\pgfqpoint{1.057591in}{0.565328in}}%
\pgfpathlineto{\pgfqpoint{1.100301in}{0.565328in}}%
\pgfpathlineto{\pgfqpoint{1.100301in}{0.557174in}}%
\pgfpathlineto{\pgfqpoint{1.143011in}{0.557174in}}%
\pgfpathlineto{\pgfqpoint{1.143011in}{0.536626in}}%
\pgfpathlineto{\pgfqpoint{1.185721in}{0.536626in}}%
\pgfpathlineto{\pgfqpoint{1.185721in}{0.533138in}}%
\pgfpathlineto{\pgfqpoint{1.228432in}{0.533138in}}%
\pgfpathlineto{\pgfqpoint{1.228432in}{0.516696in}}%
\pgfpathlineto{\pgfqpoint{1.271142in}{0.516696in}}%
\pgfpathlineto{\pgfqpoint{1.271142in}{0.511545in}}%
\pgfpathlineto{\pgfqpoint{1.313852in}{0.511545in}}%
\pgfpathlineto{\pgfqpoint{1.313852in}{0.506178in}}%
\pgfpathlineto{\pgfqpoint{1.356562in}{0.506178in}}%
\pgfpathlineto{\pgfqpoint{1.356562in}{0.494448in}}%
\pgfpathlineto{\pgfqpoint{1.399272in}{0.494448in}}%
\pgfpathlineto{\pgfqpoint{1.399272in}{0.484610in}}%
\pgfpathlineto{\pgfqpoint{1.441982in}{0.484610in}}%
\pgfpathlineto{\pgfqpoint{1.441982in}{0.477287in}}%
\pgfpathlineto{\pgfqpoint{1.484692in}{0.477287in}}%
\pgfpathlineto{\pgfqpoint{1.484692in}{0.471428in}}%
\pgfpathlineto{\pgfqpoint{1.527402in}{0.471428in}}%
\pgfpathlineto{\pgfqpoint{1.527402in}{0.472561in}}%
\pgfpathlineto{\pgfqpoint{1.570112in}{0.472561in}}%
\pgfpathlineto{\pgfqpoint{1.570112in}{0.467019in}}%
\pgfpathlineto{\pgfqpoint{1.612822in}{0.467019in}}%
\pgfpathlineto{\pgfqpoint{1.612822in}{0.466004in}}%
\pgfpathlineto{\pgfqpoint{1.655532in}{0.466004in}}%
\pgfpathlineto{\pgfqpoint{1.655532in}{0.465236in}}%
\pgfpathlineto{\pgfqpoint{1.698242in}{0.465236in}}%
\pgfpathlineto{\pgfqpoint{1.698242in}{0.460160in}}%
\pgfpathlineto{\pgfqpoint{1.740952in}{0.460160in}}%
\pgfpathlineto{\pgfqpoint{1.740952in}{0.451917in}}%
\pgfpathlineto{\pgfqpoint{1.783662in}{0.451917in}}%
\pgfpathlineto{\pgfqpoint{1.783662in}{0.448700in}}%
\pgfpathlineto{\pgfqpoint{1.826372in}{0.448700in}}%
\pgfpathlineto{\pgfqpoint{1.826372in}{0.451687in}}%
\pgfpathlineto{\pgfqpoint{1.869082in}{0.451687in}}%
\pgfpathlineto{\pgfqpoint{1.869082in}{0.449845in}}%
\pgfpathlineto{\pgfqpoint{1.911792in}{0.449845in}}%
\pgfpathlineto{\pgfqpoint{1.911792in}{0.446814in}}%
\pgfpathlineto{\pgfqpoint{1.954502in}{0.446814in}}%
\pgfpathlineto{\pgfqpoint{1.954502in}{0.448487in}}%
\pgfpathlineto{\pgfqpoint{1.997212in}{0.448487in}}%
\pgfpathlineto{\pgfqpoint{1.997212in}{0.441339in}}%
\pgfpathlineto{\pgfqpoint{2.039922in}{0.441339in}}%
\pgfpathlineto{\pgfqpoint{2.039922in}{0.440418in}}%
\pgfpathlineto{\pgfqpoint{2.082632in}{0.440418in}}%
\pgfpathlineto{\pgfqpoint{2.082632in}{0.439915in}}%
\pgfpathlineto{\pgfqpoint{2.125342in}{0.439915in}}%
\pgfpathlineto{\pgfqpoint{2.125342in}{0.437235in}}%
\pgfpathlineto{\pgfqpoint{2.168052in}{0.437235in}}%
\pgfpathlineto{\pgfqpoint{2.168052in}{0.439553in}}%
\pgfpathlineto{\pgfqpoint{2.210762in}{0.439553in}}%
\pgfpathlineto{\pgfqpoint{2.210762in}{0.437235in}}%
\pgfpathlineto{\pgfqpoint{2.253472in}{0.437235in}}%
\pgfpathlineto{\pgfqpoint{2.253472in}{0.436357in}}%
\pgfpathlineto{\pgfqpoint{2.296182in}{0.436357in}}%
\pgfpathlineto{\pgfqpoint{2.296182in}{0.437677in}}%
\pgfpathlineto{\pgfqpoint{2.338892in}{0.437677in}}%
\pgfpathlineto{\pgfqpoint{2.338892in}{0.434848in}}%
\pgfpathlineto{\pgfqpoint{2.381602in}{0.434848in}}%
\pgfpathlineto{\pgfqpoint{2.381602in}{0.433622in}}%
\pgfpathlineto{\pgfqpoint{2.424312in}{0.433622in}}%
\pgfpathlineto{\pgfqpoint{2.424312in}{0.433537in}}%
\pgfpathlineto{\pgfqpoint{2.467022in}{0.433537in}}%
\pgfpathlineto{\pgfqpoint{2.467022in}{0.422640in}}%
\pgfusepath{stroke}%
\end{pgfscope}%
\begin{pgfscope}%
\pgfsetrectcap%
\pgfsetmiterjoin%
\pgfsetlinewidth{1.003750pt}%
\definecolor{currentstroke}{rgb}{0.000000,0.000000,0.000000}%
\pgfsetstrokecolor{currentstroke}%
\pgfsetdash{}{0pt}%
\pgfpathmoveto{\pgfqpoint{0.331521in}{1.632426in}}%
\pgfpathlineto{\pgfqpoint{2.467022in}{1.632426in}}%
\pgfusepath{stroke}%
\end{pgfscope}%
\begin{pgfscope}%
\pgfsetrectcap%
\pgfsetmiterjoin%
\pgfsetlinewidth{1.003750pt}%
\definecolor{currentstroke}{rgb}{0.000000,0.000000,0.000000}%
\pgfsetstrokecolor{currentstroke}%
\pgfsetdash{}{0pt}%
\pgfpathmoveto{\pgfqpoint{2.467022in}{0.422640in}}%
\pgfpathlineto{\pgfqpoint{2.467022in}{1.632426in}}%
\pgfusepath{stroke}%
\end{pgfscope}%
\begin{pgfscope}%
\pgfsetrectcap%
\pgfsetmiterjoin%
\pgfsetlinewidth{1.003750pt}%
\definecolor{currentstroke}{rgb}{0.000000,0.000000,0.000000}%
\pgfsetstrokecolor{currentstroke}%
\pgfsetdash{}{0pt}%
\pgfpathmoveto{\pgfqpoint{0.331521in}{0.422640in}}%
\pgfpathlineto{\pgfqpoint{2.467022in}{0.422640in}}%
\pgfusepath{stroke}%
\end{pgfscope}%
\begin{pgfscope}%
\pgfsetrectcap%
\pgfsetmiterjoin%
\pgfsetlinewidth{1.003750pt}%
\definecolor{currentstroke}{rgb}{0.000000,0.000000,0.000000}%
\pgfsetstrokecolor{currentstroke}%
\pgfsetdash{}{0pt}%
\pgfpathmoveto{\pgfqpoint{0.331521in}{0.422640in}}%
\pgfpathlineto{\pgfqpoint{0.331521in}{1.632426in}}%
\pgfusepath{stroke}%
\end{pgfscope}%
\begin{pgfscope}%
\pgfsetbuttcap%
\pgfsetroundjoin%
\definecolor{currentfill}{rgb}{0.000000,0.000000,0.000000}%
\pgfsetfillcolor{currentfill}%
\pgfsetlinewidth{0.501875pt}%
\definecolor{currentstroke}{rgb}{0.000000,0.000000,0.000000}%
\pgfsetstrokecolor{currentstroke}%
\pgfsetdash{}{0pt}%
\pgfsys@defobject{currentmarker}{\pgfqpoint{0.000000in}{0.000000in}}{\pgfqpoint{0.000000in}{0.069444in}}{%
\pgfpathmoveto{\pgfqpoint{0.000000in}{0.000000in}}%
\pgfpathlineto{\pgfqpoint{0.000000in}{0.069444in}}%
\pgfusepath{stroke,fill}%
}%
\begin{pgfscope}%
\pgfsys@transformshift{0.331521in}{0.422640in}%
\pgfsys@useobject{currentmarker}{}%
\end{pgfscope}%
\end{pgfscope}%
\begin{pgfscope}%
\pgfsetbuttcap%
\pgfsetroundjoin%
\definecolor{currentfill}{rgb}{0.000000,0.000000,0.000000}%
\pgfsetfillcolor{currentfill}%
\pgfsetlinewidth{0.501875pt}%
\definecolor{currentstroke}{rgb}{0.000000,0.000000,0.000000}%
\pgfsetstrokecolor{currentstroke}%
\pgfsetdash{}{0pt}%
\pgfsys@defobject{currentmarker}{\pgfqpoint{0.000000in}{-0.069444in}}{\pgfqpoint{0.000000in}{0.000000in}}{%
\pgfpathmoveto{\pgfqpoint{0.000000in}{0.000000in}}%
\pgfpathlineto{\pgfqpoint{0.000000in}{-0.069444in}}%
\pgfusepath{stroke,fill}%
}%
\begin{pgfscope}%
\pgfsys@transformshift{0.331521in}{1.632426in}%
\pgfsys@useobject{currentmarker}{}%
\end{pgfscope}%
\end{pgfscope}%
\begin{pgfscope}%
\pgftext[x=0.331521in,y=0.353196in,,top]{\rmfamily\fontsize{8.000000}{9.600000}\selectfont 0.000}%
\end{pgfscope}%
\begin{pgfscope}%
\pgfsetbuttcap%
\pgfsetroundjoin%
\definecolor{currentfill}{rgb}{0.000000,0.000000,0.000000}%
\pgfsetfillcolor{currentfill}%
\pgfsetlinewidth{0.501875pt}%
\definecolor{currentstroke}{rgb}{0.000000,0.000000,0.000000}%
\pgfsetstrokecolor{currentstroke}%
\pgfsetdash{}{0pt}%
\pgfsys@defobject{currentmarker}{\pgfqpoint{0.000000in}{0.000000in}}{\pgfqpoint{0.000000in}{0.069444in}}{%
\pgfpathmoveto{\pgfqpoint{0.000000in}{0.000000in}}%
\pgfpathlineto{\pgfqpoint{0.000000in}{0.069444in}}%
\pgfusepath{stroke,fill}%
}%
\begin{pgfscope}%
\pgfsys@transformshift{0.636593in}{0.422640in}%
\pgfsys@useobject{currentmarker}{}%
\end{pgfscope}%
\end{pgfscope}%
\begin{pgfscope}%
\pgfsetbuttcap%
\pgfsetroundjoin%
\definecolor{currentfill}{rgb}{0.000000,0.000000,0.000000}%
\pgfsetfillcolor{currentfill}%
\pgfsetlinewidth{0.501875pt}%
\definecolor{currentstroke}{rgb}{0.000000,0.000000,0.000000}%
\pgfsetstrokecolor{currentstroke}%
\pgfsetdash{}{0pt}%
\pgfsys@defobject{currentmarker}{\pgfqpoint{0.000000in}{-0.069444in}}{\pgfqpoint{0.000000in}{0.000000in}}{%
\pgfpathmoveto{\pgfqpoint{0.000000in}{0.000000in}}%
\pgfpathlineto{\pgfqpoint{0.000000in}{-0.069444in}}%
\pgfusepath{stroke,fill}%
}%
\begin{pgfscope}%
\pgfsys@transformshift{0.636593in}{1.632426in}%
\pgfsys@useobject{currentmarker}{}%
\end{pgfscope}%
\end{pgfscope}%
\begin{pgfscope}%
\pgftext[x=0.636593in,y=0.353196in,,top]{\rmfamily\fontsize{8.000000}{9.600000}\selectfont 0.002}%
\end{pgfscope}%
\begin{pgfscope}%
\pgfsetbuttcap%
\pgfsetroundjoin%
\definecolor{currentfill}{rgb}{0.000000,0.000000,0.000000}%
\pgfsetfillcolor{currentfill}%
\pgfsetlinewidth{0.501875pt}%
\definecolor{currentstroke}{rgb}{0.000000,0.000000,0.000000}%
\pgfsetstrokecolor{currentstroke}%
\pgfsetdash{}{0pt}%
\pgfsys@defobject{currentmarker}{\pgfqpoint{0.000000in}{0.000000in}}{\pgfqpoint{0.000000in}{0.069444in}}{%
\pgfpathmoveto{\pgfqpoint{0.000000in}{0.000000in}}%
\pgfpathlineto{\pgfqpoint{0.000000in}{0.069444in}}%
\pgfusepath{stroke,fill}%
}%
\begin{pgfscope}%
\pgfsys@transformshift{0.941664in}{0.422640in}%
\pgfsys@useobject{currentmarker}{}%
\end{pgfscope}%
\end{pgfscope}%
\begin{pgfscope}%
\pgfsetbuttcap%
\pgfsetroundjoin%
\definecolor{currentfill}{rgb}{0.000000,0.000000,0.000000}%
\pgfsetfillcolor{currentfill}%
\pgfsetlinewidth{0.501875pt}%
\definecolor{currentstroke}{rgb}{0.000000,0.000000,0.000000}%
\pgfsetstrokecolor{currentstroke}%
\pgfsetdash{}{0pt}%
\pgfsys@defobject{currentmarker}{\pgfqpoint{0.000000in}{-0.069444in}}{\pgfqpoint{0.000000in}{0.000000in}}{%
\pgfpathmoveto{\pgfqpoint{0.000000in}{0.000000in}}%
\pgfpathlineto{\pgfqpoint{0.000000in}{-0.069444in}}%
\pgfusepath{stroke,fill}%
}%
\begin{pgfscope}%
\pgfsys@transformshift{0.941664in}{1.632426in}%
\pgfsys@useobject{currentmarker}{}%
\end{pgfscope}%
\end{pgfscope}%
\begin{pgfscope}%
\pgftext[x=0.941664in,y=0.353196in,,top]{\rmfamily\fontsize{8.000000}{9.600000}\selectfont 0.004}%
\end{pgfscope}%
\begin{pgfscope}%
\pgfsetbuttcap%
\pgfsetroundjoin%
\definecolor{currentfill}{rgb}{0.000000,0.000000,0.000000}%
\pgfsetfillcolor{currentfill}%
\pgfsetlinewidth{0.501875pt}%
\definecolor{currentstroke}{rgb}{0.000000,0.000000,0.000000}%
\pgfsetstrokecolor{currentstroke}%
\pgfsetdash{}{0pt}%
\pgfsys@defobject{currentmarker}{\pgfqpoint{0.000000in}{0.000000in}}{\pgfqpoint{0.000000in}{0.069444in}}{%
\pgfpathmoveto{\pgfqpoint{0.000000in}{0.000000in}}%
\pgfpathlineto{\pgfqpoint{0.000000in}{0.069444in}}%
\pgfusepath{stroke,fill}%
}%
\begin{pgfscope}%
\pgfsys@transformshift{1.246736in}{0.422640in}%
\pgfsys@useobject{currentmarker}{}%
\end{pgfscope}%
\end{pgfscope}%
\begin{pgfscope}%
\pgfsetbuttcap%
\pgfsetroundjoin%
\definecolor{currentfill}{rgb}{0.000000,0.000000,0.000000}%
\pgfsetfillcolor{currentfill}%
\pgfsetlinewidth{0.501875pt}%
\definecolor{currentstroke}{rgb}{0.000000,0.000000,0.000000}%
\pgfsetstrokecolor{currentstroke}%
\pgfsetdash{}{0pt}%
\pgfsys@defobject{currentmarker}{\pgfqpoint{0.000000in}{-0.069444in}}{\pgfqpoint{0.000000in}{0.000000in}}{%
\pgfpathmoveto{\pgfqpoint{0.000000in}{0.000000in}}%
\pgfpathlineto{\pgfqpoint{0.000000in}{-0.069444in}}%
\pgfusepath{stroke,fill}%
}%
\begin{pgfscope}%
\pgfsys@transformshift{1.246736in}{1.632426in}%
\pgfsys@useobject{currentmarker}{}%
\end{pgfscope}%
\end{pgfscope}%
\begin{pgfscope}%
\pgftext[x=1.246736in,y=0.353196in,,top]{\rmfamily\fontsize{8.000000}{9.600000}\selectfont 0.006}%
\end{pgfscope}%
\begin{pgfscope}%
\pgfsetbuttcap%
\pgfsetroundjoin%
\definecolor{currentfill}{rgb}{0.000000,0.000000,0.000000}%
\pgfsetfillcolor{currentfill}%
\pgfsetlinewidth{0.501875pt}%
\definecolor{currentstroke}{rgb}{0.000000,0.000000,0.000000}%
\pgfsetstrokecolor{currentstroke}%
\pgfsetdash{}{0pt}%
\pgfsys@defobject{currentmarker}{\pgfqpoint{0.000000in}{0.000000in}}{\pgfqpoint{0.000000in}{0.069444in}}{%
\pgfpathmoveto{\pgfqpoint{0.000000in}{0.000000in}}%
\pgfpathlineto{\pgfqpoint{0.000000in}{0.069444in}}%
\pgfusepath{stroke,fill}%
}%
\begin{pgfscope}%
\pgfsys@transformshift{1.551807in}{0.422640in}%
\pgfsys@useobject{currentmarker}{}%
\end{pgfscope}%
\end{pgfscope}%
\begin{pgfscope}%
\pgfsetbuttcap%
\pgfsetroundjoin%
\definecolor{currentfill}{rgb}{0.000000,0.000000,0.000000}%
\pgfsetfillcolor{currentfill}%
\pgfsetlinewidth{0.501875pt}%
\definecolor{currentstroke}{rgb}{0.000000,0.000000,0.000000}%
\pgfsetstrokecolor{currentstroke}%
\pgfsetdash{}{0pt}%
\pgfsys@defobject{currentmarker}{\pgfqpoint{0.000000in}{-0.069444in}}{\pgfqpoint{0.000000in}{0.000000in}}{%
\pgfpathmoveto{\pgfqpoint{0.000000in}{0.000000in}}%
\pgfpathlineto{\pgfqpoint{0.000000in}{-0.069444in}}%
\pgfusepath{stroke,fill}%
}%
\begin{pgfscope}%
\pgfsys@transformshift{1.551807in}{1.632426in}%
\pgfsys@useobject{currentmarker}{}%
\end{pgfscope}%
\end{pgfscope}%
\begin{pgfscope}%
\pgftext[x=1.551807in,y=0.353196in,,top]{\rmfamily\fontsize{8.000000}{9.600000}\selectfont 0.008}%
\end{pgfscope}%
\begin{pgfscope}%
\pgfsetbuttcap%
\pgfsetroundjoin%
\definecolor{currentfill}{rgb}{0.000000,0.000000,0.000000}%
\pgfsetfillcolor{currentfill}%
\pgfsetlinewidth{0.501875pt}%
\definecolor{currentstroke}{rgb}{0.000000,0.000000,0.000000}%
\pgfsetstrokecolor{currentstroke}%
\pgfsetdash{}{0pt}%
\pgfsys@defobject{currentmarker}{\pgfqpoint{0.000000in}{0.000000in}}{\pgfqpoint{0.000000in}{0.069444in}}{%
\pgfpathmoveto{\pgfqpoint{0.000000in}{0.000000in}}%
\pgfpathlineto{\pgfqpoint{0.000000in}{0.069444in}}%
\pgfusepath{stroke,fill}%
}%
\begin{pgfscope}%
\pgfsys@transformshift{1.856879in}{0.422640in}%
\pgfsys@useobject{currentmarker}{}%
\end{pgfscope}%
\end{pgfscope}%
\begin{pgfscope}%
\pgfsetbuttcap%
\pgfsetroundjoin%
\definecolor{currentfill}{rgb}{0.000000,0.000000,0.000000}%
\pgfsetfillcolor{currentfill}%
\pgfsetlinewidth{0.501875pt}%
\definecolor{currentstroke}{rgb}{0.000000,0.000000,0.000000}%
\pgfsetstrokecolor{currentstroke}%
\pgfsetdash{}{0pt}%
\pgfsys@defobject{currentmarker}{\pgfqpoint{0.000000in}{-0.069444in}}{\pgfqpoint{0.000000in}{0.000000in}}{%
\pgfpathmoveto{\pgfqpoint{0.000000in}{0.000000in}}%
\pgfpathlineto{\pgfqpoint{0.000000in}{-0.069444in}}%
\pgfusepath{stroke,fill}%
}%
\begin{pgfscope}%
\pgfsys@transformshift{1.856879in}{1.632426in}%
\pgfsys@useobject{currentmarker}{}%
\end{pgfscope}%
\end{pgfscope}%
\begin{pgfscope}%
\pgftext[x=1.856879in,y=0.353196in,,top]{\rmfamily\fontsize{8.000000}{9.600000}\selectfont 0.010}%
\end{pgfscope}%
\begin{pgfscope}%
\pgfsetbuttcap%
\pgfsetroundjoin%
\definecolor{currentfill}{rgb}{0.000000,0.000000,0.000000}%
\pgfsetfillcolor{currentfill}%
\pgfsetlinewidth{0.501875pt}%
\definecolor{currentstroke}{rgb}{0.000000,0.000000,0.000000}%
\pgfsetstrokecolor{currentstroke}%
\pgfsetdash{}{0pt}%
\pgfsys@defobject{currentmarker}{\pgfqpoint{0.000000in}{0.000000in}}{\pgfqpoint{0.000000in}{0.069444in}}{%
\pgfpathmoveto{\pgfqpoint{0.000000in}{0.000000in}}%
\pgfpathlineto{\pgfqpoint{0.000000in}{0.069444in}}%
\pgfusepath{stroke,fill}%
}%
\begin{pgfscope}%
\pgfsys@transformshift{2.161950in}{0.422640in}%
\pgfsys@useobject{currentmarker}{}%
\end{pgfscope}%
\end{pgfscope}%
\begin{pgfscope}%
\pgfsetbuttcap%
\pgfsetroundjoin%
\definecolor{currentfill}{rgb}{0.000000,0.000000,0.000000}%
\pgfsetfillcolor{currentfill}%
\pgfsetlinewidth{0.501875pt}%
\definecolor{currentstroke}{rgb}{0.000000,0.000000,0.000000}%
\pgfsetstrokecolor{currentstroke}%
\pgfsetdash{}{0pt}%
\pgfsys@defobject{currentmarker}{\pgfqpoint{0.000000in}{-0.069444in}}{\pgfqpoint{0.000000in}{0.000000in}}{%
\pgfpathmoveto{\pgfqpoint{0.000000in}{0.000000in}}%
\pgfpathlineto{\pgfqpoint{0.000000in}{-0.069444in}}%
\pgfusepath{stroke,fill}%
}%
\begin{pgfscope}%
\pgfsys@transformshift{2.161950in}{1.632426in}%
\pgfsys@useobject{currentmarker}{}%
\end{pgfscope}%
\end{pgfscope}%
\begin{pgfscope}%
\pgftext[x=2.161950in,y=0.353196in,,top]{\rmfamily\fontsize{8.000000}{9.600000}\selectfont 0.012}%
\end{pgfscope}%
\begin{pgfscope}%
\pgfsetbuttcap%
\pgfsetroundjoin%
\definecolor{currentfill}{rgb}{0.000000,0.000000,0.000000}%
\pgfsetfillcolor{currentfill}%
\pgfsetlinewidth{0.501875pt}%
\definecolor{currentstroke}{rgb}{0.000000,0.000000,0.000000}%
\pgfsetstrokecolor{currentstroke}%
\pgfsetdash{}{0pt}%
\pgfsys@defobject{currentmarker}{\pgfqpoint{0.000000in}{0.000000in}}{\pgfqpoint{0.000000in}{0.069444in}}{%
\pgfpathmoveto{\pgfqpoint{0.000000in}{0.000000in}}%
\pgfpathlineto{\pgfqpoint{0.000000in}{0.069444in}}%
\pgfusepath{stroke,fill}%
}%
\begin{pgfscope}%
\pgfsys@transformshift{2.467022in}{0.422640in}%
\pgfsys@useobject{currentmarker}{}%
\end{pgfscope}%
\end{pgfscope}%
\begin{pgfscope}%
\pgfsetbuttcap%
\pgfsetroundjoin%
\definecolor{currentfill}{rgb}{0.000000,0.000000,0.000000}%
\pgfsetfillcolor{currentfill}%
\pgfsetlinewidth{0.501875pt}%
\definecolor{currentstroke}{rgb}{0.000000,0.000000,0.000000}%
\pgfsetstrokecolor{currentstroke}%
\pgfsetdash{}{0pt}%
\pgfsys@defobject{currentmarker}{\pgfqpoint{0.000000in}{-0.069444in}}{\pgfqpoint{0.000000in}{0.000000in}}{%
\pgfpathmoveto{\pgfqpoint{0.000000in}{0.000000in}}%
\pgfpathlineto{\pgfqpoint{0.000000in}{-0.069444in}}%
\pgfusepath{stroke,fill}%
}%
\begin{pgfscope}%
\pgfsys@transformshift{2.467022in}{1.632426in}%
\pgfsys@useobject{currentmarker}{}%
\end{pgfscope}%
\end{pgfscope}%
\begin{pgfscope}%
\pgftext[x=2.467022in,y=0.353196in,,top]{\rmfamily\fontsize{8.000000}{9.600000}\selectfont 0.014}%
\end{pgfscope}%
\begin{pgfscope}%
\pgftext[x=1.399272in,y=0.176221in,,top]{\rmfamily\fontsize{9.000000}{10.800000}\selectfont \(\displaystyle \mathrm{cos}(\mathrm{DIRA\ angle})\)}%
\end{pgfscope}%
\begin{pgfscope}%
\pgfsetbuttcap%
\pgfsetroundjoin%
\definecolor{currentfill}{rgb}{0.000000,0.000000,0.000000}%
\pgfsetfillcolor{currentfill}%
\pgfsetlinewidth{0.501875pt}%
\definecolor{currentstroke}{rgb}{0.000000,0.000000,0.000000}%
\pgfsetstrokecolor{currentstroke}%
\pgfsetdash{}{0pt}%
\pgfsys@defobject{currentmarker}{\pgfqpoint{0.000000in}{0.000000in}}{\pgfqpoint{0.069444in}{0.000000in}}{%
\pgfpathmoveto{\pgfqpoint{0.000000in}{0.000000in}}%
\pgfpathlineto{\pgfqpoint{0.069444in}{0.000000in}}%
\pgfusepath{stroke,fill}%
}%
\begin{pgfscope}%
\pgfsys@transformshift{0.331521in}{0.422640in}%
\pgfsys@useobject{currentmarker}{}%
\end{pgfscope}%
\end{pgfscope}%
\begin{pgfscope}%
\pgfsetbuttcap%
\pgfsetroundjoin%
\definecolor{currentfill}{rgb}{0.000000,0.000000,0.000000}%
\pgfsetfillcolor{currentfill}%
\pgfsetlinewidth{0.501875pt}%
\definecolor{currentstroke}{rgb}{0.000000,0.000000,0.000000}%
\pgfsetstrokecolor{currentstroke}%
\pgfsetdash{}{0pt}%
\pgfsys@defobject{currentmarker}{\pgfqpoint{-0.069444in}{0.000000in}}{\pgfqpoint{0.000000in}{0.000000in}}{%
\pgfpathmoveto{\pgfqpoint{0.000000in}{0.000000in}}%
\pgfpathlineto{\pgfqpoint{-0.069444in}{0.000000in}}%
\pgfusepath{stroke,fill}%
}%
\begin{pgfscope}%
\pgfsys@transformshift{2.467022in}{0.422640in}%
\pgfsys@useobject{currentmarker}{}%
\end{pgfscope}%
\end{pgfscope}%
\begin{pgfscope}%
\pgftext[x=0.262077in,y=0.422640in,right,]{\rmfamily\fontsize{8.000000}{9.600000}\selectfont 0}%
\end{pgfscope}%
\begin{pgfscope}%
\pgfsetbuttcap%
\pgfsetroundjoin%
\definecolor{currentfill}{rgb}{0.000000,0.000000,0.000000}%
\pgfsetfillcolor{currentfill}%
\pgfsetlinewidth{0.501875pt}%
\definecolor{currentstroke}{rgb}{0.000000,0.000000,0.000000}%
\pgfsetstrokecolor{currentstroke}%
\pgfsetdash{}{0pt}%
\pgfsys@defobject{currentmarker}{\pgfqpoint{0.000000in}{0.000000in}}{\pgfqpoint{0.069444in}{0.000000in}}{%
\pgfpathmoveto{\pgfqpoint{0.000000in}{0.000000in}}%
\pgfpathlineto{\pgfqpoint{0.069444in}{0.000000in}}%
\pgfusepath{stroke,fill}%
}%
\begin{pgfscope}%
\pgfsys@transformshift{0.331521in}{0.573863in}%
\pgfsys@useobject{currentmarker}{}%
\end{pgfscope}%
\end{pgfscope}%
\begin{pgfscope}%
\pgfsetbuttcap%
\pgfsetroundjoin%
\definecolor{currentfill}{rgb}{0.000000,0.000000,0.000000}%
\pgfsetfillcolor{currentfill}%
\pgfsetlinewidth{0.501875pt}%
\definecolor{currentstroke}{rgb}{0.000000,0.000000,0.000000}%
\pgfsetstrokecolor{currentstroke}%
\pgfsetdash{}{0pt}%
\pgfsys@defobject{currentmarker}{\pgfqpoint{-0.069444in}{0.000000in}}{\pgfqpoint{0.000000in}{0.000000in}}{%
\pgfpathmoveto{\pgfqpoint{0.000000in}{0.000000in}}%
\pgfpathlineto{\pgfqpoint{-0.069444in}{0.000000in}}%
\pgfusepath{stroke,fill}%
}%
\begin{pgfscope}%
\pgfsys@transformshift{2.467022in}{0.573863in}%
\pgfsys@useobject{currentmarker}{}%
\end{pgfscope}%
\end{pgfscope}%
\begin{pgfscope}%
\pgftext[x=0.262077in,y=0.573863in,right,]{\rmfamily\fontsize{8.000000}{9.600000}\selectfont 50}%
\end{pgfscope}%
\begin{pgfscope}%
\pgfsetbuttcap%
\pgfsetroundjoin%
\definecolor{currentfill}{rgb}{0.000000,0.000000,0.000000}%
\pgfsetfillcolor{currentfill}%
\pgfsetlinewidth{0.501875pt}%
\definecolor{currentstroke}{rgb}{0.000000,0.000000,0.000000}%
\pgfsetstrokecolor{currentstroke}%
\pgfsetdash{}{0pt}%
\pgfsys@defobject{currentmarker}{\pgfqpoint{0.000000in}{0.000000in}}{\pgfqpoint{0.069444in}{0.000000in}}{%
\pgfpathmoveto{\pgfqpoint{0.000000in}{0.000000in}}%
\pgfpathlineto{\pgfqpoint{0.069444in}{0.000000in}}%
\pgfusepath{stroke,fill}%
}%
\begin{pgfscope}%
\pgfsys@transformshift{0.331521in}{0.725087in}%
\pgfsys@useobject{currentmarker}{}%
\end{pgfscope}%
\end{pgfscope}%
\begin{pgfscope}%
\pgfsetbuttcap%
\pgfsetroundjoin%
\definecolor{currentfill}{rgb}{0.000000,0.000000,0.000000}%
\pgfsetfillcolor{currentfill}%
\pgfsetlinewidth{0.501875pt}%
\definecolor{currentstroke}{rgb}{0.000000,0.000000,0.000000}%
\pgfsetstrokecolor{currentstroke}%
\pgfsetdash{}{0pt}%
\pgfsys@defobject{currentmarker}{\pgfqpoint{-0.069444in}{0.000000in}}{\pgfqpoint{0.000000in}{0.000000in}}{%
\pgfpathmoveto{\pgfqpoint{0.000000in}{0.000000in}}%
\pgfpathlineto{\pgfqpoint{-0.069444in}{0.000000in}}%
\pgfusepath{stroke,fill}%
}%
\begin{pgfscope}%
\pgfsys@transformshift{2.467022in}{0.725087in}%
\pgfsys@useobject{currentmarker}{}%
\end{pgfscope}%
\end{pgfscope}%
\begin{pgfscope}%
\pgftext[x=0.262077in,y=0.725087in,right,]{\rmfamily\fontsize{8.000000}{9.600000}\selectfont 100}%
\end{pgfscope}%
\begin{pgfscope}%
\pgfsetbuttcap%
\pgfsetroundjoin%
\definecolor{currentfill}{rgb}{0.000000,0.000000,0.000000}%
\pgfsetfillcolor{currentfill}%
\pgfsetlinewidth{0.501875pt}%
\definecolor{currentstroke}{rgb}{0.000000,0.000000,0.000000}%
\pgfsetstrokecolor{currentstroke}%
\pgfsetdash{}{0pt}%
\pgfsys@defobject{currentmarker}{\pgfqpoint{0.000000in}{0.000000in}}{\pgfqpoint{0.069444in}{0.000000in}}{%
\pgfpathmoveto{\pgfqpoint{0.000000in}{0.000000in}}%
\pgfpathlineto{\pgfqpoint{0.069444in}{0.000000in}}%
\pgfusepath{stroke,fill}%
}%
\begin{pgfscope}%
\pgfsys@transformshift{0.331521in}{0.876310in}%
\pgfsys@useobject{currentmarker}{}%
\end{pgfscope}%
\end{pgfscope}%
\begin{pgfscope}%
\pgfsetbuttcap%
\pgfsetroundjoin%
\definecolor{currentfill}{rgb}{0.000000,0.000000,0.000000}%
\pgfsetfillcolor{currentfill}%
\pgfsetlinewidth{0.501875pt}%
\definecolor{currentstroke}{rgb}{0.000000,0.000000,0.000000}%
\pgfsetstrokecolor{currentstroke}%
\pgfsetdash{}{0pt}%
\pgfsys@defobject{currentmarker}{\pgfqpoint{-0.069444in}{0.000000in}}{\pgfqpoint{0.000000in}{0.000000in}}{%
\pgfpathmoveto{\pgfqpoint{0.000000in}{0.000000in}}%
\pgfpathlineto{\pgfqpoint{-0.069444in}{0.000000in}}%
\pgfusepath{stroke,fill}%
}%
\begin{pgfscope}%
\pgfsys@transformshift{2.467022in}{0.876310in}%
\pgfsys@useobject{currentmarker}{}%
\end{pgfscope}%
\end{pgfscope}%
\begin{pgfscope}%
\pgftext[x=0.262077in,y=0.876310in,right,]{\rmfamily\fontsize{8.000000}{9.600000}\selectfont 150}%
\end{pgfscope}%
\begin{pgfscope}%
\pgfsetbuttcap%
\pgfsetroundjoin%
\definecolor{currentfill}{rgb}{0.000000,0.000000,0.000000}%
\pgfsetfillcolor{currentfill}%
\pgfsetlinewidth{0.501875pt}%
\definecolor{currentstroke}{rgb}{0.000000,0.000000,0.000000}%
\pgfsetstrokecolor{currentstroke}%
\pgfsetdash{}{0pt}%
\pgfsys@defobject{currentmarker}{\pgfqpoint{0.000000in}{0.000000in}}{\pgfqpoint{0.069444in}{0.000000in}}{%
\pgfpathmoveto{\pgfqpoint{0.000000in}{0.000000in}}%
\pgfpathlineto{\pgfqpoint{0.069444in}{0.000000in}}%
\pgfusepath{stroke,fill}%
}%
\begin{pgfscope}%
\pgfsys@transformshift{0.331521in}{1.027533in}%
\pgfsys@useobject{currentmarker}{}%
\end{pgfscope}%
\end{pgfscope}%
\begin{pgfscope}%
\pgfsetbuttcap%
\pgfsetroundjoin%
\definecolor{currentfill}{rgb}{0.000000,0.000000,0.000000}%
\pgfsetfillcolor{currentfill}%
\pgfsetlinewidth{0.501875pt}%
\definecolor{currentstroke}{rgb}{0.000000,0.000000,0.000000}%
\pgfsetstrokecolor{currentstroke}%
\pgfsetdash{}{0pt}%
\pgfsys@defobject{currentmarker}{\pgfqpoint{-0.069444in}{0.000000in}}{\pgfqpoint{0.000000in}{0.000000in}}{%
\pgfpathmoveto{\pgfqpoint{0.000000in}{0.000000in}}%
\pgfpathlineto{\pgfqpoint{-0.069444in}{0.000000in}}%
\pgfusepath{stroke,fill}%
}%
\begin{pgfscope}%
\pgfsys@transformshift{2.467022in}{1.027533in}%
\pgfsys@useobject{currentmarker}{}%
\end{pgfscope}%
\end{pgfscope}%
\begin{pgfscope}%
\pgftext[x=0.262077in,y=1.027533in,right,]{\rmfamily\fontsize{8.000000}{9.600000}\selectfont 200}%
\end{pgfscope}%
\begin{pgfscope}%
\pgfsetbuttcap%
\pgfsetroundjoin%
\definecolor{currentfill}{rgb}{0.000000,0.000000,0.000000}%
\pgfsetfillcolor{currentfill}%
\pgfsetlinewidth{0.501875pt}%
\definecolor{currentstroke}{rgb}{0.000000,0.000000,0.000000}%
\pgfsetstrokecolor{currentstroke}%
\pgfsetdash{}{0pt}%
\pgfsys@defobject{currentmarker}{\pgfqpoint{0.000000in}{0.000000in}}{\pgfqpoint{0.069444in}{0.000000in}}{%
\pgfpathmoveto{\pgfqpoint{0.000000in}{0.000000in}}%
\pgfpathlineto{\pgfqpoint{0.069444in}{0.000000in}}%
\pgfusepath{stroke,fill}%
}%
\begin{pgfscope}%
\pgfsys@transformshift{0.331521in}{1.178756in}%
\pgfsys@useobject{currentmarker}{}%
\end{pgfscope}%
\end{pgfscope}%
\begin{pgfscope}%
\pgfsetbuttcap%
\pgfsetroundjoin%
\definecolor{currentfill}{rgb}{0.000000,0.000000,0.000000}%
\pgfsetfillcolor{currentfill}%
\pgfsetlinewidth{0.501875pt}%
\definecolor{currentstroke}{rgb}{0.000000,0.000000,0.000000}%
\pgfsetstrokecolor{currentstroke}%
\pgfsetdash{}{0pt}%
\pgfsys@defobject{currentmarker}{\pgfqpoint{-0.069444in}{0.000000in}}{\pgfqpoint{0.000000in}{0.000000in}}{%
\pgfpathmoveto{\pgfqpoint{0.000000in}{0.000000in}}%
\pgfpathlineto{\pgfqpoint{-0.069444in}{0.000000in}}%
\pgfusepath{stroke,fill}%
}%
\begin{pgfscope}%
\pgfsys@transformshift{2.467022in}{1.178756in}%
\pgfsys@useobject{currentmarker}{}%
\end{pgfscope}%
\end{pgfscope}%
\begin{pgfscope}%
\pgftext[x=0.262077in,y=1.178756in,right,]{\rmfamily\fontsize{8.000000}{9.600000}\selectfont 250}%
\end{pgfscope}%
\begin{pgfscope}%
\pgfsetbuttcap%
\pgfsetroundjoin%
\definecolor{currentfill}{rgb}{0.000000,0.000000,0.000000}%
\pgfsetfillcolor{currentfill}%
\pgfsetlinewidth{0.501875pt}%
\definecolor{currentstroke}{rgb}{0.000000,0.000000,0.000000}%
\pgfsetstrokecolor{currentstroke}%
\pgfsetdash{}{0pt}%
\pgfsys@defobject{currentmarker}{\pgfqpoint{0.000000in}{0.000000in}}{\pgfqpoint{0.069444in}{0.000000in}}{%
\pgfpathmoveto{\pgfqpoint{0.000000in}{0.000000in}}%
\pgfpathlineto{\pgfqpoint{0.069444in}{0.000000in}}%
\pgfusepath{stroke,fill}%
}%
\begin{pgfscope}%
\pgfsys@transformshift{0.331521in}{1.329980in}%
\pgfsys@useobject{currentmarker}{}%
\end{pgfscope}%
\end{pgfscope}%
\begin{pgfscope}%
\pgfsetbuttcap%
\pgfsetroundjoin%
\definecolor{currentfill}{rgb}{0.000000,0.000000,0.000000}%
\pgfsetfillcolor{currentfill}%
\pgfsetlinewidth{0.501875pt}%
\definecolor{currentstroke}{rgb}{0.000000,0.000000,0.000000}%
\pgfsetstrokecolor{currentstroke}%
\pgfsetdash{}{0pt}%
\pgfsys@defobject{currentmarker}{\pgfqpoint{-0.069444in}{0.000000in}}{\pgfqpoint{0.000000in}{0.000000in}}{%
\pgfpathmoveto{\pgfqpoint{0.000000in}{0.000000in}}%
\pgfpathlineto{\pgfqpoint{-0.069444in}{0.000000in}}%
\pgfusepath{stroke,fill}%
}%
\begin{pgfscope}%
\pgfsys@transformshift{2.467022in}{1.329980in}%
\pgfsys@useobject{currentmarker}{}%
\end{pgfscope}%
\end{pgfscope}%
\begin{pgfscope}%
\pgftext[x=0.262077in,y=1.329980in,right,]{\rmfamily\fontsize{8.000000}{9.600000}\selectfont 300}%
\end{pgfscope}%
\begin{pgfscope}%
\pgfsetbuttcap%
\pgfsetroundjoin%
\definecolor{currentfill}{rgb}{0.000000,0.000000,0.000000}%
\pgfsetfillcolor{currentfill}%
\pgfsetlinewidth{0.501875pt}%
\definecolor{currentstroke}{rgb}{0.000000,0.000000,0.000000}%
\pgfsetstrokecolor{currentstroke}%
\pgfsetdash{}{0pt}%
\pgfsys@defobject{currentmarker}{\pgfqpoint{0.000000in}{0.000000in}}{\pgfqpoint{0.069444in}{0.000000in}}{%
\pgfpathmoveto{\pgfqpoint{0.000000in}{0.000000in}}%
\pgfpathlineto{\pgfqpoint{0.069444in}{0.000000in}}%
\pgfusepath{stroke,fill}%
}%
\begin{pgfscope}%
\pgfsys@transformshift{0.331521in}{1.481203in}%
\pgfsys@useobject{currentmarker}{}%
\end{pgfscope}%
\end{pgfscope}%
\begin{pgfscope}%
\pgfsetbuttcap%
\pgfsetroundjoin%
\definecolor{currentfill}{rgb}{0.000000,0.000000,0.000000}%
\pgfsetfillcolor{currentfill}%
\pgfsetlinewidth{0.501875pt}%
\definecolor{currentstroke}{rgb}{0.000000,0.000000,0.000000}%
\pgfsetstrokecolor{currentstroke}%
\pgfsetdash{}{0pt}%
\pgfsys@defobject{currentmarker}{\pgfqpoint{-0.069444in}{0.000000in}}{\pgfqpoint{0.000000in}{0.000000in}}{%
\pgfpathmoveto{\pgfqpoint{0.000000in}{0.000000in}}%
\pgfpathlineto{\pgfqpoint{-0.069444in}{0.000000in}}%
\pgfusepath{stroke,fill}%
}%
\begin{pgfscope}%
\pgfsys@transformshift{2.467022in}{1.481203in}%
\pgfsys@useobject{currentmarker}{}%
\end{pgfscope}%
\end{pgfscope}%
\begin{pgfscope}%
\pgftext[x=0.262077in,y=1.481203in,right,]{\rmfamily\fontsize{8.000000}{9.600000}\selectfont 350}%
\end{pgfscope}%
\begin{pgfscope}%
\pgfsetbuttcap%
\pgfsetroundjoin%
\definecolor{currentfill}{rgb}{0.000000,0.000000,0.000000}%
\pgfsetfillcolor{currentfill}%
\pgfsetlinewidth{0.501875pt}%
\definecolor{currentstroke}{rgb}{0.000000,0.000000,0.000000}%
\pgfsetstrokecolor{currentstroke}%
\pgfsetdash{}{0pt}%
\pgfsys@defobject{currentmarker}{\pgfqpoint{0.000000in}{0.000000in}}{\pgfqpoint{0.069444in}{0.000000in}}{%
\pgfpathmoveto{\pgfqpoint{0.000000in}{0.000000in}}%
\pgfpathlineto{\pgfqpoint{0.069444in}{0.000000in}}%
\pgfusepath{stroke,fill}%
}%
\begin{pgfscope}%
\pgfsys@transformshift{0.331521in}{1.632426in}%
\pgfsys@useobject{currentmarker}{}%
\end{pgfscope}%
\end{pgfscope}%
\begin{pgfscope}%
\pgfsetbuttcap%
\pgfsetroundjoin%
\definecolor{currentfill}{rgb}{0.000000,0.000000,0.000000}%
\pgfsetfillcolor{currentfill}%
\pgfsetlinewidth{0.501875pt}%
\definecolor{currentstroke}{rgb}{0.000000,0.000000,0.000000}%
\pgfsetstrokecolor{currentstroke}%
\pgfsetdash{}{0pt}%
\pgfsys@defobject{currentmarker}{\pgfqpoint{-0.069444in}{0.000000in}}{\pgfqpoint{0.000000in}{0.000000in}}{%
\pgfpathmoveto{\pgfqpoint{0.000000in}{0.000000in}}%
\pgfpathlineto{\pgfqpoint{-0.069444in}{0.000000in}}%
\pgfusepath{stroke,fill}%
}%
\begin{pgfscope}%
\pgfsys@transformshift{2.467022in}{1.632426in}%
\pgfsys@useobject{currentmarker}{}%
\end{pgfscope}%
\end{pgfscope}%
\begin{pgfscope}%
\pgftext[x=0.262077in,y=1.632426in,right,]{\rmfamily\fontsize{8.000000}{9.600000}\selectfont 400}%
\end{pgfscope}%
\end{pgfpicture}%
\makeatother%
\endgroup%

	\end{subfigure}
	\begin{subfigure}[t]{0.49\textwidth}
		\centering
    %\includegraphics[width=\textwidth]{store/variables/DATA_MC_B_ENDVERTEX_CHI2_NDOF.pdf}
    %% Creator: Matplotlib, PGF backend
%%
%% To include the figure in your LaTeX document, write
%%   \input{<filename>.pgf}
%%
%% Make sure the required packages are loaded in your preamble
%%   \usepackage{pgf}
%%
%% Figures using additional raster images can only be included by \input if
%% they are in the same directory as the main LaTeX file. For loading figures
%% from other directories you can use the `import` package
%%   \usepackage{import}
%% and then include the figures with
%%   \import{<path to file>}{<filename>.pgf}
%%
%% Matplotlib used the following preamble
%%   \usepackage{fontspec}
%%   \setmainfont{DejaVu Serif}
%%   \setsansfont{DejaVu Sans}
%%   \setmonofont{DejaVu Sans Mono}
%%
\begingroup%
\makeatletter%
\begin{pgfpicture}%
\pgfpathrectangle{\pgfpointorigin}{\pgfqpoint{2.678086in}{1.718727in}}%
\pgfusepath{use as bounding box, clip}%
\begin{pgfscope}%
\pgfsetbuttcap%
\pgfsetmiterjoin%
\definecolor{currentfill}{rgb}{1.000000,1.000000,1.000000}%
\pgfsetfillcolor{currentfill}%
\pgfsetlinewidth{0.000000pt}%
\definecolor{currentstroke}{rgb}{1.000000,1.000000,1.000000}%
\pgfsetstrokecolor{currentstroke}%
\pgfsetdash{}{0pt}%
\pgfpathmoveto{\pgfqpoint{0.000000in}{-0.000000in}}%
\pgfpathlineto{\pgfqpoint{2.678086in}{-0.000000in}}%
\pgfpathlineto{\pgfqpoint{2.678086in}{1.718727in}}%
\pgfpathlineto{\pgfqpoint{0.000000in}{1.718727in}}%
\pgfpathclose%
\pgfusepath{fill}%
\end{pgfscope}%
\begin{pgfscope}%
\pgfsetbuttcap%
\pgfsetmiterjoin%
\definecolor{currentfill}{rgb}{1.000000,1.000000,1.000000}%
\pgfsetfillcolor{currentfill}%
\pgfsetlinewidth{0.000000pt}%
\definecolor{currentstroke}{rgb}{0.000000,0.000000,0.000000}%
\pgfsetstrokecolor{currentstroke}%
\pgfsetstrokeopacity{0.000000}%
\pgfsetdash{}{0pt}%
\pgfpathmoveto{\pgfqpoint{0.296148in}{0.441418in}}%
\pgfpathlineto{\pgfqpoint{2.592740in}{0.441418in}}%
\pgfpathlineto{\pgfqpoint{2.592740in}{1.614961in}}%
\pgfpathlineto{\pgfqpoint{0.296148in}{1.614961in}}%
\pgfpathclose%
\pgfusepath{fill}%
\end{pgfscope}%
\begin{pgfscope}%
\pgfpathrectangle{\pgfqpoint{0.296148in}{0.441418in}}{\pgfqpoint{2.296592in}{1.173543in}} %
\pgfusepath{clip}%
\pgfsetbuttcap%
\pgfsetmiterjoin%
\definecolor{currentfill}{rgb}{0.215686,0.470588,0.749020}%
\pgfsetfillcolor{currentfill}%
\pgfsetlinewidth{0.000000pt}%
\definecolor{currentstroke}{rgb}{0.000000,0.000000,0.000000}%
\pgfsetstrokecolor{currentstroke}%
\pgfsetdash{}{0pt}%
\pgfpathmoveto{\pgfqpoint{0.297424in}{0.441418in}}%
\pgfpathlineto{\pgfqpoint{0.297424in}{0.609709in}}%
\pgfpathlineto{\pgfqpoint{0.343328in}{0.609709in}}%
\pgfpathlineto{\pgfqpoint{0.343328in}{0.985601in}}%
\pgfpathlineto{\pgfqpoint{0.389233in}{0.985601in}}%
\pgfpathlineto{\pgfqpoint{0.389233in}{1.251138in}}%
\pgfpathlineto{\pgfqpoint{0.435137in}{1.251138in}}%
\pgfpathlineto{\pgfqpoint{0.435137in}{1.391596in}}%
\pgfpathlineto{\pgfqpoint{0.481042in}{1.391596in}}%
\pgfpathlineto{\pgfqpoint{0.481042in}{1.426427in}}%
\pgfpathlineto{\pgfqpoint{0.526946in}{1.426427in}}%
\pgfpathlineto{\pgfqpoint{0.526946in}{1.417278in}}%
\pgfpathlineto{\pgfqpoint{0.572851in}{1.417278in}}%
\pgfpathlineto{\pgfqpoint{0.572851in}{1.349906in}}%
\pgfpathlineto{\pgfqpoint{0.618755in}{1.349906in}}%
\pgfpathlineto{\pgfqpoint{0.618755in}{1.270193in}}%
\pgfpathlineto{\pgfqpoint{0.664660in}{1.270193in}}%
\pgfpathlineto{\pgfqpoint{0.664660in}{1.189248in}}%
\pgfpathlineto{\pgfqpoint{0.710564in}{1.189248in}}%
\pgfpathlineto{\pgfqpoint{0.710564in}{1.110275in}}%
\pgfpathlineto{\pgfqpoint{0.756469in}{1.110275in}}%
\pgfpathlineto{\pgfqpoint{0.756469in}{1.039528in}}%
\pgfpathlineto{\pgfqpoint{0.802373in}{1.039528in}}%
\pgfpathlineto{\pgfqpoint{0.802373in}{0.959669in}}%
\pgfpathlineto{\pgfqpoint{0.848278in}{0.959669in}}%
\pgfpathlineto{\pgfqpoint{0.848278in}{0.901037in}}%
\pgfpathlineto{\pgfqpoint{0.894182in}{0.901037in}}%
\pgfpathlineto{\pgfqpoint{0.894182in}{0.843994in}}%
\pgfpathlineto{\pgfqpoint{0.940087in}{0.843994in}}%
\pgfpathlineto{\pgfqpoint{0.940087in}{0.784292in}}%
\pgfpathlineto{\pgfqpoint{0.985991in}{0.784292in}}%
\pgfpathlineto{\pgfqpoint{0.985991in}{0.750250in}}%
\pgfpathlineto{\pgfqpoint{1.031896in}{0.750250in}}%
\pgfpathlineto{\pgfqpoint{1.031896in}{0.705507in}}%
\pgfpathlineto{\pgfqpoint{1.077800in}{0.705507in}}%
\pgfpathlineto{\pgfqpoint{1.077800in}{0.674539in}}%
\pgfpathlineto{\pgfqpoint{1.123705in}{0.674539in}}%
\pgfpathlineto{\pgfqpoint{1.123705in}{0.634008in}}%
\pgfpathlineto{\pgfqpoint{1.169609in}{0.634008in}}%
\pgfpathlineto{\pgfqpoint{1.169609in}{0.609502in}}%
\pgfpathlineto{\pgfqpoint{1.215514in}{0.609502in}}%
\pgfpathlineto{\pgfqpoint{1.215514in}{0.586251in}}%
\pgfpathlineto{\pgfqpoint{1.261418in}{0.586251in}}%
\pgfpathlineto{\pgfqpoint{1.261418in}{0.569989in}}%
\pgfpathlineto{\pgfqpoint{1.307323in}{0.569989in}}%
\pgfpathlineto{\pgfqpoint{1.307323in}{0.551306in}}%
\pgfpathlineto{\pgfqpoint{1.353227in}{0.551306in}}%
\pgfpathlineto{\pgfqpoint{1.353227in}{0.537044in}}%
\pgfpathlineto{\pgfqpoint{1.399132in}{0.537044in}}%
\pgfpathlineto{\pgfqpoint{1.399132in}{0.526890in}}%
\pgfpathlineto{\pgfqpoint{1.445036in}{0.526890in}}%
\pgfpathlineto{\pgfqpoint{1.445036in}{0.511423in}}%
\pgfpathlineto{\pgfqpoint{1.490941in}{0.511423in}}%
\pgfpathlineto{\pgfqpoint{1.490941in}{0.499708in}}%
\pgfpathlineto{\pgfqpoint{1.536845in}{0.499708in}}%
\pgfpathlineto{\pgfqpoint{1.536845in}{0.494453in}}%
\pgfpathlineto{\pgfqpoint{1.582750in}{0.494453in}}%
\pgfpathlineto{\pgfqpoint{1.582750in}{0.488950in}}%
\pgfpathlineto{\pgfqpoint{1.628654in}{0.488950in}}%
\pgfpathlineto{\pgfqpoint{1.628654in}{0.482587in}}%
\pgfpathlineto{\pgfqpoint{1.674559in}{0.482587in}}%
\pgfpathlineto{\pgfqpoint{1.674559in}{0.478364in}}%
\pgfpathlineto{\pgfqpoint{1.720463in}{0.478364in}}%
\pgfpathlineto{\pgfqpoint{1.720463in}{0.475954in}}%
\pgfpathlineto{\pgfqpoint{1.766368in}{0.475954in}}%
\pgfpathlineto{\pgfqpoint{1.766368in}{0.471715in}}%
\pgfpathlineto{\pgfqpoint{1.812272in}{0.471715in}}%
\pgfpathlineto{\pgfqpoint{1.812272in}{0.465497in}}%
\pgfpathlineto{\pgfqpoint{1.858177in}{0.465497in}}%
\pgfpathlineto{\pgfqpoint{1.858177in}{0.465883in}}%
\pgfpathlineto{\pgfqpoint{1.904081in}{0.465883in}}%
\pgfpathlineto{\pgfqpoint{1.904081in}{0.461012in}}%
\pgfpathlineto{\pgfqpoint{1.949986in}{0.461012in}}%
\pgfpathlineto{\pgfqpoint{1.949986in}{0.460840in}}%
\pgfpathlineto{\pgfqpoint{1.995890in}{0.460840in}}%
\pgfpathlineto{\pgfqpoint{1.995890in}{0.457723in}}%
\pgfpathlineto{\pgfqpoint{2.041795in}{0.457723in}}%
\pgfpathlineto{\pgfqpoint{2.041795in}{0.456838in}}%
\pgfpathlineto{\pgfqpoint{2.087699in}{0.456838in}}%
\pgfpathlineto{\pgfqpoint{2.087699in}{0.453616in}}%
\pgfpathlineto{\pgfqpoint{2.133604in}{0.453616in}}%
\pgfpathlineto{\pgfqpoint{2.133604in}{0.452813in}}%
\pgfpathlineto{\pgfqpoint{2.179508in}{0.452813in}}%
\pgfpathlineto{\pgfqpoint{2.179508in}{0.454010in}}%
\pgfpathlineto{\pgfqpoint{2.225413in}{0.454010in}}%
\pgfpathlineto{\pgfqpoint{2.225413in}{0.450271in}}%
\pgfpathlineto{\pgfqpoint{2.271317in}{0.450271in}}%
\pgfpathlineto{\pgfqpoint{2.271317in}{0.450847in}}%
\pgfpathlineto{\pgfqpoint{2.317222in}{0.450847in}}%
\pgfpathlineto{\pgfqpoint{2.317222in}{0.450423in}}%
\pgfpathlineto{\pgfqpoint{2.363126in}{0.450423in}}%
\pgfpathlineto{\pgfqpoint{2.363126in}{0.449278in}}%
\pgfpathlineto{\pgfqpoint{2.409031in}{0.449278in}}%
\pgfpathlineto{\pgfqpoint{2.409031in}{0.449574in}}%
\pgfpathlineto{\pgfqpoint{2.454935in}{0.449574in}}%
\pgfpathlineto{\pgfqpoint{2.454935in}{0.449361in}}%
\pgfpathlineto{\pgfqpoint{2.500840in}{0.449361in}}%
\pgfpathlineto{\pgfqpoint{2.500840in}{0.447843in}}%
\pgfpathlineto{\pgfqpoint{2.546744in}{0.447843in}}%
\pgfpathlineto{\pgfqpoint{2.546744in}{0.448453in}}%
\pgfpathlineto{\pgfqpoint{2.592649in}{0.448453in}}%
\pgfpathlineto{\pgfqpoint{2.592649in}{0.441418in}}%
\pgfpathlineto{\pgfqpoint{2.546744in}{0.441418in}}%
\pgfpathlineto{\pgfqpoint{2.546744in}{0.441418in}}%
\pgfpathlineto{\pgfqpoint{2.500840in}{0.441418in}}%
\pgfpathlineto{\pgfqpoint{2.500840in}{0.441418in}}%
\pgfpathlineto{\pgfqpoint{2.454935in}{0.441418in}}%
\pgfpathlineto{\pgfqpoint{2.454935in}{0.441418in}}%
\pgfpathlineto{\pgfqpoint{2.409031in}{0.441418in}}%
\pgfpathlineto{\pgfqpoint{2.409031in}{0.441418in}}%
\pgfpathlineto{\pgfqpoint{2.363126in}{0.441418in}}%
\pgfpathlineto{\pgfqpoint{2.363126in}{0.441418in}}%
\pgfpathlineto{\pgfqpoint{2.317222in}{0.441418in}}%
\pgfpathlineto{\pgfqpoint{2.317222in}{0.441418in}}%
\pgfpathlineto{\pgfqpoint{2.271317in}{0.441418in}}%
\pgfpathlineto{\pgfqpoint{2.271317in}{0.441418in}}%
\pgfpathlineto{\pgfqpoint{2.225413in}{0.441418in}}%
\pgfpathlineto{\pgfqpoint{2.225413in}{0.441418in}}%
\pgfpathlineto{\pgfqpoint{2.179508in}{0.441418in}}%
\pgfpathlineto{\pgfqpoint{2.179508in}{0.441418in}}%
\pgfpathlineto{\pgfqpoint{2.133604in}{0.441418in}}%
\pgfpathlineto{\pgfqpoint{2.133604in}{0.441418in}}%
\pgfpathlineto{\pgfqpoint{2.087699in}{0.441418in}}%
\pgfpathlineto{\pgfqpoint{2.087699in}{0.441418in}}%
\pgfpathlineto{\pgfqpoint{2.041795in}{0.441418in}}%
\pgfpathlineto{\pgfqpoint{2.041795in}{0.441418in}}%
\pgfpathlineto{\pgfqpoint{1.995890in}{0.441418in}}%
\pgfpathlineto{\pgfqpoint{1.995890in}{0.441418in}}%
\pgfpathlineto{\pgfqpoint{1.949986in}{0.441418in}}%
\pgfpathlineto{\pgfqpoint{1.949986in}{0.441418in}}%
\pgfpathlineto{\pgfqpoint{1.904081in}{0.441418in}}%
\pgfpathlineto{\pgfqpoint{1.904081in}{0.441418in}}%
\pgfpathlineto{\pgfqpoint{1.858177in}{0.441418in}}%
\pgfpathlineto{\pgfqpoint{1.858177in}{0.441418in}}%
\pgfpathlineto{\pgfqpoint{1.812272in}{0.441418in}}%
\pgfpathlineto{\pgfqpoint{1.812272in}{0.441418in}}%
\pgfpathlineto{\pgfqpoint{1.766368in}{0.441418in}}%
\pgfpathlineto{\pgfqpoint{1.766368in}{0.441418in}}%
\pgfpathlineto{\pgfqpoint{1.720463in}{0.441418in}}%
\pgfpathlineto{\pgfqpoint{1.720463in}{0.441418in}}%
\pgfpathlineto{\pgfqpoint{1.674559in}{0.441418in}}%
\pgfpathlineto{\pgfqpoint{1.674559in}{0.441418in}}%
\pgfpathlineto{\pgfqpoint{1.628654in}{0.441418in}}%
\pgfpathlineto{\pgfqpoint{1.628654in}{0.441418in}}%
\pgfpathlineto{\pgfqpoint{1.582750in}{0.441418in}}%
\pgfpathlineto{\pgfqpoint{1.582750in}{0.441418in}}%
\pgfpathlineto{\pgfqpoint{1.536845in}{0.441418in}}%
\pgfpathlineto{\pgfqpoint{1.536845in}{0.441418in}}%
\pgfpathlineto{\pgfqpoint{1.490941in}{0.441418in}}%
\pgfpathlineto{\pgfqpoint{1.490941in}{0.441418in}}%
\pgfpathlineto{\pgfqpoint{1.445036in}{0.441418in}}%
\pgfpathlineto{\pgfqpoint{1.445036in}{0.441418in}}%
\pgfpathlineto{\pgfqpoint{1.399132in}{0.441418in}}%
\pgfpathlineto{\pgfqpoint{1.399132in}{0.441418in}}%
\pgfpathlineto{\pgfqpoint{1.353227in}{0.441418in}}%
\pgfpathlineto{\pgfqpoint{1.353227in}{0.441418in}}%
\pgfpathlineto{\pgfqpoint{1.307323in}{0.441418in}}%
\pgfpathlineto{\pgfqpoint{1.307323in}{0.441418in}}%
\pgfpathlineto{\pgfqpoint{1.261418in}{0.441418in}}%
\pgfpathlineto{\pgfqpoint{1.261418in}{0.441418in}}%
\pgfpathlineto{\pgfqpoint{1.215514in}{0.441418in}}%
\pgfpathlineto{\pgfqpoint{1.215514in}{0.441418in}}%
\pgfpathlineto{\pgfqpoint{1.169609in}{0.441418in}}%
\pgfpathlineto{\pgfqpoint{1.169609in}{0.441418in}}%
\pgfpathlineto{\pgfqpoint{1.123705in}{0.441418in}}%
\pgfpathlineto{\pgfqpoint{1.123705in}{0.441418in}}%
\pgfpathlineto{\pgfqpoint{1.077800in}{0.441418in}}%
\pgfpathlineto{\pgfqpoint{1.077800in}{0.441418in}}%
\pgfpathlineto{\pgfqpoint{1.031896in}{0.441418in}}%
\pgfpathlineto{\pgfqpoint{1.031896in}{0.441418in}}%
\pgfpathlineto{\pgfqpoint{0.985991in}{0.441418in}}%
\pgfpathlineto{\pgfqpoint{0.985991in}{0.441418in}}%
\pgfpathlineto{\pgfqpoint{0.940087in}{0.441418in}}%
\pgfpathlineto{\pgfqpoint{0.940087in}{0.441418in}}%
\pgfpathlineto{\pgfqpoint{0.894182in}{0.441418in}}%
\pgfpathlineto{\pgfqpoint{0.894182in}{0.441418in}}%
\pgfpathlineto{\pgfqpoint{0.848278in}{0.441418in}}%
\pgfpathlineto{\pgfqpoint{0.848278in}{0.441418in}}%
\pgfpathlineto{\pgfqpoint{0.802373in}{0.441418in}}%
\pgfpathlineto{\pgfqpoint{0.802373in}{0.441418in}}%
\pgfpathlineto{\pgfqpoint{0.756469in}{0.441418in}}%
\pgfpathlineto{\pgfqpoint{0.756469in}{0.441418in}}%
\pgfpathlineto{\pgfqpoint{0.710564in}{0.441418in}}%
\pgfpathlineto{\pgfqpoint{0.710564in}{0.441418in}}%
\pgfpathlineto{\pgfqpoint{0.664660in}{0.441418in}}%
\pgfpathlineto{\pgfqpoint{0.664660in}{0.441418in}}%
\pgfpathlineto{\pgfqpoint{0.618755in}{0.441418in}}%
\pgfpathlineto{\pgfqpoint{0.618755in}{0.441418in}}%
\pgfpathlineto{\pgfqpoint{0.572851in}{0.441418in}}%
\pgfpathlineto{\pgfqpoint{0.572851in}{0.441418in}}%
\pgfpathlineto{\pgfqpoint{0.526946in}{0.441418in}}%
\pgfpathlineto{\pgfqpoint{0.526946in}{0.441418in}}%
\pgfpathlineto{\pgfqpoint{0.481042in}{0.441418in}}%
\pgfpathlineto{\pgfqpoint{0.481042in}{0.441418in}}%
\pgfpathlineto{\pgfqpoint{0.435137in}{0.441418in}}%
\pgfpathlineto{\pgfqpoint{0.435137in}{0.441418in}}%
\pgfpathlineto{\pgfqpoint{0.389233in}{0.441418in}}%
\pgfpathlineto{\pgfqpoint{0.389233in}{0.441418in}}%
\pgfpathlineto{\pgfqpoint{0.343328in}{0.441418in}}%
\pgfpathlineto{\pgfqpoint{0.343328in}{0.441418in}}%
\pgfpathlineto{\pgfqpoint{0.297424in}{0.441418in}}%
\pgfusepath{fill}%
\end{pgfscope}%
\begin{pgfscope}%
\pgfpathrectangle{\pgfqpoint{0.296148in}{0.441418in}}{\pgfqpoint{2.296592in}{1.173543in}} %
\pgfusepath{clip}%
\pgfsetbuttcap%
\pgfsetmiterjoin%
\pgfsetlinewidth{0.501875pt}%
\definecolor{currentstroke}{rgb}{1.000000,0.000000,0.000000}%
\pgfsetstrokecolor{currentstroke}%
\pgfsetdash{}{0pt}%
\pgfpathmoveto{\pgfqpoint{0.297424in}{0.441418in}}%
\pgfpathlineto{\pgfqpoint{0.297424in}{0.610895in}}%
\pgfpathlineto{\pgfqpoint{0.343328in}{0.610895in}}%
\pgfpathlineto{\pgfqpoint{0.343328in}{1.002659in}}%
\pgfpathlineto{\pgfqpoint{0.389233in}{1.002659in}}%
\pgfpathlineto{\pgfqpoint{0.389233in}{1.298591in}}%
\pgfpathlineto{\pgfqpoint{0.435137in}{1.298591in}}%
\pgfpathlineto{\pgfqpoint{0.435137in}{1.449318in}}%
\pgfpathlineto{\pgfqpoint{0.481042in}{1.449318in}}%
\pgfpathlineto{\pgfqpoint{0.481042in}{1.491327in}}%
\pgfpathlineto{\pgfqpoint{0.526946in}{1.491327in}}%
\pgfpathlineto{\pgfqpoint{0.526946in}{1.469847in}}%
\pgfpathlineto{\pgfqpoint{0.572851in}{1.469847in}}%
\pgfpathlineto{\pgfqpoint{0.572851in}{1.408998in}}%
\pgfpathlineto{\pgfqpoint{0.618755in}{1.408998in}}%
\pgfpathlineto{\pgfqpoint{0.618755in}{1.322433in}}%
\pgfpathlineto{\pgfqpoint{0.664660in}{1.322433in}}%
\pgfpathlineto{\pgfqpoint{0.664660in}{1.230223in}}%
\pgfpathlineto{\pgfqpoint{0.710564in}{1.230223in}}%
\pgfpathlineto{\pgfqpoint{0.710564in}{1.123836in}}%
\pgfpathlineto{\pgfqpoint{0.756469in}{1.123836in}}%
\pgfpathlineto{\pgfqpoint{0.756469in}{1.048779in}}%
\pgfpathlineto{\pgfqpoint{0.802373in}{1.048779in}}%
\pgfpathlineto{\pgfqpoint{0.802373in}{0.961294in}}%
\pgfpathlineto{\pgfqpoint{0.848278in}{0.961294in}}%
\pgfpathlineto{\pgfqpoint{0.848278in}{0.890196in}}%
\pgfpathlineto{\pgfqpoint{0.894182in}{0.890196in}}%
\pgfpathlineto{\pgfqpoint{0.894182in}{0.823485in}}%
\pgfpathlineto{\pgfqpoint{0.940087in}{0.823485in}}%
\pgfpathlineto{\pgfqpoint{0.940087in}{0.774726in}}%
\pgfpathlineto{\pgfqpoint{0.985991in}{0.774726in}}%
\pgfpathlineto{\pgfqpoint{0.985991in}{0.724279in}}%
\pgfpathlineto{\pgfqpoint{1.031896in}{0.724279in}}%
\pgfpathlineto{\pgfqpoint{1.031896in}{0.679385in}}%
\pgfpathlineto{\pgfqpoint{1.077800in}{0.679385in}}%
\pgfpathlineto{\pgfqpoint{1.077800in}{0.647841in}}%
\pgfpathlineto{\pgfqpoint{1.123705in}{0.647841in}}%
\pgfpathlineto{\pgfqpoint{1.123705in}{0.611877in}}%
\pgfpathlineto{\pgfqpoint{1.169609in}{0.611877in}}%
\pgfpathlineto{\pgfqpoint{1.169609in}{0.589630in}}%
\pgfpathlineto{\pgfqpoint{1.215514in}{0.589630in}}%
\pgfpathlineto{\pgfqpoint{1.215514in}{0.562995in}}%
\pgfpathlineto{\pgfqpoint{1.261418in}{0.562995in}}%
\pgfpathlineto{\pgfqpoint{1.261418in}{0.547744in}}%
\pgfpathlineto{\pgfqpoint{1.307323in}{0.547744in}}%
\pgfpathlineto{\pgfqpoint{1.307323in}{0.531020in}}%
\pgfpathlineto{\pgfqpoint{1.353227in}{0.531020in}}%
\pgfpathlineto{\pgfqpoint{1.353227in}{0.515800in}}%
\pgfpathlineto{\pgfqpoint{1.399132in}{0.515800in}}%
\pgfpathlineto{\pgfqpoint{1.399132in}{0.506257in}}%
\pgfpathlineto{\pgfqpoint{1.445036in}{0.506257in}}%
\pgfpathlineto{\pgfqpoint{1.445036in}{0.495241in}}%
\pgfpathlineto{\pgfqpoint{1.490941in}{0.495241in}}%
\pgfpathlineto{\pgfqpoint{1.490941in}{0.488153in}}%
\pgfpathlineto{\pgfqpoint{1.536845in}{0.488153in}}%
\pgfpathlineto{\pgfqpoint{1.536845in}{0.483304in}}%
\pgfpathlineto{\pgfqpoint{1.582750in}{0.483304in}}%
\pgfpathlineto{\pgfqpoint{1.582750in}{0.476523in}}%
\pgfpathlineto{\pgfqpoint{1.628654in}{0.476523in}}%
\pgfpathlineto{\pgfqpoint{1.628654in}{0.472994in}}%
\pgfpathlineto{\pgfqpoint{1.674559in}{0.472994in}}%
\pgfpathlineto{\pgfqpoint{1.674559in}{0.468606in}}%
\pgfpathlineto{\pgfqpoint{1.720463in}{0.468606in}}%
\pgfpathlineto{\pgfqpoint{1.720463in}{0.467501in}}%
\pgfpathlineto{\pgfqpoint{1.766368in}{0.467501in}}%
\pgfpathlineto{\pgfqpoint{1.766368in}{0.462315in}}%
\pgfpathlineto{\pgfqpoint{1.812272in}{0.462315in}}%
\pgfpathlineto{\pgfqpoint{1.812272in}{0.460413in}}%
\pgfpathlineto{\pgfqpoint{1.858177in}{0.460413in}}%
\pgfpathlineto{\pgfqpoint{1.858177in}{0.457897in}}%
\pgfpathlineto{\pgfqpoint{1.904081in}{0.457897in}}%
\pgfpathlineto{\pgfqpoint{1.904081in}{0.456240in}}%
\pgfpathlineto{\pgfqpoint{1.949986in}{0.456240in}}%
\pgfpathlineto{\pgfqpoint{1.949986in}{0.456056in}}%
\pgfpathlineto{\pgfqpoint{1.995890in}{0.456056in}}%
\pgfpathlineto{\pgfqpoint{1.995890in}{0.452956in}}%
\pgfpathlineto{\pgfqpoint{2.041795in}{0.452956in}}%
\pgfpathlineto{\pgfqpoint{2.041795in}{0.451545in}}%
\pgfpathlineto{\pgfqpoint{2.087699in}{0.451545in}}%
\pgfpathlineto{\pgfqpoint{2.087699in}{0.451575in}}%
\pgfpathlineto{\pgfqpoint{2.133604in}{0.451575in}}%
\pgfpathlineto{\pgfqpoint{2.133604in}{0.450747in}}%
\pgfpathlineto{\pgfqpoint{2.179508in}{0.450747in}}%
\pgfpathlineto{\pgfqpoint{2.179508in}{0.449857in}}%
\pgfpathlineto{\pgfqpoint{2.225413in}{0.449857in}}%
\pgfpathlineto{\pgfqpoint{2.225413in}{0.449581in}}%
\pgfpathlineto{\pgfqpoint{2.271317in}{0.449581in}}%
\pgfpathlineto{\pgfqpoint{2.271317in}{0.448814in}}%
\pgfpathlineto{\pgfqpoint{2.317222in}{0.448814in}}%
\pgfpathlineto{\pgfqpoint{2.317222in}{0.447402in}}%
\pgfpathlineto{\pgfqpoint{2.363126in}{0.447402in}}%
\pgfpathlineto{\pgfqpoint{2.363126in}{0.446543in}}%
\pgfpathlineto{\pgfqpoint{2.409031in}{0.446543in}}%
\pgfpathlineto{\pgfqpoint{2.409031in}{0.447034in}}%
\pgfpathlineto{\pgfqpoint{2.454935in}{0.447034in}}%
\pgfpathlineto{\pgfqpoint{2.454935in}{0.446512in}}%
\pgfpathlineto{\pgfqpoint{2.500840in}{0.446512in}}%
\pgfpathlineto{\pgfqpoint{2.500840in}{0.445868in}}%
\pgfpathlineto{\pgfqpoint{2.546744in}{0.445868in}}%
\pgfpathlineto{\pgfqpoint{2.546744in}{0.445500in}}%
\pgfpathlineto{\pgfqpoint{2.592649in}{0.445500in}}%
\pgfpathlineto{\pgfqpoint{2.592649in}{0.441418in}}%
\pgfusepath{stroke}%
\end{pgfscope}%
\begin{pgfscope}%
\pgfpathrectangle{\pgfqpoint{0.296148in}{0.441418in}}{\pgfqpoint{2.296592in}{1.173543in}} %
\pgfusepath{clip}%
\pgfsetbuttcap%
\pgfsetmiterjoin%
\pgfsetlinewidth{0.501875pt}%
\definecolor{currentstroke}{rgb}{1.000000,0.647059,0.000000}%
\pgfsetstrokecolor{currentstroke}%
\pgfsetdash{}{0pt}%
\pgfpathmoveto{\pgfqpoint{0.297424in}{0.441418in}}%
\pgfpathlineto{\pgfqpoint{0.297424in}{0.615449in}}%
\pgfpathlineto{\pgfqpoint{0.343328in}{0.615449in}}%
\pgfpathlineto{\pgfqpoint{0.343328in}{1.006628in}}%
\pgfpathlineto{\pgfqpoint{0.389233in}{1.006628in}}%
\pgfpathlineto{\pgfqpoint{0.389233in}{1.283403in}}%
\pgfpathlineto{\pgfqpoint{0.435137in}{1.283403in}}%
\pgfpathlineto{\pgfqpoint{0.435137in}{1.438517in}}%
\pgfpathlineto{\pgfqpoint{0.481042in}{1.438517in}}%
\pgfpathlineto{\pgfqpoint{0.481042in}{1.489770in}}%
\pgfpathlineto{\pgfqpoint{0.526946in}{1.489770in}}%
\pgfpathlineto{\pgfqpoint{0.526946in}{1.442597in}}%
\pgfpathlineto{\pgfqpoint{0.572851in}{1.442597in}}%
\pgfpathlineto{\pgfqpoint{0.572851in}{1.421293in}}%
\pgfpathlineto{\pgfqpoint{0.618755in}{1.421293in}}%
\pgfpathlineto{\pgfqpoint{0.618755in}{1.327456in}}%
\pgfpathlineto{\pgfqpoint{0.664660in}{1.327456in}}%
\pgfpathlineto{\pgfqpoint{0.664660in}{1.217491in}}%
\pgfpathlineto{\pgfqpoint{0.710564in}{1.217491in}}%
\pgfpathlineto{\pgfqpoint{0.710564in}{1.115638in}}%
\pgfpathlineto{\pgfqpoint{0.756469in}{1.115638in}}%
\pgfpathlineto{\pgfqpoint{0.756469in}{1.053209in}}%
\pgfpathlineto{\pgfqpoint{0.802373in}{1.053209in}}%
\pgfpathlineto{\pgfqpoint{0.802373in}{0.967897in}}%
\pgfpathlineto{\pgfqpoint{0.848278in}{0.967897in}}%
\pgfpathlineto{\pgfqpoint{0.848278in}{0.894562in}}%
\pgfpathlineto{\pgfqpoint{0.894182in}{0.894562in}}%
\pgfpathlineto{\pgfqpoint{0.894182in}{0.827417in}}%
\pgfpathlineto{\pgfqpoint{0.940087in}{0.827417in}}%
\pgfpathlineto{\pgfqpoint{0.940087in}{0.771231in}}%
\pgfpathlineto{\pgfqpoint{0.985991in}{0.771231in}}%
\pgfpathlineto{\pgfqpoint{0.985991in}{0.733266in}}%
\pgfpathlineto{\pgfqpoint{1.031896in}{0.733266in}}%
\pgfpathlineto{\pgfqpoint{1.031896in}{0.680257in}}%
\pgfpathlineto{\pgfqpoint{1.077800in}{0.680257in}}%
\pgfpathlineto{\pgfqpoint{1.077800in}{0.650229in}}%
\pgfpathlineto{\pgfqpoint{1.123705in}{0.650229in}}%
\pgfpathlineto{\pgfqpoint{1.123705in}{0.615789in}}%
\pgfpathlineto{\pgfqpoint{1.169609in}{0.615789in}}%
\pgfpathlineto{\pgfqpoint{1.169609in}{0.597596in}}%
\pgfpathlineto{\pgfqpoint{1.215514in}{0.597596in}}%
\pgfpathlineto{\pgfqpoint{1.215514in}{0.565613in}}%
\pgfpathlineto{\pgfqpoint{1.261418in}{0.565613in}}%
\pgfpathlineto{\pgfqpoint{1.261418in}{0.553380in}}%
\pgfpathlineto{\pgfqpoint{1.307323in}{0.553380in}}%
\pgfpathlineto{\pgfqpoint{1.307323in}{0.531973in}}%
\pgfpathlineto{\pgfqpoint{1.353227in}{0.531973in}}%
\pgfpathlineto{\pgfqpoint{1.353227in}{0.520488in}}%
\pgfpathlineto{\pgfqpoint{1.399132in}{0.520488in}}%
\pgfpathlineto{\pgfqpoint{1.399132in}{0.504184in}}%
\pgfpathlineto{\pgfqpoint{1.445036in}{0.504184in}}%
\pgfpathlineto{\pgfqpoint{1.445036in}{0.496341in}}%
\pgfpathlineto{\pgfqpoint{1.490941in}{0.496341in}}%
\pgfpathlineto{\pgfqpoint{1.490941in}{0.488634in}}%
\pgfpathlineto{\pgfqpoint{1.536845in}{0.488634in}}%
\pgfpathlineto{\pgfqpoint{1.536845in}{0.483084in}}%
\pgfpathlineto{\pgfqpoint{1.582750in}{0.483084in}}%
\pgfpathlineto{\pgfqpoint{1.582750in}{0.478693in}}%
\pgfpathlineto{\pgfqpoint{1.628654in}{0.478693in}}%
\pgfpathlineto{\pgfqpoint{1.628654in}{0.470341in}}%
\pgfpathlineto{\pgfqpoint{1.674559in}{0.470341in}}%
\pgfpathlineto{\pgfqpoint{1.674559in}{0.467715in}}%
\pgfpathlineto{\pgfqpoint{1.720463in}{0.467715in}}%
\pgfpathlineto{\pgfqpoint{1.720463in}{0.468843in}}%
\pgfpathlineto{\pgfqpoint{1.766368in}{0.468843in}}%
\pgfpathlineto{\pgfqpoint{1.766368in}{0.460630in}}%
\pgfpathlineto{\pgfqpoint{1.812272in}{0.460630in}}%
\pgfpathlineto{\pgfqpoint{1.812272in}{0.460215in}}%
\pgfpathlineto{\pgfqpoint{1.858177in}{0.460215in}}%
\pgfpathlineto{\pgfqpoint{1.858177in}{0.455926in}}%
\pgfpathlineto{\pgfqpoint{1.904081in}{0.455926in}}%
\pgfpathlineto{\pgfqpoint{1.904081in}{0.454650in}}%
\pgfpathlineto{\pgfqpoint{1.949986in}{0.454650in}}%
\pgfpathlineto{\pgfqpoint{1.949986in}{0.457183in}}%
\pgfpathlineto{\pgfqpoint{1.995890in}{0.457183in}}%
\pgfpathlineto{\pgfqpoint{1.995890in}{0.453309in}}%
\pgfpathlineto{\pgfqpoint{2.041795in}{0.453309in}}%
\pgfpathlineto{\pgfqpoint{2.041795in}{0.450141in}}%
\pgfpathlineto{\pgfqpoint{2.087699in}{0.450141in}}%
\pgfpathlineto{\pgfqpoint{2.087699in}{0.452551in}}%
\pgfpathlineto{\pgfqpoint{2.133604in}{0.452551in}}%
\pgfpathlineto{\pgfqpoint{2.133604in}{0.452283in}}%
\pgfpathlineto{\pgfqpoint{2.179508in}{0.452283in}}%
\pgfpathlineto{\pgfqpoint{2.179508in}{0.449677in}}%
\pgfpathlineto{\pgfqpoint{2.225413in}{0.449677in}}%
\pgfpathlineto{\pgfqpoint{2.225413in}{0.449587in}}%
\pgfpathlineto{\pgfqpoint{2.271317in}{0.449587in}}%
\pgfpathlineto{\pgfqpoint{2.271317in}{0.448113in}}%
\pgfpathlineto{\pgfqpoint{2.317222in}{0.448113in}}%
\pgfpathlineto{\pgfqpoint{2.317222in}{0.446476in}}%
\pgfpathlineto{\pgfqpoint{2.363126in}{0.446476in}}%
\pgfpathlineto{\pgfqpoint{2.363126in}{0.446759in}}%
\pgfpathlineto{\pgfqpoint{2.409031in}{0.446759in}}%
\pgfpathlineto{\pgfqpoint{2.409031in}{0.447007in}}%
\pgfpathlineto{\pgfqpoint{2.454935in}{0.447007in}}%
\pgfpathlineto{\pgfqpoint{2.454935in}{0.447634in}}%
\pgfpathlineto{\pgfqpoint{2.500840in}{0.447634in}}%
\pgfpathlineto{\pgfqpoint{2.500840in}{0.446364in}}%
\pgfpathlineto{\pgfqpoint{2.546744in}{0.446364in}}%
\pgfpathlineto{\pgfqpoint{2.546744in}{0.445124in}}%
\pgfpathlineto{\pgfqpoint{2.592649in}{0.445124in}}%
\pgfpathlineto{\pgfqpoint{2.592649in}{0.441418in}}%
\pgfusepath{stroke}%
\end{pgfscope}%
\begin{pgfscope}%
\pgfsetrectcap%
\pgfsetmiterjoin%
\pgfsetlinewidth{1.003750pt}%
\definecolor{currentstroke}{rgb}{0.000000,0.000000,0.000000}%
\pgfsetstrokecolor{currentstroke}%
\pgfsetdash{}{0pt}%
\pgfpathmoveto{\pgfqpoint{0.296148in}{1.614961in}}%
\pgfpathlineto{\pgfqpoint{2.592740in}{1.614961in}}%
\pgfusepath{stroke}%
\end{pgfscope}%
\begin{pgfscope}%
\pgfsetrectcap%
\pgfsetmiterjoin%
\pgfsetlinewidth{1.003750pt}%
\definecolor{currentstroke}{rgb}{0.000000,0.000000,0.000000}%
\pgfsetstrokecolor{currentstroke}%
\pgfsetdash{}{0pt}%
\pgfpathmoveto{\pgfqpoint{2.592740in}{0.441418in}}%
\pgfpathlineto{\pgfqpoint{2.592740in}{1.614961in}}%
\pgfusepath{stroke}%
\end{pgfscope}%
\begin{pgfscope}%
\pgfsetrectcap%
\pgfsetmiterjoin%
\pgfsetlinewidth{1.003750pt}%
\definecolor{currentstroke}{rgb}{0.000000,0.000000,0.000000}%
\pgfsetstrokecolor{currentstroke}%
\pgfsetdash{}{0pt}%
\pgfpathmoveto{\pgfqpoint{0.296148in}{0.441418in}}%
\pgfpathlineto{\pgfqpoint{2.592740in}{0.441418in}}%
\pgfusepath{stroke}%
\end{pgfscope}%
\begin{pgfscope}%
\pgfsetrectcap%
\pgfsetmiterjoin%
\pgfsetlinewidth{1.003750pt}%
\definecolor{currentstroke}{rgb}{0.000000,0.000000,0.000000}%
\pgfsetstrokecolor{currentstroke}%
\pgfsetdash{}{0pt}%
\pgfpathmoveto{\pgfqpoint{0.296148in}{0.441418in}}%
\pgfpathlineto{\pgfqpoint{0.296148in}{1.614961in}}%
\pgfusepath{stroke}%
\end{pgfscope}%
\begin{pgfscope}%
\pgfsetbuttcap%
\pgfsetroundjoin%
\definecolor{currentfill}{rgb}{0.000000,0.000000,0.000000}%
\pgfsetfillcolor{currentfill}%
\pgfsetlinewidth{0.501875pt}%
\definecolor{currentstroke}{rgb}{0.000000,0.000000,0.000000}%
\pgfsetstrokecolor{currentstroke}%
\pgfsetdash{}{0pt}%
\pgfsys@defobject{currentmarker}{\pgfqpoint{0.000000in}{0.000000in}}{\pgfqpoint{0.000000in}{0.069444in}}{%
\pgfpathmoveto{\pgfqpoint{0.000000in}{0.000000in}}%
\pgfpathlineto{\pgfqpoint{0.000000in}{0.069444in}}%
\pgfusepath{stroke,fill}%
}%
\begin{pgfscope}%
\pgfsys@transformshift{0.296148in}{0.441418in}%
\pgfsys@useobject{currentmarker}{}%
\end{pgfscope}%
\end{pgfscope}%
\begin{pgfscope}%
\pgfsetbuttcap%
\pgfsetroundjoin%
\definecolor{currentfill}{rgb}{0.000000,0.000000,0.000000}%
\pgfsetfillcolor{currentfill}%
\pgfsetlinewidth{0.501875pt}%
\definecolor{currentstroke}{rgb}{0.000000,0.000000,0.000000}%
\pgfsetstrokecolor{currentstroke}%
\pgfsetdash{}{0pt}%
\pgfsys@defobject{currentmarker}{\pgfqpoint{0.000000in}{-0.069444in}}{\pgfqpoint{0.000000in}{0.000000in}}{%
\pgfpathmoveto{\pgfqpoint{0.000000in}{0.000000in}}%
\pgfpathlineto{\pgfqpoint{0.000000in}{-0.069444in}}%
\pgfusepath{stroke,fill}%
}%
\begin{pgfscope}%
\pgfsys@transformshift{0.296148in}{1.614961in}%
\pgfsys@useobject{currentmarker}{}%
\end{pgfscope}%
\end{pgfscope}%
\begin{pgfscope}%
\pgftext[x=0.296148in,y=0.371974in,,top]{\rmfamily\fontsize{8.000000}{9.600000}\selectfont 0}%
\end{pgfscope}%
\begin{pgfscope}%
\pgfsetbuttcap%
\pgfsetroundjoin%
\definecolor{currentfill}{rgb}{0.000000,0.000000,0.000000}%
\pgfsetfillcolor{currentfill}%
\pgfsetlinewidth{0.501875pt}%
\definecolor{currentstroke}{rgb}{0.000000,0.000000,0.000000}%
\pgfsetstrokecolor{currentstroke}%
\pgfsetdash{}{0pt}%
\pgfsys@defobject{currentmarker}{\pgfqpoint{0.000000in}{0.000000in}}{\pgfqpoint{0.000000in}{0.069444in}}{%
\pgfpathmoveto{\pgfqpoint{0.000000in}{0.000000in}}%
\pgfpathlineto{\pgfqpoint{0.000000in}{0.069444in}}%
\pgfusepath{stroke,fill}%
}%
\begin{pgfscope}%
\pgfsys@transformshift{0.583222in}{0.441418in}%
\pgfsys@useobject{currentmarker}{}%
\end{pgfscope}%
\end{pgfscope}%
\begin{pgfscope}%
\pgfsetbuttcap%
\pgfsetroundjoin%
\definecolor{currentfill}{rgb}{0.000000,0.000000,0.000000}%
\pgfsetfillcolor{currentfill}%
\pgfsetlinewidth{0.501875pt}%
\definecolor{currentstroke}{rgb}{0.000000,0.000000,0.000000}%
\pgfsetstrokecolor{currentstroke}%
\pgfsetdash{}{0pt}%
\pgfsys@defobject{currentmarker}{\pgfqpoint{0.000000in}{-0.069444in}}{\pgfqpoint{0.000000in}{0.000000in}}{%
\pgfpathmoveto{\pgfqpoint{0.000000in}{0.000000in}}%
\pgfpathlineto{\pgfqpoint{0.000000in}{-0.069444in}}%
\pgfusepath{stroke,fill}%
}%
\begin{pgfscope}%
\pgfsys@transformshift{0.583222in}{1.614961in}%
\pgfsys@useobject{currentmarker}{}%
\end{pgfscope}%
\end{pgfscope}%
\begin{pgfscope}%
\pgftext[x=0.583222in,y=0.371974in,,top]{\rmfamily\fontsize{8.000000}{9.600000}\selectfont 1}%
\end{pgfscope}%
\begin{pgfscope}%
\pgfsetbuttcap%
\pgfsetroundjoin%
\definecolor{currentfill}{rgb}{0.000000,0.000000,0.000000}%
\pgfsetfillcolor{currentfill}%
\pgfsetlinewidth{0.501875pt}%
\definecolor{currentstroke}{rgb}{0.000000,0.000000,0.000000}%
\pgfsetstrokecolor{currentstroke}%
\pgfsetdash{}{0pt}%
\pgfsys@defobject{currentmarker}{\pgfqpoint{0.000000in}{0.000000in}}{\pgfqpoint{0.000000in}{0.069444in}}{%
\pgfpathmoveto{\pgfqpoint{0.000000in}{0.000000in}}%
\pgfpathlineto{\pgfqpoint{0.000000in}{0.069444in}}%
\pgfusepath{stroke,fill}%
}%
\begin{pgfscope}%
\pgfsys@transformshift{0.870296in}{0.441418in}%
\pgfsys@useobject{currentmarker}{}%
\end{pgfscope}%
\end{pgfscope}%
\begin{pgfscope}%
\pgfsetbuttcap%
\pgfsetroundjoin%
\definecolor{currentfill}{rgb}{0.000000,0.000000,0.000000}%
\pgfsetfillcolor{currentfill}%
\pgfsetlinewidth{0.501875pt}%
\definecolor{currentstroke}{rgb}{0.000000,0.000000,0.000000}%
\pgfsetstrokecolor{currentstroke}%
\pgfsetdash{}{0pt}%
\pgfsys@defobject{currentmarker}{\pgfqpoint{0.000000in}{-0.069444in}}{\pgfqpoint{0.000000in}{0.000000in}}{%
\pgfpathmoveto{\pgfqpoint{0.000000in}{0.000000in}}%
\pgfpathlineto{\pgfqpoint{0.000000in}{-0.069444in}}%
\pgfusepath{stroke,fill}%
}%
\begin{pgfscope}%
\pgfsys@transformshift{0.870296in}{1.614961in}%
\pgfsys@useobject{currentmarker}{}%
\end{pgfscope}%
\end{pgfscope}%
\begin{pgfscope}%
\pgftext[x=0.870296in,y=0.371974in,,top]{\rmfamily\fontsize{8.000000}{9.600000}\selectfont 2}%
\end{pgfscope}%
\begin{pgfscope}%
\pgfsetbuttcap%
\pgfsetroundjoin%
\definecolor{currentfill}{rgb}{0.000000,0.000000,0.000000}%
\pgfsetfillcolor{currentfill}%
\pgfsetlinewidth{0.501875pt}%
\definecolor{currentstroke}{rgb}{0.000000,0.000000,0.000000}%
\pgfsetstrokecolor{currentstroke}%
\pgfsetdash{}{0pt}%
\pgfsys@defobject{currentmarker}{\pgfqpoint{0.000000in}{0.000000in}}{\pgfqpoint{0.000000in}{0.069444in}}{%
\pgfpathmoveto{\pgfqpoint{0.000000in}{0.000000in}}%
\pgfpathlineto{\pgfqpoint{0.000000in}{0.069444in}}%
\pgfusepath{stroke,fill}%
}%
\begin{pgfscope}%
\pgfsys@transformshift{1.157370in}{0.441418in}%
\pgfsys@useobject{currentmarker}{}%
\end{pgfscope}%
\end{pgfscope}%
\begin{pgfscope}%
\pgfsetbuttcap%
\pgfsetroundjoin%
\definecolor{currentfill}{rgb}{0.000000,0.000000,0.000000}%
\pgfsetfillcolor{currentfill}%
\pgfsetlinewidth{0.501875pt}%
\definecolor{currentstroke}{rgb}{0.000000,0.000000,0.000000}%
\pgfsetstrokecolor{currentstroke}%
\pgfsetdash{}{0pt}%
\pgfsys@defobject{currentmarker}{\pgfqpoint{0.000000in}{-0.069444in}}{\pgfqpoint{0.000000in}{0.000000in}}{%
\pgfpathmoveto{\pgfqpoint{0.000000in}{0.000000in}}%
\pgfpathlineto{\pgfqpoint{0.000000in}{-0.069444in}}%
\pgfusepath{stroke,fill}%
}%
\begin{pgfscope}%
\pgfsys@transformshift{1.157370in}{1.614961in}%
\pgfsys@useobject{currentmarker}{}%
\end{pgfscope}%
\end{pgfscope}%
\begin{pgfscope}%
\pgftext[x=1.157370in,y=0.371974in,,top]{\rmfamily\fontsize{8.000000}{9.600000}\selectfont 3}%
\end{pgfscope}%
\begin{pgfscope}%
\pgfsetbuttcap%
\pgfsetroundjoin%
\definecolor{currentfill}{rgb}{0.000000,0.000000,0.000000}%
\pgfsetfillcolor{currentfill}%
\pgfsetlinewidth{0.501875pt}%
\definecolor{currentstroke}{rgb}{0.000000,0.000000,0.000000}%
\pgfsetstrokecolor{currentstroke}%
\pgfsetdash{}{0pt}%
\pgfsys@defobject{currentmarker}{\pgfqpoint{0.000000in}{0.000000in}}{\pgfqpoint{0.000000in}{0.069444in}}{%
\pgfpathmoveto{\pgfqpoint{0.000000in}{0.000000in}}%
\pgfpathlineto{\pgfqpoint{0.000000in}{0.069444in}}%
\pgfusepath{stroke,fill}%
}%
\begin{pgfscope}%
\pgfsys@transformshift{1.444444in}{0.441418in}%
\pgfsys@useobject{currentmarker}{}%
\end{pgfscope}%
\end{pgfscope}%
\begin{pgfscope}%
\pgfsetbuttcap%
\pgfsetroundjoin%
\definecolor{currentfill}{rgb}{0.000000,0.000000,0.000000}%
\pgfsetfillcolor{currentfill}%
\pgfsetlinewidth{0.501875pt}%
\definecolor{currentstroke}{rgb}{0.000000,0.000000,0.000000}%
\pgfsetstrokecolor{currentstroke}%
\pgfsetdash{}{0pt}%
\pgfsys@defobject{currentmarker}{\pgfqpoint{0.000000in}{-0.069444in}}{\pgfqpoint{0.000000in}{0.000000in}}{%
\pgfpathmoveto{\pgfqpoint{0.000000in}{0.000000in}}%
\pgfpathlineto{\pgfqpoint{0.000000in}{-0.069444in}}%
\pgfusepath{stroke,fill}%
}%
\begin{pgfscope}%
\pgfsys@transformshift{1.444444in}{1.614961in}%
\pgfsys@useobject{currentmarker}{}%
\end{pgfscope}%
\end{pgfscope}%
\begin{pgfscope}%
\pgftext[x=1.444444in,y=0.371974in,,top]{\rmfamily\fontsize{8.000000}{9.600000}\selectfont 4}%
\end{pgfscope}%
\begin{pgfscope}%
\pgfsetbuttcap%
\pgfsetroundjoin%
\definecolor{currentfill}{rgb}{0.000000,0.000000,0.000000}%
\pgfsetfillcolor{currentfill}%
\pgfsetlinewidth{0.501875pt}%
\definecolor{currentstroke}{rgb}{0.000000,0.000000,0.000000}%
\pgfsetstrokecolor{currentstroke}%
\pgfsetdash{}{0pt}%
\pgfsys@defobject{currentmarker}{\pgfqpoint{0.000000in}{0.000000in}}{\pgfqpoint{0.000000in}{0.069444in}}{%
\pgfpathmoveto{\pgfqpoint{0.000000in}{0.000000in}}%
\pgfpathlineto{\pgfqpoint{0.000000in}{0.069444in}}%
\pgfusepath{stroke,fill}%
}%
\begin{pgfscope}%
\pgfsys@transformshift{1.731518in}{0.441418in}%
\pgfsys@useobject{currentmarker}{}%
\end{pgfscope}%
\end{pgfscope}%
\begin{pgfscope}%
\pgfsetbuttcap%
\pgfsetroundjoin%
\definecolor{currentfill}{rgb}{0.000000,0.000000,0.000000}%
\pgfsetfillcolor{currentfill}%
\pgfsetlinewidth{0.501875pt}%
\definecolor{currentstroke}{rgb}{0.000000,0.000000,0.000000}%
\pgfsetstrokecolor{currentstroke}%
\pgfsetdash{}{0pt}%
\pgfsys@defobject{currentmarker}{\pgfqpoint{0.000000in}{-0.069444in}}{\pgfqpoint{0.000000in}{0.000000in}}{%
\pgfpathmoveto{\pgfqpoint{0.000000in}{0.000000in}}%
\pgfpathlineto{\pgfqpoint{0.000000in}{-0.069444in}}%
\pgfusepath{stroke,fill}%
}%
\begin{pgfscope}%
\pgfsys@transformshift{1.731518in}{1.614961in}%
\pgfsys@useobject{currentmarker}{}%
\end{pgfscope}%
\end{pgfscope}%
\begin{pgfscope}%
\pgftext[x=1.731518in,y=0.371974in,,top]{\rmfamily\fontsize{8.000000}{9.600000}\selectfont 5}%
\end{pgfscope}%
\begin{pgfscope}%
\pgfsetbuttcap%
\pgfsetroundjoin%
\definecolor{currentfill}{rgb}{0.000000,0.000000,0.000000}%
\pgfsetfillcolor{currentfill}%
\pgfsetlinewidth{0.501875pt}%
\definecolor{currentstroke}{rgb}{0.000000,0.000000,0.000000}%
\pgfsetstrokecolor{currentstroke}%
\pgfsetdash{}{0pt}%
\pgfsys@defobject{currentmarker}{\pgfqpoint{0.000000in}{0.000000in}}{\pgfqpoint{0.000000in}{0.069444in}}{%
\pgfpathmoveto{\pgfqpoint{0.000000in}{0.000000in}}%
\pgfpathlineto{\pgfqpoint{0.000000in}{0.069444in}}%
\pgfusepath{stroke,fill}%
}%
\begin{pgfscope}%
\pgfsys@transformshift{2.018592in}{0.441418in}%
\pgfsys@useobject{currentmarker}{}%
\end{pgfscope}%
\end{pgfscope}%
\begin{pgfscope}%
\pgfsetbuttcap%
\pgfsetroundjoin%
\definecolor{currentfill}{rgb}{0.000000,0.000000,0.000000}%
\pgfsetfillcolor{currentfill}%
\pgfsetlinewidth{0.501875pt}%
\definecolor{currentstroke}{rgb}{0.000000,0.000000,0.000000}%
\pgfsetstrokecolor{currentstroke}%
\pgfsetdash{}{0pt}%
\pgfsys@defobject{currentmarker}{\pgfqpoint{0.000000in}{-0.069444in}}{\pgfqpoint{0.000000in}{0.000000in}}{%
\pgfpathmoveto{\pgfqpoint{0.000000in}{0.000000in}}%
\pgfpathlineto{\pgfqpoint{0.000000in}{-0.069444in}}%
\pgfusepath{stroke,fill}%
}%
\begin{pgfscope}%
\pgfsys@transformshift{2.018592in}{1.614961in}%
\pgfsys@useobject{currentmarker}{}%
\end{pgfscope}%
\end{pgfscope}%
\begin{pgfscope}%
\pgftext[x=2.018592in,y=0.371974in,,top]{\rmfamily\fontsize{8.000000}{9.600000}\selectfont 6}%
\end{pgfscope}%
\begin{pgfscope}%
\pgfsetbuttcap%
\pgfsetroundjoin%
\definecolor{currentfill}{rgb}{0.000000,0.000000,0.000000}%
\pgfsetfillcolor{currentfill}%
\pgfsetlinewidth{0.501875pt}%
\definecolor{currentstroke}{rgb}{0.000000,0.000000,0.000000}%
\pgfsetstrokecolor{currentstroke}%
\pgfsetdash{}{0pt}%
\pgfsys@defobject{currentmarker}{\pgfqpoint{0.000000in}{0.000000in}}{\pgfqpoint{0.000000in}{0.069444in}}{%
\pgfpathmoveto{\pgfqpoint{0.000000in}{0.000000in}}%
\pgfpathlineto{\pgfqpoint{0.000000in}{0.069444in}}%
\pgfusepath{stroke,fill}%
}%
\begin{pgfscope}%
\pgfsys@transformshift{2.305666in}{0.441418in}%
\pgfsys@useobject{currentmarker}{}%
\end{pgfscope}%
\end{pgfscope}%
\begin{pgfscope}%
\pgfsetbuttcap%
\pgfsetroundjoin%
\definecolor{currentfill}{rgb}{0.000000,0.000000,0.000000}%
\pgfsetfillcolor{currentfill}%
\pgfsetlinewidth{0.501875pt}%
\definecolor{currentstroke}{rgb}{0.000000,0.000000,0.000000}%
\pgfsetstrokecolor{currentstroke}%
\pgfsetdash{}{0pt}%
\pgfsys@defobject{currentmarker}{\pgfqpoint{0.000000in}{-0.069444in}}{\pgfqpoint{0.000000in}{0.000000in}}{%
\pgfpathmoveto{\pgfqpoint{0.000000in}{0.000000in}}%
\pgfpathlineto{\pgfqpoint{0.000000in}{-0.069444in}}%
\pgfusepath{stroke,fill}%
}%
\begin{pgfscope}%
\pgfsys@transformshift{2.305666in}{1.614961in}%
\pgfsys@useobject{currentmarker}{}%
\end{pgfscope}%
\end{pgfscope}%
\begin{pgfscope}%
\pgftext[x=2.305666in,y=0.371974in,,top]{\rmfamily\fontsize{8.000000}{9.600000}\selectfont 7}%
\end{pgfscope}%
\begin{pgfscope}%
\pgfsetbuttcap%
\pgfsetroundjoin%
\definecolor{currentfill}{rgb}{0.000000,0.000000,0.000000}%
\pgfsetfillcolor{currentfill}%
\pgfsetlinewidth{0.501875pt}%
\definecolor{currentstroke}{rgb}{0.000000,0.000000,0.000000}%
\pgfsetstrokecolor{currentstroke}%
\pgfsetdash{}{0pt}%
\pgfsys@defobject{currentmarker}{\pgfqpoint{0.000000in}{0.000000in}}{\pgfqpoint{0.000000in}{0.069444in}}{%
\pgfpathmoveto{\pgfqpoint{0.000000in}{0.000000in}}%
\pgfpathlineto{\pgfqpoint{0.000000in}{0.069444in}}%
\pgfusepath{stroke,fill}%
}%
\begin{pgfscope}%
\pgfsys@transformshift{2.592740in}{0.441418in}%
\pgfsys@useobject{currentmarker}{}%
\end{pgfscope}%
\end{pgfscope}%
\begin{pgfscope}%
\pgfsetbuttcap%
\pgfsetroundjoin%
\definecolor{currentfill}{rgb}{0.000000,0.000000,0.000000}%
\pgfsetfillcolor{currentfill}%
\pgfsetlinewidth{0.501875pt}%
\definecolor{currentstroke}{rgb}{0.000000,0.000000,0.000000}%
\pgfsetstrokecolor{currentstroke}%
\pgfsetdash{}{0pt}%
\pgfsys@defobject{currentmarker}{\pgfqpoint{0.000000in}{-0.069444in}}{\pgfqpoint{0.000000in}{0.000000in}}{%
\pgfpathmoveto{\pgfqpoint{0.000000in}{0.000000in}}%
\pgfpathlineto{\pgfqpoint{0.000000in}{-0.069444in}}%
\pgfusepath{stroke,fill}%
}%
\begin{pgfscope}%
\pgfsys@transformshift{2.592740in}{1.614961in}%
\pgfsys@useobject{currentmarker}{}%
\end{pgfscope}%
\end{pgfscope}%
\begin{pgfscope}%
\pgftext[x=2.592740in,y=0.371974in,,top]{\rmfamily\fontsize{8.000000}{9.600000}\selectfont 8}%
\end{pgfscope}%
\begin{pgfscope}%
\pgftext[x=1.444444in,y=0.194999in,,top]{\rmfamily\fontsize{9.000000}{10.800000}\selectfont \(\displaystyle B^0\ \mathrm{vertex}\ \chi^2 / \mathrm{ndf}\)}%
\end{pgfscope}%
\begin{pgfscope}%
\pgfsetbuttcap%
\pgfsetroundjoin%
\definecolor{currentfill}{rgb}{0.000000,0.000000,0.000000}%
\pgfsetfillcolor{currentfill}%
\pgfsetlinewidth{0.501875pt}%
\definecolor{currentstroke}{rgb}{0.000000,0.000000,0.000000}%
\pgfsetstrokecolor{currentstroke}%
\pgfsetdash{}{0pt}%
\pgfsys@defobject{currentmarker}{\pgfqpoint{0.000000in}{0.000000in}}{\pgfqpoint{0.069444in}{0.000000in}}{%
\pgfpathmoveto{\pgfqpoint{0.000000in}{0.000000in}}%
\pgfpathlineto{\pgfqpoint{0.069444in}{0.000000in}}%
\pgfusepath{stroke,fill}%
}%
\begin{pgfscope}%
\pgfsys@transformshift{0.296148in}{0.441418in}%
\pgfsys@useobject{currentmarker}{}%
\end{pgfscope}%
\end{pgfscope}%
\begin{pgfscope}%
\pgfsetbuttcap%
\pgfsetroundjoin%
\definecolor{currentfill}{rgb}{0.000000,0.000000,0.000000}%
\pgfsetfillcolor{currentfill}%
\pgfsetlinewidth{0.501875pt}%
\definecolor{currentstroke}{rgb}{0.000000,0.000000,0.000000}%
\pgfsetstrokecolor{currentstroke}%
\pgfsetdash{}{0pt}%
\pgfsys@defobject{currentmarker}{\pgfqpoint{-0.069444in}{0.000000in}}{\pgfqpoint{0.000000in}{0.000000in}}{%
\pgfpathmoveto{\pgfqpoint{0.000000in}{0.000000in}}%
\pgfpathlineto{\pgfqpoint{-0.069444in}{0.000000in}}%
\pgfusepath{stroke,fill}%
}%
\begin{pgfscope}%
\pgfsys@transformshift{2.592740in}{0.441418in}%
\pgfsys@useobject{currentmarker}{}%
\end{pgfscope}%
\end{pgfscope}%
\begin{pgfscope}%
\pgftext[x=0.226704in,y=0.441418in,right,]{\rmfamily\fontsize{8.000000}{9.600000}\selectfont 0.0}%
\end{pgfscope}%
\begin{pgfscope}%
\pgfsetbuttcap%
\pgfsetroundjoin%
\definecolor{currentfill}{rgb}{0.000000,0.000000,0.000000}%
\pgfsetfillcolor{currentfill}%
\pgfsetlinewidth{0.501875pt}%
\definecolor{currentstroke}{rgb}{0.000000,0.000000,0.000000}%
\pgfsetstrokecolor{currentstroke}%
\pgfsetdash{}{0pt}%
\pgfsys@defobject{currentmarker}{\pgfqpoint{0.000000in}{0.000000in}}{\pgfqpoint{0.069444in}{0.000000in}}{%
\pgfpathmoveto{\pgfqpoint{0.000000in}{0.000000in}}%
\pgfpathlineto{\pgfqpoint{0.069444in}{0.000000in}}%
\pgfusepath{stroke,fill}%
}%
\begin{pgfscope}%
\pgfsys@transformshift{0.296148in}{0.637009in}%
\pgfsys@useobject{currentmarker}{}%
\end{pgfscope}%
\end{pgfscope}%
\begin{pgfscope}%
\pgfsetbuttcap%
\pgfsetroundjoin%
\definecolor{currentfill}{rgb}{0.000000,0.000000,0.000000}%
\pgfsetfillcolor{currentfill}%
\pgfsetlinewidth{0.501875pt}%
\definecolor{currentstroke}{rgb}{0.000000,0.000000,0.000000}%
\pgfsetstrokecolor{currentstroke}%
\pgfsetdash{}{0pt}%
\pgfsys@defobject{currentmarker}{\pgfqpoint{-0.069444in}{0.000000in}}{\pgfqpoint{0.000000in}{0.000000in}}{%
\pgfpathmoveto{\pgfqpoint{0.000000in}{0.000000in}}%
\pgfpathlineto{\pgfqpoint{-0.069444in}{0.000000in}}%
\pgfusepath{stroke,fill}%
}%
\begin{pgfscope}%
\pgfsys@transformshift{2.592740in}{0.637009in}%
\pgfsys@useobject{currentmarker}{}%
\end{pgfscope}%
\end{pgfscope}%
\begin{pgfscope}%
\pgftext[x=0.226704in,y=0.637009in,right,]{\rmfamily\fontsize{8.000000}{9.600000}\selectfont 0.1}%
\end{pgfscope}%
\begin{pgfscope}%
\pgfsetbuttcap%
\pgfsetroundjoin%
\definecolor{currentfill}{rgb}{0.000000,0.000000,0.000000}%
\pgfsetfillcolor{currentfill}%
\pgfsetlinewidth{0.501875pt}%
\definecolor{currentstroke}{rgb}{0.000000,0.000000,0.000000}%
\pgfsetstrokecolor{currentstroke}%
\pgfsetdash{}{0pt}%
\pgfsys@defobject{currentmarker}{\pgfqpoint{0.000000in}{0.000000in}}{\pgfqpoint{0.069444in}{0.000000in}}{%
\pgfpathmoveto{\pgfqpoint{0.000000in}{0.000000in}}%
\pgfpathlineto{\pgfqpoint{0.069444in}{0.000000in}}%
\pgfusepath{stroke,fill}%
}%
\begin{pgfscope}%
\pgfsys@transformshift{0.296148in}{0.832599in}%
\pgfsys@useobject{currentmarker}{}%
\end{pgfscope}%
\end{pgfscope}%
\begin{pgfscope}%
\pgfsetbuttcap%
\pgfsetroundjoin%
\definecolor{currentfill}{rgb}{0.000000,0.000000,0.000000}%
\pgfsetfillcolor{currentfill}%
\pgfsetlinewidth{0.501875pt}%
\definecolor{currentstroke}{rgb}{0.000000,0.000000,0.000000}%
\pgfsetstrokecolor{currentstroke}%
\pgfsetdash{}{0pt}%
\pgfsys@defobject{currentmarker}{\pgfqpoint{-0.069444in}{0.000000in}}{\pgfqpoint{0.000000in}{0.000000in}}{%
\pgfpathmoveto{\pgfqpoint{0.000000in}{0.000000in}}%
\pgfpathlineto{\pgfqpoint{-0.069444in}{0.000000in}}%
\pgfusepath{stroke,fill}%
}%
\begin{pgfscope}%
\pgfsys@transformshift{2.592740in}{0.832599in}%
\pgfsys@useobject{currentmarker}{}%
\end{pgfscope}%
\end{pgfscope}%
\begin{pgfscope}%
\pgftext[x=0.226704in,y=0.832599in,right,]{\rmfamily\fontsize{8.000000}{9.600000}\selectfont 0.2}%
\end{pgfscope}%
\begin{pgfscope}%
\pgfsetbuttcap%
\pgfsetroundjoin%
\definecolor{currentfill}{rgb}{0.000000,0.000000,0.000000}%
\pgfsetfillcolor{currentfill}%
\pgfsetlinewidth{0.501875pt}%
\definecolor{currentstroke}{rgb}{0.000000,0.000000,0.000000}%
\pgfsetstrokecolor{currentstroke}%
\pgfsetdash{}{0pt}%
\pgfsys@defobject{currentmarker}{\pgfqpoint{0.000000in}{0.000000in}}{\pgfqpoint{0.069444in}{0.000000in}}{%
\pgfpathmoveto{\pgfqpoint{0.000000in}{0.000000in}}%
\pgfpathlineto{\pgfqpoint{0.069444in}{0.000000in}}%
\pgfusepath{stroke,fill}%
}%
\begin{pgfscope}%
\pgfsys@transformshift{0.296148in}{1.028190in}%
\pgfsys@useobject{currentmarker}{}%
\end{pgfscope}%
\end{pgfscope}%
\begin{pgfscope}%
\pgfsetbuttcap%
\pgfsetroundjoin%
\definecolor{currentfill}{rgb}{0.000000,0.000000,0.000000}%
\pgfsetfillcolor{currentfill}%
\pgfsetlinewidth{0.501875pt}%
\definecolor{currentstroke}{rgb}{0.000000,0.000000,0.000000}%
\pgfsetstrokecolor{currentstroke}%
\pgfsetdash{}{0pt}%
\pgfsys@defobject{currentmarker}{\pgfqpoint{-0.069444in}{0.000000in}}{\pgfqpoint{0.000000in}{0.000000in}}{%
\pgfpathmoveto{\pgfqpoint{0.000000in}{0.000000in}}%
\pgfpathlineto{\pgfqpoint{-0.069444in}{0.000000in}}%
\pgfusepath{stroke,fill}%
}%
\begin{pgfscope}%
\pgfsys@transformshift{2.592740in}{1.028190in}%
\pgfsys@useobject{currentmarker}{}%
\end{pgfscope}%
\end{pgfscope}%
\begin{pgfscope}%
\pgftext[x=0.226704in,y=1.028190in,right,]{\rmfamily\fontsize{8.000000}{9.600000}\selectfont 0.3}%
\end{pgfscope}%
\begin{pgfscope}%
\pgfsetbuttcap%
\pgfsetroundjoin%
\definecolor{currentfill}{rgb}{0.000000,0.000000,0.000000}%
\pgfsetfillcolor{currentfill}%
\pgfsetlinewidth{0.501875pt}%
\definecolor{currentstroke}{rgb}{0.000000,0.000000,0.000000}%
\pgfsetstrokecolor{currentstroke}%
\pgfsetdash{}{0pt}%
\pgfsys@defobject{currentmarker}{\pgfqpoint{0.000000in}{0.000000in}}{\pgfqpoint{0.069444in}{0.000000in}}{%
\pgfpathmoveto{\pgfqpoint{0.000000in}{0.000000in}}%
\pgfpathlineto{\pgfqpoint{0.069444in}{0.000000in}}%
\pgfusepath{stroke,fill}%
}%
\begin{pgfscope}%
\pgfsys@transformshift{0.296148in}{1.223780in}%
\pgfsys@useobject{currentmarker}{}%
\end{pgfscope}%
\end{pgfscope}%
\begin{pgfscope}%
\pgfsetbuttcap%
\pgfsetroundjoin%
\definecolor{currentfill}{rgb}{0.000000,0.000000,0.000000}%
\pgfsetfillcolor{currentfill}%
\pgfsetlinewidth{0.501875pt}%
\definecolor{currentstroke}{rgb}{0.000000,0.000000,0.000000}%
\pgfsetstrokecolor{currentstroke}%
\pgfsetdash{}{0pt}%
\pgfsys@defobject{currentmarker}{\pgfqpoint{-0.069444in}{0.000000in}}{\pgfqpoint{0.000000in}{0.000000in}}{%
\pgfpathmoveto{\pgfqpoint{0.000000in}{0.000000in}}%
\pgfpathlineto{\pgfqpoint{-0.069444in}{0.000000in}}%
\pgfusepath{stroke,fill}%
}%
\begin{pgfscope}%
\pgfsys@transformshift{2.592740in}{1.223780in}%
\pgfsys@useobject{currentmarker}{}%
\end{pgfscope}%
\end{pgfscope}%
\begin{pgfscope}%
\pgftext[x=0.226704in,y=1.223780in,right,]{\rmfamily\fontsize{8.000000}{9.600000}\selectfont 0.4}%
\end{pgfscope}%
\begin{pgfscope}%
\pgfsetbuttcap%
\pgfsetroundjoin%
\definecolor{currentfill}{rgb}{0.000000,0.000000,0.000000}%
\pgfsetfillcolor{currentfill}%
\pgfsetlinewidth{0.501875pt}%
\definecolor{currentstroke}{rgb}{0.000000,0.000000,0.000000}%
\pgfsetstrokecolor{currentstroke}%
\pgfsetdash{}{0pt}%
\pgfsys@defobject{currentmarker}{\pgfqpoint{0.000000in}{0.000000in}}{\pgfqpoint{0.069444in}{0.000000in}}{%
\pgfpathmoveto{\pgfqpoint{0.000000in}{0.000000in}}%
\pgfpathlineto{\pgfqpoint{0.069444in}{0.000000in}}%
\pgfusepath{stroke,fill}%
}%
\begin{pgfscope}%
\pgfsys@transformshift{0.296148in}{1.419371in}%
\pgfsys@useobject{currentmarker}{}%
\end{pgfscope}%
\end{pgfscope}%
\begin{pgfscope}%
\pgfsetbuttcap%
\pgfsetroundjoin%
\definecolor{currentfill}{rgb}{0.000000,0.000000,0.000000}%
\pgfsetfillcolor{currentfill}%
\pgfsetlinewidth{0.501875pt}%
\definecolor{currentstroke}{rgb}{0.000000,0.000000,0.000000}%
\pgfsetstrokecolor{currentstroke}%
\pgfsetdash{}{0pt}%
\pgfsys@defobject{currentmarker}{\pgfqpoint{-0.069444in}{0.000000in}}{\pgfqpoint{0.000000in}{0.000000in}}{%
\pgfpathmoveto{\pgfqpoint{0.000000in}{0.000000in}}%
\pgfpathlineto{\pgfqpoint{-0.069444in}{0.000000in}}%
\pgfusepath{stroke,fill}%
}%
\begin{pgfscope}%
\pgfsys@transformshift{2.592740in}{1.419371in}%
\pgfsys@useobject{currentmarker}{}%
\end{pgfscope}%
\end{pgfscope}%
\begin{pgfscope}%
\pgftext[x=0.226704in,y=1.419371in,right,]{\rmfamily\fontsize{8.000000}{9.600000}\selectfont 0.5}%
\end{pgfscope}%
\begin{pgfscope}%
\pgfsetbuttcap%
\pgfsetroundjoin%
\definecolor{currentfill}{rgb}{0.000000,0.000000,0.000000}%
\pgfsetfillcolor{currentfill}%
\pgfsetlinewidth{0.501875pt}%
\definecolor{currentstroke}{rgb}{0.000000,0.000000,0.000000}%
\pgfsetstrokecolor{currentstroke}%
\pgfsetdash{}{0pt}%
\pgfsys@defobject{currentmarker}{\pgfqpoint{0.000000in}{0.000000in}}{\pgfqpoint{0.069444in}{0.000000in}}{%
\pgfpathmoveto{\pgfqpoint{0.000000in}{0.000000in}}%
\pgfpathlineto{\pgfqpoint{0.069444in}{0.000000in}}%
\pgfusepath{stroke,fill}%
}%
\begin{pgfscope}%
\pgfsys@transformshift{0.296148in}{1.614961in}%
\pgfsys@useobject{currentmarker}{}%
\end{pgfscope}%
\end{pgfscope}%
\begin{pgfscope}%
\pgfsetbuttcap%
\pgfsetroundjoin%
\definecolor{currentfill}{rgb}{0.000000,0.000000,0.000000}%
\pgfsetfillcolor{currentfill}%
\pgfsetlinewidth{0.501875pt}%
\definecolor{currentstroke}{rgb}{0.000000,0.000000,0.000000}%
\pgfsetstrokecolor{currentstroke}%
\pgfsetdash{}{0pt}%
\pgfsys@defobject{currentmarker}{\pgfqpoint{-0.069444in}{0.000000in}}{\pgfqpoint{0.000000in}{0.000000in}}{%
\pgfpathmoveto{\pgfqpoint{0.000000in}{0.000000in}}%
\pgfpathlineto{\pgfqpoint{-0.069444in}{0.000000in}}%
\pgfusepath{stroke,fill}%
}%
\begin{pgfscope}%
\pgfsys@transformshift{2.592740in}{1.614961in}%
\pgfsys@useobject{currentmarker}{}%
\end{pgfscope}%
\end{pgfscope}%
\begin{pgfscope}%
\pgftext[x=0.226704in,y=1.614961in,right,]{\rmfamily\fontsize{8.000000}{9.600000}\selectfont 0.6}%
\end{pgfscope}%
\end{pgfpicture}%
\makeatother%
\endgroup%

	\end{subfigure}
  
	\begin{subfigure}[t]{0.49\textwidth}
		\centering
    %\includegraphics[width=\textwidth]{store/variables/DATA_MC_B_ISOLATION_BDT_Soft.pdf}
    \input{store/variables/DATA_MC_B_ISOLATION_BDT_Soft.pgf}
	\end{subfigure}
	\begin{subfigure}[t]{0.49\textwidth}
		\centering
    %\includegraphics[width=\textwidth]{store/variables/DATA_MC_B_TAU.pdf}
    %% Creator: Matplotlib, PGF backend
%%
%% To include the figure in your LaTeX document, write
%%   \input{<filename>.pgf}
%%
%% Make sure the required packages are loaded in your preamble
%%   \usepackage{pgf}
%%
%% Figures using additional raster images can only be included by \input if
%% they are in the same directory as the main LaTeX file. For loading figures
%% from other directories you can use the `import` package
%%   \usepackage{import}
%% and then include the figures with
%%   \import{<path to file>}{<filename>.pgf}
%%
%% Matplotlib used the following preamble
%%   \usepackage{fontspec}
%%   \setmainfont{DejaVu Serif}
%%   \setsansfont{DejaVu Sans}
%%   \setmonofont{DejaVu Sans Mono}
%%
\begingroup%
\makeatletter%
\begin{pgfpicture}%
\pgfpathrectangle{\pgfpointorigin}{\pgfqpoint{2.683883in}{1.741309in}}%
\pgfusepath{use as bounding box, clip}%
\begin{pgfscope}%
\pgfsetbuttcap%
\pgfsetmiterjoin%
\definecolor{currentfill}{rgb}{1.000000,1.000000,1.000000}%
\pgfsetfillcolor{currentfill}%
\pgfsetlinewidth{0.000000pt}%
\definecolor{currentstroke}{rgb}{1.000000,1.000000,1.000000}%
\pgfsetstrokecolor{currentstroke}%
\pgfsetdash{}{0pt}%
\pgfpathmoveto{\pgfqpoint{0.000000in}{0.000000in}}%
\pgfpathlineto{\pgfqpoint{2.683883in}{0.000000in}}%
\pgfpathlineto{\pgfqpoint{2.683883in}{1.741309in}}%
\pgfpathlineto{\pgfqpoint{0.000000in}{1.741309in}}%
\pgfpathclose%
\pgfusepath{fill}%
\end{pgfscope}%
\begin{pgfscope}%
\pgfsetbuttcap%
\pgfsetmiterjoin%
\definecolor{currentfill}{rgb}{1.000000,1.000000,1.000000}%
\pgfsetfillcolor{currentfill}%
\pgfsetlinewidth{0.000000pt}%
\definecolor{currentstroke}{rgb}{0.000000,0.000000,0.000000}%
\pgfsetstrokecolor{currentstroke}%
\pgfsetstrokeopacity{0.000000}%
\pgfsetdash{}{0pt}%
\pgfpathmoveto{\pgfqpoint{0.366840in}{0.417391in}}%
\pgfpathlineto{\pgfqpoint{2.563190in}{0.417391in}}%
\pgfpathlineto{\pgfqpoint{2.563190in}{1.637544in}}%
\pgfpathlineto{\pgfqpoint{0.366840in}{1.637544in}}%
\pgfpathclose%
\pgfusepath{fill}%
\end{pgfscope}%
\begin{pgfscope}%
\pgfpathrectangle{\pgfqpoint{0.366840in}{0.417391in}}{\pgfqpoint{2.196350in}{1.220153in}} %
\pgfusepath{clip}%
\pgfsetbuttcap%
\pgfsetmiterjoin%
\definecolor{currentfill}{rgb}{0.215686,0.470588,0.749020}%
\pgfsetfillcolor{currentfill}%
\pgfsetlinewidth{0.000000pt}%
\definecolor{currentstroke}{rgb}{0.000000,0.000000,0.000000}%
\pgfsetstrokecolor{currentstroke}%
\pgfsetdash{}{0pt}%
\pgfpathmoveto{\pgfqpoint{0.366840in}{0.417391in}}%
\pgfpathlineto{\pgfqpoint{0.366840in}{0.417643in}}%
\pgfpathlineto{\pgfqpoint{0.410767in}{0.417643in}}%
\pgfpathlineto{\pgfqpoint{0.410767in}{0.504071in}}%
\pgfpathlineto{\pgfqpoint{0.454694in}{0.504071in}}%
\pgfpathlineto{\pgfqpoint{0.454694in}{0.925919in}}%
\pgfpathlineto{\pgfqpoint{0.498621in}{0.925919in}}%
\pgfpathlineto{\pgfqpoint{0.498621in}{1.337314in}}%
\pgfpathlineto{\pgfqpoint{0.542548in}{1.337314in}}%
\pgfpathlineto{\pgfqpoint{0.542548in}{1.533480in}}%
\pgfpathlineto{\pgfqpoint{0.586475in}{1.533480in}}%
\pgfpathlineto{\pgfqpoint{0.586475in}{1.591442in}}%
\pgfpathlineto{\pgfqpoint{0.630402in}{1.591442in}}%
\pgfpathlineto{\pgfqpoint{0.630402in}{1.579504in}}%
\pgfpathlineto{\pgfqpoint{0.674329in}{1.579504in}}%
\pgfpathlineto{\pgfqpoint{0.674329in}{1.493477in}}%
\pgfpathlineto{\pgfqpoint{0.718256in}{1.493477in}}%
\pgfpathlineto{\pgfqpoint{0.718256in}{1.426316in}}%
\pgfpathlineto{\pgfqpoint{0.762183in}{1.426316in}}%
\pgfpathlineto{\pgfqpoint{0.762183in}{1.329781in}}%
\pgfpathlineto{\pgfqpoint{0.806110in}{1.329781in}}%
\pgfpathlineto{\pgfqpoint{0.806110in}{1.264074in}}%
\pgfpathlineto{\pgfqpoint{0.850037in}{1.264074in}}%
\pgfpathlineto{\pgfqpoint{0.850037in}{1.176642in}}%
\pgfpathlineto{\pgfqpoint{0.893964in}{1.176642in}}%
\pgfpathlineto{\pgfqpoint{0.893964in}{1.082195in}}%
\pgfpathlineto{\pgfqpoint{0.937891in}{1.082195in}}%
\pgfpathlineto{\pgfqpoint{0.937891in}{1.016310in}}%
\pgfpathlineto{\pgfqpoint{0.981818in}{1.016310in}}%
\pgfpathlineto{\pgfqpoint{0.981818in}{0.952766in}}%
\pgfpathlineto{\pgfqpoint{1.025745in}{0.952766in}}%
\pgfpathlineto{\pgfqpoint{1.025745in}{0.902095in}}%
\pgfpathlineto{\pgfqpoint{1.069672in}{0.902095in}}%
\pgfpathlineto{\pgfqpoint{1.069672in}{0.841896in}}%
\pgfpathlineto{\pgfqpoint{1.113599in}{0.841896in}}%
\pgfpathlineto{\pgfqpoint{1.113599in}{0.784115in}}%
\pgfpathlineto{\pgfqpoint{1.157526in}{0.784115in}}%
\pgfpathlineto{\pgfqpoint{1.157526in}{0.738083in}}%
\pgfpathlineto{\pgfqpoint{1.201453in}{0.738083in}}%
\pgfpathlineto{\pgfqpoint{1.201453in}{0.705430in}}%
\pgfpathlineto{\pgfqpoint{1.245380in}{0.705430in}}%
\pgfpathlineto{\pgfqpoint{1.245380in}{0.670605in}}%
\pgfpathlineto{\pgfqpoint{1.289307in}{0.670605in}}%
\pgfpathlineto{\pgfqpoint{1.289307in}{0.638023in}}%
\pgfpathlineto{\pgfqpoint{1.333234in}{0.638023in}}%
\pgfpathlineto{\pgfqpoint{1.333234in}{0.607420in}}%
\pgfpathlineto{\pgfqpoint{1.377161in}{0.607420in}}%
\pgfpathlineto{\pgfqpoint{1.377161in}{0.591077in}}%
\pgfpathlineto{\pgfqpoint{1.421088in}{0.591077in}}%
\pgfpathlineto{\pgfqpoint{1.421088in}{0.568228in}}%
\pgfpathlineto{\pgfqpoint{1.465015in}{0.568228in}}%
\pgfpathlineto{\pgfqpoint{1.465015in}{0.544456in}}%
\pgfpathlineto{\pgfqpoint{1.508942in}{0.544456in}}%
\pgfpathlineto{\pgfqpoint{1.508942in}{0.528188in}}%
\pgfpathlineto{\pgfqpoint{1.552869in}{0.528188in}}%
\pgfpathlineto{\pgfqpoint{1.552869in}{0.519380in}}%
\pgfpathlineto{\pgfqpoint{1.596796in}{0.519380in}}%
\pgfpathlineto{\pgfqpoint{1.596796in}{0.508284in}}%
\pgfpathlineto{\pgfqpoint{1.640723in}{0.508284in}}%
\pgfpathlineto{\pgfqpoint{1.640723in}{0.496888in}}%
\pgfpathlineto{\pgfqpoint{1.684650in}{0.496888in}}%
\pgfpathlineto{\pgfqpoint{1.684650in}{0.482249in}}%
\pgfpathlineto{\pgfqpoint{1.728577in}{0.482249in}}%
\pgfpathlineto{\pgfqpoint{1.728577in}{0.477051in}}%
\pgfpathlineto{\pgfqpoint{1.772504in}{0.477051in}}%
\pgfpathlineto{\pgfqpoint{1.772504in}{0.466787in}}%
\pgfpathlineto{\pgfqpoint{1.816431in}{0.466787in}}%
\pgfpathlineto{\pgfqpoint{1.816431in}{0.464930in}}%
\pgfpathlineto{\pgfqpoint{1.860358in}{0.464930in}}%
\pgfpathlineto{\pgfqpoint{1.860358in}{0.455803in}}%
\pgfpathlineto{\pgfqpoint{1.904285in}{0.455803in}}%
\pgfpathlineto{\pgfqpoint{1.904285in}{0.451594in}}%
\pgfpathlineto{\pgfqpoint{1.948212in}{0.451594in}}%
\pgfpathlineto{\pgfqpoint{1.948212in}{0.449105in}}%
\pgfpathlineto{\pgfqpoint{1.992139in}{0.449105in}}%
\pgfpathlineto{\pgfqpoint{1.992139in}{0.445646in}}%
\pgfpathlineto{\pgfqpoint{2.036066in}{0.445646in}}%
\pgfpathlineto{\pgfqpoint{2.036066in}{0.440347in}}%
\pgfpathlineto{\pgfqpoint{2.079993in}{0.440347in}}%
\pgfpathlineto{\pgfqpoint{2.079993in}{0.437649in}}%
\pgfpathlineto{\pgfqpoint{2.123920in}{0.437649in}}%
\pgfpathlineto{\pgfqpoint{2.123920in}{0.435312in}}%
\pgfpathlineto{\pgfqpoint{2.167847in}{0.435312in}}%
\pgfpathlineto{\pgfqpoint{2.167847in}{0.431999in}}%
\pgfpathlineto{\pgfqpoint{2.211774in}{0.431999in}}%
\pgfpathlineto{\pgfqpoint{2.211774in}{0.428878in}}%
\pgfpathlineto{\pgfqpoint{2.255701in}{0.428878in}}%
\pgfpathlineto{\pgfqpoint{2.255701in}{0.428442in}}%
\pgfpathlineto{\pgfqpoint{2.299628in}{0.428442in}}%
\pgfpathlineto{\pgfqpoint{2.299628in}{0.428069in}}%
\pgfpathlineto{\pgfqpoint{2.343555in}{0.428069in}}%
\pgfpathlineto{\pgfqpoint{2.343555in}{0.427075in}}%
\pgfpathlineto{\pgfqpoint{2.387482in}{0.427075in}}%
\pgfpathlineto{\pgfqpoint{2.387482in}{0.425082in}}%
\pgfpathlineto{\pgfqpoint{2.431409in}{0.425082in}}%
\pgfpathlineto{\pgfqpoint{2.431409in}{0.423956in}}%
\pgfpathlineto{\pgfqpoint{2.475336in}{0.423956in}}%
\pgfpathlineto{\pgfqpoint{2.475336in}{0.423540in}}%
\pgfpathlineto{\pgfqpoint{2.519263in}{0.423540in}}%
\pgfpathlineto{\pgfqpoint{2.519263in}{0.422847in}}%
\pgfpathlineto{\pgfqpoint{2.563190in}{0.422847in}}%
\pgfpathlineto{\pgfqpoint{2.563190in}{0.417391in}}%
\pgfpathlineto{\pgfqpoint{2.519263in}{0.417391in}}%
\pgfpathlineto{\pgfqpoint{2.519263in}{0.417391in}}%
\pgfpathlineto{\pgfqpoint{2.475336in}{0.417391in}}%
\pgfpathlineto{\pgfqpoint{2.475336in}{0.417391in}}%
\pgfpathlineto{\pgfqpoint{2.431409in}{0.417391in}}%
\pgfpathlineto{\pgfqpoint{2.431409in}{0.417391in}}%
\pgfpathlineto{\pgfqpoint{2.387482in}{0.417391in}}%
\pgfpathlineto{\pgfqpoint{2.387482in}{0.417391in}}%
\pgfpathlineto{\pgfqpoint{2.343555in}{0.417391in}}%
\pgfpathlineto{\pgfqpoint{2.343555in}{0.417391in}}%
\pgfpathlineto{\pgfqpoint{2.299628in}{0.417391in}}%
\pgfpathlineto{\pgfqpoint{2.299628in}{0.417391in}}%
\pgfpathlineto{\pgfqpoint{2.255701in}{0.417391in}}%
\pgfpathlineto{\pgfqpoint{2.255701in}{0.417391in}}%
\pgfpathlineto{\pgfqpoint{2.211774in}{0.417391in}}%
\pgfpathlineto{\pgfqpoint{2.211774in}{0.417391in}}%
\pgfpathlineto{\pgfqpoint{2.167847in}{0.417391in}}%
\pgfpathlineto{\pgfqpoint{2.167847in}{0.417391in}}%
\pgfpathlineto{\pgfqpoint{2.123920in}{0.417391in}}%
\pgfpathlineto{\pgfqpoint{2.123920in}{0.417391in}}%
\pgfpathlineto{\pgfqpoint{2.079993in}{0.417391in}}%
\pgfpathlineto{\pgfqpoint{2.079993in}{0.417391in}}%
\pgfpathlineto{\pgfqpoint{2.036066in}{0.417391in}}%
\pgfpathlineto{\pgfqpoint{2.036066in}{0.417391in}}%
\pgfpathlineto{\pgfqpoint{1.992139in}{0.417391in}}%
\pgfpathlineto{\pgfqpoint{1.992139in}{0.417391in}}%
\pgfpathlineto{\pgfqpoint{1.948212in}{0.417391in}}%
\pgfpathlineto{\pgfqpoint{1.948212in}{0.417391in}}%
\pgfpathlineto{\pgfqpoint{1.904285in}{0.417391in}}%
\pgfpathlineto{\pgfqpoint{1.904285in}{0.417391in}}%
\pgfpathlineto{\pgfqpoint{1.860358in}{0.417391in}}%
\pgfpathlineto{\pgfqpoint{1.860358in}{0.417391in}}%
\pgfpathlineto{\pgfqpoint{1.816431in}{0.417391in}}%
\pgfpathlineto{\pgfqpoint{1.816431in}{0.417391in}}%
\pgfpathlineto{\pgfqpoint{1.772504in}{0.417391in}}%
\pgfpathlineto{\pgfqpoint{1.772504in}{0.417391in}}%
\pgfpathlineto{\pgfqpoint{1.728577in}{0.417391in}}%
\pgfpathlineto{\pgfqpoint{1.728577in}{0.417391in}}%
\pgfpathlineto{\pgfqpoint{1.684650in}{0.417391in}}%
\pgfpathlineto{\pgfqpoint{1.684650in}{0.417391in}}%
\pgfpathlineto{\pgfqpoint{1.640723in}{0.417391in}}%
\pgfpathlineto{\pgfqpoint{1.640723in}{0.417391in}}%
\pgfpathlineto{\pgfqpoint{1.596796in}{0.417391in}}%
\pgfpathlineto{\pgfqpoint{1.596796in}{0.417391in}}%
\pgfpathlineto{\pgfqpoint{1.552869in}{0.417391in}}%
\pgfpathlineto{\pgfqpoint{1.552869in}{0.417391in}}%
\pgfpathlineto{\pgfqpoint{1.508942in}{0.417391in}}%
\pgfpathlineto{\pgfqpoint{1.508942in}{0.417391in}}%
\pgfpathlineto{\pgfqpoint{1.465015in}{0.417391in}}%
\pgfpathlineto{\pgfqpoint{1.465015in}{0.417391in}}%
\pgfpathlineto{\pgfqpoint{1.421088in}{0.417391in}}%
\pgfpathlineto{\pgfqpoint{1.421088in}{0.417391in}}%
\pgfpathlineto{\pgfqpoint{1.377161in}{0.417391in}}%
\pgfpathlineto{\pgfqpoint{1.377161in}{0.417391in}}%
\pgfpathlineto{\pgfqpoint{1.333234in}{0.417391in}}%
\pgfpathlineto{\pgfqpoint{1.333234in}{0.417391in}}%
\pgfpathlineto{\pgfqpoint{1.289307in}{0.417391in}}%
\pgfpathlineto{\pgfqpoint{1.289307in}{0.417391in}}%
\pgfpathlineto{\pgfqpoint{1.245380in}{0.417391in}}%
\pgfpathlineto{\pgfqpoint{1.245380in}{0.417391in}}%
\pgfpathlineto{\pgfqpoint{1.201453in}{0.417391in}}%
\pgfpathlineto{\pgfqpoint{1.201453in}{0.417391in}}%
\pgfpathlineto{\pgfqpoint{1.157526in}{0.417391in}}%
\pgfpathlineto{\pgfqpoint{1.157526in}{0.417391in}}%
\pgfpathlineto{\pgfqpoint{1.113599in}{0.417391in}}%
\pgfpathlineto{\pgfqpoint{1.113599in}{0.417391in}}%
\pgfpathlineto{\pgfqpoint{1.069672in}{0.417391in}}%
\pgfpathlineto{\pgfqpoint{1.069672in}{0.417391in}}%
\pgfpathlineto{\pgfqpoint{1.025745in}{0.417391in}}%
\pgfpathlineto{\pgfqpoint{1.025745in}{0.417391in}}%
\pgfpathlineto{\pgfqpoint{0.981818in}{0.417391in}}%
\pgfpathlineto{\pgfqpoint{0.981818in}{0.417391in}}%
\pgfpathlineto{\pgfqpoint{0.937891in}{0.417391in}}%
\pgfpathlineto{\pgfqpoint{0.937891in}{0.417391in}}%
\pgfpathlineto{\pgfqpoint{0.893964in}{0.417391in}}%
\pgfpathlineto{\pgfqpoint{0.893964in}{0.417391in}}%
\pgfpathlineto{\pgfqpoint{0.850037in}{0.417391in}}%
\pgfpathlineto{\pgfqpoint{0.850037in}{0.417391in}}%
\pgfpathlineto{\pgfqpoint{0.806110in}{0.417391in}}%
\pgfpathlineto{\pgfqpoint{0.806110in}{0.417391in}}%
\pgfpathlineto{\pgfqpoint{0.762183in}{0.417391in}}%
\pgfpathlineto{\pgfqpoint{0.762183in}{0.417391in}}%
\pgfpathlineto{\pgfqpoint{0.718256in}{0.417391in}}%
\pgfpathlineto{\pgfqpoint{0.718256in}{0.417391in}}%
\pgfpathlineto{\pgfqpoint{0.674329in}{0.417391in}}%
\pgfpathlineto{\pgfqpoint{0.674329in}{0.417391in}}%
\pgfpathlineto{\pgfqpoint{0.630402in}{0.417391in}}%
\pgfpathlineto{\pgfqpoint{0.630402in}{0.417391in}}%
\pgfpathlineto{\pgfqpoint{0.586475in}{0.417391in}}%
\pgfpathlineto{\pgfqpoint{0.586475in}{0.417391in}}%
\pgfpathlineto{\pgfqpoint{0.542548in}{0.417391in}}%
\pgfpathlineto{\pgfqpoint{0.542548in}{0.417391in}}%
\pgfpathlineto{\pgfqpoint{0.498621in}{0.417391in}}%
\pgfpathlineto{\pgfqpoint{0.498621in}{0.417391in}}%
\pgfpathlineto{\pgfqpoint{0.454694in}{0.417391in}}%
\pgfpathlineto{\pgfqpoint{0.454694in}{0.417391in}}%
\pgfpathlineto{\pgfqpoint{0.410767in}{0.417391in}}%
\pgfpathlineto{\pgfqpoint{0.410767in}{0.417391in}}%
\pgfpathlineto{\pgfqpoint{0.366840in}{0.417391in}}%
\pgfusepath{fill}%
\end{pgfscope}%
\begin{pgfscope}%
\pgfpathrectangle{\pgfqpoint{0.366840in}{0.417391in}}{\pgfqpoint{2.196350in}{1.220153in}} %
\pgfusepath{clip}%
\pgfsetbuttcap%
\pgfsetmiterjoin%
\pgfsetlinewidth{0.501875pt}%
\definecolor{currentstroke}{rgb}{1.000000,0.000000,0.000000}%
\pgfsetstrokecolor{currentstroke}%
\pgfsetdash{}{0pt}%
\pgfpathmoveto{\pgfqpoint{0.366840in}{0.417391in}}%
\pgfpathlineto{\pgfqpoint{0.366840in}{0.417506in}}%
\pgfpathlineto{\pgfqpoint{0.410767in}{0.417506in}}%
\pgfpathlineto{\pgfqpoint{0.410767in}{0.503643in}}%
\pgfpathlineto{\pgfqpoint{0.454694in}{0.503643in}}%
\pgfpathlineto{\pgfqpoint{0.454694in}{0.925696in}}%
\pgfpathlineto{\pgfqpoint{0.498621in}{0.925696in}}%
\pgfpathlineto{\pgfqpoint{0.498621in}{1.323588in}}%
\pgfpathlineto{\pgfqpoint{0.542548in}{1.323588in}}%
\pgfpathlineto{\pgfqpoint{0.542548in}{1.522745in}}%
\pgfpathlineto{\pgfqpoint{0.586475in}{1.522745in}}%
\pgfpathlineto{\pgfqpoint{0.586475in}{1.577396in}}%
\pgfpathlineto{\pgfqpoint{0.630402in}{1.577396in}}%
\pgfpathlineto{\pgfqpoint{0.630402in}{1.557530in}}%
\pgfpathlineto{\pgfqpoint{0.674329in}{1.557530in}}%
\pgfpathlineto{\pgfqpoint{0.674329in}{1.503915in}}%
\pgfpathlineto{\pgfqpoint{0.718256in}{1.503915in}}%
\pgfpathlineto{\pgfqpoint{0.718256in}{1.428018in}}%
\pgfpathlineto{\pgfqpoint{0.762183in}{1.428018in}}%
\pgfpathlineto{\pgfqpoint{0.762183in}{1.342304in}}%
\pgfpathlineto{\pgfqpoint{0.806110in}{1.342304in}}%
\pgfpathlineto{\pgfqpoint{0.806110in}{1.253789in}}%
\pgfpathlineto{\pgfqpoint{0.850037in}{1.253789in}}%
\pgfpathlineto{\pgfqpoint{0.850037in}{1.169110in}}%
\pgfpathlineto{\pgfqpoint{0.893964in}{1.169110in}}%
\pgfpathlineto{\pgfqpoint{0.893964in}{1.097125in}}%
\pgfpathlineto{\pgfqpoint{0.937891in}{1.097125in}}%
\pgfpathlineto{\pgfqpoint{0.937891in}{1.012638in}}%
\pgfpathlineto{\pgfqpoint{0.981818in}{1.012638in}}%
\pgfpathlineto{\pgfqpoint{0.981818in}{0.953692in}}%
\pgfpathlineto{\pgfqpoint{1.025745in}{0.953692in}}%
\pgfpathlineto{\pgfqpoint{1.025745in}{0.891448in}}%
\pgfpathlineto{\pgfqpoint{1.069672in}{0.891448in}}%
\pgfpathlineto{\pgfqpoint{1.069672in}{0.843625in}}%
\pgfpathlineto{\pgfqpoint{1.113599in}{0.843625in}}%
\pgfpathlineto{\pgfqpoint{1.113599in}{0.789243in}}%
\pgfpathlineto{\pgfqpoint{1.157526in}{0.789243in}}%
\pgfpathlineto{\pgfqpoint{1.157526in}{0.746903in}}%
\pgfpathlineto{\pgfqpoint{1.201453in}{0.746903in}}%
\pgfpathlineto{\pgfqpoint{1.201453in}{0.709012in}}%
\pgfpathlineto{\pgfqpoint{1.245380in}{0.709012in}}%
\pgfpathlineto{\pgfqpoint{1.245380in}{0.670393in}}%
\pgfpathlineto{\pgfqpoint{1.289307in}{0.670393in}}%
\pgfpathlineto{\pgfqpoint{1.289307in}{0.638408in}}%
\pgfpathlineto{\pgfqpoint{1.333234in}{0.638408in}}%
\pgfpathlineto{\pgfqpoint{1.333234in}{0.609760in}}%
\pgfpathlineto{\pgfqpoint{1.377161in}{0.609760in}}%
\pgfpathlineto{\pgfqpoint{1.377161in}{0.593729in}}%
\pgfpathlineto{\pgfqpoint{1.421088in}{0.593729in}}%
\pgfpathlineto{\pgfqpoint{1.421088in}{0.571064in}}%
\pgfpathlineto{\pgfqpoint{1.465015in}{0.571064in}}%
\pgfpathlineto{\pgfqpoint{1.465015in}{0.552540in}}%
\pgfpathlineto{\pgfqpoint{1.508942in}{0.552540in}}%
\pgfpathlineto{\pgfqpoint{1.508942in}{0.533326in}}%
\pgfpathlineto{\pgfqpoint{1.552869in}{0.533326in}}%
\pgfpathlineto{\pgfqpoint{1.552869in}{0.521054in}}%
\pgfpathlineto{\pgfqpoint{1.596796in}{0.521054in}}%
\pgfpathlineto{\pgfqpoint{1.596796in}{0.509280in}}%
\pgfpathlineto{\pgfqpoint{1.640723in}{0.509280in}}%
\pgfpathlineto{\pgfqpoint{1.640723in}{0.493978in}}%
\pgfpathlineto{\pgfqpoint{1.684650in}{0.493978in}}%
\pgfpathlineto{\pgfqpoint{1.684650in}{0.484390in}}%
\pgfpathlineto{\pgfqpoint{1.728577in}{0.484390in}}%
\pgfpathlineto{\pgfqpoint{1.728577in}{0.478906in}}%
\pgfpathlineto{\pgfqpoint{1.772504in}{0.478906in}}%
\pgfpathlineto{\pgfqpoint{1.772504in}{0.469970in}}%
\pgfpathlineto{\pgfqpoint{1.816431in}{0.469970in}}%
\pgfpathlineto{\pgfqpoint{1.816431in}{0.461456in}}%
\pgfpathlineto{\pgfqpoint{1.860358in}{0.461456in}}%
\pgfpathlineto{\pgfqpoint{1.860358in}{0.457583in}}%
\pgfpathlineto{\pgfqpoint{1.904285in}{0.457583in}}%
\pgfpathlineto{\pgfqpoint{1.904285in}{0.453326in}}%
\pgfpathlineto{\pgfqpoint{1.948212in}{0.453326in}}%
\pgfpathlineto{\pgfqpoint{1.948212in}{0.450718in}}%
\pgfpathlineto{\pgfqpoint{1.992139in}{0.450718in}}%
\pgfpathlineto{\pgfqpoint{1.992139in}{0.444390in}}%
\pgfpathlineto{\pgfqpoint{2.036066in}{0.444390in}}%
\pgfpathlineto{\pgfqpoint{2.036066in}{0.441399in}}%
\pgfpathlineto{\pgfqpoint{2.079993in}{0.441399in}}%
\pgfpathlineto{\pgfqpoint{2.079993in}{0.437794in}}%
\pgfpathlineto{\pgfqpoint{2.123920in}{0.437794in}}%
\pgfpathlineto{\pgfqpoint{2.123920in}{0.434496in}}%
\pgfpathlineto{\pgfqpoint{2.167847in}{0.434496in}}%
\pgfpathlineto{\pgfqpoint{2.167847in}{0.433690in}}%
\pgfpathlineto{\pgfqpoint{2.211774in}{0.433690in}}%
\pgfpathlineto{\pgfqpoint{2.211774in}{0.431351in}}%
\pgfpathlineto{\pgfqpoint{2.255701in}{0.431351in}}%
\pgfpathlineto{\pgfqpoint{2.255701in}{0.429702in}}%
\pgfpathlineto{\pgfqpoint{2.299628in}{0.429702in}}%
\pgfpathlineto{\pgfqpoint{2.299628in}{0.428129in}}%
\pgfpathlineto{\pgfqpoint{2.343555in}{0.428129in}}%
\pgfpathlineto{\pgfqpoint{2.343555in}{0.427094in}}%
\pgfpathlineto{\pgfqpoint{2.387482in}{0.427094in}}%
\pgfpathlineto{\pgfqpoint{2.387482in}{0.425291in}}%
\pgfpathlineto{\pgfqpoint{2.431409in}{0.425291in}}%
\pgfpathlineto{\pgfqpoint{2.431409in}{0.423412in}}%
\pgfpathlineto{\pgfqpoint{2.475336in}{0.423412in}}%
\pgfpathlineto{\pgfqpoint{2.475336in}{0.423796in}}%
\pgfpathlineto{\pgfqpoint{2.519263in}{0.423796in}}%
\pgfpathlineto{\pgfqpoint{2.519263in}{0.422108in}}%
\pgfpathlineto{\pgfqpoint{2.563190in}{0.422108in}}%
\pgfpathlineto{\pgfqpoint{2.563190in}{0.417391in}}%
\pgfusepath{stroke}%
\end{pgfscope}%
\begin{pgfscope}%
\pgfpathrectangle{\pgfqpoint{0.366840in}{0.417391in}}{\pgfqpoint{2.196350in}{1.220153in}} %
\pgfusepath{clip}%
\pgfsetbuttcap%
\pgfsetmiterjoin%
\pgfsetlinewidth{0.501875pt}%
\definecolor{currentstroke}{rgb}{1.000000,0.647059,0.000000}%
\pgfsetstrokecolor{currentstroke}%
\pgfsetdash{}{0pt}%
\pgfpathmoveto{\pgfqpoint{0.366840in}{0.417391in}}%
\pgfpathlineto{\pgfqpoint{0.366840in}{0.417483in}}%
\pgfpathlineto{\pgfqpoint{0.410767in}{0.417483in}}%
\pgfpathlineto{\pgfqpoint{0.410767in}{0.500422in}}%
\pgfpathlineto{\pgfqpoint{0.454694in}{0.500422in}}%
\pgfpathlineto{\pgfqpoint{0.454694in}{0.910772in}}%
\pgfpathlineto{\pgfqpoint{0.498621in}{0.910772in}}%
\pgfpathlineto{\pgfqpoint{0.498621in}{1.319019in}}%
\pgfpathlineto{\pgfqpoint{0.542548in}{1.319019in}}%
\pgfpathlineto{\pgfqpoint{0.542548in}{1.540189in}}%
\pgfpathlineto{\pgfqpoint{0.586475in}{1.540189in}}%
\pgfpathlineto{\pgfqpoint{0.586475in}{1.571673in}}%
\pgfpathlineto{\pgfqpoint{0.630402in}{1.571673in}}%
\pgfpathlineto{\pgfqpoint{0.630402in}{1.552644in}}%
\pgfpathlineto{\pgfqpoint{0.674329in}{1.552644in}}%
\pgfpathlineto{\pgfqpoint{0.674329in}{1.500264in}}%
\pgfpathlineto{\pgfqpoint{0.718256in}{1.500264in}}%
\pgfpathlineto{\pgfqpoint{0.718256in}{1.421417in}}%
\pgfpathlineto{\pgfqpoint{0.762183in}{1.421417in}}%
\pgfpathlineto{\pgfqpoint{0.762183in}{1.338593in}}%
\pgfpathlineto{\pgfqpoint{0.806110in}{1.338593in}}%
\pgfpathlineto{\pgfqpoint{0.806110in}{1.252999in}}%
\pgfpathlineto{\pgfqpoint{0.850037in}{1.252999in}}%
\pgfpathlineto{\pgfqpoint{0.850037in}{1.151058in}}%
\pgfpathlineto{\pgfqpoint{0.893964in}{1.151058in}}%
\pgfpathlineto{\pgfqpoint{0.893964in}{1.100769in}}%
\pgfpathlineto{\pgfqpoint{0.937891in}{1.100769in}}%
\pgfpathlineto{\pgfqpoint{0.937891in}{1.023478in}}%
\pgfpathlineto{\pgfqpoint{0.981818in}{1.023478in}}%
\pgfpathlineto{\pgfqpoint{0.981818in}{0.949232in}}%
\pgfpathlineto{\pgfqpoint{1.025745in}{0.949232in}}%
\pgfpathlineto{\pgfqpoint{1.025745in}{0.903440in}}%
\pgfpathlineto{\pgfqpoint{1.069672in}{0.903440in}}%
\pgfpathlineto{\pgfqpoint{1.069672in}{0.847706in}}%
\pgfpathlineto{\pgfqpoint{1.113599in}{0.847706in}}%
\pgfpathlineto{\pgfqpoint{1.113599in}{0.787402in}}%
\pgfpathlineto{\pgfqpoint{1.157526in}{0.787402in}}%
\pgfpathlineto{\pgfqpoint{1.157526in}{0.748188in}}%
\pgfpathlineto{\pgfqpoint{1.201453in}{0.748188in}}%
\pgfpathlineto{\pgfqpoint{1.201453in}{0.716605in}}%
\pgfpathlineto{\pgfqpoint{1.245380in}{0.716605in}}%
\pgfpathlineto{\pgfqpoint{1.245380in}{0.665600in}}%
\pgfpathlineto{\pgfqpoint{1.289307in}{0.665600in}}%
\pgfpathlineto{\pgfqpoint{1.289307in}{0.634633in}}%
\pgfpathlineto{\pgfqpoint{1.333234in}{0.634633in}}%
\pgfpathlineto{\pgfqpoint{1.333234in}{0.617876in}}%
\pgfpathlineto{\pgfqpoint{1.377161in}{0.617876in}}%
\pgfpathlineto{\pgfqpoint{1.377161in}{0.596320in}}%
\pgfpathlineto{\pgfqpoint{1.421088in}{0.596320in}}%
\pgfpathlineto{\pgfqpoint{1.421088in}{0.573065in}}%
\pgfpathlineto{\pgfqpoint{1.465015in}{0.573065in}}%
\pgfpathlineto{\pgfqpoint{1.465015in}{0.552928in}}%
\pgfpathlineto{\pgfqpoint{1.508942in}{0.552928in}}%
\pgfpathlineto{\pgfqpoint{1.508942in}{0.531457in}}%
\pgfpathlineto{\pgfqpoint{1.552869in}{0.531457in}}%
\pgfpathlineto{\pgfqpoint{1.552869in}{0.523748in}}%
\pgfpathlineto{\pgfqpoint{1.596796in}{0.523748in}}%
\pgfpathlineto{\pgfqpoint{1.596796in}{0.507849in}}%
\pgfpathlineto{\pgfqpoint{1.640723in}{0.507849in}}%
\pgfpathlineto{\pgfqpoint{1.640723in}{0.498759in}}%
\pgfpathlineto{\pgfqpoint{1.684650in}{0.498759in}}%
\pgfpathlineto{\pgfqpoint{1.684650in}{0.487282in}}%
\pgfpathlineto{\pgfqpoint{1.728577in}{0.487282in}}%
\pgfpathlineto{\pgfqpoint{1.728577in}{0.479598in}}%
\pgfpathlineto{\pgfqpoint{1.772504in}{0.479598in}}%
\pgfpathlineto{\pgfqpoint{1.772504in}{0.475282in}}%
\pgfpathlineto{\pgfqpoint{1.816431in}{0.475282in}}%
\pgfpathlineto{\pgfqpoint{1.816431in}{0.461732in}}%
\pgfpathlineto{\pgfqpoint{1.860358in}{0.461732in}}%
\pgfpathlineto{\pgfqpoint{1.860358in}{0.454032in}}%
\pgfpathlineto{\pgfqpoint{1.904285in}{0.454032in}}%
\pgfpathlineto{\pgfqpoint{1.904285in}{0.454830in}}%
\pgfpathlineto{\pgfqpoint{1.948212in}{0.454830in}}%
\pgfpathlineto{\pgfqpoint{1.948212in}{0.446766in}}%
\pgfpathlineto{\pgfqpoint{1.992139in}{0.446766in}}%
\pgfpathlineto{\pgfqpoint{1.992139in}{0.445217in}}%
\pgfpathlineto{\pgfqpoint{2.036066in}{0.445217in}}%
\pgfpathlineto{\pgfqpoint{2.036066in}{0.440742in}}%
\pgfpathlineto{\pgfqpoint{2.079993in}{0.440742in}}%
\pgfpathlineto{\pgfqpoint{2.079993in}{0.437830in}}%
\pgfpathlineto{\pgfqpoint{2.123920in}{0.437830in}}%
\pgfpathlineto{\pgfqpoint{2.123920in}{0.434646in}}%
\pgfpathlineto{\pgfqpoint{2.167847in}{0.434646in}}%
\pgfpathlineto{\pgfqpoint{2.167847in}{0.431522in}}%
\pgfpathlineto{\pgfqpoint{2.211774in}{0.431522in}}%
\pgfpathlineto{\pgfqpoint{2.211774in}{0.431539in}}%
\pgfpathlineto{\pgfqpoint{2.255701in}{0.431539in}}%
\pgfpathlineto{\pgfqpoint{2.255701in}{0.432018in}}%
\pgfpathlineto{\pgfqpoint{2.299628in}{0.432018in}}%
\pgfpathlineto{\pgfqpoint{2.299628in}{0.429365in}}%
\pgfpathlineto{\pgfqpoint{2.343555in}{0.429365in}}%
\pgfpathlineto{\pgfqpoint{2.343555in}{0.426845in}}%
\pgfpathlineto{\pgfqpoint{2.387482in}{0.426845in}}%
\pgfpathlineto{\pgfqpoint{2.387482in}{0.427043in}}%
\pgfpathlineto{\pgfqpoint{2.431409in}{0.427043in}}%
\pgfpathlineto{\pgfqpoint{2.431409in}{0.424444in}}%
\pgfpathlineto{\pgfqpoint{2.475336in}{0.424444in}}%
\pgfpathlineto{\pgfqpoint{2.475336in}{0.423143in}}%
\pgfpathlineto{\pgfqpoint{2.519263in}{0.423143in}}%
\pgfpathlineto{\pgfqpoint{2.519263in}{0.421996in}}%
\pgfpathlineto{\pgfqpoint{2.563190in}{0.421996in}}%
\pgfpathlineto{\pgfqpoint{2.563190in}{0.417391in}}%
\pgfusepath{stroke}%
\end{pgfscope}%
\begin{pgfscope}%
\pgfsetrectcap%
\pgfsetmiterjoin%
\pgfsetlinewidth{1.003750pt}%
\definecolor{currentstroke}{rgb}{0.000000,0.000000,0.000000}%
\pgfsetstrokecolor{currentstroke}%
\pgfsetdash{}{0pt}%
\pgfpathmoveto{\pgfqpoint{0.366840in}{1.637544in}}%
\pgfpathlineto{\pgfqpoint{2.563190in}{1.637544in}}%
\pgfusepath{stroke}%
\end{pgfscope}%
\begin{pgfscope}%
\pgfsetrectcap%
\pgfsetmiterjoin%
\pgfsetlinewidth{1.003750pt}%
\definecolor{currentstroke}{rgb}{0.000000,0.000000,0.000000}%
\pgfsetstrokecolor{currentstroke}%
\pgfsetdash{}{0pt}%
\pgfpathmoveto{\pgfqpoint{2.563190in}{0.417391in}}%
\pgfpathlineto{\pgfqpoint{2.563190in}{1.637544in}}%
\pgfusepath{stroke}%
\end{pgfscope}%
\begin{pgfscope}%
\pgfsetrectcap%
\pgfsetmiterjoin%
\pgfsetlinewidth{1.003750pt}%
\definecolor{currentstroke}{rgb}{0.000000,0.000000,0.000000}%
\pgfsetstrokecolor{currentstroke}%
\pgfsetdash{}{0pt}%
\pgfpathmoveto{\pgfqpoint{0.366840in}{0.417391in}}%
\pgfpathlineto{\pgfqpoint{2.563190in}{0.417391in}}%
\pgfusepath{stroke}%
\end{pgfscope}%
\begin{pgfscope}%
\pgfsetrectcap%
\pgfsetmiterjoin%
\pgfsetlinewidth{1.003750pt}%
\definecolor{currentstroke}{rgb}{0.000000,0.000000,0.000000}%
\pgfsetstrokecolor{currentstroke}%
\pgfsetdash{}{0pt}%
\pgfpathmoveto{\pgfqpoint{0.366840in}{0.417391in}}%
\pgfpathlineto{\pgfqpoint{0.366840in}{1.637544in}}%
\pgfusepath{stroke}%
\end{pgfscope}%
\begin{pgfscope}%
\pgfsetbuttcap%
\pgfsetroundjoin%
\definecolor{currentfill}{rgb}{0.000000,0.000000,0.000000}%
\pgfsetfillcolor{currentfill}%
\pgfsetlinewidth{0.501875pt}%
\definecolor{currentstroke}{rgb}{0.000000,0.000000,0.000000}%
\pgfsetstrokecolor{currentstroke}%
\pgfsetdash{}{0pt}%
\pgfsys@defobject{currentmarker}{\pgfqpoint{0.000000in}{0.000000in}}{\pgfqpoint{0.000000in}{0.069444in}}{%
\pgfpathmoveto{\pgfqpoint{0.000000in}{0.000000in}}%
\pgfpathlineto{\pgfqpoint{0.000000in}{0.069444in}}%
\pgfusepath{stroke,fill}%
}%
\begin{pgfscope}%
\pgfsys@transformshift{0.366840in}{0.417391in}%
\pgfsys@useobject{currentmarker}{}%
\end{pgfscope}%
\end{pgfscope}%
\begin{pgfscope}%
\pgfsetbuttcap%
\pgfsetroundjoin%
\definecolor{currentfill}{rgb}{0.000000,0.000000,0.000000}%
\pgfsetfillcolor{currentfill}%
\pgfsetlinewidth{0.501875pt}%
\definecolor{currentstroke}{rgb}{0.000000,0.000000,0.000000}%
\pgfsetstrokecolor{currentstroke}%
\pgfsetdash{}{0pt}%
\pgfsys@defobject{currentmarker}{\pgfqpoint{0.000000in}{-0.069444in}}{\pgfqpoint{0.000000in}{0.000000in}}{%
\pgfpathmoveto{\pgfqpoint{0.000000in}{0.000000in}}%
\pgfpathlineto{\pgfqpoint{0.000000in}{-0.069444in}}%
\pgfusepath{stroke,fill}%
}%
\begin{pgfscope}%
\pgfsys@transformshift{0.366840in}{1.637544in}%
\pgfsys@useobject{currentmarker}{}%
\end{pgfscope}%
\end{pgfscope}%
\begin{pgfscope}%
\pgftext[x=0.366840in,y=0.347947in,,top]{\rmfamily\fontsize{8.000000}{9.600000}\selectfont 0}%
\end{pgfscope}%
\begin{pgfscope}%
\pgfsetbuttcap%
\pgfsetroundjoin%
\definecolor{currentfill}{rgb}{0.000000,0.000000,0.000000}%
\pgfsetfillcolor{currentfill}%
\pgfsetlinewidth{0.501875pt}%
\definecolor{currentstroke}{rgb}{0.000000,0.000000,0.000000}%
\pgfsetstrokecolor{currentstroke}%
\pgfsetdash{}{0pt}%
\pgfsys@defobject{currentmarker}{\pgfqpoint{0.000000in}{0.000000in}}{\pgfqpoint{0.000000in}{0.069444in}}{%
\pgfpathmoveto{\pgfqpoint{0.000000in}{0.000000in}}%
\pgfpathlineto{\pgfqpoint{0.000000in}{0.069444in}}%
\pgfusepath{stroke,fill}%
}%
\begin{pgfscope}%
\pgfsys@transformshift{0.806110in}{0.417391in}%
\pgfsys@useobject{currentmarker}{}%
\end{pgfscope}%
\end{pgfscope}%
\begin{pgfscope}%
\pgfsetbuttcap%
\pgfsetroundjoin%
\definecolor{currentfill}{rgb}{0.000000,0.000000,0.000000}%
\pgfsetfillcolor{currentfill}%
\pgfsetlinewidth{0.501875pt}%
\definecolor{currentstroke}{rgb}{0.000000,0.000000,0.000000}%
\pgfsetstrokecolor{currentstroke}%
\pgfsetdash{}{0pt}%
\pgfsys@defobject{currentmarker}{\pgfqpoint{0.000000in}{-0.069444in}}{\pgfqpoint{0.000000in}{0.000000in}}{%
\pgfpathmoveto{\pgfqpoint{0.000000in}{0.000000in}}%
\pgfpathlineto{\pgfqpoint{0.000000in}{-0.069444in}}%
\pgfusepath{stroke,fill}%
}%
\begin{pgfscope}%
\pgfsys@transformshift{0.806110in}{1.637544in}%
\pgfsys@useobject{currentmarker}{}%
\end{pgfscope}%
\end{pgfscope}%
\begin{pgfscope}%
\pgftext[x=0.806110in,y=0.347947in,,top]{\rmfamily\fontsize{8.000000}{9.600000}\selectfont 2}%
\end{pgfscope}%
\begin{pgfscope}%
\pgfsetbuttcap%
\pgfsetroundjoin%
\definecolor{currentfill}{rgb}{0.000000,0.000000,0.000000}%
\pgfsetfillcolor{currentfill}%
\pgfsetlinewidth{0.501875pt}%
\definecolor{currentstroke}{rgb}{0.000000,0.000000,0.000000}%
\pgfsetstrokecolor{currentstroke}%
\pgfsetdash{}{0pt}%
\pgfsys@defobject{currentmarker}{\pgfqpoint{0.000000in}{0.000000in}}{\pgfqpoint{0.000000in}{0.069444in}}{%
\pgfpathmoveto{\pgfqpoint{0.000000in}{0.000000in}}%
\pgfpathlineto{\pgfqpoint{0.000000in}{0.069444in}}%
\pgfusepath{stroke,fill}%
}%
\begin{pgfscope}%
\pgfsys@transformshift{1.245380in}{0.417391in}%
\pgfsys@useobject{currentmarker}{}%
\end{pgfscope}%
\end{pgfscope}%
\begin{pgfscope}%
\pgfsetbuttcap%
\pgfsetroundjoin%
\definecolor{currentfill}{rgb}{0.000000,0.000000,0.000000}%
\pgfsetfillcolor{currentfill}%
\pgfsetlinewidth{0.501875pt}%
\definecolor{currentstroke}{rgb}{0.000000,0.000000,0.000000}%
\pgfsetstrokecolor{currentstroke}%
\pgfsetdash{}{0pt}%
\pgfsys@defobject{currentmarker}{\pgfqpoint{0.000000in}{-0.069444in}}{\pgfqpoint{0.000000in}{0.000000in}}{%
\pgfpathmoveto{\pgfqpoint{0.000000in}{0.000000in}}%
\pgfpathlineto{\pgfqpoint{0.000000in}{-0.069444in}}%
\pgfusepath{stroke,fill}%
}%
\begin{pgfscope}%
\pgfsys@transformshift{1.245380in}{1.637544in}%
\pgfsys@useobject{currentmarker}{}%
\end{pgfscope}%
\end{pgfscope}%
\begin{pgfscope}%
\pgftext[x=1.245380in,y=0.347947in,,top]{\rmfamily\fontsize{8.000000}{9.600000}\selectfont 4}%
\end{pgfscope}%
\begin{pgfscope}%
\pgfsetbuttcap%
\pgfsetroundjoin%
\definecolor{currentfill}{rgb}{0.000000,0.000000,0.000000}%
\pgfsetfillcolor{currentfill}%
\pgfsetlinewidth{0.501875pt}%
\definecolor{currentstroke}{rgb}{0.000000,0.000000,0.000000}%
\pgfsetstrokecolor{currentstroke}%
\pgfsetdash{}{0pt}%
\pgfsys@defobject{currentmarker}{\pgfqpoint{0.000000in}{0.000000in}}{\pgfqpoint{0.000000in}{0.069444in}}{%
\pgfpathmoveto{\pgfqpoint{0.000000in}{0.000000in}}%
\pgfpathlineto{\pgfqpoint{0.000000in}{0.069444in}}%
\pgfusepath{stroke,fill}%
}%
\begin{pgfscope}%
\pgfsys@transformshift{1.684650in}{0.417391in}%
\pgfsys@useobject{currentmarker}{}%
\end{pgfscope}%
\end{pgfscope}%
\begin{pgfscope}%
\pgfsetbuttcap%
\pgfsetroundjoin%
\definecolor{currentfill}{rgb}{0.000000,0.000000,0.000000}%
\pgfsetfillcolor{currentfill}%
\pgfsetlinewidth{0.501875pt}%
\definecolor{currentstroke}{rgb}{0.000000,0.000000,0.000000}%
\pgfsetstrokecolor{currentstroke}%
\pgfsetdash{}{0pt}%
\pgfsys@defobject{currentmarker}{\pgfqpoint{0.000000in}{-0.069444in}}{\pgfqpoint{0.000000in}{0.000000in}}{%
\pgfpathmoveto{\pgfqpoint{0.000000in}{0.000000in}}%
\pgfpathlineto{\pgfqpoint{0.000000in}{-0.069444in}}%
\pgfusepath{stroke,fill}%
}%
\begin{pgfscope}%
\pgfsys@transformshift{1.684650in}{1.637544in}%
\pgfsys@useobject{currentmarker}{}%
\end{pgfscope}%
\end{pgfscope}%
\begin{pgfscope}%
\pgftext[x=1.684650in,y=0.347947in,,top]{\rmfamily\fontsize{8.000000}{9.600000}\selectfont 6}%
\end{pgfscope}%
\begin{pgfscope}%
\pgfsetbuttcap%
\pgfsetroundjoin%
\definecolor{currentfill}{rgb}{0.000000,0.000000,0.000000}%
\pgfsetfillcolor{currentfill}%
\pgfsetlinewidth{0.501875pt}%
\definecolor{currentstroke}{rgb}{0.000000,0.000000,0.000000}%
\pgfsetstrokecolor{currentstroke}%
\pgfsetdash{}{0pt}%
\pgfsys@defobject{currentmarker}{\pgfqpoint{0.000000in}{0.000000in}}{\pgfqpoint{0.000000in}{0.069444in}}{%
\pgfpathmoveto{\pgfqpoint{0.000000in}{0.000000in}}%
\pgfpathlineto{\pgfqpoint{0.000000in}{0.069444in}}%
\pgfusepath{stroke,fill}%
}%
\begin{pgfscope}%
\pgfsys@transformshift{2.123920in}{0.417391in}%
\pgfsys@useobject{currentmarker}{}%
\end{pgfscope}%
\end{pgfscope}%
\begin{pgfscope}%
\pgfsetbuttcap%
\pgfsetroundjoin%
\definecolor{currentfill}{rgb}{0.000000,0.000000,0.000000}%
\pgfsetfillcolor{currentfill}%
\pgfsetlinewidth{0.501875pt}%
\definecolor{currentstroke}{rgb}{0.000000,0.000000,0.000000}%
\pgfsetstrokecolor{currentstroke}%
\pgfsetdash{}{0pt}%
\pgfsys@defobject{currentmarker}{\pgfqpoint{0.000000in}{-0.069444in}}{\pgfqpoint{0.000000in}{0.000000in}}{%
\pgfpathmoveto{\pgfqpoint{0.000000in}{0.000000in}}%
\pgfpathlineto{\pgfqpoint{0.000000in}{-0.069444in}}%
\pgfusepath{stroke,fill}%
}%
\begin{pgfscope}%
\pgfsys@transformshift{2.123920in}{1.637544in}%
\pgfsys@useobject{currentmarker}{}%
\end{pgfscope}%
\end{pgfscope}%
\begin{pgfscope}%
\pgftext[x=2.123920in,y=0.347947in,,top]{\rmfamily\fontsize{8.000000}{9.600000}\selectfont 8}%
\end{pgfscope}%
\begin{pgfscope}%
\pgfsetbuttcap%
\pgfsetroundjoin%
\definecolor{currentfill}{rgb}{0.000000,0.000000,0.000000}%
\pgfsetfillcolor{currentfill}%
\pgfsetlinewidth{0.501875pt}%
\definecolor{currentstroke}{rgb}{0.000000,0.000000,0.000000}%
\pgfsetstrokecolor{currentstroke}%
\pgfsetdash{}{0pt}%
\pgfsys@defobject{currentmarker}{\pgfqpoint{0.000000in}{0.000000in}}{\pgfqpoint{0.000000in}{0.069444in}}{%
\pgfpathmoveto{\pgfqpoint{0.000000in}{0.000000in}}%
\pgfpathlineto{\pgfqpoint{0.000000in}{0.069444in}}%
\pgfusepath{stroke,fill}%
}%
\begin{pgfscope}%
\pgfsys@transformshift{2.563190in}{0.417391in}%
\pgfsys@useobject{currentmarker}{}%
\end{pgfscope}%
\end{pgfscope}%
\begin{pgfscope}%
\pgfsetbuttcap%
\pgfsetroundjoin%
\definecolor{currentfill}{rgb}{0.000000,0.000000,0.000000}%
\pgfsetfillcolor{currentfill}%
\pgfsetlinewidth{0.501875pt}%
\definecolor{currentstroke}{rgb}{0.000000,0.000000,0.000000}%
\pgfsetstrokecolor{currentstroke}%
\pgfsetdash{}{0pt}%
\pgfsys@defobject{currentmarker}{\pgfqpoint{0.000000in}{-0.069444in}}{\pgfqpoint{0.000000in}{0.000000in}}{%
\pgfpathmoveto{\pgfqpoint{0.000000in}{0.000000in}}%
\pgfpathlineto{\pgfqpoint{0.000000in}{-0.069444in}}%
\pgfusepath{stroke,fill}%
}%
\begin{pgfscope}%
\pgfsys@transformshift{2.563190in}{1.637544in}%
\pgfsys@useobject{currentmarker}{}%
\end{pgfscope}%
\end{pgfscope}%
\begin{pgfscope}%
\pgftext[x=2.563190in,y=0.347947in,,top]{\rmfamily\fontsize{8.000000}{9.600000}\selectfont 10}%
\end{pgfscope}%
\begin{pgfscope}%
\pgftext[x=1.465015in,y=0.170972in,,top]{\rmfamily\fontsize{9.000000}{10.800000}\selectfont \(\displaystyle t_{B^0}\)}%
\end{pgfscope}%
\begin{pgfscope}%
\pgfsetbuttcap%
\pgfsetroundjoin%
\definecolor{currentfill}{rgb}{0.000000,0.000000,0.000000}%
\pgfsetfillcolor{currentfill}%
\pgfsetlinewidth{0.501875pt}%
\definecolor{currentstroke}{rgb}{0.000000,0.000000,0.000000}%
\pgfsetstrokecolor{currentstroke}%
\pgfsetdash{}{0pt}%
\pgfsys@defobject{currentmarker}{\pgfqpoint{0.000000in}{0.000000in}}{\pgfqpoint{0.069444in}{0.000000in}}{%
\pgfpathmoveto{\pgfqpoint{0.000000in}{0.000000in}}%
\pgfpathlineto{\pgfqpoint{0.069444in}{0.000000in}}%
\pgfusepath{stroke,fill}%
}%
\begin{pgfscope}%
\pgfsys@transformshift{0.366840in}{0.417391in}%
\pgfsys@useobject{currentmarker}{}%
\end{pgfscope}%
\end{pgfscope}%
\begin{pgfscope}%
\pgfsetbuttcap%
\pgfsetroundjoin%
\definecolor{currentfill}{rgb}{0.000000,0.000000,0.000000}%
\pgfsetfillcolor{currentfill}%
\pgfsetlinewidth{0.501875pt}%
\definecolor{currentstroke}{rgb}{0.000000,0.000000,0.000000}%
\pgfsetstrokecolor{currentstroke}%
\pgfsetdash{}{0pt}%
\pgfsys@defobject{currentmarker}{\pgfqpoint{-0.069444in}{0.000000in}}{\pgfqpoint{0.000000in}{0.000000in}}{%
\pgfpathmoveto{\pgfqpoint{0.000000in}{0.000000in}}%
\pgfpathlineto{\pgfqpoint{-0.069444in}{0.000000in}}%
\pgfusepath{stroke,fill}%
}%
\begin{pgfscope}%
\pgfsys@transformshift{2.563190in}{0.417391in}%
\pgfsys@useobject{currentmarker}{}%
\end{pgfscope}%
\end{pgfscope}%
\begin{pgfscope}%
\pgftext[x=0.297396in,y=0.417391in,right,]{\rmfamily\fontsize{8.000000}{9.600000}\selectfont 0.00}%
\end{pgfscope}%
\begin{pgfscope}%
\pgfsetbuttcap%
\pgfsetroundjoin%
\definecolor{currentfill}{rgb}{0.000000,0.000000,0.000000}%
\pgfsetfillcolor{currentfill}%
\pgfsetlinewidth{0.501875pt}%
\definecolor{currentstroke}{rgb}{0.000000,0.000000,0.000000}%
\pgfsetstrokecolor{currentstroke}%
\pgfsetdash{}{0pt}%
\pgfsys@defobject{currentmarker}{\pgfqpoint{0.000000in}{0.000000in}}{\pgfqpoint{0.069444in}{0.000000in}}{%
\pgfpathmoveto{\pgfqpoint{0.000000in}{0.000000in}}%
\pgfpathlineto{\pgfqpoint{0.069444in}{0.000000in}}%
\pgfusepath{stroke,fill}%
}%
\begin{pgfscope}%
\pgfsys@transformshift{0.366840in}{0.569910in}%
\pgfsys@useobject{currentmarker}{}%
\end{pgfscope}%
\end{pgfscope}%
\begin{pgfscope}%
\pgfsetbuttcap%
\pgfsetroundjoin%
\definecolor{currentfill}{rgb}{0.000000,0.000000,0.000000}%
\pgfsetfillcolor{currentfill}%
\pgfsetlinewidth{0.501875pt}%
\definecolor{currentstroke}{rgb}{0.000000,0.000000,0.000000}%
\pgfsetstrokecolor{currentstroke}%
\pgfsetdash{}{0pt}%
\pgfsys@defobject{currentmarker}{\pgfqpoint{-0.069444in}{0.000000in}}{\pgfqpoint{0.000000in}{0.000000in}}{%
\pgfpathmoveto{\pgfqpoint{0.000000in}{0.000000in}}%
\pgfpathlineto{\pgfqpoint{-0.069444in}{0.000000in}}%
\pgfusepath{stroke,fill}%
}%
\begin{pgfscope}%
\pgfsys@transformshift{2.563190in}{0.569910in}%
\pgfsys@useobject{currentmarker}{}%
\end{pgfscope}%
\end{pgfscope}%
\begin{pgfscope}%
\pgftext[x=0.297396in,y=0.569910in,right,]{\rmfamily\fontsize{8.000000}{9.600000}\selectfont 0.05}%
\end{pgfscope}%
\begin{pgfscope}%
\pgfsetbuttcap%
\pgfsetroundjoin%
\definecolor{currentfill}{rgb}{0.000000,0.000000,0.000000}%
\pgfsetfillcolor{currentfill}%
\pgfsetlinewidth{0.501875pt}%
\definecolor{currentstroke}{rgb}{0.000000,0.000000,0.000000}%
\pgfsetstrokecolor{currentstroke}%
\pgfsetdash{}{0pt}%
\pgfsys@defobject{currentmarker}{\pgfqpoint{0.000000in}{0.000000in}}{\pgfqpoint{0.069444in}{0.000000in}}{%
\pgfpathmoveto{\pgfqpoint{0.000000in}{0.000000in}}%
\pgfpathlineto{\pgfqpoint{0.069444in}{0.000000in}}%
\pgfusepath{stroke,fill}%
}%
\begin{pgfscope}%
\pgfsys@transformshift{0.366840in}{0.722429in}%
\pgfsys@useobject{currentmarker}{}%
\end{pgfscope}%
\end{pgfscope}%
\begin{pgfscope}%
\pgfsetbuttcap%
\pgfsetroundjoin%
\definecolor{currentfill}{rgb}{0.000000,0.000000,0.000000}%
\pgfsetfillcolor{currentfill}%
\pgfsetlinewidth{0.501875pt}%
\definecolor{currentstroke}{rgb}{0.000000,0.000000,0.000000}%
\pgfsetstrokecolor{currentstroke}%
\pgfsetdash{}{0pt}%
\pgfsys@defobject{currentmarker}{\pgfqpoint{-0.069444in}{0.000000in}}{\pgfqpoint{0.000000in}{0.000000in}}{%
\pgfpathmoveto{\pgfqpoint{0.000000in}{0.000000in}}%
\pgfpathlineto{\pgfqpoint{-0.069444in}{0.000000in}}%
\pgfusepath{stroke,fill}%
}%
\begin{pgfscope}%
\pgfsys@transformshift{2.563190in}{0.722429in}%
\pgfsys@useobject{currentmarker}{}%
\end{pgfscope}%
\end{pgfscope}%
\begin{pgfscope}%
\pgftext[x=0.297396in,y=0.722429in,right,]{\rmfamily\fontsize{8.000000}{9.600000}\selectfont 0.10}%
\end{pgfscope}%
\begin{pgfscope}%
\pgfsetbuttcap%
\pgfsetroundjoin%
\definecolor{currentfill}{rgb}{0.000000,0.000000,0.000000}%
\pgfsetfillcolor{currentfill}%
\pgfsetlinewidth{0.501875pt}%
\definecolor{currentstroke}{rgb}{0.000000,0.000000,0.000000}%
\pgfsetstrokecolor{currentstroke}%
\pgfsetdash{}{0pt}%
\pgfsys@defobject{currentmarker}{\pgfqpoint{0.000000in}{0.000000in}}{\pgfqpoint{0.069444in}{0.000000in}}{%
\pgfpathmoveto{\pgfqpoint{0.000000in}{0.000000in}}%
\pgfpathlineto{\pgfqpoint{0.069444in}{0.000000in}}%
\pgfusepath{stroke,fill}%
}%
\begin{pgfscope}%
\pgfsys@transformshift{0.366840in}{0.874948in}%
\pgfsys@useobject{currentmarker}{}%
\end{pgfscope}%
\end{pgfscope}%
\begin{pgfscope}%
\pgfsetbuttcap%
\pgfsetroundjoin%
\definecolor{currentfill}{rgb}{0.000000,0.000000,0.000000}%
\pgfsetfillcolor{currentfill}%
\pgfsetlinewidth{0.501875pt}%
\definecolor{currentstroke}{rgb}{0.000000,0.000000,0.000000}%
\pgfsetstrokecolor{currentstroke}%
\pgfsetdash{}{0pt}%
\pgfsys@defobject{currentmarker}{\pgfqpoint{-0.069444in}{0.000000in}}{\pgfqpoint{0.000000in}{0.000000in}}{%
\pgfpathmoveto{\pgfqpoint{0.000000in}{0.000000in}}%
\pgfpathlineto{\pgfqpoint{-0.069444in}{0.000000in}}%
\pgfusepath{stroke,fill}%
}%
\begin{pgfscope}%
\pgfsys@transformshift{2.563190in}{0.874948in}%
\pgfsys@useobject{currentmarker}{}%
\end{pgfscope}%
\end{pgfscope}%
\begin{pgfscope}%
\pgftext[x=0.297396in,y=0.874948in,right,]{\rmfamily\fontsize{8.000000}{9.600000}\selectfont 0.15}%
\end{pgfscope}%
\begin{pgfscope}%
\pgfsetbuttcap%
\pgfsetroundjoin%
\definecolor{currentfill}{rgb}{0.000000,0.000000,0.000000}%
\pgfsetfillcolor{currentfill}%
\pgfsetlinewidth{0.501875pt}%
\definecolor{currentstroke}{rgb}{0.000000,0.000000,0.000000}%
\pgfsetstrokecolor{currentstroke}%
\pgfsetdash{}{0pt}%
\pgfsys@defobject{currentmarker}{\pgfqpoint{0.000000in}{0.000000in}}{\pgfqpoint{0.069444in}{0.000000in}}{%
\pgfpathmoveto{\pgfqpoint{0.000000in}{0.000000in}}%
\pgfpathlineto{\pgfqpoint{0.069444in}{0.000000in}}%
\pgfusepath{stroke,fill}%
}%
\begin{pgfscope}%
\pgfsys@transformshift{0.366840in}{1.027467in}%
\pgfsys@useobject{currentmarker}{}%
\end{pgfscope}%
\end{pgfscope}%
\begin{pgfscope}%
\pgfsetbuttcap%
\pgfsetroundjoin%
\definecolor{currentfill}{rgb}{0.000000,0.000000,0.000000}%
\pgfsetfillcolor{currentfill}%
\pgfsetlinewidth{0.501875pt}%
\definecolor{currentstroke}{rgb}{0.000000,0.000000,0.000000}%
\pgfsetstrokecolor{currentstroke}%
\pgfsetdash{}{0pt}%
\pgfsys@defobject{currentmarker}{\pgfqpoint{-0.069444in}{0.000000in}}{\pgfqpoint{0.000000in}{0.000000in}}{%
\pgfpathmoveto{\pgfqpoint{0.000000in}{0.000000in}}%
\pgfpathlineto{\pgfqpoint{-0.069444in}{0.000000in}}%
\pgfusepath{stroke,fill}%
}%
\begin{pgfscope}%
\pgfsys@transformshift{2.563190in}{1.027467in}%
\pgfsys@useobject{currentmarker}{}%
\end{pgfscope}%
\end{pgfscope}%
\begin{pgfscope}%
\pgftext[x=0.297396in,y=1.027467in,right,]{\rmfamily\fontsize{8.000000}{9.600000}\selectfont 0.20}%
\end{pgfscope}%
\begin{pgfscope}%
\pgfsetbuttcap%
\pgfsetroundjoin%
\definecolor{currentfill}{rgb}{0.000000,0.000000,0.000000}%
\pgfsetfillcolor{currentfill}%
\pgfsetlinewidth{0.501875pt}%
\definecolor{currentstroke}{rgb}{0.000000,0.000000,0.000000}%
\pgfsetstrokecolor{currentstroke}%
\pgfsetdash{}{0pt}%
\pgfsys@defobject{currentmarker}{\pgfqpoint{0.000000in}{0.000000in}}{\pgfqpoint{0.069444in}{0.000000in}}{%
\pgfpathmoveto{\pgfqpoint{0.000000in}{0.000000in}}%
\pgfpathlineto{\pgfqpoint{0.069444in}{0.000000in}}%
\pgfusepath{stroke,fill}%
}%
\begin{pgfscope}%
\pgfsys@transformshift{0.366840in}{1.179987in}%
\pgfsys@useobject{currentmarker}{}%
\end{pgfscope}%
\end{pgfscope}%
\begin{pgfscope}%
\pgfsetbuttcap%
\pgfsetroundjoin%
\definecolor{currentfill}{rgb}{0.000000,0.000000,0.000000}%
\pgfsetfillcolor{currentfill}%
\pgfsetlinewidth{0.501875pt}%
\definecolor{currentstroke}{rgb}{0.000000,0.000000,0.000000}%
\pgfsetstrokecolor{currentstroke}%
\pgfsetdash{}{0pt}%
\pgfsys@defobject{currentmarker}{\pgfqpoint{-0.069444in}{0.000000in}}{\pgfqpoint{0.000000in}{0.000000in}}{%
\pgfpathmoveto{\pgfqpoint{0.000000in}{0.000000in}}%
\pgfpathlineto{\pgfqpoint{-0.069444in}{0.000000in}}%
\pgfusepath{stroke,fill}%
}%
\begin{pgfscope}%
\pgfsys@transformshift{2.563190in}{1.179987in}%
\pgfsys@useobject{currentmarker}{}%
\end{pgfscope}%
\end{pgfscope}%
\begin{pgfscope}%
\pgftext[x=0.297396in,y=1.179987in,right,]{\rmfamily\fontsize{8.000000}{9.600000}\selectfont 0.25}%
\end{pgfscope}%
\begin{pgfscope}%
\pgfsetbuttcap%
\pgfsetroundjoin%
\definecolor{currentfill}{rgb}{0.000000,0.000000,0.000000}%
\pgfsetfillcolor{currentfill}%
\pgfsetlinewidth{0.501875pt}%
\definecolor{currentstroke}{rgb}{0.000000,0.000000,0.000000}%
\pgfsetstrokecolor{currentstroke}%
\pgfsetdash{}{0pt}%
\pgfsys@defobject{currentmarker}{\pgfqpoint{0.000000in}{0.000000in}}{\pgfqpoint{0.069444in}{0.000000in}}{%
\pgfpathmoveto{\pgfqpoint{0.000000in}{0.000000in}}%
\pgfpathlineto{\pgfqpoint{0.069444in}{0.000000in}}%
\pgfusepath{stroke,fill}%
}%
\begin{pgfscope}%
\pgfsys@transformshift{0.366840in}{1.332506in}%
\pgfsys@useobject{currentmarker}{}%
\end{pgfscope}%
\end{pgfscope}%
\begin{pgfscope}%
\pgfsetbuttcap%
\pgfsetroundjoin%
\definecolor{currentfill}{rgb}{0.000000,0.000000,0.000000}%
\pgfsetfillcolor{currentfill}%
\pgfsetlinewidth{0.501875pt}%
\definecolor{currentstroke}{rgb}{0.000000,0.000000,0.000000}%
\pgfsetstrokecolor{currentstroke}%
\pgfsetdash{}{0pt}%
\pgfsys@defobject{currentmarker}{\pgfqpoint{-0.069444in}{0.000000in}}{\pgfqpoint{0.000000in}{0.000000in}}{%
\pgfpathmoveto{\pgfqpoint{0.000000in}{0.000000in}}%
\pgfpathlineto{\pgfqpoint{-0.069444in}{0.000000in}}%
\pgfusepath{stroke,fill}%
}%
\begin{pgfscope}%
\pgfsys@transformshift{2.563190in}{1.332506in}%
\pgfsys@useobject{currentmarker}{}%
\end{pgfscope}%
\end{pgfscope}%
\begin{pgfscope}%
\pgftext[x=0.297396in,y=1.332506in,right,]{\rmfamily\fontsize{8.000000}{9.600000}\selectfont 0.30}%
\end{pgfscope}%
\begin{pgfscope}%
\pgfsetbuttcap%
\pgfsetroundjoin%
\definecolor{currentfill}{rgb}{0.000000,0.000000,0.000000}%
\pgfsetfillcolor{currentfill}%
\pgfsetlinewidth{0.501875pt}%
\definecolor{currentstroke}{rgb}{0.000000,0.000000,0.000000}%
\pgfsetstrokecolor{currentstroke}%
\pgfsetdash{}{0pt}%
\pgfsys@defobject{currentmarker}{\pgfqpoint{0.000000in}{0.000000in}}{\pgfqpoint{0.069444in}{0.000000in}}{%
\pgfpathmoveto{\pgfqpoint{0.000000in}{0.000000in}}%
\pgfpathlineto{\pgfqpoint{0.069444in}{0.000000in}}%
\pgfusepath{stroke,fill}%
}%
\begin{pgfscope}%
\pgfsys@transformshift{0.366840in}{1.485025in}%
\pgfsys@useobject{currentmarker}{}%
\end{pgfscope}%
\end{pgfscope}%
\begin{pgfscope}%
\pgfsetbuttcap%
\pgfsetroundjoin%
\definecolor{currentfill}{rgb}{0.000000,0.000000,0.000000}%
\pgfsetfillcolor{currentfill}%
\pgfsetlinewidth{0.501875pt}%
\definecolor{currentstroke}{rgb}{0.000000,0.000000,0.000000}%
\pgfsetstrokecolor{currentstroke}%
\pgfsetdash{}{0pt}%
\pgfsys@defobject{currentmarker}{\pgfqpoint{-0.069444in}{0.000000in}}{\pgfqpoint{0.000000in}{0.000000in}}{%
\pgfpathmoveto{\pgfqpoint{0.000000in}{0.000000in}}%
\pgfpathlineto{\pgfqpoint{-0.069444in}{0.000000in}}%
\pgfusepath{stroke,fill}%
}%
\begin{pgfscope}%
\pgfsys@transformshift{2.563190in}{1.485025in}%
\pgfsys@useobject{currentmarker}{}%
\end{pgfscope}%
\end{pgfscope}%
\begin{pgfscope}%
\pgftext[x=0.297396in,y=1.485025in,right,]{\rmfamily\fontsize{8.000000}{9.600000}\selectfont 0.35}%
\end{pgfscope}%
\begin{pgfscope}%
\pgfsetbuttcap%
\pgfsetroundjoin%
\definecolor{currentfill}{rgb}{0.000000,0.000000,0.000000}%
\pgfsetfillcolor{currentfill}%
\pgfsetlinewidth{0.501875pt}%
\definecolor{currentstroke}{rgb}{0.000000,0.000000,0.000000}%
\pgfsetstrokecolor{currentstroke}%
\pgfsetdash{}{0pt}%
\pgfsys@defobject{currentmarker}{\pgfqpoint{0.000000in}{0.000000in}}{\pgfqpoint{0.069444in}{0.000000in}}{%
\pgfpathmoveto{\pgfqpoint{0.000000in}{0.000000in}}%
\pgfpathlineto{\pgfqpoint{0.069444in}{0.000000in}}%
\pgfusepath{stroke,fill}%
}%
\begin{pgfscope}%
\pgfsys@transformshift{0.366840in}{1.637544in}%
\pgfsys@useobject{currentmarker}{}%
\end{pgfscope}%
\end{pgfscope}%
\begin{pgfscope}%
\pgfsetbuttcap%
\pgfsetroundjoin%
\definecolor{currentfill}{rgb}{0.000000,0.000000,0.000000}%
\pgfsetfillcolor{currentfill}%
\pgfsetlinewidth{0.501875pt}%
\definecolor{currentstroke}{rgb}{0.000000,0.000000,0.000000}%
\pgfsetstrokecolor{currentstroke}%
\pgfsetdash{}{0pt}%
\pgfsys@defobject{currentmarker}{\pgfqpoint{-0.069444in}{0.000000in}}{\pgfqpoint{0.000000in}{0.000000in}}{%
\pgfpathmoveto{\pgfqpoint{0.000000in}{0.000000in}}%
\pgfpathlineto{\pgfqpoint{-0.069444in}{0.000000in}}%
\pgfusepath{stroke,fill}%
}%
\begin{pgfscope}%
\pgfsys@transformshift{2.563190in}{1.637544in}%
\pgfsys@useobject{currentmarker}{}%
\end{pgfscope}%
\end{pgfscope}%
\begin{pgfscope}%
\pgftext[x=0.297396in,y=1.637544in,right,]{\rmfamily\fontsize{8.000000}{9.600000}\selectfont 0.40}%
\end{pgfscope}%
\end{pgfpicture}%
\makeatother%
\endgroup%

	\end{subfigure}

	\begin{subfigure}[t]{0.49\textwidth}
		\centering
    %\includegraphics[width=\textwidth]{store/variables/DATA_MC_Kplus_PIDK.pdf}
    %% Creator: Matplotlib, PGF backend
%%
%% To include the figure in your LaTeX document, write
%%   \input{<filename>.pgf}
%%
%% Make sure the required packages are loaded in your preamble
%%   \usepackage{pgf}
%%
%% Figures using additional raster images can only be included by \input if
%% they are in the same directory as the main LaTeX file. For loading figures
%% from other directories you can use the `import` package
%%   \usepackage{import}
%% and then include the figures with
%%   \import{<path to file>}{<filename>.pgf}
%%
%% Matplotlib used the following preamble
%%   \usepackage{fontspec}
%%   \setmainfont{DejaVu Serif}
%%   \setsansfont{DejaVu Sans}
%%   \setmonofont{DejaVu Sans Mono}
%%
\begingroup%
\makeatletter%
\begin{pgfpicture}%
\pgfpathrectangle{\pgfpointorigin}{\pgfqpoint{2.684287in}{1.719349in}}%
\pgfusepath{use as bounding box, clip}%
\begin{pgfscope}%
\pgfsetbuttcap%
\pgfsetmiterjoin%
\definecolor{currentfill}{rgb}{1.000000,1.000000,1.000000}%
\pgfsetfillcolor{currentfill}%
\pgfsetlinewidth{0.000000pt}%
\definecolor{currentstroke}{rgb}{1.000000,1.000000,1.000000}%
\pgfsetstrokecolor{currentstroke}%
\pgfsetdash{}{0pt}%
\pgfpathmoveto{\pgfqpoint{0.000000in}{0.000000in}}%
\pgfpathlineto{\pgfqpoint{2.684287in}{0.000000in}}%
\pgfpathlineto{\pgfqpoint{2.684287in}{1.719349in}}%
\pgfpathlineto{\pgfqpoint{0.000000in}{1.719349in}}%
\pgfpathclose%
\pgfusepath{fill}%
\end{pgfscope}%
\begin{pgfscope}%
\pgfsetbuttcap%
\pgfsetmiterjoin%
\definecolor{currentfill}{rgb}{1.000000,1.000000,1.000000}%
\pgfsetfillcolor{currentfill}%
\pgfsetlinewidth{0.000000pt}%
\definecolor{currentstroke}{rgb}{0.000000,0.000000,0.000000}%
\pgfsetstrokecolor{currentstroke}%
\pgfsetstrokeopacity{0.000000}%
\pgfsetdash{}{0pt}%
\pgfpathmoveto{\pgfqpoint{0.437532in}{0.449983in}}%
\pgfpathlineto{\pgfqpoint{2.528249in}{0.449983in}}%
\pgfpathlineto{\pgfqpoint{2.528249in}{1.615583in}}%
\pgfpathlineto{\pgfqpoint{0.437532in}{1.615583in}}%
\pgfpathclose%
\pgfusepath{fill}%
\end{pgfscope}%
\begin{pgfscope}%
\pgfpathrectangle{\pgfqpoint{0.437532in}{0.449983in}}{\pgfqpoint{2.090716in}{1.165600in}} %
\pgfusepath{clip}%
\pgfsetbuttcap%
\pgfsetmiterjoin%
\definecolor{currentfill}{rgb}{0.215686,0.470588,0.749020}%
\pgfsetfillcolor{currentfill}%
\pgfsetlinewidth{0.000000pt}%
\definecolor{currentstroke}{rgb}{0.000000,0.000000,0.000000}%
\pgfsetstrokecolor{currentstroke}%
\pgfsetdash{}{0pt}%
\pgfpathmoveto{\pgfqpoint{0.502867in}{0.449983in}}%
\pgfpathlineto{\pgfqpoint{0.502867in}{0.449983in}}%
\pgfpathlineto{\pgfqpoint{0.540762in}{0.449983in}}%
\pgfpathlineto{\pgfqpoint{0.540762in}{0.449983in}}%
\pgfpathlineto{\pgfqpoint{0.578656in}{0.449983in}}%
\pgfpathlineto{\pgfqpoint{0.578656in}{0.449983in}}%
\pgfpathlineto{\pgfqpoint{0.616550in}{0.449983in}}%
\pgfpathlineto{\pgfqpoint{0.616550in}{0.502606in}}%
\pgfpathlineto{\pgfqpoint{0.654444in}{0.502606in}}%
\pgfpathlineto{\pgfqpoint{0.654444in}{0.688850in}}%
\pgfpathlineto{\pgfqpoint{0.692339in}{0.688850in}}%
\pgfpathlineto{\pgfqpoint{0.692339in}{0.995587in}}%
\pgfpathlineto{\pgfqpoint{0.730233in}{0.995587in}}%
\pgfpathlineto{\pgfqpoint{0.730233in}{1.083500in}}%
\pgfpathlineto{\pgfqpoint{0.768127in}{1.083500in}}%
\pgfpathlineto{\pgfqpoint{0.768127in}{1.186689in}}%
\pgfpathlineto{\pgfqpoint{0.806021in}{1.186689in}}%
\pgfpathlineto{\pgfqpoint{0.806021in}{1.302145in}}%
\pgfpathlineto{\pgfqpoint{0.843915in}{1.302145in}}%
\pgfpathlineto{\pgfqpoint{0.843915in}{1.410180in}}%
\pgfpathlineto{\pgfqpoint{0.881810in}{1.410180in}}%
\pgfpathlineto{\pgfqpoint{0.881810in}{1.488705in}}%
\pgfpathlineto{\pgfqpoint{0.919704in}{1.488705in}}%
\pgfpathlineto{\pgfqpoint{0.919704in}{1.517650in}}%
\pgfpathlineto{\pgfqpoint{0.957598in}{1.517650in}}%
\pgfpathlineto{\pgfqpoint{0.957598in}{1.501132in}}%
\pgfpathlineto{\pgfqpoint{0.995492in}{1.501132in}}%
\pgfpathlineto{\pgfqpoint{0.995492in}{1.432514in}}%
\pgfpathlineto{\pgfqpoint{1.033387in}{1.432514in}}%
\pgfpathlineto{\pgfqpoint{1.033387in}{1.332528in}}%
\pgfpathlineto{\pgfqpoint{1.071281in}{1.332528in}}%
\pgfpathlineto{\pgfqpoint{1.071281in}{1.223426in}}%
\pgfpathlineto{\pgfqpoint{1.109175in}{1.223426in}}%
\pgfpathlineto{\pgfqpoint{1.109175in}{1.103410in}}%
\pgfpathlineto{\pgfqpoint{1.147069in}{1.103410in}}%
\pgfpathlineto{\pgfqpoint{1.147069in}{1.000454in}}%
\pgfpathlineto{\pgfqpoint{1.184964in}{1.000454in}}%
\pgfpathlineto{\pgfqpoint{1.184964in}{0.904260in}}%
\pgfpathlineto{\pgfqpoint{1.222858in}{0.904260in}}%
\pgfpathlineto{\pgfqpoint{1.222858in}{0.830013in}}%
\pgfpathlineto{\pgfqpoint{1.260752in}{0.830013in}}%
\pgfpathlineto{\pgfqpoint{1.260752in}{0.759209in}}%
\pgfpathlineto{\pgfqpoint{1.298646in}{0.759209in}}%
\pgfpathlineto{\pgfqpoint{1.298646in}{0.697569in}}%
\pgfpathlineto{\pgfqpoint{1.336540in}{0.697569in}}%
\pgfpathlineto{\pgfqpoint{1.336540in}{0.654094in}}%
\pgfpathlineto{\pgfqpoint{1.374435in}{0.654094in}}%
\pgfpathlineto{\pgfqpoint{1.374435in}{0.619874in}}%
\pgfpathlineto{\pgfqpoint{1.412329in}{0.619874in}}%
\pgfpathlineto{\pgfqpoint{1.412329in}{0.580060in}}%
\pgfpathlineto{\pgfqpoint{1.450223in}{0.580060in}}%
\pgfpathlineto{\pgfqpoint{1.450223in}{0.558427in}}%
\pgfpathlineto{\pgfqpoint{1.488117in}{0.558427in}}%
\pgfpathlineto{\pgfqpoint{1.488117in}{0.533679in}}%
\pgfpathlineto{\pgfqpoint{1.526012in}{0.533679in}}%
\pgfpathlineto{\pgfqpoint{1.526012in}{0.518138in}}%
\pgfpathlineto{\pgfqpoint{1.563906in}{0.518138in}}%
\pgfpathlineto{\pgfqpoint{1.563906in}{0.500889in}}%
\pgfpathlineto{\pgfqpoint{1.601800in}{0.500889in}}%
\pgfpathlineto{\pgfqpoint{1.601800in}{0.489614in}}%
\pgfpathlineto{\pgfqpoint{1.639694in}{0.489614in}}%
\pgfpathlineto{\pgfqpoint{1.639694in}{0.482205in}}%
\pgfpathlineto{\pgfqpoint{1.677589in}{0.482205in}}%
\pgfpathlineto{\pgfqpoint{1.677589in}{0.473359in}}%
\pgfpathlineto{\pgfqpoint{1.715483in}{0.473359in}}%
\pgfpathlineto{\pgfqpoint{1.715483in}{0.469518in}}%
\pgfpathlineto{\pgfqpoint{1.753377in}{0.469518in}}%
\pgfpathlineto{\pgfqpoint{1.753377in}{0.464460in}}%
\pgfpathlineto{\pgfqpoint{1.791271in}{0.464460in}}%
\pgfpathlineto{\pgfqpoint{1.791271in}{0.460563in}}%
\pgfpathlineto{\pgfqpoint{1.829166in}{0.460563in}}%
\pgfpathlineto{\pgfqpoint{1.829166in}{0.458320in}}%
\pgfpathlineto{\pgfqpoint{1.867060in}{0.458320in}}%
\pgfpathlineto{\pgfqpoint{1.867060in}{0.455191in}}%
\pgfpathlineto{\pgfqpoint{1.904954in}{0.455191in}}%
\pgfpathlineto{\pgfqpoint{1.904954in}{0.454450in}}%
\pgfpathlineto{\pgfqpoint{1.942848in}{0.454450in}}%
\pgfpathlineto{\pgfqpoint{1.942848in}{0.453445in}}%
\pgfpathlineto{\pgfqpoint{1.980742in}{0.453445in}}%
\pgfpathlineto{\pgfqpoint{1.980742in}{0.453224in}}%
\pgfpathlineto{\pgfqpoint{2.018637in}{0.453224in}}%
\pgfpathlineto{\pgfqpoint{2.018637in}{0.451387in}}%
\pgfpathlineto{\pgfqpoint{2.056531in}{0.451387in}}%
\pgfpathlineto{\pgfqpoint{2.056531in}{0.451749in}}%
\pgfpathlineto{\pgfqpoint{2.094425in}{0.451749in}}%
\pgfpathlineto{\pgfqpoint{2.094425in}{0.450687in}}%
\pgfpathlineto{\pgfqpoint{2.132319in}{0.450687in}}%
\pgfpathlineto{\pgfqpoint{2.132319in}{0.450855in}}%
\pgfpathlineto{\pgfqpoint{2.170214in}{0.450855in}}%
\pgfpathlineto{\pgfqpoint{2.170214in}{0.450761in}}%
\pgfpathlineto{\pgfqpoint{2.208108in}{0.450761in}}%
\pgfpathlineto{\pgfqpoint{2.208108in}{0.450599in}}%
\pgfpathlineto{\pgfqpoint{2.246002in}{0.450599in}}%
\pgfpathlineto{\pgfqpoint{2.246002in}{0.450263in}}%
\pgfpathlineto{\pgfqpoint{2.283896in}{0.450263in}}%
\pgfpathlineto{\pgfqpoint{2.283896in}{0.450342in}}%
\pgfpathlineto{\pgfqpoint{2.321791in}{0.450342in}}%
\pgfpathlineto{\pgfqpoint{2.321791in}{0.450141in}}%
\pgfpathlineto{\pgfqpoint{2.359685in}{0.450141in}}%
\pgfpathlineto{\pgfqpoint{2.359685in}{0.450201in}}%
\pgfpathlineto{\pgfqpoint{2.397579in}{0.450201in}}%
\pgfpathlineto{\pgfqpoint{2.397579in}{0.449983in}}%
\pgfpathlineto{\pgfqpoint{2.359685in}{0.449983in}}%
\pgfpathlineto{\pgfqpoint{2.359685in}{0.449983in}}%
\pgfpathlineto{\pgfqpoint{2.321791in}{0.449983in}}%
\pgfpathlineto{\pgfqpoint{2.321791in}{0.449983in}}%
\pgfpathlineto{\pgfqpoint{2.283896in}{0.449983in}}%
\pgfpathlineto{\pgfqpoint{2.283896in}{0.449983in}}%
\pgfpathlineto{\pgfqpoint{2.246002in}{0.449983in}}%
\pgfpathlineto{\pgfqpoint{2.246002in}{0.449983in}}%
\pgfpathlineto{\pgfqpoint{2.208108in}{0.449983in}}%
\pgfpathlineto{\pgfqpoint{2.208108in}{0.449983in}}%
\pgfpathlineto{\pgfqpoint{2.170214in}{0.449983in}}%
\pgfpathlineto{\pgfqpoint{2.170214in}{0.449983in}}%
\pgfpathlineto{\pgfqpoint{2.132319in}{0.449983in}}%
\pgfpathlineto{\pgfqpoint{2.132319in}{0.449983in}}%
\pgfpathlineto{\pgfqpoint{2.094425in}{0.449983in}}%
\pgfpathlineto{\pgfqpoint{2.094425in}{0.449983in}}%
\pgfpathlineto{\pgfqpoint{2.056531in}{0.449983in}}%
\pgfpathlineto{\pgfqpoint{2.056531in}{0.449983in}}%
\pgfpathlineto{\pgfqpoint{2.018637in}{0.449983in}}%
\pgfpathlineto{\pgfqpoint{2.018637in}{0.449983in}}%
\pgfpathlineto{\pgfqpoint{1.980742in}{0.449983in}}%
\pgfpathlineto{\pgfqpoint{1.980742in}{0.449983in}}%
\pgfpathlineto{\pgfqpoint{1.942848in}{0.449983in}}%
\pgfpathlineto{\pgfqpoint{1.942848in}{0.449983in}}%
\pgfpathlineto{\pgfqpoint{1.904954in}{0.449983in}}%
\pgfpathlineto{\pgfqpoint{1.904954in}{0.449983in}}%
\pgfpathlineto{\pgfqpoint{1.867060in}{0.449983in}}%
\pgfpathlineto{\pgfqpoint{1.867060in}{0.449983in}}%
\pgfpathlineto{\pgfqpoint{1.829166in}{0.449983in}}%
\pgfpathlineto{\pgfqpoint{1.829166in}{0.449983in}}%
\pgfpathlineto{\pgfqpoint{1.791271in}{0.449983in}}%
\pgfpathlineto{\pgfqpoint{1.791271in}{0.449983in}}%
\pgfpathlineto{\pgfqpoint{1.753377in}{0.449983in}}%
\pgfpathlineto{\pgfqpoint{1.753377in}{0.449983in}}%
\pgfpathlineto{\pgfqpoint{1.715483in}{0.449983in}}%
\pgfpathlineto{\pgfqpoint{1.715483in}{0.449983in}}%
\pgfpathlineto{\pgfqpoint{1.677589in}{0.449983in}}%
\pgfpathlineto{\pgfqpoint{1.677589in}{0.449983in}}%
\pgfpathlineto{\pgfqpoint{1.639694in}{0.449983in}}%
\pgfpathlineto{\pgfqpoint{1.639694in}{0.449983in}}%
\pgfpathlineto{\pgfqpoint{1.601800in}{0.449983in}}%
\pgfpathlineto{\pgfqpoint{1.601800in}{0.449983in}}%
\pgfpathlineto{\pgfqpoint{1.563906in}{0.449983in}}%
\pgfpathlineto{\pgfqpoint{1.563906in}{0.449983in}}%
\pgfpathlineto{\pgfqpoint{1.526012in}{0.449983in}}%
\pgfpathlineto{\pgfqpoint{1.526012in}{0.449983in}}%
\pgfpathlineto{\pgfqpoint{1.488117in}{0.449983in}}%
\pgfpathlineto{\pgfqpoint{1.488117in}{0.449983in}}%
\pgfpathlineto{\pgfqpoint{1.450223in}{0.449983in}}%
\pgfpathlineto{\pgfqpoint{1.450223in}{0.449983in}}%
\pgfpathlineto{\pgfqpoint{1.412329in}{0.449983in}}%
\pgfpathlineto{\pgfqpoint{1.412329in}{0.449983in}}%
\pgfpathlineto{\pgfqpoint{1.374435in}{0.449983in}}%
\pgfpathlineto{\pgfqpoint{1.374435in}{0.449983in}}%
\pgfpathlineto{\pgfqpoint{1.336540in}{0.449983in}}%
\pgfpathlineto{\pgfqpoint{1.336540in}{0.449983in}}%
\pgfpathlineto{\pgfqpoint{1.298646in}{0.449983in}}%
\pgfpathlineto{\pgfqpoint{1.298646in}{0.449983in}}%
\pgfpathlineto{\pgfqpoint{1.260752in}{0.449983in}}%
\pgfpathlineto{\pgfqpoint{1.260752in}{0.449983in}}%
\pgfpathlineto{\pgfqpoint{1.222858in}{0.449983in}}%
\pgfpathlineto{\pgfqpoint{1.222858in}{0.449983in}}%
\pgfpathlineto{\pgfqpoint{1.184964in}{0.449983in}}%
\pgfpathlineto{\pgfqpoint{1.184964in}{0.449983in}}%
\pgfpathlineto{\pgfqpoint{1.147069in}{0.449983in}}%
\pgfpathlineto{\pgfqpoint{1.147069in}{0.449983in}}%
\pgfpathlineto{\pgfqpoint{1.109175in}{0.449983in}}%
\pgfpathlineto{\pgfqpoint{1.109175in}{0.449983in}}%
\pgfpathlineto{\pgfqpoint{1.071281in}{0.449983in}}%
\pgfpathlineto{\pgfqpoint{1.071281in}{0.449983in}}%
\pgfpathlineto{\pgfqpoint{1.033387in}{0.449983in}}%
\pgfpathlineto{\pgfqpoint{1.033387in}{0.449983in}}%
\pgfpathlineto{\pgfqpoint{0.995492in}{0.449983in}}%
\pgfpathlineto{\pgfqpoint{0.995492in}{0.449983in}}%
\pgfpathlineto{\pgfqpoint{0.957598in}{0.449983in}}%
\pgfpathlineto{\pgfqpoint{0.957598in}{0.449983in}}%
\pgfpathlineto{\pgfqpoint{0.919704in}{0.449983in}}%
\pgfpathlineto{\pgfqpoint{0.919704in}{0.449983in}}%
\pgfpathlineto{\pgfqpoint{0.881810in}{0.449983in}}%
\pgfpathlineto{\pgfqpoint{0.881810in}{0.449983in}}%
\pgfpathlineto{\pgfqpoint{0.843915in}{0.449983in}}%
\pgfpathlineto{\pgfqpoint{0.843915in}{0.449983in}}%
\pgfpathlineto{\pgfqpoint{0.806021in}{0.449983in}}%
\pgfpathlineto{\pgfqpoint{0.806021in}{0.449983in}}%
\pgfpathlineto{\pgfqpoint{0.768127in}{0.449983in}}%
\pgfpathlineto{\pgfqpoint{0.768127in}{0.449983in}}%
\pgfpathlineto{\pgfqpoint{0.730233in}{0.449983in}}%
\pgfpathlineto{\pgfqpoint{0.730233in}{0.449983in}}%
\pgfpathlineto{\pgfqpoint{0.692339in}{0.449983in}}%
\pgfpathlineto{\pgfqpoint{0.692339in}{0.449983in}}%
\pgfpathlineto{\pgfqpoint{0.654444in}{0.449983in}}%
\pgfpathlineto{\pgfqpoint{0.654444in}{0.449983in}}%
\pgfpathlineto{\pgfqpoint{0.616550in}{0.449983in}}%
\pgfpathlineto{\pgfqpoint{0.616550in}{0.449983in}}%
\pgfpathlineto{\pgfqpoint{0.578656in}{0.449983in}}%
\pgfpathlineto{\pgfqpoint{0.578656in}{0.449983in}}%
\pgfpathlineto{\pgfqpoint{0.540762in}{0.449983in}}%
\pgfpathlineto{\pgfqpoint{0.540762in}{0.449983in}}%
\pgfpathlineto{\pgfqpoint{0.502867in}{0.449983in}}%
\pgfusepath{fill}%
\end{pgfscope}%
\begin{pgfscope}%
\pgfpathrectangle{\pgfqpoint{0.437532in}{0.449983in}}{\pgfqpoint{2.090716in}{1.165600in}} %
\pgfusepath{clip}%
\pgfsetbuttcap%
\pgfsetmiterjoin%
\pgfsetlinewidth{0.501875pt}%
\definecolor{currentstroke}{rgb}{1.000000,0.000000,0.000000}%
\pgfsetstrokecolor{currentstroke}%
\pgfsetdash{}{0pt}%
\pgfpathmoveto{\pgfqpoint{0.502867in}{0.449983in}}%
\pgfpathlineto{\pgfqpoint{0.502867in}{0.449983in}}%
\pgfpathlineto{\pgfqpoint{0.540762in}{0.449983in}}%
\pgfpathlineto{\pgfqpoint{0.540762in}{0.449983in}}%
\pgfpathlineto{\pgfqpoint{0.578656in}{0.449983in}}%
\pgfpathlineto{\pgfqpoint{0.578656in}{0.449983in}}%
\pgfpathlineto{\pgfqpoint{0.616550in}{0.449983in}}%
\pgfpathlineto{\pgfqpoint{0.616550in}{0.484980in}}%
\pgfpathlineto{\pgfqpoint{0.654444in}{0.484980in}}%
\pgfpathlineto{\pgfqpoint{0.654444in}{0.630616in}}%
\pgfpathlineto{\pgfqpoint{0.692339in}{0.630616in}}%
\pgfpathlineto{\pgfqpoint{0.692339in}{0.921517in}}%
\pgfpathlineto{\pgfqpoint{0.730233in}{0.921517in}}%
\pgfpathlineto{\pgfqpoint{0.730233in}{0.953287in}}%
\pgfpathlineto{\pgfqpoint{0.768127in}{0.953287in}}%
\pgfpathlineto{\pgfqpoint{0.768127in}{0.990066in}}%
\pgfpathlineto{\pgfqpoint{0.806021in}{0.990066in}}%
\pgfpathlineto{\pgfqpoint{0.806021in}{1.044360in}}%
\pgfpathlineto{\pgfqpoint{0.843915in}{1.044360in}}%
\pgfpathlineto{\pgfqpoint{0.843915in}{1.123834in}}%
\pgfpathlineto{\pgfqpoint{0.881810in}{1.123834in}}%
\pgfpathlineto{\pgfqpoint{0.881810in}{1.210738in}}%
\pgfpathlineto{\pgfqpoint{0.919704in}{1.210738in}}%
\pgfpathlineto{\pgfqpoint{0.919704in}{1.258140in}}%
\pgfpathlineto{\pgfqpoint{0.957598in}{1.258140in}}%
\pgfpathlineto{\pgfqpoint{0.957598in}{1.289472in}}%
\pgfpathlineto{\pgfqpoint{0.995492in}{1.289472in}}%
\pgfpathlineto{\pgfqpoint{0.995492in}{1.292565in}}%
\pgfpathlineto{\pgfqpoint{1.033387in}{1.292565in}}%
\pgfpathlineto{\pgfqpoint{1.033387in}{1.243045in}}%
\pgfpathlineto{\pgfqpoint{1.071281in}{1.243045in}}%
\pgfpathlineto{\pgfqpoint{1.071281in}{1.195677in}}%
\pgfpathlineto{\pgfqpoint{1.109175in}{1.195677in}}%
\pgfpathlineto{\pgfqpoint{1.109175in}{1.122556in}}%
\pgfpathlineto{\pgfqpoint{1.147069in}{1.122556in}}%
\pgfpathlineto{\pgfqpoint{1.147069in}{1.064598in}}%
\pgfpathlineto{\pgfqpoint{1.184964in}{1.064598in}}%
\pgfpathlineto{\pgfqpoint{1.184964in}{0.991209in}}%
\pgfpathlineto{\pgfqpoint{1.222858in}{0.991209in}}%
\pgfpathlineto{\pgfqpoint{1.222858in}{0.938293in}}%
\pgfpathlineto{\pgfqpoint{1.260752in}{0.938293in}}%
\pgfpathlineto{\pgfqpoint{1.260752in}{0.886487in}}%
\pgfpathlineto{\pgfqpoint{1.298646in}{0.886487in}}%
\pgfpathlineto{\pgfqpoint{1.298646in}{0.847086in}}%
\pgfpathlineto{\pgfqpoint{1.336540in}{0.847086in}}%
\pgfpathlineto{\pgfqpoint{1.336540in}{0.799953in}}%
\pgfpathlineto{\pgfqpoint{1.374435in}{0.799953in}}%
\pgfpathlineto{\pgfqpoint{1.374435in}{0.759005in}}%
\pgfpathlineto{\pgfqpoint{1.412329in}{0.759005in}}%
\pgfpathlineto{\pgfqpoint{1.412329in}{0.720882in}}%
\pgfpathlineto{\pgfqpoint{1.450223in}{0.720882in}}%
\pgfpathlineto{\pgfqpoint{1.450223in}{0.691365in}}%
\pgfpathlineto{\pgfqpoint{1.488117in}{0.691365in}}%
\pgfpathlineto{\pgfqpoint{1.488117in}{0.660099in}}%
\pgfpathlineto{\pgfqpoint{1.526012in}{0.660099in}}%
\pgfpathlineto{\pgfqpoint{1.526012in}{0.632129in}}%
\pgfpathlineto{\pgfqpoint{1.563906in}{0.632129in}}%
\pgfpathlineto{\pgfqpoint{1.563906in}{0.606982in}}%
\pgfpathlineto{\pgfqpoint{1.601800in}{0.606982in}}%
\pgfpathlineto{\pgfqpoint{1.601800in}{0.577633in}}%
\pgfpathlineto{\pgfqpoint{1.639694in}{0.577633in}}%
\pgfpathlineto{\pgfqpoint{1.639694in}{0.563581in}}%
\pgfpathlineto{\pgfqpoint{1.677589in}{0.563581in}}%
\pgfpathlineto{\pgfqpoint{1.677589in}{0.540787in}}%
\pgfpathlineto{\pgfqpoint{1.715483in}{0.540787in}}%
\pgfpathlineto{\pgfqpoint{1.715483in}{0.527979in}}%
\pgfpathlineto{\pgfqpoint{1.753377in}{0.527979in}}%
\pgfpathlineto{\pgfqpoint{1.753377in}{0.516481in}}%
\pgfpathlineto{\pgfqpoint{1.791271in}{0.516481in}}%
\pgfpathlineto{\pgfqpoint{1.791271in}{0.504748in}}%
\pgfpathlineto{\pgfqpoint{1.829166in}{0.504748in}}%
\pgfpathlineto{\pgfqpoint{1.829166in}{0.492309in}}%
\pgfpathlineto{\pgfqpoint{1.867060in}{0.492309in}}%
\pgfpathlineto{\pgfqpoint{1.867060in}{0.485518in}}%
\pgfpathlineto{\pgfqpoint{1.904954in}{0.485518in}}%
\pgfpathlineto{\pgfqpoint{1.904954in}{0.479736in}}%
\pgfpathlineto{\pgfqpoint{1.942848in}{0.479736in}}%
\pgfpathlineto{\pgfqpoint{1.942848in}{0.471331in}}%
\pgfpathlineto{\pgfqpoint{1.980742in}{0.471331in}}%
\pgfpathlineto{\pgfqpoint{1.980742in}{0.467734in}}%
\pgfpathlineto{\pgfqpoint{2.018637in}{0.467734in}}%
\pgfpathlineto{\pgfqpoint{2.018637in}{0.465280in}}%
\pgfpathlineto{\pgfqpoint{2.056531in}{0.465280in}}%
\pgfpathlineto{\pgfqpoint{2.056531in}{0.461010in}}%
\pgfpathlineto{\pgfqpoint{2.094425in}{0.461010in}}%
\pgfpathlineto{\pgfqpoint{2.094425in}{0.458657in}}%
\pgfpathlineto{\pgfqpoint{2.132319in}{0.458657in}}%
\pgfpathlineto{\pgfqpoint{2.132319in}{0.456472in}}%
\pgfpathlineto{\pgfqpoint{2.170214in}{0.456472in}}%
\pgfpathlineto{\pgfqpoint{2.170214in}{0.455598in}}%
\pgfpathlineto{\pgfqpoint{2.208108in}{0.455598in}}%
\pgfpathlineto{\pgfqpoint{2.208108in}{0.454556in}}%
\pgfpathlineto{\pgfqpoint{2.246002in}{0.454556in}}%
\pgfpathlineto{\pgfqpoint{2.246002in}{0.452976in}}%
\pgfpathlineto{\pgfqpoint{2.283896in}{0.452976in}}%
\pgfpathlineto{\pgfqpoint{2.283896in}{0.452606in}}%
\pgfpathlineto{\pgfqpoint{2.321791in}{0.452606in}}%
\pgfpathlineto{\pgfqpoint{2.321791in}{0.451564in}}%
\pgfpathlineto{\pgfqpoint{2.359685in}{0.451564in}}%
\pgfpathlineto{\pgfqpoint{2.359685in}{0.451429in}}%
\pgfpathlineto{\pgfqpoint{2.397579in}{0.451429in}}%
\pgfpathlineto{\pgfqpoint{2.397579in}{0.449983in}}%
\pgfusepath{stroke}%
\end{pgfscope}%
\begin{pgfscope}%
\pgfpathrectangle{\pgfqpoint{0.437532in}{0.449983in}}{\pgfqpoint{2.090716in}{1.165600in}} %
\pgfusepath{clip}%
\pgfsetbuttcap%
\pgfsetmiterjoin%
\pgfsetlinewidth{0.501875pt}%
\definecolor{currentstroke}{rgb}{1.000000,0.647059,0.000000}%
\pgfsetstrokecolor{currentstroke}%
\pgfsetdash{}{0pt}%
\pgfpathmoveto{\pgfqpoint{0.502867in}{0.449983in}}%
\pgfpathlineto{\pgfqpoint{0.502867in}{0.449983in}}%
\pgfpathlineto{\pgfqpoint{0.540762in}{0.449983in}}%
\pgfpathlineto{\pgfqpoint{0.540762in}{0.449983in}}%
\pgfpathlineto{\pgfqpoint{0.578656in}{0.449983in}}%
\pgfpathlineto{\pgfqpoint{0.578656in}{0.449983in}}%
\pgfpathlineto{\pgfqpoint{0.616550in}{0.449983in}}%
\pgfpathlineto{\pgfqpoint{0.616550in}{0.550611in}}%
\pgfpathlineto{\pgfqpoint{0.654444in}{0.550611in}}%
\pgfpathlineto{\pgfqpoint{0.654444in}{0.827468in}}%
\pgfpathlineto{\pgfqpoint{0.692339in}{0.827468in}}%
\pgfpathlineto{\pgfqpoint{0.692339in}{1.103392in}}%
\pgfpathlineto{\pgfqpoint{0.730233in}{1.103392in}}%
\pgfpathlineto{\pgfqpoint{0.730233in}{1.140778in}}%
\pgfpathlineto{\pgfqpoint{0.768127in}{1.140778in}}%
\pgfpathlineto{\pgfqpoint{0.768127in}{1.227227in}}%
\pgfpathlineto{\pgfqpoint{0.806021in}{1.227227in}}%
\pgfpathlineto{\pgfqpoint{0.806021in}{1.291925in}}%
\pgfpathlineto{\pgfqpoint{0.843915in}{1.291925in}}%
\pgfpathlineto{\pgfqpoint{0.843915in}{1.413589in}}%
\pgfpathlineto{\pgfqpoint{0.881810in}{1.413589in}}%
\pgfpathlineto{\pgfqpoint{0.881810in}{1.489315in}}%
\pgfpathlineto{\pgfqpoint{0.919704in}{1.489315in}}%
\pgfpathlineto{\pgfqpoint{0.919704in}{1.550076in}}%
\pgfpathlineto{\pgfqpoint{0.957598in}{1.550076in}}%
\pgfpathlineto{\pgfqpoint{0.957598in}{1.525300in}}%
\pgfpathlineto{\pgfqpoint{0.995492in}{1.525300in}}%
\pgfpathlineto{\pgfqpoint{0.995492in}{1.463609in}}%
\pgfpathlineto{\pgfqpoint{1.033387in}{1.463609in}}%
\pgfpathlineto{\pgfqpoint{1.033387in}{1.314568in}}%
\pgfpathlineto{\pgfqpoint{1.071281in}{1.314568in}}%
\pgfpathlineto{\pgfqpoint{1.071281in}{1.215694in}}%
\pgfpathlineto{\pgfqpoint{1.109175in}{1.215694in}}%
\pgfpathlineto{\pgfqpoint{1.109175in}{1.087697in}}%
\pgfpathlineto{\pgfqpoint{1.147069in}{1.087697in}}%
\pgfpathlineto{\pgfqpoint{1.147069in}{0.973573in}}%
\pgfpathlineto{\pgfqpoint{1.184964in}{0.973573in}}%
\pgfpathlineto{\pgfqpoint{1.184964in}{0.873989in}}%
\pgfpathlineto{\pgfqpoint{1.222858in}{0.873989in}}%
\pgfpathlineto{\pgfqpoint{1.222858in}{0.792410in}}%
\pgfpathlineto{\pgfqpoint{1.260752in}{0.792410in}}%
\pgfpathlineto{\pgfqpoint{1.260752in}{0.716723in}}%
\pgfpathlineto{\pgfqpoint{1.298646in}{0.716723in}}%
\pgfpathlineto{\pgfqpoint{1.298646in}{0.674538in}}%
\pgfpathlineto{\pgfqpoint{1.336540in}{0.674538in}}%
\pgfpathlineto{\pgfqpoint{1.336540in}{0.631082in}}%
\pgfpathlineto{\pgfqpoint{1.374435in}{0.631082in}}%
\pgfpathlineto{\pgfqpoint{1.374435in}{0.579254in}}%
\pgfpathlineto{\pgfqpoint{1.412329in}{0.579254in}}%
\pgfpathlineto{\pgfqpoint{1.412329in}{0.551573in}}%
\pgfpathlineto{\pgfqpoint{1.450223in}{0.551573in}}%
\pgfpathlineto{\pgfqpoint{1.450223in}{0.530417in}}%
\pgfpathlineto{\pgfqpoint{1.488117in}{0.530417in}}%
\pgfpathlineto{\pgfqpoint{1.488117in}{0.506366in}}%
\pgfpathlineto{\pgfqpoint{1.526012in}{0.506366in}}%
\pgfpathlineto{\pgfqpoint{1.526012in}{0.491738in}}%
\pgfpathlineto{\pgfqpoint{1.563906in}{0.491738in}}%
\pgfpathlineto{\pgfqpoint{1.563906in}{0.481727in}}%
\pgfpathlineto{\pgfqpoint{1.601800in}{0.481727in}}%
\pgfpathlineto{\pgfqpoint{1.601800in}{0.475918in}}%
\pgfpathlineto{\pgfqpoint{1.639694in}{0.475918in}}%
\pgfpathlineto{\pgfqpoint{1.639694in}{0.468881in}}%
\pgfpathlineto{\pgfqpoint{1.677589in}{0.468881in}}%
\pgfpathlineto{\pgfqpoint{1.677589in}{0.462590in}}%
\pgfpathlineto{\pgfqpoint{1.715483in}{0.462590in}}%
\pgfpathlineto{\pgfqpoint{1.715483in}{0.462714in}}%
\pgfpathlineto{\pgfqpoint{1.753377in}{0.462714in}}%
\pgfpathlineto{\pgfqpoint{1.753377in}{0.455575in}}%
\pgfpathlineto{\pgfqpoint{1.791271in}{0.455575in}}%
\pgfpathlineto{\pgfqpoint{1.791271in}{0.455205in}}%
\pgfpathlineto{\pgfqpoint{1.829166in}{0.455205in}}%
\pgfpathlineto{\pgfqpoint{1.829166in}{0.453257in}}%
\pgfpathlineto{\pgfqpoint{1.867060in}{0.453257in}}%
\pgfpathlineto{\pgfqpoint{1.867060in}{0.452507in}}%
\pgfpathlineto{\pgfqpoint{1.904954in}{0.452507in}}%
\pgfpathlineto{\pgfqpoint{1.904954in}{0.452681in}}%
\pgfpathlineto{\pgfqpoint{1.942848in}{0.452681in}}%
\pgfpathlineto{\pgfqpoint{1.942848in}{0.451445in}}%
\pgfpathlineto{\pgfqpoint{1.980742in}{0.451445in}}%
\pgfpathlineto{\pgfqpoint{1.980742in}{0.450781in}}%
\pgfpathlineto{\pgfqpoint{2.018637in}{0.450781in}}%
\pgfpathlineto{\pgfqpoint{2.018637in}{0.450287in}}%
\pgfpathlineto{\pgfqpoint{2.056531in}{0.450287in}}%
\pgfpathlineto{\pgfqpoint{2.056531in}{0.450006in}}%
\pgfpathlineto{\pgfqpoint{2.094425in}{0.450006in}}%
\pgfpathlineto{\pgfqpoint{2.094425in}{0.450090in}}%
\pgfpathlineto{\pgfqpoint{2.132319in}{0.450090in}}%
\pgfpathlineto{\pgfqpoint{2.132319in}{0.450169in}}%
\pgfpathlineto{\pgfqpoint{2.170214in}{0.450169in}}%
\pgfpathlineto{\pgfqpoint{2.170214in}{0.450083in}}%
\pgfpathlineto{\pgfqpoint{2.208108in}{0.450083in}}%
\pgfpathlineto{\pgfqpoint{2.208108in}{0.450019in}}%
\pgfpathlineto{\pgfqpoint{2.246002in}{0.450019in}}%
\pgfpathlineto{\pgfqpoint{2.246002in}{0.450076in}}%
\pgfpathlineto{\pgfqpoint{2.283896in}{0.450076in}}%
\pgfpathlineto{\pgfqpoint{2.283896in}{0.449996in}}%
\pgfpathlineto{\pgfqpoint{2.321791in}{0.449996in}}%
\pgfpathlineto{\pgfqpoint{2.321791in}{0.450019in}}%
\pgfpathlineto{\pgfqpoint{2.359685in}{0.450019in}}%
\pgfpathlineto{\pgfqpoint{2.359685in}{0.449983in}}%
\pgfpathlineto{\pgfqpoint{2.397579in}{0.449983in}}%
\pgfpathlineto{\pgfqpoint{2.397579in}{0.449983in}}%
\pgfusepath{stroke}%
\end{pgfscope}%
\begin{pgfscope}%
\pgfsetrectcap%
\pgfsetmiterjoin%
\pgfsetlinewidth{1.003750pt}%
\definecolor{currentstroke}{rgb}{0.000000,0.000000,0.000000}%
\pgfsetstrokecolor{currentstroke}%
\pgfsetdash{}{0pt}%
\pgfpathmoveto{\pgfqpoint{0.437532in}{1.615583in}}%
\pgfpathlineto{\pgfqpoint{2.528249in}{1.615583in}}%
\pgfusepath{stroke}%
\end{pgfscope}%
\begin{pgfscope}%
\pgfsetrectcap%
\pgfsetmiterjoin%
\pgfsetlinewidth{1.003750pt}%
\definecolor{currentstroke}{rgb}{0.000000,0.000000,0.000000}%
\pgfsetstrokecolor{currentstroke}%
\pgfsetdash{}{0pt}%
\pgfpathmoveto{\pgfqpoint{2.528249in}{0.449983in}}%
\pgfpathlineto{\pgfqpoint{2.528249in}{1.615583in}}%
\pgfusepath{stroke}%
\end{pgfscope}%
\begin{pgfscope}%
\pgfsetrectcap%
\pgfsetmiterjoin%
\pgfsetlinewidth{1.003750pt}%
\definecolor{currentstroke}{rgb}{0.000000,0.000000,0.000000}%
\pgfsetstrokecolor{currentstroke}%
\pgfsetdash{}{0pt}%
\pgfpathmoveto{\pgfqpoint{0.437532in}{0.449983in}}%
\pgfpathlineto{\pgfqpoint{2.528249in}{0.449983in}}%
\pgfusepath{stroke}%
\end{pgfscope}%
\begin{pgfscope}%
\pgfsetrectcap%
\pgfsetmiterjoin%
\pgfsetlinewidth{1.003750pt}%
\definecolor{currentstroke}{rgb}{0.000000,0.000000,0.000000}%
\pgfsetstrokecolor{currentstroke}%
\pgfsetdash{}{0pt}%
\pgfpathmoveto{\pgfqpoint{0.437532in}{0.449983in}}%
\pgfpathlineto{\pgfqpoint{0.437532in}{1.615583in}}%
\pgfusepath{stroke}%
\end{pgfscope}%
\begin{pgfscope}%
\pgfsetbuttcap%
\pgfsetroundjoin%
\definecolor{currentfill}{rgb}{0.000000,0.000000,0.000000}%
\pgfsetfillcolor{currentfill}%
\pgfsetlinewidth{0.501875pt}%
\definecolor{currentstroke}{rgb}{0.000000,0.000000,0.000000}%
\pgfsetstrokecolor{currentstroke}%
\pgfsetdash{}{0pt}%
\pgfsys@defobject{currentmarker}{\pgfqpoint{0.000000in}{0.000000in}}{\pgfqpoint{0.000000in}{0.069444in}}{%
\pgfpathmoveto{\pgfqpoint{0.000000in}{0.000000in}}%
\pgfpathlineto{\pgfqpoint{0.000000in}{0.069444in}}%
\pgfusepath{stroke,fill}%
}%
\begin{pgfscope}%
\pgfsys@transformshift{0.437532in}{0.449983in}%
\pgfsys@useobject{currentmarker}{}%
\end{pgfscope}%
\end{pgfscope}%
\begin{pgfscope}%
\pgfsetbuttcap%
\pgfsetroundjoin%
\definecolor{currentfill}{rgb}{0.000000,0.000000,0.000000}%
\pgfsetfillcolor{currentfill}%
\pgfsetlinewidth{0.501875pt}%
\definecolor{currentstroke}{rgb}{0.000000,0.000000,0.000000}%
\pgfsetstrokecolor{currentstroke}%
\pgfsetdash{}{0pt}%
\pgfsys@defobject{currentmarker}{\pgfqpoint{0.000000in}{-0.069444in}}{\pgfqpoint{0.000000in}{0.000000in}}{%
\pgfpathmoveto{\pgfqpoint{0.000000in}{0.000000in}}%
\pgfpathlineto{\pgfqpoint{0.000000in}{-0.069444in}}%
\pgfusepath{stroke,fill}%
}%
\begin{pgfscope}%
\pgfsys@transformshift{0.437532in}{1.615583in}%
\pgfsys@useobject{currentmarker}{}%
\end{pgfscope}%
\end{pgfscope}%
\begin{pgfscope}%
\pgftext[x=0.437532in,y=0.380539in,,top]{\rmfamily\fontsize{8.000000}{9.600000}\selectfont −20}%
\end{pgfscope}%
\begin{pgfscope}%
\pgfsetbuttcap%
\pgfsetroundjoin%
\definecolor{currentfill}{rgb}{0.000000,0.000000,0.000000}%
\pgfsetfillcolor{currentfill}%
\pgfsetlinewidth{0.501875pt}%
\definecolor{currentstroke}{rgb}{0.000000,0.000000,0.000000}%
\pgfsetstrokecolor{currentstroke}%
\pgfsetdash{}{0pt}%
\pgfsys@defobject{currentmarker}{\pgfqpoint{0.000000in}{0.000000in}}{\pgfqpoint{0.000000in}{0.069444in}}{%
\pgfpathmoveto{\pgfqpoint{0.000000in}{0.000000in}}%
\pgfpathlineto{\pgfqpoint{0.000000in}{0.069444in}}%
\pgfusepath{stroke,fill}%
}%
\begin{pgfscope}%
\pgfsys@transformshift{0.698872in}{0.449983in}%
\pgfsys@useobject{currentmarker}{}%
\end{pgfscope}%
\end{pgfscope}%
\begin{pgfscope}%
\pgfsetbuttcap%
\pgfsetroundjoin%
\definecolor{currentfill}{rgb}{0.000000,0.000000,0.000000}%
\pgfsetfillcolor{currentfill}%
\pgfsetlinewidth{0.501875pt}%
\definecolor{currentstroke}{rgb}{0.000000,0.000000,0.000000}%
\pgfsetstrokecolor{currentstroke}%
\pgfsetdash{}{0pt}%
\pgfsys@defobject{currentmarker}{\pgfqpoint{0.000000in}{-0.069444in}}{\pgfqpoint{0.000000in}{0.000000in}}{%
\pgfpathmoveto{\pgfqpoint{0.000000in}{0.000000in}}%
\pgfpathlineto{\pgfqpoint{0.000000in}{-0.069444in}}%
\pgfusepath{stroke,fill}%
}%
\begin{pgfscope}%
\pgfsys@transformshift{0.698872in}{1.615583in}%
\pgfsys@useobject{currentmarker}{}%
\end{pgfscope}%
\end{pgfscope}%
\begin{pgfscope}%
\pgftext[x=0.698872in,y=0.380539in,,top]{\rmfamily\fontsize{8.000000}{9.600000}\selectfont 0}%
\end{pgfscope}%
\begin{pgfscope}%
\pgfsetbuttcap%
\pgfsetroundjoin%
\definecolor{currentfill}{rgb}{0.000000,0.000000,0.000000}%
\pgfsetfillcolor{currentfill}%
\pgfsetlinewidth{0.501875pt}%
\definecolor{currentstroke}{rgb}{0.000000,0.000000,0.000000}%
\pgfsetstrokecolor{currentstroke}%
\pgfsetdash{}{0pt}%
\pgfsys@defobject{currentmarker}{\pgfqpoint{0.000000in}{0.000000in}}{\pgfqpoint{0.000000in}{0.069444in}}{%
\pgfpathmoveto{\pgfqpoint{0.000000in}{0.000000in}}%
\pgfpathlineto{\pgfqpoint{0.000000in}{0.069444in}}%
\pgfusepath{stroke,fill}%
}%
\begin{pgfscope}%
\pgfsys@transformshift{0.960212in}{0.449983in}%
\pgfsys@useobject{currentmarker}{}%
\end{pgfscope}%
\end{pgfscope}%
\begin{pgfscope}%
\pgfsetbuttcap%
\pgfsetroundjoin%
\definecolor{currentfill}{rgb}{0.000000,0.000000,0.000000}%
\pgfsetfillcolor{currentfill}%
\pgfsetlinewidth{0.501875pt}%
\definecolor{currentstroke}{rgb}{0.000000,0.000000,0.000000}%
\pgfsetstrokecolor{currentstroke}%
\pgfsetdash{}{0pt}%
\pgfsys@defobject{currentmarker}{\pgfqpoint{0.000000in}{-0.069444in}}{\pgfqpoint{0.000000in}{0.000000in}}{%
\pgfpathmoveto{\pgfqpoint{0.000000in}{0.000000in}}%
\pgfpathlineto{\pgfqpoint{0.000000in}{-0.069444in}}%
\pgfusepath{stroke,fill}%
}%
\begin{pgfscope}%
\pgfsys@transformshift{0.960212in}{1.615583in}%
\pgfsys@useobject{currentmarker}{}%
\end{pgfscope}%
\end{pgfscope}%
\begin{pgfscope}%
\pgftext[x=0.960212in,y=0.380539in,,top]{\rmfamily\fontsize{8.000000}{9.600000}\selectfont 20}%
\end{pgfscope}%
\begin{pgfscope}%
\pgfsetbuttcap%
\pgfsetroundjoin%
\definecolor{currentfill}{rgb}{0.000000,0.000000,0.000000}%
\pgfsetfillcolor{currentfill}%
\pgfsetlinewidth{0.501875pt}%
\definecolor{currentstroke}{rgb}{0.000000,0.000000,0.000000}%
\pgfsetstrokecolor{currentstroke}%
\pgfsetdash{}{0pt}%
\pgfsys@defobject{currentmarker}{\pgfqpoint{0.000000in}{0.000000in}}{\pgfqpoint{0.000000in}{0.069444in}}{%
\pgfpathmoveto{\pgfqpoint{0.000000in}{0.000000in}}%
\pgfpathlineto{\pgfqpoint{0.000000in}{0.069444in}}%
\pgfusepath{stroke,fill}%
}%
\begin{pgfscope}%
\pgfsys@transformshift{1.221551in}{0.449983in}%
\pgfsys@useobject{currentmarker}{}%
\end{pgfscope}%
\end{pgfscope}%
\begin{pgfscope}%
\pgfsetbuttcap%
\pgfsetroundjoin%
\definecolor{currentfill}{rgb}{0.000000,0.000000,0.000000}%
\pgfsetfillcolor{currentfill}%
\pgfsetlinewidth{0.501875pt}%
\definecolor{currentstroke}{rgb}{0.000000,0.000000,0.000000}%
\pgfsetstrokecolor{currentstroke}%
\pgfsetdash{}{0pt}%
\pgfsys@defobject{currentmarker}{\pgfqpoint{0.000000in}{-0.069444in}}{\pgfqpoint{0.000000in}{0.000000in}}{%
\pgfpathmoveto{\pgfqpoint{0.000000in}{0.000000in}}%
\pgfpathlineto{\pgfqpoint{0.000000in}{-0.069444in}}%
\pgfusepath{stroke,fill}%
}%
\begin{pgfscope}%
\pgfsys@transformshift{1.221551in}{1.615583in}%
\pgfsys@useobject{currentmarker}{}%
\end{pgfscope}%
\end{pgfscope}%
\begin{pgfscope}%
\pgftext[x=1.221551in,y=0.380539in,,top]{\rmfamily\fontsize{8.000000}{9.600000}\selectfont 40}%
\end{pgfscope}%
\begin{pgfscope}%
\pgfsetbuttcap%
\pgfsetroundjoin%
\definecolor{currentfill}{rgb}{0.000000,0.000000,0.000000}%
\pgfsetfillcolor{currentfill}%
\pgfsetlinewidth{0.501875pt}%
\definecolor{currentstroke}{rgb}{0.000000,0.000000,0.000000}%
\pgfsetstrokecolor{currentstroke}%
\pgfsetdash{}{0pt}%
\pgfsys@defobject{currentmarker}{\pgfqpoint{0.000000in}{0.000000in}}{\pgfqpoint{0.000000in}{0.069444in}}{%
\pgfpathmoveto{\pgfqpoint{0.000000in}{0.000000in}}%
\pgfpathlineto{\pgfqpoint{0.000000in}{0.069444in}}%
\pgfusepath{stroke,fill}%
}%
\begin{pgfscope}%
\pgfsys@transformshift{1.482891in}{0.449983in}%
\pgfsys@useobject{currentmarker}{}%
\end{pgfscope}%
\end{pgfscope}%
\begin{pgfscope}%
\pgfsetbuttcap%
\pgfsetroundjoin%
\definecolor{currentfill}{rgb}{0.000000,0.000000,0.000000}%
\pgfsetfillcolor{currentfill}%
\pgfsetlinewidth{0.501875pt}%
\definecolor{currentstroke}{rgb}{0.000000,0.000000,0.000000}%
\pgfsetstrokecolor{currentstroke}%
\pgfsetdash{}{0pt}%
\pgfsys@defobject{currentmarker}{\pgfqpoint{0.000000in}{-0.069444in}}{\pgfqpoint{0.000000in}{0.000000in}}{%
\pgfpathmoveto{\pgfqpoint{0.000000in}{0.000000in}}%
\pgfpathlineto{\pgfqpoint{0.000000in}{-0.069444in}}%
\pgfusepath{stroke,fill}%
}%
\begin{pgfscope}%
\pgfsys@transformshift{1.482891in}{1.615583in}%
\pgfsys@useobject{currentmarker}{}%
\end{pgfscope}%
\end{pgfscope}%
\begin{pgfscope}%
\pgftext[x=1.482891in,y=0.380539in,,top]{\rmfamily\fontsize{8.000000}{9.600000}\selectfont 60}%
\end{pgfscope}%
\begin{pgfscope}%
\pgfsetbuttcap%
\pgfsetroundjoin%
\definecolor{currentfill}{rgb}{0.000000,0.000000,0.000000}%
\pgfsetfillcolor{currentfill}%
\pgfsetlinewidth{0.501875pt}%
\definecolor{currentstroke}{rgb}{0.000000,0.000000,0.000000}%
\pgfsetstrokecolor{currentstroke}%
\pgfsetdash{}{0pt}%
\pgfsys@defobject{currentmarker}{\pgfqpoint{0.000000in}{0.000000in}}{\pgfqpoint{0.000000in}{0.069444in}}{%
\pgfpathmoveto{\pgfqpoint{0.000000in}{0.000000in}}%
\pgfpathlineto{\pgfqpoint{0.000000in}{0.069444in}}%
\pgfusepath{stroke,fill}%
}%
\begin{pgfscope}%
\pgfsys@transformshift{1.744230in}{0.449983in}%
\pgfsys@useobject{currentmarker}{}%
\end{pgfscope}%
\end{pgfscope}%
\begin{pgfscope}%
\pgfsetbuttcap%
\pgfsetroundjoin%
\definecolor{currentfill}{rgb}{0.000000,0.000000,0.000000}%
\pgfsetfillcolor{currentfill}%
\pgfsetlinewidth{0.501875pt}%
\definecolor{currentstroke}{rgb}{0.000000,0.000000,0.000000}%
\pgfsetstrokecolor{currentstroke}%
\pgfsetdash{}{0pt}%
\pgfsys@defobject{currentmarker}{\pgfqpoint{0.000000in}{-0.069444in}}{\pgfqpoint{0.000000in}{0.000000in}}{%
\pgfpathmoveto{\pgfqpoint{0.000000in}{0.000000in}}%
\pgfpathlineto{\pgfqpoint{0.000000in}{-0.069444in}}%
\pgfusepath{stroke,fill}%
}%
\begin{pgfscope}%
\pgfsys@transformshift{1.744230in}{1.615583in}%
\pgfsys@useobject{currentmarker}{}%
\end{pgfscope}%
\end{pgfscope}%
\begin{pgfscope}%
\pgftext[x=1.744230in,y=0.380539in,,top]{\rmfamily\fontsize{8.000000}{9.600000}\selectfont 80}%
\end{pgfscope}%
\begin{pgfscope}%
\pgfsetbuttcap%
\pgfsetroundjoin%
\definecolor{currentfill}{rgb}{0.000000,0.000000,0.000000}%
\pgfsetfillcolor{currentfill}%
\pgfsetlinewidth{0.501875pt}%
\definecolor{currentstroke}{rgb}{0.000000,0.000000,0.000000}%
\pgfsetstrokecolor{currentstroke}%
\pgfsetdash{}{0pt}%
\pgfsys@defobject{currentmarker}{\pgfqpoint{0.000000in}{0.000000in}}{\pgfqpoint{0.000000in}{0.069444in}}{%
\pgfpathmoveto{\pgfqpoint{0.000000in}{0.000000in}}%
\pgfpathlineto{\pgfqpoint{0.000000in}{0.069444in}}%
\pgfusepath{stroke,fill}%
}%
\begin{pgfscope}%
\pgfsys@transformshift{2.005570in}{0.449983in}%
\pgfsys@useobject{currentmarker}{}%
\end{pgfscope}%
\end{pgfscope}%
\begin{pgfscope}%
\pgfsetbuttcap%
\pgfsetroundjoin%
\definecolor{currentfill}{rgb}{0.000000,0.000000,0.000000}%
\pgfsetfillcolor{currentfill}%
\pgfsetlinewidth{0.501875pt}%
\definecolor{currentstroke}{rgb}{0.000000,0.000000,0.000000}%
\pgfsetstrokecolor{currentstroke}%
\pgfsetdash{}{0pt}%
\pgfsys@defobject{currentmarker}{\pgfqpoint{0.000000in}{-0.069444in}}{\pgfqpoint{0.000000in}{0.000000in}}{%
\pgfpathmoveto{\pgfqpoint{0.000000in}{0.000000in}}%
\pgfpathlineto{\pgfqpoint{0.000000in}{-0.069444in}}%
\pgfusepath{stroke,fill}%
}%
\begin{pgfscope}%
\pgfsys@transformshift{2.005570in}{1.615583in}%
\pgfsys@useobject{currentmarker}{}%
\end{pgfscope}%
\end{pgfscope}%
\begin{pgfscope}%
\pgftext[x=2.005570in,y=0.380539in,,top]{\rmfamily\fontsize{8.000000}{9.600000}\selectfont 100}%
\end{pgfscope}%
\begin{pgfscope}%
\pgfsetbuttcap%
\pgfsetroundjoin%
\definecolor{currentfill}{rgb}{0.000000,0.000000,0.000000}%
\pgfsetfillcolor{currentfill}%
\pgfsetlinewidth{0.501875pt}%
\definecolor{currentstroke}{rgb}{0.000000,0.000000,0.000000}%
\pgfsetstrokecolor{currentstroke}%
\pgfsetdash{}{0pt}%
\pgfsys@defobject{currentmarker}{\pgfqpoint{0.000000in}{0.000000in}}{\pgfqpoint{0.000000in}{0.069444in}}{%
\pgfpathmoveto{\pgfqpoint{0.000000in}{0.000000in}}%
\pgfpathlineto{\pgfqpoint{0.000000in}{0.069444in}}%
\pgfusepath{stroke,fill}%
}%
\begin{pgfscope}%
\pgfsys@transformshift{2.266909in}{0.449983in}%
\pgfsys@useobject{currentmarker}{}%
\end{pgfscope}%
\end{pgfscope}%
\begin{pgfscope}%
\pgfsetbuttcap%
\pgfsetroundjoin%
\definecolor{currentfill}{rgb}{0.000000,0.000000,0.000000}%
\pgfsetfillcolor{currentfill}%
\pgfsetlinewidth{0.501875pt}%
\definecolor{currentstroke}{rgb}{0.000000,0.000000,0.000000}%
\pgfsetstrokecolor{currentstroke}%
\pgfsetdash{}{0pt}%
\pgfsys@defobject{currentmarker}{\pgfqpoint{0.000000in}{-0.069444in}}{\pgfqpoint{0.000000in}{0.000000in}}{%
\pgfpathmoveto{\pgfqpoint{0.000000in}{0.000000in}}%
\pgfpathlineto{\pgfqpoint{0.000000in}{-0.069444in}}%
\pgfusepath{stroke,fill}%
}%
\begin{pgfscope}%
\pgfsys@transformshift{2.266909in}{1.615583in}%
\pgfsys@useobject{currentmarker}{}%
\end{pgfscope}%
\end{pgfscope}%
\begin{pgfscope}%
\pgftext[x=2.266909in,y=0.380539in,,top]{\rmfamily\fontsize{8.000000}{9.600000}\selectfont 120}%
\end{pgfscope}%
\begin{pgfscope}%
\pgfsetbuttcap%
\pgfsetroundjoin%
\definecolor{currentfill}{rgb}{0.000000,0.000000,0.000000}%
\pgfsetfillcolor{currentfill}%
\pgfsetlinewidth{0.501875pt}%
\definecolor{currentstroke}{rgb}{0.000000,0.000000,0.000000}%
\pgfsetstrokecolor{currentstroke}%
\pgfsetdash{}{0pt}%
\pgfsys@defobject{currentmarker}{\pgfqpoint{0.000000in}{0.000000in}}{\pgfqpoint{0.000000in}{0.069444in}}{%
\pgfpathmoveto{\pgfqpoint{0.000000in}{0.000000in}}%
\pgfpathlineto{\pgfqpoint{0.000000in}{0.069444in}}%
\pgfusepath{stroke,fill}%
}%
\begin{pgfscope}%
\pgfsys@transformshift{2.528249in}{0.449983in}%
\pgfsys@useobject{currentmarker}{}%
\end{pgfscope}%
\end{pgfscope}%
\begin{pgfscope}%
\pgfsetbuttcap%
\pgfsetroundjoin%
\definecolor{currentfill}{rgb}{0.000000,0.000000,0.000000}%
\pgfsetfillcolor{currentfill}%
\pgfsetlinewidth{0.501875pt}%
\definecolor{currentstroke}{rgb}{0.000000,0.000000,0.000000}%
\pgfsetstrokecolor{currentstroke}%
\pgfsetdash{}{0pt}%
\pgfsys@defobject{currentmarker}{\pgfqpoint{0.000000in}{-0.069444in}}{\pgfqpoint{0.000000in}{0.000000in}}{%
\pgfpathmoveto{\pgfqpoint{0.000000in}{0.000000in}}%
\pgfpathlineto{\pgfqpoint{0.000000in}{-0.069444in}}%
\pgfusepath{stroke,fill}%
}%
\begin{pgfscope}%
\pgfsys@transformshift{2.528249in}{1.615583in}%
\pgfsys@useobject{currentmarker}{}%
\end{pgfscope}%
\end{pgfscope}%
\begin{pgfscope}%
\pgftext[x=2.528249in,y=0.380539in,,top]{\rmfamily\fontsize{8.000000}{9.600000}\selectfont 140}%
\end{pgfscope}%
\begin{pgfscope}%
\pgftext[x=1.482891in,y=0.203564in,,top]{\rmfamily\fontsize{9.000000}{10.800000}\selectfont \(\displaystyle \mathrm{DLL}_{K/\pi}(K^+)\)}%
\end{pgfscope}%
\begin{pgfscope}%
\pgfsetbuttcap%
\pgfsetroundjoin%
\definecolor{currentfill}{rgb}{0.000000,0.000000,0.000000}%
\pgfsetfillcolor{currentfill}%
\pgfsetlinewidth{0.501875pt}%
\definecolor{currentstroke}{rgb}{0.000000,0.000000,0.000000}%
\pgfsetstrokecolor{currentstroke}%
\pgfsetdash{}{0pt}%
\pgfsys@defobject{currentmarker}{\pgfqpoint{0.000000in}{0.000000in}}{\pgfqpoint{0.069444in}{0.000000in}}{%
\pgfpathmoveto{\pgfqpoint{0.000000in}{0.000000in}}%
\pgfpathlineto{\pgfqpoint{0.069444in}{0.000000in}}%
\pgfusepath{stroke,fill}%
}%
\begin{pgfscope}%
\pgfsys@transformshift{0.437532in}{0.449983in}%
\pgfsys@useobject{currentmarker}{}%
\end{pgfscope}%
\end{pgfscope}%
\begin{pgfscope}%
\pgfsetbuttcap%
\pgfsetroundjoin%
\definecolor{currentfill}{rgb}{0.000000,0.000000,0.000000}%
\pgfsetfillcolor{currentfill}%
\pgfsetlinewidth{0.501875pt}%
\definecolor{currentstroke}{rgb}{0.000000,0.000000,0.000000}%
\pgfsetstrokecolor{currentstroke}%
\pgfsetdash{}{0pt}%
\pgfsys@defobject{currentmarker}{\pgfqpoint{-0.069444in}{0.000000in}}{\pgfqpoint{0.000000in}{0.000000in}}{%
\pgfpathmoveto{\pgfqpoint{0.000000in}{0.000000in}}%
\pgfpathlineto{\pgfqpoint{-0.069444in}{0.000000in}}%
\pgfusepath{stroke,fill}%
}%
\begin{pgfscope}%
\pgfsys@transformshift{2.528249in}{0.449983in}%
\pgfsys@useobject{currentmarker}{}%
\end{pgfscope}%
\end{pgfscope}%
\begin{pgfscope}%
\pgftext[x=0.368088in,y=0.449983in,right,]{\rmfamily\fontsize{8.000000}{9.600000}\selectfont 0.000}%
\end{pgfscope}%
\begin{pgfscope}%
\pgfsetbuttcap%
\pgfsetroundjoin%
\definecolor{currentfill}{rgb}{0.000000,0.000000,0.000000}%
\pgfsetfillcolor{currentfill}%
\pgfsetlinewidth{0.501875pt}%
\definecolor{currentstroke}{rgb}{0.000000,0.000000,0.000000}%
\pgfsetstrokecolor{currentstroke}%
\pgfsetdash{}{0pt}%
\pgfsys@defobject{currentmarker}{\pgfqpoint{0.000000in}{0.000000in}}{\pgfqpoint{0.069444in}{0.000000in}}{%
\pgfpathmoveto{\pgfqpoint{0.000000in}{0.000000in}}%
\pgfpathlineto{\pgfqpoint{0.069444in}{0.000000in}}%
\pgfusepath{stroke,fill}%
}%
\begin{pgfscope}%
\pgfsys@transformshift{0.437532in}{0.644250in}%
\pgfsys@useobject{currentmarker}{}%
\end{pgfscope}%
\end{pgfscope}%
\begin{pgfscope}%
\pgfsetbuttcap%
\pgfsetroundjoin%
\definecolor{currentfill}{rgb}{0.000000,0.000000,0.000000}%
\pgfsetfillcolor{currentfill}%
\pgfsetlinewidth{0.501875pt}%
\definecolor{currentstroke}{rgb}{0.000000,0.000000,0.000000}%
\pgfsetstrokecolor{currentstroke}%
\pgfsetdash{}{0pt}%
\pgfsys@defobject{currentmarker}{\pgfqpoint{-0.069444in}{0.000000in}}{\pgfqpoint{0.000000in}{0.000000in}}{%
\pgfpathmoveto{\pgfqpoint{0.000000in}{0.000000in}}%
\pgfpathlineto{\pgfqpoint{-0.069444in}{0.000000in}}%
\pgfusepath{stroke,fill}%
}%
\begin{pgfscope}%
\pgfsys@transformshift{2.528249in}{0.644250in}%
\pgfsys@useobject{currentmarker}{}%
\end{pgfscope}%
\end{pgfscope}%
\begin{pgfscope}%
\pgftext[x=0.368088in,y=0.644250in,right,]{\rmfamily\fontsize{8.000000}{9.600000}\selectfont 0.005}%
\end{pgfscope}%
\begin{pgfscope}%
\pgfsetbuttcap%
\pgfsetroundjoin%
\definecolor{currentfill}{rgb}{0.000000,0.000000,0.000000}%
\pgfsetfillcolor{currentfill}%
\pgfsetlinewidth{0.501875pt}%
\definecolor{currentstroke}{rgb}{0.000000,0.000000,0.000000}%
\pgfsetstrokecolor{currentstroke}%
\pgfsetdash{}{0pt}%
\pgfsys@defobject{currentmarker}{\pgfqpoint{0.000000in}{0.000000in}}{\pgfqpoint{0.069444in}{0.000000in}}{%
\pgfpathmoveto{\pgfqpoint{0.000000in}{0.000000in}}%
\pgfpathlineto{\pgfqpoint{0.069444in}{0.000000in}}%
\pgfusepath{stroke,fill}%
}%
\begin{pgfscope}%
\pgfsys@transformshift{0.437532in}{0.838517in}%
\pgfsys@useobject{currentmarker}{}%
\end{pgfscope}%
\end{pgfscope}%
\begin{pgfscope}%
\pgfsetbuttcap%
\pgfsetroundjoin%
\definecolor{currentfill}{rgb}{0.000000,0.000000,0.000000}%
\pgfsetfillcolor{currentfill}%
\pgfsetlinewidth{0.501875pt}%
\definecolor{currentstroke}{rgb}{0.000000,0.000000,0.000000}%
\pgfsetstrokecolor{currentstroke}%
\pgfsetdash{}{0pt}%
\pgfsys@defobject{currentmarker}{\pgfqpoint{-0.069444in}{0.000000in}}{\pgfqpoint{0.000000in}{0.000000in}}{%
\pgfpathmoveto{\pgfqpoint{0.000000in}{0.000000in}}%
\pgfpathlineto{\pgfqpoint{-0.069444in}{0.000000in}}%
\pgfusepath{stroke,fill}%
}%
\begin{pgfscope}%
\pgfsys@transformshift{2.528249in}{0.838517in}%
\pgfsys@useobject{currentmarker}{}%
\end{pgfscope}%
\end{pgfscope}%
\begin{pgfscope}%
\pgftext[x=0.368088in,y=0.838517in,right,]{\rmfamily\fontsize{8.000000}{9.600000}\selectfont 0.010}%
\end{pgfscope}%
\begin{pgfscope}%
\pgfsetbuttcap%
\pgfsetroundjoin%
\definecolor{currentfill}{rgb}{0.000000,0.000000,0.000000}%
\pgfsetfillcolor{currentfill}%
\pgfsetlinewidth{0.501875pt}%
\definecolor{currentstroke}{rgb}{0.000000,0.000000,0.000000}%
\pgfsetstrokecolor{currentstroke}%
\pgfsetdash{}{0pt}%
\pgfsys@defobject{currentmarker}{\pgfqpoint{0.000000in}{0.000000in}}{\pgfqpoint{0.069444in}{0.000000in}}{%
\pgfpathmoveto{\pgfqpoint{0.000000in}{0.000000in}}%
\pgfpathlineto{\pgfqpoint{0.069444in}{0.000000in}}%
\pgfusepath{stroke,fill}%
}%
\begin{pgfscope}%
\pgfsys@transformshift{0.437532in}{1.032783in}%
\pgfsys@useobject{currentmarker}{}%
\end{pgfscope}%
\end{pgfscope}%
\begin{pgfscope}%
\pgfsetbuttcap%
\pgfsetroundjoin%
\definecolor{currentfill}{rgb}{0.000000,0.000000,0.000000}%
\pgfsetfillcolor{currentfill}%
\pgfsetlinewidth{0.501875pt}%
\definecolor{currentstroke}{rgb}{0.000000,0.000000,0.000000}%
\pgfsetstrokecolor{currentstroke}%
\pgfsetdash{}{0pt}%
\pgfsys@defobject{currentmarker}{\pgfqpoint{-0.069444in}{0.000000in}}{\pgfqpoint{0.000000in}{0.000000in}}{%
\pgfpathmoveto{\pgfqpoint{0.000000in}{0.000000in}}%
\pgfpathlineto{\pgfqpoint{-0.069444in}{0.000000in}}%
\pgfusepath{stroke,fill}%
}%
\begin{pgfscope}%
\pgfsys@transformshift{2.528249in}{1.032783in}%
\pgfsys@useobject{currentmarker}{}%
\end{pgfscope}%
\end{pgfscope}%
\begin{pgfscope}%
\pgftext[x=0.368088in,y=1.032783in,right,]{\rmfamily\fontsize{8.000000}{9.600000}\selectfont 0.015}%
\end{pgfscope}%
\begin{pgfscope}%
\pgfsetbuttcap%
\pgfsetroundjoin%
\definecolor{currentfill}{rgb}{0.000000,0.000000,0.000000}%
\pgfsetfillcolor{currentfill}%
\pgfsetlinewidth{0.501875pt}%
\definecolor{currentstroke}{rgb}{0.000000,0.000000,0.000000}%
\pgfsetstrokecolor{currentstroke}%
\pgfsetdash{}{0pt}%
\pgfsys@defobject{currentmarker}{\pgfqpoint{0.000000in}{0.000000in}}{\pgfqpoint{0.069444in}{0.000000in}}{%
\pgfpathmoveto{\pgfqpoint{0.000000in}{0.000000in}}%
\pgfpathlineto{\pgfqpoint{0.069444in}{0.000000in}}%
\pgfusepath{stroke,fill}%
}%
\begin{pgfscope}%
\pgfsys@transformshift{0.437532in}{1.227050in}%
\pgfsys@useobject{currentmarker}{}%
\end{pgfscope}%
\end{pgfscope}%
\begin{pgfscope}%
\pgfsetbuttcap%
\pgfsetroundjoin%
\definecolor{currentfill}{rgb}{0.000000,0.000000,0.000000}%
\pgfsetfillcolor{currentfill}%
\pgfsetlinewidth{0.501875pt}%
\definecolor{currentstroke}{rgb}{0.000000,0.000000,0.000000}%
\pgfsetstrokecolor{currentstroke}%
\pgfsetdash{}{0pt}%
\pgfsys@defobject{currentmarker}{\pgfqpoint{-0.069444in}{0.000000in}}{\pgfqpoint{0.000000in}{0.000000in}}{%
\pgfpathmoveto{\pgfqpoint{0.000000in}{0.000000in}}%
\pgfpathlineto{\pgfqpoint{-0.069444in}{0.000000in}}%
\pgfusepath{stroke,fill}%
}%
\begin{pgfscope}%
\pgfsys@transformshift{2.528249in}{1.227050in}%
\pgfsys@useobject{currentmarker}{}%
\end{pgfscope}%
\end{pgfscope}%
\begin{pgfscope}%
\pgftext[x=0.368088in,y=1.227050in,right,]{\rmfamily\fontsize{8.000000}{9.600000}\selectfont 0.020}%
\end{pgfscope}%
\begin{pgfscope}%
\pgfsetbuttcap%
\pgfsetroundjoin%
\definecolor{currentfill}{rgb}{0.000000,0.000000,0.000000}%
\pgfsetfillcolor{currentfill}%
\pgfsetlinewidth{0.501875pt}%
\definecolor{currentstroke}{rgb}{0.000000,0.000000,0.000000}%
\pgfsetstrokecolor{currentstroke}%
\pgfsetdash{}{0pt}%
\pgfsys@defobject{currentmarker}{\pgfqpoint{0.000000in}{0.000000in}}{\pgfqpoint{0.069444in}{0.000000in}}{%
\pgfpathmoveto{\pgfqpoint{0.000000in}{0.000000in}}%
\pgfpathlineto{\pgfqpoint{0.069444in}{0.000000in}}%
\pgfusepath{stroke,fill}%
}%
\begin{pgfscope}%
\pgfsys@transformshift{0.437532in}{1.421317in}%
\pgfsys@useobject{currentmarker}{}%
\end{pgfscope}%
\end{pgfscope}%
\begin{pgfscope}%
\pgfsetbuttcap%
\pgfsetroundjoin%
\definecolor{currentfill}{rgb}{0.000000,0.000000,0.000000}%
\pgfsetfillcolor{currentfill}%
\pgfsetlinewidth{0.501875pt}%
\definecolor{currentstroke}{rgb}{0.000000,0.000000,0.000000}%
\pgfsetstrokecolor{currentstroke}%
\pgfsetdash{}{0pt}%
\pgfsys@defobject{currentmarker}{\pgfqpoint{-0.069444in}{0.000000in}}{\pgfqpoint{0.000000in}{0.000000in}}{%
\pgfpathmoveto{\pgfqpoint{0.000000in}{0.000000in}}%
\pgfpathlineto{\pgfqpoint{-0.069444in}{0.000000in}}%
\pgfusepath{stroke,fill}%
}%
\begin{pgfscope}%
\pgfsys@transformshift{2.528249in}{1.421317in}%
\pgfsys@useobject{currentmarker}{}%
\end{pgfscope}%
\end{pgfscope}%
\begin{pgfscope}%
\pgftext[x=0.368088in,y=1.421317in,right,]{\rmfamily\fontsize{8.000000}{9.600000}\selectfont 0.025}%
\end{pgfscope}%
\begin{pgfscope}%
\pgfsetbuttcap%
\pgfsetroundjoin%
\definecolor{currentfill}{rgb}{0.000000,0.000000,0.000000}%
\pgfsetfillcolor{currentfill}%
\pgfsetlinewidth{0.501875pt}%
\definecolor{currentstroke}{rgb}{0.000000,0.000000,0.000000}%
\pgfsetstrokecolor{currentstroke}%
\pgfsetdash{}{0pt}%
\pgfsys@defobject{currentmarker}{\pgfqpoint{0.000000in}{0.000000in}}{\pgfqpoint{0.069444in}{0.000000in}}{%
\pgfpathmoveto{\pgfqpoint{0.000000in}{0.000000in}}%
\pgfpathlineto{\pgfqpoint{0.069444in}{0.000000in}}%
\pgfusepath{stroke,fill}%
}%
\begin{pgfscope}%
\pgfsys@transformshift{0.437532in}{1.615583in}%
\pgfsys@useobject{currentmarker}{}%
\end{pgfscope}%
\end{pgfscope}%
\begin{pgfscope}%
\pgfsetbuttcap%
\pgfsetroundjoin%
\definecolor{currentfill}{rgb}{0.000000,0.000000,0.000000}%
\pgfsetfillcolor{currentfill}%
\pgfsetlinewidth{0.501875pt}%
\definecolor{currentstroke}{rgb}{0.000000,0.000000,0.000000}%
\pgfsetstrokecolor{currentstroke}%
\pgfsetdash{}{0pt}%
\pgfsys@defobject{currentmarker}{\pgfqpoint{-0.069444in}{0.000000in}}{\pgfqpoint{0.000000in}{0.000000in}}{%
\pgfpathmoveto{\pgfqpoint{0.000000in}{0.000000in}}%
\pgfpathlineto{\pgfqpoint{-0.069444in}{0.000000in}}%
\pgfusepath{stroke,fill}%
}%
\begin{pgfscope}%
\pgfsys@transformshift{2.528249in}{1.615583in}%
\pgfsys@useobject{currentmarker}{}%
\end{pgfscope}%
\end{pgfscope}%
\begin{pgfscope}%
\pgftext[x=0.368088in,y=1.615583in,right,]{\rmfamily\fontsize{8.000000}{9.600000}\selectfont 0.030}%
\end{pgfscope}%
\end{pgfpicture}%
\makeatother%
\endgroup%

	\end{subfigure}
	\begin{subfigure}[t]{0.49\textwidth}
		\centering
    %\includegraphics[width=\textwidth]{store/variables/DATA_MC_Kplus_PIDmu.pdf}
    \input{store/variables/DATA_MC_Kplus_PIDmu.pgf}
	\end{subfigure}

	\begin{subfigure}[t]{0.49\textwidth}
		\centering
    %\includegraphics[width=\textwidth]{store/variables/DATA_MC_piminus_PIDK.pdf}
    %% Creator: Matplotlib, PGF backend
%%
%% To include the figure in your LaTeX document, write
%%   \input{<filename>.pgf}
%%
%% Make sure the required packages are loaded in your preamble
%%   \usepackage{pgf}
%%
%% Figures using additional raster images can only be included by \input if
%% they are in the same directory as the main LaTeX file. For loading figures
%% from other directories you can use the `import` package
%%   \usepackage{import}
%% and then include the figures with
%%   \import{<path to file>}{<filename>.pgf}
%%
%% Matplotlib used the following preamble
%%   \usepackage{fontspec}
%%   \setmainfont{DejaVu Serif}
%%   \setsansfont{DejaVu Sans}
%%   \setmonofont{DejaVu Sans Mono}
%%
\begingroup%
\makeatletter%
\begin{pgfpicture}%
\pgfpathrectangle{\pgfpointorigin}{\pgfqpoint{2.679091in}{1.723197in}}%
\pgfusepath{use as bounding box, clip}%
\begin{pgfscope}%
\pgfsetbuttcap%
\pgfsetmiterjoin%
\definecolor{currentfill}{rgb}{1.000000,1.000000,1.000000}%
\pgfsetfillcolor{currentfill}%
\pgfsetlinewidth{0.000000pt}%
\definecolor{currentstroke}{rgb}{1.000000,1.000000,1.000000}%
\pgfsetstrokecolor{currentstroke}%
\pgfsetdash{}{0pt}%
\pgfpathmoveto{\pgfqpoint{0.000000in}{0.000000in}}%
\pgfpathlineto{\pgfqpoint{2.679091in}{0.000000in}}%
\pgfpathlineto{\pgfqpoint{2.679091in}{1.723197in}}%
\pgfpathlineto{\pgfqpoint{0.000000in}{1.723197in}}%
\pgfpathclose%
\pgfusepath{fill}%
\end{pgfscope}%
\begin{pgfscope}%
\pgfsetbuttcap%
\pgfsetmiterjoin%
\definecolor{currentfill}{rgb}{1.000000,1.000000,1.000000}%
\pgfsetfillcolor{currentfill}%
\pgfsetlinewidth{0.000000pt}%
\definecolor{currentstroke}{rgb}{0.000000,0.000000,0.000000}%
\pgfsetstrokecolor{currentstroke}%
\pgfsetstrokeopacity{0.000000}%
\pgfsetdash{}{0pt}%
\pgfpathmoveto{\pgfqpoint{0.437532in}{0.449983in}}%
\pgfpathlineto{\pgfqpoint{2.558398in}{0.449983in}}%
\pgfpathlineto{\pgfqpoint{2.558398in}{1.619432in}}%
\pgfpathlineto{\pgfqpoint{0.437532in}{1.619432in}}%
\pgfpathclose%
\pgfusepath{fill}%
\end{pgfscope}%
\begin{pgfscope}%
\pgfpathrectangle{\pgfqpoint{0.437532in}{0.449983in}}{\pgfqpoint{2.120866in}{1.169449in}} %
\pgfusepath{clip}%
\pgfsetbuttcap%
\pgfsetmiterjoin%
\definecolor{currentfill}{rgb}{0.215686,0.470588,0.749020}%
\pgfsetfillcolor{currentfill}%
\pgfsetlinewidth{0.000000pt}%
\definecolor{currentstroke}{rgb}{0.000000,0.000000,0.000000}%
\pgfsetstrokecolor{currentstroke}%
\pgfsetdash{}{0pt}%
\pgfpathmoveto{\pgfqpoint{0.555358in}{0.449983in}}%
\pgfpathlineto{\pgfqpoint{0.555358in}{0.450353in}}%
\pgfpathlineto{\pgfqpoint{0.593063in}{0.450353in}}%
\pgfpathlineto{\pgfqpoint{0.593063in}{0.450370in}}%
\pgfpathlineto{\pgfqpoint{0.630767in}{0.450370in}}%
\pgfpathlineto{\pgfqpoint{0.630767in}{0.450488in}}%
\pgfpathlineto{\pgfqpoint{0.668471in}{0.450488in}}%
\pgfpathlineto{\pgfqpoint{0.668471in}{0.450960in}}%
\pgfpathlineto{\pgfqpoint{0.706176in}{0.450960in}}%
\pgfpathlineto{\pgfqpoint{0.706176in}{0.451180in}}%
\pgfpathlineto{\pgfqpoint{0.743880in}{0.451180in}}%
\pgfpathlineto{\pgfqpoint{0.743880in}{0.450840in}}%
\pgfpathlineto{\pgfqpoint{0.781584in}{0.450840in}}%
\pgfpathlineto{\pgfqpoint{0.781584in}{0.451692in}}%
\pgfpathlineto{\pgfqpoint{0.819288in}{0.451692in}}%
\pgfpathlineto{\pgfqpoint{0.819288in}{0.451691in}}%
\pgfpathlineto{\pgfqpoint{0.856993in}{0.451691in}}%
\pgfpathlineto{\pgfqpoint{0.856993in}{0.452964in}}%
\pgfpathlineto{\pgfqpoint{0.894697in}{0.452964in}}%
\pgfpathlineto{\pgfqpoint{0.894697in}{0.453448in}}%
\pgfpathlineto{\pgfqpoint{0.932401in}{0.453448in}}%
\pgfpathlineto{\pgfqpoint{0.932401in}{0.455066in}}%
\pgfpathlineto{\pgfqpoint{0.970105in}{0.455066in}}%
\pgfpathlineto{\pgfqpoint{0.970105in}{0.455766in}}%
\pgfpathlineto{\pgfqpoint{1.007810in}{0.455766in}}%
\pgfpathlineto{\pgfqpoint{1.007810in}{0.457408in}}%
\pgfpathlineto{\pgfqpoint{1.045514in}{0.457408in}}%
\pgfpathlineto{\pgfqpoint{1.045514in}{0.459653in}}%
\pgfpathlineto{\pgfqpoint{1.083218in}{0.459653in}}%
\pgfpathlineto{\pgfqpoint{1.083218in}{0.462845in}}%
\pgfpathlineto{\pgfqpoint{1.120923in}{0.462845in}}%
\pgfpathlineto{\pgfqpoint{1.120923in}{0.465756in}}%
\pgfpathlineto{\pgfqpoint{1.158627in}{0.465756in}}%
\pgfpathlineto{\pgfqpoint{1.158627in}{0.471033in}}%
\pgfpathlineto{\pgfqpoint{1.196331in}{0.471033in}}%
\pgfpathlineto{\pgfqpoint{1.196331in}{0.476335in}}%
\pgfpathlineto{\pgfqpoint{1.234035in}{0.476335in}}%
\pgfpathlineto{\pgfqpoint{1.234035in}{0.483709in}}%
\pgfpathlineto{\pgfqpoint{1.271740in}{0.483709in}}%
\pgfpathlineto{\pgfqpoint{1.271740in}{0.491141in}}%
\pgfpathlineto{\pgfqpoint{1.309444in}{0.491141in}}%
\pgfpathlineto{\pgfqpoint{1.309444in}{0.502084in}}%
\pgfpathlineto{\pgfqpoint{1.347148in}{0.502084in}}%
\pgfpathlineto{\pgfqpoint{1.347148in}{0.514513in}}%
\pgfpathlineto{\pgfqpoint{1.384853in}{0.514513in}}%
\pgfpathlineto{\pgfqpoint{1.384853in}{0.528412in}}%
\pgfpathlineto{\pgfqpoint{1.422557in}{0.528412in}}%
\pgfpathlineto{\pgfqpoint{1.422557in}{0.549502in}}%
\pgfpathlineto{\pgfqpoint{1.460261in}{0.549502in}}%
\pgfpathlineto{\pgfqpoint{1.460261in}{0.569721in}}%
\pgfpathlineto{\pgfqpoint{1.497965in}{0.569721in}}%
\pgfpathlineto{\pgfqpoint{1.497965in}{0.588768in}}%
\pgfpathlineto{\pgfqpoint{1.535670in}{0.588768in}}%
\pgfpathlineto{\pgfqpoint{1.535670in}{0.617212in}}%
\pgfpathlineto{\pgfqpoint{1.573374in}{0.617212in}}%
\pgfpathlineto{\pgfqpoint{1.573374in}{0.651055in}}%
\pgfpathlineto{\pgfqpoint{1.611078in}{0.651055in}}%
\pgfpathlineto{\pgfqpoint{1.611078in}{0.679039in}}%
\pgfpathlineto{\pgfqpoint{1.648783in}{0.679039in}}%
\pgfpathlineto{\pgfqpoint{1.648783in}{0.708128in}}%
\pgfpathlineto{\pgfqpoint{1.686487in}{0.708128in}}%
\pgfpathlineto{\pgfqpoint{1.686487in}{0.744994in}}%
\pgfpathlineto{\pgfqpoint{1.724191in}{0.744994in}}%
\pgfpathlineto{\pgfqpoint{1.724191in}{0.775899in}}%
\pgfpathlineto{\pgfqpoint{1.761895in}{0.775899in}}%
\pgfpathlineto{\pgfqpoint{1.761895in}{0.807665in}}%
\pgfpathlineto{\pgfqpoint{1.799600in}{0.807665in}}%
\pgfpathlineto{\pgfqpoint{1.799600in}{0.841027in}}%
\pgfpathlineto{\pgfqpoint{1.837304in}{0.841027in}}%
\pgfpathlineto{\pgfqpoint{1.837304in}{0.869876in}}%
\pgfpathlineto{\pgfqpoint{1.875008in}{0.869876in}}%
\pgfpathlineto{\pgfqpoint{1.875008in}{0.917629in}}%
\pgfpathlineto{\pgfqpoint{1.912713in}{0.917629in}}%
\pgfpathlineto{\pgfqpoint{1.912713in}{1.006563in}}%
\pgfpathlineto{\pgfqpoint{1.950417in}{1.006563in}}%
\pgfpathlineto{\pgfqpoint{1.950417in}{1.124053in}}%
\pgfpathlineto{\pgfqpoint{1.988121in}{1.124053in}}%
\pgfpathlineto{\pgfqpoint{1.988121in}{1.250490in}}%
\pgfpathlineto{\pgfqpoint{2.025825in}{1.250490in}}%
\pgfpathlineto{\pgfqpoint{2.025825in}{1.324798in}}%
\pgfpathlineto{\pgfqpoint{2.063530in}{1.324798in}}%
\pgfpathlineto{\pgfqpoint{2.063530in}{1.364481in}}%
\pgfpathlineto{\pgfqpoint{2.101234in}{1.364481in}}%
\pgfpathlineto{\pgfqpoint{2.101234in}{1.116094in}}%
\pgfpathlineto{\pgfqpoint{2.138938in}{1.116094in}}%
\pgfpathlineto{\pgfqpoint{2.138938in}{0.857681in}}%
\pgfpathlineto{\pgfqpoint{2.176643in}{0.857681in}}%
\pgfpathlineto{\pgfqpoint{2.176643in}{0.672018in}}%
\pgfpathlineto{\pgfqpoint{2.214347in}{0.672018in}}%
\pgfpathlineto{\pgfqpoint{2.214347in}{0.537805in}}%
\pgfpathlineto{\pgfqpoint{2.252051in}{0.537805in}}%
\pgfpathlineto{\pgfqpoint{2.252051in}{0.482025in}}%
\pgfpathlineto{\pgfqpoint{2.289755in}{0.482025in}}%
\pgfpathlineto{\pgfqpoint{2.289755in}{0.467911in}}%
\pgfpathlineto{\pgfqpoint{2.327460in}{0.467911in}}%
\pgfpathlineto{\pgfqpoint{2.327460in}{0.460279in}}%
\pgfpathlineto{\pgfqpoint{2.365164in}{0.460279in}}%
\pgfpathlineto{\pgfqpoint{2.365164in}{0.457394in}}%
\pgfpathlineto{\pgfqpoint{2.402868in}{0.457394in}}%
\pgfpathlineto{\pgfqpoint{2.402868in}{0.455680in}}%
\pgfpathlineto{\pgfqpoint{2.440573in}{0.455680in}}%
\pgfpathlineto{\pgfqpoint{2.440573in}{0.449983in}}%
\pgfpathlineto{\pgfqpoint{2.402868in}{0.449983in}}%
\pgfpathlineto{\pgfqpoint{2.402868in}{0.449983in}}%
\pgfpathlineto{\pgfqpoint{2.365164in}{0.449983in}}%
\pgfpathlineto{\pgfqpoint{2.365164in}{0.449983in}}%
\pgfpathlineto{\pgfqpoint{2.327460in}{0.449983in}}%
\pgfpathlineto{\pgfqpoint{2.327460in}{0.449983in}}%
\pgfpathlineto{\pgfqpoint{2.289755in}{0.449983in}}%
\pgfpathlineto{\pgfqpoint{2.289755in}{0.449983in}}%
\pgfpathlineto{\pgfqpoint{2.252051in}{0.449983in}}%
\pgfpathlineto{\pgfqpoint{2.252051in}{0.449983in}}%
\pgfpathlineto{\pgfqpoint{2.214347in}{0.449983in}}%
\pgfpathlineto{\pgfqpoint{2.214347in}{0.449983in}}%
\pgfpathlineto{\pgfqpoint{2.176643in}{0.449983in}}%
\pgfpathlineto{\pgfqpoint{2.176643in}{0.449983in}}%
\pgfpathlineto{\pgfqpoint{2.138938in}{0.449983in}}%
\pgfpathlineto{\pgfqpoint{2.138938in}{0.449983in}}%
\pgfpathlineto{\pgfqpoint{2.101234in}{0.449983in}}%
\pgfpathlineto{\pgfqpoint{2.101234in}{0.449983in}}%
\pgfpathlineto{\pgfqpoint{2.063530in}{0.449983in}}%
\pgfpathlineto{\pgfqpoint{2.063530in}{0.449983in}}%
\pgfpathlineto{\pgfqpoint{2.025825in}{0.449983in}}%
\pgfpathlineto{\pgfqpoint{2.025825in}{0.449983in}}%
\pgfpathlineto{\pgfqpoint{1.988121in}{0.449983in}}%
\pgfpathlineto{\pgfqpoint{1.988121in}{0.449983in}}%
\pgfpathlineto{\pgfqpoint{1.950417in}{0.449983in}}%
\pgfpathlineto{\pgfqpoint{1.950417in}{0.449983in}}%
\pgfpathlineto{\pgfqpoint{1.912713in}{0.449983in}}%
\pgfpathlineto{\pgfqpoint{1.912713in}{0.449983in}}%
\pgfpathlineto{\pgfqpoint{1.875008in}{0.449983in}}%
\pgfpathlineto{\pgfqpoint{1.875008in}{0.449983in}}%
\pgfpathlineto{\pgfqpoint{1.837304in}{0.449983in}}%
\pgfpathlineto{\pgfqpoint{1.837304in}{0.449983in}}%
\pgfpathlineto{\pgfqpoint{1.799600in}{0.449983in}}%
\pgfpathlineto{\pgfqpoint{1.799600in}{0.449983in}}%
\pgfpathlineto{\pgfqpoint{1.761895in}{0.449983in}}%
\pgfpathlineto{\pgfqpoint{1.761895in}{0.449983in}}%
\pgfpathlineto{\pgfqpoint{1.724191in}{0.449983in}}%
\pgfpathlineto{\pgfqpoint{1.724191in}{0.449983in}}%
\pgfpathlineto{\pgfqpoint{1.686487in}{0.449983in}}%
\pgfpathlineto{\pgfqpoint{1.686487in}{0.449983in}}%
\pgfpathlineto{\pgfqpoint{1.648783in}{0.449983in}}%
\pgfpathlineto{\pgfqpoint{1.648783in}{0.449983in}}%
\pgfpathlineto{\pgfqpoint{1.611078in}{0.449983in}}%
\pgfpathlineto{\pgfqpoint{1.611078in}{0.449983in}}%
\pgfpathlineto{\pgfqpoint{1.573374in}{0.449983in}}%
\pgfpathlineto{\pgfqpoint{1.573374in}{0.449983in}}%
\pgfpathlineto{\pgfqpoint{1.535670in}{0.449983in}}%
\pgfpathlineto{\pgfqpoint{1.535670in}{0.449983in}}%
\pgfpathlineto{\pgfqpoint{1.497965in}{0.449983in}}%
\pgfpathlineto{\pgfqpoint{1.497965in}{0.449983in}}%
\pgfpathlineto{\pgfqpoint{1.460261in}{0.449983in}}%
\pgfpathlineto{\pgfqpoint{1.460261in}{0.449983in}}%
\pgfpathlineto{\pgfqpoint{1.422557in}{0.449983in}}%
\pgfpathlineto{\pgfqpoint{1.422557in}{0.449983in}}%
\pgfpathlineto{\pgfqpoint{1.384853in}{0.449983in}}%
\pgfpathlineto{\pgfqpoint{1.384853in}{0.449983in}}%
\pgfpathlineto{\pgfqpoint{1.347148in}{0.449983in}}%
\pgfpathlineto{\pgfqpoint{1.347148in}{0.449983in}}%
\pgfpathlineto{\pgfqpoint{1.309444in}{0.449983in}}%
\pgfpathlineto{\pgfqpoint{1.309444in}{0.449983in}}%
\pgfpathlineto{\pgfqpoint{1.271740in}{0.449983in}}%
\pgfpathlineto{\pgfqpoint{1.271740in}{0.449983in}}%
\pgfpathlineto{\pgfqpoint{1.234035in}{0.449983in}}%
\pgfpathlineto{\pgfqpoint{1.234035in}{0.449983in}}%
\pgfpathlineto{\pgfqpoint{1.196331in}{0.449983in}}%
\pgfpathlineto{\pgfqpoint{1.196331in}{0.449983in}}%
\pgfpathlineto{\pgfqpoint{1.158627in}{0.449983in}}%
\pgfpathlineto{\pgfqpoint{1.158627in}{0.449983in}}%
\pgfpathlineto{\pgfqpoint{1.120923in}{0.449983in}}%
\pgfpathlineto{\pgfqpoint{1.120923in}{0.449983in}}%
\pgfpathlineto{\pgfqpoint{1.083218in}{0.449983in}}%
\pgfpathlineto{\pgfqpoint{1.083218in}{0.449983in}}%
\pgfpathlineto{\pgfqpoint{1.045514in}{0.449983in}}%
\pgfpathlineto{\pgfqpoint{1.045514in}{0.449983in}}%
\pgfpathlineto{\pgfqpoint{1.007810in}{0.449983in}}%
\pgfpathlineto{\pgfqpoint{1.007810in}{0.449983in}}%
\pgfpathlineto{\pgfqpoint{0.970105in}{0.449983in}}%
\pgfpathlineto{\pgfqpoint{0.970105in}{0.449983in}}%
\pgfpathlineto{\pgfqpoint{0.932401in}{0.449983in}}%
\pgfpathlineto{\pgfqpoint{0.932401in}{0.449983in}}%
\pgfpathlineto{\pgfqpoint{0.894697in}{0.449983in}}%
\pgfpathlineto{\pgfqpoint{0.894697in}{0.449983in}}%
\pgfpathlineto{\pgfqpoint{0.856993in}{0.449983in}}%
\pgfpathlineto{\pgfqpoint{0.856993in}{0.449983in}}%
\pgfpathlineto{\pgfqpoint{0.819288in}{0.449983in}}%
\pgfpathlineto{\pgfqpoint{0.819288in}{0.449983in}}%
\pgfpathlineto{\pgfqpoint{0.781584in}{0.449983in}}%
\pgfpathlineto{\pgfqpoint{0.781584in}{0.449983in}}%
\pgfpathlineto{\pgfqpoint{0.743880in}{0.449983in}}%
\pgfpathlineto{\pgfqpoint{0.743880in}{0.449983in}}%
\pgfpathlineto{\pgfqpoint{0.706176in}{0.449983in}}%
\pgfpathlineto{\pgfqpoint{0.706176in}{0.449983in}}%
\pgfpathlineto{\pgfqpoint{0.668471in}{0.449983in}}%
\pgfpathlineto{\pgfqpoint{0.668471in}{0.449983in}}%
\pgfpathlineto{\pgfqpoint{0.630767in}{0.449983in}}%
\pgfpathlineto{\pgfqpoint{0.630767in}{0.449983in}}%
\pgfpathlineto{\pgfqpoint{0.593063in}{0.449983in}}%
\pgfpathlineto{\pgfqpoint{0.593063in}{0.449983in}}%
\pgfpathlineto{\pgfqpoint{0.555358in}{0.449983in}}%
\pgfusepath{fill}%
\end{pgfscope}%
\begin{pgfscope}%
\pgfpathrectangle{\pgfqpoint{0.437532in}{0.449983in}}{\pgfqpoint{2.120866in}{1.169449in}} %
\pgfusepath{clip}%
\pgfsetbuttcap%
\pgfsetmiterjoin%
\pgfsetlinewidth{0.501875pt}%
\definecolor{currentstroke}{rgb}{1.000000,0.000000,0.000000}%
\pgfsetstrokecolor{currentstroke}%
\pgfsetdash{}{0pt}%
\pgfpathmoveto{\pgfqpoint{0.555358in}{0.449983in}}%
\pgfpathlineto{\pgfqpoint{0.555358in}{0.452370in}}%
\pgfpathlineto{\pgfqpoint{0.593063in}{0.452370in}}%
\pgfpathlineto{\pgfqpoint{0.593063in}{0.453701in}}%
\pgfpathlineto{\pgfqpoint{0.630767in}{0.453701in}}%
\pgfpathlineto{\pgfqpoint{0.630767in}{0.453930in}}%
\pgfpathlineto{\pgfqpoint{0.668471in}{0.453930in}}%
\pgfpathlineto{\pgfqpoint{0.668471in}{0.455284in}}%
\pgfpathlineto{\pgfqpoint{0.706176in}{0.455284in}}%
\pgfpathlineto{\pgfqpoint{0.706176in}{0.457326in}}%
\pgfpathlineto{\pgfqpoint{0.743880in}{0.457326in}}%
\pgfpathlineto{\pgfqpoint{0.743880in}{0.457785in}}%
\pgfpathlineto{\pgfqpoint{0.781584in}{0.457785in}}%
\pgfpathlineto{\pgfqpoint{0.781584in}{0.460608in}}%
\pgfpathlineto{\pgfqpoint{0.819288in}{0.460608in}}%
\pgfpathlineto{\pgfqpoint{0.819288in}{0.462007in}}%
\pgfpathlineto{\pgfqpoint{0.856993in}{0.462007in}}%
\pgfpathlineto{\pgfqpoint{0.856993in}{0.464577in}}%
\pgfpathlineto{\pgfqpoint{0.894697in}{0.464577in}}%
\pgfpathlineto{\pgfqpoint{0.894697in}{0.468846in}}%
\pgfpathlineto{\pgfqpoint{0.932401in}{0.468846in}}%
\pgfpathlineto{\pgfqpoint{0.932401in}{0.472540in}}%
\pgfpathlineto{\pgfqpoint{0.970105in}{0.472540in}}%
\pgfpathlineto{\pgfqpoint{0.970105in}{0.477221in}}%
\pgfpathlineto{\pgfqpoint{1.007810in}{0.477221in}}%
\pgfpathlineto{\pgfqpoint{1.007810in}{0.483049in}}%
\pgfpathlineto{\pgfqpoint{1.045514in}{0.483049in}}%
\pgfpathlineto{\pgfqpoint{1.045514in}{0.488396in}}%
\pgfpathlineto{\pgfqpoint{1.083218in}{0.488396in}}%
\pgfpathlineto{\pgfqpoint{1.083218in}{0.497184in}}%
\pgfpathlineto{\pgfqpoint{1.120923in}{0.497184in}}%
\pgfpathlineto{\pgfqpoint{1.120923in}{0.506730in}}%
\pgfpathlineto{\pgfqpoint{1.158627in}{0.506730in}}%
\pgfpathlineto{\pgfqpoint{1.158627in}{0.514165in}}%
\pgfpathlineto{\pgfqpoint{1.196331in}{0.514165in}}%
\pgfpathlineto{\pgfqpoint{1.196331in}{0.529333in}}%
\pgfpathlineto{\pgfqpoint{1.234035in}{0.529333in}}%
\pgfpathlineto{\pgfqpoint{1.234035in}{0.538809in}}%
\pgfpathlineto{\pgfqpoint{1.271740in}{0.538809in}}%
\pgfpathlineto{\pgfqpoint{1.271740in}{0.556455in}}%
\pgfpathlineto{\pgfqpoint{1.309444in}{0.556455in}}%
\pgfpathlineto{\pgfqpoint{1.309444in}{0.571875in}}%
\pgfpathlineto{\pgfqpoint{1.347148in}{0.571875in}}%
\pgfpathlineto{\pgfqpoint{1.347148in}{0.585070in}}%
\pgfpathlineto{\pgfqpoint{1.384853in}{0.585070in}}%
\pgfpathlineto{\pgfqpoint{1.384853in}{0.607534in}}%
\pgfpathlineto{\pgfqpoint{1.422557in}{0.607534in}}%
\pgfpathlineto{\pgfqpoint{1.422557in}{0.627681in}}%
\pgfpathlineto{\pgfqpoint{1.460261in}{0.627681in}}%
\pgfpathlineto{\pgfqpoint{1.460261in}{0.645098in}}%
\pgfpathlineto{\pgfqpoint{1.497965in}{0.645098in}}%
\pgfpathlineto{\pgfqpoint{1.497965in}{0.665841in}}%
\pgfpathlineto{\pgfqpoint{1.535670in}{0.665841in}}%
\pgfpathlineto{\pgfqpoint{1.535670in}{0.685988in}}%
\pgfpathlineto{\pgfqpoint{1.573374in}{0.685988in}}%
\pgfpathlineto{\pgfqpoint{1.573374in}{0.704621in}}%
\pgfpathlineto{\pgfqpoint{1.611078in}{0.704621in}}%
\pgfpathlineto{\pgfqpoint{1.611078in}{0.722496in}}%
\pgfpathlineto{\pgfqpoint{1.648783in}{0.722496in}}%
\pgfpathlineto{\pgfqpoint{1.648783in}{0.736769in}}%
\pgfpathlineto{\pgfqpoint{1.686487in}{0.736769in}}%
\pgfpathlineto{\pgfqpoint{1.686487in}{0.750812in}}%
\pgfpathlineto{\pgfqpoint{1.724191in}{0.750812in}}%
\pgfpathlineto{\pgfqpoint{1.724191in}{0.767472in}}%
\pgfpathlineto{\pgfqpoint{1.761895in}{0.767472in}}%
\pgfpathlineto{\pgfqpoint{1.761895in}{0.786540in}}%
\pgfpathlineto{\pgfqpoint{1.799600in}{0.786540in}}%
\pgfpathlineto{\pgfqpoint{1.799600in}{0.805700in}}%
\pgfpathlineto{\pgfqpoint{1.837304in}{0.805700in}}%
\pgfpathlineto{\pgfqpoint{1.837304in}{0.830965in}}%
\pgfpathlineto{\pgfqpoint{1.875008in}{0.830965in}}%
\pgfpathlineto{\pgfqpoint{1.875008in}{0.862287in}}%
\pgfpathlineto{\pgfqpoint{1.912713in}{0.862287in}}%
\pgfpathlineto{\pgfqpoint{1.912713in}{0.926216in}}%
\pgfpathlineto{\pgfqpoint{1.950417in}{0.926216in}}%
\pgfpathlineto{\pgfqpoint{1.950417in}{1.000103in}}%
\pgfpathlineto{\pgfqpoint{1.988121in}{1.000103in}}%
\pgfpathlineto{\pgfqpoint{1.988121in}{1.070641in}}%
\pgfpathlineto{\pgfqpoint{2.025825in}{1.070641in}}%
\pgfpathlineto{\pgfqpoint{2.025825in}{1.128719in}}%
\pgfpathlineto{\pgfqpoint{2.063530in}{1.128719in}}%
\pgfpathlineto{\pgfqpoint{2.063530in}{1.213690in}}%
\pgfpathlineto{\pgfqpoint{2.101234in}{1.213690in}}%
\pgfpathlineto{\pgfqpoint{2.101234in}{1.025276in}}%
\pgfpathlineto{\pgfqpoint{2.138938in}{1.025276in}}%
\pgfpathlineto{\pgfqpoint{2.138938in}{0.809142in}}%
\pgfpathlineto{\pgfqpoint{2.176643in}{0.809142in}}%
\pgfpathlineto{\pgfqpoint{2.176643in}{0.632821in}}%
\pgfpathlineto{\pgfqpoint{2.214347in}{0.632821in}}%
\pgfpathlineto{\pgfqpoint{2.214347in}{0.516528in}}%
\pgfpathlineto{\pgfqpoint{2.252051in}{0.516528in}}%
\pgfpathlineto{\pgfqpoint{2.252051in}{0.479860in}}%
\pgfpathlineto{\pgfqpoint{2.289755in}{0.479860in}}%
\pgfpathlineto{\pgfqpoint{2.289755in}{0.464623in}}%
\pgfpathlineto{\pgfqpoint{2.327460in}{0.464623in}}%
\pgfpathlineto{\pgfqpoint{2.327460in}{0.456890in}}%
\pgfpathlineto{\pgfqpoint{2.365164in}{0.456890in}}%
\pgfpathlineto{\pgfqpoint{2.365164in}{0.452967in}}%
\pgfpathlineto{\pgfqpoint{2.402868in}{0.452967in}}%
\pgfpathlineto{\pgfqpoint{2.402868in}{0.450947in}}%
\pgfpathlineto{\pgfqpoint{2.440573in}{0.450947in}}%
\pgfpathlineto{\pgfqpoint{2.440573in}{0.449983in}}%
\pgfusepath{stroke}%
\end{pgfscope}%
\begin{pgfscope}%
\pgfpathrectangle{\pgfqpoint{0.437532in}{0.449983in}}{\pgfqpoint{2.120866in}{1.169449in}} %
\pgfusepath{clip}%
\pgfsetbuttcap%
\pgfsetmiterjoin%
\pgfsetlinewidth{0.501875pt}%
\definecolor{currentstroke}{rgb}{1.000000,0.647059,0.000000}%
\pgfsetstrokecolor{currentstroke}%
\pgfsetdash{}{0pt}%
\pgfpathmoveto{\pgfqpoint{0.555358in}{0.449983in}}%
\pgfpathlineto{\pgfqpoint{0.555358in}{0.450107in}}%
\pgfpathlineto{\pgfqpoint{0.593063in}{0.450107in}}%
\pgfpathlineto{\pgfqpoint{0.593063in}{0.450225in}}%
\pgfpathlineto{\pgfqpoint{0.630767in}{0.450225in}}%
\pgfpathlineto{\pgfqpoint{0.630767in}{0.450175in}}%
\pgfpathlineto{\pgfqpoint{0.668471in}{0.450175in}}%
\pgfpathlineto{\pgfqpoint{0.668471in}{0.450225in}}%
\pgfpathlineto{\pgfqpoint{0.706176in}{0.450225in}}%
\pgfpathlineto{\pgfqpoint{0.706176in}{0.450111in}}%
\pgfpathlineto{\pgfqpoint{0.743880in}{0.450111in}}%
\pgfpathlineto{\pgfqpoint{0.743880in}{0.450269in}}%
\pgfpathlineto{\pgfqpoint{0.781584in}{0.450269in}}%
\pgfpathlineto{\pgfqpoint{0.781584in}{0.450918in}}%
\pgfpathlineto{\pgfqpoint{0.819288in}{0.450918in}}%
\pgfpathlineto{\pgfqpoint{0.819288in}{0.451239in}}%
\pgfpathlineto{\pgfqpoint{0.856993in}{0.451239in}}%
\pgfpathlineto{\pgfqpoint{0.856993in}{0.450654in}}%
\pgfpathlineto{\pgfqpoint{0.894697in}{0.450654in}}%
\pgfpathlineto{\pgfqpoint{0.894697in}{0.451818in}}%
\pgfpathlineto{\pgfqpoint{0.932401in}{0.451818in}}%
\pgfpathlineto{\pgfqpoint{0.932401in}{0.451885in}}%
\pgfpathlineto{\pgfqpoint{0.970105in}{0.451885in}}%
\pgfpathlineto{\pgfqpoint{0.970105in}{0.453680in}}%
\pgfpathlineto{\pgfqpoint{1.007810in}{0.453680in}}%
\pgfpathlineto{\pgfqpoint{1.007810in}{0.453961in}}%
\pgfpathlineto{\pgfqpoint{1.045514in}{0.453961in}}%
\pgfpathlineto{\pgfqpoint{1.045514in}{0.455742in}}%
\pgfpathlineto{\pgfqpoint{1.083218in}{0.455742in}}%
\pgfpathlineto{\pgfqpoint{1.083218in}{0.456245in}}%
\pgfpathlineto{\pgfqpoint{1.120923in}{0.456245in}}%
\pgfpathlineto{\pgfqpoint{1.120923in}{0.458711in}}%
\pgfpathlineto{\pgfqpoint{1.158627in}{0.458711in}}%
\pgfpathlineto{\pgfqpoint{1.158627in}{0.463975in}}%
\pgfpathlineto{\pgfqpoint{1.196331in}{0.463975in}}%
\pgfpathlineto{\pgfqpoint{1.196331in}{0.463033in}}%
\pgfpathlineto{\pgfqpoint{1.234035in}{0.463033in}}%
\pgfpathlineto{\pgfqpoint{1.234035in}{0.469496in}}%
\pgfpathlineto{\pgfqpoint{1.271740in}{0.469496in}}%
\pgfpathlineto{\pgfqpoint{1.271740in}{0.475904in}}%
\pgfpathlineto{\pgfqpoint{1.309444in}{0.475904in}}%
\pgfpathlineto{\pgfqpoint{1.309444in}{0.482581in}}%
\pgfpathlineto{\pgfqpoint{1.347148in}{0.482581in}}%
\pgfpathlineto{\pgfqpoint{1.347148in}{0.494435in}}%
\pgfpathlineto{\pgfqpoint{1.384853in}{0.494435in}}%
\pgfpathlineto{\pgfqpoint{1.384853in}{0.512236in}}%
\pgfpathlineto{\pgfqpoint{1.422557in}{0.512236in}}%
\pgfpathlineto{\pgfqpoint{1.422557in}{0.523440in}}%
\pgfpathlineto{\pgfqpoint{1.460261in}{0.523440in}}%
\pgfpathlineto{\pgfqpoint{1.460261in}{0.544227in}}%
\pgfpathlineto{\pgfqpoint{1.497965in}{0.544227in}}%
\pgfpathlineto{\pgfqpoint{1.497965in}{0.567715in}}%
\pgfpathlineto{\pgfqpoint{1.535670in}{0.567715in}}%
\pgfpathlineto{\pgfqpoint{1.535670in}{0.591727in}}%
\pgfpathlineto{\pgfqpoint{1.573374in}{0.591727in}}%
\pgfpathlineto{\pgfqpoint{1.573374in}{0.614695in}}%
\pgfpathlineto{\pgfqpoint{1.611078in}{0.614695in}}%
\pgfpathlineto{\pgfqpoint{1.611078in}{0.663824in}}%
\pgfpathlineto{\pgfqpoint{1.648783in}{0.663824in}}%
\pgfpathlineto{\pgfqpoint{1.648783in}{0.691782in}}%
\pgfpathlineto{\pgfqpoint{1.686487in}{0.691782in}}%
\pgfpathlineto{\pgfqpoint{1.686487in}{0.713595in}}%
\pgfpathlineto{\pgfqpoint{1.724191in}{0.713595in}}%
\pgfpathlineto{\pgfqpoint{1.724191in}{0.743525in}}%
\pgfpathlineto{\pgfqpoint{1.761895in}{0.743525in}}%
\pgfpathlineto{\pgfqpoint{1.761895in}{0.769699in}}%
\pgfpathlineto{\pgfqpoint{1.799600in}{0.769699in}}%
\pgfpathlineto{\pgfqpoint{1.799600in}{0.827139in}}%
\pgfpathlineto{\pgfqpoint{1.837304in}{0.827139in}}%
\pgfpathlineto{\pgfqpoint{1.837304in}{0.844090in}}%
\pgfpathlineto{\pgfqpoint{1.875008in}{0.844090in}}%
\pgfpathlineto{\pgfqpoint{1.875008in}{0.867410in}}%
\pgfpathlineto{\pgfqpoint{1.912713in}{0.867410in}}%
\pgfpathlineto{\pgfqpoint{1.912713in}{0.970332in}}%
\pgfpathlineto{\pgfqpoint{1.950417in}{0.970332in}}%
\pgfpathlineto{\pgfqpoint{1.950417in}{1.146464in}}%
\pgfpathlineto{\pgfqpoint{1.988121in}{1.146464in}}%
\pgfpathlineto{\pgfqpoint{1.988121in}{1.292951in}}%
\pgfpathlineto{\pgfqpoint{2.025825in}{1.292951in}}%
\pgfpathlineto{\pgfqpoint{2.025825in}{1.487761in}}%
\pgfpathlineto{\pgfqpoint{2.063530in}{1.487761in}}%
\pgfpathlineto{\pgfqpoint{2.063530in}{1.468735in}}%
\pgfpathlineto{\pgfqpoint{2.101234in}{1.468735in}}%
\pgfpathlineto{\pgfqpoint{2.101234in}{1.252003in}}%
\pgfpathlineto{\pgfqpoint{2.138938in}{1.252003in}}%
\pgfpathlineto{\pgfqpoint{2.138938in}{0.895887in}}%
\pgfpathlineto{\pgfqpoint{2.176643in}{0.895887in}}%
\pgfpathlineto{\pgfqpoint{2.176643in}{0.671482in}}%
\pgfpathlineto{\pgfqpoint{2.214347in}{0.671482in}}%
\pgfpathlineto{\pgfqpoint{2.214347in}{0.547591in}}%
\pgfpathlineto{\pgfqpoint{2.252051in}{0.547591in}}%
\pgfpathlineto{\pgfqpoint{2.252051in}{0.499089in}}%
\pgfpathlineto{\pgfqpoint{2.289755in}{0.499089in}}%
\pgfpathlineto{\pgfqpoint{2.289755in}{0.479429in}}%
\pgfpathlineto{\pgfqpoint{2.327460in}{0.479429in}}%
\pgfpathlineto{\pgfqpoint{2.327460in}{0.457960in}}%
\pgfpathlineto{\pgfqpoint{2.365164in}{0.457960in}}%
\pgfpathlineto{\pgfqpoint{2.365164in}{0.453046in}}%
\pgfpathlineto{\pgfqpoint{2.402868in}{0.453046in}}%
\pgfpathlineto{\pgfqpoint{2.402868in}{0.452269in}}%
\pgfpathlineto{\pgfqpoint{2.440573in}{0.452269in}}%
\pgfpathlineto{\pgfqpoint{2.440573in}{0.449983in}}%
\pgfusepath{stroke}%
\end{pgfscope}%
\begin{pgfscope}%
\pgfsetrectcap%
\pgfsetmiterjoin%
\pgfsetlinewidth{1.003750pt}%
\definecolor{currentstroke}{rgb}{0.000000,0.000000,0.000000}%
\pgfsetstrokecolor{currentstroke}%
\pgfsetdash{}{0pt}%
\pgfpathmoveto{\pgfqpoint{0.437532in}{1.619432in}}%
\pgfpathlineto{\pgfqpoint{2.558398in}{1.619432in}}%
\pgfusepath{stroke}%
\end{pgfscope}%
\begin{pgfscope}%
\pgfsetrectcap%
\pgfsetmiterjoin%
\pgfsetlinewidth{1.003750pt}%
\definecolor{currentstroke}{rgb}{0.000000,0.000000,0.000000}%
\pgfsetstrokecolor{currentstroke}%
\pgfsetdash{}{0pt}%
\pgfpathmoveto{\pgfqpoint{2.558398in}{0.449983in}}%
\pgfpathlineto{\pgfqpoint{2.558398in}{1.619432in}}%
\pgfusepath{stroke}%
\end{pgfscope}%
\begin{pgfscope}%
\pgfsetrectcap%
\pgfsetmiterjoin%
\pgfsetlinewidth{1.003750pt}%
\definecolor{currentstroke}{rgb}{0.000000,0.000000,0.000000}%
\pgfsetstrokecolor{currentstroke}%
\pgfsetdash{}{0pt}%
\pgfpathmoveto{\pgfqpoint{0.437532in}{0.449983in}}%
\pgfpathlineto{\pgfqpoint{2.558398in}{0.449983in}}%
\pgfusepath{stroke}%
\end{pgfscope}%
\begin{pgfscope}%
\pgfsetrectcap%
\pgfsetmiterjoin%
\pgfsetlinewidth{1.003750pt}%
\definecolor{currentstroke}{rgb}{0.000000,0.000000,0.000000}%
\pgfsetstrokecolor{currentstroke}%
\pgfsetdash{}{0pt}%
\pgfpathmoveto{\pgfqpoint{0.437532in}{0.449983in}}%
\pgfpathlineto{\pgfqpoint{0.437532in}{1.619432in}}%
\pgfusepath{stroke}%
\end{pgfscope}%
\begin{pgfscope}%
\pgfsetbuttcap%
\pgfsetroundjoin%
\definecolor{currentfill}{rgb}{0.000000,0.000000,0.000000}%
\pgfsetfillcolor{currentfill}%
\pgfsetlinewidth{0.501875pt}%
\definecolor{currentstroke}{rgb}{0.000000,0.000000,0.000000}%
\pgfsetstrokecolor{currentstroke}%
\pgfsetdash{}{0pt}%
\pgfsys@defobject{currentmarker}{\pgfqpoint{0.000000in}{0.000000in}}{\pgfqpoint{0.000000in}{0.069444in}}{%
\pgfpathmoveto{\pgfqpoint{0.000000in}{0.000000in}}%
\pgfpathlineto{\pgfqpoint{0.000000in}{0.069444in}}%
\pgfusepath{stroke,fill}%
}%
\begin{pgfscope}%
\pgfsys@transformshift{0.437532in}{0.449983in}%
\pgfsys@useobject{currentmarker}{}%
\end{pgfscope}%
\end{pgfscope}%
\begin{pgfscope}%
\pgfsetbuttcap%
\pgfsetroundjoin%
\definecolor{currentfill}{rgb}{0.000000,0.000000,0.000000}%
\pgfsetfillcolor{currentfill}%
\pgfsetlinewidth{0.501875pt}%
\definecolor{currentstroke}{rgb}{0.000000,0.000000,0.000000}%
\pgfsetstrokecolor{currentstroke}%
\pgfsetdash{}{0pt}%
\pgfsys@defobject{currentmarker}{\pgfqpoint{0.000000in}{-0.069444in}}{\pgfqpoint{0.000000in}{0.000000in}}{%
\pgfpathmoveto{\pgfqpoint{0.000000in}{0.000000in}}%
\pgfpathlineto{\pgfqpoint{0.000000in}{-0.069444in}}%
\pgfusepath{stroke,fill}%
}%
\begin{pgfscope}%
\pgfsys@transformshift{0.437532in}{1.619432in}%
\pgfsys@useobject{currentmarker}{}%
\end{pgfscope}%
\end{pgfscope}%
\begin{pgfscope}%
\pgftext[x=0.437532in,y=0.380539in,,top]{\rmfamily\fontsize{8.000000}{9.600000}\selectfont −140}%
\end{pgfscope}%
\begin{pgfscope}%
\pgfsetbuttcap%
\pgfsetroundjoin%
\definecolor{currentfill}{rgb}{0.000000,0.000000,0.000000}%
\pgfsetfillcolor{currentfill}%
\pgfsetlinewidth{0.501875pt}%
\definecolor{currentstroke}{rgb}{0.000000,0.000000,0.000000}%
\pgfsetstrokecolor{currentstroke}%
\pgfsetdash{}{0pt}%
\pgfsys@defobject{currentmarker}{\pgfqpoint{0.000000in}{0.000000in}}{\pgfqpoint{0.000000in}{0.069444in}}{%
\pgfpathmoveto{\pgfqpoint{0.000000in}{0.000000in}}%
\pgfpathlineto{\pgfqpoint{0.000000in}{0.069444in}}%
\pgfusepath{stroke,fill}%
}%
\begin{pgfscope}%
\pgfsys@transformshift{0.673184in}{0.449983in}%
\pgfsys@useobject{currentmarker}{}%
\end{pgfscope}%
\end{pgfscope}%
\begin{pgfscope}%
\pgfsetbuttcap%
\pgfsetroundjoin%
\definecolor{currentfill}{rgb}{0.000000,0.000000,0.000000}%
\pgfsetfillcolor{currentfill}%
\pgfsetlinewidth{0.501875pt}%
\definecolor{currentstroke}{rgb}{0.000000,0.000000,0.000000}%
\pgfsetstrokecolor{currentstroke}%
\pgfsetdash{}{0pt}%
\pgfsys@defobject{currentmarker}{\pgfqpoint{0.000000in}{-0.069444in}}{\pgfqpoint{0.000000in}{0.000000in}}{%
\pgfpathmoveto{\pgfqpoint{0.000000in}{0.000000in}}%
\pgfpathlineto{\pgfqpoint{0.000000in}{-0.069444in}}%
\pgfusepath{stroke,fill}%
}%
\begin{pgfscope}%
\pgfsys@transformshift{0.673184in}{1.619432in}%
\pgfsys@useobject{currentmarker}{}%
\end{pgfscope}%
\end{pgfscope}%
\begin{pgfscope}%
\pgftext[x=0.673184in,y=0.380539in,,top]{\rmfamily\fontsize{8.000000}{9.600000}\selectfont −120}%
\end{pgfscope}%
\begin{pgfscope}%
\pgfsetbuttcap%
\pgfsetroundjoin%
\definecolor{currentfill}{rgb}{0.000000,0.000000,0.000000}%
\pgfsetfillcolor{currentfill}%
\pgfsetlinewidth{0.501875pt}%
\definecolor{currentstroke}{rgb}{0.000000,0.000000,0.000000}%
\pgfsetstrokecolor{currentstroke}%
\pgfsetdash{}{0pt}%
\pgfsys@defobject{currentmarker}{\pgfqpoint{0.000000in}{0.000000in}}{\pgfqpoint{0.000000in}{0.069444in}}{%
\pgfpathmoveto{\pgfqpoint{0.000000in}{0.000000in}}%
\pgfpathlineto{\pgfqpoint{0.000000in}{0.069444in}}%
\pgfusepath{stroke,fill}%
}%
\begin{pgfscope}%
\pgfsys@transformshift{0.908836in}{0.449983in}%
\pgfsys@useobject{currentmarker}{}%
\end{pgfscope}%
\end{pgfscope}%
\begin{pgfscope}%
\pgfsetbuttcap%
\pgfsetroundjoin%
\definecolor{currentfill}{rgb}{0.000000,0.000000,0.000000}%
\pgfsetfillcolor{currentfill}%
\pgfsetlinewidth{0.501875pt}%
\definecolor{currentstroke}{rgb}{0.000000,0.000000,0.000000}%
\pgfsetstrokecolor{currentstroke}%
\pgfsetdash{}{0pt}%
\pgfsys@defobject{currentmarker}{\pgfqpoint{0.000000in}{-0.069444in}}{\pgfqpoint{0.000000in}{0.000000in}}{%
\pgfpathmoveto{\pgfqpoint{0.000000in}{0.000000in}}%
\pgfpathlineto{\pgfqpoint{0.000000in}{-0.069444in}}%
\pgfusepath{stroke,fill}%
}%
\begin{pgfscope}%
\pgfsys@transformshift{0.908836in}{1.619432in}%
\pgfsys@useobject{currentmarker}{}%
\end{pgfscope}%
\end{pgfscope}%
\begin{pgfscope}%
\pgftext[x=0.908836in,y=0.380539in,,top]{\rmfamily\fontsize{8.000000}{9.600000}\selectfont −100}%
\end{pgfscope}%
\begin{pgfscope}%
\pgfsetbuttcap%
\pgfsetroundjoin%
\definecolor{currentfill}{rgb}{0.000000,0.000000,0.000000}%
\pgfsetfillcolor{currentfill}%
\pgfsetlinewidth{0.501875pt}%
\definecolor{currentstroke}{rgb}{0.000000,0.000000,0.000000}%
\pgfsetstrokecolor{currentstroke}%
\pgfsetdash{}{0pt}%
\pgfsys@defobject{currentmarker}{\pgfqpoint{0.000000in}{0.000000in}}{\pgfqpoint{0.000000in}{0.069444in}}{%
\pgfpathmoveto{\pgfqpoint{0.000000in}{0.000000in}}%
\pgfpathlineto{\pgfqpoint{0.000000in}{0.069444in}}%
\pgfusepath{stroke,fill}%
}%
\begin{pgfscope}%
\pgfsys@transformshift{1.144488in}{0.449983in}%
\pgfsys@useobject{currentmarker}{}%
\end{pgfscope}%
\end{pgfscope}%
\begin{pgfscope}%
\pgfsetbuttcap%
\pgfsetroundjoin%
\definecolor{currentfill}{rgb}{0.000000,0.000000,0.000000}%
\pgfsetfillcolor{currentfill}%
\pgfsetlinewidth{0.501875pt}%
\definecolor{currentstroke}{rgb}{0.000000,0.000000,0.000000}%
\pgfsetstrokecolor{currentstroke}%
\pgfsetdash{}{0pt}%
\pgfsys@defobject{currentmarker}{\pgfqpoint{0.000000in}{-0.069444in}}{\pgfqpoint{0.000000in}{0.000000in}}{%
\pgfpathmoveto{\pgfqpoint{0.000000in}{0.000000in}}%
\pgfpathlineto{\pgfqpoint{0.000000in}{-0.069444in}}%
\pgfusepath{stroke,fill}%
}%
\begin{pgfscope}%
\pgfsys@transformshift{1.144488in}{1.619432in}%
\pgfsys@useobject{currentmarker}{}%
\end{pgfscope}%
\end{pgfscope}%
\begin{pgfscope}%
\pgftext[x=1.144488in,y=0.380539in,,top]{\rmfamily\fontsize{8.000000}{9.600000}\selectfont −80}%
\end{pgfscope}%
\begin{pgfscope}%
\pgfsetbuttcap%
\pgfsetroundjoin%
\definecolor{currentfill}{rgb}{0.000000,0.000000,0.000000}%
\pgfsetfillcolor{currentfill}%
\pgfsetlinewidth{0.501875pt}%
\definecolor{currentstroke}{rgb}{0.000000,0.000000,0.000000}%
\pgfsetstrokecolor{currentstroke}%
\pgfsetdash{}{0pt}%
\pgfsys@defobject{currentmarker}{\pgfqpoint{0.000000in}{0.000000in}}{\pgfqpoint{0.000000in}{0.069444in}}{%
\pgfpathmoveto{\pgfqpoint{0.000000in}{0.000000in}}%
\pgfpathlineto{\pgfqpoint{0.000000in}{0.069444in}}%
\pgfusepath{stroke,fill}%
}%
\begin{pgfscope}%
\pgfsys@transformshift{1.380140in}{0.449983in}%
\pgfsys@useobject{currentmarker}{}%
\end{pgfscope}%
\end{pgfscope}%
\begin{pgfscope}%
\pgfsetbuttcap%
\pgfsetroundjoin%
\definecolor{currentfill}{rgb}{0.000000,0.000000,0.000000}%
\pgfsetfillcolor{currentfill}%
\pgfsetlinewidth{0.501875pt}%
\definecolor{currentstroke}{rgb}{0.000000,0.000000,0.000000}%
\pgfsetstrokecolor{currentstroke}%
\pgfsetdash{}{0pt}%
\pgfsys@defobject{currentmarker}{\pgfqpoint{0.000000in}{-0.069444in}}{\pgfqpoint{0.000000in}{0.000000in}}{%
\pgfpathmoveto{\pgfqpoint{0.000000in}{0.000000in}}%
\pgfpathlineto{\pgfqpoint{0.000000in}{-0.069444in}}%
\pgfusepath{stroke,fill}%
}%
\begin{pgfscope}%
\pgfsys@transformshift{1.380140in}{1.619432in}%
\pgfsys@useobject{currentmarker}{}%
\end{pgfscope}%
\end{pgfscope}%
\begin{pgfscope}%
\pgftext[x=1.380140in,y=0.380539in,,top]{\rmfamily\fontsize{8.000000}{9.600000}\selectfont −60}%
\end{pgfscope}%
\begin{pgfscope}%
\pgfsetbuttcap%
\pgfsetroundjoin%
\definecolor{currentfill}{rgb}{0.000000,0.000000,0.000000}%
\pgfsetfillcolor{currentfill}%
\pgfsetlinewidth{0.501875pt}%
\definecolor{currentstroke}{rgb}{0.000000,0.000000,0.000000}%
\pgfsetstrokecolor{currentstroke}%
\pgfsetdash{}{0pt}%
\pgfsys@defobject{currentmarker}{\pgfqpoint{0.000000in}{0.000000in}}{\pgfqpoint{0.000000in}{0.069444in}}{%
\pgfpathmoveto{\pgfqpoint{0.000000in}{0.000000in}}%
\pgfpathlineto{\pgfqpoint{0.000000in}{0.069444in}}%
\pgfusepath{stroke,fill}%
}%
\begin{pgfscope}%
\pgfsys@transformshift{1.615791in}{0.449983in}%
\pgfsys@useobject{currentmarker}{}%
\end{pgfscope}%
\end{pgfscope}%
\begin{pgfscope}%
\pgfsetbuttcap%
\pgfsetroundjoin%
\definecolor{currentfill}{rgb}{0.000000,0.000000,0.000000}%
\pgfsetfillcolor{currentfill}%
\pgfsetlinewidth{0.501875pt}%
\definecolor{currentstroke}{rgb}{0.000000,0.000000,0.000000}%
\pgfsetstrokecolor{currentstroke}%
\pgfsetdash{}{0pt}%
\pgfsys@defobject{currentmarker}{\pgfqpoint{0.000000in}{-0.069444in}}{\pgfqpoint{0.000000in}{0.000000in}}{%
\pgfpathmoveto{\pgfqpoint{0.000000in}{0.000000in}}%
\pgfpathlineto{\pgfqpoint{0.000000in}{-0.069444in}}%
\pgfusepath{stroke,fill}%
}%
\begin{pgfscope}%
\pgfsys@transformshift{1.615791in}{1.619432in}%
\pgfsys@useobject{currentmarker}{}%
\end{pgfscope}%
\end{pgfscope}%
\begin{pgfscope}%
\pgftext[x=1.615791in,y=0.380539in,,top]{\rmfamily\fontsize{8.000000}{9.600000}\selectfont −40}%
\end{pgfscope}%
\begin{pgfscope}%
\pgfsetbuttcap%
\pgfsetroundjoin%
\definecolor{currentfill}{rgb}{0.000000,0.000000,0.000000}%
\pgfsetfillcolor{currentfill}%
\pgfsetlinewidth{0.501875pt}%
\definecolor{currentstroke}{rgb}{0.000000,0.000000,0.000000}%
\pgfsetstrokecolor{currentstroke}%
\pgfsetdash{}{0pt}%
\pgfsys@defobject{currentmarker}{\pgfqpoint{0.000000in}{0.000000in}}{\pgfqpoint{0.000000in}{0.069444in}}{%
\pgfpathmoveto{\pgfqpoint{0.000000in}{0.000000in}}%
\pgfpathlineto{\pgfqpoint{0.000000in}{0.069444in}}%
\pgfusepath{stroke,fill}%
}%
\begin{pgfscope}%
\pgfsys@transformshift{1.851443in}{0.449983in}%
\pgfsys@useobject{currentmarker}{}%
\end{pgfscope}%
\end{pgfscope}%
\begin{pgfscope}%
\pgfsetbuttcap%
\pgfsetroundjoin%
\definecolor{currentfill}{rgb}{0.000000,0.000000,0.000000}%
\pgfsetfillcolor{currentfill}%
\pgfsetlinewidth{0.501875pt}%
\definecolor{currentstroke}{rgb}{0.000000,0.000000,0.000000}%
\pgfsetstrokecolor{currentstroke}%
\pgfsetdash{}{0pt}%
\pgfsys@defobject{currentmarker}{\pgfqpoint{0.000000in}{-0.069444in}}{\pgfqpoint{0.000000in}{0.000000in}}{%
\pgfpathmoveto{\pgfqpoint{0.000000in}{0.000000in}}%
\pgfpathlineto{\pgfqpoint{0.000000in}{-0.069444in}}%
\pgfusepath{stroke,fill}%
}%
\begin{pgfscope}%
\pgfsys@transformshift{1.851443in}{1.619432in}%
\pgfsys@useobject{currentmarker}{}%
\end{pgfscope}%
\end{pgfscope}%
\begin{pgfscope}%
\pgftext[x=1.851443in,y=0.380539in,,top]{\rmfamily\fontsize{8.000000}{9.600000}\selectfont −20}%
\end{pgfscope}%
\begin{pgfscope}%
\pgfsetbuttcap%
\pgfsetroundjoin%
\definecolor{currentfill}{rgb}{0.000000,0.000000,0.000000}%
\pgfsetfillcolor{currentfill}%
\pgfsetlinewidth{0.501875pt}%
\definecolor{currentstroke}{rgb}{0.000000,0.000000,0.000000}%
\pgfsetstrokecolor{currentstroke}%
\pgfsetdash{}{0pt}%
\pgfsys@defobject{currentmarker}{\pgfqpoint{0.000000in}{0.000000in}}{\pgfqpoint{0.000000in}{0.069444in}}{%
\pgfpathmoveto{\pgfqpoint{0.000000in}{0.000000in}}%
\pgfpathlineto{\pgfqpoint{0.000000in}{0.069444in}}%
\pgfusepath{stroke,fill}%
}%
\begin{pgfscope}%
\pgfsys@transformshift{2.087095in}{0.449983in}%
\pgfsys@useobject{currentmarker}{}%
\end{pgfscope}%
\end{pgfscope}%
\begin{pgfscope}%
\pgfsetbuttcap%
\pgfsetroundjoin%
\definecolor{currentfill}{rgb}{0.000000,0.000000,0.000000}%
\pgfsetfillcolor{currentfill}%
\pgfsetlinewidth{0.501875pt}%
\definecolor{currentstroke}{rgb}{0.000000,0.000000,0.000000}%
\pgfsetstrokecolor{currentstroke}%
\pgfsetdash{}{0pt}%
\pgfsys@defobject{currentmarker}{\pgfqpoint{0.000000in}{-0.069444in}}{\pgfqpoint{0.000000in}{0.000000in}}{%
\pgfpathmoveto{\pgfqpoint{0.000000in}{0.000000in}}%
\pgfpathlineto{\pgfqpoint{0.000000in}{-0.069444in}}%
\pgfusepath{stroke,fill}%
}%
\begin{pgfscope}%
\pgfsys@transformshift{2.087095in}{1.619432in}%
\pgfsys@useobject{currentmarker}{}%
\end{pgfscope}%
\end{pgfscope}%
\begin{pgfscope}%
\pgftext[x=2.087095in,y=0.380539in,,top]{\rmfamily\fontsize{8.000000}{9.600000}\selectfont 0}%
\end{pgfscope}%
\begin{pgfscope}%
\pgfsetbuttcap%
\pgfsetroundjoin%
\definecolor{currentfill}{rgb}{0.000000,0.000000,0.000000}%
\pgfsetfillcolor{currentfill}%
\pgfsetlinewidth{0.501875pt}%
\definecolor{currentstroke}{rgb}{0.000000,0.000000,0.000000}%
\pgfsetstrokecolor{currentstroke}%
\pgfsetdash{}{0pt}%
\pgfsys@defobject{currentmarker}{\pgfqpoint{0.000000in}{0.000000in}}{\pgfqpoint{0.000000in}{0.069444in}}{%
\pgfpathmoveto{\pgfqpoint{0.000000in}{0.000000in}}%
\pgfpathlineto{\pgfqpoint{0.000000in}{0.069444in}}%
\pgfusepath{stroke,fill}%
}%
\begin{pgfscope}%
\pgfsys@transformshift{2.322747in}{0.449983in}%
\pgfsys@useobject{currentmarker}{}%
\end{pgfscope}%
\end{pgfscope}%
\begin{pgfscope}%
\pgfsetbuttcap%
\pgfsetroundjoin%
\definecolor{currentfill}{rgb}{0.000000,0.000000,0.000000}%
\pgfsetfillcolor{currentfill}%
\pgfsetlinewidth{0.501875pt}%
\definecolor{currentstroke}{rgb}{0.000000,0.000000,0.000000}%
\pgfsetstrokecolor{currentstroke}%
\pgfsetdash{}{0pt}%
\pgfsys@defobject{currentmarker}{\pgfqpoint{0.000000in}{-0.069444in}}{\pgfqpoint{0.000000in}{0.000000in}}{%
\pgfpathmoveto{\pgfqpoint{0.000000in}{0.000000in}}%
\pgfpathlineto{\pgfqpoint{0.000000in}{-0.069444in}}%
\pgfusepath{stroke,fill}%
}%
\begin{pgfscope}%
\pgfsys@transformshift{2.322747in}{1.619432in}%
\pgfsys@useobject{currentmarker}{}%
\end{pgfscope}%
\end{pgfscope}%
\begin{pgfscope}%
\pgftext[x=2.322747in,y=0.380539in,,top]{\rmfamily\fontsize{8.000000}{9.600000}\selectfont 20}%
\end{pgfscope}%
\begin{pgfscope}%
\pgfsetbuttcap%
\pgfsetroundjoin%
\definecolor{currentfill}{rgb}{0.000000,0.000000,0.000000}%
\pgfsetfillcolor{currentfill}%
\pgfsetlinewidth{0.501875pt}%
\definecolor{currentstroke}{rgb}{0.000000,0.000000,0.000000}%
\pgfsetstrokecolor{currentstroke}%
\pgfsetdash{}{0pt}%
\pgfsys@defobject{currentmarker}{\pgfqpoint{0.000000in}{0.000000in}}{\pgfqpoint{0.000000in}{0.069444in}}{%
\pgfpathmoveto{\pgfqpoint{0.000000in}{0.000000in}}%
\pgfpathlineto{\pgfqpoint{0.000000in}{0.069444in}}%
\pgfusepath{stroke,fill}%
}%
\begin{pgfscope}%
\pgfsys@transformshift{2.558398in}{0.449983in}%
\pgfsys@useobject{currentmarker}{}%
\end{pgfscope}%
\end{pgfscope}%
\begin{pgfscope}%
\pgfsetbuttcap%
\pgfsetroundjoin%
\definecolor{currentfill}{rgb}{0.000000,0.000000,0.000000}%
\pgfsetfillcolor{currentfill}%
\pgfsetlinewidth{0.501875pt}%
\definecolor{currentstroke}{rgb}{0.000000,0.000000,0.000000}%
\pgfsetstrokecolor{currentstroke}%
\pgfsetdash{}{0pt}%
\pgfsys@defobject{currentmarker}{\pgfqpoint{0.000000in}{-0.069444in}}{\pgfqpoint{0.000000in}{0.000000in}}{%
\pgfpathmoveto{\pgfqpoint{0.000000in}{0.000000in}}%
\pgfpathlineto{\pgfqpoint{0.000000in}{-0.069444in}}%
\pgfusepath{stroke,fill}%
}%
\begin{pgfscope}%
\pgfsys@transformshift{2.558398in}{1.619432in}%
\pgfsys@useobject{currentmarker}{}%
\end{pgfscope}%
\end{pgfscope}%
\begin{pgfscope}%
\pgftext[x=2.558398in,y=0.380539in,,top]{\rmfamily\fontsize{8.000000}{9.600000}\selectfont 40}%
\end{pgfscope}%
\begin{pgfscope}%
\pgftext[x=1.497965in,y=0.203564in,,top]{\rmfamily\fontsize{9.000000}{10.800000}\selectfont \(\displaystyle \mathrm{DLL}_{K/\pi}(\pi^-)\)}%
\end{pgfscope}%
\begin{pgfscope}%
\pgfsetbuttcap%
\pgfsetroundjoin%
\definecolor{currentfill}{rgb}{0.000000,0.000000,0.000000}%
\pgfsetfillcolor{currentfill}%
\pgfsetlinewidth{0.501875pt}%
\definecolor{currentstroke}{rgb}{0.000000,0.000000,0.000000}%
\pgfsetstrokecolor{currentstroke}%
\pgfsetdash{}{0pt}%
\pgfsys@defobject{currentmarker}{\pgfqpoint{0.000000in}{0.000000in}}{\pgfqpoint{0.069444in}{0.000000in}}{%
\pgfpathmoveto{\pgfqpoint{0.000000in}{0.000000in}}%
\pgfpathlineto{\pgfqpoint{0.069444in}{0.000000in}}%
\pgfusepath{stroke,fill}%
}%
\begin{pgfscope}%
\pgfsys@transformshift{0.437532in}{0.449983in}%
\pgfsys@useobject{currentmarker}{}%
\end{pgfscope}%
\end{pgfscope}%
\begin{pgfscope}%
\pgfsetbuttcap%
\pgfsetroundjoin%
\definecolor{currentfill}{rgb}{0.000000,0.000000,0.000000}%
\pgfsetfillcolor{currentfill}%
\pgfsetlinewidth{0.501875pt}%
\definecolor{currentstroke}{rgb}{0.000000,0.000000,0.000000}%
\pgfsetstrokecolor{currentstroke}%
\pgfsetdash{}{0pt}%
\pgfsys@defobject{currentmarker}{\pgfqpoint{-0.069444in}{0.000000in}}{\pgfqpoint{0.000000in}{0.000000in}}{%
\pgfpathmoveto{\pgfqpoint{0.000000in}{0.000000in}}%
\pgfpathlineto{\pgfqpoint{-0.069444in}{0.000000in}}%
\pgfusepath{stroke,fill}%
}%
\begin{pgfscope}%
\pgfsys@transformshift{2.558398in}{0.449983in}%
\pgfsys@useobject{currentmarker}{}%
\end{pgfscope}%
\end{pgfscope}%
\begin{pgfscope}%
\pgftext[x=0.368088in,y=0.449983in,right,]{\rmfamily\fontsize{8.000000}{9.600000}\selectfont 0.000}%
\end{pgfscope}%
\begin{pgfscope}%
\pgfsetbuttcap%
\pgfsetroundjoin%
\definecolor{currentfill}{rgb}{0.000000,0.000000,0.000000}%
\pgfsetfillcolor{currentfill}%
\pgfsetlinewidth{0.501875pt}%
\definecolor{currentstroke}{rgb}{0.000000,0.000000,0.000000}%
\pgfsetstrokecolor{currentstroke}%
\pgfsetdash{}{0pt}%
\pgfsys@defobject{currentmarker}{\pgfqpoint{0.000000in}{0.000000in}}{\pgfqpoint{0.069444in}{0.000000in}}{%
\pgfpathmoveto{\pgfqpoint{0.000000in}{0.000000in}}%
\pgfpathlineto{\pgfqpoint{0.069444in}{0.000000in}}%
\pgfusepath{stroke,fill}%
}%
\begin{pgfscope}%
\pgfsys@transformshift{0.437532in}{0.596165in}%
\pgfsys@useobject{currentmarker}{}%
\end{pgfscope}%
\end{pgfscope}%
\begin{pgfscope}%
\pgfsetbuttcap%
\pgfsetroundjoin%
\definecolor{currentfill}{rgb}{0.000000,0.000000,0.000000}%
\pgfsetfillcolor{currentfill}%
\pgfsetlinewidth{0.501875pt}%
\definecolor{currentstroke}{rgb}{0.000000,0.000000,0.000000}%
\pgfsetstrokecolor{currentstroke}%
\pgfsetdash{}{0pt}%
\pgfsys@defobject{currentmarker}{\pgfqpoint{-0.069444in}{0.000000in}}{\pgfqpoint{0.000000in}{0.000000in}}{%
\pgfpathmoveto{\pgfqpoint{0.000000in}{0.000000in}}%
\pgfpathlineto{\pgfqpoint{-0.069444in}{0.000000in}}%
\pgfusepath{stroke,fill}%
}%
\begin{pgfscope}%
\pgfsys@transformshift{2.558398in}{0.596165in}%
\pgfsys@useobject{currentmarker}{}%
\end{pgfscope}%
\end{pgfscope}%
\begin{pgfscope}%
\pgftext[x=0.368088in,y=0.596165in,right,]{\rmfamily\fontsize{8.000000}{9.600000}\selectfont 0.005}%
\end{pgfscope}%
\begin{pgfscope}%
\pgfsetbuttcap%
\pgfsetroundjoin%
\definecolor{currentfill}{rgb}{0.000000,0.000000,0.000000}%
\pgfsetfillcolor{currentfill}%
\pgfsetlinewidth{0.501875pt}%
\definecolor{currentstroke}{rgb}{0.000000,0.000000,0.000000}%
\pgfsetstrokecolor{currentstroke}%
\pgfsetdash{}{0pt}%
\pgfsys@defobject{currentmarker}{\pgfqpoint{0.000000in}{0.000000in}}{\pgfqpoint{0.069444in}{0.000000in}}{%
\pgfpathmoveto{\pgfqpoint{0.000000in}{0.000000in}}%
\pgfpathlineto{\pgfqpoint{0.069444in}{0.000000in}}%
\pgfusepath{stroke,fill}%
}%
\begin{pgfscope}%
\pgfsys@transformshift{0.437532in}{0.742346in}%
\pgfsys@useobject{currentmarker}{}%
\end{pgfscope}%
\end{pgfscope}%
\begin{pgfscope}%
\pgfsetbuttcap%
\pgfsetroundjoin%
\definecolor{currentfill}{rgb}{0.000000,0.000000,0.000000}%
\pgfsetfillcolor{currentfill}%
\pgfsetlinewidth{0.501875pt}%
\definecolor{currentstroke}{rgb}{0.000000,0.000000,0.000000}%
\pgfsetstrokecolor{currentstroke}%
\pgfsetdash{}{0pt}%
\pgfsys@defobject{currentmarker}{\pgfqpoint{-0.069444in}{0.000000in}}{\pgfqpoint{0.000000in}{0.000000in}}{%
\pgfpathmoveto{\pgfqpoint{0.000000in}{0.000000in}}%
\pgfpathlineto{\pgfqpoint{-0.069444in}{0.000000in}}%
\pgfusepath{stroke,fill}%
}%
\begin{pgfscope}%
\pgfsys@transformshift{2.558398in}{0.742346in}%
\pgfsys@useobject{currentmarker}{}%
\end{pgfscope}%
\end{pgfscope}%
\begin{pgfscope}%
\pgftext[x=0.368088in,y=0.742346in,right,]{\rmfamily\fontsize{8.000000}{9.600000}\selectfont 0.010}%
\end{pgfscope}%
\begin{pgfscope}%
\pgfsetbuttcap%
\pgfsetroundjoin%
\definecolor{currentfill}{rgb}{0.000000,0.000000,0.000000}%
\pgfsetfillcolor{currentfill}%
\pgfsetlinewidth{0.501875pt}%
\definecolor{currentstroke}{rgb}{0.000000,0.000000,0.000000}%
\pgfsetstrokecolor{currentstroke}%
\pgfsetdash{}{0pt}%
\pgfsys@defobject{currentmarker}{\pgfqpoint{0.000000in}{0.000000in}}{\pgfqpoint{0.069444in}{0.000000in}}{%
\pgfpathmoveto{\pgfqpoint{0.000000in}{0.000000in}}%
\pgfpathlineto{\pgfqpoint{0.069444in}{0.000000in}}%
\pgfusepath{stroke,fill}%
}%
\begin{pgfscope}%
\pgfsys@transformshift{0.437532in}{0.888527in}%
\pgfsys@useobject{currentmarker}{}%
\end{pgfscope}%
\end{pgfscope}%
\begin{pgfscope}%
\pgfsetbuttcap%
\pgfsetroundjoin%
\definecolor{currentfill}{rgb}{0.000000,0.000000,0.000000}%
\pgfsetfillcolor{currentfill}%
\pgfsetlinewidth{0.501875pt}%
\definecolor{currentstroke}{rgb}{0.000000,0.000000,0.000000}%
\pgfsetstrokecolor{currentstroke}%
\pgfsetdash{}{0pt}%
\pgfsys@defobject{currentmarker}{\pgfqpoint{-0.069444in}{0.000000in}}{\pgfqpoint{0.000000in}{0.000000in}}{%
\pgfpathmoveto{\pgfqpoint{0.000000in}{0.000000in}}%
\pgfpathlineto{\pgfqpoint{-0.069444in}{0.000000in}}%
\pgfusepath{stroke,fill}%
}%
\begin{pgfscope}%
\pgfsys@transformshift{2.558398in}{0.888527in}%
\pgfsys@useobject{currentmarker}{}%
\end{pgfscope}%
\end{pgfscope}%
\begin{pgfscope}%
\pgftext[x=0.368088in,y=0.888527in,right,]{\rmfamily\fontsize{8.000000}{9.600000}\selectfont 0.015}%
\end{pgfscope}%
\begin{pgfscope}%
\pgfsetbuttcap%
\pgfsetroundjoin%
\definecolor{currentfill}{rgb}{0.000000,0.000000,0.000000}%
\pgfsetfillcolor{currentfill}%
\pgfsetlinewidth{0.501875pt}%
\definecolor{currentstroke}{rgb}{0.000000,0.000000,0.000000}%
\pgfsetstrokecolor{currentstroke}%
\pgfsetdash{}{0pt}%
\pgfsys@defobject{currentmarker}{\pgfqpoint{0.000000in}{0.000000in}}{\pgfqpoint{0.069444in}{0.000000in}}{%
\pgfpathmoveto{\pgfqpoint{0.000000in}{0.000000in}}%
\pgfpathlineto{\pgfqpoint{0.069444in}{0.000000in}}%
\pgfusepath{stroke,fill}%
}%
\begin{pgfscope}%
\pgfsys@transformshift{0.437532in}{1.034708in}%
\pgfsys@useobject{currentmarker}{}%
\end{pgfscope}%
\end{pgfscope}%
\begin{pgfscope}%
\pgfsetbuttcap%
\pgfsetroundjoin%
\definecolor{currentfill}{rgb}{0.000000,0.000000,0.000000}%
\pgfsetfillcolor{currentfill}%
\pgfsetlinewidth{0.501875pt}%
\definecolor{currentstroke}{rgb}{0.000000,0.000000,0.000000}%
\pgfsetstrokecolor{currentstroke}%
\pgfsetdash{}{0pt}%
\pgfsys@defobject{currentmarker}{\pgfqpoint{-0.069444in}{0.000000in}}{\pgfqpoint{0.000000in}{0.000000in}}{%
\pgfpathmoveto{\pgfqpoint{0.000000in}{0.000000in}}%
\pgfpathlineto{\pgfqpoint{-0.069444in}{0.000000in}}%
\pgfusepath{stroke,fill}%
}%
\begin{pgfscope}%
\pgfsys@transformshift{2.558398in}{1.034708in}%
\pgfsys@useobject{currentmarker}{}%
\end{pgfscope}%
\end{pgfscope}%
\begin{pgfscope}%
\pgftext[x=0.368088in,y=1.034708in,right,]{\rmfamily\fontsize{8.000000}{9.600000}\selectfont 0.020}%
\end{pgfscope}%
\begin{pgfscope}%
\pgfsetbuttcap%
\pgfsetroundjoin%
\definecolor{currentfill}{rgb}{0.000000,0.000000,0.000000}%
\pgfsetfillcolor{currentfill}%
\pgfsetlinewidth{0.501875pt}%
\definecolor{currentstroke}{rgb}{0.000000,0.000000,0.000000}%
\pgfsetstrokecolor{currentstroke}%
\pgfsetdash{}{0pt}%
\pgfsys@defobject{currentmarker}{\pgfqpoint{0.000000in}{0.000000in}}{\pgfqpoint{0.069444in}{0.000000in}}{%
\pgfpathmoveto{\pgfqpoint{0.000000in}{0.000000in}}%
\pgfpathlineto{\pgfqpoint{0.069444in}{0.000000in}}%
\pgfusepath{stroke,fill}%
}%
\begin{pgfscope}%
\pgfsys@transformshift{0.437532in}{1.180889in}%
\pgfsys@useobject{currentmarker}{}%
\end{pgfscope}%
\end{pgfscope}%
\begin{pgfscope}%
\pgfsetbuttcap%
\pgfsetroundjoin%
\definecolor{currentfill}{rgb}{0.000000,0.000000,0.000000}%
\pgfsetfillcolor{currentfill}%
\pgfsetlinewidth{0.501875pt}%
\definecolor{currentstroke}{rgb}{0.000000,0.000000,0.000000}%
\pgfsetstrokecolor{currentstroke}%
\pgfsetdash{}{0pt}%
\pgfsys@defobject{currentmarker}{\pgfqpoint{-0.069444in}{0.000000in}}{\pgfqpoint{0.000000in}{0.000000in}}{%
\pgfpathmoveto{\pgfqpoint{0.000000in}{0.000000in}}%
\pgfpathlineto{\pgfqpoint{-0.069444in}{0.000000in}}%
\pgfusepath{stroke,fill}%
}%
\begin{pgfscope}%
\pgfsys@transformshift{2.558398in}{1.180889in}%
\pgfsys@useobject{currentmarker}{}%
\end{pgfscope}%
\end{pgfscope}%
\begin{pgfscope}%
\pgftext[x=0.368088in,y=1.180889in,right,]{\rmfamily\fontsize{8.000000}{9.600000}\selectfont 0.025}%
\end{pgfscope}%
\begin{pgfscope}%
\pgfsetbuttcap%
\pgfsetroundjoin%
\definecolor{currentfill}{rgb}{0.000000,0.000000,0.000000}%
\pgfsetfillcolor{currentfill}%
\pgfsetlinewidth{0.501875pt}%
\definecolor{currentstroke}{rgb}{0.000000,0.000000,0.000000}%
\pgfsetstrokecolor{currentstroke}%
\pgfsetdash{}{0pt}%
\pgfsys@defobject{currentmarker}{\pgfqpoint{0.000000in}{0.000000in}}{\pgfqpoint{0.069444in}{0.000000in}}{%
\pgfpathmoveto{\pgfqpoint{0.000000in}{0.000000in}}%
\pgfpathlineto{\pgfqpoint{0.069444in}{0.000000in}}%
\pgfusepath{stroke,fill}%
}%
\begin{pgfscope}%
\pgfsys@transformshift{0.437532in}{1.327070in}%
\pgfsys@useobject{currentmarker}{}%
\end{pgfscope}%
\end{pgfscope}%
\begin{pgfscope}%
\pgfsetbuttcap%
\pgfsetroundjoin%
\definecolor{currentfill}{rgb}{0.000000,0.000000,0.000000}%
\pgfsetfillcolor{currentfill}%
\pgfsetlinewidth{0.501875pt}%
\definecolor{currentstroke}{rgb}{0.000000,0.000000,0.000000}%
\pgfsetstrokecolor{currentstroke}%
\pgfsetdash{}{0pt}%
\pgfsys@defobject{currentmarker}{\pgfqpoint{-0.069444in}{0.000000in}}{\pgfqpoint{0.000000in}{0.000000in}}{%
\pgfpathmoveto{\pgfqpoint{0.000000in}{0.000000in}}%
\pgfpathlineto{\pgfqpoint{-0.069444in}{0.000000in}}%
\pgfusepath{stroke,fill}%
}%
\begin{pgfscope}%
\pgfsys@transformshift{2.558398in}{1.327070in}%
\pgfsys@useobject{currentmarker}{}%
\end{pgfscope}%
\end{pgfscope}%
\begin{pgfscope}%
\pgftext[x=0.368088in,y=1.327070in,right,]{\rmfamily\fontsize{8.000000}{9.600000}\selectfont 0.030}%
\end{pgfscope}%
\begin{pgfscope}%
\pgfsetbuttcap%
\pgfsetroundjoin%
\definecolor{currentfill}{rgb}{0.000000,0.000000,0.000000}%
\pgfsetfillcolor{currentfill}%
\pgfsetlinewidth{0.501875pt}%
\definecolor{currentstroke}{rgb}{0.000000,0.000000,0.000000}%
\pgfsetstrokecolor{currentstroke}%
\pgfsetdash{}{0pt}%
\pgfsys@defobject{currentmarker}{\pgfqpoint{0.000000in}{0.000000in}}{\pgfqpoint{0.069444in}{0.000000in}}{%
\pgfpathmoveto{\pgfqpoint{0.000000in}{0.000000in}}%
\pgfpathlineto{\pgfqpoint{0.069444in}{0.000000in}}%
\pgfusepath{stroke,fill}%
}%
\begin{pgfscope}%
\pgfsys@transformshift{0.437532in}{1.473251in}%
\pgfsys@useobject{currentmarker}{}%
\end{pgfscope}%
\end{pgfscope}%
\begin{pgfscope}%
\pgfsetbuttcap%
\pgfsetroundjoin%
\definecolor{currentfill}{rgb}{0.000000,0.000000,0.000000}%
\pgfsetfillcolor{currentfill}%
\pgfsetlinewidth{0.501875pt}%
\definecolor{currentstroke}{rgb}{0.000000,0.000000,0.000000}%
\pgfsetstrokecolor{currentstroke}%
\pgfsetdash{}{0pt}%
\pgfsys@defobject{currentmarker}{\pgfqpoint{-0.069444in}{0.000000in}}{\pgfqpoint{0.000000in}{0.000000in}}{%
\pgfpathmoveto{\pgfqpoint{0.000000in}{0.000000in}}%
\pgfpathlineto{\pgfqpoint{-0.069444in}{0.000000in}}%
\pgfusepath{stroke,fill}%
}%
\begin{pgfscope}%
\pgfsys@transformshift{2.558398in}{1.473251in}%
\pgfsys@useobject{currentmarker}{}%
\end{pgfscope}%
\end{pgfscope}%
\begin{pgfscope}%
\pgftext[x=0.368088in,y=1.473251in,right,]{\rmfamily\fontsize{8.000000}{9.600000}\selectfont 0.035}%
\end{pgfscope}%
\begin{pgfscope}%
\pgfsetbuttcap%
\pgfsetroundjoin%
\definecolor{currentfill}{rgb}{0.000000,0.000000,0.000000}%
\pgfsetfillcolor{currentfill}%
\pgfsetlinewidth{0.501875pt}%
\definecolor{currentstroke}{rgb}{0.000000,0.000000,0.000000}%
\pgfsetstrokecolor{currentstroke}%
\pgfsetdash{}{0pt}%
\pgfsys@defobject{currentmarker}{\pgfqpoint{0.000000in}{0.000000in}}{\pgfqpoint{0.069444in}{0.000000in}}{%
\pgfpathmoveto{\pgfqpoint{0.000000in}{0.000000in}}%
\pgfpathlineto{\pgfqpoint{0.069444in}{0.000000in}}%
\pgfusepath{stroke,fill}%
}%
\begin{pgfscope}%
\pgfsys@transformshift{0.437532in}{1.619432in}%
\pgfsys@useobject{currentmarker}{}%
\end{pgfscope}%
\end{pgfscope}%
\begin{pgfscope}%
\pgfsetbuttcap%
\pgfsetroundjoin%
\definecolor{currentfill}{rgb}{0.000000,0.000000,0.000000}%
\pgfsetfillcolor{currentfill}%
\pgfsetlinewidth{0.501875pt}%
\definecolor{currentstroke}{rgb}{0.000000,0.000000,0.000000}%
\pgfsetstrokecolor{currentstroke}%
\pgfsetdash{}{0pt}%
\pgfsys@defobject{currentmarker}{\pgfqpoint{-0.069444in}{0.000000in}}{\pgfqpoint{0.000000in}{0.000000in}}{%
\pgfpathmoveto{\pgfqpoint{0.000000in}{0.000000in}}%
\pgfpathlineto{\pgfqpoint{-0.069444in}{0.000000in}}%
\pgfusepath{stroke,fill}%
}%
\begin{pgfscope}%
\pgfsys@transformshift{2.558398in}{1.619432in}%
\pgfsys@useobject{currentmarker}{}%
\end{pgfscope}%
\end{pgfscope}%
\begin{pgfscope}%
\pgftext[x=0.368088in,y=1.619432in,right,]{\rmfamily\fontsize{8.000000}{9.600000}\selectfont 0.040}%
\end{pgfscope}%
\end{pgfpicture}%
\makeatother%
\endgroup%

	\end{subfigure}
	\begin{subfigure}[t]{0.49\textwidth}
		\centering
    %\includegraphics[width=\textwidth]{store/variables/DATA_MC_piminus_PIDmu.pdf}
    \input{store/variables/DATA_MC_piminus_PIDmu.pgf}
	\end{subfigure}

	\begin{subfigure}[t]{0.49\textwidth}
		\centering
    %\includegraphics[width=\textwidth]{store/variables/DATA_MC_muminus_PIDmu.pdf}
    \input{store/variables/DATA_MC_muminus_PIDmu.pgf}
	\end{subfigure}
	\begin{subfigure}[t]{0.49\textwidth}
		\centering
    %\includegraphics[width=\textwidth]{store/variables/DATA_MC_muplus_PIDmu.pdf}
    \input{store/variables/DATA_MC_muplus_PIDmu.pgf}
	\end{subfigure}

	\caption{
    Input variables after bin-based reweighting and resampling of PID variables (orange).
    The original simulated dataset (red) and data distribution (blue) are  given as a reference.
  }
  \thisfloatpagestyle{empty}
  \label{fig:mcfeaturesresampled}
\end{figure}

\subsection{Classifier-driven reweighting}
\label{reweighting}

An alternative reweighting scheme is used as follows:
A classifier is trained to discriminate between a real and simulated sample of a control channel (in this case $\PBzero\to\PJpsi\PKstar$).
The classifier is then used to predict posterior class membership probabilities $p$ for the dataset that needs to be reweighted.
The ratio $p / (1 - p)$ is used as a weight that removes data-simulation differences.

In this case, the classifier has been applied to the same simulated dataset used to train it using $2$-fold cross-validation.
The weighted input variables are given in figure \ref{fig:mcfeaturesreweighted}.
The classifier response is given in figure \ref{fig:reweightedresponse}.

A large reduction in the efficiency ratio down to a discrepancy of \SI{1}{\percent} can be observed.
This validates the basic feasibility of the reweighting method.
It remains to be seen how transferable the reweighting procedure is, i.e. in how far an application of the trained classifier to the signal channel $\PBzero\to\APDzero\APmuon\Pmuon$ also leads to a data-like classifier response.

\begin{figure}
  \centering
  \input{store/variables/DATA_MC_REWEIGHT_clf.pgf}
  \caption{
    Classifier response on the fully reweighted simulated sample.
    The original simulated dataset (red) and data distribution (blue) are  given as a reference.
  }
  \label{fig:reweightedresponse}
\end{figure}

\begin{figure}	
	\centering
	\begin{subfigure}[t]{0.49\textwidth}
		\centering
    %\includegraphics[width=\textwidth]{store/variables/DATA_MC_REWEIGHTED_B_DiraAngle.pdf}
    \input{store/variables/DATA_MC_REWEIGHT_B_DiraAngle.pgf}
	\end{subfigure}
	\begin{subfigure}[t]{0.49\textwidth}
		\centering
    %\includegraphics[width=\textwidth]{store/variables/DATA_MC_REWEIGHTED_B_ENDVERTEX_CHI2_NDOF.pdf}
    %% Creator: Matplotlib, PGF backend
%%
%% To include the figure in your LaTeX document, write
%%   \input{<filename>.pgf}
%%
%% Make sure the required packages are loaded in your preamble
%%   \usepackage{pgf}
%%
%% Figures using additional raster images can only be included by \input if
%% they are in the same directory as the main LaTeX file. For loading figures
%% from other directories you can use the `import` package
%%   \usepackage{import}
%% and then include the figures with
%%   \import{<path to file>}{<filename>.pgf}
%%
%% Matplotlib used the following preamble
%%   \usepackage{fontspec}
%%   \setmainfont{DejaVu Serif}
%%   \setsansfont{DejaVu Sans}
%%   \setmonofont{DejaVu Sans Mono}
%%
\begingroup%
\makeatletter%
\begin{pgfpicture}%
\pgfpathrectangle{\pgfpointorigin}{\pgfqpoint{2.678086in}{1.718727in}}%
\pgfusepath{use as bounding box, clip}%
\begin{pgfscope}%
\pgfsetbuttcap%
\pgfsetmiterjoin%
\definecolor{currentfill}{rgb}{1.000000,1.000000,1.000000}%
\pgfsetfillcolor{currentfill}%
\pgfsetlinewidth{0.000000pt}%
\definecolor{currentstroke}{rgb}{1.000000,1.000000,1.000000}%
\pgfsetstrokecolor{currentstroke}%
\pgfsetdash{}{0pt}%
\pgfpathmoveto{\pgfqpoint{0.000000in}{-0.000000in}}%
\pgfpathlineto{\pgfqpoint{2.678086in}{-0.000000in}}%
\pgfpathlineto{\pgfqpoint{2.678086in}{1.718727in}}%
\pgfpathlineto{\pgfqpoint{0.000000in}{1.718727in}}%
\pgfpathclose%
\pgfusepath{fill}%
\end{pgfscope}%
\begin{pgfscope}%
\pgfsetbuttcap%
\pgfsetmiterjoin%
\definecolor{currentfill}{rgb}{1.000000,1.000000,1.000000}%
\pgfsetfillcolor{currentfill}%
\pgfsetlinewidth{0.000000pt}%
\definecolor{currentstroke}{rgb}{0.000000,0.000000,0.000000}%
\pgfsetstrokecolor{currentstroke}%
\pgfsetstrokeopacity{0.000000}%
\pgfsetdash{}{0pt}%
\pgfpathmoveto{\pgfqpoint{0.296148in}{0.441418in}}%
\pgfpathlineto{\pgfqpoint{2.592740in}{0.441418in}}%
\pgfpathlineto{\pgfqpoint{2.592740in}{1.614961in}}%
\pgfpathlineto{\pgfqpoint{0.296148in}{1.614961in}}%
\pgfpathclose%
\pgfusepath{fill}%
\end{pgfscope}%
\begin{pgfscope}%
\pgfpathrectangle{\pgfqpoint{0.296148in}{0.441418in}}{\pgfqpoint{2.296592in}{1.173543in}} %
\pgfusepath{clip}%
\pgfsetbuttcap%
\pgfsetmiterjoin%
\definecolor{currentfill}{rgb}{0.215686,0.470588,0.749020}%
\pgfsetfillcolor{currentfill}%
\pgfsetlinewidth{0.000000pt}%
\definecolor{currentstroke}{rgb}{0.000000,0.000000,0.000000}%
\pgfsetstrokecolor{currentstroke}%
\pgfsetdash{}{0pt}%
\pgfpathmoveto{\pgfqpoint{0.297424in}{0.441418in}}%
\pgfpathlineto{\pgfqpoint{0.297424in}{0.609709in}}%
\pgfpathlineto{\pgfqpoint{0.343328in}{0.609709in}}%
\pgfpathlineto{\pgfqpoint{0.343328in}{0.985601in}}%
\pgfpathlineto{\pgfqpoint{0.389233in}{0.985601in}}%
\pgfpathlineto{\pgfqpoint{0.389233in}{1.251138in}}%
\pgfpathlineto{\pgfqpoint{0.435137in}{1.251138in}}%
\pgfpathlineto{\pgfqpoint{0.435137in}{1.391596in}}%
\pgfpathlineto{\pgfqpoint{0.481042in}{1.391596in}}%
\pgfpathlineto{\pgfqpoint{0.481042in}{1.426427in}}%
\pgfpathlineto{\pgfqpoint{0.526946in}{1.426427in}}%
\pgfpathlineto{\pgfqpoint{0.526946in}{1.417278in}}%
\pgfpathlineto{\pgfqpoint{0.572851in}{1.417278in}}%
\pgfpathlineto{\pgfqpoint{0.572851in}{1.349906in}}%
\pgfpathlineto{\pgfqpoint{0.618755in}{1.349906in}}%
\pgfpathlineto{\pgfqpoint{0.618755in}{1.270193in}}%
\pgfpathlineto{\pgfqpoint{0.664660in}{1.270193in}}%
\pgfpathlineto{\pgfqpoint{0.664660in}{1.189248in}}%
\pgfpathlineto{\pgfqpoint{0.710564in}{1.189248in}}%
\pgfpathlineto{\pgfqpoint{0.710564in}{1.110275in}}%
\pgfpathlineto{\pgfqpoint{0.756469in}{1.110275in}}%
\pgfpathlineto{\pgfqpoint{0.756469in}{1.039528in}}%
\pgfpathlineto{\pgfqpoint{0.802373in}{1.039528in}}%
\pgfpathlineto{\pgfqpoint{0.802373in}{0.959669in}}%
\pgfpathlineto{\pgfqpoint{0.848278in}{0.959669in}}%
\pgfpathlineto{\pgfqpoint{0.848278in}{0.901037in}}%
\pgfpathlineto{\pgfqpoint{0.894182in}{0.901037in}}%
\pgfpathlineto{\pgfqpoint{0.894182in}{0.843994in}}%
\pgfpathlineto{\pgfqpoint{0.940087in}{0.843994in}}%
\pgfpathlineto{\pgfqpoint{0.940087in}{0.784292in}}%
\pgfpathlineto{\pgfqpoint{0.985991in}{0.784292in}}%
\pgfpathlineto{\pgfqpoint{0.985991in}{0.750250in}}%
\pgfpathlineto{\pgfqpoint{1.031896in}{0.750250in}}%
\pgfpathlineto{\pgfqpoint{1.031896in}{0.705507in}}%
\pgfpathlineto{\pgfqpoint{1.077800in}{0.705507in}}%
\pgfpathlineto{\pgfqpoint{1.077800in}{0.674539in}}%
\pgfpathlineto{\pgfqpoint{1.123705in}{0.674539in}}%
\pgfpathlineto{\pgfqpoint{1.123705in}{0.634008in}}%
\pgfpathlineto{\pgfqpoint{1.169609in}{0.634008in}}%
\pgfpathlineto{\pgfqpoint{1.169609in}{0.609502in}}%
\pgfpathlineto{\pgfqpoint{1.215514in}{0.609502in}}%
\pgfpathlineto{\pgfqpoint{1.215514in}{0.586251in}}%
\pgfpathlineto{\pgfqpoint{1.261418in}{0.586251in}}%
\pgfpathlineto{\pgfqpoint{1.261418in}{0.569989in}}%
\pgfpathlineto{\pgfqpoint{1.307323in}{0.569989in}}%
\pgfpathlineto{\pgfqpoint{1.307323in}{0.551306in}}%
\pgfpathlineto{\pgfqpoint{1.353227in}{0.551306in}}%
\pgfpathlineto{\pgfqpoint{1.353227in}{0.537044in}}%
\pgfpathlineto{\pgfqpoint{1.399132in}{0.537044in}}%
\pgfpathlineto{\pgfqpoint{1.399132in}{0.526890in}}%
\pgfpathlineto{\pgfqpoint{1.445036in}{0.526890in}}%
\pgfpathlineto{\pgfqpoint{1.445036in}{0.511423in}}%
\pgfpathlineto{\pgfqpoint{1.490941in}{0.511423in}}%
\pgfpathlineto{\pgfqpoint{1.490941in}{0.499708in}}%
\pgfpathlineto{\pgfqpoint{1.536845in}{0.499708in}}%
\pgfpathlineto{\pgfqpoint{1.536845in}{0.494453in}}%
\pgfpathlineto{\pgfqpoint{1.582750in}{0.494453in}}%
\pgfpathlineto{\pgfqpoint{1.582750in}{0.488950in}}%
\pgfpathlineto{\pgfqpoint{1.628654in}{0.488950in}}%
\pgfpathlineto{\pgfqpoint{1.628654in}{0.482587in}}%
\pgfpathlineto{\pgfqpoint{1.674559in}{0.482587in}}%
\pgfpathlineto{\pgfqpoint{1.674559in}{0.478364in}}%
\pgfpathlineto{\pgfqpoint{1.720463in}{0.478364in}}%
\pgfpathlineto{\pgfqpoint{1.720463in}{0.475954in}}%
\pgfpathlineto{\pgfqpoint{1.766368in}{0.475954in}}%
\pgfpathlineto{\pgfqpoint{1.766368in}{0.471715in}}%
\pgfpathlineto{\pgfqpoint{1.812272in}{0.471715in}}%
\pgfpathlineto{\pgfqpoint{1.812272in}{0.465497in}}%
\pgfpathlineto{\pgfqpoint{1.858177in}{0.465497in}}%
\pgfpathlineto{\pgfqpoint{1.858177in}{0.465883in}}%
\pgfpathlineto{\pgfqpoint{1.904081in}{0.465883in}}%
\pgfpathlineto{\pgfqpoint{1.904081in}{0.461012in}}%
\pgfpathlineto{\pgfqpoint{1.949986in}{0.461012in}}%
\pgfpathlineto{\pgfqpoint{1.949986in}{0.460840in}}%
\pgfpathlineto{\pgfqpoint{1.995890in}{0.460840in}}%
\pgfpathlineto{\pgfqpoint{1.995890in}{0.457723in}}%
\pgfpathlineto{\pgfqpoint{2.041795in}{0.457723in}}%
\pgfpathlineto{\pgfqpoint{2.041795in}{0.456838in}}%
\pgfpathlineto{\pgfqpoint{2.087699in}{0.456838in}}%
\pgfpathlineto{\pgfqpoint{2.087699in}{0.453616in}}%
\pgfpathlineto{\pgfqpoint{2.133604in}{0.453616in}}%
\pgfpathlineto{\pgfqpoint{2.133604in}{0.452813in}}%
\pgfpathlineto{\pgfqpoint{2.179508in}{0.452813in}}%
\pgfpathlineto{\pgfqpoint{2.179508in}{0.454010in}}%
\pgfpathlineto{\pgfqpoint{2.225413in}{0.454010in}}%
\pgfpathlineto{\pgfqpoint{2.225413in}{0.450271in}}%
\pgfpathlineto{\pgfqpoint{2.271317in}{0.450271in}}%
\pgfpathlineto{\pgfqpoint{2.271317in}{0.450847in}}%
\pgfpathlineto{\pgfqpoint{2.317222in}{0.450847in}}%
\pgfpathlineto{\pgfqpoint{2.317222in}{0.450423in}}%
\pgfpathlineto{\pgfqpoint{2.363126in}{0.450423in}}%
\pgfpathlineto{\pgfqpoint{2.363126in}{0.449278in}}%
\pgfpathlineto{\pgfqpoint{2.409031in}{0.449278in}}%
\pgfpathlineto{\pgfqpoint{2.409031in}{0.449574in}}%
\pgfpathlineto{\pgfqpoint{2.454935in}{0.449574in}}%
\pgfpathlineto{\pgfqpoint{2.454935in}{0.449361in}}%
\pgfpathlineto{\pgfqpoint{2.500840in}{0.449361in}}%
\pgfpathlineto{\pgfqpoint{2.500840in}{0.447843in}}%
\pgfpathlineto{\pgfqpoint{2.546744in}{0.447843in}}%
\pgfpathlineto{\pgfqpoint{2.546744in}{0.448453in}}%
\pgfpathlineto{\pgfqpoint{2.592649in}{0.448453in}}%
\pgfpathlineto{\pgfqpoint{2.592649in}{0.441418in}}%
\pgfpathlineto{\pgfqpoint{2.546744in}{0.441418in}}%
\pgfpathlineto{\pgfqpoint{2.546744in}{0.441418in}}%
\pgfpathlineto{\pgfqpoint{2.500840in}{0.441418in}}%
\pgfpathlineto{\pgfqpoint{2.500840in}{0.441418in}}%
\pgfpathlineto{\pgfqpoint{2.454935in}{0.441418in}}%
\pgfpathlineto{\pgfqpoint{2.454935in}{0.441418in}}%
\pgfpathlineto{\pgfqpoint{2.409031in}{0.441418in}}%
\pgfpathlineto{\pgfqpoint{2.409031in}{0.441418in}}%
\pgfpathlineto{\pgfqpoint{2.363126in}{0.441418in}}%
\pgfpathlineto{\pgfqpoint{2.363126in}{0.441418in}}%
\pgfpathlineto{\pgfqpoint{2.317222in}{0.441418in}}%
\pgfpathlineto{\pgfqpoint{2.317222in}{0.441418in}}%
\pgfpathlineto{\pgfqpoint{2.271317in}{0.441418in}}%
\pgfpathlineto{\pgfqpoint{2.271317in}{0.441418in}}%
\pgfpathlineto{\pgfqpoint{2.225413in}{0.441418in}}%
\pgfpathlineto{\pgfqpoint{2.225413in}{0.441418in}}%
\pgfpathlineto{\pgfqpoint{2.179508in}{0.441418in}}%
\pgfpathlineto{\pgfqpoint{2.179508in}{0.441418in}}%
\pgfpathlineto{\pgfqpoint{2.133604in}{0.441418in}}%
\pgfpathlineto{\pgfqpoint{2.133604in}{0.441418in}}%
\pgfpathlineto{\pgfqpoint{2.087699in}{0.441418in}}%
\pgfpathlineto{\pgfqpoint{2.087699in}{0.441418in}}%
\pgfpathlineto{\pgfqpoint{2.041795in}{0.441418in}}%
\pgfpathlineto{\pgfqpoint{2.041795in}{0.441418in}}%
\pgfpathlineto{\pgfqpoint{1.995890in}{0.441418in}}%
\pgfpathlineto{\pgfqpoint{1.995890in}{0.441418in}}%
\pgfpathlineto{\pgfqpoint{1.949986in}{0.441418in}}%
\pgfpathlineto{\pgfqpoint{1.949986in}{0.441418in}}%
\pgfpathlineto{\pgfqpoint{1.904081in}{0.441418in}}%
\pgfpathlineto{\pgfqpoint{1.904081in}{0.441418in}}%
\pgfpathlineto{\pgfqpoint{1.858177in}{0.441418in}}%
\pgfpathlineto{\pgfqpoint{1.858177in}{0.441418in}}%
\pgfpathlineto{\pgfqpoint{1.812272in}{0.441418in}}%
\pgfpathlineto{\pgfqpoint{1.812272in}{0.441418in}}%
\pgfpathlineto{\pgfqpoint{1.766368in}{0.441418in}}%
\pgfpathlineto{\pgfqpoint{1.766368in}{0.441418in}}%
\pgfpathlineto{\pgfqpoint{1.720463in}{0.441418in}}%
\pgfpathlineto{\pgfqpoint{1.720463in}{0.441418in}}%
\pgfpathlineto{\pgfqpoint{1.674559in}{0.441418in}}%
\pgfpathlineto{\pgfqpoint{1.674559in}{0.441418in}}%
\pgfpathlineto{\pgfqpoint{1.628654in}{0.441418in}}%
\pgfpathlineto{\pgfqpoint{1.628654in}{0.441418in}}%
\pgfpathlineto{\pgfqpoint{1.582750in}{0.441418in}}%
\pgfpathlineto{\pgfqpoint{1.582750in}{0.441418in}}%
\pgfpathlineto{\pgfqpoint{1.536845in}{0.441418in}}%
\pgfpathlineto{\pgfqpoint{1.536845in}{0.441418in}}%
\pgfpathlineto{\pgfqpoint{1.490941in}{0.441418in}}%
\pgfpathlineto{\pgfqpoint{1.490941in}{0.441418in}}%
\pgfpathlineto{\pgfqpoint{1.445036in}{0.441418in}}%
\pgfpathlineto{\pgfqpoint{1.445036in}{0.441418in}}%
\pgfpathlineto{\pgfqpoint{1.399132in}{0.441418in}}%
\pgfpathlineto{\pgfqpoint{1.399132in}{0.441418in}}%
\pgfpathlineto{\pgfqpoint{1.353227in}{0.441418in}}%
\pgfpathlineto{\pgfqpoint{1.353227in}{0.441418in}}%
\pgfpathlineto{\pgfqpoint{1.307323in}{0.441418in}}%
\pgfpathlineto{\pgfqpoint{1.307323in}{0.441418in}}%
\pgfpathlineto{\pgfqpoint{1.261418in}{0.441418in}}%
\pgfpathlineto{\pgfqpoint{1.261418in}{0.441418in}}%
\pgfpathlineto{\pgfqpoint{1.215514in}{0.441418in}}%
\pgfpathlineto{\pgfqpoint{1.215514in}{0.441418in}}%
\pgfpathlineto{\pgfqpoint{1.169609in}{0.441418in}}%
\pgfpathlineto{\pgfqpoint{1.169609in}{0.441418in}}%
\pgfpathlineto{\pgfqpoint{1.123705in}{0.441418in}}%
\pgfpathlineto{\pgfqpoint{1.123705in}{0.441418in}}%
\pgfpathlineto{\pgfqpoint{1.077800in}{0.441418in}}%
\pgfpathlineto{\pgfqpoint{1.077800in}{0.441418in}}%
\pgfpathlineto{\pgfqpoint{1.031896in}{0.441418in}}%
\pgfpathlineto{\pgfqpoint{1.031896in}{0.441418in}}%
\pgfpathlineto{\pgfqpoint{0.985991in}{0.441418in}}%
\pgfpathlineto{\pgfqpoint{0.985991in}{0.441418in}}%
\pgfpathlineto{\pgfqpoint{0.940087in}{0.441418in}}%
\pgfpathlineto{\pgfqpoint{0.940087in}{0.441418in}}%
\pgfpathlineto{\pgfqpoint{0.894182in}{0.441418in}}%
\pgfpathlineto{\pgfqpoint{0.894182in}{0.441418in}}%
\pgfpathlineto{\pgfqpoint{0.848278in}{0.441418in}}%
\pgfpathlineto{\pgfqpoint{0.848278in}{0.441418in}}%
\pgfpathlineto{\pgfqpoint{0.802373in}{0.441418in}}%
\pgfpathlineto{\pgfqpoint{0.802373in}{0.441418in}}%
\pgfpathlineto{\pgfqpoint{0.756469in}{0.441418in}}%
\pgfpathlineto{\pgfqpoint{0.756469in}{0.441418in}}%
\pgfpathlineto{\pgfqpoint{0.710564in}{0.441418in}}%
\pgfpathlineto{\pgfqpoint{0.710564in}{0.441418in}}%
\pgfpathlineto{\pgfqpoint{0.664660in}{0.441418in}}%
\pgfpathlineto{\pgfqpoint{0.664660in}{0.441418in}}%
\pgfpathlineto{\pgfqpoint{0.618755in}{0.441418in}}%
\pgfpathlineto{\pgfqpoint{0.618755in}{0.441418in}}%
\pgfpathlineto{\pgfqpoint{0.572851in}{0.441418in}}%
\pgfpathlineto{\pgfqpoint{0.572851in}{0.441418in}}%
\pgfpathlineto{\pgfqpoint{0.526946in}{0.441418in}}%
\pgfpathlineto{\pgfqpoint{0.526946in}{0.441418in}}%
\pgfpathlineto{\pgfqpoint{0.481042in}{0.441418in}}%
\pgfpathlineto{\pgfqpoint{0.481042in}{0.441418in}}%
\pgfpathlineto{\pgfqpoint{0.435137in}{0.441418in}}%
\pgfpathlineto{\pgfqpoint{0.435137in}{0.441418in}}%
\pgfpathlineto{\pgfqpoint{0.389233in}{0.441418in}}%
\pgfpathlineto{\pgfqpoint{0.389233in}{0.441418in}}%
\pgfpathlineto{\pgfqpoint{0.343328in}{0.441418in}}%
\pgfpathlineto{\pgfqpoint{0.343328in}{0.441418in}}%
\pgfpathlineto{\pgfqpoint{0.297424in}{0.441418in}}%
\pgfusepath{fill}%
\end{pgfscope}%
\begin{pgfscope}%
\pgfpathrectangle{\pgfqpoint{0.296148in}{0.441418in}}{\pgfqpoint{2.296592in}{1.173543in}} %
\pgfusepath{clip}%
\pgfsetbuttcap%
\pgfsetmiterjoin%
\pgfsetlinewidth{0.501875pt}%
\definecolor{currentstroke}{rgb}{1.000000,0.000000,0.000000}%
\pgfsetstrokecolor{currentstroke}%
\pgfsetdash{}{0pt}%
\pgfpathmoveto{\pgfqpoint{0.297424in}{0.441418in}}%
\pgfpathlineto{\pgfqpoint{0.297424in}{0.610895in}}%
\pgfpathlineto{\pgfqpoint{0.343328in}{0.610895in}}%
\pgfpathlineto{\pgfqpoint{0.343328in}{1.002659in}}%
\pgfpathlineto{\pgfqpoint{0.389233in}{1.002659in}}%
\pgfpathlineto{\pgfqpoint{0.389233in}{1.298591in}}%
\pgfpathlineto{\pgfqpoint{0.435137in}{1.298591in}}%
\pgfpathlineto{\pgfqpoint{0.435137in}{1.449318in}}%
\pgfpathlineto{\pgfqpoint{0.481042in}{1.449318in}}%
\pgfpathlineto{\pgfqpoint{0.481042in}{1.491327in}}%
\pgfpathlineto{\pgfqpoint{0.526946in}{1.491327in}}%
\pgfpathlineto{\pgfqpoint{0.526946in}{1.469847in}}%
\pgfpathlineto{\pgfqpoint{0.572851in}{1.469847in}}%
\pgfpathlineto{\pgfqpoint{0.572851in}{1.408998in}}%
\pgfpathlineto{\pgfqpoint{0.618755in}{1.408998in}}%
\pgfpathlineto{\pgfqpoint{0.618755in}{1.322433in}}%
\pgfpathlineto{\pgfqpoint{0.664660in}{1.322433in}}%
\pgfpathlineto{\pgfqpoint{0.664660in}{1.230223in}}%
\pgfpathlineto{\pgfqpoint{0.710564in}{1.230223in}}%
\pgfpathlineto{\pgfqpoint{0.710564in}{1.123836in}}%
\pgfpathlineto{\pgfqpoint{0.756469in}{1.123836in}}%
\pgfpathlineto{\pgfqpoint{0.756469in}{1.048779in}}%
\pgfpathlineto{\pgfqpoint{0.802373in}{1.048779in}}%
\pgfpathlineto{\pgfqpoint{0.802373in}{0.961294in}}%
\pgfpathlineto{\pgfqpoint{0.848278in}{0.961294in}}%
\pgfpathlineto{\pgfqpoint{0.848278in}{0.890196in}}%
\pgfpathlineto{\pgfqpoint{0.894182in}{0.890196in}}%
\pgfpathlineto{\pgfqpoint{0.894182in}{0.823485in}}%
\pgfpathlineto{\pgfqpoint{0.940087in}{0.823485in}}%
\pgfpathlineto{\pgfqpoint{0.940087in}{0.774726in}}%
\pgfpathlineto{\pgfqpoint{0.985991in}{0.774726in}}%
\pgfpathlineto{\pgfqpoint{0.985991in}{0.724279in}}%
\pgfpathlineto{\pgfqpoint{1.031896in}{0.724279in}}%
\pgfpathlineto{\pgfqpoint{1.031896in}{0.679385in}}%
\pgfpathlineto{\pgfqpoint{1.077800in}{0.679385in}}%
\pgfpathlineto{\pgfqpoint{1.077800in}{0.647841in}}%
\pgfpathlineto{\pgfqpoint{1.123705in}{0.647841in}}%
\pgfpathlineto{\pgfqpoint{1.123705in}{0.611877in}}%
\pgfpathlineto{\pgfqpoint{1.169609in}{0.611877in}}%
\pgfpathlineto{\pgfqpoint{1.169609in}{0.589630in}}%
\pgfpathlineto{\pgfqpoint{1.215514in}{0.589630in}}%
\pgfpathlineto{\pgfqpoint{1.215514in}{0.562995in}}%
\pgfpathlineto{\pgfqpoint{1.261418in}{0.562995in}}%
\pgfpathlineto{\pgfqpoint{1.261418in}{0.547744in}}%
\pgfpathlineto{\pgfqpoint{1.307323in}{0.547744in}}%
\pgfpathlineto{\pgfqpoint{1.307323in}{0.531020in}}%
\pgfpathlineto{\pgfqpoint{1.353227in}{0.531020in}}%
\pgfpathlineto{\pgfqpoint{1.353227in}{0.515800in}}%
\pgfpathlineto{\pgfqpoint{1.399132in}{0.515800in}}%
\pgfpathlineto{\pgfqpoint{1.399132in}{0.506257in}}%
\pgfpathlineto{\pgfqpoint{1.445036in}{0.506257in}}%
\pgfpathlineto{\pgfqpoint{1.445036in}{0.495241in}}%
\pgfpathlineto{\pgfqpoint{1.490941in}{0.495241in}}%
\pgfpathlineto{\pgfqpoint{1.490941in}{0.488153in}}%
\pgfpathlineto{\pgfqpoint{1.536845in}{0.488153in}}%
\pgfpathlineto{\pgfqpoint{1.536845in}{0.483304in}}%
\pgfpathlineto{\pgfqpoint{1.582750in}{0.483304in}}%
\pgfpathlineto{\pgfqpoint{1.582750in}{0.476523in}}%
\pgfpathlineto{\pgfqpoint{1.628654in}{0.476523in}}%
\pgfpathlineto{\pgfqpoint{1.628654in}{0.472994in}}%
\pgfpathlineto{\pgfqpoint{1.674559in}{0.472994in}}%
\pgfpathlineto{\pgfqpoint{1.674559in}{0.468606in}}%
\pgfpathlineto{\pgfqpoint{1.720463in}{0.468606in}}%
\pgfpathlineto{\pgfqpoint{1.720463in}{0.467501in}}%
\pgfpathlineto{\pgfqpoint{1.766368in}{0.467501in}}%
\pgfpathlineto{\pgfqpoint{1.766368in}{0.462315in}}%
\pgfpathlineto{\pgfqpoint{1.812272in}{0.462315in}}%
\pgfpathlineto{\pgfqpoint{1.812272in}{0.460413in}}%
\pgfpathlineto{\pgfqpoint{1.858177in}{0.460413in}}%
\pgfpathlineto{\pgfqpoint{1.858177in}{0.457897in}}%
\pgfpathlineto{\pgfqpoint{1.904081in}{0.457897in}}%
\pgfpathlineto{\pgfqpoint{1.904081in}{0.456240in}}%
\pgfpathlineto{\pgfqpoint{1.949986in}{0.456240in}}%
\pgfpathlineto{\pgfqpoint{1.949986in}{0.456056in}}%
\pgfpathlineto{\pgfqpoint{1.995890in}{0.456056in}}%
\pgfpathlineto{\pgfqpoint{1.995890in}{0.452956in}}%
\pgfpathlineto{\pgfqpoint{2.041795in}{0.452956in}}%
\pgfpathlineto{\pgfqpoint{2.041795in}{0.451545in}}%
\pgfpathlineto{\pgfqpoint{2.087699in}{0.451545in}}%
\pgfpathlineto{\pgfqpoint{2.087699in}{0.451575in}}%
\pgfpathlineto{\pgfqpoint{2.133604in}{0.451575in}}%
\pgfpathlineto{\pgfqpoint{2.133604in}{0.450747in}}%
\pgfpathlineto{\pgfqpoint{2.179508in}{0.450747in}}%
\pgfpathlineto{\pgfqpoint{2.179508in}{0.449857in}}%
\pgfpathlineto{\pgfqpoint{2.225413in}{0.449857in}}%
\pgfpathlineto{\pgfqpoint{2.225413in}{0.449581in}}%
\pgfpathlineto{\pgfqpoint{2.271317in}{0.449581in}}%
\pgfpathlineto{\pgfqpoint{2.271317in}{0.448814in}}%
\pgfpathlineto{\pgfqpoint{2.317222in}{0.448814in}}%
\pgfpathlineto{\pgfqpoint{2.317222in}{0.447402in}}%
\pgfpathlineto{\pgfqpoint{2.363126in}{0.447402in}}%
\pgfpathlineto{\pgfqpoint{2.363126in}{0.446543in}}%
\pgfpathlineto{\pgfqpoint{2.409031in}{0.446543in}}%
\pgfpathlineto{\pgfqpoint{2.409031in}{0.447034in}}%
\pgfpathlineto{\pgfqpoint{2.454935in}{0.447034in}}%
\pgfpathlineto{\pgfqpoint{2.454935in}{0.446512in}}%
\pgfpathlineto{\pgfqpoint{2.500840in}{0.446512in}}%
\pgfpathlineto{\pgfqpoint{2.500840in}{0.445868in}}%
\pgfpathlineto{\pgfqpoint{2.546744in}{0.445868in}}%
\pgfpathlineto{\pgfqpoint{2.546744in}{0.445500in}}%
\pgfpathlineto{\pgfqpoint{2.592649in}{0.445500in}}%
\pgfpathlineto{\pgfqpoint{2.592649in}{0.441418in}}%
\pgfusepath{stroke}%
\end{pgfscope}%
\begin{pgfscope}%
\pgfpathrectangle{\pgfqpoint{0.296148in}{0.441418in}}{\pgfqpoint{2.296592in}{1.173543in}} %
\pgfusepath{clip}%
\pgfsetbuttcap%
\pgfsetmiterjoin%
\pgfsetlinewidth{1.003750pt}%
\definecolor{currentstroke}{rgb}{1.000000,0.647059,0.000000}%
\pgfsetstrokecolor{currentstroke}%
\pgfsetdash{}{0pt}%
\pgfpathmoveto{\pgfqpoint{0.297424in}{0.441418in}}%
\pgfpathlineto{\pgfqpoint{0.297424in}{0.603626in}}%
\pgfpathlineto{\pgfqpoint{0.343328in}{0.603626in}}%
\pgfpathlineto{\pgfqpoint{0.343328in}{0.976006in}}%
\pgfpathlineto{\pgfqpoint{0.389233in}{0.976006in}}%
\pgfpathlineto{\pgfqpoint{0.389233in}{1.255838in}}%
\pgfpathlineto{\pgfqpoint{0.435137in}{1.255838in}}%
\pgfpathlineto{\pgfqpoint{0.435137in}{1.399265in}}%
\pgfpathlineto{\pgfqpoint{0.481042in}{1.399265in}}%
\pgfpathlineto{\pgfqpoint{0.481042in}{1.444638in}}%
\pgfpathlineto{\pgfqpoint{0.526946in}{1.444638in}}%
\pgfpathlineto{\pgfqpoint{0.526946in}{1.418375in}}%
\pgfpathlineto{\pgfqpoint{0.572851in}{1.418375in}}%
\pgfpathlineto{\pgfqpoint{0.572851in}{1.360521in}}%
\pgfpathlineto{\pgfqpoint{0.618755in}{1.360521in}}%
\pgfpathlineto{\pgfqpoint{0.618755in}{1.280076in}}%
\pgfpathlineto{\pgfqpoint{0.664660in}{1.280076in}}%
\pgfpathlineto{\pgfqpoint{0.664660in}{1.196681in}}%
\pgfpathlineto{\pgfqpoint{0.710564in}{1.196681in}}%
\pgfpathlineto{\pgfqpoint{0.710564in}{1.107636in}}%
\pgfpathlineto{\pgfqpoint{0.756469in}{1.107636in}}%
\pgfpathlineto{\pgfqpoint{0.756469in}{1.039494in}}%
\pgfpathlineto{\pgfqpoint{0.802373in}{1.039494in}}%
\pgfpathlineto{\pgfqpoint{0.802373in}{0.955752in}}%
\pgfpathlineto{\pgfqpoint{0.848278in}{0.955752in}}%
\pgfpathlineto{\pgfqpoint{0.848278in}{0.901018in}}%
\pgfpathlineto{\pgfqpoint{0.894182in}{0.901018in}}%
\pgfpathlineto{\pgfqpoint{0.894182in}{0.838401in}}%
\pgfpathlineto{\pgfqpoint{0.940087in}{0.838401in}}%
\pgfpathlineto{\pgfqpoint{0.940087in}{0.788888in}}%
\pgfpathlineto{\pgfqpoint{0.985991in}{0.788888in}}%
\pgfpathlineto{\pgfqpoint{0.985991in}{0.746198in}}%
\pgfpathlineto{\pgfqpoint{1.031896in}{0.746198in}}%
\pgfpathlineto{\pgfqpoint{1.031896in}{0.700878in}}%
\pgfpathlineto{\pgfqpoint{1.077800in}{0.700878in}}%
\pgfpathlineto{\pgfqpoint{1.077800in}{0.674255in}}%
\pgfpathlineto{\pgfqpoint{1.123705in}{0.674255in}}%
\pgfpathlineto{\pgfqpoint{1.123705in}{0.633016in}}%
\pgfpathlineto{\pgfqpoint{1.169609in}{0.633016in}}%
\pgfpathlineto{\pgfqpoint{1.169609in}{0.610975in}}%
\pgfpathlineto{\pgfqpoint{1.215514in}{0.610975in}}%
\pgfpathlineto{\pgfqpoint{1.215514in}{0.582688in}}%
\pgfpathlineto{\pgfqpoint{1.261418in}{0.582688in}}%
\pgfpathlineto{\pgfqpoint{1.261418in}{0.570028in}}%
\pgfpathlineto{\pgfqpoint{1.307323in}{0.570028in}}%
\pgfpathlineto{\pgfqpoint{1.307323in}{0.551973in}}%
\pgfpathlineto{\pgfqpoint{1.353227in}{0.551973in}}%
\pgfpathlineto{\pgfqpoint{1.353227in}{0.534428in}}%
\pgfpathlineto{\pgfqpoint{1.399132in}{0.534428in}}%
\pgfpathlineto{\pgfqpoint{1.399132in}{0.524707in}}%
\pgfpathlineto{\pgfqpoint{1.445036in}{0.524707in}}%
\pgfpathlineto{\pgfqpoint{1.445036in}{0.510048in}}%
\pgfpathlineto{\pgfqpoint{1.490941in}{0.510048in}}%
\pgfpathlineto{\pgfqpoint{1.490941in}{0.500346in}}%
\pgfpathlineto{\pgfqpoint{1.536845in}{0.500346in}}%
\pgfpathlineto{\pgfqpoint{1.536845in}{0.493994in}}%
\pgfpathlineto{\pgfqpoint{1.582750in}{0.493994in}}%
\pgfpathlineto{\pgfqpoint{1.582750in}{0.486369in}}%
\pgfpathlineto{\pgfqpoint{1.628654in}{0.486369in}}%
\pgfpathlineto{\pgfqpoint{1.628654in}{0.481514in}}%
\pgfpathlineto{\pgfqpoint{1.674559in}{0.481514in}}%
\pgfpathlineto{\pgfqpoint{1.674559in}{0.475943in}}%
\pgfpathlineto{\pgfqpoint{1.720463in}{0.475943in}}%
\pgfpathlineto{\pgfqpoint{1.720463in}{0.474649in}}%
\pgfpathlineto{\pgfqpoint{1.766368in}{0.474649in}}%
\pgfpathlineto{\pgfqpoint{1.766368in}{0.468027in}}%
\pgfpathlineto{\pgfqpoint{1.812272in}{0.468027in}}%
\pgfpathlineto{\pgfqpoint{1.812272in}{0.465500in}}%
\pgfpathlineto{\pgfqpoint{1.858177in}{0.465500in}}%
\pgfpathlineto{\pgfqpoint{1.858177in}{0.462736in}}%
\pgfpathlineto{\pgfqpoint{1.904081in}{0.462736in}}%
\pgfpathlineto{\pgfqpoint{1.904081in}{0.461337in}}%
\pgfpathlineto{\pgfqpoint{1.949986in}{0.461337in}}%
\pgfpathlineto{\pgfqpoint{1.949986in}{0.460886in}}%
\pgfpathlineto{\pgfqpoint{1.995890in}{0.460886in}}%
\pgfpathlineto{\pgfqpoint{1.995890in}{0.456710in}}%
\pgfpathlineto{\pgfqpoint{2.041795in}{0.456710in}}%
\pgfpathlineto{\pgfqpoint{2.041795in}{0.454595in}}%
\pgfpathlineto{\pgfqpoint{2.087699in}{0.454595in}}%
\pgfpathlineto{\pgfqpoint{2.087699in}{0.454692in}}%
\pgfpathlineto{\pgfqpoint{2.133604in}{0.454692in}}%
\pgfpathlineto{\pgfqpoint{2.133604in}{0.453988in}}%
\pgfpathlineto{\pgfqpoint{2.179508in}{0.453988in}}%
\pgfpathlineto{\pgfqpoint{2.179508in}{0.452982in}}%
\pgfpathlineto{\pgfqpoint{2.225413in}{0.452982in}}%
\pgfpathlineto{\pgfqpoint{2.225413in}{0.452766in}}%
\pgfpathlineto{\pgfqpoint{2.271317in}{0.452766in}}%
\pgfpathlineto{\pgfqpoint{2.271317in}{0.451454in}}%
\pgfpathlineto{\pgfqpoint{2.317222in}{0.451454in}}%
\pgfpathlineto{\pgfqpoint{2.317222in}{0.449129in}}%
\pgfpathlineto{\pgfqpoint{2.363126in}{0.449129in}}%
\pgfpathlineto{\pgfqpoint{2.363126in}{0.447950in}}%
\pgfpathlineto{\pgfqpoint{2.409031in}{0.447950in}}%
\pgfpathlineto{\pgfqpoint{2.409031in}{0.449259in}}%
\pgfpathlineto{\pgfqpoint{2.454935in}{0.449259in}}%
\pgfpathlineto{\pgfqpoint{2.454935in}{0.448051in}}%
\pgfpathlineto{\pgfqpoint{2.500840in}{0.448051in}}%
\pgfpathlineto{\pgfqpoint{2.500840in}{0.447522in}}%
\pgfpathlineto{\pgfqpoint{2.546744in}{0.447522in}}%
\pgfpathlineto{\pgfqpoint{2.546744in}{0.446804in}}%
\pgfpathlineto{\pgfqpoint{2.592649in}{0.446804in}}%
\pgfpathlineto{\pgfqpoint{2.592649in}{0.441418in}}%
\pgfusepath{stroke}%
\end{pgfscope}%
\begin{pgfscope}%
\pgfsetrectcap%
\pgfsetmiterjoin%
\pgfsetlinewidth{1.003750pt}%
\definecolor{currentstroke}{rgb}{0.000000,0.000000,0.000000}%
\pgfsetstrokecolor{currentstroke}%
\pgfsetdash{}{0pt}%
\pgfpathmoveto{\pgfqpoint{0.296148in}{1.614961in}}%
\pgfpathlineto{\pgfqpoint{2.592740in}{1.614961in}}%
\pgfusepath{stroke}%
\end{pgfscope}%
\begin{pgfscope}%
\pgfsetrectcap%
\pgfsetmiterjoin%
\pgfsetlinewidth{1.003750pt}%
\definecolor{currentstroke}{rgb}{0.000000,0.000000,0.000000}%
\pgfsetstrokecolor{currentstroke}%
\pgfsetdash{}{0pt}%
\pgfpathmoveto{\pgfqpoint{2.592740in}{0.441418in}}%
\pgfpathlineto{\pgfqpoint{2.592740in}{1.614961in}}%
\pgfusepath{stroke}%
\end{pgfscope}%
\begin{pgfscope}%
\pgfsetrectcap%
\pgfsetmiterjoin%
\pgfsetlinewidth{1.003750pt}%
\definecolor{currentstroke}{rgb}{0.000000,0.000000,0.000000}%
\pgfsetstrokecolor{currentstroke}%
\pgfsetdash{}{0pt}%
\pgfpathmoveto{\pgfqpoint{0.296148in}{0.441418in}}%
\pgfpathlineto{\pgfqpoint{2.592740in}{0.441418in}}%
\pgfusepath{stroke}%
\end{pgfscope}%
\begin{pgfscope}%
\pgfsetrectcap%
\pgfsetmiterjoin%
\pgfsetlinewidth{1.003750pt}%
\definecolor{currentstroke}{rgb}{0.000000,0.000000,0.000000}%
\pgfsetstrokecolor{currentstroke}%
\pgfsetdash{}{0pt}%
\pgfpathmoveto{\pgfqpoint{0.296148in}{0.441418in}}%
\pgfpathlineto{\pgfqpoint{0.296148in}{1.614961in}}%
\pgfusepath{stroke}%
\end{pgfscope}%
\begin{pgfscope}%
\pgfsetbuttcap%
\pgfsetroundjoin%
\definecolor{currentfill}{rgb}{0.000000,0.000000,0.000000}%
\pgfsetfillcolor{currentfill}%
\pgfsetlinewidth{0.501875pt}%
\definecolor{currentstroke}{rgb}{0.000000,0.000000,0.000000}%
\pgfsetstrokecolor{currentstroke}%
\pgfsetdash{}{0pt}%
\pgfsys@defobject{currentmarker}{\pgfqpoint{0.000000in}{0.000000in}}{\pgfqpoint{0.000000in}{0.069444in}}{%
\pgfpathmoveto{\pgfqpoint{0.000000in}{0.000000in}}%
\pgfpathlineto{\pgfqpoint{0.000000in}{0.069444in}}%
\pgfusepath{stroke,fill}%
}%
\begin{pgfscope}%
\pgfsys@transformshift{0.296148in}{0.441418in}%
\pgfsys@useobject{currentmarker}{}%
\end{pgfscope}%
\end{pgfscope}%
\begin{pgfscope}%
\pgfsetbuttcap%
\pgfsetroundjoin%
\definecolor{currentfill}{rgb}{0.000000,0.000000,0.000000}%
\pgfsetfillcolor{currentfill}%
\pgfsetlinewidth{0.501875pt}%
\definecolor{currentstroke}{rgb}{0.000000,0.000000,0.000000}%
\pgfsetstrokecolor{currentstroke}%
\pgfsetdash{}{0pt}%
\pgfsys@defobject{currentmarker}{\pgfqpoint{0.000000in}{-0.069444in}}{\pgfqpoint{0.000000in}{0.000000in}}{%
\pgfpathmoveto{\pgfqpoint{0.000000in}{0.000000in}}%
\pgfpathlineto{\pgfqpoint{0.000000in}{-0.069444in}}%
\pgfusepath{stroke,fill}%
}%
\begin{pgfscope}%
\pgfsys@transformshift{0.296148in}{1.614961in}%
\pgfsys@useobject{currentmarker}{}%
\end{pgfscope}%
\end{pgfscope}%
\begin{pgfscope}%
\pgftext[x=0.296148in,y=0.371974in,,top]{\rmfamily\fontsize{8.000000}{9.600000}\selectfont 0}%
\end{pgfscope}%
\begin{pgfscope}%
\pgfsetbuttcap%
\pgfsetroundjoin%
\definecolor{currentfill}{rgb}{0.000000,0.000000,0.000000}%
\pgfsetfillcolor{currentfill}%
\pgfsetlinewidth{0.501875pt}%
\definecolor{currentstroke}{rgb}{0.000000,0.000000,0.000000}%
\pgfsetstrokecolor{currentstroke}%
\pgfsetdash{}{0pt}%
\pgfsys@defobject{currentmarker}{\pgfqpoint{0.000000in}{0.000000in}}{\pgfqpoint{0.000000in}{0.069444in}}{%
\pgfpathmoveto{\pgfqpoint{0.000000in}{0.000000in}}%
\pgfpathlineto{\pgfqpoint{0.000000in}{0.069444in}}%
\pgfusepath{stroke,fill}%
}%
\begin{pgfscope}%
\pgfsys@transformshift{0.583222in}{0.441418in}%
\pgfsys@useobject{currentmarker}{}%
\end{pgfscope}%
\end{pgfscope}%
\begin{pgfscope}%
\pgfsetbuttcap%
\pgfsetroundjoin%
\definecolor{currentfill}{rgb}{0.000000,0.000000,0.000000}%
\pgfsetfillcolor{currentfill}%
\pgfsetlinewidth{0.501875pt}%
\definecolor{currentstroke}{rgb}{0.000000,0.000000,0.000000}%
\pgfsetstrokecolor{currentstroke}%
\pgfsetdash{}{0pt}%
\pgfsys@defobject{currentmarker}{\pgfqpoint{0.000000in}{-0.069444in}}{\pgfqpoint{0.000000in}{0.000000in}}{%
\pgfpathmoveto{\pgfqpoint{0.000000in}{0.000000in}}%
\pgfpathlineto{\pgfqpoint{0.000000in}{-0.069444in}}%
\pgfusepath{stroke,fill}%
}%
\begin{pgfscope}%
\pgfsys@transformshift{0.583222in}{1.614961in}%
\pgfsys@useobject{currentmarker}{}%
\end{pgfscope}%
\end{pgfscope}%
\begin{pgfscope}%
\pgftext[x=0.583222in,y=0.371974in,,top]{\rmfamily\fontsize{8.000000}{9.600000}\selectfont 1}%
\end{pgfscope}%
\begin{pgfscope}%
\pgfsetbuttcap%
\pgfsetroundjoin%
\definecolor{currentfill}{rgb}{0.000000,0.000000,0.000000}%
\pgfsetfillcolor{currentfill}%
\pgfsetlinewidth{0.501875pt}%
\definecolor{currentstroke}{rgb}{0.000000,0.000000,0.000000}%
\pgfsetstrokecolor{currentstroke}%
\pgfsetdash{}{0pt}%
\pgfsys@defobject{currentmarker}{\pgfqpoint{0.000000in}{0.000000in}}{\pgfqpoint{0.000000in}{0.069444in}}{%
\pgfpathmoveto{\pgfqpoint{0.000000in}{0.000000in}}%
\pgfpathlineto{\pgfqpoint{0.000000in}{0.069444in}}%
\pgfusepath{stroke,fill}%
}%
\begin{pgfscope}%
\pgfsys@transformshift{0.870296in}{0.441418in}%
\pgfsys@useobject{currentmarker}{}%
\end{pgfscope}%
\end{pgfscope}%
\begin{pgfscope}%
\pgfsetbuttcap%
\pgfsetroundjoin%
\definecolor{currentfill}{rgb}{0.000000,0.000000,0.000000}%
\pgfsetfillcolor{currentfill}%
\pgfsetlinewidth{0.501875pt}%
\definecolor{currentstroke}{rgb}{0.000000,0.000000,0.000000}%
\pgfsetstrokecolor{currentstroke}%
\pgfsetdash{}{0pt}%
\pgfsys@defobject{currentmarker}{\pgfqpoint{0.000000in}{-0.069444in}}{\pgfqpoint{0.000000in}{0.000000in}}{%
\pgfpathmoveto{\pgfqpoint{0.000000in}{0.000000in}}%
\pgfpathlineto{\pgfqpoint{0.000000in}{-0.069444in}}%
\pgfusepath{stroke,fill}%
}%
\begin{pgfscope}%
\pgfsys@transformshift{0.870296in}{1.614961in}%
\pgfsys@useobject{currentmarker}{}%
\end{pgfscope}%
\end{pgfscope}%
\begin{pgfscope}%
\pgftext[x=0.870296in,y=0.371974in,,top]{\rmfamily\fontsize{8.000000}{9.600000}\selectfont 2}%
\end{pgfscope}%
\begin{pgfscope}%
\pgfsetbuttcap%
\pgfsetroundjoin%
\definecolor{currentfill}{rgb}{0.000000,0.000000,0.000000}%
\pgfsetfillcolor{currentfill}%
\pgfsetlinewidth{0.501875pt}%
\definecolor{currentstroke}{rgb}{0.000000,0.000000,0.000000}%
\pgfsetstrokecolor{currentstroke}%
\pgfsetdash{}{0pt}%
\pgfsys@defobject{currentmarker}{\pgfqpoint{0.000000in}{0.000000in}}{\pgfqpoint{0.000000in}{0.069444in}}{%
\pgfpathmoveto{\pgfqpoint{0.000000in}{0.000000in}}%
\pgfpathlineto{\pgfqpoint{0.000000in}{0.069444in}}%
\pgfusepath{stroke,fill}%
}%
\begin{pgfscope}%
\pgfsys@transformshift{1.157370in}{0.441418in}%
\pgfsys@useobject{currentmarker}{}%
\end{pgfscope}%
\end{pgfscope}%
\begin{pgfscope}%
\pgfsetbuttcap%
\pgfsetroundjoin%
\definecolor{currentfill}{rgb}{0.000000,0.000000,0.000000}%
\pgfsetfillcolor{currentfill}%
\pgfsetlinewidth{0.501875pt}%
\definecolor{currentstroke}{rgb}{0.000000,0.000000,0.000000}%
\pgfsetstrokecolor{currentstroke}%
\pgfsetdash{}{0pt}%
\pgfsys@defobject{currentmarker}{\pgfqpoint{0.000000in}{-0.069444in}}{\pgfqpoint{0.000000in}{0.000000in}}{%
\pgfpathmoveto{\pgfqpoint{0.000000in}{0.000000in}}%
\pgfpathlineto{\pgfqpoint{0.000000in}{-0.069444in}}%
\pgfusepath{stroke,fill}%
}%
\begin{pgfscope}%
\pgfsys@transformshift{1.157370in}{1.614961in}%
\pgfsys@useobject{currentmarker}{}%
\end{pgfscope}%
\end{pgfscope}%
\begin{pgfscope}%
\pgftext[x=1.157370in,y=0.371974in,,top]{\rmfamily\fontsize{8.000000}{9.600000}\selectfont 3}%
\end{pgfscope}%
\begin{pgfscope}%
\pgfsetbuttcap%
\pgfsetroundjoin%
\definecolor{currentfill}{rgb}{0.000000,0.000000,0.000000}%
\pgfsetfillcolor{currentfill}%
\pgfsetlinewidth{0.501875pt}%
\definecolor{currentstroke}{rgb}{0.000000,0.000000,0.000000}%
\pgfsetstrokecolor{currentstroke}%
\pgfsetdash{}{0pt}%
\pgfsys@defobject{currentmarker}{\pgfqpoint{0.000000in}{0.000000in}}{\pgfqpoint{0.000000in}{0.069444in}}{%
\pgfpathmoveto{\pgfqpoint{0.000000in}{0.000000in}}%
\pgfpathlineto{\pgfqpoint{0.000000in}{0.069444in}}%
\pgfusepath{stroke,fill}%
}%
\begin{pgfscope}%
\pgfsys@transformshift{1.444444in}{0.441418in}%
\pgfsys@useobject{currentmarker}{}%
\end{pgfscope}%
\end{pgfscope}%
\begin{pgfscope}%
\pgfsetbuttcap%
\pgfsetroundjoin%
\definecolor{currentfill}{rgb}{0.000000,0.000000,0.000000}%
\pgfsetfillcolor{currentfill}%
\pgfsetlinewidth{0.501875pt}%
\definecolor{currentstroke}{rgb}{0.000000,0.000000,0.000000}%
\pgfsetstrokecolor{currentstroke}%
\pgfsetdash{}{0pt}%
\pgfsys@defobject{currentmarker}{\pgfqpoint{0.000000in}{-0.069444in}}{\pgfqpoint{0.000000in}{0.000000in}}{%
\pgfpathmoveto{\pgfqpoint{0.000000in}{0.000000in}}%
\pgfpathlineto{\pgfqpoint{0.000000in}{-0.069444in}}%
\pgfusepath{stroke,fill}%
}%
\begin{pgfscope}%
\pgfsys@transformshift{1.444444in}{1.614961in}%
\pgfsys@useobject{currentmarker}{}%
\end{pgfscope}%
\end{pgfscope}%
\begin{pgfscope}%
\pgftext[x=1.444444in,y=0.371974in,,top]{\rmfamily\fontsize{8.000000}{9.600000}\selectfont 4}%
\end{pgfscope}%
\begin{pgfscope}%
\pgfsetbuttcap%
\pgfsetroundjoin%
\definecolor{currentfill}{rgb}{0.000000,0.000000,0.000000}%
\pgfsetfillcolor{currentfill}%
\pgfsetlinewidth{0.501875pt}%
\definecolor{currentstroke}{rgb}{0.000000,0.000000,0.000000}%
\pgfsetstrokecolor{currentstroke}%
\pgfsetdash{}{0pt}%
\pgfsys@defobject{currentmarker}{\pgfqpoint{0.000000in}{0.000000in}}{\pgfqpoint{0.000000in}{0.069444in}}{%
\pgfpathmoveto{\pgfqpoint{0.000000in}{0.000000in}}%
\pgfpathlineto{\pgfqpoint{0.000000in}{0.069444in}}%
\pgfusepath{stroke,fill}%
}%
\begin{pgfscope}%
\pgfsys@transformshift{1.731518in}{0.441418in}%
\pgfsys@useobject{currentmarker}{}%
\end{pgfscope}%
\end{pgfscope}%
\begin{pgfscope}%
\pgfsetbuttcap%
\pgfsetroundjoin%
\definecolor{currentfill}{rgb}{0.000000,0.000000,0.000000}%
\pgfsetfillcolor{currentfill}%
\pgfsetlinewidth{0.501875pt}%
\definecolor{currentstroke}{rgb}{0.000000,0.000000,0.000000}%
\pgfsetstrokecolor{currentstroke}%
\pgfsetdash{}{0pt}%
\pgfsys@defobject{currentmarker}{\pgfqpoint{0.000000in}{-0.069444in}}{\pgfqpoint{0.000000in}{0.000000in}}{%
\pgfpathmoveto{\pgfqpoint{0.000000in}{0.000000in}}%
\pgfpathlineto{\pgfqpoint{0.000000in}{-0.069444in}}%
\pgfusepath{stroke,fill}%
}%
\begin{pgfscope}%
\pgfsys@transformshift{1.731518in}{1.614961in}%
\pgfsys@useobject{currentmarker}{}%
\end{pgfscope}%
\end{pgfscope}%
\begin{pgfscope}%
\pgftext[x=1.731518in,y=0.371974in,,top]{\rmfamily\fontsize{8.000000}{9.600000}\selectfont 5}%
\end{pgfscope}%
\begin{pgfscope}%
\pgfsetbuttcap%
\pgfsetroundjoin%
\definecolor{currentfill}{rgb}{0.000000,0.000000,0.000000}%
\pgfsetfillcolor{currentfill}%
\pgfsetlinewidth{0.501875pt}%
\definecolor{currentstroke}{rgb}{0.000000,0.000000,0.000000}%
\pgfsetstrokecolor{currentstroke}%
\pgfsetdash{}{0pt}%
\pgfsys@defobject{currentmarker}{\pgfqpoint{0.000000in}{0.000000in}}{\pgfqpoint{0.000000in}{0.069444in}}{%
\pgfpathmoveto{\pgfqpoint{0.000000in}{0.000000in}}%
\pgfpathlineto{\pgfqpoint{0.000000in}{0.069444in}}%
\pgfusepath{stroke,fill}%
}%
\begin{pgfscope}%
\pgfsys@transformshift{2.018592in}{0.441418in}%
\pgfsys@useobject{currentmarker}{}%
\end{pgfscope}%
\end{pgfscope}%
\begin{pgfscope}%
\pgfsetbuttcap%
\pgfsetroundjoin%
\definecolor{currentfill}{rgb}{0.000000,0.000000,0.000000}%
\pgfsetfillcolor{currentfill}%
\pgfsetlinewidth{0.501875pt}%
\definecolor{currentstroke}{rgb}{0.000000,0.000000,0.000000}%
\pgfsetstrokecolor{currentstroke}%
\pgfsetdash{}{0pt}%
\pgfsys@defobject{currentmarker}{\pgfqpoint{0.000000in}{-0.069444in}}{\pgfqpoint{0.000000in}{0.000000in}}{%
\pgfpathmoveto{\pgfqpoint{0.000000in}{0.000000in}}%
\pgfpathlineto{\pgfqpoint{0.000000in}{-0.069444in}}%
\pgfusepath{stroke,fill}%
}%
\begin{pgfscope}%
\pgfsys@transformshift{2.018592in}{1.614961in}%
\pgfsys@useobject{currentmarker}{}%
\end{pgfscope}%
\end{pgfscope}%
\begin{pgfscope}%
\pgftext[x=2.018592in,y=0.371974in,,top]{\rmfamily\fontsize{8.000000}{9.600000}\selectfont 6}%
\end{pgfscope}%
\begin{pgfscope}%
\pgfsetbuttcap%
\pgfsetroundjoin%
\definecolor{currentfill}{rgb}{0.000000,0.000000,0.000000}%
\pgfsetfillcolor{currentfill}%
\pgfsetlinewidth{0.501875pt}%
\definecolor{currentstroke}{rgb}{0.000000,0.000000,0.000000}%
\pgfsetstrokecolor{currentstroke}%
\pgfsetdash{}{0pt}%
\pgfsys@defobject{currentmarker}{\pgfqpoint{0.000000in}{0.000000in}}{\pgfqpoint{0.000000in}{0.069444in}}{%
\pgfpathmoveto{\pgfqpoint{0.000000in}{0.000000in}}%
\pgfpathlineto{\pgfqpoint{0.000000in}{0.069444in}}%
\pgfusepath{stroke,fill}%
}%
\begin{pgfscope}%
\pgfsys@transformshift{2.305666in}{0.441418in}%
\pgfsys@useobject{currentmarker}{}%
\end{pgfscope}%
\end{pgfscope}%
\begin{pgfscope}%
\pgfsetbuttcap%
\pgfsetroundjoin%
\definecolor{currentfill}{rgb}{0.000000,0.000000,0.000000}%
\pgfsetfillcolor{currentfill}%
\pgfsetlinewidth{0.501875pt}%
\definecolor{currentstroke}{rgb}{0.000000,0.000000,0.000000}%
\pgfsetstrokecolor{currentstroke}%
\pgfsetdash{}{0pt}%
\pgfsys@defobject{currentmarker}{\pgfqpoint{0.000000in}{-0.069444in}}{\pgfqpoint{0.000000in}{0.000000in}}{%
\pgfpathmoveto{\pgfqpoint{0.000000in}{0.000000in}}%
\pgfpathlineto{\pgfqpoint{0.000000in}{-0.069444in}}%
\pgfusepath{stroke,fill}%
}%
\begin{pgfscope}%
\pgfsys@transformshift{2.305666in}{1.614961in}%
\pgfsys@useobject{currentmarker}{}%
\end{pgfscope}%
\end{pgfscope}%
\begin{pgfscope}%
\pgftext[x=2.305666in,y=0.371974in,,top]{\rmfamily\fontsize{8.000000}{9.600000}\selectfont 7}%
\end{pgfscope}%
\begin{pgfscope}%
\pgfsetbuttcap%
\pgfsetroundjoin%
\definecolor{currentfill}{rgb}{0.000000,0.000000,0.000000}%
\pgfsetfillcolor{currentfill}%
\pgfsetlinewidth{0.501875pt}%
\definecolor{currentstroke}{rgb}{0.000000,0.000000,0.000000}%
\pgfsetstrokecolor{currentstroke}%
\pgfsetdash{}{0pt}%
\pgfsys@defobject{currentmarker}{\pgfqpoint{0.000000in}{0.000000in}}{\pgfqpoint{0.000000in}{0.069444in}}{%
\pgfpathmoveto{\pgfqpoint{0.000000in}{0.000000in}}%
\pgfpathlineto{\pgfqpoint{0.000000in}{0.069444in}}%
\pgfusepath{stroke,fill}%
}%
\begin{pgfscope}%
\pgfsys@transformshift{2.592740in}{0.441418in}%
\pgfsys@useobject{currentmarker}{}%
\end{pgfscope}%
\end{pgfscope}%
\begin{pgfscope}%
\pgfsetbuttcap%
\pgfsetroundjoin%
\definecolor{currentfill}{rgb}{0.000000,0.000000,0.000000}%
\pgfsetfillcolor{currentfill}%
\pgfsetlinewidth{0.501875pt}%
\definecolor{currentstroke}{rgb}{0.000000,0.000000,0.000000}%
\pgfsetstrokecolor{currentstroke}%
\pgfsetdash{}{0pt}%
\pgfsys@defobject{currentmarker}{\pgfqpoint{0.000000in}{-0.069444in}}{\pgfqpoint{0.000000in}{0.000000in}}{%
\pgfpathmoveto{\pgfqpoint{0.000000in}{0.000000in}}%
\pgfpathlineto{\pgfqpoint{0.000000in}{-0.069444in}}%
\pgfusepath{stroke,fill}%
}%
\begin{pgfscope}%
\pgfsys@transformshift{2.592740in}{1.614961in}%
\pgfsys@useobject{currentmarker}{}%
\end{pgfscope}%
\end{pgfscope}%
\begin{pgfscope}%
\pgftext[x=2.592740in,y=0.371974in,,top]{\rmfamily\fontsize{8.000000}{9.600000}\selectfont 8}%
\end{pgfscope}%
\begin{pgfscope}%
\pgftext[x=1.444444in,y=0.194999in,,top]{\rmfamily\fontsize{9.000000}{10.800000}\selectfont \(\displaystyle B^0\ \mathrm{vertex}\ \chi^2 / \mathrm{ndf}\)}%
\end{pgfscope}%
\begin{pgfscope}%
\pgfsetbuttcap%
\pgfsetroundjoin%
\definecolor{currentfill}{rgb}{0.000000,0.000000,0.000000}%
\pgfsetfillcolor{currentfill}%
\pgfsetlinewidth{0.501875pt}%
\definecolor{currentstroke}{rgb}{0.000000,0.000000,0.000000}%
\pgfsetstrokecolor{currentstroke}%
\pgfsetdash{}{0pt}%
\pgfsys@defobject{currentmarker}{\pgfqpoint{0.000000in}{0.000000in}}{\pgfqpoint{0.069444in}{0.000000in}}{%
\pgfpathmoveto{\pgfqpoint{0.000000in}{0.000000in}}%
\pgfpathlineto{\pgfqpoint{0.069444in}{0.000000in}}%
\pgfusepath{stroke,fill}%
}%
\begin{pgfscope}%
\pgfsys@transformshift{0.296148in}{0.441418in}%
\pgfsys@useobject{currentmarker}{}%
\end{pgfscope}%
\end{pgfscope}%
\begin{pgfscope}%
\pgfsetbuttcap%
\pgfsetroundjoin%
\definecolor{currentfill}{rgb}{0.000000,0.000000,0.000000}%
\pgfsetfillcolor{currentfill}%
\pgfsetlinewidth{0.501875pt}%
\definecolor{currentstroke}{rgb}{0.000000,0.000000,0.000000}%
\pgfsetstrokecolor{currentstroke}%
\pgfsetdash{}{0pt}%
\pgfsys@defobject{currentmarker}{\pgfqpoint{-0.069444in}{0.000000in}}{\pgfqpoint{0.000000in}{0.000000in}}{%
\pgfpathmoveto{\pgfqpoint{0.000000in}{0.000000in}}%
\pgfpathlineto{\pgfqpoint{-0.069444in}{0.000000in}}%
\pgfusepath{stroke,fill}%
}%
\begin{pgfscope}%
\pgfsys@transformshift{2.592740in}{0.441418in}%
\pgfsys@useobject{currentmarker}{}%
\end{pgfscope}%
\end{pgfscope}%
\begin{pgfscope}%
\pgftext[x=0.226704in,y=0.441418in,right,]{\rmfamily\fontsize{8.000000}{9.600000}\selectfont 0.0}%
\end{pgfscope}%
\begin{pgfscope}%
\pgfsetbuttcap%
\pgfsetroundjoin%
\definecolor{currentfill}{rgb}{0.000000,0.000000,0.000000}%
\pgfsetfillcolor{currentfill}%
\pgfsetlinewidth{0.501875pt}%
\definecolor{currentstroke}{rgb}{0.000000,0.000000,0.000000}%
\pgfsetstrokecolor{currentstroke}%
\pgfsetdash{}{0pt}%
\pgfsys@defobject{currentmarker}{\pgfqpoint{0.000000in}{0.000000in}}{\pgfqpoint{0.069444in}{0.000000in}}{%
\pgfpathmoveto{\pgfqpoint{0.000000in}{0.000000in}}%
\pgfpathlineto{\pgfqpoint{0.069444in}{0.000000in}}%
\pgfusepath{stroke,fill}%
}%
\begin{pgfscope}%
\pgfsys@transformshift{0.296148in}{0.637009in}%
\pgfsys@useobject{currentmarker}{}%
\end{pgfscope}%
\end{pgfscope}%
\begin{pgfscope}%
\pgfsetbuttcap%
\pgfsetroundjoin%
\definecolor{currentfill}{rgb}{0.000000,0.000000,0.000000}%
\pgfsetfillcolor{currentfill}%
\pgfsetlinewidth{0.501875pt}%
\definecolor{currentstroke}{rgb}{0.000000,0.000000,0.000000}%
\pgfsetstrokecolor{currentstroke}%
\pgfsetdash{}{0pt}%
\pgfsys@defobject{currentmarker}{\pgfqpoint{-0.069444in}{0.000000in}}{\pgfqpoint{0.000000in}{0.000000in}}{%
\pgfpathmoveto{\pgfqpoint{0.000000in}{0.000000in}}%
\pgfpathlineto{\pgfqpoint{-0.069444in}{0.000000in}}%
\pgfusepath{stroke,fill}%
}%
\begin{pgfscope}%
\pgfsys@transformshift{2.592740in}{0.637009in}%
\pgfsys@useobject{currentmarker}{}%
\end{pgfscope}%
\end{pgfscope}%
\begin{pgfscope}%
\pgftext[x=0.226704in,y=0.637009in,right,]{\rmfamily\fontsize{8.000000}{9.600000}\selectfont 0.1}%
\end{pgfscope}%
\begin{pgfscope}%
\pgfsetbuttcap%
\pgfsetroundjoin%
\definecolor{currentfill}{rgb}{0.000000,0.000000,0.000000}%
\pgfsetfillcolor{currentfill}%
\pgfsetlinewidth{0.501875pt}%
\definecolor{currentstroke}{rgb}{0.000000,0.000000,0.000000}%
\pgfsetstrokecolor{currentstroke}%
\pgfsetdash{}{0pt}%
\pgfsys@defobject{currentmarker}{\pgfqpoint{0.000000in}{0.000000in}}{\pgfqpoint{0.069444in}{0.000000in}}{%
\pgfpathmoveto{\pgfqpoint{0.000000in}{0.000000in}}%
\pgfpathlineto{\pgfqpoint{0.069444in}{0.000000in}}%
\pgfusepath{stroke,fill}%
}%
\begin{pgfscope}%
\pgfsys@transformshift{0.296148in}{0.832599in}%
\pgfsys@useobject{currentmarker}{}%
\end{pgfscope}%
\end{pgfscope}%
\begin{pgfscope}%
\pgfsetbuttcap%
\pgfsetroundjoin%
\definecolor{currentfill}{rgb}{0.000000,0.000000,0.000000}%
\pgfsetfillcolor{currentfill}%
\pgfsetlinewidth{0.501875pt}%
\definecolor{currentstroke}{rgb}{0.000000,0.000000,0.000000}%
\pgfsetstrokecolor{currentstroke}%
\pgfsetdash{}{0pt}%
\pgfsys@defobject{currentmarker}{\pgfqpoint{-0.069444in}{0.000000in}}{\pgfqpoint{0.000000in}{0.000000in}}{%
\pgfpathmoveto{\pgfqpoint{0.000000in}{0.000000in}}%
\pgfpathlineto{\pgfqpoint{-0.069444in}{0.000000in}}%
\pgfusepath{stroke,fill}%
}%
\begin{pgfscope}%
\pgfsys@transformshift{2.592740in}{0.832599in}%
\pgfsys@useobject{currentmarker}{}%
\end{pgfscope}%
\end{pgfscope}%
\begin{pgfscope}%
\pgftext[x=0.226704in,y=0.832599in,right,]{\rmfamily\fontsize{8.000000}{9.600000}\selectfont 0.2}%
\end{pgfscope}%
\begin{pgfscope}%
\pgfsetbuttcap%
\pgfsetroundjoin%
\definecolor{currentfill}{rgb}{0.000000,0.000000,0.000000}%
\pgfsetfillcolor{currentfill}%
\pgfsetlinewidth{0.501875pt}%
\definecolor{currentstroke}{rgb}{0.000000,0.000000,0.000000}%
\pgfsetstrokecolor{currentstroke}%
\pgfsetdash{}{0pt}%
\pgfsys@defobject{currentmarker}{\pgfqpoint{0.000000in}{0.000000in}}{\pgfqpoint{0.069444in}{0.000000in}}{%
\pgfpathmoveto{\pgfqpoint{0.000000in}{0.000000in}}%
\pgfpathlineto{\pgfqpoint{0.069444in}{0.000000in}}%
\pgfusepath{stroke,fill}%
}%
\begin{pgfscope}%
\pgfsys@transformshift{0.296148in}{1.028190in}%
\pgfsys@useobject{currentmarker}{}%
\end{pgfscope}%
\end{pgfscope}%
\begin{pgfscope}%
\pgfsetbuttcap%
\pgfsetroundjoin%
\definecolor{currentfill}{rgb}{0.000000,0.000000,0.000000}%
\pgfsetfillcolor{currentfill}%
\pgfsetlinewidth{0.501875pt}%
\definecolor{currentstroke}{rgb}{0.000000,0.000000,0.000000}%
\pgfsetstrokecolor{currentstroke}%
\pgfsetdash{}{0pt}%
\pgfsys@defobject{currentmarker}{\pgfqpoint{-0.069444in}{0.000000in}}{\pgfqpoint{0.000000in}{0.000000in}}{%
\pgfpathmoveto{\pgfqpoint{0.000000in}{0.000000in}}%
\pgfpathlineto{\pgfqpoint{-0.069444in}{0.000000in}}%
\pgfusepath{stroke,fill}%
}%
\begin{pgfscope}%
\pgfsys@transformshift{2.592740in}{1.028190in}%
\pgfsys@useobject{currentmarker}{}%
\end{pgfscope}%
\end{pgfscope}%
\begin{pgfscope}%
\pgftext[x=0.226704in,y=1.028190in,right,]{\rmfamily\fontsize{8.000000}{9.600000}\selectfont 0.3}%
\end{pgfscope}%
\begin{pgfscope}%
\pgfsetbuttcap%
\pgfsetroundjoin%
\definecolor{currentfill}{rgb}{0.000000,0.000000,0.000000}%
\pgfsetfillcolor{currentfill}%
\pgfsetlinewidth{0.501875pt}%
\definecolor{currentstroke}{rgb}{0.000000,0.000000,0.000000}%
\pgfsetstrokecolor{currentstroke}%
\pgfsetdash{}{0pt}%
\pgfsys@defobject{currentmarker}{\pgfqpoint{0.000000in}{0.000000in}}{\pgfqpoint{0.069444in}{0.000000in}}{%
\pgfpathmoveto{\pgfqpoint{0.000000in}{0.000000in}}%
\pgfpathlineto{\pgfqpoint{0.069444in}{0.000000in}}%
\pgfusepath{stroke,fill}%
}%
\begin{pgfscope}%
\pgfsys@transformshift{0.296148in}{1.223780in}%
\pgfsys@useobject{currentmarker}{}%
\end{pgfscope}%
\end{pgfscope}%
\begin{pgfscope}%
\pgfsetbuttcap%
\pgfsetroundjoin%
\definecolor{currentfill}{rgb}{0.000000,0.000000,0.000000}%
\pgfsetfillcolor{currentfill}%
\pgfsetlinewidth{0.501875pt}%
\definecolor{currentstroke}{rgb}{0.000000,0.000000,0.000000}%
\pgfsetstrokecolor{currentstroke}%
\pgfsetdash{}{0pt}%
\pgfsys@defobject{currentmarker}{\pgfqpoint{-0.069444in}{0.000000in}}{\pgfqpoint{0.000000in}{0.000000in}}{%
\pgfpathmoveto{\pgfqpoint{0.000000in}{0.000000in}}%
\pgfpathlineto{\pgfqpoint{-0.069444in}{0.000000in}}%
\pgfusepath{stroke,fill}%
}%
\begin{pgfscope}%
\pgfsys@transformshift{2.592740in}{1.223780in}%
\pgfsys@useobject{currentmarker}{}%
\end{pgfscope}%
\end{pgfscope}%
\begin{pgfscope}%
\pgftext[x=0.226704in,y=1.223780in,right,]{\rmfamily\fontsize{8.000000}{9.600000}\selectfont 0.4}%
\end{pgfscope}%
\begin{pgfscope}%
\pgfsetbuttcap%
\pgfsetroundjoin%
\definecolor{currentfill}{rgb}{0.000000,0.000000,0.000000}%
\pgfsetfillcolor{currentfill}%
\pgfsetlinewidth{0.501875pt}%
\definecolor{currentstroke}{rgb}{0.000000,0.000000,0.000000}%
\pgfsetstrokecolor{currentstroke}%
\pgfsetdash{}{0pt}%
\pgfsys@defobject{currentmarker}{\pgfqpoint{0.000000in}{0.000000in}}{\pgfqpoint{0.069444in}{0.000000in}}{%
\pgfpathmoveto{\pgfqpoint{0.000000in}{0.000000in}}%
\pgfpathlineto{\pgfqpoint{0.069444in}{0.000000in}}%
\pgfusepath{stroke,fill}%
}%
\begin{pgfscope}%
\pgfsys@transformshift{0.296148in}{1.419371in}%
\pgfsys@useobject{currentmarker}{}%
\end{pgfscope}%
\end{pgfscope}%
\begin{pgfscope}%
\pgfsetbuttcap%
\pgfsetroundjoin%
\definecolor{currentfill}{rgb}{0.000000,0.000000,0.000000}%
\pgfsetfillcolor{currentfill}%
\pgfsetlinewidth{0.501875pt}%
\definecolor{currentstroke}{rgb}{0.000000,0.000000,0.000000}%
\pgfsetstrokecolor{currentstroke}%
\pgfsetdash{}{0pt}%
\pgfsys@defobject{currentmarker}{\pgfqpoint{-0.069444in}{0.000000in}}{\pgfqpoint{0.000000in}{0.000000in}}{%
\pgfpathmoveto{\pgfqpoint{0.000000in}{0.000000in}}%
\pgfpathlineto{\pgfqpoint{-0.069444in}{0.000000in}}%
\pgfusepath{stroke,fill}%
}%
\begin{pgfscope}%
\pgfsys@transformshift{2.592740in}{1.419371in}%
\pgfsys@useobject{currentmarker}{}%
\end{pgfscope}%
\end{pgfscope}%
\begin{pgfscope}%
\pgftext[x=0.226704in,y=1.419371in,right,]{\rmfamily\fontsize{8.000000}{9.600000}\selectfont 0.5}%
\end{pgfscope}%
\begin{pgfscope}%
\pgfsetbuttcap%
\pgfsetroundjoin%
\definecolor{currentfill}{rgb}{0.000000,0.000000,0.000000}%
\pgfsetfillcolor{currentfill}%
\pgfsetlinewidth{0.501875pt}%
\definecolor{currentstroke}{rgb}{0.000000,0.000000,0.000000}%
\pgfsetstrokecolor{currentstroke}%
\pgfsetdash{}{0pt}%
\pgfsys@defobject{currentmarker}{\pgfqpoint{0.000000in}{0.000000in}}{\pgfqpoint{0.069444in}{0.000000in}}{%
\pgfpathmoveto{\pgfqpoint{0.000000in}{0.000000in}}%
\pgfpathlineto{\pgfqpoint{0.069444in}{0.000000in}}%
\pgfusepath{stroke,fill}%
}%
\begin{pgfscope}%
\pgfsys@transformshift{0.296148in}{1.614961in}%
\pgfsys@useobject{currentmarker}{}%
\end{pgfscope}%
\end{pgfscope}%
\begin{pgfscope}%
\pgfsetbuttcap%
\pgfsetroundjoin%
\definecolor{currentfill}{rgb}{0.000000,0.000000,0.000000}%
\pgfsetfillcolor{currentfill}%
\pgfsetlinewidth{0.501875pt}%
\definecolor{currentstroke}{rgb}{0.000000,0.000000,0.000000}%
\pgfsetstrokecolor{currentstroke}%
\pgfsetdash{}{0pt}%
\pgfsys@defobject{currentmarker}{\pgfqpoint{-0.069444in}{0.000000in}}{\pgfqpoint{0.000000in}{0.000000in}}{%
\pgfpathmoveto{\pgfqpoint{0.000000in}{0.000000in}}%
\pgfpathlineto{\pgfqpoint{-0.069444in}{0.000000in}}%
\pgfusepath{stroke,fill}%
}%
\begin{pgfscope}%
\pgfsys@transformshift{2.592740in}{1.614961in}%
\pgfsys@useobject{currentmarker}{}%
\end{pgfscope}%
\end{pgfscope}%
\begin{pgfscope}%
\pgftext[x=0.226704in,y=1.614961in,right,]{\rmfamily\fontsize{8.000000}{9.600000}\selectfont 0.6}%
\end{pgfscope}%
\end{pgfpicture}%
\makeatother%
\endgroup%

	\end{subfigure}
  
	\begin{subfigure}[t]{0.49\textwidth}
		\centering
    %\includegraphics[width=\textwidth]{store/variables/DATA_MC_REWEIGHTED_B_ISOLATION_BDT_Soft.pdf}
    %% Creator: Matplotlib, PGF backend
%%
%% To include the figure in your LaTeX document, write
%%   \input{<filename>.pgf}
%%
%% Make sure the required packages are loaded in your preamble
%%   \usepackage{pgf}
%%
%% Figures using additional raster images can only be included by \input if
%% they are in the same directory as the main LaTeX file. For loading figures
%% from other directories you can use the `import` package
%%   \usepackage{import}
%% and then include the figures with
%%   \import{<path to file>}{<filename>.pgf}
%%
%% Matplotlib used the following preamble
%%   \usepackage{fontspec}
%%   \setmainfont{DejaVu Serif}
%%   \setsansfont{DejaVu Sans}
%%   \setmonofont{DejaVu Sans Mono}
%%
\begingroup%
\makeatletter%
\begin{pgfpicture}%
\pgfpathrectangle{\pgfpointorigin}{\pgfqpoint{2.680146in}{1.787280in}}%
\pgfusepath{use as bounding box, clip}%
\begin{pgfscope}%
\pgfsetbuttcap%
\pgfsetmiterjoin%
\definecolor{currentfill}{rgb}{1.000000,1.000000,1.000000}%
\pgfsetfillcolor{currentfill}%
\pgfsetlinewidth{0.000000pt}%
\definecolor{currentstroke}{rgb}{1.000000,1.000000,1.000000}%
\pgfsetstrokecolor{currentstroke}%
\pgfsetdash{}{0pt}%
\pgfpathmoveto{\pgfqpoint{0.000000in}{0.000000in}}%
\pgfpathlineto{\pgfqpoint{2.680146in}{0.000000in}}%
\pgfpathlineto{\pgfqpoint{2.680146in}{1.787280in}}%
\pgfpathlineto{\pgfqpoint{0.000000in}{1.787280in}}%
\pgfpathclose%
\pgfusepath{fill}%
\end{pgfscope}%
\begin{pgfscope}%
\pgfsetbuttcap%
\pgfsetmiterjoin%
\definecolor{currentfill}{rgb}{1.000000,1.000000,1.000000}%
\pgfsetfillcolor{currentfill}%
\pgfsetlinewidth{0.000000pt}%
\definecolor{currentstroke}{rgb}{0.000000,0.000000,0.000000}%
\pgfsetstrokecolor{currentstroke}%
\pgfsetstrokeopacity{0.000000}%
\pgfsetdash{}{0pt}%
\pgfpathmoveto{\pgfqpoint{0.296148in}{0.417391in}}%
\pgfpathlineto{\pgfqpoint{2.541794in}{0.417391in}}%
\pgfpathlineto{\pgfqpoint{2.541794in}{1.683515in}}%
\pgfpathlineto{\pgfqpoint{0.296148in}{1.683515in}}%
\pgfpathclose%
\pgfusepath{fill}%
\end{pgfscope}%
\begin{pgfscope}%
\pgfpathrectangle{\pgfqpoint{0.296148in}{0.417391in}}{\pgfqpoint{2.245646in}{1.266124in}} %
\pgfusepath{clip}%
\pgfsetbuttcap%
\pgfsetmiterjoin%
\definecolor{currentfill}{rgb}{0.215686,0.470588,0.749020}%
\pgfsetfillcolor{currentfill}%
\pgfsetlinewidth{0.000000pt}%
\definecolor{currentstroke}{rgb}{0.000000,0.000000,0.000000}%
\pgfsetstrokecolor{currentstroke}%
\pgfsetdash{}{0pt}%
\pgfpathmoveto{\pgfqpoint{0.296148in}{0.417391in}}%
\pgfpathlineto{\pgfqpoint{0.296148in}{1.526353in}}%
\pgfpathlineto{\pgfqpoint{0.335506in}{1.526353in}}%
\pgfpathlineto{\pgfqpoint{0.335506in}{0.417391in}}%
\pgfpathlineto{\pgfqpoint{0.374864in}{0.417391in}}%
\pgfpathlineto{\pgfqpoint{0.374864in}{0.417391in}}%
\pgfpathlineto{\pgfqpoint{0.414221in}{0.417391in}}%
\pgfpathlineto{\pgfqpoint{0.414221in}{0.417391in}}%
\pgfpathlineto{\pgfqpoint{0.453579in}{0.417391in}}%
\pgfpathlineto{\pgfqpoint{0.453579in}{0.417391in}}%
\pgfpathlineto{\pgfqpoint{0.492937in}{0.417391in}}%
\pgfpathlineto{\pgfqpoint{0.492937in}{0.417391in}}%
\pgfpathlineto{\pgfqpoint{0.532295in}{0.417391in}}%
\pgfpathlineto{\pgfqpoint{0.532295in}{0.417391in}}%
\pgfpathlineto{\pgfqpoint{0.571652in}{0.417391in}}%
\pgfpathlineto{\pgfqpoint{0.571652in}{0.417391in}}%
\pgfpathlineto{\pgfqpoint{0.611010in}{0.417391in}}%
\pgfpathlineto{\pgfqpoint{0.611010in}{0.417391in}}%
\pgfpathlineto{\pgfqpoint{0.650368in}{0.417391in}}%
\pgfpathlineto{\pgfqpoint{0.650368in}{0.417391in}}%
\pgfpathlineto{\pgfqpoint{0.689726in}{0.417391in}}%
\pgfpathlineto{\pgfqpoint{0.689726in}{0.417391in}}%
\pgfpathlineto{\pgfqpoint{0.729083in}{0.417391in}}%
\pgfpathlineto{\pgfqpoint{0.729083in}{0.417391in}}%
\pgfpathlineto{\pgfqpoint{0.768441in}{0.417391in}}%
\pgfpathlineto{\pgfqpoint{0.768441in}{0.417391in}}%
\pgfpathlineto{\pgfqpoint{0.807799in}{0.417391in}}%
\pgfpathlineto{\pgfqpoint{0.807799in}{0.417391in}}%
\pgfpathlineto{\pgfqpoint{0.847157in}{0.417391in}}%
\pgfpathlineto{\pgfqpoint{0.847157in}{0.417391in}}%
\pgfpathlineto{\pgfqpoint{0.886514in}{0.417391in}}%
\pgfpathlineto{\pgfqpoint{0.886514in}{0.417391in}}%
\pgfpathlineto{\pgfqpoint{0.925872in}{0.417391in}}%
\pgfpathlineto{\pgfqpoint{0.925872in}{0.417391in}}%
\pgfpathlineto{\pgfqpoint{0.965230in}{0.417391in}}%
\pgfpathlineto{\pgfqpoint{0.965230in}{0.417391in}}%
\pgfpathlineto{\pgfqpoint{1.004588in}{0.417391in}}%
\pgfpathlineto{\pgfqpoint{1.004588in}{0.417391in}}%
\pgfpathlineto{\pgfqpoint{1.043945in}{0.417391in}}%
\pgfpathlineto{\pgfqpoint{1.043945in}{0.418247in}}%
\pgfpathlineto{\pgfqpoint{1.083303in}{0.418247in}}%
\pgfpathlineto{\pgfqpoint{1.083303in}{0.420360in}}%
\pgfpathlineto{\pgfqpoint{1.122661in}{0.420360in}}%
\pgfpathlineto{\pgfqpoint{1.122661in}{0.421642in}}%
\pgfpathlineto{\pgfqpoint{1.162019in}{0.421642in}}%
\pgfpathlineto{\pgfqpoint{1.162019in}{0.440915in}}%
\pgfpathlineto{\pgfqpoint{1.201376in}{0.440915in}}%
\pgfpathlineto{\pgfqpoint{1.201376in}{0.441587in}}%
\pgfpathlineto{\pgfqpoint{1.240734in}{0.441587in}}%
\pgfpathlineto{\pgfqpoint{1.240734in}{0.460514in}}%
\pgfpathlineto{\pgfqpoint{1.280092in}{0.460514in}}%
\pgfpathlineto{\pgfqpoint{1.280092in}{0.473106in}}%
\pgfpathlineto{\pgfqpoint{1.319450in}{0.473106in}}%
\pgfpathlineto{\pgfqpoint{1.319450in}{0.480505in}}%
\pgfpathlineto{\pgfqpoint{1.358807in}{0.480505in}}%
\pgfpathlineto{\pgfqpoint{1.358807in}{0.489116in}}%
\pgfpathlineto{\pgfqpoint{1.398165in}{0.489116in}}%
\pgfpathlineto{\pgfqpoint{1.398165in}{0.525940in}}%
\pgfpathlineto{\pgfqpoint{1.437523in}{0.525940in}}%
\pgfpathlineto{\pgfqpoint{1.437523in}{0.516444in}}%
\pgfpathlineto{\pgfqpoint{1.476881in}{0.516444in}}%
\pgfpathlineto{\pgfqpoint{1.476881in}{0.565888in}}%
\pgfpathlineto{\pgfqpoint{1.516238in}{0.565888in}}%
\pgfpathlineto{\pgfqpoint{1.516238in}{0.619494in}}%
\pgfpathlineto{\pgfqpoint{1.555596in}{0.619494in}}%
\pgfpathlineto{\pgfqpoint{1.555596in}{0.705779in}}%
\pgfpathlineto{\pgfqpoint{1.594954in}{0.705779in}}%
\pgfpathlineto{\pgfqpoint{1.594954in}{0.626069in}}%
\pgfpathlineto{\pgfqpoint{1.634312in}{0.626069in}}%
\pgfpathlineto{\pgfqpoint{1.634312in}{0.695000in}}%
\pgfpathlineto{\pgfqpoint{1.673669in}{0.695000in}}%
\pgfpathlineto{\pgfqpoint{1.673669in}{0.719928in}}%
\pgfpathlineto{\pgfqpoint{1.713027in}{0.719928in}}%
\pgfpathlineto{\pgfqpoint{1.713027in}{0.951367in}}%
\pgfpathlineto{\pgfqpoint{1.752385in}{0.951367in}}%
\pgfpathlineto{\pgfqpoint{1.752385in}{0.774528in}}%
\pgfpathlineto{\pgfqpoint{1.791743in}{0.774528in}}%
\pgfpathlineto{\pgfqpoint{1.791743in}{0.763351in}}%
\pgfpathlineto{\pgfqpoint{1.831100in}{0.763351in}}%
\pgfpathlineto{\pgfqpoint{1.831100in}{0.602207in}}%
\pgfpathlineto{\pgfqpoint{1.870458in}{0.602207in}}%
\pgfpathlineto{\pgfqpoint{1.870458in}{0.554745in}}%
\pgfpathlineto{\pgfqpoint{1.909816in}{0.554745in}}%
\pgfpathlineto{\pgfqpoint{1.909816in}{0.713414in}}%
\pgfpathlineto{\pgfqpoint{1.949174in}{0.713414in}}%
\pgfpathlineto{\pgfqpoint{1.949174in}{0.515791in}}%
\pgfpathlineto{\pgfqpoint{1.988531in}{0.515791in}}%
\pgfpathlineto{\pgfqpoint{1.988531in}{0.549127in}}%
\pgfpathlineto{\pgfqpoint{2.027889in}{0.549127in}}%
\pgfpathlineto{\pgfqpoint{2.027889in}{0.519606in}}%
\pgfpathlineto{\pgfqpoint{2.067247in}{0.519606in}}%
\pgfpathlineto{\pgfqpoint{2.067247in}{0.468719in}}%
\pgfpathlineto{\pgfqpoint{2.106605in}{0.468719in}}%
\pgfpathlineto{\pgfqpoint{2.106605in}{0.443615in}}%
\pgfpathlineto{\pgfqpoint{2.145962in}{0.443615in}}%
\pgfpathlineto{\pgfqpoint{2.145962in}{0.451028in}}%
\pgfpathlineto{\pgfqpoint{2.185320in}{0.451028in}}%
\pgfpathlineto{\pgfqpoint{2.185320in}{0.431290in}}%
\pgfpathlineto{\pgfqpoint{2.224678in}{0.431290in}}%
\pgfpathlineto{\pgfqpoint{2.224678in}{0.422066in}}%
\pgfpathlineto{\pgfqpoint{2.264036in}{0.422066in}}%
\pgfpathlineto{\pgfqpoint{2.264036in}{0.417391in}}%
\pgfpathlineto{\pgfqpoint{2.224678in}{0.417391in}}%
\pgfpathlineto{\pgfqpoint{2.224678in}{0.417391in}}%
\pgfpathlineto{\pgfqpoint{2.185320in}{0.417391in}}%
\pgfpathlineto{\pgfqpoint{2.185320in}{0.417391in}}%
\pgfpathlineto{\pgfqpoint{2.145962in}{0.417391in}}%
\pgfpathlineto{\pgfqpoint{2.145962in}{0.417391in}}%
\pgfpathlineto{\pgfqpoint{2.106605in}{0.417391in}}%
\pgfpathlineto{\pgfqpoint{2.106605in}{0.417391in}}%
\pgfpathlineto{\pgfqpoint{2.067247in}{0.417391in}}%
\pgfpathlineto{\pgfqpoint{2.067247in}{0.417391in}}%
\pgfpathlineto{\pgfqpoint{2.027889in}{0.417391in}}%
\pgfpathlineto{\pgfqpoint{2.027889in}{0.417391in}}%
\pgfpathlineto{\pgfqpoint{1.988531in}{0.417391in}}%
\pgfpathlineto{\pgfqpoint{1.988531in}{0.417391in}}%
\pgfpathlineto{\pgfqpoint{1.949174in}{0.417391in}}%
\pgfpathlineto{\pgfqpoint{1.949174in}{0.417391in}}%
\pgfpathlineto{\pgfqpoint{1.909816in}{0.417391in}}%
\pgfpathlineto{\pgfqpoint{1.909816in}{0.417391in}}%
\pgfpathlineto{\pgfqpoint{1.870458in}{0.417391in}}%
\pgfpathlineto{\pgfqpoint{1.870458in}{0.417391in}}%
\pgfpathlineto{\pgfqpoint{1.831100in}{0.417391in}}%
\pgfpathlineto{\pgfqpoint{1.831100in}{0.417391in}}%
\pgfpathlineto{\pgfqpoint{1.791743in}{0.417391in}}%
\pgfpathlineto{\pgfqpoint{1.791743in}{0.417391in}}%
\pgfpathlineto{\pgfqpoint{1.752385in}{0.417391in}}%
\pgfpathlineto{\pgfqpoint{1.752385in}{0.417391in}}%
\pgfpathlineto{\pgfqpoint{1.713027in}{0.417391in}}%
\pgfpathlineto{\pgfqpoint{1.713027in}{0.417391in}}%
\pgfpathlineto{\pgfqpoint{1.673669in}{0.417391in}}%
\pgfpathlineto{\pgfqpoint{1.673669in}{0.417391in}}%
\pgfpathlineto{\pgfqpoint{1.634312in}{0.417391in}}%
\pgfpathlineto{\pgfqpoint{1.634312in}{0.417391in}}%
\pgfpathlineto{\pgfqpoint{1.594954in}{0.417391in}}%
\pgfpathlineto{\pgfqpoint{1.594954in}{0.417391in}}%
\pgfpathlineto{\pgfqpoint{1.555596in}{0.417391in}}%
\pgfpathlineto{\pgfqpoint{1.555596in}{0.417391in}}%
\pgfpathlineto{\pgfqpoint{1.516238in}{0.417391in}}%
\pgfpathlineto{\pgfqpoint{1.516238in}{0.417391in}}%
\pgfpathlineto{\pgfqpoint{1.476881in}{0.417391in}}%
\pgfpathlineto{\pgfqpoint{1.476881in}{0.417391in}}%
\pgfpathlineto{\pgfqpoint{1.437523in}{0.417391in}}%
\pgfpathlineto{\pgfqpoint{1.437523in}{0.417391in}}%
\pgfpathlineto{\pgfqpoint{1.398165in}{0.417391in}}%
\pgfpathlineto{\pgfqpoint{1.398165in}{0.417391in}}%
\pgfpathlineto{\pgfqpoint{1.358807in}{0.417391in}}%
\pgfpathlineto{\pgfqpoint{1.358807in}{0.417391in}}%
\pgfpathlineto{\pgfqpoint{1.319450in}{0.417391in}}%
\pgfpathlineto{\pgfqpoint{1.319450in}{0.417391in}}%
\pgfpathlineto{\pgfqpoint{1.280092in}{0.417391in}}%
\pgfpathlineto{\pgfqpoint{1.280092in}{0.417391in}}%
\pgfpathlineto{\pgfqpoint{1.240734in}{0.417391in}}%
\pgfpathlineto{\pgfqpoint{1.240734in}{0.417391in}}%
\pgfpathlineto{\pgfqpoint{1.201376in}{0.417391in}}%
\pgfpathlineto{\pgfqpoint{1.201376in}{0.417391in}}%
\pgfpathlineto{\pgfqpoint{1.162019in}{0.417391in}}%
\pgfpathlineto{\pgfqpoint{1.162019in}{0.417391in}}%
\pgfpathlineto{\pgfqpoint{1.122661in}{0.417391in}}%
\pgfpathlineto{\pgfqpoint{1.122661in}{0.417391in}}%
\pgfpathlineto{\pgfqpoint{1.083303in}{0.417391in}}%
\pgfpathlineto{\pgfqpoint{1.083303in}{0.417391in}}%
\pgfpathlineto{\pgfqpoint{1.043945in}{0.417391in}}%
\pgfpathlineto{\pgfqpoint{1.043945in}{0.417391in}}%
\pgfpathlineto{\pgfqpoint{1.004588in}{0.417391in}}%
\pgfpathlineto{\pgfqpoint{1.004588in}{0.417391in}}%
\pgfpathlineto{\pgfqpoint{0.965230in}{0.417391in}}%
\pgfpathlineto{\pgfqpoint{0.965230in}{0.417391in}}%
\pgfpathlineto{\pgfqpoint{0.925872in}{0.417391in}}%
\pgfpathlineto{\pgfqpoint{0.925872in}{0.417391in}}%
\pgfpathlineto{\pgfqpoint{0.886514in}{0.417391in}}%
\pgfpathlineto{\pgfqpoint{0.886514in}{0.417391in}}%
\pgfpathlineto{\pgfqpoint{0.847157in}{0.417391in}}%
\pgfpathlineto{\pgfqpoint{0.847157in}{0.417391in}}%
\pgfpathlineto{\pgfqpoint{0.807799in}{0.417391in}}%
\pgfpathlineto{\pgfqpoint{0.807799in}{0.417391in}}%
\pgfpathlineto{\pgfqpoint{0.768441in}{0.417391in}}%
\pgfpathlineto{\pgfqpoint{0.768441in}{0.417391in}}%
\pgfpathlineto{\pgfqpoint{0.729083in}{0.417391in}}%
\pgfpathlineto{\pgfqpoint{0.729083in}{0.417391in}}%
\pgfpathlineto{\pgfqpoint{0.689726in}{0.417391in}}%
\pgfpathlineto{\pgfqpoint{0.689726in}{0.417391in}}%
\pgfpathlineto{\pgfqpoint{0.650368in}{0.417391in}}%
\pgfpathlineto{\pgfqpoint{0.650368in}{0.417391in}}%
\pgfpathlineto{\pgfqpoint{0.611010in}{0.417391in}}%
\pgfpathlineto{\pgfqpoint{0.611010in}{0.417391in}}%
\pgfpathlineto{\pgfqpoint{0.571652in}{0.417391in}}%
\pgfpathlineto{\pgfqpoint{0.571652in}{0.417391in}}%
\pgfpathlineto{\pgfqpoint{0.532295in}{0.417391in}}%
\pgfpathlineto{\pgfqpoint{0.532295in}{0.417391in}}%
\pgfpathlineto{\pgfqpoint{0.492937in}{0.417391in}}%
\pgfpathlineto{\pgfqpoint{0.492937in}{0.417391in}}%
\pgfpathlineto{\pgfqpoint{0.453579in}{0.417391in}}%
\pgfpathlineto{\pgfqpoint{0.453579in}{0.417391in}}%
\pgfpathlineto{\pgfqpoint{0.414221in}{0.417391in}}%
\pgfpathlineto{\pgfqpoint{0.414221in}{0.417391in}}%
\pgfpathlineto{\pgfqpoint{0.374864in}{0.417391in}}%
\pgfpathlineto{\pgfqpoint{0.374864in}{0.417391in}}%
\pgfpathlineto{\pgfqpoint{0.335506in}{0.417391in}}%
\pgfpathlineto{\pgfqpoint{0.335506in}{0.417391in}}%
\pgfpathlineto{\pgfqpoint{0.296148in}{0.417391in}}%
\pgfusepath{fill}%
\end{pgfscope}%
\begin{pgfscope}%
\pgfpathrectangle{\pgfqpoint{0.296148in}{0.417391in}}{\pgfqpoint{2.245646in}{1.266124in}} %
\pgfusepath{clip}%
\pgfsetbuttcap%
\pgfsetmiterjoin%
\pgfsetlinewidth{0.501875pt}%
\definecolor{currentstroke}{rgb}{1.000000,0.000000,0.000000}%
\pgfsetstrokecolor{currentstroke}%
\pgfsetdash{}{0pt}%
\pgfpathmoveto{\pgfqpoint{0.296148in}{0.417391in}}%
\pgfpathlineto{\pgfqpoint{0.296148in}{1.629397in}}%
\pgfpathlineto{\pgfqpoint{0.335506in}{1.629397in}}%
\pgfpathlineto{\pgfqpoint{0.335506in}{0.417391in}}%
\pgfpathlineto{\pgfqpoint{0.374864in}{0.417391in}}%
\pgfpathlineto{\pgfqpoint{0.374864in}{0.417391in}}%
\pgfpathlineto{\pgfqpoint{0.414221in}{0.417391in}}%
\pgfpathlineto{\pgfqpoint{0.414221in}{0.417391in}}%
\pgfpathlineto{\pgfqpoint{0.453579in}{0.417391in}}%
\pgfpathlineto{\pgfqpoint{0.453579in}{0.417391in}}%
\pgfpathlineto{\pgfqpoint{0.492937in}{0.417391in}}%
\pgfpathlineto{\pgfqpoint{0.492937in}{0.417391in}}%
\pgfpathlineto{\pgfqpoint{0.532295in}{0.417391in}}%
\pgfpathlineto{\pgfqpoint{0.532295in}{0.417391in}}%
\pgfpathlineto{\pgfqpoint{0.571652in}{0.417391in}}%
\pgfpathlineto{\pgfqpoint{0.571652in}{0.417391in}}%
\pgfpathlineto{\pgfqpoint{0.611010in}{0.417391in}}%
\pgfpathlineto{\pgfqpoint{0.611010in}{0.417391in}}%
\pgfpathlineto{\pgfqpoint{0.650368in}{0.417391in}}%
\pgfpathlineto{\pgfqpoint{0.650368in}{0.417391in}}%
\pgfpathlineto{\pgfqpoint{0.689726in}{0.417391in}}%
\pgfpathlineto{\pgfqpoint{0.689726in}{0.417391in}}%
\pgfpathlineto{\pgfqpoint{0.729083in}{0.417391in}}%
\pgfpathlineto{\pgfqpoint{0.729083in}{0.417391in}}%
\pgfpathlineto{\pgfqpoint{0.768441in}{0.417391in}}%
\pgfpathlineto{\pgfqpoint{0.768441in}{0.417391in}}%
\pgfpathlineto{\pgfqpoint{0.807799in}{0.417391in}}%
\pgfpathlineto{\pgfqpoint{0.807799in}{0.417391in}}%
\pgfpathlineto{\pgfqpoint{0.847157in}{0.417391in}}%
\pgfpathlineto{\pgfqpoint{0.847157in}{0.417391in}}%
\pgfpathlineto{\pgfqpoint{0.886514in}{0.417391in}}%
\pgfpathlineto{\pgfqpoint{0.886514in}{0.417391in}}%
\pgfpathlineto{\pgfqpoint{0.925872in}{0.417391in}}%
\pgfpathlineto{\pgfqpoint{0.925872in}{0.417391in}}%
\pgfpathlineto{\pgfqpoint{0.965230in}{0.417391in}}%
\pgfpathlineto{\pgfqpoint{0.965230in}{0.417391in}}%
\pgfpathlineto{\pgfqpoint{1.004588in}{0.417391in}}%
\pgfpathlineto{\pgfqpoint{1.004588in}{0.417391in}}%
\pgfpathlineto{\pgfqpoint{1.043945in}{0.417391in}}%
\pgfpathlineto{\pgfqpoint{1.043945in}{0.418492in}}%
\pgfpathlineto{\pgfqpoint{1.083303in}{0.418492in}}%
\pgfpathlineto{\pgfqpoint{1.083303in}{0.422573in}}%
\pgfpathlineto{\pgfqpoint{1.122661in}{0.422573in}}%
\pgfpathlineto{\pgfqpoint{1.122661in}{0.422788in}}%
\pgfpathlineto{\pgfqpoint{1.162019in}{0.422788in}}%
\pgfpathlineto{\pgfqpoint{1.162019in}{0.447005in}}%
\pgfpathlineto{\pgfqpoint{1.201376in}{0.447005in}}%
\pgfpathlineto{\pgfqpoint{1.201376in}{0.448173in}}%
\pgfpathlineto{\pgfqpoint{1.240734in}{0.448173in}}%
\pgfpathlineto{\pgfqpoint{1.240734in}{0.472593in}}%
\pgfpathlineto{\pgfqpoint{1.280092in}{0.472593in}}%
\pgfpathlineto{\pgfqpoint{1.280092in}{0.489494in}}%
\pgfpathlineto{\pgfqpoint{1.319450in}{0.489494in}}%
\pgfpathlineto{\pgfqpoint{1.319450in}{0.497696in}}%
\pgfpathlineto{\pgfqpoint{1.358807in}{0.497696in}}%
\pgfpathlineto{\pgfqpoint{1.358807in}{0.505563in}}%
\pgfpathlineto{\pgfqpoint{1.398165in}{0.505563in}}%
\pgfpathlineto{\pgfqpoint{1.398165in}{0.549515in}}%
\pgfpathlineto{\pgfqpoint{1.437523in}{0.549515in}}%
\pgfpathlineto{\pgfqpoint{1.437523in}{0.537581in}}%
\pgfpathlineto{\pgfqpoint{1.476881in}{0.537581in}}%
\pgfpathlineto{\pgfqpoint{1.476881in}{0.595816in}}%
\pgfpathlineto{\pgfqpoint{1.516238in}{0.595816in}}%
\pgfpathlineto{\pgfqpoint{1.516238in}{0.643903in}}%
\pgfpathlineto{\pgfqpoint{1.555596in}{0.643903in}}%
\pgfpathlineto{\pgfqpoint{1.555596in}{0.734089in}}%
\pgfpathlineto{\pgfqpoint{1.594954in}{0.734089in}}%
\pgfpathlineto{\pgfqpoint{1.594954in}{0.631606in}}%
\pgfpathlineto{\pgfqpoint{1.634312in}{0.631606in}}%
\pgfpathlineto{\pgfqpoint{1.634312in}{0.686982in}}%
\pgfpathlineto{\pgfqpoint{1.673669in}{0.686982in}}%
\pgfpathlineto{\pgfqpoint{1.673669in}{0.712542in}}%
\pgfpathlineto{\pgfqpoint{1.713027in}{0.712542in}}%
\pgfpathlineto{\pgfqpoint{1.713027in}{0.920367in}}%
\pgfpathlineto{\pgfqpoint{1.752385in}{0.920367in}}%
\pgfpathlineto{\pgfqpoint{1.752385in}{0.741123in}}%
\pgfpathlineto{\pgfqpoint{1.791743in}{0.741123in}}%
\pgfpathlineto{\pgfqpoint{1.791743in}{0.723739in}}%
\pgfpathlineto{\pgfqpoint{1.831100in}{0.723739in}}%
\pgfpathlineto{\pgfqpoint{1.831100in}{0.574122in}}%
\pgfpathlineto{\pgfqpoint{1.870458in}{0.574122in}}%
\pgfpathlineto{\pgfqpoint{1.870458in}{0.534211in}}%
\pgfpathlineto{\pgfqpoint{1.909816in}{0.534211in}}%
\pgfpathlineto{\pgfqpoint{1.909816in}{0.658630in}}%
\pgfpathlineto{\pgfqpoint{1.949174in}{0.658630in}}%
\pgfpathlineto{\pgfqpoint{1.949174in}{0.496582in}}%
\pgfpathlineto{\pgfqpoint{1.988531in}{0.496582in}}%
\pgfpathlineto{\pgfqpoint{1.988531in}{0.521887in}}%
\pgfpathlineto{\pgfqpoint{2.027889in}{0.521887in}}%
\pgfpathlineto{\pgfqpoint{2.027889in}{0.497482in}}%
\pgfpathlineto{\pgfqpoint{2.067247in}{0.497482in}}%
\pgfpathlineto{\pgfqpoint{2.067247in}{0.460229in}}%
\pgfpathlineto{\pgfqpoint{2.106605in}{0.460229in}}%
\pgfpathlineto{\pgfqpoint{2.106605in}{0.438145in}}%
\pgfpathlineto{\pgfqpoint{2.145962in}{0.438145in}}%
\pgfpathlineto{\pgfqpoint{2.145962in}{0.445448in}}%
\pgfpathlineto{\pgfqpoint{2.185320in}{0.445448in}}%
\pgfpathlineto{\pgfqpoint{2.185320in}{0.428923in}}%
\pgfpathlineto{\pgfqpoint{2.224678in}{0.428923in}}%
\pgfpathlineto{\pgfqpoint{2.224678in}{0.421042in}}%
\pgfpathlineto{\pgfqpoint{2.264036in}{0.421042in}}%
\pgfpathlineto{\pgfqpoint{2.264036in}{0.417391in}}%
\pgfusepath{stroke}%
\end{pgfscope}%
\begin{pgfscope}%
\pgfpathrectangle{\pgfqpoint{0.296148in}{0.417391in}}{\pgfqpoint{2.245646in}{1.266124in}} %
\pgfusepath{clip}%
\pgfsetbuttcap%
\pgfsetmiterjoin%
\pgfsetlinewidth{1.003750pt}%
\definecolor{currentstroke}{rgb}{1.000000,0.647059,0.000000}%
\pgfsetstrokecolor{currentstroke}%
\pgfsetdash{}{0pt}%
\pgfpathmoveto{\pgfqpoint{0.296148in}{0.417391in}}%
\pgfpathlineto{\pgfqpoint{0.296148in}{1.504845in}}%
\pgfpathlineto{\pgfqpoint{0.335506in}{1.504845in}}%
\pgfpathlineto{\pgfqpoint{0.335506in}{0.417391in}}%
\pgfpathlineto{\pgfqpoint{0.374864in}{0.417391in}}%
\pgfpathlineto{\pgfqpoint{0.374864in}{0.417391in}}%
\pgfpathlineto{\pgfqpoint{0.414221in}{0.417391in}}%
\pgfpathlineto{\pgfqpoint{0.414221in}{0.417391in}}%
\pgfpathlineto{\pgfqpoint{0.453579in}{0.417391in}}%
\pgfpathlineto{\pgfqpoint{0.453579in}{0.417391in}}%
\pgfpathlineto{\pgfqpoint{0.492937in}{0.417391in}}%
\pgfpathlineto{\pgfqpoint{0.492937in}{0.417391in}}%
\pgfpathlineto{\pgfqpoint{0.532295in}{0.417391in}}%
\pgfpathlineto{\pgfqpoint{0.532295in}{0.417391in}}%
\pgfpathlineto{\pgfqpoint{0.571652in}{0.417391in}}%
\pgfpathlineto{\pgfqpoint{0.571652in}{0.417391in}}%
\pgfpathlineto{\pgfqpoint{0.611010in}{0.417391in}}%
\pgfpathlineto{\pgfqpoint{0.611010in}{0.417391in}}%
\pgfpathlineto{\pgfqpoint{0.650368in}{0.417391in}}%
\pgfpathlineto{\pgfqpoint{0.650368in}{0.417391in}}%
\pgfpathlineto{\pgfqpoint{0.689726in}{0.417391in}}%
\pgfpathlineto{\pgfqpoint{0.689726in}{0.417391in}}%
\pgfpathlineto{\pgfqpoint{0.729083in}{0.417391in}}%
\pgfpathlineto{\pgfqpoint{0.729083in}{0.417391in}}%
\pgfpathlineto{\pgfqpoint{0.768441in}{0.417391in}}%
\pgfpathlineto{\pgfqpoint{0.768441in}{0.417391in}}%
\pgfpathlineto{\pgfqpoint{0.807799in}{0.417391in}}%
\pgfpathlineto{\pgfqpoint{0.807799in}{0.417391in}}%
\pgfpathlineto{\pgfqpoint{0.847157in}{0.417391in}}%
\pgfpathlineto{\pgfqpoint{0.847157in}{0.417391in}}%
\pgfpathlineto{\pgfqpoint{0.886514in}{0.417391in}}%
\pgfpathlineto{\pgfqpoint{0.886514in}{0.417391in}}%
\pgfpathlineto{\pgfqpoint{0.925872in}{0.417391in}}%
\pgfpathlineto{\pgfqpoint{0.925872in}{0.417391in}}%
\pgfpathlineto{\pgfqpoint{0.965230in}{0.417391in}}%
\pgfpathlineto{\pgfqpoint{0.965230in}{0.417391in}}%
\pgfpathlineto{\pgfqpoint{1.004588in}{0.417391in}}%
\pgfpathlineto{\pgfqpoint{1.004588in}{0.417391in}}%
\pgfpathlineto{\pgfqpoint{1.043945in}{0.417391in}}%
\pgfpathlineto{\pgfqpoint{1.043945in}{0.418325in}}%
\pgfpathlineto{\pgfqpoint{1.083303in}{0.418325in}}%
\pgfpathlineto{\pgfqpoint{1.083303in}{0.421474in}}%
\pgfpathlineto{\pgfqpoint{1.122661in}{0.421474in}}%
\pgfpathlineto{\pgfqpoint{1.122661in}{0.421808in}}%
\pgfpathlineto{\pgfqpoint{1.162019in}{0.421808in}}%
\pgfpathlineto{\pgfqpoint{1.162019in}{0.442064in}}%
\pgfpathlineto{\pgfqpoint{1.201376in}{0.442064in}}%
\pgfpathlineto{\pgfqpoint{1.201376in}{0.442327in}}%
\pgfpathlineto{\pgfqpoint{1.240734in}{0.442327in}}%
\pgfpathlineto{\pgfqpoint{1.240734in}{0.462315in}}%
\pgfpathlineto{\pgfqpoint{1.280092in}{0.462315in}}%
\pgfpathlineto{\pgfqpoint{1.280092in}{0.476463in}}%
\pgfpathlineto{\pgfqpoint{1.319450in}{0.476463in}}%
\pgfpathlineto{\pgfqpoint{1.319450in}{0.484519in}}%
\pgfpathlineto{\pgfqpoint{1.358807in}{0.484519in}}%
\pgfpathlineto{\pgfqpoint{1.358807in}{0.490618in}}%
\pgfpathlineto{\pgfqpoint{1.398165in}{0.490618in}}%
\pgfpathlineto{\pgfqpoint{1.398165in}{0.529066in}}%
\pgfpathlineto{\pgfqpoint{1.437523in}{0.529066in}}%
\pgfpathlineto{\pgfqpoint{1.437523in}{0.519866in}}%
\pgfpathlineto{\pgfqpoint{1.476881in}{0.519866in}}%
\pgfpathlineto{\pgfqpoint{1.476881in}{0.570247in}}%
\pgfpathlineto{\pgfqpoint{1.516238in}{0.570247in}}%
\pgfpathlineto{\pgfqpoint{1.516238in}{0.621601in}}%
\pgfpathlineto{\pgfqpoint{1.555596in}{0.621601in}}%
\pgfpathlineto{\pgfqpoint{1.555596in}{0.713014in}}%
\pgfpathlineto{\pgfqpoint{1.594954in}{0.713014in}}%
\pgfpathlineto{\pgfqpoint{1.594954in}{0.626628in}}%
\pgfpathlineto{\pgfqpoint{1.634312in}{0.626628in}}%
\pgfpathlineto{\pgfqpoint{1.634312in}{0.689414in}}%
\pgfpathlineto{\pgfqpoint{1.673669in}{0.689414in}}%
\pgfpathlineto{\pgfqpoint{1.673669in}{0.720530in}}%
\pgfpathlineto{\pgfqpoint{1.713027in}{0.720530in}}%
\pgfpathlineto{\pgfqpoint{1.713027in}{0.959000in}}%
\pgfpathlineto{\pgfqpoint{1.752385in}{0.959000in}}%
\pgfpathlineto{\pgfqpoint{1.752385in}{0.776441in}}%
\pgfpathlineto{\pgfqpoint{1.791743in}{0.776441in}}%
\pgfpathlineto{\pgfqpoint{1.791743in}{0.765390in}}%
\pgfpathlineto{\pgfqpoint{1.831100in}{0.765390in}}%
\pgfpathlineto{\pgfqpoint{1.831100in}{0.599027in}}%
\pgfpathlineto{\pgfqpoint{1.870458in}{0.599027in}}%
\pgfpathlineto{\pgfqpoint{1.870458in}{0.555060in}}%
\pgfpathlineto{\pgfqpoint{1.909816in}{0.555060in}}%
\pgfpathlineto{\pgfqpoint{1.909816in}{0.709111in}}%
\pgfpathlineto{\pgfqpoint{1.949174in}{0.709111in}}%
\pgfpathlineto{\pgfqpoint{1.949174in}{0.513246in}}%
\pgfpathlineto{\pgfqpoint{1.988531in}{0.513246in}}%
\pgfpathlineto{\pgfqpoint{1.988531in}{0.543518in}}%
\pgfpathlineto{\pgfqpoint{2.027889in}{0.543518in}}%
\pgfpathlineto{\pgfqpoint{2.027889in}{0.514432in}}%
\pgfpathlineto{\pgfqpoint{2.067247in}{0.514432in}}%
\pgfpathlineto{\pgfqpoint{2.067247in}{0.470180in}}%
\pgfpathlineto{\pgfqpoint{2.106605in}{0.470180in}}%
\pgfpathlineto{\pgfqpoint{2.106605in}{0.443125in}}%
\pgfpathlineto{\pgfqpoint{2.145962in}{0.443125in}}%
\pgfpathlineto{\pgfqpoint{2.145962in}{0.451061in}}%
\pgfpathlineto{\pgfqpoint{2.185320in}{0.451061in}}%
\pgfpathlineto{\pgfqpoint{2.185320in}{0.431438in}}%
\pgfpathlineto{\pgfqpoint{2.224678in}{0.431438in}}%
\pgfpathlineto{\pgfqpoint{2.224678in}{0.421587in}}%
\pgfpathlineto{\pgfqpoint{2.264036in}{0.421587in}}%
\pgfpathlineto{\pgfqpoint{2.264036in}{0.417391in}}%
\pgfusepath{stroke}%
\end{pgfscope}%
\begin{pgfscope}%
\pgfsetrectcap%
\pgfsetmiterjoin%
\pgfsetlinewidth{1.003750pt}%
\definecolor{currentstroke}{rgb}{0.000000,0.000000,0.000000}%
\pgfsetstrokecolor{currentstroke}%
\pgfsetdash{}{0pt}%
\pgfpathmoveto{\pgfqpoint{0.296148in}{1.683515in}}%
\pgfpathlineto{\pgfqpoint{2.541794in}{1.683515in}}%
\pgfusepath{stroke}%
\end{pgfscope}%
\begin{pgfscope}%
\pgfsetrectcap%
\pgfsetmiterjoin%
\pgfsetlinewidth{1.003750pt}%
\definecolor{currentstroke}{rgb}{0.000000,0.000000,0.000000}%
\pgfsetstrokecolor{currentstroke}%
\pgfsetdash{}{0pt}%
\pgfpathmoveto{\pgfqpoint{2.541794in}{0.417391in}}%
\pgfpathlineto{\pgfqpoint{2.541794in}{1.683515in}}%
\pgfusepath{stroke}%
\end{pgfscope}%
\begin{pgfscope}%
\pgfsetrectcap%
\pgfsetmiterjoin%
\pgfsetlinewidth{1.003750pt}%
\definecolor{currentstroke}{rgb}{0.000000,0.000000,0.000000}%
\pgfsetstrokecolor{currentstroke}%
\pgfsetdash{}{0pt}%
\pgfpathmoveto{\pgfqpoint{0.296148in}{0.417391in}}%
\pgfpathlineto{\pgfqpoint{2.541794in}{0.417391in}}%
\pgfusepath{stroke}%
\end{pgfscope}%
\begin{pgfscope}%
\pgfsetrectcap%
\pgfsetmiterjoin%
\pgfsetlinewidth{1.003750pt}%
\definecolor{currentstroke}{rgb}{0.000000,0.000000,0.000000}%
\pgfsetstrokecolor{currentstroke}%
\pgfsetdash{}{0pt}%
\pgfpathmoveto{\pgfqpoint{0.296148in}{0.417391in}}%
\pgfpathlineto{\pgfqpoint{0.296148in}{1.683515in}}%
\pgfusepath{stroke}%
\end{pgfscope}%
\begin{pgfscope}%
\pgfsetbuttcap%
\pgfsetroundjoin%
\definecolor{currentfill}{rgb}{0.000000,0.000000,0.000000}%
\pgfsetfillcolor{currentfill}%
\pgfsetlinewidth{0.501875pt}%
\definecolor{currentstroke}{rgb}{0.000000,0.000000,0.000000}%
\pgfsetstrokecolor{currentstroke}%
\pgfsetdash{}{0pt}%
\pgfsys@defobject{currentmarker}{\pgfqpoint{0.000000in}{0.000000in}}{\pgfqpoint{0.000000in}{0.069444in}}{%
\pgfpathmoveto{\pgfqpoint{0.000000in}{0.000000in}}%
\pgfpathlineto{\pgfqpoint{0.000000in}{0.069444in}}%
\pgfusepath{stroke,fill}%
}%
\begin{pgfscope}%
\pgfsys@transformshift{0.296148in}{0.417391in}%
\pgfsys@useobject{currentmarker}{}%
\end{pgfscope}%
\end{pgfscope}%
\begin{pgfscope}%
\pgfsetbuttcap%
\pgfsetroundjoin%
\definecolor{currentfill}{rgb}{0.000000,0.000000,0.000000}%
\pgfsetfillcolor{currentfill}%
\pgfsetlinewidth{0.501875pt}%
\definecolor{currentstroke}{rgb}{0.000000,0.000000,0.000000}%
\pgfsetstrokecolor{currentstroke}%
\pgfsetdash{}{0pt}%
\pgfsys@defobject{currentmarker}{\pgfqpoint{0.000000in}{-0.069444in}}{\pgfqpoint{0.000000in}{0.000000in}}{%
\pgfpathmoveto{\pgfqpoint{0.000000in}{0.000000in}}%
\pgfpathlineto{\pgfqpoint{0.000000in}{-0.069444in}}%
\pgfusepath{stroke,fill}%
}%
\begin{pgfscope}%
\pgfsys@transformshift{0.296148in}{1.683515in}%
\pgfsys@useobject{currentmarker}{}%
\end{pgfscope}%
\end{pgfscope}%
\begin{pgfscope}%
\pgftext[x=0.296148in,y=0.347947in,,top]{\rmfamily\fontsize{8.000000}{9.600000}\selectfont −2.0}%
\end{pgfscope}%
\begin{pgfscope}%
\pgfsetbuttcap%
\pgfsetroundjoin%
\definecolor{currentfill}{rgb}{0.000000,0.000000,0.000000}%
\pgfsetfillcolor{currentfill}%
\pgfsetlinewidth{0.501875pt}%
\definecolor{currentstroke}{rgb}{0.000000,0.000000,0.000000}%
\pgfsetstrokecolor{currentstroke}%
\pgfsetdash{}{0pt}%
\pgfsys@defobject{currentmarker}{\pgfqpoint{0.000000in}{0.000000in}}{\pgfqpoint{0.000000in}{0.069444in}}{%
\pgfpathmoveto{\pgfqpoint{0.000000in}{0.000000in}}%
\pgfpathlineto{\pgfqpoint{0.000000in}{0.069444in}}%
\pgfusepath{stroke,fill}%
}%
\begin{pgfscope}%
\pgfsys@transformshift{0.670422in}{0.417391in}%
\pgfsys@useobject{currentmarker}{}%
\end{pgfscope}%
\end{pgfscope}%
\begin{pgfscope}%
\pgfsetbuttcap%
\pgfsetroundjoin%
\definecolor{currentfill}{rgb}{0.000000,0.000000,0.000000}%
\pgfsetfillcolor{currentfill}%
\pgfsetlinewidth{0.501875pt}%
\definecolor{currentstroke}{rgb}{0.000000,0.000000,0.000000}%
\pgfsetstrokecolor{currentstroke}%
\pgfsetdash{}{0pt}%
\pgfsys@defobject{currentmarker}{\pgfqpoint{0.000000in}{-0.069444in}}{\pgfqpoint{0.000000in}{0.000000in}}{%
\pgfpathmoveto{\pgfqpoint{0.000000in}{0.000000in}}%
\pgfpathlineto{\pgfqpoint{0.000000in}{-0.069444in}}%
\pgfusepath{stroke,fill}%
}%
\begin{pgfscope}%
\pgfsys@transformshift{0.670422in}{1.683515in}%
\pgfsys@useobject{currentmarker}{}%
\end{pgfscope}%
\end{pgfscope}%
\begin{pgfscope}%
\pgftext[x=0.670422in,y=0.347947in,,top]{\rmfamily\fontsize{8.000000}{9.600000}\selectfont −1.5}%
\end{pgfscope}%
\begin{pgfscope}%
\pgfsetbuttcap%
\pgfsetroundjoin%
\definecolor{currentfill}{rgb}{0.000000,0.000000,0.000000}%
\pgfsetfillcolor{currentfill}%
\pgfsetlinewidth{0.501875pt}%
\definecolor{currentstroke}{rgb}{0.000000,0.000000,0.000000}%
\pgfsetstrokecolor{currentstroke}%
\pgfsetdash{}{0pt}%
\pgfsys@defobject{currentmarker}{\pgfqpoint{0.000000in}{0.000000in}}{\pgfqpoint{0.000000in}{0.069444in}}{%
\pgfpathmoveto{\pgfqpoint{0.000000in}{0.000000in}}%
\pgfpathlineto{\pgfqpoint{0.000000in}{0.069444in}}%
\pgfusepath{stroke,fill}%
}%
\begin{pgfscope}%
\pgfsys@transformshift{1.044697in}{0.417391in}%
\pgfsys@useobject{currentmarker}{}%
\end{pgfscope}%
\end{pgfscope}%
\begin{pgfscope}%
\pgfsetbuttcap%
\pgfsetroundjoin%
\definecolor{currentfill}{rgb}{0.000000,0.000000,0.000000}%
\pgfsetfillcolor{currentfill}%
\pgfsetlinewidth{0.501875pt}%
\definecolor{currentstroke}{rgb}{0.000000,0.000000,0.000000}%
\pgfsetstrokecolor{currentstroke}%
\pgfsetdash{}{0pt}%
\pgfsys@defobject{currentmarker}{\pgfqpoint{0.000000in}{-0.069444in}}{\pgfqpoint{0.000000in}{0.000000in}}{%
\pgfpathmoveto{\pgfqpoint{0.000000in}{0.000000in}}%
\pgfpathlineto{\pgfqpoint{0.000000in}{-0.069444in}}%
\pgfusepath{stroke,fill}%
}%
\begin{pgfscope}%
\pgfsys@transformshift{1.044697in}{1.683515in}%
\pgfsys@useobject{currentmarker}{}%
\end{pgfscope}%
\end{pgfscope}%
\begin{pgfscope}%
\pgftext[x=1.044697in,y=0.347947in,,top]{\rmfamily\fontsize{8.000000}{9.600000}\selectfont −1.0}%
\end{pgfscope}%
\begin{pgfscope}%
\pgfsetbuttcap%
\pgfsetroundjoin%
\definecolor{currentfill}{rgb}{0.000000,0.000000,0.000000}%
\pgfsetfillcolor{currentfill}%
\pgfsetlinewidth{0.501875pt}%
\definecolor{currentstroke}{rgb}{0.000000,0.000000,0.000000}%
\pgfsetstrokecolor{currentstroke}%
\pgfsetdash{}{0pt}%
\pgfsys@defobject{currentmarker}{\pgfqpoint{0.000000in}{0.000000in}}{\pgfqpoint{0.000000in}{0.069444in}}{%
\pgfpathmoveto{\pgfqpoint{0.000000in}{0.000000in}}%
\pgfpathlineto{\pgfqpoint{0.000000in}{0.069444in}}%
\pgfusepath{stroke,fill}%
}%
\begin{pgfscope}%
\pgfsys@transformshift{1.418971in}{0.417391in}%
\pgfsys@useobject{currentmarker}{}%
\end{pgfscope}%
\end{pgfscope}%
\begin{pgfscope}%
\pgfsetbuttcap%
\pgfsetroundjoin%
\definecolor{currentfill}{rgb}{0.000000,0.000000,0.000000}%
\pgfsetfillcolor{currentfill}%
\pgfsetlinewidth{0.501875pt}%
\definecolor{currentstroke}{rgb}{0.000000,0.000000,0.000000}%
\pgfsetstrokecolor{currentstroke}%
\pgfsetdash{}{0pt}%
\pgfsys@defobject{currentmarker}{\pgfqpoint{0.000000in}{-0.069444in}}{\pgfqpoint{0.000000in}{0.000000in}}{%
\pgfpathmoveto{\pgfqpoint{0.000000in}{0.000000in}}%
\pgfpathlineto{\pgfqpoint{0.000000in}{-0.069444in}}%
\pgfusepath{stroke,fill}%
}%
\begin{pgfscope}%
\pgfsys@transformshift{1.418971in}{1.683515in}%
\pgfsys@useobject{currentmarker}{}%
\end{pgfscope}%
\end{pgfscope}%
\begin{pgfscope}%
\pgftext[x=1.418971in,y=0.347947in,,top]{\rmfamily\fontsize{8.000000}{9.600000}\selectfont −0.5}%
\end{pgfscope}%
\begin{pgfscope}%
\pgfsetbuttcap%
\pgfsetroundjoin%
\definecolor{currentfill}{rgb}{0.000000,0.000000,0.000000}%
\pgfsetfillcolor{currentfill}%
\pgfsetlinewidth{0.501875pt}%
\definecolor{currentstroke}{rgb}{0.000000,0.000000,0.000000}%
\pgfsetstrokecolor{currentstroke}%
\pgfsetdash{}{0pt}%
\pgfsys@defobject{currentmarker}{\pgfqpoint{0.000000in}{0.000000in}}{\pgfqpoint{0.000000in}{0.069444in}}{%
\pgfpathmoveto{\pgfqpoint{0.000000in}{0.000000in}}%
\pgfpathlineto{\pgfqpoint{0.000000in}{0.069444in}}%
\pgfusepath{stroke,fill}%
}%
\begin{pgfscope}%
\pgfsys@transformshift{1.793245in}{0.417391in}%
\pgfsys@useobject{currentmarker}{}%
\end{pgfscope}%
\end{pgfscope}%
\begin{pgfscope}%
\pgfsetbuttcap%
\pgfsetroundjoin%
\definecolor{currentfill}{rgb}{0.000000,0.000000,0.000000}%
\pgfsetfillcolor{currentfill}%
\pgfsetlinewidth{0.501875pt}%
\definecolor{currentstroke}{rgb}{0.000000,0.000000,0.000000}%
\pgfsetstrokecolor{currentstroke}%
\pgfsetdash{}{0pt}%
\pgfsys@defobject{currentmarker}{\pgfqpoint{0.000000in}{-0.069444in}}{\pgfqpoint{0.000000in}{0.000000in}}{%
\pgfpathmoveto{\pgfqpoint{0.000000in}{0.000000in}}%
\pgfpathlineto{\pgfqpoint{0.000000in}{-0.069444in}}%
\pgfusepath{stroke,fill}%
}%
\begin{pgfscope}%
\pgfsys@transformshift{1.793245in}{1.683515in}%
\pgfsys@useobject{currentmarker}{}%
\end{pgfscope}%
\end{pgfscope}%
\begin{pgfscope}%
\pgftext[x=1.793245in,y=0.347947in,,top]{\rmfamily\fontsize{8.000000}{9.600000}\selectfont 0.0}%
\end{pgfscope}%
\begin{pgfscope}%
\pgfsetbuttcap%
\pgfsetroundjoin%
\definecolor{currentfill}{rgb}{0.000000,0.000000,0.000000}%
\pgfsetfillcolor{currentfill}%
\pgfsetlinewidth{0.501875pt}%
\definecolor{currentstroke}{rgb}{0.000000,0.000000,0.000000}%
\pgfsetstrokecolor{currentstroke}%
\pgfsetdash{}{0pt}%
\pgfsys@defobject{currentmarker}{\pgfqpoint{0.000000in}{0.000000in}}{\pgfqpoint{0.000000in}{0.069444in}}{%
\pgfpathmoveto{\pgfqpoint{0.000000in}{0.000000in}}%
\pgfpathlineto{\pgfqpoint{0.000000in}{0.069444in}}%
\pgfusepath{stroke,fill}%
}%
\begin{pgfscope}%
\pgfsys@transformshift{2.167520in}{0.417391in}%
\pgfsys@useobject{currentmarker}{}%
\end{pgfscope}%
\end{pgfscope}%
\begin{pgfscope}%
\pgfsetbuttcap%
\pgfsetroundjoin%
\definecolor{currentfill}{rgb}{0.000000,0.000000,0.000000}%
\pgfsetfillcolor{currentfill}%
\pgfsetlinewidth{0.501875pt}%
\definecolor{currentstroke}{rgb}{0.000000,0.000000,0.000000}%
\pgfsetstrokecolor{currentstroke}%
\pgfsetdash{}{0pt}%
\pgfsys@defobject{currentmarker}{\pgfqpoint{0.000000in}{-0.069444in}}{\pgfqpoint{0.000000in}{0.000000in}}{%
\pgfpathmoveto{\pgfqpoint{0.000000in}{0.000000in}}%
\pgfpathlineto{\pgfqpoint{0.000000in}{-0.069444in}}%
\pgfusepath{stroke,fill}%
}%
\begin{pgfscope}%
\pgfsys@transformshift{2.167520in}{1.683515in}%
\pgfsys@useobject{currentmarker}{}%
\end{pgfscope}%
\end{pgfscope}%
\begin{pgfscope}%
\pgftext[x=2.167520in,y=0.347947in,,top]{\rmfamily\fontsize{8.000000}{9.600000}\selectfont 0.5}%
\end{pgfscope}%
\begin{pgfscope}%
\pgfsetbuttcap%
\pgfsetroundjoin%
\definecolor{currentfill}{rgb}{0.000000,0.000000,0.000000}%
\pgfsetfillcolor{currentfill}%
\pgfsetlinewidth{0.501875pt}%
\definecolor{currentstroke}{rgb}{0.000000,0.000000,0.000000}%
\pgfsetstrokecolor{currentstroke}%
\pgfsetdash{}{0pt}%
\pgfsys@defobject{currentmarker}{\pgfqpoint{0.000000in}{0.000000in}}{\pgfqpoint{0.000000in}{0.069444in}}{%
\pgfpathmoveto{\pgfqpoint{0.000000in}{0.000000in}}%
\pgfpathlineto{\pgfqpoint{0.000000in}{0.069444in}}%
\pgfusepath{stroke,fill}%
}%
\begin{pgfscope}%
\pgfsys@transformshift{2.541794in}{0.417391in}%
\pgfsys@useobject{currentmarker}{}%
\end{pgfscope}%
\end{pgfscope}%
\begin{pgfscope}%
\pgfsetbuttcap%
\pgfsetroundjoin%
\definecolor{currentfill}{rgb}{0.000000,0.000000,0.000000}%
\pgfsetfillcolor{currentfill}%
\pgfsetlinewidth{0.501875pt}%
\definecolor{currentstroke}{rgb}{0.000000,0.000000,0.000000}%
\pgfsetstrokecolor{currentstroke}%
\pgfsetdash{}{0pt}%
\pgfsys@defobject{currentmarker}{\pgfqpoint{0.000000in}{-0.069444in}}{\pgfqpoint{0.000000in}{0.000000in}}{%
\pgfpathmoveto{\pgfqpoint{0.000000in}{0.000000in}}%
\pgfpathlineto{\pgfqpoint{0.000000in}{-0.069444in}}%
\pgfusepath{stroke,fill}%
}%
\begin{pgfscope}%
\pgfsys@transformshift{2.541794in}{1.683515in}%
\pgfsys@useobject{currentmarker}{}%
\end{pgfscope}%
\end{pgfscope}%
\begin{pgfscope}%
\pgftext[x=2.541794in,y=0.347947in,,top]{\rmfamily\fontsize{8.000000}{9.600000}\selectfont 1.0}%
\end{pgfscope}%
\begin{pgfscope}%
\pgftext[x=1.418971in,y=0.170972in,,top]{\rmfamily\fontsize{9.000000}{10.800000}\selectfont muon isolation BDT response}%
\end{pgfscope}%
\begin{pgfscope}%
\pgfsetbuttcap%
\pgfsetroundjoin%
\definecolor{currentfill}{rgb}{0.000000,0.000000,0.000000}%
\pgfsetfillcolor{currentfill}%
\pgfsetlinewidth{0.501875pt}%
\definecolor{currentstroke}{rgb}{0.000000,0.000000,0.000000}%
\pgfsetstrokecolor{currentstroke}%
\pgfsetdash{}{0pt}%
\pgfsys@defobject{currentmarker}{\pgfqpoint{0.000000in}{0.000000in}}{\pgfqpoint{0.069444in}{0.000000in}}{%
\pgfpathmoveto{\pgfqpoint{0.000000in}{0.000000in}}%
\pgfpathlineto{\pgfqpoint{0.069444in}{0.000000in}}%
\pgfusepath{stroke,fill}%
}%
\begin{pgfscope}%
\pgfsys@transformshift{0.296148in}{0.417391in}%
\pgfsys@useobject{currentmarker}{}%
\end{pgfscope}%
\end{pgfscope}%
\begin{pgfscope}%
\pgfsetbuttcap%
\pgfsetroundjoin%
\definecolor{currentfill}{rgb}{0.000000,0.000000,0.000000}%
\pgfsetfillcolor{currentfill}%
\pgfsetlinewidth{0.501875pt}%
\definecolor{currentstroke}{rgb}{0.000000,0.000000,0.000000}%
\pgfsetstrokecolor{currentstroke}%
\pgfsetdash{}{0pt}%
\pgfsys@defobject{currentmarker}{\pgfqpoint{-0.069444in}{0.000000in}}{\pgfqpoint{0.000000in}{0.000000in}}{%
\pgfpathmoveto{\pgfqpoint{0.000000in}{0.000000in}}%
\pgfpathlineto{\pgfqpoint{-0.069444in}{0.000000in}}%
\pgfusepath{stroke,fill}%
}%
\begin{pgfscope}%
\pgfsys@transformshift{2.541794in}{0.417391in}%
\pgfsys@useobject{currentmarker}{}%
\end{pgfscope}%
\end{pgfscope}%
\begin{pgfscope}%
\pgftext[x=0.226704in,y=0.417391in,right,]{\rmfamily\fontsize{8.000000}{9.600000}\selectfont 0.0}%
\end{pgfscope}%
\begin{pgfscope}%
\pgfsetbuttcap%
\pgfsetroundjoin%
\definecolor{currentfill}{rgb}{0.000000,0.000000,0.000000}%
\pgfsetfillcolor{currentfill}%
\pgfsetlinewidth{0.501875pt}%
\definecolor{currentstroke}{rgb}{0.000000,0.000000,0.000000}%
\pgfsetstrokecolor{currentstroke}%
\pgfsetdash{}{0pt}%
\pgfsys@defobject{currentmarker}{\pgfqpoint{0.000000in}{0.000000in}}{\pgfqpoint{0.069444in}{0.000000in}}{%
\pgfpathmoveto{\pgfqpoint{0.000000in}{0.000000in}}%
\pgfpathlineto{\pgfqpoint{0.069444in}{0.000000in}}%
\pgfusepath{stroke,fill}%
}%
\begin{pgfscope}%
\pgfsys@transformshift{0.296148in}{0.558071in}%
\pgfsys@useobject{currentmarker}{}%
\end{pgfscope}%
\end{pgfscope}%
\begin{pgfscope}%
\pgfsetbuttcap%
\pgfsetroundjoin%
\definecolor{currentfill}{rgb}{0.000000,0.000000,0.000000}%
\pgfsetfillcolor{currentfill}%
\pgfsetlinewidth{0.501875pt}%
\definecolor{currentstroke}{rgb}{0.000000,0.000000,0.000000}%
\pgfsetstrokecolor{currentstroke}%
\pgfsetdash{}{0pt}%
\pgfsys@defobject{currentmarker}{\pgfqpoint{-0.069444in}{0.000000in}}{\pgfqpoint{0.000000in}{0.000000in}}{%
\pgfpathmoveto{\pgfqpoint{0.000000in}{0.000000in}}%
\pgfpathlineto{\pgfqpoint{-0.069444in}{0.000000in}}%
\pgfusepath{stroke,fill}%
}%
\begin{pgfscope}%
\pgfsys@transformshift{2.541794in}{0.558071in}%
\pgfsys@useobject{currentmarker}{}%
\end{pgfscope}%
\end{pgfscope}%
\begin{pgfscope}%
\pgftext[x=0.226704in,y=0.558071in,right,]{\rmfamily\fontsize{8.000000}{9.600000}\selectfont 0.5}%
\end{pgfscope}%
\begin{pgfscope}%
\pgfsetbuttcap%
\pgfsetroundjoin%
\definecolor{currentfill}{rgb}{0.000000,0.000000,0.000000}%
\pgfsetfillcolor{currentfill}%
\pgfsetlinewidth{0.501875pt}%
\definecolor{currentstroke}{rgb}{0.000000,0.000000,0.000000}%
\pgfsetstrokecolor{currentstroke}%
\pgfsetdash{}{0pt}%
\pgfsys@defobject{currentmarker}{\pgfqpoint{0.000000in}{0.000000in}}{\pgfqpoint{0.069444in}{0.000000in}}{%
\pgfpathmoveto{\pgfqpoint{0.000000in}{0.000000in}}%
\pgfpathlineto{\pgfqpoint{0.069444in}{0.000000in}}%
\pgfusepath{stroke,fill}%
}%
\begin{pgfscope}%
\pgfsys@transformshift{0.296148in}{0.698752in}%
\pgfsys@useobject{currentmarker}{}%
\end{pgfscope}%
\end{pgfscope}%
\begin{pgfscope}%
\pgfsetbuttcap%
\pgfsetroundjoin%
\definecolor{currentfill}{rgb}{0.000000,0.000000,0.000000}%
\pgfsetfillcolor{currentfill}%
\pgfsetlinewidth{0.501875pt}%
\definecolor{currentstroke}{rgb}{0.000000,0.000000,0.000000}%
\pgfsetstrokecolor{currentstroke}%
\pgfsetdash{}{0pt}%
\pgfsys@defobject{currentmarker}{\pgfqpoint{-0.069444in}{0.000000in}}{\pgfqpoint{0.000000in}{0.000000in}}{%
\pgfpathmoveto{\pgfqpoint{0.000000in}{0.000000in}}%
\pgfpathlineto{\pgfqpoint{-0.069444in}{0.000000in}}%
\pgfusepath{stroke,fill}%
}%
\begin{pgfscope}%
\pgfsys@transformshift{2.541794in}{0.698752in}%
\pgfsys@useobject{currentmarker}{}%
\end{pgfscope}%
\end{pgfscope}%
\begin{pgfscope}%
\pgftext[x=0.226704in,y=0.698752in,right,]{\rmfamily\fontsize{8.000000}{9.600000}\selectfont 1.0}%
\end{pgfscope}%
\begin{pgfscope}%
\pgfsetbuttcap%
\pgfsetroundjoin%
\definecolor{currentfill}{rgb}{0.000000,0.000000,0.000000}%
\pgfsetfillcolor{currentfill}%
\pgfsetlinewidth{0.501875pt}%
\definecolor{currentstroke}{rgb}{0.000000,0.000000,0.000000}%
\pgfsetstrokecolor{currentstroke}%
\pgfsetdash{}{0pt}%
\pgfsys@defobject{currentmarker}{\pgfqpoint{0.000000in}{0.000000in}}{\pgfqpoint{0.069444in}{0.000000in}}{%
\pgfpathmoveto{\pgfqpoint{0.000000in}{0.000000in}}%
\pgfpathlineto{\pgfqpoint{0.069444in}{0.000000in}}%
\pgfusepath{stroke,fill}%
}%
\begin{pgfscope}%
\pgfsys@transformshift{0.296148in}{0.839432in}%
\pgfsys@useobject{currentmarker}{}%
\end{pgfscope}%
\end{pgfscope}%
\begin{pgfscope}%
\pgfsetbuttcap%
\pgfsetroundjoin%
\definecolor{currentfill}{rgb}{0.000000,0.000000,0.000000}%
\pgfsetfillcolor{currentfill}%
\pgfsetlinewidth{0.501875pt}%
\definecolor{currentstroke}{rgb}{0.000000,0.000000,0.000000}%
\pgfsetstrokecolor{currentstroke}%
\pgfsetdash{}{0pt}%
\pgfsys@defobject{currentmarker}{\pgfqpoint{-0.069444in}{0.000000in}}{\pgfqpoint{0.000000in}{0.000000in}}{%
\pgfpathmoveto{\pgfqpoint{0.000000in}{0.000000in}}%
\pgfpathlineto{\pgfqpoint{-0.069444in}{0.000000in}}%
\pgfusepath{stroke,fill}%
}%
\begin{pgfscope}%
\pgfsys@transformshift{2.541794in}{0.839432in}%
\pgfsys@useobject{currentmarker}{}%
\end{pgfscope}%
\end{pgfscope}%
\begin{pgfscope}%
\pgftext[x=0.226704in,y=0.839432in,right,]{\rmfamily\fontsize{8.000000}{9.600000}\selectfont 1.5}%
\end{pgfscope}%
\begin{pgfscope}%
\pgfsetbuttcap%
\pgfsetroundjoin%
\definecolor{currentfill}{rgb}{0.000000,0.000000,0.000000}%
\pgfsetfillcolor{currentfill}%
\pgfsetlinewidth{0.501875pt}%
\definecolor{currentstroke}{rgb}{0.000000,0.000000,0.000000}%
\pgfsetstrokecolor{currentstroke}%
\pgfsetdash{}{0pt}%
\pgfsys@defobject{currentmarker}{\pgfqpoint{0.000000in}{0.000000in}}{\pgfqpoint{0.069444in}{0.000000in}}{%
\pgfpathmoveto{\pgfqpoint{0.000000in}{0.000000in}}%
\pgfpathlineto{\pgfqpoint{0.069444in}{0.000000in}}%
\pgfusepath{stroke,fill}%
}%
\begin{pgfscope}%
\pgfsys@transformshift{0.296148in}{0.980113in}%
\pgfsys@useobject{currentmarker}{}%
\end{pgfscope}%
\end{pgfscope}%
\begin{pgfscope}%
\pgfsetbuttcap%
\pgfsetroundjoin%
\definecolor{currentfill}{rgb}{0.000000,0.000000,0.000000}%
\pgfsetfillcolor{currentfill}%
\pgfsetlinewidth{0.501875pt}%
\definecolor{currentstroke}{rgb}{0.000000,0.000000,0.000000}%
\pgfsetstrokecolor{currentstroke}%
\pgfsetdash{}{0pt}%
\pgfsys@defobject{currentmarker}{\pgfqpoint{-0.069444in}{0.000000in}}{\pgfqpoint{0.000000in}{0.000000in}}{%
\pgfpathmoveto{\pgfqpoint{0.000000in}{0.000000in}}%
\pgfpathlineto{\pgfqpoint{-0.069444in}{0.000000in}}%
\pgfusepath{stroke,fill}%
}%
\begin{pgfscope}%
\pgfsys@transformshift{2.541794in}{0.980113in}%
\pgfsys@useobject{currentmarker}{}%
\end{pgfscope}%
\end{pgfscope}%
\begin{pgfscope}%
\pgftext[x=0.226704in,y=0.980113in,right,]{\rmfamily\fontsize{8.000000}{9.600000}\selectfont 2.0}%
\end{pgfscope}%
\begin{pgfscope}%
\pgfsetbuttcap%
\pgfsetroundjoin%
\definecolor{currentfill}{rgb}{0.000000,0.000000,0.000000}%
\pgfsetfillcolor{currentfill}%
\pgfsetlinewidth{0.501875pt}%
\definecolor{currentstroke}{rgb}{0.000000,0.000000,0.000000}%
\pgfsetstrokecolor{currentstroke}%
\pgfsetdash{}{0pt}%
\pgfsys@defobject{currentmarker}{\pgfqpoint{0.000000in}{0.000000in}}{\pgfqpoint{0.069444in}{0.000000in}}{%
\pgfpathmoveto{\pgfqpoint{0.000000in}{0.000000in}}%
\pgfpathlineto{\pgfqpoint{0.069444in}{0.000000in}}%
\pgfusepath{stroke,fill}%
}%
\begin{pgfscope}%
\pgfsys@transformshift{0.296148in}{1.120793in}%
\pgfsys@useobject{currentmarker}{}%
\end{pgfscope}%
\end{pgfscope}%
\begin{pgfscope}%
\pgfsetbuttcap%
\pgfsetroundjoin%
\definecolor{currentfill}{rgb}{0.000000,0.000000,0.000000}%
\pgfsetfillcolor{currentfill}%
\pgfsetlinewidth{0.501875pt}%
\definecolor{currentstroke}{rgb}{0.000000,0.000000,0.000000}%
\pgfsetstrokecolor{currentstroke}%
\pgfsetdash{}{0pt}%
\pgfsys@defobject{currentmarker}{\pgfqpoint{-0.069444in}{0.000000in}}{\pgfqpoint{0.000000in}{0.000000in}}{%
\pgfpathmoveto{\pgfqpoint{0.000000in}{0.000000in}}%
\pgfpathlineto{\pgfqpoint{-0.069444in}{0.000000in}}%
\pgfusepath{stroke,fill}%
}%
\begin{pgfscope}%
\pgfsys@transformshift{2.541794in}{1.120793in}%
\pgfsys@useobject{currentmarker}{}%
\end{pgfscope}%
\end{pgfscope}%
\begin{pgfscope}%
\pgftext[x=0.226704in,y=1.120793in,right,]{\rmfamily\fontsize{8.000000}{9.600000}\selectfont 2.5}%
\end{pgfscope}%
\begin{pgfscope}%
\pgfsetbuttcap%
\pgfsetroundjoin%
\definecolor{currentfill}{rgb}{0.000000,0.000000,0.000000}%
\pgfsetfillcolor{currentfill}%
\pgfsetlinewidth{0.501875pt}%
\definecolor{currentstroke}{rgb}{0.000000,0.000000,0.000000}%
\pgfsetstrokecolor{currentstroke}%
\pgfsetdash{}{0pt}%
\pgfsys@defobject{currentmarker}{\pgfqpoint{0.000000in}{0.000000in}}{\pgfqpoint{0.069444in}{0.000000in}}{%
\pgfpathmoveto{\pgfqpoint{0.000000in}{0.000000in}}%
\pgfpathlineto{\pgfqpoint{0.069444in}{0.000000in}}%
\pgfusepath{stroke,fill}%
}%
\begin{pgfscope}%
\pgfsys@transformshift{0.296148in}{1.261474in}%
\pgfsys@useobject{currentmarker}{}%
\end{pgfscope}%
\end{pgfscope}%
\begin{pgfscope}%
\pgfsetbuttcap%
\pgfsetroundjoin%
\definecolor{currentfill}{rgb}{0.000000,0.000000,0.000000}%
\pgfsetfillcolor{currentfill}%
\pgfsetlinewidth{0.501875pt}%
\definecolor{currentstroke}{rgb}{0.000000,0.000000,0.000000}%
\pgfsetstrokecolor{currentstroke}%
\pgfsetdash{}{0pt}%
\pgfsys@defobject{currentmarker}{\pgfqpoint{-0.069444in}{0.000000in}}{\pgfqpoint{0.000000in}{0.000000in}}{%
\pgfpathmoveto{\pgfqpoint{0.000000in}{0.000000in}}%
\pgfpathlineto{\pgfqpoint{-0.069444in}{0.000000in}}%
\pgfusepath{stroke,fill}%
}%
\begin{pgfscope}%
\pgfsys@transformshift{2.541794in}{1.261474in}%
\pgfsys@useobject{currentmarker}{}%
\end{pgfscope}%
\end{pgfscope}%
\begin{pgfscope}%
\pgftext[x=0.226704in,y=1.261474in,right,]{\rmfamily\fontsize{8.000000}{9.600000}\selectfont 3.0}%
\end{pgfscope}%
\begin{pgfscope}%
\pgfsetbuttcap%
\pgfsetroundjoin%
\definecolor{currentfill}{rgb}{0.000000,0.000000,0.000000}%
\pgfsetfillcolor{currentfill}%
\pgfsetlinewidth{0.501875pt}%
\definecolor{currentstroke}{rgb}{0.000000,0.000000,0.000000}%
\pgfsetstrokecolor{currentstroke}%
\pgfsetdash{}{0pt}%
\pgfsys@defobject{currentmarker}{\pgfqpoint{0.000000in}{0.000000in}}{\pgfqpoint{0.069444in}{0.000000in}}{%
\pgfpathmoveto{\pgfqpoint{0.000000in}{0.000000in}}%
\pgfpathlineto{\pgfqpoint{0.069444in}{0.000000in}}%
\pgfusepath{stroke,fill}%
}%
\begin{pgfscope}%
\pgfsys@transformshift{0.296148in}{1.402154in}%
\pgfsys@useobject{currentmarker}{}%
\end{pgfscope}%
\end{pgfscope}%
\begin{pgfscope}%
\pgfsetbuttcap%
\pgfsetroundjoin%
\definecolor{currentfill}{rgb}{0.000000,0.000000,0.000000}%
\pgfsetfillcolor{currentfill}%
\pgfsetlinewidth{0.501875pt}%
\definecolor{currentstroke}{rgb}{0.000000,0.000000,0.000000}%
\pgfsetstrokecolor{currentstroke}%
\pgfsetdash{}{0pt}%
\pgfsys@defobject{currentmarker}{\pgfqpoint{-0.069444in}{0.000000in}}{\pgfqpoint{0.000000in}{0.000000in}}{%
\pgfpathmoveto{\pgfqpoint{0.000000in}{0.000000in}}%
\pgfpathlineto{\pgfqpoint{-0.069444in}{0.000000in}}%
\pgfusepath{stroke,fill}%
}%
\begin{pgfscope}%
\pgfsys@transformshift{2.541794in}{1.402154in}%
\pgfsys@useobject{currentmarker}{}%
\end{pgfscope}%
\end{pgfscope}%
\begin{pgfscope}%
\pgftext[x=0.226704in,y=1.402154in,right,]{\rmfamily\fontsize{8.000000}{9.600000}\selectfont 3.5}%
\end{pgfscope}%
\begin{pgfscope}%
\pgfsetbuttcap%
\pgfsetroundjoin%
\definecolor{currentfill}{rgb}{0.000000,0.000000,0.000000}%
\pgfsetfillcolor{currentfill}%
\pgfsetlinewidth{0.501875pt}%
\definecolor{currentstroke}{rgb}{0.000000,0.000000,0.000000}%
\pgfsetstrokecolor{currentstroke}%
\pgfsetdash{}{0pt}%
\pgfsys@defobject{currentmarker}{\pgfqpoint{0.000000in}{0.000000in}}{\pgfqpoint{0.069444in}{0.000000in}}{%
\pgfpathmoveto{\pgfqpoint{0.000000in}{0.000000in}}%
\pgfpathlineto{\pgfqpoint{0.069444in}{0.000000in}}%
\pgfusepath{stroke,fill}%
}%
\begin{pgfscope}%
\pgfsys@transformshift{0.296148in}{1.542835in}%
\pgfsys@useobject{currentmarker}{}%
\end{pgfscope}%
\end{pgfscope}%
\begin{pgfscope}%
\pgfsetbuttcap%
\pgfsetroundjoin%
\definecolor{currentfill}{rgb}{0.000000,0.000000,0.000000}%
\pgfsetfillcolor{currentfill}%
\pgfsetlinewidth{0.501875pt}%
\definecolor{currentstroke}{rgb}{0.000000,0.000000,0.000000}%
\pgfsetstrokecolor{currentstroke}%
\pgfsetdash{}{0pt}%
\pgfsys@defobject{currentmarker}{\pgfqpoint{-0.069444in}{0.000000in}}{\pgfqpoint{0.000000in}{0.000000in}}{%
\pgfpathmoveto{\pgfqpoint{0.000000in}{0.000000in}}%
\pgfpathlineto{\pgfqpoint{-0.069444in}{0.000000in}}%
\pgfusepath{stroke,fill}%
}%
\begin{pgfscope}%
\pgfsys@transformshift{2.541794in}{1.542835in}%
\pgfsys@useobject{currentmarker}{}%
\end{pgfscope}%
\end{pgfscope}%
\begin{pgfscope}%
\pgftext[x=0.226704in,y=1.542835in,right,]{\rmfamily\fontsize{8.000000}{9.600000}\selectfont 4.0}%
\end{pgfscope}%
\begin{pgfscope}%
\pgfsetbuttcap%
\pgfsetroundjoin%
\definecolor{currentfill}{rgb}{0.000000,0.000000,0.000000}%
\pgfsetfillcolor{currentfill}%
\pgfsetlinewidth{0.501875pt}%
\definecolor{currentstroke}{rgb}{0.000000,0.000000,0.000000}%
\pgfsetstrokecolor{currentstroke}%
\pgfsetdash{}{0pt}%
\pgfsys@defobject{currentmarker}{\pgfqpoint{0.000000in}{0.000000in}}{\pgfqpoint{0.069444in}{0.000000in}}{%
\pgfpathmoveto{\pgfqpoint{0.000000in}{0.000000in}}%
\pgfpathlineto{\pgfqpoint{0.069444in}{0.000000in}}%
\pgfusepath{stroke,fill}%
}%
\begin{pgfscope}%
\pgfsys@transformshift{0.296148in}{1.683515in}%
\pgfsys@useobject{currentmarker}{}%
\end{pgfscope}%
\end{pgfscope}%
\begin{pgfscope}%
\pgfsetbuttcap%
\pgfsetroundjoin%
\definecolor{currentfill}{rgb}{0.000000,0.000000,0.000000}%
\pgfsetfillcolor{currentfill}%
\pgfsetlinewidth{0.501875pt}%
\definecolor{currentstroke}{rgb}{0.000000,0.000000,0.000000}%
\pgfsetstrokecolor{currentstroke}%
\pgfsetdash{}{0pt}%
\pgfsys@defobject{currentmarker}{\pgfqpoint{-0.069444in}{0.000000in}}{\pgfqpoint{0.000000in}{0.000000in}}{%
\pgfpathmoveto{\pgfqpoint{0.000000in}{0.000000in}}%
\pgfpathlineto{\pgfqpoint{-0.069444in}{0.000000in}}%
\pgfusepath{stroke,fill}%
}%
\begin{pgfscope}%
\pgfsys@transformshift{2.541794in}{1.683515in}%
\pgfsys@useobject{currentmarker}{}%
\end{pgfscope}%
\end{pgfscope}%
\begin{pgfscope}%
\pgftext[x=0.226704in,y=1.683515in,right,]{\rmfamily\fontsize{8.000000}{9.600000}\selectfont 4.5}%
\end{pgfscope}%
\end{pgfpicture}%
\makeatother%
\endgroup%

	\end{subfigure}
	\begin{subfigure}[t]{0.49\textwidth}
		\centering
    %\includegraphics[width=\textwidth]{store/variables/DATA_MC_REWEIGHTED_B_TAU.pdf}
    %% Creator: Matplotlib, PGF backend
%%
%% To include the figure in your LaTeX document, write
%%   \input{<filename>.pgf}
%%
%% Make sure the required packages are loaded in your preamble
%%   \usepackage{pgf}
%%
%% Figures using additional raster images can only be included by \input if
%% they are in the same directory as the main LaTeX file. For loading figures
%% from other directories you can use the `import` package
%%   \usepackage{import}
%% and then include the figures with
%%   \import{<path to file>}{<filename>.pgf}
%%
%% Matplotlib used the following preamble
%%   \usepackage{fontspec}
%%   \setmainfont{DejaVu Serif}
%%   \setsansfont{DejaVu Sans}
%%   \setmonofont{DejaVu Sans Mono}
%%
\begingroup%
\makeatletter%
\begin{pgfpicture}%
\pgfpathrectangle{\pgfpointorigin}{\pgfqpoint{2.683883in}{1.741309in}}%
\pgfusepath{use as bounding box, clip}%
\begin{pgfscope}%
\pgfsetbuttcap%
\pgfsetmiterjoin%
\definecolor{currentfill}{rgb}{1.000000,1.000000,1.000000}%
\pgfsetfillcolor{currentfill}%
\pgfsetlinewidth{0.000000pt}%
\definecolor{currentstroke}{rgb}{1.000000,1.000000,1.000000}%
\pgfsetstrokecolor{currentstroke}%
\pgfsetdash{}{0pt}%
\pgfpathmoveto{\pgfqpoint{0.000000in}{0.000000in}}%
\pgfpathlineto{\pgfqpoint{2.683883in}{0.000000in}}%
\pgfpathlineto{\pgfqpoint{2.683883in}{1.741309in}}%
\pgfpathlineto{\pgfqpoint{0.000000in}{1.741309in}}%
\pgfpathclose%
\pgfusepath{fill}%
\end{pgfscope}%
\begin{pgfscope}%
\pgfsetbuttcap%
\pgfsetmiterjoin%
\definecolor{currentfill}{rgb}{1.000000,1.000000,1.000000}%
\pgfsetfillcolor{currentfill}%
\pgfsetlinewidth{0.000000pt}%
\definecolor{currentstroke}{rgb}{0.000000,0.000000,0.000000}%
\pgfsetstrokecolor{currentstroke}%
\pgfsetstrokeopacity{0.000000}%
\pgfsetdash{}{0pt}%
\pgfpathmoveto{\pgfqpoint{0.366840in}{0.417391in}}%
\pgfpathlineto{\pgfqpoint{2.563190in}{0.417391in}}%
\pgfpathlineto{\pgfqpoint{2.563190in}{1.637544in}}%
\pgfpathlineto{\pgfqpoint{0.366840in}{1.637544in}}%
\pgfpathclose%
\pgfusepath{fill}%
\end{pgfscope}%
\begin{pgfscope}%
\pgfpathrectangle{\pgfqpoint{0.366840in}{0.417391in}}{\pgfqpoint{2.196350in}{1.220153in}} %
\pgfusepath{clip}%
\pgfsetbuttcap%
\pgfsetmiterjoin%
\definecolor{currentfill}{rgb}{0.215686,0.470588,0.749020}%
\pgfsetfillcolor{currentfill}%
\pgfsetlinewidth{0.000000pt}%
\definecolor{currentstroke}{rgb}{0.000000,0.000000,0.000000}%
\pgfsetstrokecolor{currentstroke}%
\pgfsetdash{}{0pt}%
\pgfpathmoveto{\pgfqpoint{0.366840in}{0.417391in}}%
\pgfpathlineto{\pgfqpoint{0.366840in}{0.417643in}}%
\pgfpathlineto{\pgfqpoint{0.410767in}{0.417643in}}%
\pgfpathlineto{\pgfqpoint{0.410767in}{0.504071in}}%
\pgfpathlineto{\pgfqpoint{0.454694in}{0.504071in}}%
\pgfpathlineto{\pgfqpoint{0.454694in}{0.925919in}}%
\pgfpathlineto{\pgfqpoint{0.498621in}{0.925919in}}%
\pgfpathlineto{\pgfqpoint{0.498621in}{1.337314in}}%
\pgfpathlineto{\pgfqpoint{0.542548in}{1.337314in}}%
\pgfpathlineto{\pgfqpoint{0.542548in}{1.533480in}}%
\pgfpathlineto{\pgfqpoint{0.586475in}{1.533480in}}%
\pgfpathlineto{\pgfqpoint{0.586475in}{1.591442in}}%
\pgfpathlineto{\pgfqpoint{0.630402in}{1.591442in}}%
\pgfpathlineto{\pgfqpoint{0.630402in}{1.579504in}}%
\pgfpathlineto{\pgfqpoint{0.674329in}{1.579504in}}%
\pgfpathlineto{\pgfqpoint{0.674329in}{1.493477in}}%
\pgfpathlineto{\pgfqpoint{0.718256in}{1.493477in}}%
\pgfpathlineto{\pgfqpoint{0.718256in}{1.426316in}}%
\pgfpathlineto{\pgfqpoint{0.762183in}{1.426316in}}%
\pgfpathlineto{\pgfqpoint{0.762183in}{1.329781in}}%
\pgfpathlineto{\pgfqpoint{0.806110in}{1.329781in}}%
\pgfpathlineto{\pgfqpoint{0.806110in}{1.264074in}}%
\pgfpathlineto{\pgfqpoint{0.850037in}{1.264074in}}%
\pgfpathlineto{\pgfqpoint{0.850037in}{1.176642in}}%
\pgfpathlineto{\pgfqpoint{0.893964in}{1.176642in}}%
\pgfpathlineto{\pgfqpoint{0.893964in}{1.082195in}}%
\pgfpathlineto{\pgfqpoint{0.937891in}{1.082195in}}%
\pgfpathlineto{\pgfqpoint{0.937891in}{1.016310in}}%
\pgfpathlineto{\pgfqpoint{0.981818in}{1.016310in}}%
\pgfpathlineto{\pgfqpoint{0.981818in}{0.952766in}}%
\pgfpathlineto{\pgfqpoint{1.025745in}{0.952766in}}%
\pgfpathlineto{\pgfqpoint{1.025745in}{0.902095in}}%
\pgfpathlineto{\pgfqpoint{1.069672in}{0.902095in}}%
\pgfpathlineto{\pgfqpoint{1.069672in}{0.841896in}}%
\pgfpathlineto{\pgfqpoint{1.113599in}{0.841896in}}%
\pgfpathlineto{\pgfqpoint{1.113599in}{0.784115in}}%
\pgfpathlineto{\pgfqpoint{1.157526in}{0.784115in}}%
\pgfpathlineto{\pgfqpoint{1.157526in}{0.738083in}}%
\pgfpathlineto{\pgfqpoint{1.201453in}{0.738083in}}%
\pgfpathlineto{\pgfqpoint{1.201453in}{0.705430in}}%
\pgfpathlineto{\pgfqpoint{1.245380in}{0.705430in}}%
\pgfpathlineto{\pgfqpoint{1.245380in}{0.670605in}}%
\pgfpathlineto{\pgfqpoint{1.289307in}{0.670605in}}%
\pgfpathlineto{\pgfqpoint{1.289307in}{0.638023in}}%
\pgfpathlineto{\pgfqpoint{1.333234in}{0.638023in}}%
\pgfpathlineto{\pgfqpoint{1.333234in}{0.607420in}}%
\pgfpathlineto{\pgfqpoint{1.377161in}{0.607420in}}%
\pgfpathlineto{\pgfqpoint{1.377161in}{0.591077in}}%
\pgfpathlineto{\pgfqpoint{1.421088in}{0.591077in}}%
\pgfpathlineto{\pgfqpoint{1.421088in}{0.568228in}}%
\pgfpathlineto{\pgfqpoint{1.465015in}{0.568228in}}%
\pgfpathlineto{\pgfqpoint{1.465015in}{0.544456in}}%
\pgfpathlineto{\pgfqpoint{1.508942in}{0.544456in}}%
\pgfpathlineto{\pgfqpoint{1.508942in}{0.528188in}}%
\pgfpathlineto{\pgfqpoint{1.552869in}{0.528188in}}%
\pgfpathlineto{\pgfqpoint{1.552869in}{0.519380in}}%
\pgfpathlineto{\pgfqpoint{1.596796in}{0.519380in}}%
\pgfpathlineto{\pgfqpoint{1.596796in}{0.508284in}}%
\pgfpathlineto{\pgfqpoint{1.640723in}{0.508284in}}%
\pgfpathlineto{\pgfqpoint{1.640723in}{0.496888in}}%
\pgfpathlineto{\pgfqpoint{1.684650in}{0.496888in}}%
\pgfpathlineto{\pgfqpoint{1.684650in}{0.482249in}}%
\pgfpathlineto{\pgfqpoint{1.728577in}{0.482249in}}%
\pgfpathlineto{\pgfqpoint{1.728577in}{0.477051in}}%
\pgfpathlineto{\pgfqpoint{1.772504in}{0.477051in}}%
\pgfpathlineto{\pgfqpoint{1.772504in}{0.466787in}}%
\pgfpathlineto{\pgfqpoint{1.816431in}{0.466787in}}%
\pgfpathlineto{\pgfqpoint{1.816431in}{0.464930in}}%
\pgfpathlineto{\pgfqpoint{1.860358in}{0.464930in}}%
\pgfpathlineto{\pgfqpoint{1.860358in}{0.455803in}}%
\pgfpathlineto{\pgfqpoint{1.904285in}{0.455803in}}%
\pgfpathlineto{\pgfqpoint{1.904285in}{0.451594in}}%
\pgfpathlineto{\pgfqpoint{1.948212in}{0.451594in}}%
\pgfpathlineto{\pgfqpoint{1.948212in}{0.449105in}}%
\pgfpathlineto{\pgfqpoint{1.992139in}{0.449105in}}%
\pgfpathlineto{\pgfqpoint{1.992139in}{0.445646in}}%
\pgfpathlineto{\pgfqpoint{2.036066in}{0.445646in}}%
\pgfpathlineto{\pgfqpoint{2.036066in}{0.440347in}}%
\pgfpathlineto{\pgfqpoint{2.079993in}{0.440347in}}%
\pgfpathlineto{\pgfqpoint{2.079993in}{0.437649in}}%
\pgfpathlineto{\pgfqpoint{2.123920in}{0.437649in}}%
\pgfpathlineto{\pgfqpoint{2.123920in}{0.435312in}}%
\pgfpathlineto{\pgfqpoint{2.167847in}{0.435312in}}%
\pgfpathlineto{\pgfqpoint{2.167847in}{0.431999in}}%
\pgfpathlineto{\pgfqpoint{2.211774in}{0.431999in}}%
\pgfpathlineto{\pgfqpoint{2.211774in}{0.428878in}}%
\pgfpathlineto{\pgfqpoint{2.255701in}{0.428878in}}%
\pgfpathlineto{\pgfqpoint{2.255701in}{0.428442in}}%
\pgfpathlineto{\pgfqpoint{2.299628in}{0.428442in}}%
\pgfpathlineto{\pgfqpoint{2.299628in}{0.428069in}}%
\pgfpathlineto{\pgfqpoint{2.343555in}{0.428069in}}%
\pgfpathlineto{\pgfqpoint{2.343555in}{0.427075in}}%
\pgfpathlineto{\pgfqpoint{2.387482in}{0.427075in}}%
\pgfpathlineto{\pgfqpoint{2.387482in}{0.425082in}}%
\pgfpathlineto{\pgfqpoint{2.431409in}{0.425082in}}%
\pgfpathlineto{\pgfqpoint{2.431409in}{0.423956in}}%
\pgfpathlineto{\pgfqpoint{2.475336in}{0.423956in}}%
\pgfpathlineto{\pgfqpoint{2.475336in}{0.423540in}}%
\pgfpathlineto{\pgfqpoint{2.519263in}{0.423540in}}%
\pgfpathlineto{\pgfqpoint{2.519263in}{0.422847in}}%
\pgfpathlineto{\pgfqpoint{2.563190in}{0.422847in}}%
\pgfpathlineto{\pgfqpoint{2.563190in}{0.417391in}}%
\pgfpathlineto{\pgfqpoint{2.519263in}{0.417391in}}%
\pgfpathlineto{\pgfqpoint{2.519263in}{0.417391in}}%
\pgfpathlineto{\pgfqpoint{2.475336in}{0.417391in}}%
\pgfpathlineto{\pgfqpoint{2.475336in}{0.417391in}}%
\pgfpathlineto{\pgfqpoint{2.431409in}{0.417391in}}%
\pgfpathlineto{\pgfqpoint{2.431409in}{0.417391in}}%
\pgfpathlineto{\pgfqpoint{2.387482in}{0.417391in}}%
\pgfpathlineto{\pgfqpoint{2.387482in}{0.417391in}}%
\pgfpathlineto{\pgfqpoint{2.343555in}{0.417391in}}%
\pgfpathlineto{\pgfqpoint{2.343555in}{0.417391in}}%
\pgfpathlineto{\pgfqpoint{2.299628in}{0.417391in}}%
\pgfpathlineto{\pgfqpoint{2.299628in}{0.417391in}}%
\pgfpathlineto{\pgfqpoint{2.255701in}{0.417391in}}%
\pgfpathlineto{\pgfqpoint{2.255701in}{0.417391in}}%
\pgfpathlineto{\pgfqpoint{2.211774in}{0.417391in}}%
\pgfpathlineto{\pgfqpoint{2.211774in}{0.417391in}}%
\pgfpathlineto{\pgfqpoint{2.167847in}{0.417391in}}%
\pgfpathlineto{\pgfqpoint{2.167847in}{0.417391in}}%
\pgfpathlineto{\pgfqpoint{2.123920in}{0.417391in}}%
\pgfpathlineto{\pgfqpoint{2.123920in}{0.417391in}}%
\pgfpathlineto{\pgfqpoint{2.079993in}{0.417391in}}%
\pgfpathlineto{\pgfqpoint{2.079993in}{0.417391in}}%
\pgfpathlineto{\pgfqpoint{2.036066in}{0.417391in}}%
\pgfpathlineto{\pgfqpoint{2.036066in}{0.417391in}}%
\pgfpathlineto{\pgfqpoint{1.992139in}{0.417391in}}%
\pgfpathlineto{\pgfqpoint{1.992139in}{0.417391in}}%
\pgfpathlineto{\pgfqpoint{1.948212in}{0.417391in}}%
\pgfpathlineto{\pgfqpoint{1.948212in}{0.417391in}}%
\pgfpathlineto{\pgfqpoint{1.904285in}{0.417391in}}%
\pgfpathlineto{\pgfqpoint{1.904285in}{0.417391in}}%
\pgfpathlineto{\pgfqpoint{1.860358in}{0.417391in}}%
\pgfpathlineto{\pgfqpoint{1.860358in}{0.417391in}}%
\pgfpathlineto{\pgfqpoint{1.816431in}{0.417391in}}%
\pgfpathlineto{\pgfqpoint{1.816431in}{0.417391in}}%
\pgfpathlineto{\pgfqpoint{1.772504in}{0.417391in}}%
\pgfpathlineto{\pgfqpoint{1.772504in}{0.417391in}}%
\pgfpathlineto{\pgfqpoint{1.728577in}{0.417391in}}%
\pgfpathlineto{\pgfqpoint{1.728577in}{0.417391in}}%
\pgfpathlineto{\pgfqpoint{1.684650in}{0.417391in}}%
\pgfpathlineto{\pgfqpoint{1.684650in}{0.417391in}}%
\pgfpathlineto{\pgfqpoint{1.640723in}{0.417391in}}%
\pgfpathlineto{\pgfqpoint{1.640723in}{0.417391in}}%
\pgfpathlineto{\pgfqpoint{1.596796in}{0.417391in}}%
\pgfpathlineto{\pgfqpoint{1.596796in}{0.417391in}}%
\pgfpathlineto{\pgfqpoint{1.552869in}{0.417391in}}%
\pgfpathlineto{\pgfqpoint{1.552869in}{0.417391in}}%
\pgfpathlineto{\pgfqpoint{1.508942in}{0.417391in}}%
\pgfpathlineto{\pgfqpoint{1.508942in}{0.417391in}}%
\pgfpathlineto{\pgfqpoint{1.465015in}{0.417391in}}%
\pgfpathlineto{\pgfqpoint{1.465015in}{0.417391in}}%
\pgfpathlineto{\pgfqpoint{1.421088in}{0.417391in}}%
\pgfpathlineto{\pgfqpoint{1.421088in}{0.417391in}}%
\pgfpathlineto{\pgfqpoint{1.377161in}{0.417391in}}%
\pgfpathlineto{\pgfqpoint{1.377161in}{0.417391in}}%
\pgfpathlineto{\pgfqpoint{1.333234in}{0.417391in}}%
\pgfpathlineto{\pgfqpoint{1.333234in}{0.417391in}}%
\pgfpathlineto{\pgfqpoint{1.289307in}{0.417391in}}%
\pgfpathlineto{\pgfqpoint{1.289307in}{0.417391in}}%
\pgfpathlineto{\pgfqpoint{1.245380in}{0.417391in}}%
\pgfpathlineto{\pgfqpoint{1.245380in}{0.417391in}}%
\pgfpathlineto{\pgfqpoint{1.201453in}{0.417391in}}%
\pgfpathlineto{\pgfqpoint{1.201453in}{0.417391in}}%
\pgfpathlineto{\pgfqpoint{1.157526in}{0.417391in}}%
\pgfpathlineto{\pgfqpoint{1.157526in}{0.417391in}}%
\pgfpathlineto{\pgfqpoint{1.113599in}{0.417391in}}%
\pgfpathlineto{\pgfqpoint{1.113599in}{0.417391in}}%
\pgfpathlineto{\pgfqpoint{1.069672in}{0.417391in}}%
\pgfpathlineto{\pgfqpoint{1.069672in}{0.417391in}}%
\pgfpathlineto{\pgfqpoint{1.025745in}{0.417391in}}%
\pgfpathlineto{\pgfqpoint{1.025745in}{0.417391in}}%
\pgfpathlineto{\pgfqpoint{0.981818in}{0.417391in}}%
\pgfpathlineto{\pgfqpoint{0.981818in}{0.417391in}}%
\pgfpathlineto{\pgfqpoint{0.937891in}{0.417391in}}%
\pgfpathlineto{\pgfqpoint{0.937891in}{0.417391in}}%
\pgfpathlineto{\pgfqpoint{0.893964in}{0.417391in}}%
\pgfpathlineto{\pgfqpoint{0.893964in}{0.417391in}}%
\pgfpathlineto{\pgfqpoint{0.850037in}{0.417391in}}%
\pgfpathlineto{\pgfqpoint{0.850037in}{0.417391in}}%
\pgfpathlineto{\pgfqpoint{0.806110in}{0.417391in}}%
\pgfpathlineto{\pgfqpoint{0.806110in}{0.417391in}}%
\pgfpathlineto{\pgfqpoint{0.762183in}{0.417391in}}%
\pgfpathlineto{\pgfqpoint{0.762183in}{0.417391in}}%
\pgfpathlineto{\pgfqpoint{0.718256in}{0.417391in}}%
\pgfpathlineto{\pgfqpoint{0.718256in}{0.417391in}}%
\pgfpathlineto{\pgfqpoint{0.674329in}{0.417391in}}%
\pgfpathlineto{\pgfqpoint{0.674329in}{0.417391in}}%
\pgfpathlineto{\pgfqpoint{0.630402in}{0.417391in}}%
\pgfpathlineto{\pgfqpoint{0.630402in}{0.417391in}}%
\pgfpathlineto{\pgfqpoint{0.586475in}{0.417391in}}%
\pgfpathlineto{\pgfqpoint{0.586475in}{0.417391in}}%
\pgfpathlineto{\pgfqpoint{0.542548in}{0.417391in}}%
\pgfpathlineto{\pgfqpoint{0.542548in}{0.417391in}}%
\pgfpathlineto{\pgfqpoint{0.498621in}{0.417391in}}%
\pgfpathlineto{\pgfqpoint{0.498621in}{0.417391in}}%
\pgfpathlineto{\pgfqpoint{0.454694in}{0.417391in}}%
\pgfpathlineto{\pgfqpoint{0.454694in}{0.417391in}}%
\pgfpathlineto{\pgfqpoint{0.410767in}{0.417391in}}%
\pgfpathlineto{\pgfqpoint{0.410767in}{0.417391in}}%
\pgfpathlineto{\pgfqpoint{0.366840in}{0.417391in}}%
\pgfusepath{fill}%
\end{pgfscope}%
\begin{pgfscope}%
\pgfpathrectangle{\pgfqpoint{0.366840in}{0.417391in}}{\pgfqpoint{2.196350in}{1.220153in}} %
\pgfusepath{clip}%
\pgfsetbuttcap%
\pgfsetmiterjoin%
\pgfsetlinewidth{0.501875pt}%
\definecolor{currentstroke}{rgb}{1.000000,0.000000,0.000000}%
\pgfsetstrokecolor{currentstroke}%
\pgfsetdash{}{0pt}%
\pgfpathmoveto{\pgfqpoint{0.366840in}{0.417391in}}%
\pgfpathlineto{\pgfqpoint{0.366840in}{0.417506in}}%
\pgfpathlineto{\pgfqpoint{0.410767in}{0.417506in}}%
\pgfpathlineto{\pgfqpoint{0.410767in}{0.503643in}}%
\pgfpathlineto{\pgfqpoint{0.454694in}{0.503643in}}%
\pgfpathlineto{\pgfqpoint{0.454694in}{0.925696in}}%
\pgfpathlineto{\pgfqpoint{0.498621in}{0.925696in}}%
\pgfpathlineto{\pgfqpoint{0.498621in}{1.323588in}}%
\pgfpathlineto{\pgfqpoint{0.542548in}{1.323588in}}%
\pgfpathlineto{\pgfqpoint{0.542548in}{1.522745in}}%
\pgfpathlineto{\pgfqpoint{0.586475in}{1.522745in}}%
\pgfpathlineto{\pgfqpoint{0.586475in}{1.577396in}}%
\pgfpathlineto{\pgfqpoint{0.630402in}{1.577396in}}%
\pgfpathlineto{\pgfqpoint{0.630402in}{1.557530in}}%
\pgfpathlineto{\pgfqpoint{0.674329in}{1.557530in}}%
\pgfpathlineto{\pgfqpoint{0.674329in}{1.503915in}}%
\pgfpathlineto{\pgfqpoint{0.718256in}{1.503915in}}%
\pgfpathlineto{\pgfqpoint{0.718256in}{1.428018in}}%
\pgfpathlineto{\pgfqpoint{0.762183in}{1.428018in}}%
\pgfpathlineto{\pgfqpoint{0.762183in}{1.342304in}}%
\pgfpathlineto{\pgfqpoint{0.806110in}{1.342304in}}%
\pgfpathlineto{\pgfqpoint{0.806110in}{1.253789in}}%
\pgfpathlineto{\pgfqpoint{0.850037in}{1.253789in}}%
\pgfpathlineto{\pgfqpoint{0.850037in}{1.169110in}}%
\pgfpathlineto{\pgfqpoint{0.893964in}{1.169110in}}%
\pgfpathlineto{\pgfqpoint{0.893964in}{1.097125in}}%
\pgfpathlineto{\pgfqpoint{0.937891in}{1.097125in}}%
\pgfpathlineto{\pgfqpoint{0.937891in}{1.012638in}}%
\pgfpathlineto{\pgfqpoint{0.981818in}{1.012638in}}%
\pgfpathlineto{\pgfqpoint{0.981818in}{0.953692in}}%
\pgfpathlineto{\pgfqpoint{1.025745in}{0.953692in}}%
\pgfpathlineto{\pgfqpoint{1.025745in}{0.891448in}}%
\pgfpathlineto{\pgfqpoint{1.069672in}{0.891448in}}%
\pgfpathlineto{\pgfqpoint{1.069672in}{0.843625in}}%
\pgfpathlineto{\pgfqpoint{1.113599in}{0.843625in}}%
\pgfpathlineto{\pgfqpoint{1.113599in}{0.789243in}}%
\pgfpathlineto{\pgfqpoint{1.157526in}{0.789243in}}%
\pgfpathlineto{\pgfqpoint{1.157526in}{0.746903in}}%
\pgfpathlineto{\pgfqpoint{1.201453in}{0.746903in}}%
\pgfpathlineto{\pgfqpoint{1.201453in}{0.709012in}}%
\pgfpathlineto{\pgfqpoint{1.245380in}{0.709012in}}%
\pgfpathlineto{\pgfqpoint{1.245380in}{0.670393in}}%
\pgfpathlineto{\pgfqpoint{1.289307in}{0.670393in}}%
\pgfpathlineto{\pgfqpoint{1.289307in}{0.638408in}}%
\pgfpathlineto{\pgfqpoint{1.333234in}{0.638408in}}%
\pgfpathlineto{\pgfqpoint{1.333234in}{0.609760in}}%
\pgfpathlineto{\pgfqpoint{1.377161in}{0.609760in}}%
\pgfpathlineto{\pgfqpoint{1.377161in}{0.593729in}}%
\pgfpathlineto{\pgfqpoint{1.421088in}{0.593729in}}%
\pgfpathlineto{\pgfqpoint{1.421088in}{0.571064in}}%
\pgfpathlineto{\pgfqpoint{1.465015in}{0.571064in}}%
\pgfpathlineto{\pgfqpoint{1.465015in}{0.552540in}}%
\pgfpathlineto{\pgfqpoint{1.508942in}{0.552540in}}%
\pgfpathlineto{\pgfqpoint{1.508942in}{0.533326in}}%
\pgfpathlineto{\pgfqpoint{1.552869in}{0.533326in}}%
\pgfpathlineto{\pgfqpoint{1.552869in}{0.521054in}}%
\pgfpathlineto{\pgfqpoint{1.596796in}{0.521054in}}%
\pgfpathlineto{\pgfqpoint{1.596796in}{0.509280in}}%
\pgfpathlineto{\pgfqpoint{1.640723in}{0.509280in}}%
\pgfpathlineto{\pgfqpoint{1.640723in}{0.493978in}}%
\pgfpathlineto{\pgfqpoint{1.684650in}{0.493978in}}%
\pgfpathlineto{\pgfqpoint{1.684650in}{0.484390in}}%
\pgfpathlineto{\pgfqpoint{1.728577in}{0.484390in}}%
\pgfpathlineto{\pgfqpoint{1.728577in}{0.478906in}}%
\pgfpathlineto{\pgfqpoint{1.772504in}{0.478906in}}%
\pgfpathlineto{\pgfqpoint{1.772504in}{0.469970in}}%
\pgfpathlineto{\pgfqpoint{1.816431in}{0.469970in}}%
\pgfpathlineto{\pgfqpoint{1.816431in}{0.461456in}}%
\pgfpathlineto{\pgfqpoint{1.860358in}{0.461456in}}%
\pgfpathlineto{\pgfqpoint{1.860358in}{0.457583in}}%
\pgfpathlineto{\pgfqpoint{1.904285in}{0.457583in}}%
\pgfpathlineto{\pgfqpoint{1.904285in}{0.453326in}}%
\pgfpathlineto{\pgfqpoint{1.948212in}{0.453326in}}%
\pgfpathlineto{\pgfqpoint{1.948212in}{0.450718in}}%
\pgfpathlineto{\pgfqpoint{1.992139in}{0.450718in}}%
\pgfpathlineto{\pgfqpoint{1.992139in}{0.444390in}}%
\pgfpathlineto{\pgfqpoint{2.036066in}{0.444390in}}%
\pgfpathlineto{\pgfqpoint{2.036066in}{0.441399in}}%
\pgfpathlineto{\pgfqpoint{2.079993in}{0.441399in}}%
\pgfpathlineto{\pgfqpoint{2.079993in}{0.437794in}}%
\pgfpathlineto{\pgfqpoint{2.123920in}{0.437794in}}%
\pgfpathlineto{\pgfqpoint{2.123920in}{0.434496in}}%
\pgfpathlineto{\pgfqpoint{2.167847in}{0.434496in}}%
\pgfpathlineto{\pgfqpoint{2.167847in}{0.433690in}}%
\pgfpathlineto{\pgfqpoint{2.211774in}{0.433690in}}%
\pgfpathlineto{\pgfqpoint{2.211774in}{0.431351in}}%
\pgfpathlineto{\pgfqpoint{2.255701in}{0.431351in}}%
\pgfpathlineto{\pgfqpoint{2.255701in}{0.429702in}}%
\pgfpathlineto{\pgfqpoint{2.299628in}{0.429702in}}%
\pgfpathlineto{\pgfqpoint{2.299628in}{0.428129in}}%
\pgfpathlineto{\pgfqpoint{2.343555in}{0.428129in}}%
\pgfpathlineto{\pgfqpoint{2.343555in}{0.427094in}}%
\pgfpathlineto{\pgfqpoint{2.387482in}{0.427094in}}%
\pgfpathlineto{\pgfqpoint{2.387482in}{0.425291in}}%
\pgfpathlineto{\pgfqpoint{2.431409in}{0.425291in}}%
\pgfpathlineto{\pgfqpoint{2.431409in}{0.423412in}}%
\pgfpathlineto{\pgfqpoint{2.475336in}{0.423412in}}%
\pgfpathlineto{\pgfqpoint{2.475336in}{0.423796in}}%
\pgfpathlineto{\pgfqpoint{2.519263in}{0.423796in}}%
\pgfpathlineto{\pgfqpoint{2.519263in}{0.422108in}}%
\pgfpathlineto{\pgfqpoint{2.563190in}{0.422108in}}%
\pgfpathlineto{\pgfqpoint{2.563190in}{0.417391in}}%
\pgfusepath{stroke}%
\end{pgfscope}%
\begin{pgfscope}%
\pgfpathrectangle{\pgfqpoint{0.366840in}{0.417391in}}{\pgfqpoint{2.196350in}{1.220153in}} %
\pgfusepath{clip}%
\pgfsetbuttcap%
\pgfsetmiterjoin%
\pgfsetlinewidth{1.003750pt}%
\definecolor{currentstroke}{rgb}{1.000000,0.647059,0.000000}%
\pgfsetstrokecolor{currentstroke}%
\pgfsetdash{}{0pt}%
\pgfpathmoveto{\pgfqpoint{0.366840in}{0.417391in}}%
\pgfpathlineto{\pgfqpoint{0.366840in}{0.417555in}}%
\pgfpathlineto{\pgfqpoint{0.410767in}{0.417555in}}%
\pgfpathlineto{\pgfqpoint{0.410767in}{0.502434in}}%
\pgfpathlineto{\pgfqpoint{0.454694in}{0.502434in}}%
\pgfpathlineto{\pgfqpoint{0.454694in}{0.927448in}}%
\pgfpathlineto{\pgfqpoint{0.498621in}{0.927448in}}%
\pgfpathlineto{\pgfqpoint{0.498621in}{1.342817in}}%
\pgfpathlineto{\pgfqpoint{0.542548in}{1.342817in}}%
\pgfpathlineto{\pgfqpoint{0.542548in}{1.542269in}}%
\pgfpathlineto{\pgfqpoint{0.586475in}{1.542269in}}%
\pgfpathlineto{\pgfqpoint{0.586475in}{1.587976in}}%
\pgfpathlineto{\pgfqpoint{0.630402in}{1.587976in}}%
\pgfpathlineto{\pgfqpoint{0.630402in}{1.572701in}}%
\pgfpathlineto{\pgfqpoint{0.674329in}{1.572701in}}%
\pgfpathlineto{\pgfqpoint{0.674329in}{1.522115in}}%
\pgfpathlineto{\pgfqpoint{0.718256in}{1.522115in}}%
\pgfpathlineto{\pgfqpoint{0.718256in}{1.437680in}}%
\pgfpathlineto{\pgfqpoint{0.762183in}{1.437680in}}%
\pgfpathlineto{\pgfqpoint{0.762183in}{1.345209in}}%
\pgfpathlineto{\pgfqpoint{0.806110in}{1.345209in}}%
\pgfpathlineto{\pgfqpoint{0.806110in}{1.251166in}}%
\pgfpathlineto{\pgfqpoint{0.850037in}{1.251166in}}%
\pgfpathlineto{\pgfqpoint{0.850037in}{1.168414in}}%
\pgfpathlineto{\pgfqpoint{0.893964in}{1.168414in}}%
\pgfpathlineto{\pgfqpoint{0.893964in}{1.095334in}}%
\pgfpathlineto{\pgfqpoint{0.937891in}{1.095334in}}%
\pgfpathlineto{\pgfqpoint{0.937891in}{1.009206in}}%
\pgfpathlineto{\pgfqpoint{0.981818in}{1.009206in}}%
\pgfpathlineto{\pgfqpoint{0.981818in}{0.949252in}}%
\pgfpathlineto{\pgfqpoint{1.025745in}{0.949252in}}%
\pgfpathlineto{\pgfqpoint{1.025745in}{0.885645in}}%
\pgfpathlineto{\pgfqpoint{1.069672in}{0.885645in}}%
\pgfpathlineto{\pgfqpoint{1.069672in}{0.835533in}}%
\pgfpathlineto{\pgfqpoint{1.113599in}{0.835533in}}%
\pgfpathlineto{\pgfqpoint{1.113599in}{0.782074in}}%
\pgfpathlineto{\pgfqpoint{1.157526in}{0.782074in}}%
\pgfpathlineto{\pgfqpoint{1.157526in}{0.743744in}}%
\pgfpathlineto{\pgfqpoint{1.201453in}{0.743744in}}%
\pgfpathlineto{\pgfqpoint{1.201453in}{0.704392in}}%
\pgfpathlineto{\pgfqpoint{1.245380in}{0.704392in}}%
\pgfpathlineto{\pgfqpoint{1.245380in}{0.667480in}}%
\pgfpathlineto{\pgfqpoint{1.289307in}{0.667480in}}%
\pgfpathlineto{\pgfqpoint{1.289307in}{0.633115in}}%
\pgfpathlineto{\pgfqpoint{1.333234in}{0.633115in}}%
\pgfpathlineto{\pgfqpoint{1.333234in}{0.602866in}}%
\pgfpathlineto{\pgfqpoint{1.377161in}{0.602866in}}%
\pgfpathlineto{\pgfqpoint{1.377161in}{0.589708in}}%
\pgfpathlineto{\pgfqpoint{1.421088in}{0.589708in}}%
\pgfpathlineto{\pgfqpoint{1.421088in}{0.567647in}}%
\pgfpathlineto{\pgfqpoint{1.465015in}{0.567647in}}%
\pgfpathlineto{\pgfqpoint{1.465015in}{0.550689in}}%
\pgfpathlineto{\pgfqpoint{1.508942in}{0.550689in}}%
\pgfpathlineto{\pgfqpoint{1.508942in}{0.529471in}}%
\pgfpathlineto{\pgfqpoint{1.552869in}{0.529471in}}%
\pgfpathlineto{\pgfqpoint{1.552869in}{0.518088in}}%
\pgfpathlineto{\pgfqpoint{1.596796in}{0.518088in}}%
\pgfpathlineto{\pgfqpoint{1.596796in}{0.507111in}}%
\pgfpathlineto{\pgfqpoint{1.640723in}{0.507111in}}%
\pgfpathlineto{\pgfqpoint{1.640723in}{0.492313in}}%
\pgfpathlineto{\pgfqpoint{1.684650in}{0.492313in}}%
\pgfpathlineto{\pgfqpoint{1.684650in}{0.482042in}}%
\pgfpathlineto{\pgfqpoint{1.728577in}{0.482042in}}%
\pgfpathlineto{\pgfqpoint{1.728577in}{0.477035in}}%
\pgfpathlineto{\pgfqpoint{1.772504in}{0.477035in}}%
\pgfpathlineto{\pgfqpoint{1.772504in}{0.468810in}}%
\pgfpathlineto{\pgfqpoint{1.816431in}{0.468810in}}%
\pgfpathlineto{\pgfqpoint{1.816431in}{0.458496in}}%
\pgfpathlineto{\pgfqpoint{1.860358in}{0.458496in}}%
\pgfpathlineto{\pgfqpoint{1.860358in}{0.455600in}}%
\pgfpathlineto{\pgfqpoint{1.904285in}{0.455600in}}%
\pgfpathlineto{\pgfqpoint{1.904285in}{0.452072in}}%
\pgfpathlineto{\pgfqpoint{1.948212in}{0.452072in}}%
\pgfpathlineto{\pgfqpoint{1.948212in}{0.449359in}}%
\pgfpathlineto{\pgfqpoint{1.992139in}{0.449359in}}%
\pgfpathlineto{\pgfqpoint{1.992139in}{0.443422in}}%
\pgfpathlineto{\pgfqpoint{2.036066in}{0.443422in}}%
\pgfpathlineto{\pgfqpoint{2.036066in}{0.440763in}}%
\pgfpathlineto{\pgfqpoint{2.079993in}{0.440763in}}%
\pgfpathlineto{\pgfqpoint{2.079993in}{0.437122in}}%
\pgfpathlineto{\pgfqpoint{2.123920in}{0.437122in}}%
\pgfpathlineto{\pgfqpoint{2.123920in}{0.434040in}}%
\pgfpathlineto{\pgfqpoint{2.167847in}{0.434040in}}%
\pgfpathlineto{\pgfqpoint{2.167847in}{0.432895in}}%
\pgfpathlineto{\pgfqpoint{2.211774in}{0.432895in}}%
\pgfpathlineto{\pgfqpoint{2.211774in}{0.431160in}}%
\pgfpathlineto{\pgfqpoint{2.255701in}{0.431160in}}%
\pgfpathlineto{\pgfqpoint{2.255701in}{0.428760in}}%
\pgfpathlineto{\pgfqpoint{2.299628in}{0.428760in}}%
\pgfpathlineto{\pgfqpoint{2.299628in}{0.427539in}}%
\pgfpathlineto{\pgfqpoint{2.343555in}{0.427539in}}%
\pgfpathlineto{\pgfqpoint{2.343555in}{0.426539in}}%
\pgfpathlineto{\pgfqpoint{2.387482in}{0.426539in}}%
\pgfpathlineto{\pgfqpoint{2.387482in}{0.425457in}}%
\pgfpathlineto{\pgfqpoint{2.431409in}{0.425457in}}%
\pgfpathlineto{\pgfqpoint{2.431409in}{0.423595in}}%
\pgfpathlineto{\pgfqpoint{2.475336in}{0.423595in}}%
\pgfpathlineto{\pgfqpoint{2.475336in}{0.423419in}}%
\pgfpathlineto{\pgfqpoint{2.519263in}{0.423419in}}%
\pgfpathlineto{\pgfqpoint{2.519263in}{0.421882in}}%
\pgfpathlineto{\pgfqpoint{2.563190in}{0.421882in}}%
\pgfpathlineto{\pgfqpoint{2.563190in}{0.417391in}}%
\pgfusepath{stroke}%
\end{pgfscope}%
\begin{pgfscope}%
\pgfsetrectcap%
\pgfsetmiterjoin%
\pgfsetlinewidth{1.003750pt}%
\definecolor{currentstroke}{rgb}{0.000000,0.000000,0.000000}%
\pgfsetstrokecolor{currentstroke}%
\pgfsetdash{}{0pt}%
\pgfpathmoveto{\pgfqpoint{0.366840in}{1.637544in}}%
\pgfpathlineto{\pgfqpoint{2.563190in}{1.637544in}}%
\pgfusepath{stroke}%
\end{pgfscope}%
\begin{pgfscope}%
\pgfsetrectcap%
\pgfsetmiterjoin%
\pgfsetlinewidth{1.003750pt}%
\definecolor{currentstroke}{rgb}{0.000000,0.000000,0.000000}%
\pgfsetstrokecolor{currentstroke}%
\pgfsetdash{}{0pt}%
\pgfpathmoveto{\pgfqpoint{2.563190in}{0.417391in}}%
\pgfpathlineto{\pgfqpoint{2.563190in}{1.637544in}}%
\pgfusepath{stroke}%
\end{pgfscope}%
\begin{pgfscope}%
\pgfsetrectcap%
\pgfsetmiterjoin%
\pgfsetlinewidth{1.003750pt}%
\definecolor{currentstroke}{rgb}{0.000000,0.000000,0.000000}%
\pgfsetstrokecolor{currentstroke}%
\pgfsetdash{}{0pt}%
\pgfpathmoveto{\pgfqpoint{0.366840in}{0.417391in}}%
\pgfpathlineto{\pgfqpoint{2.563190in}{0.417391in}}%
\pgfusepath{stroke}%
\end{pgfscope}%
\begin{pgfscope}%
\pgfsetrectcap%
\pgfsetmiterjoin%
\pgfsetlinewidth{1.003750pt}%
\definecolor{currentstroke}{rgb}{0.000000,0.000000,0.000000}%
\pgfsetstrokecolor{currentstroke}%
\pgfsetdash{}{0pt}%
\pgfpathmoveto{\pgfqpoint{0.366840in}{0.417391in}}%
\pgfpathlineto{\pgfqpoint{0.366840in}{1.637544in}}%
\pgfusepath{stroke}%
\end{pgfscope}%
\begin{pgfscope}%
\pgfsetbuttcap%
\pgfsetroundjoin%
\definecolor{currentfill}{rgb}{0.000000,0.000000,0.000000}%
\pgfsetfillcolor{currentfill}%
\pgfsetlinewidth{0.501875pt}%
\definecolor{currentstroke}{rgb}{0.000000,0.000000,0.000000}%
\pgfsetstrokecolor{currentstroke}%
\pgfsetdash{}{0pt}%
\pgfsys@defobject{currentmarker}{\pgfqpoint{0.000000in}{0.000000in}}{\pgfqpoint{0.000000in}{0.069444in}}{%
\pgfpathmoveto{\pgfqpoint{0.000000in}{0.000000in}}%
\pgfpathlineto{\pgfqpoint{0.000000in}{0.069444in}}%
\pgfusepath{stroke,fill}%
}%
\begin{pgfscope}%
\pgfsys@transformshift{0.366840in}{0.417391in}%
\pgfsys@useobject{currentmarker}{}%
\end{pgfscope}%
\end{pgfscope}%
\begin{pgfscope}%
\pgfsetbuttcap%
\pgfsetroundjoin%
\definecolor{currentfill}{rgb}{0.000000,0.000000,0.000000}%
\pgfsetfillcolor{currentfill}%
\pgfsetlinewidth{0.501875pt}%
\definecolor{currentstroke}{rgb}{0.000000,0.000000,0.000000}%
\pgfsetstrokecolor{currentstroke}%
\pgfsetdash{}{0pt}%
\pgfsys@defobject{currentmarker}{\pgfqpoint{0.000000in}{-0.069444in}}{\pgfqpoint{0.000000in}{0.000000in}}{%
\pgfpathmoveto{\pgfqpoint{0.000000in}{0.000000in}}%
\pgfpathlineto{\pgfqpoint{0.000000in}{-0.069444in}}%
\pgfusepath{stroke,fill}%
}%
\begin{pgfscope}%
\pgfsys@transformshift{0.366840in}{1.637544in}%
\pgfsys@useobject{currentmarker}{}%
\end{pgfscope}%
\end{pgfscope}%
\begin{pgfscope}%
\pgftext[x=0.366840in,y=0.347947in,,top]{\rmfamily\fontsize{8.000000}{9.600000}\selectfont 0}%
\end{pgfscope}%
\begin{pgfscope}%
\pgfsetbuttcap%
\pgfsetroundjoin%
\definecolor{currentfill}{rgb}{0.000000,0.000000,0.000000}%
\pgfsetfillcolor{currentfill}%
\pgfsetlinewidth{0.501875pt}%
\definecolor{currentstroke}{rgb}{0.000000,0.000000,0.000000}%
\pgfsetstrokecolor{currentstroke}%
\pgfsetdash{}{0pt}%
\pgfsys@defobject{currentmarker}{\pgfqpoint{0.000000in}{0.000000in}}{\pgfqpoint{0.000000in}{0.069444in}}{%
\pgfpathmoveto{\pgfqpoint{0.000000in}{0.000000in}}%
\pgfpathlineto{\pgfqpoint{0.000000in}{0.069444in}}%
\pgfusepath{stroke,fill}%
}%
\begin{pgfscope}%
\pgfsys@transformshift{0.806110in}{0.417391in}%
\pgfsys@useobject{currentmarker}{}%
\end{pgfscope}%
\end{pgfscope}%
\begin{pgfscope}%
\pgfsetbuttcap%
\pgfsetroundjoin%
\definecolor{currentfill}{rgb}{0.000000,0.000000,0.000000}%
\pgfsetfillcolor{currentfill}%
\pgfsetlinewidth{0.501875pt}%
\definecolor{currentstroke}{rgb}{0.000000,0.000000,0.000000}%
\pgfsetstrokecolor{currentstroke}%
\pgfsetdash{}{0pt}%
\pgfsys@defobject{currentmarker}{\pgfqpoint{0.000000in}{-0.069444in}}{\pgfqpoint{0.000000in}{0.000000in}}{%
\pgfpathmoveto{\pgfqpoint{0.000000in}{0.000000in}}%
\pgfpathlineto{\pgfqpoint{0.000000in}{-0.069444in}}%
\pgfusepath{stroke,fill}%
}%
\begin{pgfscope}%
\pgfsys@transformshift{0.806110in}{1.637544in}%
\pgfsys@useobject{currentmarker}{}%
\end{pgfscope}%
\end{pgfscope}%
\begin{pgfscope}%
\pgftext[x=0.806110in,y=0.347947in,,top]{\rmfamily\fontsize{8.000000}{9.600000}\selectfont 2}%
\end{pgfscope}%
\begin{pgfscope}%
\pgfsetbuttcap%
\pgfsetroundjoin%
\definecolor{currentfill}{rgb}{0.000000,0.000000,0.000000}%
\pgfsetfillcolor{currentfill}%
\pgfsetlinewidth{0.501875pt}%
\definecolor{currentstroke}{rgb}{0.000000,0.000000,0.000000}%
\pgfsetstrokecolor{currentstroke}%
\pgfsetdash{}{0pt}%
\pgfsys@defobject{currentmarker}{\pgfqpoint{0.000000in}{0.000000in}}{\pgfqpoint{0.000000in}{0.069444in}}{%
\pgfpathmoveto{\pgfqpoint{0.000000in}{0.000000in}}%
\pgfpathlineto{\pgfqpoint{0.000000in}{0.069444in}}%
\pgfusepath{stroke,fill}%
}%
\begin{pgfscope}%
\pgfsys@transformshift{1.245380in}{0.417391in}%
\pgfsys@useobject{currentmarker}{}%
\end{pgfscope}%
\end{pgfscope}%
\begin{pgfscope}%
\pgfsetbuttcap%
\pgfsetroundjoin%
\definecolor{currentfill}{rgb}{0.000000,0.000000,0.000000}%
\pgfsetfillcolor{currentfill}%
\pgfsetlinewidth{0.501875pt}%
\definecolor{currentstroke}{rgb}{0.000000,0.000000,0.000000}%
\pgfsetstrokecolor{currentstroke}%
\pgfsetdash{}{0pt}%
\pgfsys@defobject{currentmarker}{\pgfqpoint{0.000000in}{-0.069444in}}{\pgfqpoint{0.000000in}{0.000000in}}{%
\pgfpathmoveto{\pgfqpoint{0.000000in}{0.000000in}}%
\pgfpathlineto{\pgfqpoint{0.000000in}{-0.069444in}}%
\pgfusepath{stroke,fill}%
}%
\begin{pgfscope}%
\pgfsys@transformshift{1.245380in}{1.637544in}%
\pgfsys@useobject{currentmarker}{}%
\end{pgfscope}%
\end{pgfscope}%
\begin{pgfscope}%
\pgftext[x=1.245380in,y=0.347947in,,top]{\rmfamily\fontsize{8.000000}{9.600000}\selectfont 4}%
\end{pgfscope}%
\begin{pgfscope}%
\pgfsetbuttcap%
\pgfsetroundjoin%
\definecolor{currentfill}{rgb}{0.000000,0.000000,0.000000}%
\pgfsetfillcolor{currentfill}%
\pgfsetlinewidth{0.501875pt}%
\definecolor{currentstroke}{rgb}{0.000000,0.000000,0.000000}%
\pgfsetstrokecolor{currentstroke}%
\pgfsetdash{}{0pt}%
\pgfsys@defobject{currentmarker}{\pgfqpoint{0.000000in}{0.000000in}}{\pgfqpoint{0.000000in}{0.069444in}}{%
\pgfpathmoveto{\pgfqpoint{0.000000in}{0.000000in}}%
\pgfpathlineto{\pgfqpoint{0.000000in}{0.069444in}}%
\pgfusepath{stroke,fill}%
}%
\begin{pgfscope}%
\pgfsys@transformshift{1.684650in}{0.417391in}%
\pgfsys@useobject{currentmarker}{}%
\end{pgfscope}%
\end{pgfscope}%
\begin{pgfscope}%
\pgfsetbuttcap%
\pgfsetroundjoin%
\definecolor{currentfill}{rgb}{0.000000,0.000000,0.000000}%
\pgfsetfillcolor{currentfill}%
\pgfsetlinewidth{0.501875pt}%
\definecolor{currentstroke}{rgb}{0.000000,0.000000,0.000000}%
\pgfsetstrokecolor{currentstroke}%
\pgfsetdash{}{0pt}%
\pgfsys@defobject{currentmarker}{\pgfqpoint{0.000000in}{-0.069444in}}{\pgfqpoint{0.000000in}{0.000000in}}{%
\pgfpathmoveto{\pgfqpoint{0.000000in}{0.000000in}}%
\pgfpathlineto{\pgfqpoint{0.000000in}{-0.069444in}}%
\pgfusepath{stroke,fill}%
}%
\begin{pgfscope}%
\pgfsys@transformshift{1.684650in}{1.637544in}%
\pgfsys@useobject{currentmarker}{}%
\end{pgfscope}%
\end{pgfscope}%
\begin{pgfscope}%
\pgftext[x=1.684650in,y=0.347947in,,top]{\rmfamily\fontsize{8.000000}{9.600000}\selectfont 6}%
\end{pgfscope}%
\begin{pgfscope}%
\pgfsetbuttcap%
\pgfsetroundjoin%
\definecolor{currentfill}{rgb}{0.000000,0.000000,0.000000}%
\pgfsetfillcolor{currentfill}%
\pgfsetlinewidth{0.501875pt}%
\definecolor{currentstroke}{rgb}{0.000000,0.000000,0.000000}%
\pgfsetstrokecolor{currentstroke}%
\pgfsetdash{}{0pt}%
\pgfsys@defobject{currentmarker}{\pgfqpoint{0.000000in}{0.000000in}}{\pgfqpoint{0.000000in}{0.069444in}}{%
\pgfpathmoveto{\pgfqpoint{0.000000in}{0.000000in}}%
\pgfpathlineto{\pgfqpoint{0.000000in}{0.069444in}}%
\pgfusepath{stroke,fill}%
}%
\begin{pgfscope}%
\pgfsys@transformshift{2.123920in}{0.417391in}%
\pgfsys@useobject{currentmarker}{}%
\end{pgfscope}%
\end{pgfscope}%
\begin{pgfscope}%
\pgfsetbuttcap%
\pgfsetroundjoin%
\definecolor{currentfill}{rgb}{0.000000,0.000000,0.000000}%
\pgfsetfillcolor{currentfill}%
\pgfsetlinewidth{0.501875pt}%
\definecolor{currentstroke}{rgb}{0.000000,0.000000,0.000000}%
\pgfsetstrokecolor{currentstroke}%
\pgfsetdash{}{0pt}%
\pgfsys@defobject{currentmarker}{\pgfqpoint{0.000000in}{-0.069444in}}{\pgfqpoint{0.000000in}{0.000000in}}{%
\pgfpathmoveto{\pgfqpoint{0.000000in}{0.000000in}}%
\pgfpathlineto{\pgfqpoint{0.000000in}{-0.069444in}}%
\pgfusepath{stroke,fill}%
}%
\begin{pgfscope}%
\pgfsys@transformshift{2.123920in}{1.637544in}%
\pgfsys@useobject{currentmarker}{}%
\end{pgfscope}%
\end{pgfscope}%
\begin{pgfscope}%
\pgftext[x=2.123920in,y=0.347947in,,top]{\rmfamily\fontsize{8.000000}{9.600000}\selectfont 8}%
\end{pgfscope}%
\begin{pgfscope}%
\pgfsetbuttcap%
\pgfsetroundjoin%
\definecolor{currentfill}{rgb}{0.000000,0.000000,0.000000}%
\pgfsetfillcolor{currentfill}%
\pgfsetlinewidth{0.501875pt}%
\definecolor{currentstroke}{rgb}{0.000000,0.000000,0.000000}%
\pgfsetstrokecolor{currentstroke}%
\pgfsetdash{}{0pt}%
\pgfsys@defobject{currentmarker}{\pgfqpoint{0.000000in}{0.000000in}}{\pgfqpoint{0.000000in}{0.069444in}}{%
\pgfpathmoveto{\pgfqpoint{0.000000in}{0.000000in}}%
\pgfpathlineto{\pgfqpoint{0.000000in}{0.069444in}}%
\pgfusepath{stroke,fill}%
}%
\begin{pgfscope}%
\pgfsys@transformshift{2.563190in}{0.417391in}%
\pgfsys@useobject{currentmarker}{}%
\end{pgfscope}%
\end{pgfscope}%
\begin{pgfscope}%
\pgfsetbuttcap%
\pgfsetroundjoin%
\definecolor{currentfill}{rgb}{0.000000,0.000000,0.000000}%
\pgfsetfillcolor{currentfill}%
\pgfsetlinewidth{0.501875pt}%
\definecolor{currentstroke}{rgb}{0.000000,0.000000,0.000000}%
\pgfsetstrokecolor{currentstroke}%
\pgfsetdash{}{0pt}%
\pgfsys@defobject{currentmarker}{\pgfqpoint{0.000000in}{-0.069444in}}{\pgfqpoint{0.000000in}{0.000000in}}{%
\pgfpathmoveto{\pgfqpoint{0.000000in}{0.000000in}}%
\pgfpathlineto{\pgfqpoint{0.000000in}{-0.069444in}}%
\pgfusepath{stroke,fill}%
}%
\begin{pgfscope}%
\pgfsys@transformshift{2.563190in}{1.637544in}%
\pgfsys@useobject{currentmarker}{}%
\end{pgfscope}%
\end{pgfscope}%
\begin{pgfscope}%
\pgftext[x=2.563190in,y=0.347947in,,top]{\rmfamily\fontsize{8.000000}{9.600000}\selectfont 10}%
\end{pgfscope}%
\begin{pgfscope}%
\pgftext[x=1.465015in,y=0.170972in,,top]{\rmfamily\fontsize{9.000000}{10.800000}\selectfont \(\displaystyle t_{B^0}\)}%
\end{pgfscope}%
\begin{pgfscope}%
\pgfsetbuttcap%
\pgfsetroundjoin%
\definecolor{currentfill}{rgb}{0.000000,0.000000,0.000000}%
\pgfsetfillcolor{currentfill}%
\pgfsetlinewidth{0.501875pt}%
\definecolor{currentstroke}{rgb}{0.000000,0.000000,0.000000}%
\pgfsetstrokecolor{currentstroke}%
\pgfsetdash{}{0pt}%
\pgfsys@defobject{currentmarker}{\pgfqpoint{0.000000in}{0.000000in}}{\pgfqpoint{0.069444in}{0.000000in}}{%
\pgfpathmoveto{\pgfqpoint{0.000000in}{0.000000in}}%
\pgfpathlineto{\pgfqpoint{0.069444in}{0.000000in}}%
\pgfusepath{stroke,fill}%
}%
\begin{pgfscope}%
\pgfsys@transformshift{0.366840in}{0.417391in}%
\pgfsys@useobject{currentmarker}{}%
\end{pgfscope}%
\end{pgfscope}%
\begin{pgfscope}%
\pgfsetbuttcap%
\pgfsetroundjoin%
\definecolor{currentfill}{rgb}{0.000000,0.000000,0.000000}%
\pgfsetfillcolor{currentfill}%
\pgfsetlinewidth{0.501875pt}%
\definecolor{currentstroke}{rgb}{0.000000,0.000000,0.000000}%
\pgfsetstrokecolor{currentstroke}%
\pgfsetdash{}{0pt}%
\pgfsys@defobject{currentmarker}{\pgfqpoint{-0.069444in}{0.000000in}}{\pgfqpoint{0.000000in}{0.000000in}}{%
\pgfpathmoveto{\pgfqpoint{0.000000in}{0.000000in}}%
\pgfpathlineto{\pgfqpoint{-0.069444in}{0.000000in}}%
\pgfusepath{stroke,fill}%
}%
\begin{pgfscope}%
\pgfsys@transformshift{2.563190in}{0.417391in}%
\pgfsys@useobject{currentmarker}{}%
\end{pgfscope}%
\end{pgfscope}%
\begin{pgfscope}%
\pgftext[x=0.297396in,y=0.417391in,right,]{\rmfamily\fontsize{8.000000}{9.600000}\selectfont 0.00}%
\end{pgfscope}%
\begin{pgfscope}%
\pgfsetbuttcap%
\pgfsetroundjoin%
\definecolor{currentfill}{rgb}{0.000000,0.000000,0.000000}%
\pgfsetfillcolor{currentfill}%
\pgfsetlinewidth{0.501875pt}%
\definecolor{currentstroke}{rgb}{0.000000,0.000000,0.000000}%
\pgfsetstrokecolor{currentstroke}%
\pgfsetdash{}{0pt}%
\pgfsys@defobject{currentmarker}{\pgfqpoint{0.000000in}{0.000000in}}{\pgfqpoint{0.069444in}{0.000000in}}{%
\pgfpathmoveto{\pgfqpoint{0.000000in}{0.000000in}}%
\pgfpathlineto{\pgfqpoint{0.069444in}{0.000000in}}%
\pgfusepath{stroke,fill}%
}%
\begin{pgfscope}%
\pgfsys@transformshift{0.366840in}{0.569910in}%
\pgfsys@useobject{currentmarker}{}%
\end{pgfscope}%
\end{pgfscope}%
\begin{pgfscope}%
\pgfsetbuttcap%
\pgfsetroundjoin%
\definecolor{currentfill}{rgb}{0.000000,0.000000,0.000000}%
\pgfsetfillcolor{currentfill}%
\pgfsetlinewidth{0.501875pt}%
\definecolor{currentstroke}{rgb}{0.000000,0.000000,0.000000}%
\pgfsetstrokecolor{currentstroke}%
\pgfsetdash{}{0pt}%
\pgfsys@defobject{currentmarker}{\pgfqpoint{-0.069444in}{0.000000in}}{\pgfqpoint{0.000000in}{0.000000in}}{%
\pgfpathmoveto{\pgfqpoint{0.000000in}{0.000000in}}%
\pgfpathlineto{\pgfqpoint{-0.069444in}{0.000000in}}%
\pgfusepath{stroke,fill}%
}%
\begin{pgfscope}%
\pgfsys@transformshift{2.563190in}{0.569910in}%
\pgfsys@useobject{currentmarker}{}%
\end{pgfscope}%
\end{pgfscope}%
\begin{pgfscope}%
\pgftext[x=0.297396in,y=0.569910in,right,]{\rmfamily\fontsize{8.000000}{9.600000}\selectfont 0.05}%
\end{pgfscope}%
\begin{pgfscope}%
\pgfsetbuttcap%
\pgfsetroundjoin%
\definecolor{currentfill}{rgb}{0.000000,0.000000,0.000000}%
\pgfsetfillcolor{currentfill}%
\pgfsetlinewidth{0.501875pt}%
\definecolor{currentstroke}{rgb}{0.000000,0.000000,0.000000}%
\pgfsetstrokecolor{currentstroke}%
\pgfsetdash{}{0pt}%
\pgfsys@defobject{currentmarker}{\pgfqpoint{0.000000in}{0.000000in}}{\pgfqpoint{0.069444in}{0.000000in}}{%
\pgfpathmoveto{\pgfqpoint{0.000000in}{0.000000in}}%
\pgfpathlineto{\pgfqpoint{0.069444in}{0.000000in}}%
\pgfusepath{stroke,fill}%
}%
\begin{pgfscope}%
\pgfsys@transformshift{0.366840in}{0.722429in}%
\pgfsys@useobject{currentmarker}{}%
\end{pgfscope}%
\end{pgfscope}%
\begin{pgfscope}%
\pgfsetbuttcap%
\pgfsetroundjoin%
\definecolor{currentfill}{rgb}{0.000000,0.000000,0.000000}%
\pgfsetfillcolor{currentfill}%
\pgfsetlinewidth{0.501875pt}%
\definecolor{currentstroke}{rgb}{0.000000,0.000000,0.000000}%
\pgfsetstrokecolor{currentstroke}%
\pgfsetdash{}{0pt}%
\pgfsys@defobject{currentmarker}{\pgfqpoint{-0.069444in}{0.000000in}}{\pgfqpoint{0.000000in}{0.000000in}}{%
\pgfpathmoveto{\pgfqpoint{0.000000in}{0.000000in}}%
\pgfpathlineto{\pgfqpoint{-0.069444in}{0.000000in}}%
\pgfusepath{stroke,fill}%
}%
\begin{pgfscope}%
\pgfsys@transformshift{2.563190in}{0.722429in}%
\pgfsys@useobject{currentmarker}{}%
\end{pgfscope}%
\end{pgfscope}%
\begin{pgfscope}%
\pgftext[x=0.297396in,y=0.722429in,right,]{\rmfamily\fontsize{8.000000}{9.600000}\selectfont 0.10}%
\end{pgfscope}%
\begin{pgfscope}%
\pgfsetbuttcap%
\pgfsetroundjoin%
\definecolor{currentfill}{rgb}{0.000000,0.000000,0.000000}%
\pgfsetfillcolor{currentfill}%
\pgfsetlinewidth{0.501875pt}%
\definecolor{currentstroke}{rgb}{0.000000,0.000000,0.000000}%
\pgfsetstrokecolor{currentstroke}%
\pgfsetdash{}{0pt}%
\pgfsys@defobject{currentmarker}{\pgfqpoint{0.000000in}{0.000000in}}{\pgfqpoint{0.069444in}{0.000000in}}{%
\pgfpathmoveto{\pgfqpoint{0.000000in}{0.000000in}}%
\pgfpathlineto{\pgfqpoint{0.069444in}{0.000000in}}%
\pgfusepath{stroke,fill}%
}%
\begin{pgfscope}%
\pgfsys@transformshift{0.366840in}{0.874948in}%
\pgfsys@useobject{currentmarker}{}%
\end{pgfscope}%
\end{pgfscope}%
\begin{pgfscope}%
\pgfsetbuttcap%
\pgfsetroundjoin%
\definecolor{currentfill}{rgb}{0.000000,0.000000,0.000000}%
\pgfsetfillcolor{currentfill}%
\pgfsetlinewidth{0.501875pt}%
\definecolor{currentstroke}{rgb}{0.000000,0.000000,0.000000}%
\pgfsetstrokecolor{currentstroke}%
\pgfsetdash{}{0pt}%
\pgfsys@defobject{currentmarker}{\pgfqpoint{-0.069444in}{0.000000in}}{\pgfqpoint{0.000000in}{0.000000in}}{%
\pgfpathmoveto{\pgfqpoint{0.000000in}{0.000000in}}%
\pgfpathlineto{\pgfqpoint{-0.069444in}{0.000000in}}%
\pgfusepath{stroke,fill}%
}%
\begin{pgfscope}%
\pgfsys@transformshift{2.563190in}{0.874948in}%
\pgfsys@useobject{currentmarker}{}%
\end{pgfscope}%
\end{pgfscope}%
\begin{pgfscope}%
\pgftext[x=0.297396in,y=0.874948in,right,]{\rmfamily\fontsize{8.000000}{9.600000}\selectfont 0.15}%
\end{pgfscope}%
\begin{pgfscope}%
\pgfsetbuttcap%
\pgfsetroundjoin%
\definecolor{currentfill}{rgb}{0.000000,0.000000,0.000000}%
\pgfsetfillcolor{currentfill}%
\pgfsetlinewidth{0.501875pt}%
\definecolor{currentstroke}{rgb}{0.000000,0.000000,0.000000}%
\pgfsetstrokecolor{currentstroke}%
\pgfsetdash{}{0pt}%
\pgfsys@defobject{currentmarker}{\pgfqpoint{0.000000in}{0.000000in}}{\pgfqpoint{0.069444in}{0.000000in}}{%
\pgfpathmoveto{\pgfqpoint{0.000000in}{0.000000in}}%
\pgfpathlineto{\pgfqpoint{0.069444in}{0.000000in}}%
\pgfusepath{stroke,fill}%
}%
\begin{pgfscope}%
\pgfsys@transformshift{0.366840in}{1.027467in}%
\pgfsys@useobject{currentmarker}{}%
\end{pgfscope}%
\end{pgfscope}%
\begin{pgfscope}%
\pgfsetbuttcap%
\pgfsetroundjoin%
\definecolor{currentfill}{rgb}{0.000000,0.000000,0.000000}%
\pgfsetfillcolor{currentfill}%
\pgfsetlinewidth{0.501875pt}%
\definecolor{currentstroke}{rgb}{0.000000,0.000000,0.000000}%
\pgfsetstrokecolor{currentstroke}%
\pgfsetdash{}{0pt}%
\pgfsys@defobject{currentmarker}{\pgfqpoint{-0.069444in}{0.000000in}}{\pgfqpoint{0.000000in}{0.000000in}}{%
\pgfpathmoveto{\pgfqpoint{0.000000in}{0.000000in}}%
\pgfpathlineto{\pgfqpoint{-0.069444in}{0.000000in}}%
\pgfusepath{stroke,fill}%
}%
\begin{pgfscope}%
\pgfsys@transformshift{2.563190in}{1.027467in}%
\pgfsys@useobject{currentmarker}{}%
\end{pgfscope}%
\end{pgfscope}%
\begin{pgfscope}%
\pgftext[x=0.297396in,y=1.027467in,right,]{\rmfamily\fontsize{8.000000}{9.600000}\selectfont 0.20}%
\end{pgfscope}%
\begin{pgfscope}%
\pgfsetbuttcap%
\pgfsetroundjoin%
\definecolor{currentfill}{rgb}{0.000000,0.000000,0.000000}%
\pgfsetfillcolor{currentfill}%
\pgfsetlinewidth{0.501875pt}%
\definecolor{currentstroke}{rgb}{0.000000,0.000000,0.000000}%
\pgfsetstrokecolor{currentstroke}%
\pgfsetdash{}{0pt}%
\pgfsys@defobject{currentmarker}{\pgfqpoint{0.000000in}{0.000000in}}{\pgfqpoint{0.069444in}{0.000000in}}{%
\pgfpathmoveto{\pgfqpoint{0.000000in}{0.000000in}}%
\pgfpathlineto{\pgfqpoint{0.069444in}{0.000000in}}%
\pgfusepath{stroke,fill}%
}%
\begin{pgfscope}%
\pgfsys@transformshift{0.366840in}{1.179987in}%
\pgfsys@useobject{currentmarker}{}%
\end{pgfscope}%
\end{pgfscope}%
\begin{pgfscope}%
\pgfsetbuttcap%
\pgfsetroundjoin%
\definecolor{currentfill}{rgb}{0.000000,0.000000,0.000000}%
\pgfsetfillcolor{currentfill}%
\pgfsetlinewidth{0.501875pt}%
\definecolor{currentstroke}{rgb}{0.000000,0.000000,0.000000}%
\pgfsetstrokecolor{currentstroke}%
\pgfsetdash{}{0pt}%
\pgfsys@defobject{currentmarker}{\pgfqpoint{-0.069444in}{0.000000in}}{\pgfqpoint{0.000000in}{0.000000in}}{%
\pgfpathmoveto{\pgfqpoint{0.000000in}{0.000000in}}%
\pgfpathlineto{\pgfqpoint{-0.069444in}{0.000000in}}%
\pgfusepath{stroke,fill}%
}%
\begin{pgfscope}%
\pgfsys@transformshift{2.563190in}{1.179987in}%
\pgfsys@useobject{currentmarker}{}%
\end{pgfscope}%
\end{pgfscope}%
\begin{pgfscope}%
\pgftext[x=0.297396in,y=1.179987in,right,]{\rmfamily\fontsize{8.000000}{9.600000}\selectfont 0.25}%
\end{pgfscope}%
\begin{pgfscope}%
\pgfsetbuttcap%
\pgfsetroundjoin%
\definecolor{currentfill}{rgb}{0.000000,0.000000,0.000000}%
\pgfsetfillcolor{currentfill}%
\pgfsetlinewidth{0.501875pt}%
\definecolor{currentstroke}{rgb}{0.000000,0.000000,0.000000}%
\pgfsetstrokecolor{currentstroke}%
\pgfsetdash{}{0pt}%
\pgfsys@defobject{currentmarker}{\pgfqpoint{0.000000in}{0.000000in}}{\pgfqpoint{0.069444in}{0.000000in}}{%
\pgfpathmoveto{\pgfqpoint{0.000000in}{0.000000in}}%
\pgfpathlineto{\pgfqpoint{0.069444in}{0.000000in}}%
\pgfusepath{stroke,fill}%
}%
\begin{pgfscope}%
\pgfsys@transformshift{0.366840in}{1.332506in}%
\pgfsys@useobject{currentmarker}{}%
\end{pgfscope}%
\end{pgfscope}%
\begin{pgfscope}%
\pgfsetbuttcap%
\pgfsetroundjoin%
\definecolor{currentfill}{rgb}{0.000000,0.000000,0.000000}%
\pgfsetfillcolor{currentfill}%
\pgfsetlinewidth{0.501875pt}%
\definecolor{currentstroke}{rgb}{0.000000,0.000000,0.000000}%
\pgfsetstrokecolor{currentstroke}%
\pgfsetdash{}{0pt}%
\pgfsys@defobject{currentmarker}{\pgfqpoint{-0.069444in}{0.000000in}}{\pgfqpoint{0.000000in}{0.000000in}}{%
\pgfpathmoveto{\pgfqpoint{0.000000in}{0.000000in}}%
\pgfpathlineto{\pgfqpoint{-0.069444in}{0.000000in}}%
\pgfusepath{stroke,fill}%
}%
\begin{pgfscope}%
\pgfsys@transformshift{2.563190in}{1.332506in}%
\pgfsys@useobject{currentmarker}{}%
\end{pgfscope}%
\end{pgfscope}%
\begin{pgfscope}%
\pgftext[x=0.297396in,y=1.332506in,right,]{\rmfamily\fontsize{8.000000}{9.600000}\selectfont 0.30}%
\end{pgfscope}%
\begin{pgfscope}%
\pgfsetbuttcap%
\pgfsetroundjoin%
\definecolor{currentfill}{rgb}{0.000000,0.000000,0.000000}%
\pgfsetfillcolor{currentfill}%
\pgfsetlinewidth{0.501875pt}%
\definecolor{currentstroke}{rgb}{0.000000,0.000000,0.000000}%
\pgfsetstrokecolor{currentstroke}%
\pgfsetdash{}{0pt}%
\pgfsys@defobject{currentmarker}{\pgfqpoint{0.000000in}{0.000000in}}{\pgfqpoint{0.069444in}{0.000000in}}{%
\pgfpathmoveto{\pgfqpoint{0.000000in}{0.000000in}}%
\pgfpathlineto{\pgfqpoint{0.069444in}{0.000000in}}%
\pgfusepath{stroke,fill}%
}%
\begin{pgfscope}%
\pgfsys@transformshift{0.366840in}{1.485025in}%
\pgfsys@useobject{currentmarker}{}%
\end{pgfscope}%
\end{pgfscope}%
\begin{pgfscope}%
\pgfsetbuttcap%
\pgfsetroundjoin%
\definecolor{currentfill}{rgb}{0.000000,0.000000,0.000000}%
\pgfsetfillcolor{currentfill}%
\pgfsetlinewidth{0.501875pt}%
\definecolor{currentstroke}{rgb}{0.000000,0.000000,0.000000}%
\pgfsetstrokecolor{currentstroke}%
\pgfsetdash{}{0pt}%
\pgfsys@defobject{currentmarker}{\pgfqpoint{-0.069444in}{0.000000in}}{\pgfqpoint{0.000000in}{0.000000in}}{%
\pgfpathmoveto{\pgfqpoint{0.000000in}{0.000000in}}%
\pgfpathlineto{\pgfqpoint{-0.069444in}{0.000000in}}%
\pgfusepath{stroke,fill}%
}%
\begin{pgfscope}%
\pgfsys@transformshift{2.563190in}{1.485025in}%
\pgfsys@useobject{currentmarker}{}%
\end{pgfscope}%
\end{pgfscope}%
\begin{pgfscope}%
\pgftext[x=0.297396in,y=1.485025in,right,]{\rmfamily\fontsize{8.000000}{9.600000}\selectfont 0.35}%
\end{pgfscope}%
\begin{pgfscope}%
\pgfsetbuttcap%
\pgfsetroundjoin%
\definecolor{currentfill}{rgb}{0.000000,0.000000,0.000000}%
\pgfsetfillcolor{currentfill}%
\pgfsetlinewidth{0.501875pt}%
\definecolor{currentstroke}{rgb}{0.000000,0.000000,0.000000}%
\pgfsetstrokecolor{currentstroke}%
\pgfsetdash{}{0pt}%
\pgfsys@defobject{currentmarker}{\pgfqpoint{0.000000in}{0.000000in}}{\pgfqpoint{0.069444in}{0.000000in}}{%
\pgfpathmoveto{\pgfqpoint{0.000000in}{0.000000in}}%
\pgfpathlineto{\pgfqpoint{0.069444in}{0.000000in}}%
\pgfusepath{stroke,fill}%
}%
\begin{pgfscope}%
\pgfsys@transformshift{0.366840in}{1.637544in}%
\pgfsys@useobject{currentmarker}{}%
\end{pgfscope}%
\end{pgfscope}%
\begin{pgfscope}%
\pgfsetbuttcap%
\pgfsetroundjoin%
\definecolor{currentfill}{rgb}{0.000000,0.000000,0.000000}%
\pgfsetfillcolor{currentfill}%
\pgfsetlinewidth{0.501875pt}%
\definecolor{currentstroke}{rgb}{0.000000,0.000000,0.000000}%
\pgfsetstrokecolor{currentstroke}%
\pgfsetdash{}{0pt}%
\pgfsys@defobject{currentmarker}{\pgfqpoint{-0.069444in}{0.000000in}}{\pgfqpoint{0.000000in}{0.000000in}}{%
\pgfpathmoveto{\pgfqpoint{0.000000in}{0.000000in}}%
\pgfpathlineto{\pgfqpoint{-0.069444in}{0.000000in}}%
\pgfusepath{stroke,fill}%
}%
\begin{pgfscope}%
\pgfsys@transformshift{2.563190in}{1.637544in}%
\pgfsys@useobject{currentmarker}{}%
\end{pgfscope}%
\end{pgfscope}%
\begin{pgfscope}%
\pgftext[x=0.297396in,y=1.637544in,right,]{\rmfamily\fontsize{8.000000}{9.600000}\selectfont 0.40}%
\end{pgfscope}%
\end{pgfpicture}%
\makeatother%
\endgroup%

	\end{subfigure}

	\begin{subfigure}[t]{0.49\textwidth}
		\centering
    %\includegraphics[width=\textwidth]{store/variables/DATA_MC_REWEIGHTED_Kplus_PIDK.pdf}
    \input{store/variables/DATA_MC_REWEIGHT_Kplus_PIDK.pgf}
	\end{subfigure}
	\begin{subfigure}[t]{0.49\textwidth}
		\centering
    %\includegraphics[width=\textwidth]{store/variables/DATA_MC_REWEIGHTED_Kplus_PIDmu.pdf}
    %% Creator: Matplotlib, PGF backend
%%
%% To include the figure in your LaTeX document, write
%%   \input{<filename>.pgf}
%%
%% Make sure the required packages are loaded in your preamble
%%   \usepackage{pgf}
%%
%% Figures using additional raster images can only be included by \input if
%% they are in the same directory as the main LaTeX file. For loading figures
%% from other directories you can use the `import` package
%%   \usepackage{import}
%% and then include the figures with
%%   \import{<path to file>}{<filename>.pgf}
%%
%% Matplotlib used the following preamble
%%   \usepackage{fontspec}
%%   \setmainfont{DejaVu Serif}
%%   \setsansfont{DejaVu Sans}
%%   \setmonofont{DejaVu Sans Mono}
%%
\begingroup%
\makeatletter%
\begin{pgfpicture}%
\pgfpathrectangle{\pgfpointorigin}{\pgfqpoint{2.682342in}{1.719349in}}%
\pgfusepath{use as bounding box, clip}%
\begin{pgfscope}%
\pgfsetbuttcap%
\pgfsetmiterjoin%
\definecolor{currentfill}{rgb}{1.000000,1.000000,1.000000}%
\pgfsetfillcolor{currentfill}%
\pgfsetlinewidth{0.000000pt}%
\definecolor{currentstroke}{rgb}{1.000000,1.000000,1.000000}%
\pgfsetstrokecolor{currentstroke}%
\pgfsetdash{}{0pt}%
\pgfpathmoveto{\pgfqpoint{0.000000in}{0.000000in}}%
\pgfpathlineto{\pgfqpoint{2.682342in}{0.000000in}}%
\pgfpathlineto{\pgfqpoint{2.682342in}{1.719349in}}%
\pgfpathlineto{\pgfqpoint{0.000000in}{1.719349in}}%
\pgfpathclose%
\pgfusepath{fill}%
\end{pgfscope}%
\begin{pgfscope}%
\pgfsetbuttcap%
\pgfsetmiterjoin%
\definecolor{currentfill}{rgb}{1.000000,1.000000,1.000000}%
\pgfsetfillcolor{currentfill}%
\pgfsetlinewidth{0.000000pt}%
\definecolor{currentstroke}{rgb}{0.000000,0.000000,0.000000}%
\pgfsetstrokecolor{currentstroke}%
\pgfsetstrokeopacity{0.000000}%
\pgfsetdash{}{0pt}%
\pgfpathmoveto{\pgfqpoint{0.366840in}{0.449983in}}%
\pgfpathlineto{\pgfqpoint{2.561650in}{0.449983in}}%
\pgfpathlineto{\pgfqpoint{2.561650in}{1.615583in}}%
\pgfpathlineto{\pgfqpoint{0.366840in}{1.615583in}}%
\pgfpathclose%
\pgfusepath{fill}%
\end{pgfscope}%
\begin{pgfscope}%
\pgfpathrectangle{\pgfqpoint{0.366840in}{0.449983in}}{\pgfqpoint{2.194810in}{1.165600in}} %
\pgfusepath{clip}%
\pgfsetbuttcap%
\pgfsetmiterjoin%
\definecolor{currentfill}{rgb}{0.215686,0.470588,0.749020}%
\pgfsetfillcolor{currentfill}%
\pgfsetlinewidth{0.000000pt}%
\definecolor{currentstroke}{rgb}{0.000000,0.000000,0.000000}%
\pgfsetstrokecolor{currentstroke}%
\pgfsetdash{}{0pt}%
\pgfpathmoveto{\pgfqpoint{0.366840in}{0.449983in}}%
\pgfpathlineto{\pgfqpoint{0.366840in}{0.449983in}}%
\pgfpathlineto{\pgfqpoint{0.410736in}{0.449983in}}%
\pgfpathlineto{\pgfqpoint{0.410736in}{0.449983in}}%
\pgfpathlineto{\pgfqpoint{0.454633in}{0.449983in}}%
\pgfpathlineto{\pgfqpoint{0.454633in}{0.449983in}}%
\pgfpathlineto{\pgfqpoint{0.498529in}{0.449983in}}%
\pgfpathlineto{\pgfqpoint{0.498529in}{0.450096in}}%
\pgfpathlineto{\pgfqpoint{0.542425in}{0.450096in}}%
\pgfpathlineto{\pgfqpoint{0.542425in}{0.453752in}}%
\pgfpathlineto{\pgfqpoint{0.586321in}{0.453752in}}%
\pgfpathlineto{\pgfqpoint{0.586321in}{0.471679in}}%
\pgfpathlineto{\pgfqpoint{0.630217in}{0.471679in}}%
\pgfpathlineto{\pgfqpoint{0.630217in}{0.490396in}}%
\pgfpathlineto{\pgfqpoint{0.674114in}{0.490396in}}%
\pgfpathlineto{\pgfqpoint{0.674114in}{0.510763in}}%
\pgfpathlineto{\pgfqpoint{0.718010in}{0.510763in}}%
\pgfpathlineto{\pgfqpoint{0.718010in}{0.535038in}}%
\pgfpathlineto{\pgfqpoint{0.761906in}{0.535038in}}%
\pgfpathlineto{\pgfqpoint{0.761906in}{0.577889in}}%
\pgfpathlineto{\pgfqpoint{0.805802in}{0.577889in}}%
\pgfpathlineto{\pgfqpoint{0.805802in}{0.663550in}}%
\pgfpathlineto{\pgfqpoint{0.849698in}{0.663550in}}%
\pgfpathlineto{\pgfqpoint{0.849698in}{0.759114in}}%
\pgfpathlineto{\pgfqpoint{0.893595in}{0.759114in}}%
\pgfpathlineto{\pgfqpoint{0.893595in}{0.874112in}}%
\pgfpathlineto{\pgfqpoint{0.937491in}{0.874112in}}%
\pgfpathlineto{\pgfqpoint{0.937491in}{1.062169in}}%
\pgfpathlineto{\pgfqpoint{0.981387in}{1.062169in}}%
\pgfpathlineto{\pgfqpoint{0.981387in}{1.190239in}}%
\pgfpathlineto{\pgfqpoint{1.025283in}{1.190239in}}%
\pgfpathlineto{\pgfqpoint{1.025283in}{1.303157in}}%
\pgfpathlineto{\pgfqpoint{1.069179in}{1.303157in}}%
\pgfpathlineto{\pgfqpoint{1.069179in}{1.398944in}}%
\pgfpathlineto{\pgfqpoint{1.113076in}{1.398944in}}%
\pgfpathlineto{\pgfqpoint{1.113076in}{1.500412in}}%
\pgfpathlineto{\pgfqpoint{1.156972in}{1.500412in}}%
\pgfpathlineto{\pgfqpoint{1.156972in}{1.532030in}}%
\pgfpathlineto{\pgfqpoint{1.200868in}{1.532030in}}%
\pgfpathlineto{\pgfqpoint{1.200868in}{1.505774in}}%
\pgfpathlineto{\pgfqpoint{1.244764in}{1.505774in}}%
\pgfpathlineto{\pgfqpoint{1.244764in}{1.471524in}}%
\pgfpathlineto{\pgfqpoint{1.288660in}{1.471524in}}%
\pgfpathlineto{\pgfqpoint{1.288660in}{1.377598in}}%
\pgfpathlineto{\pgfqpoint{1.332556in}{1.377598in}}%
\pgfpathlineto{\pgfqpoint{1.332556in}{1.240507in}}%
\pgfpathlineto{\pgfqpoint{1.376453in}{1.240507in}}%
\pgfpathlineto{\pgfqpoint{1.376453in}{1.133712in}}%
\pgfpathlineto{\pgfqpoint{1.420349in}{1.133712in}}%
\pgfpathlineto{\pgfqpoint{1.420349in}{1.151045in}}%
\pgfpathlineto{\pgfqpoint{1.464245in}{1.151045in}}%
\pgfpathlineto{\pgfqpoint{1.464245in}{0.938404in}}%
\pgfpathlineto{\pgfqpoint{1.508141in}{0.938404in}}%
\pgfpathlineto{\pgfqpoint{1.508141in}{0.816441in}}%
\pgfpathlineto{\pgfqpoint{1.552037in}{0.816441in}}%
\pgfpathlineto{\pgfqpoint{1.552037in}{0.742613in}}%
\pgfpathlineto{\pgfqpoint{1.595934in}{0.742613in}}%
\pgfpathlineto{\pgfqpoint{1.595934in}{0.678168in}}%
\pgfpathlineto{\pgfqpoint{1.639830in}{0.678168in}}%
\pgfpathlineto{\pgfqpoint{1.639830in}{0.631397in}}%
\pgfpathlineto{\pgfqpoint{1.683726in}{0.631397in}}%
\pgfpathlineto{\pgfqpoint{1.683726in}{0.594197in}}%
\pgfpathlineto{\pgfqpoint{1.727622in}{0.594197in}}%
\pgfpathlineto{\pgfqpoint{1.727622in}{0.563087in}}%
\pgfpathlineto{\pgfqpoint{1.771518in}{0.563087in}}%
\pgfpathlineto{\pgfqpoint{1.771518in}{0.539066in}}%
\pgfpathlineto{\pgfqpoint{1.815415in}{0.539066in}}%
\pgfpathlineto{\pgfqpoint{1.815415in}{0.516861in}}%
\pgfpathlineto{\pgfqpoint{1.859311in}{0.516861in}}%
\pgfpathlineto{\pgfqpoint{1.859311in}{0.496995in}}%
\pgfpathlineto{\pgfqpoint{1.903207in}{0.496995in}}%
\pgfpathlineto{\pgfqpoint{1.903207in}{0.488291in}}%
\pgfpathlineto{\pgfqpoint{1.947103in}{0.488291in}}%
\pgfpathlineto{\pgfqpoint{1.947103in}{0.478094in}}%
\pgfpathlineto{\pgfqpoint{1.990999in}{0.478094in}}%
\pgfpathlineto{\pgfqpoint{1.990999in}{0.465711in}}%
\pgfpathlineto{\pgfqpoint{2.034896in}{0.465711in}}%
\pgfpathlineto{\pgfqpoint{2.034896in}{0.461340in}}%
\pgfpathlineto{\pgfqpoint{2.078792in}{0.461340in}}%
\pgfpathlineto{\pgfqpoint{2.078792in}{0.455919in}}%
\pgfpathlineto{\pgfqpoint{2.122688in}{0.455919in}}%
\pgfpathlineto{\pgfqpoint{2.122688in}{0.453039in}}%
\pgfpathlineto{\pgfqpoint{2.166584in}{0.453039in}}%
\pgfpathlineto{\pgfqpoint{2.166584in}{0.451129in}}%
\pgfpathlineto{\pgfqpoint{2.210480in}{0.451129in}}%
\pgfpathlineto{\pgfqpoint{2.210480in}{0.450651in}}%
\pgfpathlineto{\pgfqpoint{2.254377in}{0.450651in}}%
\pgfpathlineto{\pgfqpoint{2.254377in}{0.450316in}}%
\pgfpathlineto{\pgfqpoint{2.298273in}{0.450316in}}%
\pgfpathlineto{\pgfqpoint{2.298273in}{0.450100in}}%
\pgfpathlineto{\pgfqpoint{2.342169in}{0.450100in}}%
\pgfpathlineto{\pgfqpoint{2.342169in}{0.450094in}}%
\pgfpathlineto{\pgfqpoint{2.386065in}{0.450094in}}%
\pgfpathlineto{\pgfqpoint{2.386065in}{0.450002in}}%
\pgfpathlineto{\pgfqpoint{2.429961in}{0.450002in}}%
\pgfpathlineto{\pgfqpoint{2.429961in}{0.450029in}}%
\pgfpathlineto{\pgfqpoint{2.473857in}{0.450029in}}%
\pgfpathlineto{\pgfqpoint{2.473857in}{0.449983in}}%
\pgfpathlineto{\pgfqpoint{2.517754in}{0.449983in}}%
\pgfpathlineto{\pgfqpoint{2.517754in}{0.449983in}}%
\pgfpathlineto{\pgfqpoint{2.561650in}{0.449983in}}%
\pgfpathlineto{\pgfqpoint{2.561650in}{0.449983in}}%
\pgfpathlineto{\pgfqpoint{2.517754in}{0.449983in}}%
\pgfpathlineto{\pgfqpoint{2.517754in}{0.449983in}}%
\pgfpathlineto{\pgfqpoint{2.473857in}{0.449983in}}%
\pgfpathlineto{\pgfqpoint{2.473857in}{0.449983in}}%
\pgfpathlineto{\pgfqpoint{2.429961in}{0.449983in}}%
\pgfpathlineto{\pgfqpoint{2.429961in}{0.449983in}}%
\pgfpathlineto{\pgfqpoint{2.386065in}{0.449983in}}%
\pgfpathlineto{\pgfqpoint{2.386065in}{0.449983in}}%
\pgfpathlineto{\pgfqpoint{2.342169in}{0.449983in}}%
\pgfpathlineto{\pgfqpoint{2.342169in}{0.449983in}}%
\pgfpathlineto{\pgfqpoint{2.298273in}{0.449983in}}%
\pgfpathlineto{\pgfqpoint{2.298273in}{0.449983in}}%
\pgfpathlineto{\pgfqpoint{2.254377in}{0.449983in}}%
\pgfpathlineto{\pgfqpoint{2.254377in}{0.449983in}}%
\pgfpathlineto{\pgfqpoint{2.210480in}{0.449983in}}%
\pgfpathlineto{\pgfqpoint{2.210480in}{0.449983in}}%
\pgfpathlineto{\pgfqpoint{2.166584in}{0.449983in}}%
\pgfpathlineto{\pgfqpoint{2.166584in}{0.449983in}}%
\pgfpathlineto{\pgfqpoint{2.122688in}{0.449983in}}%
\pgfpathlineto{\pgfqpoint{2.122688in}{0.449983in}}%
\pgfpathlineto{\pgfqpoint{2.078792in}{0.449983in}}%
\pgfpathlineto{\pgfqpoint{2.078792in}{0.449983in}}%
\pgfpathlineto{\pgfqpoint{2.034896in}{0.449983in}}%
\pgfpathlineto{\pgfqpoint{2.034896in}{0.449983in}}%
\pgfpathlineto{\pgfqpoint{1.990999in}{0.449983in}}%
\pgfpathlineto{\pgfqpoint{1.990999in}{0.449983in}}%
\pgfpathlineto{\pgfqpoint{1.947103in}{0.449983in}}%
\pgfpathlineto{\pgfqpoint{1.947103in}{0.449983in}}%
\pgfpathlineto{\pgfqpoint{1.903207in}{0.449983in}}%
\pgfpathlineto{\pgfqpoint{1.903207in}{0.449983in}}%
\pgfpathlineto{\pgfqpoint{1.859311in}{0.449983in}}%
\pgfpathlineto{\pgfqpoint{1.859311in}{0.449983in}}%
\pgfpathlineto{\pgfqpoint{1.815415in}{0.449983in}}%
\pgfpathlineto{\pgfqpoint{1.815415in}{0.449983in}}%
\pgfpathlineto{\pgfqpoint{1.771518in}{0.449983in}}%
\pgfpathlineto{\pgfqpoint{1.771518in}{0.449983in}}%
\pgfpathlineto{\pgfqpoint{1.727622in}{0.449983in}}%
\pgfpathlineto{\pgfqpoint{1.727622in}{0.449983in}}%
\pgfpathlineto{\pgfqpoint{1.683726in}{0.449983in}}%
\pgfpathlineto{\pgfqpoint{1.683726in}{0.449983in}}%
\pgfpathlineto{\pgfqpoint{1.639830in}{0.449983in}}%
\pgfpathlineto{\pgfqpoint{1.639830in}{0.449983in}}%
\pgfpathlineto{\pgfqpoint{1.595934in}{0.449983in}}%
\pgfpathlineto{\pgfqpoint{1.595934in}{0.449983in}}%
\pgfpathlineto{\pgfqpoint{1.552037in}{0.449983in}}%
\pgfpathlineto{\pgfqpoint{1.552037in}{0.449983in}}%
\pgfpathlineto{\pgfqpoint{1.508141in}{0.449983in}}%
\pgfpathlineto{\pgfqpoint{1.508141in}{0.449983in}}%
\pgfpathlineto{\pgfqpoint{1.464245in}{0.449983in}}%
\pgfpathlineto{\pgfqpoint{1.464245in}{0.449983in}}%
\pgfpathlineto{\pgfqpoint{1.420349in}{0.449983in}}%
\pgfpathlineto{\pgfqpoint{1.420349in}{0.449983in}}%
\pgfpathlineto{\pgfqpoint{1.376453in}{0.449983in}}%
\pgfpathlineto{\pgfqpoint{1.376453in}{0.449983in}}%
\pgfpathlineto{\pgfqpoint{1.332556in}{0.449983in}}%
\pgfpathlineto{\pgfqpoint{1.332556in}{0.449983in}}%
\pgfpathlineto{\pgfqpoint{1.288660in}{0.449983in}}%
\pgfpathlineto{\pgfqpoint{1.288660in}{0.449983in}}%
\pgfpathlineto{\pgfqpoint{1.244764in}{0.449983in}}%
\pgfpathlineto{\pgfqpoint{1.244764in}{0.449983in}}%
\pgfpathlineto{\pgfqpoint{1.200868in}{0.449983in}}%
\pgfpathlineto{\pgfqpoint{1.200868in}{0.449983in}}%
\pgfpathlineto{\pgfqpoint{1.156972in}{0.449983in}}%
\pgfpathlineto{\pgfqpoint{1.156972in}{0.449983in}}%
\pgfpathlineto{\pgfqpoint{1.113076in}{0.449983in}}%
\pgfpathlineto{\pgfqpoint{1.113076in}{0.449983in}}%
\pgfpathlineto{\pgfqpoint{1.069179in}{0.449983in}}%
\pgfpathlineto{\pgfqpoint{1.069179in}{0.449983in}}%
\pgfpathlineto{\pgfqpoint{1.025283in}{0.449983in}}%
\pgfpathlineto{\pgfqpoint{1.025283in}{0.449983in}}%
\pgfpathlineto{\pgfqpoint{0.981387in}{0.449983in}}%
\pgfpathlineto{\pgfqpoint{0.981387in}{0.449983in}}%
\pgfpathlineto{\pgfqpoint{0.937491in}{0.449983in}}%
\pgfpathlineto{\pgfqpoint{0.937491in}{0.449983in}}%
\pgfpathlineto{\pgfqpoint{0.893595in}{0.449983in}}%
\pgfpathlineto{\pgfqpoint{0.893595in}{0.449983in}}%
\pgfpathlineto{\pgfqpoint{0.849698in}{0.449983in}}%
\pgfpathlineto{\pgfqpoint{0.849698in}{0.449983in}}%
\pgfpathlineto{\pgfqpoint{0.805802in}{0.449983in}}%
\pgfpathlineto{\pgfqpoint{0.805802in}{0.449983in}}%
\pgfpathlineto{\pgfqpoint{0.761906in}{0.449983in}}%
\pgfpathlineto{\pgfqpoint{0.761906in}{0.449983in}}%
\pgfpathlineto{\pgfqpoint{0.718010in}{0.449983in}}%
\pgfpathlineto{\pgfqpoint{0.718010in}{0.449983in}}%
\pgfpathlineto{\pgfqpoint{0.674114in}{0.449983in}}%
\pgfpathlineto{\pgfqpoint{0.674114in}{0.449983in}}%
\pgfpathlineto{\pgfqpoint{0.630217in}{0.449983in}}%
\pgfpathlineto{\pgfqpoint{0.630217in}{0.449983in}}%
\pgfpathlineto{\pgfqpoint{0.586321in}{0.449983in}}%
\pgfpathlineto{\pgfqpoint{0.586321in}{0.449983in}}%
\pgfpathlineto{\pgfqpoint{0.542425in}{0.449983in}}%
\pgfpathlineto{\pgfqpoint{0.542425in}{0.449983in}}%
\pgfpathlineto{\pgfqpoint{0.498529in}{0.449983in}}%
\pgfpathlineto{\pgfqpoint{0.498529in}{0.449983in}}%
\pgfpathlineto{\pgfqpoint{0.454633in}{0.449983in}}%
\pgfpathlineto{\pgfqpoint{0.454633in}{0.449983in}}%
\pgfpathlineto{\pgfqpoint{0.410736in}{0.449983in}}%
\pgfpathlineto{\pgfqpoint{0.410736in}{0.449983in}}%
\pgfpathlineto{\pgfqpoint{0.366840in}{0.449983in}}%
\pgfusepath{fill}%
\end{pgfscope}%
\begin{pgfscope}%
\pgfpathrectangle{\pgfqpoint{0.366840in}{0.449983in}}{\pgfqpoint{2.194810in}{1.165600in}} %
\pgfusepath{clip}%
\pgfsetbuttcap%
\pgfsetmiterjoin%
\pgfsetlinewidth{0.501875pt}%
\definecolor{currentstroke}{rgb}{1.000000,0.000000,0.000000}%
\pgfsetstrokecolor{currentstroke}%
\pgfsetdash{}{0pt}%
\pgfpathmoveto{\pgfqpoint{0.366840in}{0.449983in}}%
\pgfpathlineto{\pgfqpoint{0.366840in}{0.449983in}}%
\pgfpathlineto{\pgfqpoint{0.410736in}{0.449983in}}%
\pgfpathlineto{\pgfqpoint{0.410736in}{0.449983in}}%
\pgfpathlineto{\pgfqpoint{0.454633in}{0.449983in}}%
\pgfpathlineto{\pgfqpoint{0.454633in}{0.449983in}}%
\pgfpathlineto{\pgfqpoint{0.498529in}{0.449983in}}%
\pgfpathlineto{\pgfqpoint{0.498529in}{0.450262in}}%
\pgfpathlineto{\pgfqpoint{0.542425in}{0.450262in}}%
\pgfpathlineto{\pgfqpoint{0.542425in}{0.456145in}}%
\pgfpathlineto{\pgfqpoint{0.586321in}{0.456145in}}%
\pgfpathlineto{\pgfqpoint{0.586321in}{0.475117in}}%
\pgfpathlineto{\pgfqpoint{0.630217in}{0.475117in}}%
\pgfpathlineto{\pgfqpoint{0.630217in}{0.491269in}}%
\pgfpathlineto{\pgfqpoint{0.674114in}{0.491269in}}%
\pgfpathlineto{\pgfqpoint{0.674114in}{0.511668in}}%
\pgfpathlineto{\pgfqpoint{0.718010in}{0.511668in}}%
\pgfpathlineto{\pgfqpoint{0.718010in}{0.545679in}}%
\pgfpathlineto{\pgfqpoint{0.761906in}{0.545679in}}%
\pgfpathlineto{\pgfqpoint{0.761906in}{0.611645in}}%
\pgfpathlineto{\pgfqpoint{0.805802in}{0.611645in}}%
\pgfpathlineto{\pgfqpoint{0.805802in}{0.718689in}}%
\pgfpathlineto{\pgfqpoint{0.849698in}{0.718689in}}%
\pgfpathlineto{\pgfqpoint{0.849698in}{0.841885in}}%
\pgfpathlineto{\pgfqpoint{0.893595in}{0.841885in}}%
\pgfpathlineto{\pgfqpoint{0.893595in}{1.011170in}}%
\pgfpathlineto{\pgfqpoint{0.937491in}{1.011170in}}%
\pgfpathlineto{\pgfqpoint{0.937491in}{1.223343in}}%
\pgfpathlineto{\pgfqpoint{0.981387in}{1.223343in}}%
\pgfpathlineto{\pgfqpoint{0.981387in}{1.324434in}}%
\pgfpathlineto{\pgfqpoint{1.025283in}{1.324434in}}%
\pgfpathlineto{\pgfqpoint{1.025283in}{1.428797in}}%
\pgfpathlineto{\pgfqpoint{1.069179in}{1.428797in}}%
\pgfpathlineto{\pgfqpoint{1.069179in}{1.504023in}}%
\pgfpathlineto{\pgfqpoint{1.113076in}{1.504023in}}%
\pgfpathlineto{\pgfqpoint{1.113076in}{1.538834in}}%
\pgfpathlineto{\pgfqpoint{1.156972in}{1.538834in}}%
\pgfpathlineto{\pgfqpoint{1.156972in}{1.539565in}}%
\pgfpathlineto{\pgfqpoint{1.200868in}{1.539565in}}%
\pgfpathlineto{\pgfqpoint{1.200868in}{1.484077in}}%
\pgfpathlineto{\pgfqpoint{1.244764in}{1.484077in}}%
\pgfpathlineto{\pgfqpoint{1.244764in}{1.385179in}}%
\pgfpathlineto{\pgfqpoint{1.288660in}{1.385179in}}%
\pgfpathlineto{\pgfqpoint{1.288660in}{1.274793in}}%
\pgfpathlineto{\pgfqpoint{1.332556in}{1.274793in}}%
\pgfpathlineto{\pgfqpoint{1.332556in}{1.131268in}}%
\pgfpathlineto{\pgfqpoint{1.376453in}{1.131268in}}%
\pgfpathlineto{\pgfqpoint{1.376453in}{1.062377in}}%
\pgfpathlineto{\pgfqpoint{1.420349in}{1.062377in}}%
\pgfpathlineto{\pgfqpoint{1.420349in}{1.080897in}}%
\pgfpathlineto{\pgfqpoint{1.464245in}{1.080897in}}%
\pgfpathlineto{\pgfqpoint{1.464245in}{0.860926in}}%
\pgfpathlineto{\pgfqpoint{1.508141in}{0.860926in}}%
\pgfpathlineto{\pgfqpoint{1.508141in}{0.750402in}}%
\pgfpathlineto{\pgfqpoint{1.552037in}{0.750402in}}%
\pgfpathlineto{\pgfqpoint{1.552037in}{0.676359in}}%
\pgfpathlineto{\pgfqpoint{1.595934in}{0.676359in}}%
\pgfpathlineto{\pgfqpoint{1.595934in}{0.619547in}}%
\pgfpathlineto{\pgfqpoint{1.639830in}{0.619547in}}%
\pgfpathlineto{\pgfqpoint{1.639830in}{0.583936in}}%
\pgfpathlineto{\pgfqpoint{1.683726in}{0.583936in}}%
\pgfpathlineto{\pgfqpoint{1.683726in}{0.556470in}}%
\pgfpathlineto{\pgfqpoint{1.727622in}{0.556470in}}%
\pgfpathlineto{\pgfqpoint{1.727622in}{0.538299in}}%
\pgfpathlineto{\pgfqpoint{1.771518in}{0.538299in}}%
\pgfpathlineto{\pgfqpoint{1.771518in}{0.518352in}}%
\pgfpathlineto{\pgfqpoint{1.815415in}{0.518352in}}%
\pgfpathlineto{\pgfqpoint{1.815415in}{0.504637in}}%
\pgfpathlineto{\pgfqpoint{1.859311in}{0.504637in}}%
\pgfpathlineto{\pgfqpoint{1.859311in}{0.490921in}}%
\pgfpathlineto{\pgfqpoint{1.903207in}{0.490921in}}%
\pgfpathlineto{\pgfqpoint{1.903207in}{0.482462in}}%
\pgfpathlineto{\pgfqpoint{1.947103in}{0.482462in}}%
\pgfpathlineto{\pgfqpoint{1.947103in}{0.472437in}}%
\pgfpathlineto{\pgfqpoint{1.990999in}{0.472437in}}%
\pgfpathlineto{\pgfqpoint{1.990999in}{0.462585in}}%
\pgfpathlineto{\pgfqpoint{2.034896in}{0.462585in}}%
\pgfpathlineto{\pgfqpoint{2.034896in}{0.457503in}}%
\pgfpathlineto{\pgfqpoint{2.078792in}{0.457503in}}%
\pgfpathlineto{\pgfqpoint{2.078792in}{0.454370in}}%
\pgfpathlineto{\pgfqpoint{2.122688in}{0.454370in}}%
\pgfpathlineto{\pgfqpoint{2.122688in}{0.452768in}}%
\pgfpathlineto{\pgfqpoint{2.166584in}{0.452768in}}%
\pgfpathlineto{\pgfqpoint{2.166584in}{0.451341in}}%
\pgfpathlineto{\pgfqpoint{2.210480in}{0.451341in}}%
\pgfpathlineto{\pgfqpoint{2.210480in}{0.450506in}}%
\pgfpathlineto{\pgfqpoint{2.254377in}{0.450506in}}%
\pgfpathlineto{\pgfqpoint{2.254377in}{0.450262in}}%
\pgfpathlineto{\pgfqpoint{2.298273in}{0.450262in}}%
\pgfpathlineto{\pgfqpoint{2.298273in}{0.450192in}}%
\pgfpathlineto{\pgfqpoint{2.342169in}{0.450192in}}%
\pgfpathlineto{\pgfqpoint{2.342169in}{0.450053in}}%
\pgfpathlineto{\pgfqpoint{2.386065in}{0.450053in}}%
\pgfpathlineto{\pgfqpoint{2.386065in}{0.449983in}}%
\pgfpathlineto{\pgfqpoint{2.429961in}{0.449983in}}%
\pgfpathlineto{\pgfqpoint{2.429961in}{0.450018in}}%
\pgfpathlineto{\pgfqpoint{2.473857in}{0.450018in}}%
\pgfpathlineto{\pgfqpoint{2.473857in}{0.449983in}}%
\pgfpathlineto{\pgfqpoint{2.517754in}{0.449983in}}%
\pgfpathlineto{\pgfqpoint{2.517754in}{0.449983in}}%
\pgfpathlineto{\pgfqpoint{2.561650in}{0.449983in}}%
\pgfpathlineto{\pgfqpoint{2.561650in}{0.449983in}}%
\pgfusepath{stroke}%
\end{pgfscope}%
\begin{pgfscope}%
\pgfpathrectangle{\pgfqpoint{0.366840in}{0.449983in}}{\pgfqpoint{2.194810in}{1.165600in}} %
\pgfusepath{clip}%
\pgfsetbuttcap%
\pgfsetmiterjoin%
\pgfsetlinewidth{1.003750pt}%
\definecolor{currentstroke}{rgb}{1.000000,0.647059,0.000000}%
\pgfsetstrokecolor{currentstroke}%
\pgfsetdash{}{0pt}%
\pgfpathmoveto{\pgfqpoint{0.366840in}{0.449983in}}%
\pgfpathlineto{\pgfqpoint{0.366840in}{0.449983in}}%
\pgfpathlineto{\pgfqpoint{0.410736in}{0.449983in}}%
\pgfpathlineto{\pgfqpoint{0.410736in}{0.449983in}}%
\pgfpathlineto{\pgfqpoint{0.454633in}{0.449983in}}%
\pgfpathlineto{\pgfqpoint{0.454633in}{0.449983in}}%
\pgfpathlineto{\pgfqpoint{0.498529in}{0.449983in}}%
\pgfpathlineto{\pgfqpoint{0.498529in}{0.450232in}}%
\pgfpathlineto{\pgfqpoint{0.542425in}{0.450232in}}%
\pgfpathlineto{\pgfqpoint{0.542425in}{0.456303in}}%
\pgfpathlineto{\pgfqpoint{0.586321in}{0.456303in}}%
\pgfpathlineto{\pgfqpoint{0.586321in}{0.475728in}}%
\pgfpathlineto{\pgfqpoint{0.630217in}{0.475728in}}%
\pgfpathlineto{\pgfqpoint{0.630217in}{0.493644in}}%
\pgfpathlineto{\pgfqpoint{0.674114in}{0.493644in}}%
\pgfpathlineto{\pgfqpoint{0.674114in}{0.506878in}}%
\pgfpathlineto{\pgfqpoint{0.718010in}{0.506878in}}%
\pgfpathlineto{\pgfqpoint{0.718010in}{0.533486in}}%
\pgfpathlineto{\pgfqpoint{0.761906in}{0.533486in}}%
\pgfpathlineto{\pgfqpoint{0.761906in}{0.582211in}}%
\pgfpathlineto{\pgfqpoint{0.805802in}{0.582211in}}%
\pgfpathlineto{\pgfqpoint{0.805802in}{0.669388in}}%
\pgfpathlineto{\pgfqpoint{0.849698in}{0.669388in}}%
\pgfpathlineto{\pgfqpoint{0.849698in}{0.758313in}}%
\pgfpathlineto{\pgfqpoint{0.893595in}{0.758313in}}%
\pgfpathlineto{\pgfqpoint{0.893595in}{0.893203in}}%
\pgfpathlineto{\pgfqpoint{0.937491in}{0.893203in}}%
\pgfpathlineto{\pgfqpoint{0.937491in}{1.076566in}}%
\pgfpathlineto{\pgfqpoint{0.981387in}{1.076566in}}%
\pgfpathlineto{\pgfqpoint{0.981387in}{1.192679in}}%
\pgfpathlineto{\pgfqpoint{1.025283in}{1.192679in}}%
\pgfpathlineto{\pgfqpoint{1.025283in}{1.302981in}}%
\pgfpathlineto{\pgfqpoint{1.069179in}{1.302981in}}%
\pgfpathlineto{\pgfqpoint{1.069179in}{1.412897in}}%
\pgfpathlineto{\pgfqpoint{1.113076in}{1.412897in}}%
\pgfpathlineto{\pgfqpoint{1.113076in}{1.494021in}}%
\pgfpathlineto{\pgfqpoint{1.156972in}{1.494021in}}%
\pgfpathlineto{\pgfqpoint{1.156972in}{1.528988in}}%
\pgfpathlineto{\pgfqpoint{1.200868in}{1.528988in}}%
\pgfpathlineto{\pgfqpoint{1.200868in}{1.510163in}}%
\pgfpathlineto{\pgfqpoint{1.244764in}{1.510163in}}%
\pgfpathlineto{\pgfqpoint{1.244764in}{1.446864in}}%
\pgfpathlineto{\pgfqpoint{1.288660in}{1.446864in}}%
\pgfpathlineto{\pgfqpoint{1.288660in}{1.358044in}}%
\pgfpathlineto{\pgfqpoint{1.332556in}{1.358044in}}%
\pgfpathlineto{\pgfqpoint{1.332556in}{1.206639in}}%
\pgfpathlineto{\pgfqpoint{1.376453in}{1.206639in}}%
\pgfpathlineto{\pgfqpoint{1.376453in}{1.151651in}}%
\pgfpathlineto{\pgfqpoint{1.420349in}{1.151651in}}%
\pgfpathlineto{\pgfqpoint{1.420349in}{1.188617in}}%
\pgfpathlineto{\pgfqpoint{1.464245in}{1.188617in}}%
\pgfpathlineto{\pgfqpoint{1.464245in}{0.938596in}}%
\pgfpathlineto{\pgfqpoint{1.508141in}{0.938596in}}%
\pgfpathlineto{\pgfqpoint{1.508141in}{0.814279in}}%
\pgfpathlineto{\pgfqpoint{1.552037in}{0.814279in}}%
\pgfpathlineto{\pgfqpoint{1.552037in}{0.730846in}}%
\pgfpathlineto{\pgfqpoint{1.595934in}{0.730846in}}%
\pgfpathlineto{\pgfqpoint{1.595934in}{0.662278in}}%
\pgfpathlineto{\pgfqpoint{1.639830in}{0.662278in}}%
\pgfpathlineto{\pgfqpoint{1.639830in}{0.620839in}}%
\pgfpathlineto{\pgfqpoint{1.683726in}{0.620839in}}%
\pgfpathlineto{\pgfqpoint{1.683726in}{0.585884in}}%
\pgfpathlineto{\pgfqpoint{1.727622in}{0.585884in}}%
\pgfpathlineto{\pgfqpoint{1.727622in}{0.563603in}}%
\pgfpathlineto{\pgfqpoint{1.771518in}{0.563603in}}%
\pgfpathlineto{\pgfqpoint{1.771518in}{0.536760in}}%
\pgfpathlineto{\pgfqpoint{1.815415in}{0.536760in}}%
\pgfpathlineto{\pgfqpoint{1.815415in}{0.520254in}}%
\pgfpathlineto{\pgfqpoint{1.859311in}{0.520254in}}%
\pgfpathlineto{\pgfqpoint{1.859311in}{0.502737in}}%
\pgfpathlineto{\pgfqpoint{1.903207in}{0.502737in}}%
\pgfpathlineto{\pgfqpoint{1.903207in}{0.491305in}}%
\pgfpathlineto{\pgfqpoint{1.947103in}{0.491305in}}%
\pgfpathlineto{\pgfqpoint{1.947103in}{0.479515in}}%
\pgfpathlineto{\pgfqpoint{1.990999in}{0.479515in}}%
\pgfpathlineto{\pgfqpoint{1.990999in}{0.466468in}}%
\pgfpathlineto{\pgfqpoint{2.034896in}{0.466468in}}%
\pgfpathlineto{\pgfqpoint{2.034896in}{0.459705in}}%
\pgfpathlineto{\pgfqpoint{2.078792in}{0.459705in}}%
\pgfpathlineto{\pgfqpoint{2.078792in}{0.456045in}}%
\pgfpathlineto{\pgfqpoint{2.122688in}{0.456045in}}%
\pgfpathlineto{\pgfqpoint{2.122688in}{0.453493in}}%
\pgfpathlineto{\pgfqpoint{2.166584in}{0.453493in}}%
\pgfpathlineto{\pgfqpoint{2.166584in}{0.452024in}}%
\pgfpathlineto{\pgfqpoint{2.210480in}{0.452024in}}%
\pgfpathlineto{\pgfqpoint{2.210480in}{0.450538in}}%
\pgfpathlineto{\pgfqpoint{2.254377in}{0.450538in}}%
\pgfpathlineto{\pgfqpoint{2.254377in}{0.450375in}}%
\pgfpathlineto{\pgfqpoint{2.298273in}{0.450375in}}%
\pgfpathlineto{\pgfqpoint{2.298273in}{0.450270in}}%
\pgfpathlineto{\pgfqpoint{2.342169in}{0.450270in}}%
\pgfpathlineto{\pgfqpoint{2.342169in}{0.450075in}}%
\pgfpathlineto{\pgfqpoint{2.386065in}{0.450075in}}%
\pgfpathlineto{\pgfqpoint{2.386065in}{0.449983in}}%
\pgfpathlineto{\pgfqpoint{2.429961in}{0.449983in}}%
\pgfpathlineto{\pgfqpoint{2.429961in}{0.450080in}}%
\pgfpathlineto{\pgfqpoint{2.473857in}{0.450080in}}%
\pgfpathlineto{\pgfqpoint{2.473857in}{0.449983in}}%
\pgfpathlineto{\pgfqpoint{2.517754in}{0.449983in}}%
\pgfpathlineto{\pgfqpoint{2.517754in}{0.449983in}}%
\pgfpathlineto{\pgfqpoint{2.561650in}{0.449983in}}%
\pgfpathlineto{\pgfqpoint{2.561650in}{0.449983in}}%
\pgfusepath{stroke}%
\end{pgfscope}%
\begin{pgfscope}%
\pgfsetrectcap%
\pgfsetmiterjoin%
\pgfsetlinewidth{1.003750pt}%
\definecolor{currentstroke}{rgb}{0.000000,0.000000,0.000000}%
\pgfsetstrokecolor{currentstroke}%
\pgfsetdash{}{0pt}%
\pgfpathmoveto{\pgfqpoint{0.366840in}{1.615583in}}%
\pgfpathlineto{\pgfqpoint{2.561650in}{1.615583in}}%
\pgfusepath{stroke}%
\end{pgfscope}%
\begin{pgfscope}%
\pgfsetrectcap%
\pgfsetmiterjoin%
\pgfsetlinewidth{1.003750pt}%
\definecolor{currentstroke}{rgb}{0.000000,0.000000,0.000000}%
\pgfsetstrokecolor{currentstroke}%
\pgfsetdash{}{0pt}%
\pgfpathmoveto{\pgfqpoint{2.561650in}{0.449983in}}%
\pgfpathlineto{\pgfqpoint{2.561650in}{1.615583in}}%
\pgfusepath{stroke}%
\end{pgfscope}%
\begin{pgfscope}%
\pgfsetrectcap%
\pgfsetmiterjoin%
\pgfsetlinewidth{1.003750pt}%
\definecolor{currentstroke}{rgb}{0.000000,0.000000,0.000000}%
\pgfsetstrokecolor{currentstroke}%
\pgfsetdash{}{0pt}%
\pgfpathmoveto{\pgfqpoint{0.366840in}{0.449983in}}%
\pgfpathlineto{\pgfqpoint{2.561650in}{0.449983in}}%
\pgfusepath{stroke}%
\end{pgfscope}%
\begin{pgfscope}%
\pgfsetrectcap%
\pgfsetmiterjoin%
\pgfsetlinewidth{1.003750pt}%
\definecolor{currentstroke}{rgb}{0.000000,0.000000,0.000000}%
\pgfsetstrokecolor{currentstroke}%
\pgfsetdash{}{0pt}%
\pgfpathmoveto{\pgfqpoint{0.366840in}{0.449983in}}%
\pgfpathlineto{\pgfqpoint{0.366840in}{1.615583in}}%
\pgfusepath{stroke}%
\end{pgfscope}%
\begin{pgfscope}%
\pgfsetbuttcap%
\pgfsetroundjoin%
\definecolor{currentfill}{rgb}{0.000000,0.000000,0.000000}%
\pgfsetfillcolor{currentfill}%
\pgfsetlinewidth{0.501875pt}%
\definecolor{currentstroke}{rgb}{0.000000,0.000000,0.000000}%
\pgfsetstrokecolor{currentstroke}%
\pgfsetdash{}{0pt}%
\pgfsys@defobject{currentmarker}{\pgfqpoint{0.000000in}{0.000000in}}{\pgfqpoint{0.000000in}{0.069444in}}{%
\pgfpathmoveto{\pgfqpoint{0.000000in}{0.000000in}}%
\pgfpathlineto{\pgfqpoint{0.000000in}{0.069444in}}%
\pgfusepath{stroke,fill}%
}%
\begin{pgfscope}%
\pgfsys@transformshift{0.366840in}{0.449983in}%
\pgfsys@useobject{currentmarker}{}%
\end{pgfscope}%
\end{pgfscope}%
\begin{pgfscope}%
\pgfsetbuttcap%
\pgfsetroundjoin%
\definecolor{currentfill}{rgb}{0.000000,0.000000,0.000000}%
\pgfsetfillcolor{currentfill}%
\pgfsetlinewidth{0.501875pt}%
\definecolor{currentstroke}{rgb}{0.000000,0.000000,0.000000}%
\pgfsetstrokecolor{currentstroke}%
\pgfsetdash{}{0pt}%
\pgfsys@defobject{currentmarker}{\pgfqpoint{0.000000in}{-0.069444in}}{\pgfqpoint{0.000000in}{0.000000in}}{%
\pgfpathmoveto{\pgfqpoint{0.000000in}{0.000000in}}%
\pgfpathlineto{\pgfqpoint{0.000000in}{-0.069444in}}%
\pgfusepath{stroke,fill}%
}%
\begin{pgfscope}%
\pgfsys@transformshift{0.366840in}{1.615583in}%
\pgfsys@useobject{currentmarker}{}%
\end{pgfscope}%
\end{pgfscope}%
\begin{pgfscope}%
\pgftext[x=0.366840in,y=0.380539in,,top]{\rmfamily\fontsize{8.000000}{9.600000}\selectfont −15}%
\end{pgfscope}%
\begin{pgfscope}%
\pgfsetbuttcap%
\pgfsetroundjoin%
\definecolor{currentfill}{rgb}{0.000000,0.000000,0.000000}%
\pgfsetfillcolor{currentfill}%
\pgfsetlinewidth{0.501875pt}%
\definecolor{currentstroke}{rgb}{0.000000,0.000000,0.000000}%
\pgfsetstrokecolor{currentstroke}%
\pgfsetdash{}{0pt}%
\pgfsys@defobject{currentmarker}{\pgfqpoint{0.000000in}{0.000000in}}{\pgfqpoint{0.000000in}{0.069444in}}{%
\pgfpathmoveto{\pgfqpoint{0.000000in}{0.000000in}}%
\pgfpathlineto{\pgfqpoint{0.000000in}{0.069444in}}%
\pgfusepath{stroke,fill}%
}%
\begin{pgfscope}%
\pgfsys@transformshift{0.732642in}{0.449983in}%
\pgfsys@useobject{currentmarker}{}%
\end{pgfscope}%
\end{pgfscope}%
\begin{pgfscope}%
\pgfsetbuttcap%
\pgfsetroundjoin%
\definecolor{currentfill}{rgb}{0.000000,0.000000,0.000000}%
\pgfsetfillcolor{currentfill}%
\pgfsetlinewidth{0.501875pt}%
\definecolor{currentstroke}{rgb}{0.000000,0.000000,0.000000}%
\pgfsetstrokecolor{currentstroke}%
\pgfsetdash{}{0pt}%
\pgfsys@defobject{currentmarker}{\pgfqpoint{0.000000in}{-0.069444in}}{\pgfqpoint{0.000000in}{0.000000in}}{%
\pgfpathmoveto{\pgfqpoint{0.000000in}{0.000000in}}%
\pgfpathlineto{\pgfqpoint{0.000000in}{-0.069444in}}%
\pgfusepath{stroke,fill}%
}%
\begin{pgfscope}%
\pgfsys@transformshift{0.732642in}{1.615583in}%
\pgfsys@useobject{currentmarker}{}%
\end{pgfscope}%
\end{pgfscope}%
\begin{pgfscope}%
\pgftext[x=0.732642in,y=0.380539in,,top]{\rmfamily\fontsize{8.000000}{9.600000}\selectfont −10}%
\end{pgfscope}%
\begin{pgfscope}%
\pgfsetbuttcap%
\pgfsetroundjoin%
\definecolor{currentfill}{rgb}{0.000000,0.000000,0.000000}%
\pgfsetfillcolor{currentfill}%
\pgfsetlinewidth{0.501875pt}%
\definecolor{currentstroke}{rgb}{0.000000,0.000000,0.000000}%
\pgfsetstrokecolor{currentstroke}%
\pgfsetdash{}{0pt}%
\pgfsys@defobject{currentmarker}{\pgfqpoint{0.000000in}{0.000000in}}{\pgfqpoint{0.000000in}{0.069444in}}{%
\pgfpathmoveto{\pgfqpoint{0.000000in}{0.000000in}}%
\pgfpathlineto{\pgfqpoint{0.000000in}{0.069444in}}%
\pgfusepath{stroke,fill}%
}%
\begin{pgfscope}%
\pgfsys@transformshift{1.098443in}{0.449983in}%
\pgfsys@useobject{currentmarker}{}%
\end{pgfscope}%
\end{pgfscope}%
\begin{pgfscope}%
\pgfsetbuttcap%
\pgfsetroundjoin%
\definecolor{currentfill}{rgb}{0.000000,0.000000,0.000000}%
\pgfsetfillcolor{currentfill}%
\pgfsetlinewidth{0.501875pt}%
\definecolor{currentstroke}{rgb}{0.000000,0.000000,0.000000}%
\pgfsetstrokecolor{currentstroke}%
\pgfsetdash{}{0pt}%
\pgfsys@defobject{currentmarker}{\pgfqpoint{0.000000in}{-0.069444in}}{\pgfqpoint{0.000000in}{0.000000in}}{%
\pgfpathmoveto{\pgfqpoint{0.000000in}{0.000000in}}%
\pgfpathlineto{\pgfqpoint{0.000000in}{-0.069444in}}%
\pgfusepath{stroke,fill}%
}%
\begin{pgfscope}%
\pgfsys@transformshift{1.098443in}{1.615583in}%
\pgfsys@useobject{currentmarker}{}%
\end{pgfscope}%
\end{pgfscope}%
\begin{pgfscope}%
\pgftext[x=1.098443in,y=0.380539in,,top]{\rmfamily\fontsize{8.000000}{9.600000}\selectfont −5}%
\end{pgfscope}%
\begin{pgfscope}%
\pgfsetbuttcap%
\pgfsetroundjoin%
\definecolor{currentfill}{rgb}{0.000000,0.000000,0.000000}%
\pgfsetfillcolor{currentfill}%
\pgfsetlinewidth{0.501875pt}%
\definecolor{currentstroke}{rgb}{0.000000,0.000000,0.000000}%
\pgfsetstrokecolor{currentstroke}%
\pgfsetdash{}{0pt}%
\pgfsys@defobject{currentmarker}{\pgfqpoint{0.000000in}{0.000000in}}{\pgfqpoint{0.000000in}{0.069444in}}{%
\pgfpathmoveto{\pgfqpoint{0.000000in}{0.000000in}}%
\pgfpathlineto{\pgfqpoint{0.000000in}{0.069444in}}%
\pgfusepath{stroke,fill}%
}%
\begin{pgfscope}%
\pgfsys@transformshift{1.464245in}{0.449983in}%
\pgfsys@useobject{currentmarker}{}%
\end{pgfscope}%
\end{pgfscope}%
\begin{pgfscope}%
\pgfsetbuttcap%
\pgfsetroundjoin%
\definecolor{currentfill}{rgb}{0.000000,0.000000,0.000000}%
\pgfsetfillcolor{currentfill}%
\pgfsetlinewidth{0.501875pt}%
\definecolor{currentstroke}{rgb}{0.000000,0.000000,0.000000}%
\pgfsetstrokecolor{currentstroke}%
\pgfsetdash{}{0pt}%
\pgfsys@defobject{currentmarker}{\pgfqpoint{0.000000in}{-0.069444in}}{\pgfqpoint{0.000000in}{0.000000in}}{%
\pgfpathmoveto{\pgfqpoint{0.000000in}{0.000000in}}%
\pgfpathlineto{\pgfqpoint{0.000000in}{-0.069444in}}%
\pgfusepath{stroke,fill}%
}%
\begin{pgfscope}%
\pgfsys@transformshift{1.464245in}{1.615583in}%
\pgfsys@useobject{currentmarker}{}%
\end{pgfscope}%
\end{pgfscope}%
\begin{pgfscope}%
\pgftext[x=1.464245in,y=0.380539in,,top]{\rmfamily\fontsize{8.000000}{9.600000}\selectfont 0}%
\end{pgfscope}%
\begin{pgfscope}%
\pgfsetbuttcap%
\pgfsetroundjoin%
\definecolor{currentfill}{rgb}{0.000000,0.000000,0.000000}%
\pgfsetfillcolor{currentfill}%
\pgfsetlinewidth{0.501875pt}%
\definecolor{currentstroke}{rgb}{0.000000,0.000000,0.000000}%
\pgfsetstrokecolor{currentstroke}%
\pgfsetdash{}{0pt}%
\pgfsys@defobject{currentmarker}{\pgfqpoint{0.000000in}{0.000000in}}{\pgfqpoint{0.000000in}{0.069444in}}{%
\pgfpathmoveto{\pgfqpoint{0.000000in}{0.000000in}}%
\pgfpathlineto{\pgfqpoint{0.000000in}{0.069444in}}%
\pgfusepath{stroke,fill}%
}%
\begin{pgfscope}%
\pgfsys@transformshift{1.830047in}{0.449983in}%
\pgfsys@useobject{currentmarker}{}%
\end{pgfscope}%
\end{pgfscope}%
\begin{pgfscope}%
\pgfsetbuttcap%
\pgfsetroundjoin%
\definecolor{currentfill}{rgb}{0.000000,0.000000,0.000000}%
\pgfsetfillcolor{currentfill}%
\pgfsetlinewidth{0.501875pt}%
\definecolor{currentstroke}{rgb}{0.000000,0.000000,0.000000}%
\pgfsetstrokecolor{currentstroke}%
\pgfsetdash{}{0pt}%
\pgfsys@defobject{currentmarker}{\pgfqpoint{0.000000in}{-0.069444in}}{\pgfqpoint{0.000000in}{0.000000in}}{%
\pgfpathmoveto{\pgfqpoint{0.000000in}{0.000000in}}%
\pgfpathlineto{\pgfqpoint{0.000000in}{-0.069444in}}%
\pgfusepath{stroke,fill}%
}%
\begin{pgfscope}%
\pgfsys@transformshift{1.830047in}{1.615583in}%
\pgfsys@useobject{currentmarker}{}%
\end{pgfscope}%
\end{pgfscope}%
\begin{pgfscope}%
\pgftext[x=1.830047in,y=0.380539in,,top]{\rmfamily\fontsize{8.000000}{9.600000}\selectfont 5}%
\end{pgfscope}%
\begin{pgfscope}%
\pgfsetbuttcap%
\pgfsetroundjoin%
\definecolor{currentfill}{rgb}{0.000000,0.000000,0.000000}%
\pgfsetfillcolor{currentfill}%
\pgfsetlinewidth{0.501875pt}%
\definecolor{currentstroke}{rgb}{0.000000,0.000000,0.000000}%
\pgfsetstrokecolor{currentstroke}%
\pgfsetdash{}{0pt}%
\pgfsys@defobject{currentmarker}{\pgfqpoint{0.000000in}{0.000000in}}{\pgfqpoint{0.000000in}{0.069444in}}{%
\pgfpathmoveto{\pgfqpoint{0.000000in}{0.000000in}}%
\pgfpathlineto{\pgfqpoint{0.000000in}{0.069444in}}%
\pgfusepath{stroke,fill}%
}%
\begin{pgfscope}%
\pgfsys@transformshift{2.195848in}{0.449983in}%
\pgfsys@useobject{currentmarker}{}%
\end{pgfscope}%
\end{pgfscope}%
\begin{pgfscope}%
\pgfsetbuttcap%
\pgfsetroundjoin%
\definecolor{currentfill}{rgb}{0.000000,0.000000,0.000000}%
\pgfsetfillcolor{currentfill}%
\pgfsetlinewidth{0.501875pt}%
\definecolor{currentstroke}{rgb}{0.000000,0.000000,0.000000}%
\pgfsetstrokecolor{currentstroke}%
\pgfsetdash{}{0pt}%
\pgfsys@defobject{currentmarker}{\pgfqpoint{0.000000in}{-0.069444in}}{\pgfqpoint{0.000000in}{0.000000in}}{%
\pgfpathmoveto{\pgfqpoint{0.000000in}{0.000000in}}%
\pgfpathlineto{\pgfqpoint{0.000000in}{-0.069444in}}%
\pgfusepath{stroke,fill}%
}%
\begin{pgfscope}%
\pgfsys@transformshift{2.195848in}{1.615583in}%
\pgfsys@useobject{currentmarker}{}%
\end{pgfscope}%
\end{pgfscope}%
\begin{pgfscope}%
\pgftext[x=2.195848in,y=0.380539in,,top]{\rmfamily\fontsize{8.000000}{9.600000}\selectfont 10}%
\end{pgfscope}%
\begin{pgfscope}%
\pgfsetbuttcap%
\pgfsetroundjoin%
\definecolor{currentfill}{rgb}{0.000000,0.000000,0.000000}%
\pgfsetfillcolor{currentfill}%
\pgfsetlinewidth{0.501875pt}%
\definecolor{currentstroke}{rgb}{0.000000,0.000000,0.000000}%
\pgfsetstrokecolor{currentstroke}%
\pgfsetdash{}{0pt}%
\pgfsys@defobject{currentmarker}{\pgfqpoint{0.000000in}{0.000000in}}{\pgfqpoint{0.000000in}{0.069444in}}{%
\pgfpathmoveto{\pgfqpoint{0.000000in}{0.000000in}}%
\pgfpathlineto{\pgfqpoint{0.000000in}{0.069444in}}%
\pgfusepath{stroke,fill}%
}%
\begin{pgfscope}%
\pgfsys@transformshift{2.561650in}{0.449983in}%
\pgfsys@useobject{currentmarker}{}%
\end{pgfscope}%
\end{pgfscope}%
\begin{pgfscope}%
\pgfsetbuttcap%
\pgfsetroundjoin%
\definecolor{currentfill}{rgb}{0.000000,0.000000,0.000000}%
\pgfsetfillcolor{currentfill}%
\pgfsetlinewidth{0.501875pt}%
\definecolor{currentstroke}{rgb}{0.000000,0.000000,0.000000}%
\pgfsetstrokecolor{currentstroke}%
\pgfsetdash{}{0pt}%
\pgfsys@defobject{currentmarker}{\pgfqpoint{0.000000in}{-0.069444in}}{\pgfqpoint{0.000000in}{0.000000in}}{%
\pgfpathmoveto{\pgfqpoint{0.000000in}{0.000000in}}%
\pgfpathlineto{\pgfqpoint{0.000000in}{-0.069444in}}%
\pgfusepath{stroke,fill}%
}%
\begin{pgfscope}%
\pgfsys@transformshift{2.561650in}{1.615583in}%
\pgfsys@useobject{currentmarker}{}%
\end{pgfscope}%
\end{pgfscope}%
\begin{pgfscope}%
\pgftext[x=2.561650in,y=0.380539in,,top]{\rmfamily\fontsize{8.000000}{9.600000}\selectfont 15}%
\end{pgfscope}%
\begin{pgfscope}%
\pgftext[x=1.464245in,y=0.203564in,,top]{\rmfamily\fontsize{9.000000}{10.800000}\selectfont \(\displaystyle \mathrm{DLL}_{\mu/\pi}(K^+)\)}%
\end{pgfscope}%
\begin{pgfscope}%
\pgfsetbuttcap%
\pgfsetroundjoin%
\definecolor{currentfill}{rgb}{0.000000,0.000000,0.000000}%
\pgfsetfillcolor{currentfill}%
\pgfsetlinewidth{0.501875pt}%
\definecolor{currentstroke}{rgb}{0.000000,0.000000,0.000000}%
\pgfsetstrokecolor{currentstroke}%
\pgfsetdash{}{0pt}%
\pgfsys@defobject{currentmarker}{\pgfqpoint{0.000000in}{0.000000in}}{\pgfqpoint{0.069444in}{0.000000in}}{%
\pgfpathmoveto{\pgfqpoint{0.000000in}{0.000000in}}%
\pgfpathlineto{\pgfqpoint{0.069444in}{0.000000in}}%
\pgfusepath{stroke,fill}%
}%
\begin{pgfscope}%
\pgfsys@transformshift{0.366840in}{0.449983in}%
\pgfsys@useobject{currentmarker}{}%
\end{pgfscope}%
\end{pgfscope}%
\begin{pgfscope}%
\pgfsetbuttcap%
\pgfsetroundjoin%
\definecolor{currentfill}{rgb}{0.000000,0.000000,0.000000}%
\pgfsetfillcolor{currentfill}%
\pgfsetlinewidth{0.501875pt}%
\definecolor{currentstroke}{rgb}{0.000000,0.000000,0.000000}%
\pgfsetstrokecolor{currentstroke}%
\pgfsetdash{}{0pt}%
\pgfsys@defobject{currentmarker}{\pgfqpoint{-0.069444in}{0.000000in}}{\pgfqpoint{0.000000in}{0.000000in}}{%
\pgfpathmoveto{\pgfqpoint{0.000000in}{0.000000in}}%
\pgfpathlineto{\pgfqpoint{-0.069444in}{0.000000in}}%
\pgfusepath{stroke,fill}%
}%
\begin{pgfscope}%
\pgfsys@transformshift{2.561650in}{0.449983in}%
\pgfsys@useobject{currentmarker}{}%
\end{pgfscope}%
\end{pgfscope}%
\begin{pgfscope}%
\pgftext[x=0.297396in,y=0.449983in,right,]{\rmfamily\fontsize{8.000000}{9.600000}\selectfont 0.00}%
\end{pgfscope}%
\begin{pgfscope}%
\pgfsetbuttcap%
\pgfsetroundjoin%
\definecolor{currentfill}{rgb}{0.000000,0.000000,0.000000}%
\pgfsetfillcolor{currentfill}%
\pgfsetlinewidth{0.501875pt}%
\definecolor{currentstroke}{rgb}{0.000000,0.000000,0.000000}%
\pgfsetstrokecolor{currentstroke}%
\pgfsetdash{}{0pt}%
\pgfsys@defobject{currentmarker}{\pgfqpoint{0.000000in}{0.000000in}}{\pgfqpoint{0.069444in}{0.000000in}}{%
\pgfpathmoveto{\pgfqpoint{0.000000in}{0.000000in}}%
\pgfpathlineto{\pgfqpoint{0.069444in}{0.000000in}}%
\pgfusepath{stroke,fill}%
}%
\begin{pgfscope}%
\pgfsys@transformshift{0.366840in}{0.616498in}%
\pgfsys@useobject{currentmarker}{}%
\end{pgfscope}%
\end{pgfscope}%
\begin{pgfscope}%
\pgfsetbuttcap%
\pgfsetroundjoin%
\definecolor{currentfill}{rgb}{0.000000,0.000000,0.000000}%
\pgfsetfillcolor{currentfill}%
\pgfsetlinewidth{0.501875pt}%
\definecolor{currentstroke}{rgb}{0.000000,0.000000,0.000000}%
\pgfsetstrokecolor{currentstroke}%
\pgfsetdash{}{0pt}%
\pgfsys@defobject{currentmarker}{\pgfqpoint{-0.069444in}{0.000000in}}{\pgfqpoint{0.000000in}{0.000000in}}{%
\pgfpathmoveto{\pgfqpoint{0.000000in}{0.000000in}}%
\pgfpathlineto{\pgfqpoint{-0.069444in}{0.000000in}}%
\pgfusepath{stroke,fill}%
}%
\begin{pgfscope}%
\pgfsys@transformshift{2.561650in}{0.616498in}%
\pgfsys@useobject{currentmarker}{}%
\end{pgfscope}%
\end{pgfscope}%
\begin{pgfscope}%
\pgftext[x=0.297396in,y=0.616498in,right,]{\rmfamily\fontsize{8.000000}{9.600000}\selectfont 0.02}%
\end{pgfscope}%
\begin{pgfscope}%
\pgfsetbuttcap%
\pgfsetroundjoin%
\definecolor{currentfill}{rgb}{0.000000,0.000000,0.000000}%
\pgfsetfillcolor{currentfill}%
\pgfsetlinewidth{0.501875pt}%
\definecolor{currentstroke}{rgb}{0.000000,0.000000,0.000000}%
\pgfsetstrokecolor{currentstroke}%
\pgfsetdash{}{0pt}%
\pgfsys@defobject{currentmarker}{\pgfqpoint{0.000000in}{0.000000in}}{\pgfqpoint{0.069444in}{0.000000in}}{%
\pgfpathmoveto{\pgfqpoint{0.000000in}{0.000000in}}%
\pgfpathlineto{\pgfqpoint{0.069444in}{0.000000in}}%
\pgfusepath{stroke,fill}%
}%
\begin{pgfscope}%
\pgfsys@transformshift{0.366840in}{0.783012in}%
\pgfsys@useobject{currentmarker}{}%
\end{pgfscope}%
\end{pgfscope}%
\begin{pgfscope}%
\pgfsetbuttcap%
\pgfsetroundjoin%
\definecolor{currentfill}{rgb}{0.000000,0.000000,0.000000}%
\pgfsetfillcolor{currentfill}%
\pgfsetlinewidth{0.501875pt}%
\definecolor{currentstroke}{rgb}{0.000000,0.000000,0.000000}%
\pgfsetstrokecolor{currentstroke}%
\pgfsetdash{}{0pt}%
\pgfsys@defobject{currentmarker}{\pgfqpoint{-0.069444in}{0.000000in}}{\pgfqpoint{0.000000in}{0.000000in}}{%
\pgfpathmoveto{\pgfqpoint{0.000000in}{0.000000in}}%
\pgfpathlineto{\pgfqpoint{-0.069444in}{0.000000in}}%
\pgfusepath{stroke,fill}%
}%
\begin{pgfscope}%
\pgfsys@transformshift{2.561650in}{0.783012in}%
\pgfsys@useobject{currentmarker}{}%
\end{pgfscope}%
\end{pgfscope}%
\begin{pgfscope}%
\pgftext[x=0.297396in,y=0.783012in,right,]{\rmfamily\fontsize{8.000000}{9.600000}\selectfont 0.04}%
\end{pgfscope}%
\begin{pgfscope}%
\pgfsetbuttcap%
\pgfsetroundjoin%
\definecolor{currentfill}{rgb}{0.000000,0.000000,0.000000}%
\pgfsetfillcolor{currentfill}%
\pgfsetlinewidth{0.501875pt}%
\definecolor{currentstroke}{rgb}{0.000000,0.000000,0.000000}%
\pgfsetstrokecolor{currentstroke}%
\pgfsetdash{}{0pt}%
\pgfsys@defobject{currentmarker}{\pgfqpoint{0.000000in}{0.000000in}}{\pgfqpoint{0.069444in}{0.000000in}}{%
\pgfpathmoveto{\pgfqpoint{0.000000in}{0.000000in}}%
\pgfpathlineto{\pgfqpoint{0.069444in}{0.000000in}}%
\pgfusepath{stroke,fill}%
}%
\begin{pgfscope}%
\pgfsys@transformshift{0.366840in}{0.949526in}%
\pgfsys@useobject{currentmarker}{}%
\end{pgfscope}%
\end{pgfscope}%
\begin{pgfscope}%
\pgfsetbuttcap%
\pgfsetroundjoin%
\definecolor{currentfill}{rgb}{0.000000,0.000000,0.000000}%
\pgfsetfillcolor{currentfill}%
\pgfsetlinewidth{0.501875pt}%
\definecolor{currentstroke}{rgb}{0.000000,0.000000,0.000000}%
\pgfsetstrokecolor{currentstroke}%
\pgfsetdash{}{0pt}%
\pgfsys@defobject{currentmarker}{\pgfqpoint{-0.069444in}{0.000000in}}{\pgfqpoint{0.000000in}{0.000000in}}{%
\pgfpathmoveto{\pgfqpoint{0.000000in}{0.000000in}}%
\pgfpathlineto{\pgfqpoint{-0.069444in}{0.000000in}}%
\pgfusepath{stroke,fill}%
}%
\begin{pgfscope}%
\pgfsys@transformshift{2.561650in}{0.949526in}%
\pgfsys@useobject{currentmarker}{}%
\end{pgfscope}%
\end{pgfscope}%
\begin{pgfscope}%
\pgftext[x=0.297396in,y=0.949526in,right,]{\rmfamily\fontsize{8.000000}{9.600000}\selectfont 0.06}%
\end{pgfscope}%
\begin{pgfscope}%
\pgfsetbuttcap%
\pgfsetroundjoin%
\definecolor{currentfill}{rgb}{0.000000,0.000000,0.000000}%
\pgfsetfillcolor{currentfill}%
\pgfsetlinewidth{0.501875pt}%
\definecolor{currentstroke}{rgb}{0.000000,0.000000,0.000000}%
\pgfsetstrokecolor{currentstroke}%
\pgfsetdash{}{0pt}%
\pgfsys@defobject{currentmarker}{\pgfqpoint{0.000000in}{0.000000in}}{\pgfqpoint{0.069444in}{0.000000in}}{%
\pgfpathmoveto{\pgfqpoint{0.000000in}{0.000000in}}%
\pgfpathlineto{\pgfqpoint{0.069444in}{0.000000in}}%
\pgfusepath{stroke,fill}%
}%
\begin{pgfscope}%
\pgfsys@transformshift{0.366840in}{1.116041in}%
\pgfsys@useobject{currentmarker}{}%
\end{pgfscope}%
\end{pgfscope}%
\begin{pgfscope}%
\pgfsetbuttcap%
\pgfsetroundjoin%
\definecolor{currentfill}{rgb}{0.000000,0.000000,0.000000}%
\pgfsetfillcolor{currentfill}%
\pgfsetlinewidth{0.501875pt}%
\definecolor{currentstroke}{rgb}{0.000000,0.000000,0.000000}%
\pgfsetstrokecolor{currentstroke}%
\pgfsetdash{}{0pt}%
\pgfsys@defobject{currentmarker}{\pgfqpoint{-0.069444in}{0.000000in}}{\pgfqpoint{0.000000in}{0.000000in}}{%
\pgfpathmoveto{\pgfqpoint{0.000000in}{0.000000in}}%
\pgfpathlineto{\pgfqpoint{-0.069444in}{0.000000in}}%
\pgfusepath{stroke,fill}%
}%
\begin{pgfscope}%
\pgfsys@transformshift{2.561650in}{1.116041in}%
\pgfsys@useobject{currentmarker}{}%
\end{pgfscope}%
\end{pgfscope}%
\begin{pgfscope}%
\pgftext[x=0.297396in,y=1.116041in,right,]{\rmfamily\fontsize{8.000000}{9.600000}\selectfont 0.08}%
\end{pgfscope}%
\begin{pgfscope}%
\pgfsetbuttcap%
\pgfsetroundjoin%
\definecolor{currentfill}{rgb}{0.000000,0.000000,0.000000}%
\pgfsetfillcolor{currentfill}%
\pgfsetlinewidth{0.501875pt}%
\definecolor{currentstroke}{rgb}{0.000000,0.000000,0.000000}%
\pgfsetstrokecolor{currentstroke}%
\pgfsetdash{}{0pt}%
\pgfsys@defobject{currentmarker}{\pgfqpoint{0.000000in}{0.000000in}}{\pgfqpoint{0.069444in}{0.000000in}}{%
\pgfpathmoveto{\pgfqpoint{0.000000in}{0.000000in}}%
\pgfpathlineto{\pgfqpoint{0.069444in}{0.000000in}}%
\pgfusepath{stroke,fill}%
}%
\begin{pgfscope}%
\pgfsys@transformshift{0.366840in}{1.282555in}%
\pgfsys@useobject{currentmarker}{}%
\end{pgfscope}%
\end{pgfscope}%
\begin{pgfscope}%
\pgfsetbuttcap%
\pgfsetroundjoin%
\definecolor{currentfill}{rgb}{0.000000,0.000000,0.000000}%
\pgfsetfillcolor{currentfill}%
\pgfsetlinewidth{0.501875pt}%
\definecolor{currentstroke}{rgb}{0.000000,0.000000,0.000000}%
\pgfsetstrokecolor{currentstroke}%
\pgfsetdash{}{0pt}%
\pgfsys@defobject{currentmarker}{\pgfqpoint{-0.069444in}{0.000000in}}{\pgfqpoint{0.000000in}{0.000000in}}{%
\pgfpathmoveto{\pgfqpoint{0.000000in}{0.000000in}}%
\pgfpathlineto{\pgfqpoint{-0.069444in}{0.000000in}}%
\pgfusepath{stroke,fill}%
}%
\begin{pgfscope}%
\pgfsys@transformshift{2.561650in}{1.282555in}%
\pgfsys@useobject{currentmarker}{}%
\end{pgfscope}%
\end{pgfscope}%
\begin{pgfscope}%
\pgftext[x=0.297396in,y=1.282555in,right,]{\rmfamily\fontsize{8.000000}{9.600000}\selectfont 0.10}%
\end{pgfscope}%
\begin{pgfscope}%
\pgfsetbuttcap%
\pgfsetroundjoin%
\definecolor{currentfill}{rgb}{0.000000,0.000000,0.000000}%
\pgfsetfillcolor{currentfill}%
\pgfsetlinewidth{0.501875pt}%
\definecolor{currentstroke}{rgb}{0.000000,0.000000,0.000000}%
\pgfsetstrokecolor{currentstroke}%
\pgfsetdash{}{0pt}%
\pgfsys@defobject{currentmarker}{\pgfqpoint{0.000000in}{0.000000in}}{\pgfqpoint{0.069444in}{0.000000in}}{%
\pgfpathmoveto{\pgfqpoint{0.000000in}{0.000000in}}%
\pgfpathlineto{\pgfqpoint{0.069444in}{0.000000in}}%
\pgfusepath{stroke,fill}%
}%
\begin{pgfscope}%
\pgfsys@transformshift{0.366840in}{1.449069in}%
\pgfsys@useobject{currentmarker}{}%
\end{pgfscope}%
\end{pgfscope}%
\begin{pgfscope}%
\pgfsetbuttcap%
\pgfsetroundjoin%
\definecolor{currentfill}{rgb}{0.000000,0.000000,0.000000}%
\pgfsetfillcolor{currentfill}%
\pgfsetlinewidth{0.501875pt}%
\definecolor{currentstroke}{rgb}{0.000000,0.000000,0.000000}%
\pgfsetstrokecolor{currentstroke}%
\pgfsetdash{}{0pt}%
\pgfsys@defobject{currentmarker}{\pgfqpoint{-0.069444in}{0.000000in}}{\pgfqpoint{0.000000in}{0.000000in}}{%
\pgfpathmoveto{\pgfqpoint{0.000000in}{0.000000in}}%
\pgfpathlineto{\pgfqpoint{-0.069444in}{0.000000in}}%
\pgfusepath{stroke,fill}%
}%
\begin{pgfscope}%
\pgfsys@transformshift{2.561650in}{1.449069in}%
\pgfsys@useobject{currentmarker}{}%
\end{pgfscope}%
\end{pgfscope}%
\begin{pgfscope}%
\pgftext[x=0.297396in,y=1.449069in,right,]{\rmfamily\fontsize{8.000000}{9.600000}\selectfont 0.12}%
\end{pgfscope}%
\begin{pgfscope}%
\pgfsetbuttcap%
\pgfsetroundjoin%
\definecolor{currentfill}{rgb}{0.000000,0.000000,0.000000}%
\pgfsetfillcolor{currentfill}%
\pgfsetlinewidth{0.501875pt}%
\definecolor{currentstroke}{rgb}{0.000000,0.000000,0.000000}%
\pgfsetstrokecolor{currentstroke}%
\pgfsetdash{}{0pt}%
\pgfsys@defobject{currentmarker}{\pgfqpoint{0.000000in}{0.000000in}}{\pgfqpoint{0.069444in}{0.000000in}}{%
\pgfpathmoveto{\pgfqpoint{0.000000in}{0.000000in}}%
\pgfpathlineto{\pgfqpoint{0.069444in}{0.000000in}}%
\pgfusepath{stroke,fill}%
}%
\begin{pgfscope}%
\pgfsys@transformshift{0.366840in}{1.615583in}%
\pgfsys@useobject{currentmarker}{}%
\end{pgfscope}%
\end{pgfscope}%
\begin{pgfscope}%
\pgfsetbuttcap%
\pgfsetroundjoin%
\definecolor{currentfill}{rgb}{0.000000,0.000000,0.000000}%
\pgfsetfillcolor{currentfill}%
\pgfsetlinewidth{0.501875pt}%
\definecolor{currentstroke}{rgb}{0.000000,0.000000,0.000000}%
\pgfsetstrokecolor{currentstroke}%
\pgfsetdash{}{0pt}%
\pgfsys@defobject{currentmarker}{\pgfqpoint{-0.069444in}{0.000000in}}{\pgfqpoint{0.000000in}{0.000000in}}{%
\pgfpathmoveto{\pgfqpoint{0.000000in}{0.000000in}}%
\pgfpathlineto{\pgfqpoint{-0.069444in}{0.000000in}}%
\pgfusepath{stroke,fill}%
}%
\begin{pgfscope}%
\pgfsys@transformshift{2.561650in}{1.615583in}%
\pgfsys@useobject{currentmarker}{}%
\end{pgfscope}%
\end{pgfscope}%
\begin{pgfscope}%
\pgftext[x=0.297396in,y=1.615583in,right,]{\rmfamily\fontsize{8.000000}{9.600000}\selectfont 0.14}%
\end{pgfscope}%
\end{pgfpicture}%
\makeatother%
\endgroup%

	\end{subfigure}

	\begin{subfigure}[t]{0.49\textwidth}
		\centering
    %\includegraphics[width=\textwidth]{store/variables/DATA_MC_REWEIGHTED_piminus_PIDK.pdf}
    %% Creator: Matplotlib, PGF backend
%%
%% To include the figure in your LaTeX document, write
%%   \input{<filename>.pgf}
%%
%% Make sure the required packages are loaded in your preamble
%%   \usepackage{pgf}
%%
%% Figures using additional raster images can only be included by \input if
%% they are in the same directory as the main LaTeX file. For loading figures
%% from other directories you can use the `import` package
%%   \usepackage{import}
%% and then include the figures with
%%   \import{<path to file>}{<filename>.pgf}
%%
%% Matplotlib used the following preamble
%%   \usepackage{fontspec}
%%   \setmainfont{DejaVu Serif}
%%   \setsansfont{DejaVu Sans}
%%   \setmonofont{DejaVu Sans Mono}
%%
\begingroup%
\makeatletter%
\begin{pgfpicture}%
\pgfpathrectangle{\pgfpointorigin}{\pgfqpoint{2.679091in}{1.723197in}}%
\pgfusepath{use as bounding box, clip}%
\begin{pgfscope}%
\pgfsetbuttcap%
\pgfsetmiterjoin%
\definecolor{currentfill}{rgb}{1.000000,1.000000,1.000000}%
\pgfsetfillcolor{currentfill}%
\pgfsetlinewidth{0.000000pt}%
\definecolor{currentstroke}{rgb}{1.000000,1.000000,1.000000}%
\pgfsetstrokecolor{currentstroke}%
\pgfsetdash{}{0pt}%
\pgfpathmoveto{\pgfqpoint{0.000000in}{0.000000in}}%
\pgfpathlineto{\pgfqpoint{2.679091in}{0.000000in}}%
\pgfpathlineto{\pgfqpoint{2.679091in}{1.723197in}}%
\pgfpathlineto{\pgfqpoint{0.000000in}{1.723197in}}%
\pgfpathclose%
\pgfusepath{fill}%
\end{pgfscope}%
\begin{pgfscope}%
\pgfsetbuttcap%
\pgfsetmiterjoin%
\definecolor{currentfill}{rgb}{1.000000,1.000000,1.000000}%
\pgfsetfillcolor{currentfill}%
\pgfsetlinewidth{0.000000pt}%
\definecolor{currentstroke}{rgb}{0.000000,0.000000,0.000000}%
\pgfsetstrokecolor{currentstroke}%
\pgfsetstrokeopacity{0.000000}%
\pgfsetdash{}{0pt}%
\pgfpathmoveto{\pgfqpoint{0.437532in}{0.449983in}}%
\pgfpathlineto{\pgfqpoint{2.558398in}{0.449983in}}%
\pgfpathlineto{\pgfqpoint{2.558398in}{1.619432in}}%
\pgfpathlineto{\pgfqpoint{0.437532in}{1.619432in}}%
\pgfpathclose%
\pgfusepath{fill}%
\end{pgfscope}%
\begin{pgfscope}%
\pgfpathrectangle{\pgfqpoint{0.437532in}{0.449983in}}{\pgfqpoint{2.120866in}{1.169449in}} %
\pgfusepath{clip}%
\pgfsetbuttcap%
\pgfsetmiterjoin%
\definecolor{currentfill}{rgb}{0.215686,0.470588,0.749020}%
\pgfsetfillcolor{currentfill}%
\pgfsetlinewidth{0.000000pt}%
\definecolor{currentstroke}{rgb}{0.000000,0.000000,0.000000}%
\pgfsetstrokecolor{currentstroke}%
\pgfsetdash{}{0pt}%
\pgfpathmoveto{\pgfqpoint{0.555358in}{0.449983in}}%
\pgfpathlineto{\pgfqpoint{0.555358in}{0.450406in}}%
\pgfpathlineto{\pgfqpoint{0.593063in}{0.450406in}}%
\pgfpathlineto{\pgfqpoint{0.593063in}{0.450425in}}%
\pgfpathlineto{\pgfqpoint{0.630767in}{0.450425in}}%
\pgfpathlineto{\pgfqpoint{0.630767in}{0.450560in}}%
\pgfpathlineto{\pgfqpoint{0.668471in}{0.450560in}}%
\pgfpathlineto{\pgfqpoint{0.668471in}{0.451099in}}%
\pgfpathlineto{\pgfqpoint{0.706176in}{0.451099in}}%
\pgfpathlineto{\pgfqpoint{0.706176in}{0.451351in}}%
\pgfpathlineto{\pgfqpoint{0.743880in}{0.451351in}}%
\pgfpathlineto{\pgfqpoint{0.743880in}{0.450963in}}%
\pgfpathlineto{\pgfqpoint{0.781584in}{0.450963in}}%
\pgfpathlineto{\pgfqpoint{0.781584in}{0.451936in}}%
\pgfpathlineto{\pgfqpoint{0.819288in}{0.451936in}}%
\pgfpathlineto{\pgfqpoint{0.819288in}{0.451935in}}%
\pgfpathlineto{\pgfqpoint{0.856993in}{0.451935in}}%
\pgfpathlineto{\pgfqpoint{0.856993in}{0.453389in}}%
\pgfpathlineto{\pgfqpoint{0.894697in}{0.453389in}}%
\pgfpathlineto{\pgfqpoint{0.894697in}{0.453942in}}%
\pgfpathlineto{\pgfqpoint{0.932401in}{0.453942in}}%
\pgfpathlineto{\pgfqpoint{0.932401in}{0.455792in}}%
\pgfpathlineto{\pgfqpoint{0.970105in}{0.455792in}}%
\pgfpathlineto{\pgfqpoint{0.970105in}{0.456593in}}%
\pgfpathlineto{\pgfqpoint{1.007810in}{0.456593in}}%
\pgfpathlineto{\pgfqpoint{1.007810in}{0.458468in}}%
\pgfpathlineto{\pgfqpoint{1.045514in}{0.458468in}}%
\pgfpathlineto{\pgfqpoint{1.045514in}{0.461035in}}%
\pgfpathlineto{\pgfqpoint{1.083218in}{0.461035in}}%
\pgfpathlineto{\pgfqpoint{1.083218in}{0.464682in}}%
\pgfpathlineto{\pgfqpoint{1.120923in}{0.464682in}}%
\pgfpathlineto{\pgfqpoint{1.120923in}{0.468009in}}%
\pgfpathlineto{\pgfqpoint{1.158627in}{0.468009in}}%
\pgfpathlineto{\pgfqpoint{1.158627in}{0.474040in}}%
\pgfpathlineto{\pgfqpoint{1.196331in}{0.474040in}}%
\pgfpathlineto{\pgfqpoint{1.196331in}{0.480099in}}%
\pgfpathlineto{\pgfqpoint{1.234035in}{0.480099in}}%
\pgfpathlineto{\pgfqpoint{1.234035in}{0.488526in}}%
\pgfpathlineto{\pgfqpoint{1.271740in}{0.488526in}}%
\pgfpathlineto{\pgfqpoint{1.271740in}{0.497020in}}%
\pgfpathlineto{\pgfqpoint{1.309444in}{0.497020in}}%
\pgfpathlineto{\pgfqpoint{1.309444in}{0.509527in}}%
\pgfpathlineto{\pgfqpoint{1.347148in}{0.509527in}}%
\pgfpathlineto{\pgfqpoint{1.347148in}{0.523731in}}%
\pgfpathlineto{\pgfqpoint{1.384853in}{0.523731in}}%
\pgfpathlineto{\pgfqpoint{1.384853in}{0.539616in}}%
\pgfpathlineto{\pgfqpoint{1.422557in}{0.539616in}}%
\pgfpathlineto{\pgfqpoint{1.422557in}{0.563719in}}%
\pgfpathlineto{\pgfqpoint{1.460261in}{0.563719in}}%
\pgfpathlineto{\pgfqpoint{1.460261in}{0.586826in}}%
\pgfpathlineto{\pgfqpoint{1.497965in}{0.586826in}}%
\pgfpathlineto{\pgfqpoint{1.497965in}{0.608594in}}%
\pgfpathlineto{\pgfqpoint{1.535670in}{0.608594in}}%
\pgfpathlineto{\pgfqpoint{1.535670in}{0.641102in}}%
\pgfpathlineto{\pgfqpoint{1.573374in}{0.641102in}}%
\pgfpathlineto{\pgfqpoint{1.573374in}{0.679779in}}%
\pgfpathlineto{\pgfqpoint{1.611078in}{0.679779in}}%
\pgfpathlineto{\pgfqpoint{1.611078in}{0.711761in}}%
\pgfpathlineto{\pgfqpoint{1.648783in}{0.711761in}}%
\pgfpathlineto{\pgfqpoint{1.648783in}{0.745005in}}%
\pgfpathlineto{\pgfqpoint{1.686487in}{0.745005in}}%
\pgfpathlineto{\pgfqpoint{1.686487in}{0.787138in}}%
\pgfpathlineto{\pgfqpoint{1.724191in}{0.787138in}}%
\pgfpathlineto{\pgfqpoint{1.724191in}{0.822458in}}%
\pgfpathlineto{\pgfqpoint{1.761895in}{0.822458in}}%
\pgfpathlineto{\pgfqpoint{1.761895in}{0.858762in}}%
\pgfpathlineto{\pgfqpoint{1.799600in}{0.858762in}}%
\pgfpathlineto{\pgfqpoint{1.799600in}{0.896891in}}%
\pgfpathlineto{\pgfqpoint{1.837304in}{0.896891in}}%
\pgfpathlineto{\pgfqpoint{1.837304in}{0.929861in}}%
\pgfpathlineto{\pgfqpoint{1.875008in}{0.929861in}}%
\pgfpathlineto{\pgfqpoint{1.875008in}{0.984436in}}%
\pgfpathlineto{\pgfqpoint{1.912713in}{0.984436in}}%
\pgfpathlineto{\pgfqpoint{1.912713in}{1.086075in}}%
\pgfpathlineto{\pgfqpoint{1.950417in}{1.086075in}}%
\pgfpathlineto{\pgfqpoint{1.950417in}{1.220349in}}%
\pgfpathlineto{\pgfqpoint{1.988121in}{1.220349in}}%
\pgfpathlineto{\pgfqpoint{1.988121in}{1.364848in}}%
\pgfpathlineto{\pgfqpoint{2.025825in}{1.364848in}}%
\pgfpathlineto{\pgfqpoint{2.025825in}{1.449771in}}%
\pgfpathlineto{\pgfqpoint{2.063530in}{1.449771in}}%
\pgfpathlineto{\pgfqpoint{2.063530in}{1.495124in}}%
\pgfpathlineto{\pgfqpoint{2.101234in}{1.495124in}}%
\pgfpathlineto{\pgfqpoint{2.101234in}{1.211253in}}%
\pgfpathlineto{\pgfqpoint{2.138938in}{1.211253in}}%
\pgfpathlineto{\pgfqpoint{2.138938in}{0.915924in}}%
\pgfpathlineto{\pgfqpoint{2.176643in}{0.915924in}}%
\pgfpathlineto{\pgfqpoint{2.176643in}{0.703737in}}%
\pgfpathlineto{\pgfqpoint{2.214347in}{0.703737in}}%
\pgfpathlineto{\pgfqpoint{2.214347in}{0.550351in}}%
\pgfpathlineto{\pgfqpoint{2.252051in}{0.550351in}}%
\pgfpathlineto{\pgfqpoint{2.252051in}{0.486603in}}%
\pgfpathlineto{\pgfqpoint{2.289755in}{0.486603in}}%
\pgfpathlineto{\pgfqpoint{2.289755in}{0.470472in}}%
\pgfpathlineto{\pgfqpoint{2.327460in}{0.470472in}}%
\pgfpathlineto{\pgfqpoint{2.327460in}{0.461750in}}%
\pgfpathlineto{\pgfqpoint{2.365164in}{0.461750in}}%
\pgfpathlineto{\pgfqpoint{2.365164in}{0.458453in}}%
\pgfpathlineto{\pgfqpoint{2.402868in}{0.458453in}}%
\pgfpathlineto{\pgfqpoint{2.402868in}{0.456494in}}%
\pgfpathlineto{\pgfqpoint{2.440573in}{0.456494in}}%
\pgfpathlineto{\pgfqpoint{2.440573in}{0.449983in}}%
\pgfpathlineto{\pgfqpoint{2.402868in}{0.449983in}}%
\pgfpathlineto{\pgfqpoint{2.402868in}{0.449983in}}%
\pgfpathlineto{\pgfqpoint{2.365164in}{0.449983in}}%
\pgfpathlineto{\pgfqpoint{2.365164in}{0.449983in}}%
\pgfpathlineto{\pgfqpoint{2.327460in}{0.449983in}}%
\pgfpathlineto{\pgfqpoint{2.327460in}{0.449983in}}%
\pgfpathlineto{\pgfqpoint{2.289755in}{0.449983in}}%
\pgfpathlineto{\pgfqpoint{2.289755in}{0.449983in}}%
\pgfpathlineto{\pgfqpoint{2.252051in}{0.449983in}}%
\pgfpathlineto{\pgfqpoint{2.252051in}{0.449983in}}%
\pgfpathlineto{\pgfqpoint{2.214347in}{0.449983in}}%
\pgfpathlineto{\pgfqpoint{2.214347in}{0.449983in}}%
\pgfpathlineto{\pgfqpoint{2.176643in}{0.449983in}}%
\pgfpathlineto{\pgfqpoint{2.176643in}{0.449983in}}%
\pgfpathlineto{\pgfqpoint{2.138938in}{0.449983in}}%
\pgfpathlineto{\pgfqpoint{2.138938in}{0.449983in}}%
\pgfpathlineto{\pgfqpoint{2.101234in}{0.449983in}}%
\pgfpathlineto{\pgfqpoint{2.101234in}{0.449983in}}%
\pgfpathlineto{\pgfqpoint{2.063530in}{0.449983in}}%
\pgfpathlineto{\pgfqpoint{2.063530in}{0.449983in}}%
\pgfpathlineto{\pgfqpoint{2.025825in}{0.449983in}}%
\pgfpathlineto{\pgfqpoint{2.025825in}{0.449983in}}%
\pgfpathlineto{\pgfqpoint{1.988121in}{0.449983in}}%
\pgfpathlineto{\pgfqpoint{1.988121in}{0.449983in}}%
\pgfpathlineto{\pgfqpoint{1.950417in}{0.449983in}}%
\pgfpathlineto{\pgfqpoint{1.950417in}{0.449983in}}%
\pgfpathlineto{\pgfqpoint{1.912713in}{0.449983in}}%
\pgfpathlineto{\pgfqpoint{1.912713in}{0.449983in}}%
\pgfpathlineto{\pgfqpoint{1.875008in}{0.449983in}}%
\pgfpathlineto{\pgfqpoint{1.875008in}{0.449983in}}%
\pgfpathlineto{\pgfqpoint{1.837304in}{0.449983in}}%
\pgfpathlineto{\pgfqpoint{1.837304in}{0.449983in}}%
\pgfpathlineto{\pgfqpoint{1.799600in}{0.449983in}}%
\pgfpathlineto{\pgfqpoint{1.799600in}{0.449983in}}%
\pgfpathlineto{\pgfqpoint{1.761895in}{0.449983in}}%
\pgfpathlineto{\pgfqpoint{1.761895in}{0.449983in}}%
\pgfpathlineto{\pgfqpoint{1.724191in}{0.449983in}}%
\pgfpathlineto{\pgfqpoint{1.724191in}{0.449983in}}%
\pgfpathlineto{\pgfqpoint{1.686487in}{0.449983in}}%
\pgfpathlineto{\pgfqpoint{1.686487in}{0.449983in}}%
\pgfpathlineto{\pgfqpoint{1.648783in}{0.449983in}}%
\pgfpathlineto{\pgfqpoint{1.648783in}{0.449983in}}%
\pgfpathlineto{\pgfqpoint{1.611078in}{0.449983in}}%
\pgfpathlineto{\pgfqpoint{1.611078in}{0.449983in}}%
\pgfpathlineto{\pgfqpoint{1.573374in}{0.449983in}}%
\pgfpathlineto{\pgfqpoint{1.573374in}{0.449983in}}%
\pgfpathlineto{\pgfqpoint{1.535670in}{0.449983in}}%
\pgfpathlineto{\pgfqpoint{1.535670in}{0.449983in}}%
\pgfpathlineto{\pgfqpoint{1.497965in}{0.449983in}}%
\pgfpathlineto{\pgfqpoint{1.497965in}{0.449983in}}%
\pgfpathlineto{\pgfqpoint{1.460261in}{0.449983in}}%
\pgfpathlineto{\pgfqpoint{1.460261in}{0.449983in}}%
\pgfpathlineto{\pgfqpoint{1.422557in}{0.449983in}}%
\pgfpathlineto{\pgfqpoint{1.422557in}{0.449983in}}%
\pgfpathlineto{\pgfqpoint{1.384853in}{0.449983in}}%
\pgfpathlineto{\pgfqpoint{1.384853in}{0.449983in}}%
\pgfpathlineto{\pgfqpoint{1.347148in}{0.449983in}}%
\pgfpathlineto{\pgfqpoint{1.347148in}{0.449983in}}%
\pgfpathlineto{\pgfqpoint{1.309444in}{0.449983in}}%
\pgfpathlineto{\pgfqpoint{1.309444in}{0.449983in}}%
\pgfpathlineto{\pgfqpoint{1.271740in}{0.449983in}}%
\pgfpathlineto{\pgfqpoint{1.271740in}{0.449983in}}%
\pgfpathlineto{\pgfqpoint{1.234035in}{0.449983in}}%
\pgfpathlineto{\pgfqpoint{1.234035in}{0.449983in}}%
\pgfpathlineto{\pgfqpoint{1.196331in}{0.449983in}}%
\pgfpathlineto{\pgfqpoint{1.196331in}{0.449983in}}%
\pgfpathlineto{\pgfqpoint{1.158627in}{0.449983in}}%
\pgfpathlineto{\pgfqpoint{1.158627in}{0.449983in}}%
\pgfpathlineto{\pgfqpoint{1.120923in}{0.449983in}}%
\pgfpathlineto{\pgfqpoint{1.120923in}{0.449983in}}%
\pgfpathlineto{\pgfqpoint{1.083218in}{0.449983in}}%
\pgfpathlineto{\pgfqpoint{1.083218in}{0.449983in}}%
\pgfpathlineto{\pgfqpoint{1.045514in}{0.449983in}}%
\pgfpathlineto{\pgfqpoint{1.045514in}{0.449983in}}%
\pgfpathlineto{\pgfqpoint{1.007810in}{0.449983in}}%
\pgfpathlineto{\pgfqpoint{1.007810in}{0.449983in}}%
\pgfpathlineto{\pgfqpoint{0.970105in}{0.449983in}}%
\pgfpathlineto{\pgfqpoint{0.970105in}{0.449983in}}%
\pgfpathlineto{\pgfqpoint{0.932401in}{0.449983in}}%
\pgfpathlineto{\pgfqpoint{0.932401in}{0.449983in}}%
\pgfpathlineto{\pgfqpoint{0.894697in}{0.449983in}}%
\pgfpathlineto{\pgfqpoint{0.894697in}{0.449983in}}%
\pgfpathlineto{\pgfqpoint{0.856993in}{0.449983in}}%
\pgfpathlineto{\pgfqpoint{0.856993in}{0.449983in}}%
\pgfpathlineto{\pgfqpoint{0.819288in}{0.449983in}}%
\pgfpathlineto{\pgfqpoint{0.819288in}{0.449983in}}%
\pgfpathlineto{\pgfqpoint{0.781584in}{0.449983in}}%
\pgfpathlineto{\pgfqpoint{0.781584in}{0.449983in}}%
\pgfpathlineto{\pgfqpoint{0.743880in}{0.449983in}}%
\pgfpathlineto{\pgfqpoint{0.743880in}{0.449983in}}%
\pgfpathlineto{\pgfqpoint{0.706176in}{0.449983in}}%
\pgfpathlineto{\pgfqpoint{0.706176in}{0.449983in}}%
\pgfpathlineto{\pgfqpoint{0.668471in}{0.449983in}}%
\pgfpathlineto{\pgfqpoint{0.668471in}{0.449983in}}%
\pgfpathlineto{\pgfqpoint{0.630767in}{0.449983in}}%
\pgfpathlineto{\pgfqpoint{0.630767in}{0.449983in}}%
\pgfpathlineto{\pgfqpoint{0.593063in}{0.449983in}}%
\pgfpathlineto{\pgfqpoint{0.593063in}{0.449983in}}%
\pgfpathlineto{\pgfqpoint{0.555358in}{0.449983in}}%
\pgfusepath{fill}%
\end{pgfscope}%
\begin{pgfscope}%
\pgfpathrectangle{\pgfqpoint{0.437532in}{0.449983in}}{\pgfqpoint{2.120866in}{1.169449in}} %
\pgfusepath{clip}%
\pgfsetbuttcap%
\pgfsetmiterjoin%
\pgfsetlinewidth{0.501875pt}%
\definecolor{currentstroke}{rgb}{1.000000,0.000000,0.000000}%
\pgfsetstrokecolor{currentstroke}%
\pgfsetdash{}{0pt}%
\pgfpathmoveto{\pgfqpoint{0.555358in}{0.449983in}}%
\pgfpathlineto{\pgfqpoint{0.555358in}{0.452711in}}%
\pgfpathlineto{\pgfqpoint{0.593063in}{0.452711in}}%
\pgfpathlineto{\pgfqpoint{0.593063in}{0.454232in}}%
\pgfpathlineto{\pgfqpoint{0.630767in}{0.454232in}}%
\pgfpathlineto{\pgfqpoint{0.630767in}{0.454494in}}%
\pgfpathlineto{\pgfqpoint{0.668471in}{0.454494in}}%
\pgfpathlineto{\pgfqpoint{0.668471in}{0.456041in}}%
\pgfpathlineto{\pgfqpoint{0.706176in}{0.456041in}}%
\pgfpathlineto{\pgfqpoint{0.706176in}{0.458375in}}%
\pgfpathlineto{\pgfqpoint{0.743880in}{0.458375in}}%
\pgfpathlineto{\pgfqpoint{0.743880in}{0.458900in}}%
\pgfpathlineto{\pgfqpoint{0.781584in}{0.458900in}}%
\pgfpathlineto{\pgfqpoint{0.781584in}{0.462125in}}%
\pgfpathlineto{\pgfqpoint{0.819288in}{0.462125in}}%
\pgfpathlineto{\pgfqpoint{0.819288in}{0.463725in}}%
\pgfpathlineto{\pgfqpoint{0.856993in}{0.463725in}}%
\pgfpathlineto{\pgfqpoint{0.856993in}{0.466662in}}%
\pgfpathlineto{\pgfqpoint{0.894697in}{0.466662in}}%
\pgfpathlineto{\pgfqpoint{0.894697in}{0.471540in}}%
\pgfpathlineto{\pgfqpoint{0.932401in}{0.471540in}}%
\pgfpathlineto{\pgfqpoint{0.932401in}{0.475762in}}%
\pgfpathlineto{\pgfqpoint{0.970105in}{0.475762in}}%
\pgfpathlineto{\pgfqpoint{0.970105in}{0.481112in}}%
\pgfpathlineto{\pgfqpoint{1.007810in}{0.481112in}}%
\pgfpathlineto{\pgfqpoint{1.007810in}{0.487773in}}%
\pgfpathlineto{\pgfqpoint{1.045514in}{0.487773in}}%
\pgfpathlineto{\pgfqpoint{1.045514in}{0.493883in}}%
\pgfpathlineto{\pgfqpoint{1.083218in}{0.493883in}}%
\pgfpathlineto{\pgfqpoint{1.083218in}{0.503927in}}%
\pgfpathlineto{\pgfqpoint{1.120923in}{0.503927in}}%
\pgfpathlineto{\pgfqpoint{1.120923in}{0.514837in}}%
\pgfpathlineto{\pgfqpoint{1.158627in}{0.514837in}}%
\pgfpathlineto{\pgfqpoint{1.158627in}{0.523334in}}%
\pgfpathlineto{\pgfqpoint{1.196331in}{0.523334in}}%
\pgfpathlineto{\pgfqpoint{1.196331in}{0.540668in}}%
\pgfpathlineto{\pgfqpoint{1.234035in}{0.540668in}}%
\pgfpathlineto{\pgfqpoint{1.234035in}{0.551499in}}%
\pgfpathlineto{\pgfqpoint{1.271740in}{0.551499in}}%
\pgfpathlineto{\pgfqpoint{1.271740in}{0.571666in}}%
\pgfpathlineto{\pgfqpoint{1.309444in}{0.571666in}}%
\pgfpathlineto{\pgfqpoint{1.309444in}{0.589289in}}%
\pgfpathlineto{\pgfqpoint{1.347148in}{0.589289in}}%
\pgfpathlineto{\pgfqpoint{1.347148in}{0.604368in}}%
\pgfpathlineto{\pgfqpoint{1.384853in}{0.604368in}}%
\pgfpathlineto{\pgfqpoint{1.384853in}{0.630042in}}%
\pgfpathlineto{\pgfqpoint{1.422557in}{0.630042in}}%
\pgfpathlineto{\pgfqpoint{1.422557in}{0.653067in}}%
\pgfpathlineto{\pgfqpoint{1.460261in}{0.653067in}}%
\pgfpathlineto{\pgfqpoint{1.460261in}{0.672971in}}%
\pgfpathlineto{\pgfqpoint{1.497965in}{0.672971in}}%
\pgfpathlineto{\pgfqpoint{1.497965in}{0.696678in}}%
\pgfpathlineto{\pgfqpoint{1.535670in}{0.696678in}}%
\pgfpathlineto{\pgfqpoint{1.535670in}{0.719703in}}%
\pgfpathlineto{\pgfqpoint{1.573374in}{0.719703in}}%
\pgfpathlineto{\pgfqpoint{1.573374in}{0.740998in}}%
\pgfpathlineto{\pgfqpoint{1.611078in}{0.740998in}}%
\pgfpathlineto{\pgfqpoint{1.611078in}{0.761427in}}%
\pgfpathlineto{\pgfqpoint{1.648783in}{0.761427in}}%
\pgfpathlineto{\pgfqpoint{1.648783in}{0.777738in}}%
\pgfpathlineto{\pgfqpoint{1.686487in}{0.777738in}}%
\pgfpathlineto{\pgfqpoint{1.686487in}{0.793788in}}%
\pgfpathlineto{\pgfqpoint{1.724191in}{0.793788in}}%
\pgfpathlineto{\pgfqpoint{1.724191in}{0.812827in}}%
\pgfpathlineto{\pgfqpoint{1.761895in}{0.812827in}}%
\pgfpathlineto{\pgfqpoint{1.761895in}{0.834620in}}%
\pgfpathlineto{\pgfqpoint{1.799600in}{0.834620in}}%
\pgfpathlineto{\pgfqpoint{1.799600in}{0.856517in}}%
\pgfpathlineto{\pgfqpoint{1.837304in}{0.856517in}}%
\pgfpathlineto{\pgfqpoint{1.837304in}{0.885390in}}%
\pgfpathlineto{\pgfqpoint{1.875008in}{0.885390in}}%
\pgfpathlineto{\pgfqpoint{1.875008in}{0.921187in}}%
\pgfpathlineto{\pgfqpoint{1.912713in}{0.921187in}}%
\pgfpathlineto{\pgfqpoint{1.912713in}{0.994249in}}%
\pgfpathlineto{\pgfqpoint{1.950417in}{0.994249in}}%
\pgfpathlineto{\pgfqpoint{1.950417in}{1.078692in}}%
\pgfpathlineto{\pgfqpoint{1.988121in}{1.078692in}}%
\pgfpathlineto{\pgfqpoint{1.988121in}{1.159306in}}%
\pgfpathlineto{\pgfqpoint{2.025825in}{1.159306in}}%
\pgfpathlineto{\pgfqpoint{2.025825in}{1.225681in}}%
\pgfpathlineto{\pgfqpoint{2.063530in}{1.225681in}}%
\pgfpathlineto{\pgfqpoint{2.063530in}{1.322790in}}%
\pgfpathlineto{\pgfqpoint{2.101234in}{1.322790in}}%
\pgfpathlineto{\pgfqpoint{2.101234in}{1.107460in}}%
\pgfpathlineto{\pgfqpoint{2.138938in}{1.107460in}}%
\pgfpathlineto{\pgfqpoint{2.138938in}{0.860451in}}%
\pgfpathlineto{\pgfqpoint{2.176643in}{0.860451in}}%
\pgfpathlineto{\pgfqpoint{2.176643in}{0.658941in}}%
\pgfpathlineto{\pgfqpoint{2.214347in}{0.658941in}}%
\pgfpathlineto{\pgfqpoint{2.214347in}{0.526035in}}%
\pgfpathlineto{\pgfqpoint{2.252051in}{0.526035in}}%
\pgfpathlineto{\pgfqpoint{2.252051in}{0.484128in}}%
\pgfpathlineto{\pgfqpoint{2.289755in}{0.484128in}}%
\pgfpathlineto{\pgfqpoint{2.289755in}{0.466715in}}%
\pgfpathlineto{\pgfqpoint{2.327460in}{0.466715in}}%
\pgfpathlineto{\pgfqpoint{2.327460in}{0.457877in}}%
\pgfpathlineto{\pgfqpoint{2.365164in}{0.457877in}}%
\pgfpathlineto{\pgfqpoint{2.365164in}{0.453393in}}%
\pgfpathlineto{\pgfqpoint{2.402868in}{0.453393in}}%
\pgfpathlineto{\pgfqpoint{2.402868in}{0.451085in}}%
\pgfpathlineto{\pgfqpoint{2.440573in}{0.451085in}}%
\pgfpathlineto{\pgfqpoint{2.440573in}{0.449983in}}%
\pgfusepath{stroke}%
\end{pgfscope}%
\begin{pgfscope}%
\pgfpathrectangle{\pgfqpoint{0.437532in}{0.449983in}}{\pgfqpoint{2.120866in}{1.169449in}} %
\pgfusepath{clip}%
\pgfsetbuttcap%
\pgfsetmiterjoin%
\pgfsetlinewidth{1.003750pt}%
\definecolor{currentstroke}{rgb}{1.000000,0.647059,0.000000}%
\pgfsetstrokecolor{currentstroke}%
\pgfsetdash{}{0pt}%
\pgfpathmoveto{\pgfqpoint{0.555358in}{0.449983in}}%
\pgfpathlineto{\pgfqpoint{0.555358in}{0.450417in}}%
\pgfpathlineto{\pgfqpoint{0.593063in}{0.450417in}}%
\pgfpathlineto{\pgfqpoint{0.593063in}{0.450689in}}%
\pgfpathlineto{\pgfqpoint{0.630767in}{0.450689in}}%
\pgfpathlineto{\pgfqpoint{0.630767in}{0.450767in}}%
\pgfpathlineto{\pgfqpoint{0.668471in}{0.450767in}}%
\pgfpathlineto{\pgfqpoint{0.668471in}{0.451053in}}%
\pgfpathlineto{\pgfqpoint{0.706176in}{0.451053in}}%
\pgfpathlineto{\pgfqpoint{0.706176in}{0.451517in}}%
\pgfpathlineto{\pgfqpoint{0.743880in}{0.451517in}}%
\pgfpathlineto{\pgfqpoint{0.743880in}{0.451650in}}%
\pgfpathlineto{\pgfqpoint{0.781584in}{0.451650in}}%
\pgfpathlineto{\pgfqpoint{0.781584in}{0.452216in}}%
\pgfpathlineto{\pgfqpoint{0.819288in}{0.452216in}}%
\pgfpathlineto{\pgfqpoint{0.819288in}{0.452658in}}%
\pgfpathlineto{\pgfqpoint{0.856993in}{0.452658in}}%
\pgfpathlineto{\pgfqpoint{0.856993in}{0.453698in}}%
\pgfpathlineto{\pgfqpoint{0.894697in}{0.453698in}}%
\pgfpathlineto{\pgfqpoint{0.894697in}{0.454885in}}%
\pgfpathlineto{\pgfqpoint{0.932401in}{0.454885in}}%
\pgfpathlineto{\pgfqpoint{0.932401in}{0.456214in}}%
\pgfpathlineto{\pgfqpoint{0.970105in}{0.456214in}}%
\pgfpathlineto{\pgfqpoint{0.970105in}{0.457334in}}%
\pgfpathlineto{\pgfqpoint{1.007810in}{0.457334in}}%
\pgfpathlineto{\pgfqpoint{1.007810in}{0.459339in}}%
\pgfpathlineto{\pgfqpoint{1.045514in}{0.459339in}}%
\pgfpathlineto{\pgfqpoint{1.045514in}{0.461670in}}%
\pgfpathlineto{\pgfqpoint{1.083218in}{0.461670in}}%
\pgfpathlineto{\pgfqpoint{1.083218in}{0.465360in}}%
\pgfpathlineto{\pgfqpoint{1.120923in}{0.465360in}}%
\pgfpathlineto{\pgfqpoint{1.120923in}{0.469694in}}%
\pgfpathlineto{\pgfqpoint{1.158627in}{0.469694in}}%
\pgfpathlineto{\pgfqpoint{1.158627in}{0.474254in}}%
\pgfpathlineto{\pgfqpoint{1.196331in}{0.474254in}}%
\pgfpathlineto{\pgfqpoint{1.196331in}{0.481087in}}%
\pgfpathlineto{\pgfqpoint{1.234035in}{0.481087in}}%
\pgfpathlineto{\pgfqpoint{1.234035in}{0.489331in}}%
\pgfpathlineto{\pgfqpoint{1.271740in}{0.489331in}}%
\pgfpathlineto{\pgfqpoint{1.271740in}{0.499984in}}%
\pgfpathlineto{\pgfqpoint{1.309444in}{0.499984in}}%
\pgfpathlineto{\pgfqpoint{1.309444in}{0.511701in}}%
\pgfpathlineto{\pgfqpoint{1.347148in}{0.511701in}}%
\pgfpathlineto{\pgfqpoint{1.347148in}{0.523837in}}%
\pgfpathlineto{\pgfqpoint{1.384853in}{0.523837in}}%
\pgfpathlineto{\pgfqpoint{1.384853in}{0.542389in}}%
\pgfpathlineto{\pgfqpoint{1.422557in}{0.542389in}}%
\pgfpathlineto{\pgfqpoint{1.422557in}{0.564969in}}%
\pgfpathlineto{\pgfqpoint{1.460261in}{0.564969in}}%
\pgfpathlineto{\pgfqpoint{1.460261in}{0.588358in}}%
\pgfpathlineto{\pgfqpoint{1.497965in}{0.588358in}}%
\pgfpathlineto{\pgfqpoint{1.497965in}{0.611880in}}%
\pgfpathlineto{\pgfqpoint{1.535670in}{0.611880in}}%
\pgfpathlineto{\pgfqpoint{1.535670in}{0.642117in}}%
\pgfpathlineto{\pgfqpoint{1.573374in}{0.642117in}}%
\pgfpathlineto{\pgfqpoint{1.573374in}{0.681354in}}%
\pgfpathlineto{\pgfqpoint{1.611078in}{0.681354in}}%
\pgfpathlineto{\pgfqpoint{1.611078in}{0.714395in}}%
\pgfpathlineto{\pgfqpoint{1.648783in}{0.714395in}}%
\pgfpathlineto{\pgfqpoint{1.648783in}{0.748851in}}%
\pgfpathlineto{\pgfqpoint{1.686487in}{0.748851in}}%
\pgfpathlineto{\pgfqpoint{1.686487in}{0.788744in}}%
\pgfpathlineto{\pgfqpoint{1.724191in}{0.788744in}}%
\pgfpathlineto{\pgfqpoint{1.724191in}{0.823672in}}%
\pgfpathlineto{\pgfqpoint{1.761895in}{0.823672in}}%
\pgfpathlineto{\pgfqpoint{1.761895in}{0.861087in}}%
\pgfpathlineto{\pgfqpoint{1.799600in}{0.861087in}}%
\pgfpathlineto{\pgfqpoint{1.799600in}{0.900412in}}%
\pgfpathlineto{\pgfqpoint{1.837304in}{0.900412in}}%
\pgfpathlineto{\pgfqpoint{1.837304in}{0.934159in}}%
\pgfpathlineto{\pgfqpoint{1.875008in}{0.934159in}}%
\pgfpathlineto{\pgfqpoint{1.875008in}{0.983027in}}%
\pgfpathlineto{\pgfqpoint{1.912713in}{0.983027in}}%
\pgfpathlineto{\pgfqpoint{1.912713in}{1.079936in}}%
\pgfpathlineto{\pgfqpoint{1.950417in}{1.079936in}}%
\pgfpathlineto{\pgfqpoint{1.950417in}{1.210958in}}%
\pgfpathlineto{\pgfqpoint{1.988121in}{1.210958in}}%
\pgfpathlineto{\pgfqpoint{1.988121in}{1.324185in}}%
\pgfpathlineto{\pgfqpoint{2.025825in}{1.324185in}}%
\pgfpathlineto{\pgfqpoint{2.025825in}{1.427726in}}%
\pgfpathlineto{\pgfqpoint{2.063530in}{1.427726in}}%
\pgfpathlineto{\pgfqpoint{2.063530in}{1.493943in}}%
\pgfpathlineto{\pgfqpoint{2.101234in}{1.493943in}}%
\pgfpathlineto{\pgfqpoint{2.101234in}{1.223995in}}%
\pgfpathlineto{\pgfqpoint{2.138938in}{1.223995in}}%
\pgfpathlineto{\pgfqpoint{2.138938in}{0.943153in}}%
\pgfpathlineto{\pgfqpoint{2.176643in}{0.943153in}}%
\pgfpathlineto{\pgfqpoint{2.176643in}{0.706683in}}%
\pgfpathlineto{\pgfqpoint{2.214347in}{0.706683in}}%
\pgfpathlineto{\pgfqpoint{2.214347in}{0.543156in}}%
\pgfpathlineto{\pgfqpoint{2.252051in}{0.543156in}}%
\pgfpathlineto{\pgfqpoint{2.252051in}{0.491162in}}%
\pgfpathlineto{\pgfqpoint{2.289755in}{0.491162in}}%
\pgfpathlineto{\pgfqpoint{2.289755in}{0.470579in}}%
\pgfpathlineto{\pgfqpoint{2.327460in}{0.470579in}}%
\pgfpathlineto{\pgfqpoint{2.327460in}{0.460845in}}%
\pgfpathlineto{\pgfqpoint{2.365164in}{0.460845in}}%
\pgfpathlineto{\pgfqpoint{2.365164in}{0.457410in}}%
\pgfpathlineto{\pgfqpoint{2.402868in}{0.457410in}}%
\pgfpathlineto{\pgfqpoint{2.402868in}{0.452177in}}%
\pgfpathlineto{\pgfqpoint{2.440573in}{0.452177in}}%
\pgfpathlineto{\pgfqpoint{2.440573in}{0.449983in}}%
\pgfusepath{stroke}%
\end{pgfscope}%
\begin{pgfscope}%
\pgfsetrectcap%
\pgfsetmiterjoin%
\pgfsetlinewidth{1.003750pt}%
\definecolor{currentstroke}{rgb}{0.000000,0.000000,0.000000}%
\pgfsetstrokecolor{currentstroke}%
\pgfsetdash{}{0pt}%
\pgfpathmoveto{\pgfqpoint{0.437532in}{1.619432in}}%
\pgfpathlineto{\pgfqpoint{2.558398in}{1.619432in}}%
\pgfusepath{stroke}%
\end{pgfscope}%
\begin{pgfscope}%
\pgfsetrectcap%
\pgfsetmiterjoin%
\pgfsetlinewidth{1.003750pt}%
\definecolor{currentstroke}{rgb}{0.000000,0.000000,0.000000}%
\pgfsetstrokecolor{currentstroke}%
\pgfsetdash{}{0pt}%
\pgfpathmoveto{\pgfqpoint{2.558398in}{0.449983in}}%
\pgfpathlineto{\pgfqpoint{2.558398in}{1.619432in}}%
\pgfusepath{stroke}%
\end{pgfscope}%
\begin{pgfscope}%
\pgfsetrectcap%
\pgfsetmiterjoin%
\pgfsetlinewidth{1.003750pt}%
\definecolor{currentstroke}{rgb}{0.000000,0.000000,0.000000}%
\pgfsetstrokecolor{currentstroke}%
\pgfsetdash{}{0pt}%
\pgfpathmoveto{\pgfqpoint{0.437532in}{0.449983in}}%
\pgfpathlineto{\pgfqpoint{2.558398in}{0.449983in}}%
\pgfusepath{stroke}%
\end{pgfscope}%
\begin{pgfscope}%
\pgfsetrectcap%
\pgfsetmiterjoin%
\pgfsetlinewidth{1.003750pt}%
\definecolor{currentstroke}{rgb}{0.000000,0.000000,0.000000}%
\pgfsetstrokecolor{currentstroke}%
\pgfsetdash{}{0pt}%
\pgfpathmoveto{\pgfqpoint{0.437532in}{0.449983in}}%
\pgfpathlineto{\pgfqpoint{0.437532in}{1.619432in}}%
\pgfusepath{stroke}%
\end{pgfscope}%
\begin{pgfscope}%
\pgfsetbuttcap%
\pgfsetroundjoin%
\definecolor{currentfill}{rgb}{0.000000,0.000000,0.000000}%
\pgfsetfillcolor{currentfill}%
\pgfsetlinewidth{0.501875pt}%
\definecolor{currentstroke}{rgb}{0.000000,0.000000,0.000000}%
\pgfsetstrokecolor{currentstroke}%
\pgfsetdash{}{0pt}%
\pgfsys@defobject{currentmarker}{\pgfqpoint{0.000000in}{0.000000in}}{\pgfqpoint{0.000000in}{0.069444in}}{%
\pgfpathmoveto{\pgfqpoint{0.000000in}{0.000000in}}%
\pgfpathlineto{\pgfqpoint{0.000000in}{0.069444in}}%
\pgfusepath{stroke,fill}%
}%
\begin{pgfscope}%
\pgfsys@transformshift{0.437532in}{0.449983in}%
\pgfsys@useobject{currentmarker}{}%
\end{pgfscope}%
\end{pgfscope}%
\begin{pgfscope}%
\pgfsetbuttcap%
\pgfsetroundjoin%
\definecolor{currentfill}{rgb}{0.000000,0.000000,0.000000}%
\pgfsetfillcolor{currentfill}%
\pgfsetlinewidth{0.501875pt}%
\definecolor{currentstroke}{rgb}{0.000000,0.000000,0.000000}%
\pgfsetstrokecolor{currentstroke}%
\pgfsetdash{}{0pt}%
\pgfsys@defobject{currentmarker}{\pgfqpoint{0.000000in}{-0.069444in}}{\pgfqpoint{0.000000in}{0.000000in}}{%
\pgfpathmoveto{\pgfqpoint{0.000000in}{0.000000in}}%
\pgfpathlineto{\pgfqpoint{0.000000in}{-0.069444in}}%
\pgfusepath{stroke,fill}%
}%
\begin{pgfscope}%
\pgfsys@transformshift{0.437532in}{1.619432in}%
\pgfsys@useobject{currentmarker}{}%
\end{pgfscope}%
\end{pgfscope}%
\begin{pgfscope}%
\pgftext[x=0.437532in,y=0.380539in,,top]{\rmfamily\fontsize{8.000000}{9.600000}\selectfont −140}%
\end{pgfscope}%
\begin{pgfscope}%
\pgfsetbuttcap%
\pgfsetroundjoin%
\definecolor{currentfill}{rgb}{0.000000,0.000000,0.000000}%
\pgfsetfillcolor{currentfill}%
\pgfsetlinewidth{0.501875pt}%
\definecolor{currentstroke}{rgb}{0.000000,0.000000,0.000000}%
\pgfsetstrokecolor{currentstroke}%
\pgfsetdash{}{0pt}%
\pgfsys@defobject{currentmarker}{\pgfqpoint{0.000000in}{0.000000in}}{\pgfqpoint{0.000000in}{0.069444in}}{%
\pgfpathmoveto{\pgfqpoint{0.000000in}{0.000000in}}%
\pgfpathlineto{\pgfqpoint{0.000000in}{0.069444in}}%
\pgfusepath{stroke,fill}%
}%
\begin{pgfscope}%
\pgfsys@transformshift{0.673184in}{0.449983in}%
\pgfsys@useobject{currentmarker}{}%
\end{pgfscope}%
\end{pgfscope}%
\begin{pgfscope}%
\pgfsetbuttcap%
\pgfsetroundjoin%
\definecolor{currentfill}{rgb}{0.000000,0.000000,0.000000}%
\pgfsetfillcolor{currentfill}%
\pgfsetlinewidth{0.501875pt}%
\definecolor{currentstroke}{rgb}{0.000000,0.000000,0.000000}%
\pgfsetstrokecolor{currentstroke}%
\pgfsetdash{}{0pt}%
\pgfsys@defobject{currentmarker}{\pgfqpoint{0.000000in}{-0.069444in}}{\pgfqpoint{0.000000in}{0.000000in}}{%
\pgfpathmoveto{\pgfqpoint{0.000000in}{0.000000in}}%
\pgfpathlineto{\pgfqpoint{0.000000in}{-0.069444in}}%
\pgfusepath{stroke,fill}%
}%
\begin{pgfscope}%
\pgfsys@transformshift{0.673184in}{1.619432in}%
\pgfsys@useobject{currentmarker}{}%
\end{pgfscope}%
\end{pgfscope}%
\begin{pgfscope}%
\pgftext[x=0.673184in,y=0.380539in,,top]{\rmfamily\fontsize{8.000000}{9.600000}\selectfont −120}%
\end{pgfscope}%
\begin{pgfscope}%
\pgfsetbuttcap%
\pgfsetroundjoin%
\definecolor{currentfill}{rgb}{0.000000,0.000000,0.000000}%
\pgfsetfillcolor{currentfill}%
\pgfsetlinewidth{0.501875pt}%
\definecolor{currentstroke}{rgb}{0.000000,0.000000,0.000000}%
\pgfsetstrokecolor{currentstroke}%
\pgfsetdash{}{0pt}%
\pgfsys@defobject{currentmarker}{\pgfqpoint{0.000000in}{0.000000in}}{\pgfqpoint{0.000000in}{0.069444in}}{%
\pgfpathmoveto{\pgfqpoint{0.000000in}{0.000000in}}%
\pgfpathlineto{\pgfqpoint{0.000000in}{0.069444in}}%
\pgfusepath{stroke,fill}%
}%
\begin{pgfscope}%
\pgfsys@transformshift{0.908836in}{0.449983in}%
\pgfsys@useobject{currentmarker}{}%
\end{pgfscope}%
\end{pgfscope}%
\begin{pgfscope}%
\pgfsetbuttcap%
\pgfsetroundjoin%
\definecolor{currentfill}{rgb}{0.000000,0.000000,0.000000}%
\pgfsetfillcolor{currentfill}%
\pgfsetlinewidth{0.501875pt}%
\definecolor{currentstroke}{rgb}{0.000000,0.000000,0.000000}%
\pgfsetstrokecolor{currentstroke}%
\pgfsetdash{}{0pt}%
\pgfsys@defobject{currentmarker}{\pgfqpoint{0.000000in}{-0.069444in}}{\pgfqpoint{0.000000in}{0.000000in}}{%
\pgfpathmoveto{\pgfqpoint{0.000000in}{0.000000in}}%
\pgfpathlineto{\pgfqpoint{0.000000in}{-0.069444in}}%
\pgfusepath{stroke,fill}%
}%
\begin{pgfscope}%
\pgfsys@transformshift{0.908836in}{1.619432in}%
\pgfsys@useobject{currentmarker}{}%
\end{pgfscope}%
\end{pgfscope}%
\begin{pgfscope}%
\pgftext[x=0.908836in,y=0.380539in,,top]{\rmfamily\fontsize{8.000000}{9.600000}\selectfont −100}%
\end{pgfscope}%
\begin{pgfscope}%
\pgfsetbuttcap%
\pgfsetroundjoin%
\definecolor{currentfill}{rgb}{0.000000,0.000000,0.000000}%
\pgfsetfillcolor{currentfill}%
\pgfsetlinewidth{0.501875pt}%
\definecolor{currentstroke}{rgb}{0.000000,0.000000,0.000000}%
\pgfsetstrokecolor{currentstroke}%
\pgfsetdash{}{0pt}%
\pgfsys@defobject{currentmarker}{\pgfqpoint{0.000000in}{0.000000in}}{\pgfqpoint{0.000000in}{0.069444in}}{%
\pgfpathmoveto{\pgfqpoint{0.000000in}{0.000000in}}%
\pgfpathlineto{\pgfqpoint{0.000000in}{0.069444in}}%
\pgfusepath{stroke,fill}%
}%
\begin{pgfscope}%
\pgfsys@transformshift{1.144488in}{0.449983in}%
\pgfsys@useobject{currentmarker}{}%
\end{pgfscope}%
\end{pgfscope}%
\begin{pgfscope}%
\pgfsetbuttcap%
\pgfsetroundjoin%
\definecolor{currentfill}{rgb}{0.000000,0.000000,0.000000}%
\pgfsetfillcolor{currentfill}%
\pgfsetlinewidth{0.501875pt}%
\definecolor{currentstroke}{rgb}{0.000000,0.000000,0.000000}%
\pgfsetstrokecolor{currentstroke}%
\pgfsetdash{}{0pt}%
\pgfsys@defobject{currentmarker}{\pgfqpoint{0.000000in}{-0.069444in}}{\pgfqpoint{0.000000in}{0.000000in}}{%
\pgfpathmoveto{\pgfqpoint{0.000000in}{0.000000in}}%
\pgfpathlineto{\pgfqpoint{0.000000in}{-0.069444in}}%
\pgfusepath{stroke,fill}%
}%
\begin{pgfscope}%
\pgfsys@transformshift{1.144488in}{1.619432in}%
\pgfsys@useobject{currentmarker}{}%
\end{pgfscope}%
\end{pgfscope}%
\begin{pgfscope}%
\pgftext[x=1.144488in,y=0.380539in,,top]{\rmfamily\fontsize{8.000000}{9.600000}\selectfont −80}%
\end{pgfscope}%
\begin{pgfscope}%
\pgfsetbuttcap%
\pgfsetroundjoin%
\definecolor{currentfill}{rgb}{0.000000,0.000000,0.000000}%
\pgfsetfillcolor{currentfill}%
\pgfsetlinewidth{0.501875pt}%
\definecolor{currentstroke}{rgb}{0.000000,0.000000,0.000000}%
\pgfsetstrokecolor{currentstroke}%
\pgfsetdash{}{0pt}%
\pgfsys@defobject{currentmarker}{\pgfqpoint{0.000000in}{0.000000in}}{\pgfqpoint{0.000000in}{0.069444in}}{%
\pgfpathmoveto{\pgfqpoint{0.000000in}{0.000000in}}%
\pgfpathlineto{\pgfqpoint{0.000000in}{0.069444in}}%
\pgfusepath{stroke,fill}%
}%
\begin{pgfscope}%
\pgfsys@transformshift{1.380140in}{0.449983in}%
\pgfsys@useobject{currentmarker}{}%
\end{pgfscope}%
\end{pgfscope}%
\begin{pgfscope}%
\pgfsetbuttcap%
\pgfsetroundjoin%
\definecolor{currentfill}{rgb}{0.000000,0.000000,0.000000}%
\pgfsetfillcolor{currentfill}%
\pgfsetlinewidth{0.501875pt}%
\definecolor{currentstroke}{rgb}{0.000000,0.000000,0.000000}%
\pgfsetstrokecolor{currentstroke}%
\pgfsetdash{}{0pt}%
\pgfsys@defobject{currentmarker}{\pgfqpoint{0.000000in}{-0.069444in}}{\pgfqpoint{0.000000in}{0.000000in}}{%
\pgfpathmoveto{\pgfqpoint{0.000000in}{0.000000in}}%
\pgfpathlineto{\pgfqpoint{0.000000in}{-0.069444in}}%
\pgfusepath{stroke,fill}%
}%
\begin{pgfscope}%
\pgfsys@transformshift{1.380140in}{1.619432in}%
\pgfsys@useobject{currentmarker}{}%
\end{pgfscope}%
\end{pgfscope}%
\begin{pgfscope}%
\pgftext[x=1.380140in,y=0.380539in,,top]{\rmfamily\fontsize{8.000000}{9.600000}\selectfont −60}%
\end{pgfscope}%
\begin{pgfscope}%
\pgfsetbuttcap%
\pgfsetroundjoin%
\definecolor{currentfill}{rgb}{0.000000,0.000000,0.000000}%
\pgfsetfillcolor{currentfill}%
\pgfsetlinewidth{0.501875pt}%
\definecolor{currentstroke}{rgb}{0.000000,0.000000,0.000000}%
\pgfsetstrokecolor{currentstroke}%
\pgfsetdash{}{0pt}%
\pgfsys@defobject{currentmarker}{\pgfqpoint{0.000000in}{0.000000in}}{\pgfqpoint{0.000000in}{0.069444in}}{%
\pgfpathmoveto{\pgfqpoint{0.000000in}{0.000000in}}%
\pgfpathlineto{\pgfqpoint{0.000000in}{0.069444in}}%
\pgfusepath{stroke,fill}%
}%
\begin{pgfscope}%
\pgfsys@transformshift{1.615791in}{0.449983in}%
\pgfsys@useobject{currentmarker}{}%
\end{pgfscope}%
\end{pgfscope}%
\begin{pgfscope}%
\pgfsetbuttcap%
\pgfsetroundjoin%
\definecolor{currentfill}{rgb}{0.000000,0.000000,0.000000}%
\pgfsetfillcolor{currentfill}%
\pgfsetlinewidth{0.501875pt}%
\definecolor{currentstroke}{rgb}{0.000000,0.000000,0.000000}%
\pgfsetstrokecolor{currentstroke}%
\pgfsetdash{}{0pt}%
\pgfsys@defobject{currentmarker}{\pgfqpoint{0.000000in}{-0.069444in}}{\pgfqpoint{0.000000in}{0.000000in}}{%
\pgfpathmoveto{\pgfqpoint{0.000000in}{0.000000in}}%
\pgfpathlineto{\pgfqpoint{0.000000in}{-0.069444in}}%
\pgfusepath{stroke,fill}%
}%
\begin{pgfscope}%
\pgfsys@transformshift{1.615791in}{1.619432in}%
\pgfsys@useobject{currentmarker}{}%
\end{pgfscope}%
\end{pgfscope}%
\begin{pgfscope}%
\pgftext[x=1.615791in,y=0.380539in,,top]{\rmfamily\fontsize{8.000000}{9.600000}\selectfont −40}%
\end{pgfscope}%
\begin{pgfscope}%
\pgfsetbuttcap%
\pgfsetroundjoin%
\definecolor{currentfill}{rgb}{0.000000,0.000000,0.000000}%
\pgfsetfillcolor{currentfill}%
\pgfsetlinewidth{0.501875pt}%
\definecolor{currentstroke}{rgb}{0.000000,0.000000,0.000000}%
\pgfsetstrokecolor{currentstroke}%
\pgfsetdash{}{0pt}%
\pgfsys@defobject{currentmarker}{\pgfqpoint{0.000000in}{0.000000in}}{\pgfqpoint{0.000000in}{0.069444in}}{%
\pgfpathmoveto{\pgfqpoint{0.000000in}{0.000000in}}%
\pgfpathlineto{\pgfqpoint{0.000000in}{0.069444in}}%
\pgfusepath{stroke,fill}%
}%
\begin{pgfscope}%
\pgfsys@transformshift{1.851443in}{0.449983in}%
\pgfsys@useobject{currentmarker}{}%
\end{pgfscope}%
\end{pgfscope}%
\begin{pgfscope}%
\pgfsetbuttcap%
\pgfsetroundjoin%
\definecolor{currentfill}{rgb}{0.000000,0.000000,0.000000}%
\pgfsetfillcolor{currentfill}%
\pgfsetlinewidth{0.501875pt}%
\definecolor{currentstroke}{rgb}{0.000000,0.000000,0.000000}%
\pgfsetstrokecolor{currentstroke}%
\pgfsetdash{}{0pt}%
\pgfsys@defobject{currentmarker}{\pgfqpoint{0.000000in}{-0.069444in}}{\pgfqpoint{0.000000in}{0.000000in}}{%
\pgfpathmoveto{\pgfqpoint{0.000000in}{0.000000in}}%
\pgfpathlineto{\pgfqpoint{0.000000in}{-0.069444in}}%
\pgfusepath{stroke,fill}%
}%
\begin{pgfscope}%
\pgfsys@transformshift{1.851443in}{1.619432in}%
\pgfsys@useobject{currentmarker}{}%
\end{pgfscope}%
\end{pgfscope}%
\begin{pgfscope}%
\pgftext[x=1.851443in,y=0.380539in,,top]{\rmfamily\fontsize{8.000000}{9.600000}\selectfont −20}%
\end{pgfscope}%
\begin{pgfscope}%
\pgfsetbuttcap%
\pgfsetroundjoin%
\definecolor{currentfill}{rgb}{0.000000,0.000000,0.000000}%
\pgfsetfillcolor{currentfill}%
\pgfsetlinewidth{0.501875pt}%
\definecolor{currentstroke}{rgb}{0.000000,0.000000,0.000000}%
\pgfsetstrokecolor{currentstroke}%
\pgfsetdash{}{0pt}%
\pgfsys@defobject{currentmarker}{\pgfqpoint{0.000000in}{0.000000in}}{\pgfqpoint{0.000000in}{0.069444in}}{%
\pgfpathmoveto{\pgfqpoint{0.000000in}{0.000000in}}%
\pgfpathlineto{\pgfqpoint{0.000000in}{0.069444in}}%
\pgfusepath{stroke,fill}%
}%
\begin{pgfscope}%
\pgfsys@transformshift{2.087095in}{0.449983in}%
\pgfsys@useobject{currentmarker}{}%
\end{pgfscope}%
\end{pgfscope}%
\begin{pgfscope}%
\pgfsetbuttcap%
\pgfsetroundjoin%
\definecolor{currentfill}{rgb}{0.000000,0.000000,0.000000}%
\pgfsetfillcolor{currentfill}%
\pgfsetlinewidth{0.501875pt}%
\definecolor{currentstroke}{rgb}{0.000000,0.000000,0.000000}%
\pgfsetstrokecolor{currentstroke}%
\pgfsetdash{}{0pt}%
\pgfsys@defobject{currentmarker}{\pgfqpoint{0.000000in}{-0.069444in}}{\pgfqpoint{0.000000in}{0.000000in}}{%
\pgfpathmoveto{\pgfqpoint{0.000000in}{0.000000in}}%
\pgfpathlineto{\pgfqpoint{0.000000in}{-0.069444in}}%
\pgfusepath{stroke,fill}%
}%
\begin{pgfscope}%
\pgfsys@transformshift{2.087095in}{1.619432in}%
\pgfsys@useobject{currentmarker}{}%
\end{pgfscope}%
\end{pgfscope}%
\begin{pgfscope}%
\pgftext[x=2.087095in,y=0.380539in,,top]{\rmfamily\fontsize{8.000000}{9.600000}\selectfont 0}%
\end{pgfscope}%
\begin{pgfscope}%
\pgfsetbuttcap%
\pgfsetroundjoin%
\definecolor{currentfill}{rgb}{0.000000,0.000000,0.000000}%
\pgfsetfillcolor{currentfill}%
\pgfsetlinewidth{0.501875pt}%
\definecolor{currentstroke}{rgb}{0.000000,0.000000,0.000000}%
\pgfsetstrokecolor{currentstroke}%
\pgfsetdash{}{0pt}%
\pgfsys@defobject{currentmarker}{\pgfqpoint{0.000000in}{0.000000in}}{\pgfqpoint{0.000000in}{0.069444in}}{%
\pgfpathmoveto{\pgfqpoint{0.000000in}{0.000000in}}%
\pgfpathlineto{\pgfqpoint{0.000000in}{0.069444in}}%
\pgfusepath{stroke,fill}%
}%
\begin{pgfscope}%
\pgfsys@transformshift{2.322747in}{0.449983in}%
\pgfsys@useobject{currentmarker}{}%
\end{pgfscope}%
\end{pgfscope}%
\begin{pgfscope}%
\pgfsetbuttcap%
\pgfsetroundjoin%
\definecolor{currentfill}{rgb}{0.000000,0.000000,0.000000}%
\pgfsetfillcolor{currentfill}%
\pgfsetlinewidth{0.501875pt}%
\definecolor{currentstroke}{rgb}{0.000000,0.000000,0.000000}%
\pgfsetstrokecolor{currentstroke}%
\pgfsetdash{}{0pt}%
\pgfsys@defobject{currentmarker}{\pgfqpoint{0.000000in}{-0.069444in}}{\pgfqpoint{0.000000in}{0.000000in}}{%
\pgfpathmoveto{\pgfqpoint{0.000000in}{0.000000in}}%
\pgfpathlineto{\pgfqpoint{0.000000in}{-0.069444in}}%
\pgfusepath{stroke,fill}%
}%
\begin{pgfscope}%
\pgfsys@transformshift{2.322747in}{1.619432in}%
\pgfsys@useobject{currentmarker}{}%
\end{pgfscope}%
\end{pgfscope}%
\begin{pgfscope}%
\pgftext[x=2.322747in,y=0.380539in,,top]{\rmfamily\fontsize{8.000000}{9.600000}\selectfont 20}%
\end{pgfscope}%
\begin{pgfscope}%
\pgfsetbuttcap%
\pgfsetroundjoin%
\definecolor{currentfill}{rgb}{0.000000,0.000000,0.000000}%
\pgfsetfillcolor{currentfill}%
\pgfsetlinewidth{0.501875pt}%
\definecolor{currentstroke}{rgb}{0.000000,0.000000,0.000000}%
\pgfsetstrokecolor{currentstroke}%
\pgfsetdash{}{0pt}%
\pgfsys@defobject{currentmarker}{\pgfqpoint{0.000000in}{0.000000in}}{\pgfqpoint{0.000000in}{0.069444in}}{%
\pgfpathmoveto{\pgfqpoint{0.000000in}{0.000000in}}%
\pgfpathlineto{\pgfqpoint{0.000000in}{0.069444in}}%
\pgfusepath{stroke,fill}%
}%
\begin{pgfscope}%
\pgfsys@transformshift{2.558398in}{0.449983in}%
\pgfsys@useobject{currentmarker}{}%
\end{pgfscope}%
\end{pgfscope}%
\begin{pgfscope}%
\pgfsetbuttcap%
\pgfsetroundjoin%
\definecolor{currentfill}{rgb}{0.000000,0.000000,0.000000}%
\pgfsetfillcolor{currentfill}%
\pgfsetlinewidth{0.501875pt}%
\definecolor{currentstroke}{rgb}{0.000000,0.000000,0.000000}%
\pgfsetstrokecolor{currentstroke}%
\pgfsetdash{}{0pt}%
\pgfsys@defobject{currentmarker}{\pgfqpoint{0.000000in}{-0.069444in}}{\pgfqpoint{0.000000in}{0.000000in}}{%
\pgfpathmoveto{\pgfqpoint{0.000000in}{0.000000in}}%
\pgfpathlineto{\pgfqpoint{0.000000in}{-0.069444in}}%
\pgfusepath{stroke,fill}%
}%
\begin{pgfscope}%
\pgfsys@transformshift{2.558398in}{1.619432in}%
\pgfsys@useobject{currentmarker}{}%
\end{pgfscope}%
\end{pgfscope}%
\begin{pgfscope}%
\pgftext[x=2.558398in,y=0.380539in,,top]{\rmfamily\fontsize{8.000000}{9.600000}\selectfont 40}%
\end{pgfscope}%
\begin{pgfscope}%
\pgftext[x=1.497965in,y=0.203564in,,top]{\rmfamily\fontsize{9.000000}{10.800000}\selectfont \(\displaystyle \mathrm{DLL}_{K/\pi}(\pi^-)\)}%
\end{pgfscope}%
\begin{pgfscope}%
\pgfsetbuttcap%
\pgfsetroundjoin%
\definecolor{currentfill}{rgb}{0.000000,0.000000,0.000000}%
\pgfsetfillcolor{currentfill}%
\pgfsetlinewidth{0.501875pt}%
\definecolor{currentstroke}{rgb}{0.000000,0.000000,0.000000}%
\pgfsetstrokecolor{currentstroke}%
\pgfsetdash{}{0pt}%
\pgfsys@defobject{currentmarker}{\pgfqpoint{0.000000in}{0.000000in}}{\pgfqpoint{0.069444in}{0.000000in}}{%
\pgfpathmoveto{\pgfqpoint{0.000000in}{0.000000in}}%
\pgfpathlineto{\pgfqpoint{0.069444in}{0.000000in}}%
\pgfusepath{stroke,fill}%
}%
\begin{pgfscope}%
\pgfsys@transformshift{0.437532in}{0.449983in}%
\pgfsys@useobject{currentmarker}{}%
\end{pgfscope}%
\end{pgfscope}%
\begin{pgfscope}%
\pgfsetbuttcap%
\pgfsetroundjoin%
\definecolor{currentfill}{rgb}{0.000000,0.000000,0.000000}%
\pgfsetfillcolor{currentfill}%
\pgfsetlinewidth{0.501875pt}%
\definecolor{currentstroke}{rgb}{0.000000,0.000000,0.000000}%
\pgfsetstrokecolor{currentstroke}%
\pgfsetdash{}{0pt}%
\pgfsys@defobject{currentmarker}{\pgfqpoint{-0.069444in}{0.000000in}}{\pgfqpoint{0.000000in}{0.000000in}}{%
\pgfpathmoveto{\pgfqpoint{0.000000in}{0.000000in}}%
\pgfpathlineto{\pgfqpoint{-0.069444in}{0.000000in}}%
\pgfusepath{stroke,fill}%
}%
\begin{pgfscope}%
\pgfsys@transformshift{2.558398in}{0.449983in}%
\pgfsys@useobject{currentmarker}{}%
\end{pgfscope}%
\end{pgfscope}%
\begin{pgfscope}%
\pgftext[x=0.368088in,y=0.449983in,right,]{\rmfamily\fontsize{8.000000}{9.600000}\selectfont 0.000}%
\end{pgfscope}%
\begin{pgfscope}%
\pgfsetbuttcap%
\pgfsetroundjoin%
\definecolor{currentfill}{rgb}{0.000000,0.000000,0.000000}%
\pgfsetfillcolor{currentfill}%
\pgfsetlinewidth{0.501875pt}%
\definecolor{currentstroke}{rgb}{0.000000,0.000000,0.000000}%
\pgfsetstrokecolor{currentstroke}%
\pgfsetdash{}{0pt}%
\pgfsys@defobject{currentmarker}{\pgfqpoint{0.000000in}{0.000000in}}{\pgfqpoint{0.069444in}{0.000000in}}{%
\pgfpathmoveto{\pgfqpoint{0.000000in}{0.000000in}}%
\pgfpathlineto{\pgfqpoint{0.069444in}{0.000000in}}%
\pgfusepath{stroke,fill}%
}%
\begin{pgfscope}%
\pgfsys@transformshift{0.437532in}{0.617048in}%
\pgfsys@useobject{currentmarker}{}%
\end{pgfscope}%
\end{pgfscope}%
\begin{pgfscope}%
\pgfsetbuttcap%
\pgfsetroundjoin%
\definecolor{currentfill}{rgb}{0.000000,0.000000,0.000000}%
\pgfsetfillcolor{currentfill}%
\pgfsetlinewidth{0.501875pt}%
\definecolor{currentstroke}{rgb}{0.000000,0.000000,0.000000}%
\pgfsetstrokecolor{currentstroke}%
\pgfsetdash{}{0pt}%
\pgfsys@defobject{currentmarker}{\pgfqpoint{-0.069444in}{0.000000in}}{\pgfqpoint{0.000000in}{0.000000in}}{%
\pgfpathmoveto{\pgfqpoint{0.000000in}{0.000000in}}%
\pgfpathlineto{\pgfqpoint{-0.069444in}{0.000000in}}%
\pgfusepath{stroke,fill}%
}%
\begin{pgfscope}%
\pgfsys@transformshift{2.558398in}{0.617048in}%
\pgfsys@useobject{currentmarker}{}%
\end{pgfscope}%
\end{pgfscope}%
\begin{pgfscope}%
\pgftext[x=0.368088in,y=0.617048in,right,]{\rmfamily\fontsize{8.000000}{9.600000}\selectfont 0.005}%
\end{pgfscope}%
\begin{pgfscope}%
\pgfsetbuttcap%
\pgfsetroundjoin%
\definecolor{currentfill}{rgb}{0.000000,0.000000,0.000000}%
\pgfsetfillcolor{currentfill}%
\pgfsetlinewidth{0.501875pt}%
\definecolor{currentstroke}{rgb}{0.000000,0.000000,0.000000}%
\pgfsetstrokecolor{currentstroke}%
\pgfsetdash{}{0pt}%
\pgfsys@defobject{currentmarker}{\pgfqpoint{0.000000in}{0.000000in}}{\pgfqpoint{0.069444in}{0.000000in}}{%
\pgfpathmoveto{\pgfqpoint{0.000000in}{0.000000in}}%
\pgfpathlineto{\pgfqpoint{0.069444in}{0.000000in}}%
\pgfusepath{stroke,fill}%
}%
\begin{pgfscope}%
\pgfsys@transformshift{0.437532in}{0.784112in}%
\pgfsys@useobject{currentmarker}{}%
\end{pgfscope}%
\end{pgfscope}%
\begin{pgfscope}%
\pgfsetbuttcap%
\pgfsetroundjoin%
\definecolor{currentfill}{rgb}{0.000000,0.000000,0.000000}%
\pgfsetfillcolor{currentfill}%
\pgfsetlinewidth{0.501875pt}%
\definecolor{currentstroke}{rgb}{0.000000,0.000000,0.000000}%
\pgfsetstrokecolor{currentstroke}%
\pgfsetdash{}{0pt}%
\pgfsys@defobject{currentmarker}{\pgfqpoint{-0.069444in}{0.000000in}}{\pgfqpoint{0.000000in}{0.000000in}}{%
\pgfpathmoveto{\pgfqpoint{0.000000in}{0.000000in}}%
\pgfpathlineto{\pgfqpoint{-0.069444in}{0.000000in}}%
\pgfusepath{stroke,fill}%
}%
\begin{pgfscope}%
\pgfsys@transformshift{2.558398in}{0.784112in}%
\pgfsys@useobject{currentmarker}{}%
\end{pgfscope}%
\end{pgfscope}%
\begin{pgfscope}%
\pgftext[x=0.368088in,y=0.784112in,right,]{\rmfamily\fontsize{8.000000}{9.600000}\selectfont 0.010}%
\end{pgfscope}%
\begin{pgfscope}%
\pgfsetbuttcap%
\pgfsetroundjoin%
\definecolor{currentfill}{rgb}{0.000000,0.000000,0.000000}%
\pgfsetfillcolor{currentfill}%
\pgfsetlinewidth{0.501875pt}%
\definecolor{currentstroke}{rgb}{0.000000,0.000000,0.000000}%
\pgfsetstrokecolor{currentstroke}%
\pgfsetdash{}{0pt}%
\pgfsys@defobject{currentmarker}{\pgfqpoint{0.000000in}{0.000000in}}{\pgfqpoint{0.069444in}{0.000000in}}{%
\pgfpathmoveto{\pgfqpoint{0.000000in}{0.000000in}}%
\pgfpathlineto{\pgfqpoint{0.069444in}{0.000000in}}%
\pgfusepath{stroke,fill}%
}%
\begin{pgfscope}%
\pgfsys@transformshift{0.437532in}{0.951176in}%
\pgfsys@useobject{currentmarker}{}%
\end{pgfscope}%
\end{pgfscope}%
\begin{pgfscope}%
\pgfsetbuttcap%
\pgfsetroundjoin%
\definecolor{currentfill}{rgb}{0.000000,0.000000,0.000000}%
\pgfsetfillcolor{currentfill}%
\pgfsetlinewidth{0.501875pt}%
\definecolor{currentstroke}{rgb}{0.000000,0.000000,0.000000}%
\pgfsetstrokecolor{currentstroke}%
\pgfsetdash{}{0pt}%
\pgfsys@defobject{currentmarker}{\pgfqpoint{-0.069444in}{0.000000in}}{\pgfqpoint{0.000000in}{0.000000in}}{%
\pgfpathmoveto{\pgfqpoint{0.000000in}{0.000000in}}%
\pgfpathlineto{\pgfqpoint{-0.069444in}{0.000000in}}%
\pgfusepath{stroke,fill}%
}%
\begin{pgfscope}%
\pgfsys@transformshift{2.558398in}{0.951176in}%
\pgfsys@useobject{currentmarker}{}%
\end{pgfscope}%
\end{pgfscope}%
\begin{pgfscope}%
\pgftext[x=0.368088in,y=0.951176in,right,]{\rmfamily\fontsize{8.000000}{9.600000}\selectfont 0.015}%
\end{pgfscope}%
\begin{pgfscope}%
\pgfsetbuttcap%
\pgfsetroundjoin%
\definecolor{currentfill}{rgb}{0.000000,0.000000,0.000000}%
\pgfsetfillcolor{currentfill}%
\pgfsetlinewidth{0.501875pt}%
\definecolor{currentstroke}{rgb}{0.000000,0.000000,0.000000}%
\pgfsetstrokecolor{currentstroke}%
\pgfsetdash{}{0pt}%
\pgfsys@defobject{currentmarker}{\pgfqpoint{0.000000in}{0.000000in}}{\pgfqpoint{0.069444in}{0.000000in}}{%
\pgfpathmoveto{\pgfqpoint{0.000000in}{0.000000in}}%
\pgfpathlineto{\pgfqpoint{0.069444in}{0.000000in}}%
\pgfusepath{stroke,fill}%
}%
\begin{pgfscope}%
\pgfsys@transformshift{0.437532in}{1.118240in}%
\pgfsys@useobject{currentmarker}{}%
\end{pgfscope}%
\end{pgfscope}%
\begin{pgfscope}%
\pgfsetbuttcap%
\pgfsetroundjoin%
\definecolor{currentfill}{rgb}{0.000000,0.000000,0.000000}%
\pgfsetfillcolor{currentfill}%
\pgfsetlinewidth{0.501875pt}%
\definecolor{currentstroke}{rgb}{0.000000,0.000000,0.000000}%
\pgfsetstrokecolor{currentstroke}%
\pgfsetdash{}{0pt}%
\pgfsys@defobject{currentmarker}{\pgfqpoint{-0.069444in}{0.000000in}}{\pgfqpoint{0.000000in}{0.000000in}}{%
\pgfpathmoveto{\pgfqpoint{0.000000in}{0.000000in}}%
\pgfpathlineto{\pgfqpoint{-0.069444in}{0.000000in}}%
\pgfusepath{stroke,fill}%
}%
\begin{pgfscope}%
\pgfsys@transformshift{2.558398in}{1.118240in}%
\pgfsys@useobject{currentmarker}{}%
\end{pgfscope}%
\end{pgfscope}%
\begin{pgfscope}%
\pgftext[x=0.368088in,y=1.118240in,right,]{\rmfamily\fontsize{8.000000}{9.600000}\selectfont 0.020}%
\end{pgfscope}%
\begin{pgfscope}%
\pgfsetbuttcap%
\pgfsetroundjoin%
\definecolor{currentfill}{rgb}{0.000000,0.000000,0.000000}%
\pgfsetfillcolor{currentfill}%
\pgfsetlinewidth{0.501875pt}%
\definecolor{currentstroke}{rgb}{0.000000,0.000000,0.000000}%
\pgfsetstrokecolor{currentstroke}%
\pgfsetdash{}{0pt}%
\pgfsys@defobject{currentmarker}{\pgfqpoint{0.000000in}{0.000000in}}{\pgfqpoint{0.069444in}{0.000000in}}{%
\pgfpathmoveto{\pgfqpoint{0.000000in}{0.000000in}}%
\pgfpathlineto{\pgfqpoint{0.069444in}{0.000000in}}%
\pgfusepath{stroke,fill}%
}%
\begin{pgfscope}%
\pgfsys@transformshift{0.437532in}{1.285304in}%
\pgfsys@useobject{currentmarker}{}%
\end{pgfscope}%
\end{pgfscope}%
\begin{pgfscope}%
\pgfsetbuttcap%
\pgfsetroundjoin%
\definecolor{currentfill}{rgb}{0.000000,0.000000,0.000000}%
\pgfsetfillcolor{currentfill}%
\pgfsetlinewidth{0.501875pt}%
\definecolor{currentstroke}{rgb}{0.000000,0.000000,0.000000}%
\pgfsetstrokecolor{currentstroke}%
\pgfsetdash{}{0pt}%
\pgfsys@defobject{currentmarker}{\pgfqpoint{-0.069444in}{0.000000in}}{\pgfqpoint{0.000000in}{0.000000in}}{%
\pgfpathmoveto{\pgfqpoint{0.000000in}{0.000000in}}%
\pgfpathlineto{\pgfqpoint{-0.069444in}{0.000000in}}%
\pgfusepath{stroke,fill}%
}%
\begin{pgfscope}%
\pgfsys@transformshift{2.558398in}{1.285304in}%
\pgfsys@useobject{currentmarker}{}%
\end{pgfscope}%
\end{pgfscope}%
\begin{pgfscope}%
\pgftext[x=0.368088in,y=1.285304in,right,]{\rmfamily\fontsize{8.000000}{9.600000}\selectfont 0.025}%
\end{pgfscope}%
\begin{pgfscope}%
\pgfsetbuttcap%
\pgfsetroundjoin%
\definecolor{currentfill}{rgb}{0.000000,0.000000,0.000000}%
\pgfsetfillcolor{currentfill}%
\pgfsetlinewidth{0.501875pt}%
\definecolor{currentstroke}{rgb}{0.000000,0.000000,0.000000}%
\pgfsetstrokecolor{currentstroke}%
\pgfsetdash{}{0pt}%
\pgfsys@defobject{currentmarker}{\pgfqpoint{0.000000in}{0.000000in}}{\pgfqpoint{0.069444in}{0.000000in}}{%
\pgfpathmoveto{\pgfqpoint{0.000000in}{0.000000in}}%
\pgfpathlineto{\pgfqpoint{0.069444in}{0.000000in}}%
\pgfusepath{stroke,fill}%
}%
\begin{pgfscope}%
\pgfsys@transformshift{0.437532in}{1.452368in}%
\pgfsys@useobject{currentmarker}{}%
\end{pgfscope}%
\end{pgfscope}%
\begin{pgfscope}%
\pgfsetbuttcap%
\pgfsetroundjoin%
\definecolor{currentfill}{rgb}{0.000000,0.000000,0.000000}%
\pgfsetfillcolor{currentfill}%
\pgfsetlinewidth{0.501875pt}%
\definecolor{currentstroke}{rgb}{0.000000,0.000000,0.000000}%
\pgfsetstrokecolor{currentstroke}%
\pgfsetdash{}{0pt}%
\pgfsys@defobject{currentmarker}{\pgfqpoint{-0.069444in}{0.000000in}}{\pgfqpoint{0.000000in}{0.000000in}}{%
\pgfpathmoveto{\pgfqpoint{0.000000in}{0.000000in}}%
\pgfpathlineto{\pgfqpoint{-0.069444in}{0.000000in}}%
\pgfusepath{stroke,fill}%
}%
\begin{pgfscope}%
\pgfsys@transformshift{2.558398in}{1.452368in}%
\pgfsys@useobject{currentmarker}{}%
\end{pgfscope}%
\end{pgfscope}%
\begin{pgfscope}%
\pgftext[x=0.368088in,y=1.452368in,right,]{\rmfamily\fontsize{8.000000}{9.600000}\selectfont 0.030}%
\end{pgfscope}%
\begin{pgfscope}%
\pgfsetbuttcap%
\pgfsetroundjoin%
\definecolor{currentfill}{rgb}{0.000000,0.000000,0.000000}%
\pgfsetfillcolor{currentfill}%
\pgfsetlinewidth{0.501875pt}%
\definecolor{currentstroke}{rgb}{0.000000,0.000000,0.000000}%
\pgfsetstrokecolor{currentstroke}%
\pgfsetdash{}{0pt}%
\pgfsys@defobject{currentmarker}{\pgfqpoint{0.000000in}{0.000000in}}{\pgfqpoint{0.069444in}{0.000000in}}{%
\pgfpathmoveto{\pgfqpoint{0.000000in}{0.000000in}}%
\pgfpathlineto{\pgfqpoint{0.069444in}{0.000000in}}%
\pgfusepath{stroke,fill}%
}%
\begin{pgfscope}%
\pgfsys@transformshift{0.437532in}{1.619432in}%
\pgfsys@useobject{currentmarker}{}%
\end{pgfscope}%
\end{pgfscope}%
\begin{pgfscope}%
\pgfsetbuttcap%
\pgfsetroundjoin%
\definecolor{currentfill}{rgb}{0.000000,0.000000,0.000000}%
\pgfsetfillcolor{currentfill}%
\pgfsetlinewidth{0.501875pt}%
\definecolor{currentstroke}{rgb}{0.000000,0.000000,0.000000}%
\pgfsetstrokecolor{currentstroke}%
\pgfsetdash{}{0pt}%
\pgfsys@defobject{currentmarker}{\pgfqpoint{-0.069444in}{0.000000in}}{\pgfqpoint{0.000000in}{0.000000in}}{%
\pgfpathmoveto{\pgfqpoint{0.000000in}{0.000000in}}%
\pgfpathlineto{\pgfqpoint{-0.069444in}{0.000000in}}%
\pgfusepath{stroke,fill}%
}%
\begin{pgfscope}%
\pgfsys@transformshift{2.558398in}{1.619432in}%
\pgfsys@useobject{currentmarker}{}%
\end{pgfscope}%
\end{pgfscope}%
\begin{pgfscope}%
\pgftext[x=0.368088in,y=1.619432in,right,]{\rmfamily\fontsize{8.000000}{9.600000}\selectfont 0.035}%
\end{pgfscope}%
\end{pgfpicture}%
\makeatother%
\endgroup%

	\end{subfigure}
	\begin{subfigure}[t]{0.49\textwidth}
		\centering
    %\includegraphics[width=\textwidth]{store/variables/DATA_MC_REWEIGHTED_piminus_PIDmu.pdf}
    %% Creator: Matplotlib, PGF backend
%%
%% To include the figure in your LaTeX document, write
%%   \input{<filename>.pgf}
%%
%% Make sure the required packages are loaded in your preamble
%%   \usepackage{pgf}
%%
%% Figures using additional raster images can only be included by \input if
%% they are in the same directory as the main LaTeX file. For loading figures
%% from other directories you can use the `import` package
%%   \usepackage{import}
%% and then include the figures with
%%   \import{<path to file>}{<filename>.pgf}
%%
%% Matplotlib used the following preamble
%%   \usepackage{fontspec}
%%   \setmainfont{DejaVu Serif}
%%   \setsansfont{DejaVu Sans}
%%   \setmonofont{DejaVu Sans Mono}
%%
\begingroup%
\makeatletter%
\begin{pgfpicture}%
\pgfpathrectangle{\pgfpointorigin}{\pgfqpoint{2.684763in}{1.723197in}}%
\pgfusepath{use as bounding box, clip}%
\begin{pgfscope}%
\pgfsetbuttcap%
\pgfsetmiterjoin%
\definecolor{currentfill}{rgb}{1.000000,1.000000,1.000000}%
\pgfsetfillcolor{currentfill}%
\pgfsetlinewidth{0.000000pt}%
\definecolor{currentstroke}{rgb}{1.000000,1.000000,1.000000}%
\pgfsetstrokecolor{currentstroke}%
\pgfsetdash{}{0pt}%
\pgfpathmoveto{\pgfqpoint{0.000000in}{0.000000in}}%
\pgfpathlineto{\pgfqpoint{2.684763in}{0.000000in}}%
\pgfpathlineto{\pgfqpoint{2.684763in}{1.723197in}}%
\pgfpathlineto{\pgfqpoint{0.000000in}{1.723197in}}%
\pgfpathclose%
\pgfusepath{fill}%
\end{pgfscope}%
\begin{pgfscope}%
\pgfsetbuttcap%
\pgfsetmiterjoin%
\definecolor{currentfill}{rgb}{1.000000,1.000000,1.000000}%
\pgfsetfillcolor{currentfill}%
\pgfsetlinewidth{0.000000pt}%
\definecolor{currentstroke}{rgb}{0.000000,0.000000,0.000000}%
\pgfsetstrokecolor{currentstroke}%
\pgfsetstrokeopacity{0.000000}%
\pgfsetdash{}{0pt}%
\pgfpathmoveto{\pgfqpoint{0.366840in}{0.449983in}}%
\pgfpathlineto{\pgfqpoint{2.564071in}{0.449983in}}%
\pgfpathlineto{\pgfqpoint{2.564071in}{1.619432in}}%
\pgfpathlineto{\pgfqpoint{0.366840in}{1.619432in}}%
\pgfpathclose%
\pgfusepath{fill}%
\end{pgfscope}%
\begin{pgfscope}%
\pgfpathrectangle{\pgfqpoint{0.366840in}{0.449983in}}{\pgfqpoint{2.197230in}{1.169449in}} %
\pgfusepath{clip}%
\pgfsetbuttcap%
\pgfsetmiterjoin%
\definecolor{currentfill}{rgb}{0.215686,0.470588,0.749020}%
\pgfsetfillcolor{currentfill}%
\pgfsetlinewidth{0.000000pt}%
\definecolor{currentstroke}{rgb}{0.000000,0.000000,0.000000}%
\pgfsetstrokecolor{currentstroke}%
\pgfsetdash{}{0pt}%
\pgfpathmoveto{\pgfqpoint{0.366840in}{0.449983in}}%
\pgfpathlineto{\pgfqpoint{0.366840in}{0.449983in}}%
\pgfpathlineto{\pgfqpoint{0.410785in}{0.449983in}}%
\pgfpathlineto{\pgfqpoint{0.410785in}{0.450311in}}%
\pgfpathlineto{\pgfqpoint{0.454729in}{0.450311in}}%
\pgfpathlineto{\pgfqpoint{0.454729in}{0.450222in}}%
\pgfpathlineto{\pgfqpoint{0.498674in}{0.450222in}}%
\pgfpathlineto{\pgfqpoint{0.498674in}{0.457072in}}%
\pgfpathlineto{\pgfqpoint{0.542619in}{0.457072in}}%
\pgfpathlineto{\pgfqpoint{0.542619in}{0.530639in}}%
\pgfpathlineto{\pgfqpoint{0.586563in}{0.530639in}}%
\pgfpathlineto{\pgfqpoint{0.586563in}{0.656572in}}%
\pgfpathlineto{\pgfqpoint{0.630508in}{0.656572in}}%
\pgfpathlineto{\pgfqpoint{0.630508in}{0.660057in}}%
\pgfpathlineto{\pgfqpoint{0.674453in}{0.660057in}}%
\pgfpathlineto{\pgfqpoint{0.674453in}{0.666954in}}%
\pgfpathlineto{\pgfqpoint{0.718397in}{0.666954in}}%
\pgfpathlineto{\pgfqpoint{0.718397in}{0.681830in}}%
\pgfpathlineto{\pgfqpoint{0.762342in}{0.681830in}}%
\pgfpathlineto{\pgfqpoint{0.762342in}{0.774530in}}%
\pgfpathlineto{\pgfqpoint{0.806286in}{0.774530in}}%
\pgfpathlineto{\pgfqpoint{0.806286in}{0.925331in}}%
\pgfpathlineto{\pgfqpoint{0.850231in}{0.925331in}}%
\pgfpathlineto{\pgfqpoint{0.850231in}{0.882922in}}%
\pgfpathlineto{\pgfqpoint{0.894176in}{0.882922in}}%
\pgfpathlineto{\pgfqpoint{0.894176in}{0.909573in}}%
\pgfpathlineto{\pgfqpoint{0.938120in}{0.909573in}}%
\pgfpathlineto{\pgfqpoint{0.938120in}{1.283699in}}%
\pgfpathlineto{\pgfqpoint{0.982065in}{1.283699in}}%
\pgfpathlineto{\pgfqpoint{0.982065in}{1.088179in}}%
\pgfpathlineto{\pgfqpoint{1.026009in}{1.088179in}}%
\pgfpathlineto{\pgfqpoint{1.026009in}{1.058620in}}%
\pgfpathlineto{\pgfqpoint{1.069954in}{1.058620in}}%
\pgfpathlineto{\pgfqpoint{1.069954in}{0.996215in}}%
\pgfpathlineto{\pgfqpoint{1.113899in}{0.996215in}}%
\pgfpathlineto{\pgfqpoint{1.113899in}{0.968280in}}%
\pgfpathlineto{\pgfqpoint{1.157843in}{0.968280in}}%
\pgfpathlineto{\pgfqpoint{1.157843in}{0.933305in}}%
\pgfpathlineto{\pgfqpoint{1.201788in}{0.933305in}}%
\pgfpathlineto{\pgfqpoint{1.201788in}{0.891155in}}%
\pgfpathlineto{\pgfqpoint{1.245732in}{0.891155in}}%
\pgfpathlineto{\pgfqpoint{1.245732in}{0.884162in}}%
\pgfpathlineto{\pgfqpoint{1.289677in}{0.884162in}}%
\pgfpathlineto{\pgfqpoint{1.289677in}{0.863586in}}%
\pgfpathlineto{\pgfqpoint{1.333622in}{0.863586in}}%
\pgfpathlineto{\pgfqpoint{1.333622in}{0.821378in}}%
\pgfpathlineto{\pgfqpoint{1.377566in}{0.821378in}}%
\pgfpathlineto{\pgfqpoint{1.377566in}{0.774955in}}%
\pgfpathlineto{\pgfqpoint{1.421511in}{0.774955in}}%
\pgfpathlineto{\pgfqpoint{1.421511in}{0.767563in}}%
\pgfpathlineto{\pgfqpoint{1.465455in}{0.767563in}}%
\pgfpathlineto{\pgfqpoint{1.465455in}{0.714745in}}%
\pgfpathlineto{\pgfqpoint{1.509400in}{0.714745in}}%
\pgfpathlineto{\pgfqpoint{1.509400in}{0.651732in}}%
\pgfpathlineto{\pgfqpoint{1.553345in}{0.651732in}}%
\pgfpathlineto{\pgfqpoint{1.553345in}{0.609909in}}%
\pgfpathlineto{\pgfqpoint{1.597289in}{0.609909in}}%
\pgfpathlineto{\pgfqpoint{1.597289in}{0.571748in}}%
\pgfpathlineto{\pgfqpoint{1.641234in}{0.571748in}}%
\pgfpathlineto{\pgfqpoint{1.641234in}{0.553943in}}%
\pgfpathlineto{\pgfqpoint{1.685178in}{0.553943in}}%
\pgfpathlineto{\pgfqpoint{1.685178in}{0.532042in}}%
\pgfpathlineto{\pgfqpoint{1.729123in}{0.532042in}}%
\pgfpathlineto{\pgfqpoint{1.729123in}{0.514543in}}%
\pgfpathlineto{\pgfqpoint{1.773068in}{0.514543in}}%
\pgfpathlineto{\pgfqpoint{1.773068in}{0.499441in}}%
\pgfpathlineto{\pgfqpoint{1.817012in}{0.499441in}}%
\pgfpathlineto{\pgfqpoint{1.817012in}{0.487711in}}%
\pgfpathlineto{\pgfqpoint{1.860957in}{0.487711in}}%
\pgfpathlineto{\pgfqpoint{1.860957in}{0.476985in}}%
\pgfpathlineto{\pgfqpoint{1.904902in}{0.476985in}}%
\pgfpathlineto{\pgfqpoint{1.904902in}{0.470225in}}%
\pgfpathlineto{\pgfqpoint{1.948846in}{0.470225in}}%
\pgfpathlineto{\pgfqpoint{1.948846in}{0.465184in}}%
\pgfpathlineto{\pgfqpoint{1.992791in}{0.465184in}}%
\pgfpathlineto{\pgfqpoint{1.992791in}{0.457263in}}%
\pgfpathlineto{\pgfqpoint{2.036735in}{0.457263in}}%
\pgfpathlineto{\pgfqpoint{2.036735in}{0.455272in}}%
\pgfpathlineto{\pgfqpoint{2.080680in}{0.455272in}}%
\pgfpathlineto{\pgfqpoint{2.080680in}{0.452441in}}%
\pgfpathlineto{\pgfqpoint{2.124625in}{0.452441in}}%
\pgfpathlineto{\pgfqpoint{2.124625in}{0.451553in}}%
\pgfpathlineto{\pgfqpoint{2.168569in}{0.451553in}}%
\pgfpathlineto{\pgfqpoint{2.168569in}{0.451753in}}%
\pgfpathlineto{\pgfqpoint{2.212514in}{0.451753in}}%
\pgfpathlineto{\pgfqpoint{2.212514in}{0.450305in}}%
\pgfpathlineto{\pgfqpoint{2.256458in}{0.450305in}}%
\pgfpathlineto{\pgfqpoint{2.256458in}{0.450411in}}%
\pgfpathlineto{\pgfqpoint{2.300403in}{0.450411in}}%
\pgfpathlineto{\pgfqpoint{2.300403in}{0.450100in}}%
\pgfpathlineto{\pgfqpoint{2.344348in}{0.450100in}}%
\pgfpathlineto{\pgfqpoint{2.344348in}{0.450069in}}%
\pgfpathlineto{\pgfqpoint{2.388292in}{0.450069in}}%
\pgfpathlineto{\pgfqpoint{2.388292in}{0.450211in}}%
\pgfpathlineto{\pgfqpoint{2.432237in}{0.450211in}}%
\pgfpathlineto{\pgfqpoint{2.432237in}{0.449932in}}%
\pgfpathlineto{\pgfqpoint{2.476181in}{0.449932in}}%
\pgfpathlineto{\pgfqpoint{2.476181in}{0.449983in}}%
\pgfpathlineto{\pgfqpoint{2.520126in}{0.449983in}}%
\pgfpathlineto{\pgfqpoint{2.520126in}{0.449983in}}%
\pgfpathlineto{\pgfqpoint{2.564071in}{0.449983in}}%
\pgfpathlineto{\pgfqpoint{2.564071in}{0.449983in}}%
\pgfpathlineto{\pgfqpoint{2.520126in}{0.449983in}}%
\pgfpathlineto{\pgfqpoint{2.520126in}{0.449983in}}%
\pgfpathlineto{\pgfqpoint{2.476181in}{0.449983in}}%
\pgfpathlineto{\pgfqpoint{2.476181in}{0.449983in}}%
\pgfpathlineto{\pgfqpoint{2.432237in}{0.449983in}}%
\pgfpathlineto{\pgfqpoint{2.432237in}{0.449983in}}%
\pgfpathlineto{\pgfqpoint{2.388292in}{0.449983in}}%
\pgfpathlineto{\pgfqpoint{2.388292in}{0.449983in}}%
\pgfpathlineto{\pgfqpoint{2.344348in}{0.449983in}}%
\pgfpathlineto{\pgfqpoint{2.344348in}{0.449983in}}%
\pgfpathlineto{\pgfqpoint{2.300403in}{0.449983in}}%
\pgfpathlineto{\pgfqpoint{2.300403in}{0.449983in}}%
\pgfpathlineto{\pgfqpoint{2.256458in}{0.449983in}}%
\pgfpathlineto{\pgfqpoint{2.256458in}{0.449983in}}%
\pgfpathlineto{\pgfqpoint{2.212514in}{0.449983in}}%
\pgfpathlineto{\pgfqpoint{2.212514in}{0.449983in}}%
\pgfpathlineto{\pgfqpoint{2.168569in}{0.449983in}}%
\pgfpathlineto{\pgfqpoint{2.168569in}{0.449983in}}%
\pgfpathlineto{\pgfqpoint{2.124625in}{0.449983in}}%
\pgfpathlineto{\pgfqpoint{2.124625in}{0.449983in}}%
\pgfpathlineto{\pgfqpoint{2.080680in}{0.449983in}}%
\pgfpathlineto{\pgfqpoint{2.080680in}{0.449983in}}%
\pgfpathlineto{\pgfqpoint{2.036735in}{0.449983in}}%
\pgfpathlineto{\pgfqpoint{2.036735in}{0.449983in}}%
\pgfpathlineto{\pgfqpoint{1.992791in}{0.449983in}}%
\pgfpathlineto{\pgfqpoint{1.992791in}{0.449983in}}%
\pgfpathlineto{\pgfqpoint{1.948846in}{0.449983in}}%
\pgfpathlineto{\pgfqpoint{1.948846in}{0.449983in}}%
\pgfpathlineto{\pgfqpoint{1.904902in}{0.449983in}}%
\pgfpathlineto{\pgfqpoint{1.904902in}{0.449983in}}%
\pgfpathlineto{\pgfqpoint{1.860957in}{0.449983in}}%
\pgfpathlineto{\pgfqpoint{1.860957in}{0.449983in}}%
\pgfpathlineto{\pgfqpoint{1.817012in}{0.449983in}}%
\pgfpathlineto{\pgfqpoint{1.817012in}{0.449983in}}%
\pgfpathlineto{\pgfqpoint{1.773068in}{0.449983in}}%
\pgfpathlineto{\pgfqpoint{1.773068in}{0.449983in}}%
\pgfpathlineto{\pgfqpoint{1.729123in}{0.449983in}}%
\pgfpathlineto{\pgfqpoint{1.729123in}{0.449983in}}%
\pgfpathlineto{\pgfqpoint{1.685178in}{0.449983in}}%
\pgfpathlineto{\pgfqpoint{1.685178in}{0.449983in}}%
\pgfpathlineto{\pgfqpoint{1.641234in}{0.449983in}}%
\pgfpathlineto{\pgfqpoint{1.641234in}{0.449983in}}%
\pgfpathlineto{\pgfqpoint{1.597289in}{0.449983in}}%
\pgfpathlineto{\pgfqpoint{1.597289in}{0.449983in}}%
\pgfpathlineto{\pgfqpoint{1.553345in}{0.449983in}}%
\pgfpathlineto{\pgfqpoint{1.553345in}{0.449983in}}%
\pgfpathlineto{\pgfqpoint{1.509400in}{0.449983in}}%
\pgfpathlineto{\pgfqpoint{1.509400in}{0.449983in}}%
\pgfpathlineto{\pgfqpoint{1.465455in}{0.449983in}}%
\pgfpathlineto{\pgfqpoint{1.465455in}{0.449983in}}%
\pgfpathlineto{\pgfqpoint{1.421511in}{0.449983in}}%
\pgfpathlineto{\pgfqpoint{1.421511in}{0.449983in}}%
\pgfpathlineto{\pgfqpoint{1.377566in}{0.449983in}}%
\pgfpathlineto{\pgfqpoint{1.377566in}{0.449983in}}%
\pgfpathlineto{\pgfqpoint{1.333622in}{0.449983in}}%
\pgfpathlineto{\pgfqpoint{1.333622in}{0.449983in}}%
\pgfpathlineto{\pgfqpoint{1.289677in}{0.449983in}}%
\pgfpathlineto{\pgfqpoint{1.289677in}{0.449983in}}%
\pgfpathlineto{\pgfqpoint{1.245732in}{0.449983in}}%
\pgfpathlineto{\pgfqpoint{1.245732in}{0.449983in}}%
\pgfpathlineto{\pgfqpoint{1.201788in}{0.449983in}}%
\pgfpathlineto{\pgfqpoint{1.201788in}{0.449983in}}%
\pgfpathlineto{\pgfqpoint{1.157843in}{0.449983in}}%
\pgfpathlineto{\pgfqpoint{1.157843in}{0.449983in}}%
\pgfpathlineto{\pgfqpoint{1.113899in}{0.449983in}}%
\pgfpathlineto{\pgfqpoint{1.113899in}{0.449983in}}%
\pgfpathlineto{\pgfqpoint{1.069954in}{0.449983in}}%
\pgfpathlineto{\pgfqpoint{1.069954in}{0.449983in}}%
\pgfpathlineto{\pgfqpoint{1.026009in}{0.449983in}}%
\pgfpathlineto{\pgfqpoint{1.026009in}{0.449983in}}%
\pgfpathlineto{\pgfqpoint{0.982065in}{0.449983in}}%
\pgfpathlineto{\pgfqpoint{0.982065in}{0.449983in}}%
\pgfpathlineto{\pgfqpoint{0.938120in}{0.449983in}}%
\pgfpathlineto{\pgfqpoint{0.938120in}{0.449983in}}%
\pgfpathlineto{\pgfqpoint{0.894176in}{0.449983in}}%
\pgfpathlineto{\pgfqpoint{0.894176in}{0.449983in}}%
\pgfpathlineto{\pgfqpoint{0.850231in}{0.449983in}}%
\pgfpathlineto{\pgfqpoint{0.850231in}{0.449983in}}%
\pgfpathlineto{\pgfqpoint{0.806286in}{0.449983in}}%
\pgfpathlineto{\pgfqpoint{0.806286in}{0.449983in}}%
\pgfpathlineto{\pgfqpoint{0.762342in}{0.449983in}}%
\pgfpathlineto{\pgfqpoint{0.762342in}{0.449983in}}%
\pgfpathlineto{\pgfqpoint{0.718397in}{0.449983in}}%
\pgfpathlineto{\pgfqpoint{0.718397in}{0.449983in}}%
\pgfpathlineto{\pgfqpoint{0.674453in}{0.449983in}}%
\pgfpathlineto{\pgfqpoint{0.674453in}{0.449983in}}%
\pgfpathlineto{\pgfqpoint{0.630508in}{0.449983in}}%
\pgfpathlineto{\pgfqpoint{0.630508in}{0.449983in}}%
\pgfpathlineto{\pgfqpoint{0.586563in}{0.449983in}}%
\pgfpathlineto{\pgfqpoint{0.586563in}{0.449983in}}%
\pgfpathlineto{\pgfqpoint{0.542619in}{0.449983in}}%
\pgfpathlineto{\pgfqpoint{0.542619in}{0.449983in}}%
\pgfpathlineto{\pgfqpoint{0.498674in}{0.449983in}}%
\pgfpathlineto{\pgfqpoint{0.498674in}{0.449983in}}%
\pgfpathlineto{\pgfqpoint{0.454729in}{0.449983in}}%
\pgfpathlineto{\pgfqpoint{0.454729in}{0.449983in}}%
\pgfpathlineto{\pgfqpoint{0.410785in}{0.449983in}}%
\pgfpathlineto{\pgfqpoint{0.410785in}{0.449983in}}%
\pgfpathlineto{\pgfqpoint{0.366840in}{0.449983in}}%
\pgfusepath{fill}%
\end{pgfscope}%
\begin{pgfscope}%
\pgfpathrectangle{\pgfqpoint{0.366840in}{0.449983in}}{\pgfqpoint{2.197230in}{1.169449in}} %
\pgfusepath{clip}%
\pgfsetbuttcap%
\pgfsetmiterjoin%
\pgfsetlinewidth{0.501875pt}%
\definecolor{currentstroke}{rgb}{1.000000,0.000000,0.000000}%
\pgfsetstrokecolor{currentstroke}%
\pgfsetdash{}{0pt}%
\pgfpathmoveto{\pgfqpoint{0.366840in}{0.449983in}}%
\pgfpathlineto{\pgfqpoint{0.366840in}{0.449983in}}%
\pgfpathlineto{\pgfqpoint{0.410785in}{0.449983in}}%
\pgfpathlineto{\pgfqpoint{0.410785in}{0.450497in}}%
\pgfpathlineto{\pgfqpoint{0.454729in}{0.450497in}}%
\pgfpathlineto{\pgfqpoint{0.454729in}{0.450766in}}%
\pgfpathlineto{\pgfqpoint{0.498674in}{0.450766in}}%
\pgfpathlineto{\pgfqpoint{0.498674in}{0.468540in}}%
\pgfpathlineto{\pgfqpoint{0.542619in}{0.468540in}}%
\pgfpathlineto{\pgfqpoint{0.542619in}{0.622367in}}%
\pgfpathlineto{\pgfqpoint{0.586563in}{0.622367in}}%
\pgfpathlineto{\pgfqpoint{0.586563in}{0.755194in}}%
\pgfpathlineto{\pgfqpoint{0.630508in}{0.755194in}}%
\pgfpathlineto{\pgfqpoint{0.630508in}{0.707642in}}%
\pgfpathlineto{\pgfqpoint{0.674453in}{0.707642in}}%
\pgfpathlineto{\pgfqpoint{0.674453in}{0.701628in}}%
\pgfpathlineto{\pgfqpoint{0.718397in}{0.701628in}}%
\pgfpathlineto{\pgfqpoint{0.718397in}{0.727347in}}%
\pgfpathlineto{\pgfqpoint{0.762342in}{0.727347in}}%
\pgfpathlineto{\pgfqpoint{0.762342in}{0.854624in}}%
\pgfpathlineto{\pgfqpoint{0.806286in}{0.854624in}}%
\pgfpathlineto{\pgfqpoint{0.806286in}{0.971290in}}%
\pgfpathlineto{\pgfqpoint{0.850231in}{0.971290in}}%
\pgfpathlineto{\pgfqpoint{0.850231in}{0.969286in}}%
\pgfpathlineto{\pgfqpoint{0.894176in}{0.969286in}}%
\pgfpathlineto{\pgfqpoint{0.894176in}{0.973051in}}%
\pgfpathlineto{\pgfqpoint{0.938120in}{0.973051in}}%
\pgfpathlineto{\pgfqpoint{0.938120in}{1.400990in}}%
\pgfpathlineto{\pgfqpoint{0.982065in}{1.400990in}}%
\pgfpathlineto{\pgfqpoint{0.982065in}{1.152769in}}%
\pgfpathlineto{\pgfqpoint{1.026009in}{1.152769in}}%
\pgfpathlineto{\pgfqpoint{1.026009in}{1.091135in}}%
\pgfpathlineto{\pgfqpoint{1.069954in}{1.091135in}}%
\pgfpathlineto{\pgfqpoint{1.069954in}{1.002755in}}%
\pgfpathlineto{\pgfqpoint{1.113899in}{1.002755in}}%
\pgfpathlineto{\pgfqpoint{1.113899in}{0.941073in}}%
\pgfpathlineto{\pgfqpoint{1.157843in}{0.941073in}}%
\pgfpathlineto{\pgfqpoint{1.157843in}{0.884060in}}%
\pgfpathlineto{\pgfqpoint{1.201788in}{0.884060in}}%
\pgfpathlineto{\pgfqpoint{1.201788in}{0.840444in}}%
\pgfpathlineto{\pgfqpoint{1.245732in}{0.840444in}}%
\pgfpathlineto{\pgfqpoint{1.245732in}{0.812573in}}%
\pgfpathlineto{\pgfqpoint{1.289677in}{0.812573in}}%
\pgfpathlineto{\pgfqpoint{1.289677in}{0.787001in}}%
\pgfpathlineto{\pgfqpoint{1.333622in}{0.787001in}}%
\pgfpathlineto{\pgfqpoint{1.333622in}{0.754045in}}%
\pgfpathlineto{\pgfqpoint{1.377566in}{0.754045in}}%
\pgfpathlineto{\pgfqpoint{1.377566in}{0.702386in}}%
\pgfpathlineto{\pgfqpoint{1.421511in}{0.702386in}}%
\pgfpathlineto{\pgfqpoint{1.421511in}{0.703413in}}%
\pgfpathlineto{\pgfqpoint{1.465455in}{0.703413in}}%
\pgfpathlineto{\pgfqpoint{1.465455in}{0.650898in}}%
\pgfpathlineto{\pgfqpoint{1.509400in}{0.650898in}}%
\pgfpathlineto{\pgfqpoint{1.509400in}{0.591758in}}%
\pgfpathlineto{\pgfqpoint{1.553345in}{0.591758in}}%
\pgfpathlineto{\pgfqpoint{1.553345in}{0.558240in}}%
\pgfpathlineto{\pgfqpoint{1.597289in}{0.558240in}}%
\pgfpathlineto{\pgfqpoint{1.597289in}{0.534672in}}%
\pgfpathlineto{\pgfqpoint{1.641234in}{0.534672in}}%
\pgfpathlineto{\pgfqpoint{1.641234in}{0.515724in}}%
\pgfpathlineto{\pgfqpoint{1.685178in}{0.515724in}}%
\pgfpathlineto{\pgfqpoint{1.685178in}{0.502033in}}%
\pgfpathlineto{\pgfqpoint{1.729123in}{0.502033in}}%
\pgfpathlineto{\pgfqpoint{1.729123in}{0.488367in}}%
\pgfpathlineto{\pgfqpoint{1.773068in}{0.488367in}}%
\pgfpathlineto{\pgfqpoint{1.773068in}{0.479566in}}%
\pgfpathlineto{\pgfqpoint{1.817012in}{0.479566in}}%
\pgfpathlineto{\pgfqpoint{1.817012in}{0.473185in}}%
\pgfpathlineto{\pgfqpoint{1.860957in}{0.473185in}}%
\pgfpathlineto{\pgfqpoint{1.860957in}{0.466950in}}%
\pgfpathlineto{\pgfqpoint{1.904902in}{0.466950in}}%
\pgfpathlineto{\pgfqpoint{1.904902in}{0.463601in}}%
\pgfpathlineto{\pgfqpoint{1.948846in}{0.463601in}}%
\pgfpathlineto{\pgfqpoint{1.948846in}{0.463063in}}%
\pgfpathlineto{\pgfqpoint{1.992791in}{0.463063in}}%
\pgfpathlineto{\pgfqpoint{1.992791in}{0.456487in}}%
\pgfpathlineto{\pgfqpoint{2.036735in}{0.456487in}}%
\pgfpathlineto{\pgfqpoint{2.036735in}{0.455558in}}%
\pgfpathlineto{\pgfqpoint{2.080680in}{0.455558in}}%
\pgfpathlineto{\pgfqpoint{2.080680in}{0.454360in}}%
\pgfpathlineto{\pgfqpoint{2.124625in}{0.454360in}}%
\pgfpathlineto{\pgfqpoint{2.124625in}{0.454115in}}%
\pgfpathlineto{\pgfqpoint{2.168569in}{0.454115in}}%
\pgfpathlineto{\pgfqpoint{2.168569in}{0.453455in}}%
\pgfpathlineto{\pgfqpoint{2.212514in}{0.453455in}}%
\pgfpathlineto{\pgfqpoint{2.212514in}{0.452771in}}%
\pgfpathlineto{\pgfqpoint{2.256458in}{0.452771in}}%
\pgfpathlineto{\pgfqpoint{2.256458in}{0.452013in}}%
\pgfpathlineto{\pgfqpoint{2.300403in}{0.452013in}}%
\pgfpathlineto{\pgfqpoint{2.300403in}{0.451768in}}%
\pgfpathlineto{\pgfqpoint{2.344348in}{0.451768in}}%
\pgfpathlineto{\pgfqpoint{2.344348in}{0.450790in}}%
\pgfpathlineto{\pgfqpoint{2.388292in}{0.450790in}}%
\pgfpathlineto{\pgfqpoint{2.388292in}{0.450228in}}%
\pgfpathlineto{\pgfqpoint{2.432237in}{0.450228in}}%
\pgfpathlineto{\pgfqpoint{2.432237in}{0.450081in}}%
\pgfpathlineto{\pgfqpoint{2.476181in}{0.450081in}}%
\pgfpathlineto{\pgfqpoint{2.476181in}{0.450032in}}%
\pgfpathlineto{\pgfqpoint{2.520126in}{0.450032in}}%
\pgfpathlineto{\pgfqpoint{2.520126in}{0.450008in}}%
\pgfpathlineto{\pgfqpoint{2.564071in}{0.450008in}}%
\pgfpathlineto{\pgfqpoint{2.564071in}{0.449983in}}%
\pgfusepath{stroke}%
\end{pgfscope}%
\begin{pgfscope}%
\pgfpathrectangle{\pgfqpoint{0.366840in}{0.449983in}}{\pgfqpoint{2.197230in}{1.169449in}} %
\pgfusepath{clip}%
\pgfsetbuttcap%
\pgfsetmiterjoin%
\pgfsetlinewidth{1.003750pt}%
\definecolor{currentstroke}{rgb}{1.000000,0.647059,0.000000}%
\pgfsetstrokecolor{currentstroke}%
\pgfsetdash{}{0pt}%
\pgfpathmoveto{\pgfqpoint{0.366840in}{0.449983in}}%
\pgfpathlineto{\pgfqpoint{0.366840in}{0.449983in}}%
\pgfpathlineto{\pgfqpoint{0.410785in}{0.449983in}}%
\pgfpathlineto{\pgfqpoint{0.410785in}{0.450208in}}%
\pgfpathlineto{\pgfqpoint{0.454729in}{0.450208in}}%
\pgfpathlineto{\pgfqpoint{0.454729in}{0.450328in}}%
\pgfpathlineto{\pgfqpoint{0.498674in}{0.450328in}}%
\pgfpathlineto{\pgfqpoint{0.498674in}{0.458128in}}%
\pgfpathlineto{\pgfqpoint{0.542619in}{0.458128in}}%
\pgfpathlineto{\pgfqpoint{0.542619in}{0.539423in}}%
\pgfpathlineto{\pgfqpoint{0.586563in}{0.539423in}}%
\pgfpathlineto{\pgfqpoint{0.586563in}{0.663168in}}%
\pgfpathlineto{\pgfqpoint{0.630508in}{0.663168in}}%
\pgfpathlineto{\pgfqpoint{0.630508in}{0.662799in}}%
\pgfpathlineto{\pgfqpoint{0.674453in}{0.662799in}}%
\pgfpathlineto{\pgfqpoint{0.674453in}{0.662719in}}%
\pgfpathlineto{\pgfqpoint{0.718397in}{0.662719in}}%
\pgfpathlineto{\pgfqpoint{0.718397in}{0.683918in}}%
\pgfpathlineto{\pgfqpoint{0.762342in}{0.683918in}}%
\pgfpathlineto{\pgfqpoint{0.762342in}{0.791663in}}%
\pgfpathlineto{\pgfqpoint{0.806286in}{0.791663in}}%
\pgfpathlineto{\pgfqpoint{0.806286in}{0.906286in}}%
\pgfpathlineto{\pgfqpoint{0.850231in}{0.906286in}}%
\pgfpathlineto{\pgfqpoint{0.850231in}{0.890941in}}%
\pgfpathlineto{\pgfqpoint{0.894176in}{0.890941in}}%
\pgfpathlineto{\pgfqpoint{0.894176in}{0.899313in}}%
\pgfpathlineto{\pgfqpoint{0.938120in}{0.899313in}}%
\pgfpathlineto{\pgfqpoint{0.938120in}{1.302895in}}%
\pgfpathlineto{\pgfqpoint{0.982065in}{1.302895in}}%
\pgfpathlineto{\pgfqpoint{0.982065in}{1.088618in}}%
\pgfpathlineto{\pgfqpoint{1.026009in}{1.088618in}}%
\pgfpathlineto{\pgfqpoint{1.026009in}{1.061725in}}%
\pgfpathlineto{\pgfqpoint{1.069954in}{1.061725in}}%
\pgfpathlineto{\pgfqpoint{1.069954in}{0.998444in}}%
\pgfpathlineto{\pgfqpoint{1.113899in}{0.998444in}}%
\pgfpathlineto{\pgfqpoint{1.113899in}{0.972279in}}%
\pgfpathlineto{\pgfqpoint{1.157843in}{0.972279in}}%
\pgfpathlineto{\pgfqpoint{1.157843in}{0.931489in}}%
\pgfpathlineto{\pgfqpoint{1.201788in}{0.931489in}}%
\pgfpathlineto{\pgfqpoint{1.201788in}{0.892226in}}%
\pgfpathlineto{\pgfqpoint{1.245732in}{0.892226in}}%
\pgfpathlineto{\pgfqpoint{1.245732in}{0.871672in}}%
\pgfpathlineto{\pgfqpoint{1.289677in}{0.871672in}}%
\pgfpathlineto{\pgfqpoint{1.289677in}{0.851342in}}%
\pgfpathlineto{\pgfqpoint{1.333622in}{0.851342in}}%
\pgfpathlineto{\pgfqpoint{1.333622in}{0.819487in}}%
\pgfpathlineto{\pgfqpoint{1.377566in}{0.819487in}}%
\pgfpathlineto{\pgfqpoint{1.377566in}{0.764522in}}%
\pgfpathlineto{\pgfqpoint{1.421511in}{0.764522in}}%
\pgfpathlineto{\pgfqpoint{1.421511in}{0.777580in}}%
\pgfpathlineto{\pgfqpoint{1.465455in}{0.777580in}}%
\pgfpathlineto{\pgfqpoint{1.465455in}{0.719476in}}%
\pgfpathlineto{\pgfqpoint{1.509400in}{0.719476in}}%
\pgfpathlineto{\pgfqpoint{1.509400in}{0.650321in}}%
\pgfpathlineto{\pgfqpoint{1.553345in}{0.650321in}}%
\pgfpathlineto{\pgfqpoint{1.553345in}{0.605991in}}%
\pgfpathlineto{\pgfqpoint{1.597289in}{0.605991in}}%
\pgfpathlineto{\pgfqpoint{1.597289in}{0.573383in}}%
\pgfpathlineto{\pgfqpoint{1.641234in}{0.573383in}}%
\pgfpathlineto{\pgfqpoint{1.641234in}{0.550266in}}%
\pgfpathlineto{\pgfqpoint{1.685178in}{0.550266in}}%
\pgfpathlineto{\pgfqpoint{1.685178in}{0.529333in}}%
\pgfpathlineto{\pgfqpoint{1.729123in}{0.529333in}}%
\pgfpathlineto{\pgfqpoint{1.729123in}{0.508423in}}%
\pgfpathlineto{\pgfqpoint{1.773068in}{0.508423in}}%
\pgfpathlineto{\pgfqpoint{1.773068in}{0.495542in}}%
\pgfpathlineto{\pgfqpoint{1.817012in}{0.495542in}}%
\pgfpathlineto{\pgfqpoint{1.817012in}{0.485300in}}%
\pgfpathlineto{\pgfqpoint{1.860957in}{0.485300in}}%
\pgfpathlineto{\pgfqpoint{1.860957in}{0.475001in}}%
\pgfpathlineto{\pgfqpoint{1.904902in}{0.475001in}}%
\pgfpathlineto{\pgfqpoint{1.904902in}{0.470216in}}%
\pgfpathlineto{\pgfqpoint{1.948846in}{0.470216in}}%
\pgfpathlineto{\pgfqpoint{1.948846in}{0.466177in}}%
\pgfpathlineto{\pgfqpoint{1.992791in}{0.466177in}}%
\pgfpathlineto{\pgfqpoint{1.992791in}{0.457747in}}%
\pgfpathlineto{\pgfqpoint{2.036735in}{0.457747in}}%
\pgfpathlineto{\pgfqpoint{2.036735in}{0.455614in}}%
\pgfpathlineto{\pgfqpoint{2.080680in}{0.455614in}}%
\pgfpathlineto{\pgfqpoint{2.080680in}{0.453189in}}%
\pgfpathlineto{\pgfqpoint{2.124625in}{0.453189in}}%
\pgfpathlineto{\pgfqpoint{2.124625in}{0.452362in}}%
\pgfpathlineto{\pgfqpoint{2.168569in}{0.452362in}}%
\pgfpathlineto{\pgfqpoint{2.168569in}{0.452039in}}%
\pgfpathlineto{\pgfqpoint{2.212514in}{0.452039in}}%
\pgfpathlineto{\pgfqpoint{2.212514in}{0.451269in}}%
\pgfpathlineto{\pgfqpoint{2.256458in}{0.451269in}}%
\pgfpathlineto{\pgfqpoint{2.256458in}{0.450763in}}%
\pgfpathlineto{\pgfqpoint{2.300403in}{0.450763in}}%
\pgfpathlineto{\pgfqpoint{2.300403in}{0.450654in}}%
\pgfpathlineto{\pgfqpoint{2.344348in}{0.450654in}}%
\pgfpathlineto{\pgfqpoint{2.344348in}{0.450292in}}%
\pgfpathlineto{\pgfqpoint{2.388292in}{0.450292in}}%
\pgfpathlineto{\pgfqpoint{2.388292in}{0.450066in}}%
\pgfpathlineto{\pgfqpoint{2.432237in}{0.450066in}}%
\pgfpathlineto{\pgfqpoint{2.432237in}{0.450013in}}%
\pgfpathlineto{\pgfqpoint{2.476181in}{0.450013in}}%
\pgfpathlineto{\pgfqpoint{2.476181in}{0.449992in}}%
\pgfpathlineto{\pgfqpoint{2.520126in}{0.449992in}}%
\pgfpathlineto{\pgfqpoint{2.520126in}{0.449993in}}%
\pgfpathlineto{\pgfqpoint{2.564071in}{0.449993in}}%
\pgfpathlineto{\pgfqpoint{2.564071in}{0.449983in}}%
\pgfusepath{stroke}%
\end{pgfscope}%
\begin{pgfscope}%
\pgfsetrectcap%
\pgfsetmiterjoin%
\pgfsetlinewidth{1.003750pt}%
\definecolor{currentstroke}{rgb}{0.000000,0.000000,0.000000}%
\pgfsetstrokecolor{currentstroke}%
\pgfsetdash{}{0pt}%
\pgfpathmoveto{\pgfqpoint{0.366840in}{1.619432in}}%
\pgfpathlineto{\pgfqpoint{2.564071in}{1.619432in}}%
\pgfusepath{stroke}%
\end{pgfscope}%
\begin{pgfscope}%
\pgfsetrectcap%
\pgfsetmiterjoin%
\pgfsetlinewidth{1.003750pt}%
\definecolor{currentstroke}{rgb}{0.000000,0.000000,0.000000}%
\pgfsetstrokecolor{currentstroke}%
\pgfsetdash{}{0pt}%
\pgfpathmoveto{\pgfqpoint{2.564071in}{0.449983in}}%
\pgfpathlineto{\pgfqpoint{2.564071in}{1.619432in}}%
\pgfusepath{stroke}%
\end{pgfscope}%
\begin{pgfscope}%
\pgfsetrectcap%
\pgfsetmiterjoin%
\pgfsetlinewidth{1.003750pt}%
\definecolor{currentstroke}{rgb}{0.000000,0.000000,0.000000}%
\pgfsetstrokecolor{currentstroke}%
\pgfsetdash{}{0pt}%
\pgfpathmoveto{\pgfqpoint{0.366840in}{0.449983in}}%
\pgfpathlineto{\pgfqpoint{2.564071in}{0.449983in}}%
\pgfusepath{stroke}%
\end{pgfscope}%
\begin{pgfscope}%
\pgfsetrectcap%
\pgfsetmiterjoin%
\pgfsetlinewidth{1.003750pt}%
\definecolor{currentstroke}{rgb}{0.000000,0.000000,0.000000}%
\pgfsetstrokecolor{currentstroke}%
\pgfsetdash{}{0pt}%
\pgfpathmoveto{\pgfqpoint{0.366840in}{0.449983in}}%
\pgfpathlineto{\pgfqpoint{0.366840in}{1.619432in}}%
\pgfusepath{stroke}%
\end{pgfscope}%
\begin{pgfscope}%
\pgfsetbuttcap%
\pgfsetroundjoin%
\definecolor{currentfill}{rgb}{0.000000,0.000000,0.000000}%
\pgfsetfillcolor{currentfill}%
\pgfsetlinewidth{0.501875pt}%
\definecolor{currentstroke}{rgb}{0.000000,0.000000,0.000000}%
\pgfsetstrokecolor{currentstroke}%
\pgfsetdash{}{0pt}%
\pgfsys@defobject{currentmarker}{\pgfqpoint{0.000000in}{0.000000in}}{\pgfqpoint{0.000000in}{0.069444in}}{%
\pgfpathmoveto{\pgfqpoint{0.000000in}{0.000000in}}%
\pgfpathlineto{\pgfqpoint{0.000000in}{0.069444in}}%
\pgfusepath{stroke,fill}%
}%
\begin{pgfscope}%
\pgfsys@transformshift{0.366840in}{0.449983in}%
\pgfsys@useobject{currentmarker}{}%
\end{pgfscope}%
\end{pgfscope}%
\begin{pgfscope}%
\pgfsetbuttcap%
\pgfsetroundjoin%
\definecolor{currentfill}{rgb}{0.000000,0.000000,0.000000}%
\pgfsetfillcolor{currentfill}%
\pgfsetlinewidth{0.501875pt}%
\definecolor{currentstroke}{rgb}{0.000000,0.000000,0.000000}%
\pgfsetstrokecolor{currentstroke}%
\pgfsetdash{}{0pt}%
\pgfsys@defobject{currentmarker}{\pgfqpoint{0.000000in}{-0.069444in}}{\pgfqpoint{0.000000in}{0.000000in}}{%
\pgfpathmoveto{\pgfqpoint{0.000000in}{0.000000in}}%
\pgfpathlineto{\pgfqpoint{0.000000in}{-0.069444in}}%
\pgfusepath{stroke,fill}%
}%
\begin{pgfscope}%
\pgfsys@transformshift{0.366840in}{1.619432in}%
\pgfsys@useobject{currentmarker}{}%
\end{pgfscope}%
\end{pgfscope}%
\begin{pgfscope}%
\pgftext[x=0.366840in,y=0.380539in,,top]{\rmfamily\fontsize{8.000000}{9.600000}\selectfont −15}%
\end{pgfscope}%
\begin{pgfscope}%
\pgfsetbuttcap%
\pgfsetroundjoin%
\definecolor{currentfill}{rgb}{0.000000,0.000000,0.000000}%
\pgfsetfillcolor{currentfill}%
\pgfsetlinewidth{0.501875pt}%
\definecolor{currentstroke}{rgb}{0.000000,0.000000,0.000000}%
\pgfsetstrokecolor{currentstroke}%
\pgfsetdash{}{0pt}%
\pgfsys@defobject{currentmarker}{\pgfqpoint{0.000000in}{0.000000in}}{\pgfqpoint{0.000000in}{0.069444in}}{%
\pgfpathmoveto{\pgfqpoint{0.000000in}{0.000000in}}%
\pgfpathlineto{\pgfqpoint{0.000000in}{0.069444in}}%
\pgfusepath{stroke,fill}%
}%
\begin{pgfscope}%
\pgfsys@transformshift{0.733045in}{0.449983in}%
\pgfsys@useobject{currentmarker}{}%
\end{pgfscope}%
\end{pgfscope}%
\begin{pgfscope}%
\pgfsetbuttcap%
\pgfsetroundjoin%
\definecolor{currentfill}{rgb}{0.000000,0.000000,0.000000}%
\pgfsetfillcolor{currentfill}%
\pgfsetlinewidth{0.501875pt}%
\definecolor{currentstroke}{rgb}{0.000000,0.000000,0.000000}%
\pgfsetstrokecolor{currentstroke}%
\pgfsetdash{}{0pt}%
\pgfsys@defobject{currentmarker}{\pgfqpoint{0.000000in}{-0.069444in}}{\pgfqpoint{0.000000in}{0.000000in}}{%
\pgfpathmoveto{\pgfqpoint{0.000000in}{0.000000in}}%
\pgfpathlineto{\pgfqpoint{0.000000in}{-0.069444in}}%
\pgfusepath{stroke,fill}%
}%
\begin{pgfscope}%
\pgfsys@transformshift{0.733045in}{1.619432in}%
\pgfsys@useobject{currentmarker}{}%
\end{pgfscope}%
\end{pgfscope}%
\begin{pgfscope}%
\pgftext[x=0.733045in,y=0.380539in,,top]{\rmfamily\fontsize{8.000000}{9.600000}\selectfont −10}%
\end{pgfscope}%
\begin{pgfscope}%
\pgfsetbuttcap%
\pgfsetroundjoin%
\definecolor{currentfill}{rgb}{0.000000,0.000000,0.000000}%
\pgfsetfillcolor{currentfill}%
\pgfsetlinewidth{0.501875pt}%
\definecolor{currentstroke}{rgb}{0.000000,0.000000,0.000000}%
\pgfsetstrokecolor{currentstroke}%
\pgfsetdash{}{0pt}%
\pgfsys@defobject{currentmarker}{\pgfqpoint{0.000000in}{0.000000in}}{\pgfqpoint{0.000000in}{0.069444in}}{%
\pgfpathmoveto{\pgfqpoint{0.000000in}{0.000000in}}%
\pgfpathlineto{\pgfqpoint{0.000000in}{0.069444in}}%
\pgfusepath{stroke,fill}%
}%
\begin{pgfscope}%
\pgfsys@transformshift{1.099250in}{0.449983in}%
\pgfsys@useobject{currentmarker}{}%
\end{pgfscope}%
\end{pgfscope}%
\begin{pgfscope}%
\pgfsetbuttcap%
\pgfsetroundjoin%
\definecolor{currentfill}{rgb}{0.000000,0.000000,0.000000}%
\pgfsetfillcolor{currentfill}%
\pgfsetlinewidth{0.501875pt}%
\definecolor{currentstroke}{rgb}{0.000000,0.000000,0.000000}%
\pgfsetstrokecolor{currentstroke}%
\pgfsetdash{}{0pt}%
\pgfsys@defobject{currentmarker}{\pgfqpoint{0.000000in}{-0.069444in}}{\pgfqpoint{0.000000in}{0.000000in}}{%
\pgfpathmoveto{\pgfqpoint{0.000000in}{0.000000in}}%
\pgfpathlineto{\pgfqpoint{0.000000in}{-0.069444in}}%
\pgfusepath{stroke,fill}%
}%
\begin{pgfscope}%
\pgfsys@transformshift{1.099250in}{1.619432in}%
\pgfsys@useobject{currentmarker}{}%
\end{pgfscope}%
\end{pgfscope}%
\begin{pgfscope}%
\pgftext[x=1.099250in,y=0.380539in,,top]{\rmfamily\fontsize{8.000000}{9.600000}\selectfont −5}%
\end{pgfscope}%
\begin{pgfscope}%
\pgfsetbuttcap%
\pgfsetroundjoin%
\definecolor{currentfill}{rgb}{0.000000,0.000000,0.000000}%
\pgfsetfillcolor{currentfill}%
\pgfsetlinewidth{0.501875pt}%
\definecolor{currentstroke}{rgb}{0.000000,0.000000,0.000000}%
\pgfsetstrokecolor{currentstroke}%
\pgfsetdash{}{0pt}%
\pgfsys@defobject{currentmarker}{\pgfqpoint{0.000000in}{0.000000in}}{\pgfqpoint{0.000000in}{0.069444in}}{%
\pgfpathmoveto{\pgfqpoint{0.000000in}{0.000000in}}%
\pgfpathlineto{\pgfqpoint{0.000000in}{0.069444in}}%
\pgfusepath{stroke,fill}%
}%
\begin{pgfscope}%
\pgfsys@transformshift{1.465455in}{0.449983in}%
\pgfsys@useobject{currentmarker}{}%
\end{pgfscope}%
\end{pgfscope}%
\begin{pgfscope}%
\pgfsetbuttcap%
\pgfsetroundjoin%
\definecolor{currentfill}{rgb}{0.000000,0.000000,0.000000}%
\pgfsetfillcolor{currentfill}%
\pgfsetlinewidth{0.501875pt}%
\definecolor{currentstroke}{rgb}{0.000000,0.000000,0.000000}%
\pgfsetstrokecolor{currentstroke}%
\pgfsetdash{}{0pt}%
\pgfsys@defobject{currentmarker}{\pgfqpoint{0.000000in}{-0.069444in}}{\pgfqpoint{0.000000in}{0.000000in}}{%
\pgfpathmoveto{\pgfqpoint{0.000000in}{0.000000in}}%
\pgfpathlineto{\pgfqpoint{0.000000in}{-0.069444in}}%
\pgfusepath{stroke,fill}%
}%
\begin{pgfscope}%
\pgfsys@transformshift{1.465455in}{1.619432in}%
\pgfsys@useobject{currentmarker}{}%
\end{pgfscope}%
\end{pgfscope}%
\begin{pgfscope}%
\pgftext[x=1.465455in,y=0.380539in,,top]{\rmfamily\fontsize{8.000000}{9.600000}\selectfont 0}%
\end{pgfscope}%
\begin{pgfscope}%
\pgfsetbuttcap%
\pgfsetroundjoin%
\definecolor{currentfill}{rgb}{0.000000,0.000000,0.000000}%
\pgfsetfillcolor{currentfill}%
\pgfsetlinewidth{0.501875pt}%
\definecolor{currentstroke}{rgb}{0.000000,0.000000,0.000000}%
\pgfsetstrokecolor{currentstroke}%
\pgfsetdash{}{0pt}%
\pgfsys@defobject{currentmarker}{\pgfqpoint{0.000000in}{0.000000in}}{\pgfqpoint{0.000000in}{0.069444in}}{%
\pgfpathmoveto{\pgfqpoint{0.000000in}{0.000000in}}%
\pgfpathlineto{\pgfqpoint{0.000000in}{0.069444in}}%
\pgfusepath{stroke,fill}%
}%
\begin{pgfscope}%
\pgfsys@transformshift{1.831661in}{0.449983in}%
\pgfsys@useobject{currentmarker}{}%
\end{pgfscope}%
\end{pgfscope}%
\begin{pgfscope}%
\pgfsetbuttcap%
\pgfsetroundjoin%
\definecolor{currentfill}{rgb}{0.000000,0.000000,0.000000}%
\pgfsetfillcolor{currentfill}%
\pgfsetlinewidth{0.501875pt}%
\definecolor{currentstroke}{rgb}{0.000000,0.000000,0.000000}%
\pgfsetstrokecolor{currentstroke}%
\pgfsetdash{}{0pt}%
\pgfsys@defobject{currentmarker}{\pgfqpoint{0.000000in}{-0.069444in}}{\pgfqpoint{0.000000in}{0.000000in}}{%
\pgfpathmoveto{\pgfqpoint{0.000000in}{0.000000in}}%
\pgfpathlineto{\pgfqpoint{0.000000in}{-0.069444in}}%
\pgfusepath{stroke,fill}%
}%
\begin{pgfscope}%
\pgfsys@transformshift{1.831661in}{1.619432in}%
\pgfsys@useobject{currentmarker}{}%
\end{pgfscope}%
\end{pgfscope}%
\begin{pgfscope}%
\pgftext[x=1.831661in,y=0.380539in,,top]{\rmfamily\fontsize{8.000000}{9.600000}\selectfont 5}%
\end{pgfscope}%
\begin{pgfscope}%
\pgfsetbuttcap%
\pgfsetroundjoin%
\definecolor{currentfill}{rgb}{0.000000,0.000000,0.000000}%
\pgfsetfillcolor{currentfill}%
\pgfsetlinewidth{0.501875pt}%
\definecolor{currentstroke}{rgb}{0.000000,0.000000,0.000000}%
\pgfsetstrokecolor{currentstroke}%
\pgfsetdash{}{0pt}%
\pgfsys@defobject{currentmarker}{\pgfqpoint{0.000000in}{0.000000in}}{\pgfqpoint{0.000000in}{0.069444in}}{%
\pgfpathmoveto{\pgfqpoint{0.000000in}{0.000000in}}%
\pgfpathlineto{\pgfqpoint{0.000000in}{0.069444in}}%
\pgfusepath{stroke,fill}%
}%
\begin{pgfscope}%
\pgfsys@transformshift{2.197866in}{0.449983in}%
\pgfsys@useobject{currentmarker}{}%
\end{pgfscope}%
\end{pgfscope}%
\begin{pgfscope}%
\pgfsetbuttcap%
\pgfsetroundjoin%
\definecolor{currentfill}{rgb}{0.000000,0.000000,0.000000}%
\pgfsetfillcolor{currentfill}%
\pgfsetlinewidth{0.501875pt}%
\definecolor{currentstroke}{rgb}{0.000000,0.000000,0.000000}%
\pgfsetstrokecolor{currentstroke}%
\pgfsetdash{}{0pt}%
\pgfsys@defobject{currentmarker}{\pgfqpoint{0.000000in}{-0.069444in}}{\pgfqpoint{0.000000in}{0.000000in}}{%
\pgfpathmoveto{\pgfqpoint{0.000000in}{0.000000in}}%
\pgfpathlineto{\pgfqpoint{0.000000in}{-0.069444in}}%
\pgfusepath{stroke,fill}%
}%
\begin{pgfscope}%
\pgfsys@transformshift{2.197866in}{1.619432in}%
\pgfsys@useobject{currentmarker}{}%
\end{pgfscope}%
\end{pgfscope}%
\begin{pgfscope}%
\pgftext[x=2.197866in,y=0.380539in,,top]{\rmfamily\fontsize{8.000000}{9.600000}\selectfont 10}%
\end{pgfscope}%
\begin{pgfscope}%
\pgfsetbuttcap%
\pgfsetroundjoin%
\definecolor{currentfill}{rgb}{0.000000,0.000000,0.000000}%
\pgfsetfillcolor{currentfill}%
\pgfsetlinewidth{0.501875pt}%
\definecolor{currentstroke}{rgb}{0.000000,0.000000,0.000000}%
\pgfsetstrokecolor{currentstroke}%
\pgfsetdash{}{0pt}%
\pgfsys@defobject{currentmarker}{\pgfqpoint{0.000000in}{0.000000in}}{\pgfqpoint{0.000000in}{0.069444in}}{%
\pgfpathmoveto{\pgfqpoint{0.000000in}{0.000000in}}%
\pgfpathlineto{\pgfqpoint{0.000000in}{0.069444in}}%
\pgfusepath{stroke,fill}%
}%
\begin{pgfscope}%
\pgfsys@transformshift{2.564071in}{0.449983in}%
\pgfsys@useobject{currentmarker}{}%
\end{pgfscope}%
\end{pgfscope}%
\begin{pgfscope}%
\pgfsetbuttcap%
\pgfsetroundjoin%
\definecolor{currentfill}{rgb}{0.000000,0.000000,0.000000}%
\pgfsetfillcolor{currentfill}%
\pgfsetlinewidth{0.501875pt}%
\definecolor{currentstroke}{rgb}{0.000000,0.000000,0.000000}%
\pgfsetstrokecolor{currentstroke}%
\pgfsetdash{}{0pt}%
\pgfsys@defobject{currentmarker}{\pgfqpoint{0.000000in}{-0.069444in}}{\pgfqpoint{0.000000in}{0.000000in}}{%
\pgfpathmoveto{\pgfqpoint{0.000000in}{0.000000in}}%
\pgfpathlineto{\pgfqpoint{0.000000in}{-0.069444in}}%
\pgfusepath{stroke,fill}%
}%
\begin{pgfscope}%
\pgfsys@transformshift{2.564071in}{1.619432in}%
\pgfsys@useobject{currentmarker}{}%
\end{pgfscope}%
\end{pgfscope}%
\begin{pgfscope}%
\pgftext[x=2.564071in,y=0.380539in,,top]{\rmfamily\fontsize{8.000000}{9.600000}\selectfont 15}%
\end{pgfscope}%
\begin{pgfscope}%
\pgftext[x=1.465455in,y=0.203564in,,top]{\rmfamily\fontsize{9.000000}{10.800000}\selectfont \(\displaystyle \mathrm{DLL}_{\mu/\pi}(\pi^-)\)}%
\end{pgfscope}%
\begin{pgfscope}%
\pgfsetbuttcap%
\pgfsetroundjoin%
\definecolor{currentfill}{rgb}{0.000000,0.000000,0.000000}%
\pgfsetfillcolor{currentfill}%
\pgfsetlinewidth{0.501875pt}%
\definecolor{currentstroke}{rgb}{0.000000,0.000000,0.000000}%
\pgfsetstrokecolor{currentstroke}%
\pgfsetdash{}{0pt}%
\pgfsys@defobject{currentmarker}{\pgfqpoint{0.000000in}{0.000000in}}{\pgfqpoint{0.069444in}{0.000000in}}{%
\pgfpathmoveto{\pgfqpoint{0.000000in}{0.000000in}}%
\pgfpathlineto{\pgfqpoint{0.069444in}{0.000000in}}%
\pgfusepath{stroke,fill}%
}%
\begin{pgfscope}%
\pgfsys@transformshift{0.366840in}{0.449983in}%
\pgfsys@useobject{currentmarker}{}%
\end{pgfscope}%
\end{pgfscope}%
\begin{pgfscope}%
\pgfsetbuttcap%
\pgfsetroundjoin%
\definecolor{currentfill}{rgb}{0.000000,0.000000,0.000000}%
\pgfsetfillcolor{currentfill}%
\pgfsetlinewidth{0.501875pt}%
\definecolor{currentstroke}{rgb}{0.000000,0.000000,0.000000}%
\pgfsetstrokecolor{currentstroke}%
\pgfsetdash{}{0pt}%
\pgfsys@defobject{currentmarker}{\pgfqpoint{-0.069444in}{0.000000in}}{\pgfqpoint{0.000000in}{0.000000in}}{%
\pgfpathmoveto{\pgfqpoint{0.000000in}{0.000000in}}%
\pgfpathlineto{\pgfqpoint{-0.069444in}{0.000000in}}%
\pgfusepath{stroke,fill}%
}%
\begin{pgfscope}%
\pgfsys@transformshift{2.564071in}{0.449983in}%
\pgfsys@useobject{currentmarker}{}%
\end{pgfscope}%
\end{pgfscope}%
\begin{pgfscope}%
\pgftext[x=0.297396in,y=0.449983in,right,]{\rmfamily\fontsize{8.000000}{9.600000}\selectfont 0.00}%
\end{pgfscope}%
\begin{pgfscope}%
\pgfsetbuttcap%
\pgfsetroundjoin%
\definecolor{currentfill}{rgb}{0.000000,0.000000,0.000000}%
\pgfsetfillcolor{currentfill}%
\pgfsetlinewidth{0.501875pt}%
\definecolor{currentstroke}{rgb}{0.000000,0.000000,0.000000}%
\pgfsetstrokecolor{currentstroke}%
\pgfsetdash{}{0pt}%
\pgfsys@defobject{currentmarker}{\pgfqpoint{0.000000in}{0.000000in}}{\pgfqpoint{0.069444in}{0.000000in}}{%
\pgfpathmoveto{\pgfqpoint{0.000000in}{0.000000in}}%
\pgfpathlineto{\pgfqpoint{0.069444in}{0.000000in}}%
\pgfusepath{stroke,fill}%
}%
\begin{pgfscope}%
\pgfsys@transformshift{0.366840in}{0.742346in}%
\pgfsys@useobject{currentmarker}{}%
\end{pgfscope}%
\end{pgfscope}%
\begin{pgfscope}%
\pgfsetbuttcap%
\pgfsetroundjoin%
\definecolor{currentfill}{rgb}{0.000000,0.000000,0.000000}%
\pgfsetfillcolor{currentfill}%
\pgfsetlinewidth{0.501875pt}%
\definecolor{currentstroke}{rgb}{0.000000,0.000000,0.000000}%
\pgfsetstrokecolor{currentstroke}%
\pgfsetdash{}{0pt}%
\pgfsys@defobject{currentmarker}{\pgfqpoint{-0.069444in}{0.000000in}}{\pgfqpoint{0.000000in}{0.000000in}}{%
\pgfpathmoveto{\pgfqpoint{0.000000in}{0.000000in}}%
\pgfpathlineto{\pgfqpoint{-0.069444in}{0.000000in}}%
\pgfusepath{stroke,fill}%
}%
\begin{pgfscope}%
\pgfsys@transformshift{2.564071in}{0.742346in}%
\pgfsys@useobject{currentmarker}{}%
\end{pgfscope}%
\end{pgfscope}%
\begin{pgfscope}%
\pgftext[x=0.297396in,y=0.742346in,right,]{\rmfamily\fontsize{8.000000}{9.600000}\selectfont 0.05}%
\end{pgfscope}%
\begin{pgfscope}%
\pgfsetbuttcap%
\pgfsetroundjoin%
\definecolor{currentfill}{rgb}{0.000000,0.000000,0.000000}%
\pgfsetfillcolor{currentfill}%
\pgfsetlinewidth{0.501875pt}%
\definecolor{currentstroke}{rgb}{0.000000,0.000000,0.000000}%
\pgfsetstrokecolor{currentstroke}%
\pgfsetdash{}{0pt}%
\pgfsys@defobject{currentmarker}{\pgfqpoint{0.000000in}{0.000000in}}{\pgfqpoint{0.069444in}{0.000000in}}{%
\pgfpathmoveto{\pgfqpoint{0.000000in}{0.000000in}}%
\pgfpathlineto{\pgfqpoint{0.069444in}{0.000000in}}%
\pgfusepath{stroke,fill}%
}%
\begin{pgfscope}%
\pgfsys@transformshift{0.366840in}{1.034708in}%
\pgfsys@useobject{currentmarker}{}%
\end{pgfscope}%
\end{pgfscope}%
\begin{pgfscope}%
\pgfsetbuttcap%
\pgfsetroundjoin%
\definecolor{currentfill}{rgb}{0.000000,0.000000,0.000000}%
\pgfsetfillcolor{currentfill}%
\pgfsetlinewidth{0.501875pt}%
\definecolor{currentstroke}{rgb}{0.000000,0.000000,0.000000}%
\pgfsetstrokecolor{currentstroke}%
\pgfsetdash{}{0pt}%
\pgfsys@defobject{currentmarker}{\pgfqpoint{-0.069444in}{0.000000in}}{\pgfqpoint{0.000000in}{0.000000in}}{%
\pgfpathmoveto{\pgfqpoint{0.000000in}{0.000000in}}%
\pgfpathlineto{\pgfqpoint{-0.069444in}{0.000000in}}%
\pgfusepath{stroke,fill}%
}%
\begin{pgfscope}%
\pgfsys@transformshift{2.564071in}{1.034708in}%
\pgfsys@useobject{currentmarker}{}%
\end{pgfscope}%
\end{pgfscope}%
\begin{pgfscope}%
\pgftext[x=0.297396in,y=1.034708in,right,]{\rmfamily\fontsize{8.000000}{9.600000}\selectfont 0.10}%
\end{pgfscope}%
\begin{pgfscope}%
\pgfsetbuttcap%
\pgfsetroundjoin%
\definecolor{currentfill}{rgb}{0.000000,0.000000,0.000000}%
\pgfsetfillcolor{currentfill}%
\pgfsetlinewidth{0.501875pt}%
\definecolor{currentstroke}{rgb}{0.000000,0.000000,0.000000}%
\pgfsetstrokecolor{currentstroke}%
\pgfsetdash{}{0pt}%
\pgfsys@defobject{currentmarker}{\pgfqpoint{0.000000in}{0.000000in}}{\pgfqpoint{0.069444in}{0.000000in}}{%
\pgfpathmoveto{\pgfqpoint{0.000000in}{0.000000in}}%
\pgfpathlineto{\pgfqpoint{0.069444in}{0.000000in}}%
\pgfusepath{stroke,fill}%
}%
\begin{pgfscope}%
\pgfsys@transformshift{0.366840in}{1.327070in}%
\pgfsys@useobject{currentmarker}{}%
\end{pgfscope}%
\end{pgfscope}%
\begin{pgfscope}%
\pgfsetbuttcap%
\pgfsetroundjoin%
\definecolor{currentfill}{rgb}{0.000000,0.000000,0.000000}%
\pgfsetfillcolor{currentfill}%
\pgfsetlinewidth{0.501875pt}%
\definecolor{currentstroke}{rgb}{0.000000,0.000000,0.000000}%
\pgfsetstrokecolor{currentstroke}%
\pgfsetdash{}{0pt}%
\pgfsys@defobject{currentmarker}{\pgfqpoint{-0.069444in}{0.000000in}}{\pgfqpoint{0.000000in}{0.000000in}}{%
\pgfpathmoveto{\pgfqpoint{0.000000in}{0.000000in}}%
\pgfpathlineto{\pgfqpoint{-0.069444in}{0.000000in}}%
\pgfusepath{stroke,fill}%
}%
\begin{pgfscope}%
\pgfsys@transformshift{2.564071in}{1.327070in}%
\pgfsys@useobject{currentmarker}{}%
\end{pgfscope}%
\end{pgfscope}%
\begin{pgfscope}%
\pgftext[x=0.297396in,y=1.327070in,right,]{\rmfamily\fontsize{8.000000}{9.600000}\selectfont 0.15}%
\end{pgfscope}%
\begin{pgfscope}%
\pgfsetbuttcap%
\pgfsetroundjoin%
\definecolor{currentfill}{rgb}{0.000000,0.000000,0.000000}%
\pgfsetfillcolor{currentfill}%
\pgfsetlinewidth{0.501875pt}%
\definecolor{currentstroke}{rgb}{0.000000,0.000000,0.000000}%
\pgfsetstrokecolor{currentstroke}%
\pgfsetdash{}{0pt}%
\pgfsys@defobject{currentmarker}{\pgfqpoint{0.000000in}{0.000000in}}{\pgfqpoint{0.069444in}{0.000000in}}{%
\pgfpathmoveto{\pgfqpoint{0.000000in}{0.000000in}}%
\pgfpathlineto{\pgfqpoint{0.069444in}{0.000000in}}%
\pgfusepath{stroke,fill}%
}%
\begin{pgfscope}%
\pgfsys@transformshift{0.366840in}{1.619432in}%
\pgfsys@useobject{currentmarker}{}%
\end{pgfscope}%
\end{pgfscope}%
\begin{pgfscope}%
\pgfsetbuttcap%
\pgfsetroundjoin%
\definecolor{currentfill}{rgb}{0.000000,0.000000,0.000000}%
\pgfsetfillcolor{currentfill}%
\pgfsetlinewidth{0.501875pt}%
\definecolor{currentstroke}{rgb}{0.000000,0.000000,0.000000}%
\pgfsetstrokecolor{currentstroke}%
\pgfsetdash{}{0pt}%
\pgfsys@defobject{currentmarker}{\pgfqpoint{-0.069444in}{0.000000in}}{\pgfqpoint{0.000000in}{0.000000in}}{%
\pgfpathmoveto{\pgfqpoint{0.000000in}{0.000000in}}%
\pgfpathlineto{\pgfqpoint{-0.069444in}{0.000000in}}%
\pgfusepath{stroke,fill}%
}%
\begin{pgfscope}%
\pgfsys@transformshift{2.564071in}{1.619432in}%
\pgfsys@useobject{currentmarker}{}%
\end{pgfscope}%
\end{pgfscope}%
\begin{pgfscope}%
\pgftext[x=0.297396in,y=1.619432in,right,]{\rmfamily\fontsize{8.000000}{9.600000}\selectfont 0.20}%
\end{pgfscope}%
\end{pgfpicture}%
\makeatother%
\endgroup%

	\end{subfigure}

	\begin{subfigure}[t]{0.49\textwidth}
		\centering
    %\includegraphics[width=\textwidth]{store/variables/DATA_MC_REWEIGHTED_muminus_PIDmu.pdf}
    %% Creator: Matplotlib, PGF backend
%%
%% To include the figure in your LaTeX document, write
%%   \input{<filename>.pgf}
%%
%% Make sure the required packages are loaded in your preamble
%%   \usepackage{pgf}
%%
%% Figures using additional raster images can only be included by \input if
%% they are in the same directory as the main LaTeX file. For loading figures
%% from other directories you can use the `import` package
%%   \usepackage{import}
%% and then include the figures with
%%   \import{<path to file>}{<filename>.pgf}
%%
%% Matplotlib used the following preamble
%%   \usepackage{fontspec}
%%   \setmainfont{DejaVu Serif}
%%   \setsansfont{DejaVu Sans}
%%   \setmonofont{DejaVu Sans Mono}
%%
\begingroup%
\makeatletter%
\begin{pgfpicture}%
\pgfpathrectangle{\pgfpointorigin}{\pgfqpoint{2.682342in}{1.723197in}}%
\pgfusepath{use as bounding box, clip}%
\begin{pgfscope}%
\pgfsetbuttcap%
\pgfsetmiterjoin%
\definecolor{currentfill}{rgb}{1.000000,1.000000,1.000000}%
\pgfsetfillcolor{currentfill}%
\pgfsetlinewidth{0.000000pt}%
\definecolor{currentstroke}{rgb}{1.000000,1.000000,1.000000}%
\pgfsetstrokecolor{currentstroke}%
\pgfsetdash{}{0pt}%
\pgfpathmoveto{\pgfqpoint{0.000000in}{0.000000in}}%
\pgfpathlineto{\pgfqpoint{2.682342in}{0.000000in}}%
\pgfpathlineto{\pgfqpoint{2.682342in}{1.723197in}}%
\pgfpathlineto{\pgfqpoint{0.000000in}{1.723197in}}%
\pgfpathclose%
\pgfusepath{fill}%
\end{pgfscope}%
\begin{pgfscope}%
\pgfsetbuttcap%
\pgfsetmiterjoin%
\definecolor{currentfill}{rgb}{1.000000,1.000000,1.000000}%
\pgfsetfillcolor{currentfill}%
\pgfsetlinewidth{0.000000pt}%
\definecolor{currentstroke}{rgb}{0.000000,0.000000,0.000000}%
\pgfsetstrokecolor{currentstroke}%
\pgfsetstrokeopacity{0.000000}%
\pgfsetdash{}{0pt}%
\pgfpathmoveto{\pgfqpoint{0.366840in}{0.449983in}}%
\pgfpathlineto{\pgfqpoint{2.561650in}{0.449983in}}%
\pgfpathlineto{\pgfqpoint{2.561650in}{1.619432in}}%
\pgfpathlineto{\pgfqpoint{0.366840in}{1.619432in}}%
\pgfpathclose%
\pgfusepath{fill}%
\end{pgfscope}%
\begin{pgfscope}%
\pgfpathrectangle{\pgfqpoint{0.366840in}{0.449983in}}{\pgfqpoint{2.194810in}{1.169449in}} %
\pgfusepath{clip}%
\pgfsetbuttcap%
\pgfsetmiterjoin%
\definecolor{currentfill}{rgb}{0.215686,0.470588,0.749020}%
\pgfsetfillcolor{currentfill}%
\pgfsetlinewidth{0.000000pt}%
\definecolor{currentstroke}{rgb}{0.000000,0.000000,0.000000}%
\pgfsetstrokecolor{currentstroke}%
\pgfsetdash{}{0pt}%
\pgfpathmoveto{\pgfqpoint{0.586346in}{0.449983in}}%
\pgfpathlineto{\pgfqpoint{0.586346in}{0.469900in}}%
\pgfpathlineto{\pgfqpoint{0.625401in}{0.469900in}}%
\pgfpathlineto{\pgfqpoint{0.625401in}{0.475052in}}%
\pgfpathlineto{\pgfqpoint{0.664457in}{0.475052in}}%
\pgfpathlineto{\pgfqpoint{0.664457in}{0.478651in}}%
\pgfpathlineto{\pgfqpoint{0.703512in}{0.478651in}}%
\pgfpathlineto{\pgfqpoint{0.703512in}{0.492213in}}%
\pgfpathlineto{\pgfqpoint{0.742567in}{0.492213in}}%
\pgfpathlineto{\pgfqpoint{0.742567in}{0.497279in}}%
\pgfpathlineto{\pgfqpoint{0.781623in}{0.497279in}}%
\pgfpathlineto{\pgfqpoint{0.781623in}{0.513888in}}%
\pgfpathlineto{\pgfqpoint{0.820678in}{0.513888in}}%
\pgfpathlineto{\pgfqpoint{0.820678in}{0.519253in}}%
\pgfpathlineto{\pgfqpoint{0.859734in}{0.519253in}}%
\pgfpathlineto{\pgfqpoint{0.859734in}{0.541169in}}%
\pgfpathlineto{\pgfqpoint{0.898789in}{0.541169in}}%
\pgfpathlineto{\pgfqpoint{0.898789in}{0.550647in}}%
\pgfpathlineto{\pgfqpoint{0.937845in}{0.550647in}}%
\pgfpathlineto{\pgfqpoint{0.937845in}{0.581601in}}%
\pgfpathlineto{\pgfqpoint{0.976900in}{0.581601in}}%
\pgfpathlineto{\pgfqpoint{0.976900in}{0.600527in}}%
\pgfpathlineto{\pgfqpoint{1.015955in}{0.600527in}}%
\pgfpathlineto{\pgfqpoint{1.015955in}{0.628016in}}%
\pgfpathlineto{\pgfqpoint{1.055011in}{0.628016in}}%
\pgfpathlineto{\pgfqpoint{1.055011in}{0.663491in}}%
\pgfpathlineto{\pgfqpoint{1.094066in}{0.663491in}}%
\pgfpathlineto{\pgfqpoint{1.094066in}{0.710622in}}%
\pgfpathlineto{\pgfqpoint{1.133122in}{0.710622in}}%
\pgfpathlineto{\pgfqpoint{1.133122in}{0.757767in}}%
\pgfpathlineto{\pgfqpoint{1.172177in}{0.757767in}}%
\pgfpathlineto{\pgfqpoint{1.172177in}{0.820606in}}%
\pgfpathlineto{\pgfqpoint{1.211233in}{0.820606in}}%
\pgfpathlineto{\pgfqpoint{1.211233in}{0.885918in}}%
\pgfpathlineto{\pgfqpoint{1.250288in}{0.885918in}}%
\pgfpathlineto{\pgfqpoint{1.250288in}{0.959487in}}%
\pgfpathlineto{\pgfqpoint{1.289343in}{0.959487in}}%
\pgfpathlineto{\pgfqpoint{1.289343in}{1.077903in}}%
\pgfpathlineto{\pgfqpoint{1.328399in}{1.077903in}}%
\pgfpathlineto{\pgfqpoint{1.328399in}{1.232588in}}%
\pgfpathlineto{\pgfqpoint{1.367454in}{1.232588in}}%
\pgfpathlineto{\pgfqpoint{1.367454in}{1.340602in}}%
\pgfpathlineto{\pgfqpoint{1.406510in}{1.340602in}}%
\pgfpathlineto{\pgfqpoint{1.406510in}{1.255742in}}%
\pgfpathlineto{\pgfqpoint{1.445565in}{1.255742in}}%
\pgfpathlineto{\pgfqpoint{1.445565in}{1.296041in}}%
\pgfpathlineto{\pgfqpoint{1.484621in}{1.296041in}}%
\pgfpathlineto{\pgfqpoint{1.484621in}{1.412574in}}%
\pgfpathlineto{\pgfqpoint{1.523676in}{1.412574in}}%
\pgfpathlineto{\pgfqpoint{1.523676in}{1.442389in}}%
\pgfpathlineto{\pgfqpoint{1.562731in}{1.442389in}}%
\pgfpathlineto{\pgfqpoint{1.562731in}{1.464656in}}%
\pgfpathlineto{\pgfqpoint{1.601787in}{1.464656in}}%
\pgfpathlineto{\pgfqpoint{1.601787in}{1.510291in}}%
\pgfpathlineto{\pgfqpoint{1.640842in}{1.510291in}}%
\pgfpathlineto{\pgfqpoint{1.640842in}{1.571990in}}%
\pgfpathlineto{\pgfqpoint{1.679898in}{1.571990in}}%
\pgfpathlineto{\pgfqpoint{1.679898in}{1.552750in}}%
\pgfpathlineto{\pgfqpoint{1.718953in}{1.552750in}}%
\pgfpathlineto{\pgfqpoint{1.718953in}{1.520728in}}%
\pgfpathlineto{\pgfqpoint{1.758009in}{1.520728in}}%
\pgfpathlineto{\pgfqpoint{1.758009in}{1.395052in}}%
\pgfpathlineto{\pgfqpoint{1.797064in}{1.395052in}}%
\pgfpathlineto{\pgfqpoint{1.797064in}{1.241911in}}%
\pgfpathlineto{\pgfqpoint{1.836119in}{1.241911in}}%
\pgfpathlineto{\pgfqpoint{1.836119in}{1.098046in}}%
\pgfpathlineto{\pgfqpoint{1.875175in}{1.098046in}}%
\pgfpathlineto{\pgfqpoint{1.875175in}{0.958130in}}%
\pgfpathlineto{\pgfqpoint{1.914230in}{0.958130in}}%
\pgfpathlineto{\pgfqpoint{1.914230in}{0.915808in}}%
\pgfpathlineto{\pgfqpoint{1.953286in}{0.915808in}}%
\pgfpathlineto{\pgfqpoint{1.953286in}{0.879322in}}%
\pgfpathlineto{\pgfqpoint{1.992341in}{0.879322in}}%
\pgfpathlineto{\pgfqpoint{1.992341in}{0.837609in}}%
\pgfpathlineto{\pgfqpoint{2.031396in}{0.837609in}}%
\pgfpathlineto{\pgfqpoint{2.031396in}{0.821021in}}%
\pgfpathlineto{\pgfqpoint{2.070452in}{0.821021in}}%
\pgfpathlineto{\pgfqpoint{2.070452in}{0.811457in}}%
\pgfpathlineto{\pgfqpoint{2.109507in}{0.811457in}}%
\pgfpathlineto{\pgfqpoint{2.109507in}{0.797010in}}%
\pgfpathlineto{\pgfqpoint{2.148563in}{0.797010in}}%
\pgfpathlineto{\pgfqpoint{2.148563in}{0.747840in}}%
\pgfpathlineto{\pgfqpoint{2.187618in}{0.747840in}}%
\pgfpathlineto{\pgfqpoint{2.187618in}{0.679052in}}%
\pgfpathlineto{\pgfqpoint{2.226674in}{0.679052in}}%
\pgfpathlineto{\pgfqpoint{2.226674in}{0.620655in}}%
\pgfpathlineto{\pgfqpoint{2.265729in}{0.620655in}}%
\pgfpathlineto{\pgfqpoint{2.265729in}{0.555642in}}%
\pgfpathlineto{\pgfqpoint{2.304784in}{0.555642in}}%
\pgfpathlineto{\pgfqpoint{2.304784in}{0.521943in}}%
\pgfpathlineto{\pgfqpoint{2.343840in}{0.521943in}}%
\pgfpathlineto{\pgfqpoint{2.343840in}{0.489856in}}%
\pgfpathlineto{\pgfqpoint{2.382895in}{0.489856in}}%
\pgfpathlineto{\pgfqpoint{2.382895in}{0.475781in}}%
\pgfpathlineto{\pgfqpoint{2.421951in}{0.475781in}}%
\pgfpathlineto{\pgfqpoint{2.421951in}{0.462257in}}%
\pgfpathlineto{\pgfqpoint{2.461006in}{0.462257in}}%
\pgfpathlineto{\pgfqpoint{2.461006in}{0.453115in}}%
\pgfpathlineto{\pgfqpoint{2.500062in}{0.453115in}}%
\pgfpathlineto{\pgfqpoint{2.500062in}{0.450780in}}%
\pgfpathlineto{\pgfqpoint{2.539117in}{0.450780in}}%
\pgfpathlineto{\pgfqpoint{2.539117in}{0.449983in}}%
\pgfpathlineto{\pgfqpoint{2.500062in}{0.449983in}}%
\pgfpathlineto{\pgfqpoint{2.500062in}{0.449983in}}%
\pgfpathlineto{\pgfqpoint{2.461006in}{0.449983in}}%
\pgfpathlineto{\pgfqpoint{2.461006in}{0.449983in}}%
\pgfpathlineto{\pgfqpoint{2.421951in}{0.449983in}}%
\pgfpathlineto{\pgfqpoint{2.421951in}{0.449983in}}%
\pgfpathlineto{\pgfqpoint{2.382895in}{0.449983in}}%
\pgfpathlineto{\pgfqpoint{2.382895in}{0.449983in}}%
\pgfpathlineto{\pgfqpoint{2.343840in}{0.449983in}}%
\pgfpathlineto{\pgfqpoint{2.343840in}{0.449983in}}%
\pgfpathlineto{\pgfqpoint{2.304784in}{0.449983in}}%
\pgfpathlineto{\pgfqpoint{2.304784in}{0.449983in}}%
\pgfpathlineto{\pgfqpoint{2.265729in}{0.449983in}}%
\pgfpathlineto{\pgfqpoint{2.265729in}{0.449983in}}%
\pgfpathlineto{\pgfqpoint{2.226674in}{0.449983in}}%
\pgfpathlineto{\pgfqpoint{2.226674in}{0.449983in}}%
\pgfpathlineto{\pgfqpoint{2.187618in}{0.449983in}}%
\pgfpathlineto{\pgfqpoint{2.187618in}{0.449983in}}%
\pgfpathlineto{\pgfqpoint{2.148563in}{0.449983in}}%
\pgfpathlineto{\pgfqpoint{2.148563in}{0.449983in}}%
\pgfpathlineto{\pgfqpoint{2.109507in}{0.449983in}}%
\pgfpathlineto{\pgfqpoint{2.109507in}{0.449983in}}%
\pgfpathlineto{\pgfqpoint{2.070452in}{0.449983in}}%
\pgfpathlineto{\pgfqpoint{2.070452in}{0.449983in}}%
\pgfpathlineto{\pgfqpoint{2.031396in}{0.449983in}}%
\pgfpathlineto{\pgfqpoint{2.031396in}{0.449983in}}%
\pgfpathlineto{\pgfqpoint{1.992341in}{0.449983in}}%
\pgfpathlineto{\pgfqpoint{1.992341in}{0.449983in}}%
\pgfpathlineto{\pgfqpoint{1.953286in}{0.449983in}}%
\pgfpathlineto{\pgfqpoint{1.953286in}{0.449983in}}%
\pgfpathlineto{\pgfqpoint{1.914230in}{0.449983in}}%
\pgfpathlineto{\pgfqpoint{1.914230in}{0.449983in}}%
\pgfpathlineto{\pgfqpoint{1.875175in}{0.449983in}}%
\pgfpathlineto{\pgfqpoint{1.875175in}{0.449983in}}%
\pgfpathlineto{\pgfqpoint{1.836119in}{0.449983in}}%
\pgfpathlineto{\pgfqpoint{1.836119in}{0.449983in}}%
\pgfpathlineto{\pgfqpoint{1.797064in}{0.449983in}}%
\pgfpathlineto{\pgfqpoint{1.797064in}{0.449983in}}%
\pgfpathlineto{\pgfqpoint{1.758009in}{0.449983in}}%
\pgfpathlineto{\pgfqpoint{1.758009in}{0.449983in}}%
\pgfpathlineto{\pgfqpoint{1.718953in}{0.449983in}}%
\pgfpathlineto{\pgfqpoint{1.718953in}{0.449983in}}%
\pgfpathlineto{\pgfqpoint{1.679898in}{0.449983in}}%
\pgfpathlineto{\pgfqpoint{1.679898in}{0.449983in}}%
\pgfpathlineto{\pgfqpoint{1.640842in}{0.449983in}}%
\pgfpathlineto{\pgfqpoint{1.640842in}{0.449983in}}%
\pgfpathlineto{\pgfqpoint{1.601787in}{0.449983in}}%
\pgfpathlineto{\pgfqpoint{1.601787in}{0.449983in}}%
\pgfpathlineto{\pgfqpoint{1.562731in}{0.449983in}}%
\pgfpathlineto{\pgfqpoint{1.562731in}{0.449983in}}%
\pgfpathlineto{\pgfqpoint{1.523676in}{0.449983in}}%
\pgfpathlineto{\pgfqpoint{1.523676in}{0.449983in}}%
\pgfpathlineto{\pgfqpoint{1.484621in}{0.449983in}}%
\pgfpathlineto{\pgfqpoint{1.484621in}{0.449983in}}%
\pgfpathlineto{\pgfqpoint{1.445565in}{0.449983in}}%
\pgfpathlineto{\pgfqpoint{1.445565in}{0.449983in}}%
\pgfpathlineto{\pgfqpoint{1.406510in}{0.449983in}}%
\pgfpathlineto{\pgfqpoint{1.406510in}{0.449983in}}%
\pgfpathlineto{\pgfqpoint{1.367454in}{0.449983in}}%
\pgfpathlineto{\pgfqpoint{1.367454in}{0.449983in}}%
\pgfpathlineto{\pgfqpoint{1.328399in}{0.449983in}}%
\pgfpathlineto{\pgfqpoint{1.328399in}{0.449983in}}%
\pgfpathlineto{\pgfqpoint{1.289343in}{0.449983in}}%
\pgfpathlineto{\pgfqpoint{1.289343in}{0.449983in}}%
\pgfpathlineto{\pgfqpoint{1.250288in}{0.449983in}}%
\pgfpathlineto{\pgfqpoint{1.250288in}{0.449983in}}%
\pgfpathlineto{\pgfqpoint{1.211233in}{0.449983in}}%
\pgfpathlineto{\pgfqpoint{1.211233in}{0.449983in}}%
\pgfpathlineto{\pgfqpoint{1.172177in}{0.449983in}}%
\pgfpathlineto{\pgfqpoint{1.172177in}{0.449983in}}%
\pgfpathlineto{\pgfqpoint{1.133122in}{0.449983in}}%
\pgfpathlineto{\pgfqpoint{1.133122in}{0.449983in}}%
\pgfpathlineto{\pgfqpoint{1.094066in}{0.449983in}}%
\pgfpathlineto{\pgfqpoint{1.094066in}{0.449983in}}%
\pgfpathlineto{\pgfqpoint{1.055011in}{0.449983in}}%
\pgfpathlineto{\pgfqpoint{1.055011in}{0.449983in}}%
\pgfpathlineto{\pgfqpoint{1.015955in}{0.449983in}}%
\pgfpathlineto{\pgfqpoint{1.015955in}{0.449983in}}%
\pgfpathlineto{\pgfqpoint{0.976900in}{0.449983in}}%
\pgfpathlineto{\pgfqpoint{0.976900in}{0.449983in}}%
\pgfpathlineto{\pgfqpoint{0.937845in}{0.449983in}}%
\pgfpathlineto{\pgfqpoint{0.937845in}{0.449983in}}%
\pgfpathlineto{\pgfqpoint{0.898789in}{0.449983in}}%
\pgfpathlineto{\pgfqpoint{0.898789in}{0.449983in}}%
\pgfpathlineto{\pgfqpoint{0.859734in}{0.449983in}}%
\pgfpathlineto{\pgfqpoint{0.859734in}{0.449983in}}%
\pgfpathlineto{\pgfqpoint{0.820678in}{0.449983in}}%
\pgfpathlineto{\pgfqpoint{0.820678in}{0.449983in}}%
\pgfpathlineto{\pgfqpoint{0.781623in}{0.449983in}}%
\pgfpathlineto{\pgfqpoint{0.781623in}{0.449983in}}%
\pgfpathlineto{\pgfqpoint{0.742567in}{0.449983in}}%
\pgfpathlineto{\pgfqpoint{0.742567in}{0.449983in}}%
\pgfpathlineto{\pgfqpoint{0.703512in}{0.449983in}}%
\pgfpathlineto{\pgfqpoint{0.703512in}{0.449983in}}%
\pgfpathlineto{\pgfqpoint{0.664457in}{0.449983in}}%
\pgfpathlineto{\pgfqpoint{0.664457in}{0.449983in}}%
\pgfpathlineto{\pgfqpoint{0.625401in}{0.449983in}}%
\pgfpathlineto{\pgfqpoint{0.625401in}{0.449983in}}%
\pgfpathlineto{\pgfqpoint{0.586346in}{0.449983in}}%
\pgfusepath{fill}%
\end{pgfscope}%
\begin{pgfscope}%
\pgfpathrectangle{\pgfqpoint{0.366840in}{0.449983in}}{\pgfqpoint{2.194810in}{1.169449in}} %
\pgfusepath{clip}%
\pgfsetbuttcap%
\pgfsetmiterjoin%
\pgfsetlinewidth{0.501875pt}%
\definecolor{currentstroke}{rgb}{1.000000,0.000000,0.000000}%
\pgfsetstrokecolor{currentstroke}%
\pgfsetdash{}{0pt}%
\pgfpathmoveto{\pgfqpoint{0.586346in}{0.449983in}}%
\pgfpathlineto{\pgfqpoint{0.586346in}{0.465131in}}%
\pgfpathlineto{\pgfqpoint{0.625401in}{0.465131in}}%
\pgfpathlineto{\pgfqpoint{0.625401in}{0.470695in}}%
\pgfpathlineto{\pgfqpoint{0.664457in}{0.470695in}}%
\pgfpathlineto{\pgfqpoint{0.664457in}{0.473271in}}%
\pgfpathlineto{\pgfqpoint{0.703512in}{0.473271in}}%
\pgfpathlineto{\pgfqpoint{0.703512in}{0.482339in}}%
\pgfpathlineto{\pgfqpoint{0.742567in}{0.482339in}}%
\pgfpathlineto{\pgfqpoint{0.742567in}{0.487079in}}%
\pgfpathlineto{\pgfqpoint{0.781623in}{0.487079in}}%
\pgfpathlineto{\pgfqpoint{0.781623in}{0.498929in}}%
\pgfpathlineto{\pgfqpoint{0.820678in}{0.498929in}}%
\pgfpathlineto{\pgfqpoint{0.820678in}{0.508358in}}%
\pgfpathlineto{\pgfqpoint{0.859734in}{0.508358in}}%
\pgfpathlineto{\pgfqpoint{0.859734in}{0.524948in}}%
\pgfpathlineto{\pgfqpoint{0.898789in}{0.524948in}}%
\pgfpathlineto{\pgfqpoint{0.898789in}{0.542414in}}%
\pgfpathlineto{\pgfqpoint{0.937845in}{0.542414in}}%
\pgfpathlineto{\pgfqpoint{0.937845in}{0.560395in}}%
\pgfpathlineto{\pgfqpoint{0.976900in}{0.560395in}}%
\pgfpathlineto{\pgfqpoint{0.976900in}{0.590226in}}%
\pgfpathlineto{\pgfqpoint{1.015955in}{0.590226in}}%
\pgfpathlineto{\pgfqpoint{1.015955in}{0.623046in}}%
\pgfpathlineto{\pgfqpoint{1.055011in}{0.623046in}}%
\pgfpathlineto{\pgfqpoint{1.055011in}{0.656329in}}%
\pgfpathlineto{\pgfqpoint{1.094066in}{0.656329in}}%
\pgfpathlineto{\pgfqpoint{1.094066in}{0.692961in}}%
\pgfpathlineto{\pgfqpoint{1.133122in}{0.692961in}}%
\pgfpathlineto{\pgfqpoint{1.133122in}{0.739434in}}%
\pgfpathlineto{\pgfqpoint{1.172177in}{0.739434in}}%
\pgfpathlineto{\pgfqpoint{1.172177in}{0.807700in}}%
\pgfpathlineto{\pgfqpoint{1.211233in}{0.807700in}}%
\pgfpathlineto{\pgfqpoint{1.211233in}{0.854070in}}%
\pgfpathlineto{\pgfqpoint{1.250288in}{0.854070in}}%
\pgfpathlineto{\pgfqpoint{1.250288in}{0.944233in}}%
\pgfpathlineto{\pgfqpoint{1.289343in}{0.944233in}}%
\pgfpathlineto{\pgfqpoint{1.289343in}{1.069071in}}%
\pgfpathlineto{\pgfqpoint{1.328399in}{1.069071in}}%
\pgfpathlineto{\pgfqpoint{1.328399in}{1.235074in}}%
\pgfpathlineto{\pgfqpoint{1.367454in}{1.235074in}}%
\pgfpathlineto{\pgfqpoint{1.367454in}{1.343889in}}%
\pgfpathlineto{\pgfqpoint{1.406510in}{1.343889in}}%
\pgfpathlineto{\pgfqpoint{1.406510in}{1.233220in}}%
\pgfpathlineto{\pgfqpoint{1.445565in}{1.233220in}}%
\pgfpathlineto{\pgfqpoint{1.445565in}{1.306072in}}%
\pgfpathlineto{\pgfqpoint{1.484621in}{1.306072in}}%
\pgfpathlineto{\pgfqpoint{1.484621in}{1.416225in}}%
\pgfpathlineto{\pgfqpoint{1.523676in}{1.416225in}}%
\pgfpathlineto{\pgfqpoint{1.523676in}{1.439719in}}%
\pgfpathlineto{\pgfqpoint{1.562731in}{1.439719in}}%
\pgfpathlineto{\pgfqpoint{1.562731in}{1.479134in}}%
\pgfpathlineto{\pgfqpoint{1.601787in}{1.479134in}}%
\pgfpathlineto{\pgfqpoint{1.601787in}{1.519733in}}%
\pgfpathlineto{\pgfqpoint{1.640842in}{1.519733in}}%
\pgfpathlineto{\pgfqpoint{1.640842in}{1.552295in}}%
\pgfpathlineto{\pgfqpoint{1.679898in}{1.552295in}}%
\pgfpathlineto{\pgfqpoint{1.679898in}{1.536993in}}%
\pgfpathlineto{\pgfqpoint{1.718953in}{1.536993in}}%
\pgfpathlineto{\pgfqpoint{1.718953in}{1.474806in}}%
\pgfpathlineto{\pgfqpoint{1.758009in}{1.474806in}}%
\pgfpathlineto{\pgfqpoint{1.758009in}{1.350020in}}%
\pgfpathlineto{\pgfqpoint{1.797064in}{1.350020in}}%
\pgfpathlineto{\pgfqpoint{1.797064in}{1.188292in}}%
\pgfpathlineto{\pgfqpoint{1.836119in}{1.188292in}}%
\pgfpathlineto{\pgfqpoint{1.836119in}{1.049389in}}%
\pgfpathlineto{\pgfqpoint{1.875175in}{1.049389in}}%
\pgfpathlineto{\pgfqpoint{1.875175in}{0.957938in}}%
\pgfpathlineto{\pgfqpoint{1.914230in}{0.957938in}}%
\pgfpathlineto{\pgfqpoint{1.914230in}{0.891063in}}%
\pgfpathlineto{\pgfqpoint{1.953286in}{0.891063in}}%
\pgfpathlineto{\pgfqpoint{1.953286in}{0.865868in}}%
\pgfpathlineto{\pgfqpoint{1.992341in}{0.865868in}}%
\pgfpathlineto{\pgfqpoint{1.992341in}{0.846393in}}%
\pgfpathlineto{\pgfqpoint{2.031396in}{0.846393in}}%
\pgfpathlineto{\pgfqpoint{2.031396in}{0.830370in}}%
\pgfpathlineto{\pgfqpoint{2.070452in}{0.830370in}}%
\pgfpathlineto{\pgfqpoint{2.070452in}{0.822590in}}%
\pgfpathlineto{\pgfqpoint{2.109507in}{0.822590in}}%
\pgfpathlineto{\pgfqpoint{2.109507in}{0.817953in}}%
\pgfpathlineto{\pgfqpoint{2.148563in}{0.817953in}}%
\pgfpathlineto{\pgfqpoint{2.148563in}{0.785958in}}%
\pgfpathlineto{\pgfqpoint{2.187618in}{0.785958in}}%
\pgfpathlineto{\pgfqpoint{2.187618in}{0.738815in}}%
\pgfpathlineto{\pgfqpoint{2.226674in}{0.738815in}}%
\pgfpathlineto{\pgfqpoint{2.226674in}{0.699040in}}%
\pgfpathlineto{\pgfqpoint{2.265729in}{0.699040in}}%
\pgfpathlineto{\pgfqpoint{2.265729in}{0.635926in}}%
\pgfpathlineto{\pgfqpoint{2.304784in}{0.635926in}}%
\pgfpathlineto{\pgfqpoint{2.304784in}{0.596151in}}%
\pgfpathlineto{\pgfqpoint{2.343840in}{0.596151in}}%
\pgfpathlineto{\pgfqpoint{2.343840in}{0.539735in}}%
\pgfpathlineto{\pgfqpoint{2.382895in}{0.539735in}}%
\pgfpathlineto{\pgfqpoint{2.382895in}{0.503412in}}%
\pgfpathlineto{\pgfqpoint{2.421951in}{0.503412in}}%
\pgfpathlineto{\pgfqpoint{2.421951in}{0.477032in}}%
\pgfpathlineto{\pgfqpoint{2.461006in}{0.477032in}}%
\pgfpathlineto{\pgfqpoint{2.461006in}{0.460185in}}%
\pgfpathlineto{\pgfqpoint{2.500062in}{0.460185in}}%
\pgfpathlineto{\pgfqpoint{2.500062in}{0.452714in}}%
\pgfpathlineto{\pgfqpoint{2.539117in}{0.452714in}}%
\pgfpathlineto{\pgfqpoint{2.539117in}{0.449983in}}%
\pgfusepath{stroke}%
\end{pgfscope}%
\begin{pgfscope}%
\pgfpathrectangle{\pgfqpoint{0.366840in}{0.449983in}}{\pgfqpoint{2.194810in}{1.169449in}} %
\pgfusepath{clip}%
\pgfsetbuttcap%
\pgfsetmiterjoin%
\pgfsetlinewidth{1.003750pt}%
\definecolor{currentstroke}{rgb}{1.000000,0.647059,0.000000}%
\pgfsetstrokecolor{currentstroke}%
\pgfsetdash{}{0pt}%
\pgfpathmoveto{\pgfqpoint{0.586346in}{0.449983in}}%
\pgfpathlineto{\pgfqpoint{0.586346in}{0.468592in}}%
\pgfpathlineto{\pgfqpoint{0.625401in}{0.468592in}}%
\pgfpathlineto{\pgfqpoint{0.625401in}{0.474622in}}%
\pgfpathlineto{\pgfqpoint{0.664457in}{0.474622in}}%
\pgfpathlineto{\pgfqpoint{0.664457in}{0.476974in}}%
\pgfpathlineto{\pgfqpoint{0.703512in}{0.476974in}}%
\pgfpathlineto{\pgfqpoint{0.703512in}{0.489017in}}%
\pgfpathlineto{\pgfqpoint{0.742567in}{0.489017in}}%
\pgfpathlineto{\pgfqpoint{0.742567in}{0.492971in}}%
\pgfpathlineto{\pgfqpoint{0.781623in}{0.492971in}}%
\pgfpathlineto{\pgfqpoint{0.781623in}{0.506975in}}%
\pgfpathlineto{\pgfqpoint{0.820678in}{0.506975in}}%
\pgfpathlineto{\pgfqpoint{0.820678in}{0.520623in}}%
\pgfpathlineto{\pgfqpoint{0.859734in}{0.520623in}}%
\pgfpathlineto{\pgfqpoint{0.859734in}{0.535181in}}%
\pgfpathlineto{\pgfqpoint{0.898789in}{0.535181in}}%
\pgfpathlineto{\pgfqpoint{0.898789in}{0.555496in}}%
\pgfpathlineto{\pgfqpoint{0.937845in}{0.555496in}}%
\pgfpathlineto{\pgfqpoint{0.937845in}{0.574723in}}%
\pgfpathlineto{\pgfqpoint{0.976900in}{0.574723in}}%
\pgfpathlineto{\pgfqpoint{0.976900in}{0.600066in}}%
\pgfpathlineto{\pgfqpoint{1.015955in}{0.600066in}}%
\pgfpathlineto{\pgfqpoint{1.015955in}{0.634458in}}%
\pgfpathlineto{\pgfqpoint{1.055011in}{0.634458in}}%
\pgfpathlineto{\pgfqpoint{1.055011in}{0.668526in}}%
\pgfpathlineto{\pgfqpoint{1.094066in}{0.668526in}}%
\pgfpathlineto{\pgfqpoint{1.094066in}{0.703724in}}%
\pgfpathlineto{\pgfqpoint{1.133122in}{0.703724in}}%
\pgfpathlineto{\pgfqpoint{1.133122in}{0.755714in}}%
\pgfpathlineto{\pgfqpoint{1.172177in}{0.755714in}}%
\pgfpathlineto{\pgfqpoint{1.172177in}{0.824822in}}%
\pgfpathlineto{\pgfqpoint{1.211233in}{0.824822in}}%
\pgfpathlineto{\pgfqpoint{1.211233in}{0.868737in}}%
\pgfpathlineto{\pgfqpoint{1.250288in}{0.868737in}}%
\pgfpathlineto{\pgfqpoint{1.250288in}{0.958573in}}%
\pgfpathlineto{\pgfqpoint{1.289343in}{0.958573in}}%
\pgfpathlineto{\pgfqpoint{1.289343in}{1.092753in}}%
\pgfpathlineto{\pgfqpoint{1.328399in}{1.092753in}}%
\pgfpathlineto{\pgfqpoint{1.328399in}{1.269218in}}%
\pgfpathlineto{\pgfqpoint{1.367454in}{1.269218in}}%
\pgfpathlineto{\pgfqpoint{1.367454in}{1.373001in}}%
\pgfpathlineto{\pgfqpoint{1.406510in}{1.373001in}}%
\pgfpathlineto{\pgfqpoint{1.406510in}{1.241247in}}%
\pgfpathlineto{\pgfqpoint{1.445565in}{1.241247in}}%
\pgfpathlineto{\pgfqpoint{1.445565in}{1.311847in}}%
\pgfpathlineto{\pgfqpoint{1.484621in}{1.311847in}}%
\pgfpathlineto{\pgfqpoint{1.484621in}{1.419196in}}%
\pgfpathlineto{\pgfqpoint{1.523676in}{1.419196in}}%
\pgfpathlineto{\pgfqpoint{1.523676in}{1.433072in}}%
\pgfpathlineto{\pgfqpoint{1.562731in}{1.433072in}}%
\pgfpathlineto{\pgfqpoint{1.562731in}{1.466663in}}%
\pgfpathlineto{\pgfqpoint{1.601787in}{1.466663in}}%
\pgfpathlineto{\pgfqpoint{1.601787in}{1.522718in}}%
\pgfpathlineto{\pgfqpoint{1.640842in}{1.522718in}}%
\pgfpathlineto{\pgfqpoint{1.640842in}{1.564909in}}%
\pgfpathlineto{\pgfqpoint{1.679898in}{1.564909in}}%
\pgfpathlineto{\pgfqpoint{1.679898in}{1.553210in}}%
\pgfpathlineto{\pgfqpoint{1.718953in}{1.553210in}}%
\pgfpathlineto{\pgfqpoint{1.718953in}{1.490897in}}%
\pgfpathlineto{\pgfqpoint{1.758009in}{1.490897in}}%
\pgfpathlineto{\pgfqpoint{1.758009in}{1.367785in}}%
\pgfpathlineto{\pgfqpoint{1.797064in}{1.367785in}}%
\pgfpathlineto{\pgfqpoint{1.797064in}{1.209767in}}%
\pgfpathlineto{\pgfqpoint{1.836119in}{1.209767in}}%
\pgfpathlineto{\pgfqpoint{1.836119in}{1.069414in}}%
\pgfpathlineto{\pgfqpoint{1.875175in}{1.069414in}}%
\pgfpathlineto{\pgfqpoint{1.875175in}{0.973239in}}%
\pgfpathlineto{\pgfqpoint{1.914230in}{0.973239in}}%
\pgfpathlineto{\pgfqpoint{1.914230in}{0.904292in}}%
\pgfpathlineto{\pgfqpoint{1.953286in}{0.904292in}}%
\pgfpathlineto{\pgfqpoint{1.953286in}{0.874092in}}%
\pgfpathlineto{\pgfqpoint{1.992341in}{0.874092in}}%
\pgfpathlineto{\pgfqpoint{1.992341in}{0.850021in}}%
\pgfpathlineto{\pgfqpoint{2.031396in}{0.850021in}}%
\pgfpathlineto{\pgfqpoint{2.031396in}{0.823306in}}%
\pgfpathlineto{\pgfqpoint{2.070452in}{0.823306in}}%
\pgfpathlineto{\pgfqpoint{2.070452in}{0.816338in}}%
\pgfpathlineto{\pgfqpoint{2.109507in}{0.816338in}}%
\pgfpathlineto{\pgfqpoint{2.109507in}{0.797478in}}%
\pgfpathlineto{\pgfqpoint{2.148563in}{0.797478in}}%
\pgfpathlineto{\pgfqpoint{2.148563in}{0.754849in}}%
\pgfpathlineto{\pgfqpoint{2.187618in}{0.754849in}}%
\pgfpathlineto{\pgfqpoint{2.187618in}{0.685364in}}%
\pgfpathlineto{\pgfqpoint{2.226674in}{0.685364in}}%
\pgfpathlineto{\pgfqpoint{2.226674in}{0.627395in}}%
\pgfpathlineto{\pgfqpoint{2.265729in}{0.627395in}}%
\pgfpathlineto{\pgfqpoint{2.265729in}{0.561331in}}%
\pgfpathlineto{\pgfqpoint{2.304784in}{0.561331in}}%
\pgfpathlineto{\pgfqpoint{2.304784in}{0.528671in}}%
\pgfpathlineto{\pgfqpoint{2.343840in}{0.528671in}}%
\pgfpathlineto{\pgfqpoint{2.343840in}{0.496487in}}%
\pgfpathlineto{\pgfqpoint{2.382895in}{0.496487in}}%
\pgfpathlineto{\pgfqpoint{2.382895in}{0.478796in}}%
\pgfpathlineto{\pgfqpoint{2.421951in}{0.478796in}}%
\pgfpathlineto{\pgfqpoint{2.421951in}{0.463435in}}%
\pgfpathlineto{\pgfqpoint{2.461006in}{0.463435in}}%
\pgfpathlineto{\pgfqpoint{2.461006in}{0.454764in}}%
\pgfpathlineto{\pgfqpoint{2.500062in}{0.454764in}}%
\pgfpathlineto{\pgfqpoint{2.500062in}{0.451280in}}%
\pgfpathlineto{\pgfqpoint{2.539117in}{0.451280in}}%
\pgfpathlineto{\pgfqpoint{2.539117in}{0.449983in}}%
\pgfusepath{stroke}%
\end{pgfscope}%
\begin{pgfscope}%
\pgfsetrectcap%
\pgfsetmiterjoin%
\pgfsetlinewidth{1.003750pt}%
\definecolor{currentstroke}{rgb}{0.000000,0.000000,0.000000}%
\pgfsetstrokecolor{currentstroke}%
\pgfsetdash{}{0pt}%
\pgfpathmoveto{\pgfqpoint{0.366840in}{1.619432in}}%
\pgfpathlineto{\pgfqpoint{2.561650in}{1.619432in}}%
\pgfusepath{stroke}%
\end{pgfscope}%
\begin{pgfscope}%
\pgfsetrectcap%
\pgfsetmiterjoin%
\pgfsetlinewidth{1.003750pt}%
\definecolor{currentstroke}{rgb}{0.000000,0.000000,0.000000}%
\pgfsetstrokecolor{currentstroke}%
\pgfsetdash{}{0pt}%
\pgfpathmoveto{\pgfqpoint{2.561650in}{0.449983in}}%
\pgfpathlineto{\pgfqpoint{2.561650in}{1.619432in}}%
\pgfusepath{stroke}%
\end{pgfscope}%
\begin{pgfscope}%
\pgfsetrectcap%
\pgfsetmiterjoin%
\pgfsetlinewidth{1.003750pt}%
\definecolor{currentstroke}{rgb}{0.000000,0.000000,0.000000}%
\pgfsetstrokecolor{currentstroke}%
\pgfsetdash{}{0pt}%
\pgfpathmoveto{\pgfqpoint{0.366840in}{0.449983in}}%
\pgfpathlineto{\pgfqpoint{2.561650in}{0.449983in}}%
\pgfusepath{stroke}%
\end{pgfscope}%
\begin{pgfscope}%
\pgfsetrectcap%
\pgfsetmiterjoin%
\pgfsetlinewidth{1.003750pt}%
\definecolor{currentstroke}{rgb}{0.000000,0.000000,0.000000}%
\pgfsetstrokecolor{currentstroke}%
\pgfsetdash{}{0pt}%
\pgfpathmoveto{\pgfqpoint{0.366840in}{0.449983in}}%
\pgfpathlineto{\pgfqpoint{0.366840in}{1.619432in}}%
\pgfusepath{stroke}%
\end{pgfscope}%
\begin{pgfscope}%
\pgfsetbuttcap%
\pgfsetroundjoin%
\definecolor{currentfill}{rgb}{0.000000,0.000000,0.000000}%
\pgfsetfillcolor{currentfill}%
\pgfsetlinewidth{0.501875pt}%
\definecolor{currentstroke}{rgb}{0.000000,0.000000,0.000000}%
\pgfsetstrokecolor{currentstroke}%
\pgfsetdash{}{0pt}%
\pgfsys@defobject{currentmarker}{\pgfqpoint{0.000000in}{0.000000in}}{\pgfqpoint{0.000000in}{0.069444in}}{%
\pgfpathmoveto{\pgfqpoint{0.000000in}{0.000000in}}%
\pgfpathlineto{\pgfqpoint{0.000000in}{0.069444in}}%
\pgfusepath{stroke,fill}%
}%
\begin{pgfscope}%
\pgfsys@transformshift{0.366840in}{0.449983in}%
\pgfsys@useobject{currentmarker}{}%
\end{pgfscope}%
\end{pgfscope}%
\begin{pgfscope}%
\pgfsetbuttcap%
\pgfsetroundjoin%
\definecolor{currentfill}{rgb}{0.000000,0.000000,0.000000}%
\pgfsetfillcolor{currentfill}%
\pgfsetlinewidth{0.501875pt}%
\definecolor{currentstroke}{rgb}{0.000000,0.000000,0.000000}%
\pgfsetstrokecolor{currentstroke}%
\pgfsetdash{}{0pt}%
\pgfsys@defobject{currentmarker}{\pgfqpoint{0.000000in}{-0.069444in}}{\pgfqpoint{0.000000in}{0.000000in}}{%
\pgfpathmoveto{\pgfqpoint{0.000000in}{0.000000in}}%
\pgfpathlineto{\pgfqpoint{0.000000in}{-0.069444in}}%
\pgfusepath{stroke,fill}%
}%
\begin{pgfscope}%
\pgfsys@transformshift{0.366840in}{1.619432in}%
\pgfsys@useobject{currentmarker}{}%
\end{pgfscope}%
\end{pgfscope}%
\begin{pgfscope}%
\pgftext[x=0.366840in,y=0.380539in,,top]{\rmfamily\fontsize{8.000000}{9.600000}\selectfont −5}%
\end{pgfscope}%
\begin{pgfscope}%
\pgfsetbuttcap%
\pgfsetroundjoin%
\definecolor{currentfill}{rgb}{0.000000,0.000000,0.000000}%
\pgfsetfillcolor{currentfill}%
\pgfsetlinewidth{0.501875pt}%
\definecolor{currentstroke}{rgb}{0.000000,0.000000,0.000000}%
\pgfsetstrokecolor{currentstroke}%
\pgfsetdash{}{0pt}%
\pgfsys@defobject{currentmarker}{\pgfqpoint{0.000000in}{0.000000in}}{\pgfqpoint{0.000000in}{0.069444in}}{%
\pgfpathmoveto{\pgfqpoint{0.000000in}{0.000000in}}%
\pgfpathlineto{\pgfqpoint{0.000000in}{0.069444in}}%
\pgfusepath{stroke,fill}%
}%
\begin{pgfscope}%
\pgfsys@transformshift{0.915543in}{0.449983in}%
\pgfsys@useobject{currentmarker}{}%
\end{pgfscope}%
\end{pgfscope}%
\begin{pgfscope}%
\pgfsetbuttcap%
\pgfsetroundjoin%
\definecolor{currentfill}{rgb}{0.000000,0.000000,0.000000}%
\pgfsetfillcolor{currentfill}%
\pgfsetlinewidth{0.501875pt}%
\definecolor{currentstroke}{rgb}{0.000000,0.000000,0.000000}%
\pgfsetstrokecolor{currentstroke}%
\pgfsetdash{}{0pt}%
\pgfsys@defobject{currentmarker}{\pgfqpoint{0.000000in}{-0.069444in}}{\pgfqpoint{0.000000in}{0.000000in}}{%
\pgfpathmoveto{\pgfqpoint{0.000000in}{0.000000in}}%
\pgfpathlineto{\pgfqpoint{0.000000in}{-0.069444in}}%
\pgfusepath{stroke,fill}%
}%
\begin{pgfscope}%
\pgfsys@transformshift{0.915543in}{1.619432in}%
\pgfsys@useobject{currentmarker}{}%
\end{pgfscope}%
\end{pgfscope}%
\begin{pgfscope}%
\pgftext[x=0.915543in,y=0.380539in,,top]{\rmfamily\fontsize{8.000000}{9.600000}\selectfont 0}%
\end{pgfscope}%
\begin{pgfscope}%
\pgfsetbuttcap%
\pgfsetroundjoin%
\definecolor{currentfill}{rgb}{0.000000,0.000000,0.000000}%
\pgfsetfillcolor{currentfill}%
\pgfsetlinewidth{0.501875pt}%
\definecolor{currentstroke}{rgb}{0.000000,0.000000,0.000000}%
\pgfsetstrokecolor{currentstroke}%
\pgfsetdash{}{0pt}%
\pgfsys@defobject{currentmarker}{\pgfqpoint{0.000000in}{0.000000in}}{\pgfqpoint{0.000000in}{0.069444in}}{%
\pgfpathmoveto{\pgfqpoint{0.000000in}{0.000000in}}%
\pgfpathlineto{\pgfqpoint{0.000000in}{0.069444in}}%
\pgfusepath{stroke,fill}%
}%
\begin{pgfscope}%
\pgfsys@transformshift{1.464245in}{0.449983in}%
\pgfsys@useobject{currentmarker}{}%
\end{pgfscope}%
\end{pgfscope}%
\begin{pgfscope}%
\pgfsetbuttcap%
\pgfsetroundjoin%
\definecolor{currentfill}{rgb}{0.000000,0.000000,0.000000}%
\pgfsetfillcolor{currentfill}%
\pgfsetlinewidth{0.501875pt}%
\definecolor{currentstroke}{rgb}{0.000000,0.000000,0.000000}%
\pgfsetstrokecolor{currentstroke}%
\pgfsetdash{}{0pt}%
\pgfsys@defobject{currentmarker}{\pgfqpoint{0.000000in}{-0.069444in}}{\pgfqpoint{0.000000in}{0.000000in}}{%
\pgfpathmoveto{\pgfqpoint{0.000000in}{0.000000in}}%
\pgfpathlineto{\pgfqpoint{0.000000in}{-0.069444in}}%
\pgfusepath{stroke,fill}%
}%
\begin{pgfscope}%
\pgfsys@transformshift{1.464245in}{1.619432in}%
\pgfsys@useobject{currentmarker}{}%
\end{pgfscope}%
\end{pgfscope}%
\begin{pgfscope}%
\pgftext[x=1.464245in,y=0.380539in,,top]{\rmfamily\fontsize{8.000000}{9.600000}\selectfont 5}%
\end{pgfscope}%
\begin{pgfscope}%
\pgfsetbuttcap%
\pgfsetroundjoin%
\definecolor{currentfill}{rgb}{0.000000,0.000000,0.000000}%
\pgfsetfillcolor{currentfill}%
\pgfsetlinewidth{0.501875pt}%
\definecolor{currentstroke}{rgb}{0.000000,0.000000,0.000000}%
\pgfsetstrokecolor{currentstroke}%
\pgfsetdash{}{0pt}%
\pgfsys@defobject{currentmarker}{\pgfqpoint{0.000000in}{0.000000in}}{\pgfqpoint{0.000000in}{0.069444in}}{%
\pgfpathmoveto{\pgfqpoint{0.000000in}{0.000000in}}%
\pgfpathlineto{\pgfqpoint{0.000000in}{0.069444in}}%
\pgfusepath{stroke,fill}%
}%
\begin{pgfscope}%
\pgfsys@transformshift{2.012947in}{0.449983in}%
\pgfsys@useobject{currentmarker}{}%
\end{pgfscope}%
\end{pgfscope}%
\begin{pgfscope}%
\pgfsetbuttcap%
\pgfsetroundjoin%
\definecolor{currentfill}{rgb}{0.000000,0.000000,0.000000}%
\pgfsetfillcolor{currentfill}%
\pgfsetlinewidth{0.501875pt}%
\definecolor{currentstroke}{rgb}{0.000000,0.000000,0.000000}%
\pgfsetstrokecolor{currentstroke}%
\pgfsetdash{}{0pt}%
\pgfsys@defobject{currentmarker}{\pgfqpoint{0.000000in}{-0.069444in}}{\pgfqpoint{0.000000in}{0.000000in}}{%
\pgfpathmoveto{\pgfqpoint{0.000000in}{0.000000in}}%
\pgfpathlineto{\pgfqpoint{0.000000in}{-0.069444in}}%
\pgfusepath{stroke,fill}%
}%
\begin{pgfscope}%
\pgfsys@transformshift{2.012947in}{1.619432in}%
\pgfsys@useobject{currentmarker}{}%
\end{pgfscope}%
\end{pgfscope}%
\begin{pgfscope}%
\pgftext[x=2.012947in,y=0.380539in,,top]{\rmfamily\fontsize{8.000000}{9.600000}\selectfont 10}%
\end{pgfscope}%
\begin{pgfscope}%
\pgfsetbuttcap%
\pgfsetroundjoin%
\definecolor{currentfill}{rgb}{0.000000,0.000000,0.000000}%
\pgfsetfillcolor{currentfill}%
\pgfsetlinewidth{0.501875pt}%
\definecolor{currentstroke}{rgb}{0.000000,0.000000,0.000000}%
\pgfsetstrokecolor{currentstroke}%
\pgfsetdash{}{0pt}%
\pgfsys@defobject{currentmarker}{\pgfqpoint{0.000000in}{0.000000in}}{\pgfqpoint{0.000000in}{0.069444in}}{%
\pgfpathmoveto{\pgfqpoint{0.000000in}{0.000000in}}%
\pgfpathlineto{\pgfqpoint{0.000000in}{0.069444in}}%
\pgfusepath{stroke,fill}%
}%
\begin{pgfscope}%
\pgfsys@transformshift{2.561650in}{0.449983in}%
\pgfsys@useobject{currentmarker}{}%
\end{pgfscope}%
\end{pgfscope}%
\begin{pgfscope}%
\pgfsetbuttcap%
\pgfsetroundjoin%
\definecolor{currentfill}{rgb}{0.000000,0.000000,0.000000}%
\pgfsetfillcolor{currentfill}%
\pgfsetlinewidth{0.501875pt}%
\definecolor{currentstroke}{rgb}{0.000000,0.000000,0.000000}%
\pgfsetstrokecolor{currentstroke}%
\pgfsetdash{}{0pt}%
\pgfsys@defobject{currentmarker}{\pgfqpoint{0.000000in}{-0.069444in}}{\pgfqpoint{0.000000in}{0.000000in}}{%
\pgfpathmoveto{\pgfqpoint{0.000000in}{0.000000in}}%
\pgfpathlineto{\pgfqpoint{0.000000in}{-0.069444in}}%
\pgfusepath{stroke,fill}%
}%
\begin{pgfscope}%
\pgfsys@transformshift{2.561650in}{1.619432in}%
\pgfsys@useobject{currentmarker}{}%
\end{pgfscope}%
\end{pgfscope}%
\begin{pgfscope}%
\pgftext[x=2.561650in,y=0.380539in,,top]{\rmfamily\fontsize{8.000000}{9.600000}\selectfont 15}%
\end{pgfscope}%
\begin{pgfscope}%
\pgftext[x=1.464245in,y=0.203564in,,top]{\rmfamily\fontsize{9.000000}{10.800000}\selectfont \(\displaystyle \mathrm{DLL}_{\mu/\pi}(\mu^-)\)}%
\end{pgfscope}%
\begin{pgfscope}%
\pgfsetbuttcap%
\pgfsetroundjoin%
\definecolor{currentfill}{rgb}{0.000000,0.000000,0.000000}%
\pgfsetfillcolor{currentfill}%
\pgfsetlinewidth{0.501875pt}%
\definecolor{currentstroke}{rgb}{0.000000,0.000000,0.000000}%
\pgfsetstrokecolor{currentstroke}%
\pgfsetdash{}{0pt}%
\pgfsys@defobject{currentmarker}{\pgfqpoint{0.000000in}{0.000000in}}{\pgfqpoint{0.069444in}{0.000000in}}{%
\pgfpathmoveto{\pgfqpoint{0.000000in}{0.000000in}}%
\pgfpathlineto{\pgfqpoint{0.069444in}{0.000000in}}%
\pgfusepath{stroke,fill}%
}%
\begin{pgfscope}%
\pgfsys@transformshift{0.366840in}{0.449983in}%
\pgfsys@useobject{currentmarker}{}%
\end{pgfscope}%
\end{pgfscope}%
\begin{pgfscope}%
\pgfsetbuttcap%
\pgfsetroundjoin%
\definecolor{currentfill}{rgb}{0.000000,0.000000,0.000000}%
\pgfsetfillcolor{currentfill}%
\pgfsetlinewidth{0.501875pt}%
\definecolor{currentstroke}{rgb}{0.000000,0.000000,0.000000}%
\pgfsetstrokecolor{currentstroke}%
\pgfsetdash{}{0pt}%
\pgfsys@defobject{currentmarker}{\pgfqpoint{-0.069444in}{0.000000in}}{\pgfqpoint{0.000000in}{0.000000in}}{%
\pgfpathmoveto{\pgfqpoint{0.000000in}{0.000000in}}%
\pgfpathlineto{\pgfqpoint{-0.069444in}{0.000000in}}%
\pgfusepath{stroke,fill}%
}%
\begin{pgfscope}%
\pgfsys@transformshift{2.561650in}{0.449983in}%
\pgfsys@useobject{currentmarker}{}%
\end{pgfscope}%
\end{pgfscope}%
\begin{pgfscope}%
\pgftext[x=0.297396in,y=0.449983in,right,]{\rmfamily\fontsize{8.000000}{9.600000}\selectfont 0.00}%
\end{pgfscope}%
\begin{pgfscope}%
\pgfsetbuttcap%
\pgfsetroundjoin%
\definecolor{currentfill}{rgb}{0.000000,0.000000,0.000000}%
\pgfsetfillcolor{currentfill}%
\pgfsetlinewidth{0.501875pt}%
\definecolor{currentstroke}{rgb}{0.000000,0.000000,0.000000}%
\pgfsetstrokecolor{currentstroke}%
\pgfsetdash{}{0pt}%
\pgfsys@defobject{currentmarker}{\pgfqpoint{0.000000in}{0.000000in}}{\pgfqpoint{0.069444in}{0.000000in}}{%
\pgfpathmoveto{\pgfqpoint{0.000000in}{0.000000in}}%
\pgfpathlineto{\pgfqpoint{0.069444in}{0.000000in}}%
\pgfusepath{stroke,fill}%
}%
\begin{pgfscope}%
\pgfsys@transformshift{0.366840in}{0.596165in}%
\pgfsys@useobject{currentmarker}{}%
\end{pgfscope}%
\end{pgfscope}%
\begin{pgfscope}%
\pgfsetbuttcap%
\pgfsetroundjoin%
\definecolor{currentfill}{rgb}{0.000000,0.000000,0.000000}%
\pgfsetfillcolor{currentfill}%
\pgfsetlinewidth{0.501875pt}%
\definecolor{currentstroke}{rgb}{0.000000,0.000000,0.000000}%
\pgfsetstrokecolor{currentstroke}%
\pgfsetdash{}{0pt}%
\pgfsys@defobject{currentmarker}{\pgfqpoint{-0.069444in}{0.000000in}}{\pgfqpoint{0.000000in}{0.000000in}}{%
\pgfpathmoveto{\pgfqpoint{0.000000in}{0.000000in}}%
\pgfpathlineto{\pgfqpoint{-0.069444in}{0.000000in}}%
\pgfusepath{stroke,fill}%
}%
\begin{pgfscope}%
\pgfsys@transformshift{2.561650in}{0.596165in}%
\pgfsys@useobject{currentmarker}{}%
\end{pgfscope}%
\end{pgfscope}%
\begin{pgfscope}%
\pgftext[x=0.297396in,y=0.596165in,right,]{\rmfamily\fontsize{8.000000}{9.600000}\selectfont 0.02}%
\end{pgfscope}%
\begin{pgfscope}%
\pgfsetbuttcap%
\pgfsetroundjoin%
\definecolor{currentfill}{rgb}{0.000000,0.000000,0.000000}%
\pgfsetfillcolor{currentfill}%
\pgfsetlinewidth{0.501875pt}%
\definecolor{currentstroke}{rgb}{0.000000,0.000000,0.000000}%
\pgfsetstrokecolor{currentstroke}%
\pgfsetdash{}{0pt}%
\pgfsys@defobject{currentmarker}{\pgfqpoint{0.000000in}{0.000000in}}{\pgfqpoint{0.069444in}{0.000000in}}{%
\pgfpathmoveto{\pgfqpoint{0.000000in}{0.000000in}}%
\pgfpathlineto{\pgfqpoint{0.069444in}{0.000000in}}%
\pgfusepath{stroke,fill}%
}%
\begin{pgfscope}%
\pgfsys@transformshift{0.366840in}{0.742346in}%
\pgfsys@useobject{currentmarker}{}%
\end{pgfscope}%
\end{pgfscope}%
\begin{pgfscope}%
\pgfsetbuttcap%
\pgfsetroundjoin%
\definecolor{currentfill}{rgb}{0.000000,0.000000,0.000000}%
\pgfsetfillcolor{currentfill}%
\pgfsetlinewidth{0.501875pt}%
\definecolor{currentstroke}{rgb}{0.000000,0.000000,0.000000}%
\pgfsetstrokecolor{currentstroke}%
\pgfsetdash{}{0pt}%
\pgfsys@defobject{currentmarker}{\pgfqpoint{-0.069444in}{0.000000in}}{\pgfqpoint{0.000000in}{0.000000in}}{%
\pgfpathmoveto{\pgfqpoint{0.000000in}{0.000000in}}%
\pgfpathlineto{\pgfqpoint{-0.069444in}{0.000000in}}%
\pgfusepath{stroke,fill}%
}%
\begin{pgfscope}%
\pgfsys@transformshift{2.561650in}{0.742346in}%
\pgfsys@useobject{currentmarker}{}%
\end{pgfscope}%
\end{pgfscope}%
\begin{pgfscope}%
\pgftext[x=0.297396in,y=0.742346in,right,]{\rmfamily\fontsize{8.000000}{9.600000}\selectfont 0.04}%
\end{pgfscope}%
\begin{pgfscope}%
\pgfsetbuttcap%
\pgfsetroundjoin%
\definecolor{currentfill}{rgb}{0.000000,0.000000,0.000000}%
\pgfsetfillcolor{currentfill}%
\pgfsetlinewidth{0.501875pt}%
\definecolor{currentstroke}{rgb}{0.000000,0.000000,0.000000}%
\pgfsetstrokecolor{currentstroke}%
\pgfsetdash{}{0pt}%
\pgfsys@defobject{currentmarker}{\pgfqpoint{0.000000in}{0.000000in}}{\pgfqpoint{0.069444in}{0.000000in}}{%
\pgfpathmoveto{\pgfqpoint{0.000000in}{0.000000in}}%
\pgfpathlineto{\pgfqpoint{0.069444in}{0.000000in}}%
\pgfusepath{stroke,fill}%
}%
\begin{pgfscope}%
\pgfsys@transformshift{0.366840in}{0.888527in}%
\pgfsys@useobject{currentmarker}{}%
\end{pgfscope}%
\end{pgfscope}%
\begin{pgfscope}%
\pgfsetbuttcap%
\pgfsetroundjoin%
\definecolor{currentfill}{rgb}{0.000000,0.000000,0.000000}%
\pgfsetfillcolor{currentfill}%
\pgfsetlinewidth{0.501875pt}%
\definecolor{currentstroke}{rgb}{0.000000,0.000000,0.000000}%
\pgfsetstrokecolor{currentstroke}%
\pgfsetdash{}{0pt}%
\pgfsys@defobject{currentmarker}{\pgfqpoint{-0.069444in}{0.000000in}}{\pgfqpoint{0.000000in}{0.000000in}}{%
\pgfpathmoveto{\pgfqpoint{0.000000in}{0.000000in}}%
\pgfpathlineto{\pgfqpoint{-0.069444in}{0.000000in}}%
\pgfusepath{stroke,fill}%
}%
\begin{pgfscope}%
\pgfsys@transformshift{2.561650in}{0.888527in}%
\pgfsys@useobject{currentmarker}{}%
\end{pgfscope}%
\end{pgfscope}%
\begin{pgfscope}%
\pgftext[x=0.297396in,y=0.888527in,right,]{\rmfamily\fontsize{8.000000}{9.600000}\selectfont 0.06}%
\end{pgfscope}%
\begin{pgfscope}%
\pgfsetbuttcap%
\pgfsetroundjoin%
\definecolor{currentfill}{rgb}{0.000000,0.000000,0.000000}%
\pgfsetfillcolor{currentfill}%
\pgfsetlinewidth{0.501875pt}%
\definecolor{currentstroke}{rgb}{0.000000,0.000000,0.000000}%
\pgfsetstrokecolor{currentstroke}%
\pgfsetdash{}{0pt}%
\pgfsys@defobject{currentmarker}{\pgfqpoint{0.000000in}{0.000000in}}{\pgfqpoint{0.069444in}{0.000000in}}{%
\pgfpathmoveto{\pgfqpoint{0.000000in}{0.000000in}}%
\pgfpathlineto{\pgfqpoint{0.069444in}{0.000000in}}%
\pgfusepath{stroke,fill}%
}%
\begin{pgfscope}%
\pgfsys@transformshift{0.366840in}{1.034708in}%
\pgfsys@useobject{currentmarker}{}%
\end{pgfscope}%
\end{pgfscope}%
\begin{pgfscope}%
\pgfsetbuttcap%
\pgfsetroundjoin%
\definecolor{currentfill}{rgb}{0.000000,0.000000,0.000000}%
\pgfsetfillcolor{currentfill}%
\pgfsetlinewidth{0.501875pt}%
\definecolor{currentstroke}{rgb}{0.000000,0.000000,0.000000}%
\pgfsetstrokecolor{currentstroke}%
\pgfsetdash{}{0pt}%
\pgfsys@defobject{currentmarker}{\pgfqpoint{-0.069444in}{0.000000in}}{\pgfqpoint{0.000000in}{0.000000in}}{%
\pgfpathmoveto{\pgfqpoint{0.000000in}{0.000000in}}%
\pgfpathlineto{\pgfqpoint{-0.069444in}{0.000000in}}%
\pgfusepath{stroke,fill}%
}%
\begin{pgfscope}%
\pgfsys@transformshift{2.561650in}{1.034708in}%
\pgfsys@useobject{currentmarker}{}%
\end{pgfscope}%
\end{pgfscope}%
\begin{pgfscope}%
\pgftext[x=0.297396in,y=1.034708in,right,]{\rmfamily\fontsize{8.000000}{9.600000}\selectfont 0.08}%
\end{pgfscope}%
\begin{pgfscope}%
\pgfsetbuttcap%
\pgfsetroundjoin%
\definecolor{currentfill}{rgb}{0.000000,0.000000,0.000000}%
\pgfsetfillcolor{currentfill}%
\pgfsetlinewidth{0.501875pt}%
\definecolor{currentstroke}{rgb}{0.000000,0.000000,0.000000}%
\pgfsetstrokecolor{currentstroke}%
\pgfsetdash{}{0pt}%
\pgfsys@defobject{currentmarker}{\pgfqpoint{0.000000in}{0.000000in}}{\pgfqpoint{0.069444in}{0.000000in}}{%
\pgfpathmoveto{\pgfqpoint{0.000000in}{0.000000in}}%
\pgfpathlineto{\pgfqpoint{0.069444in}{0.000000in}}%
\pgfusepath{stroke,fill}%
}%
\begin{pgfscope}%
\pgfsys@transformshift{0.366840in}{1.180889in}%
\pgfsys@useobject{currentmarker}{}%
\end{pgfscope}%
\end{pgfscope}%
\begin{pgfscope}%
\pgfsetbuttcap%
\pgfsetroundjoin%
\definecolor{currentfill}{rgb}{0.000000,0.000000,0.000000}%
\pgfsetfillcolor{currentfill}%
\pgfsetlinewidth{0.501875pt}%
\definecolor{currentstroke}{rgb}{0.000000,0.000000,0.000000}%
\pgfsetstrokecolor{currentstroke}%
\pgfsetdash{}{0pt}%
\pgfsys@defobject{currentmarker}{\pgfqpoint{-0.069444in}{0.000000in}}{\pgfqpoint{0.000000in}{0.000000in}}{%
\pgfpathmoveto{\pgfqpoint{0.000000in}{0.000000in}}%
\pgfpathlineto{\pgfqpoint{-0.069444in}{0.000000in}}%
\pgfusepath{stroke,fill}%
}%
\begin{pgfscope}%
\pgfsys@transformshift{2.561650in}{1.180889in}%
\pgfsys@useobject{currentmarker}{}%
\end{pgfscope}%
\end{pgfscope}%
\begin{pgfscope}%
\pgftext[x=0.297396in,y=1.180889in,right,]{\rmfamily\fontsize{8.000000}{9.600000}\selectfont 0.10}%
\end{pgfscope}%
\begin{pgfscope}%
\pgfsetbuttcap%
\pgfsetroundjoin%
\definecolor{currentfill}{rgb}{0.000000,0.000000,0.000000}%
\pgfsetfillcolor{currentfill}%
\pgfsetlinewidth{0.501875pt}%
\definecolor{currentstroke}{rgb}{0.000000,0.000000,0.000000}%
\pgfsetstrokecolor{currentstroke}%
\pgfsetdash{}{0pt}%
\pgfsys@defobject{currentmarker}{\pgfqpoint{0.000000in}{0.000000in}}{\pgfqpoint{0.069444in}{0.000000in}}{%
\pgfpathmoveto{\pgfqpoint{0.000000in}{0.000000in}}%
\pgfpathlineto{\pgfqpoint{0.069444in}{0.000000in}}%
\pgfusepath{stroke,fill}%
}%
\begin{pgfscope}%
\pgfsys@transformshift{0.366840in}{1.327070in}%
\pgfsys@useobject{currentmarker}{}%
\end{pgfscope}%
\end{pgfscope}%
\begin{pgfscope}%
\pgfsetbuttcap%
\pgfsetroundjoin%
\definecolor{currentfill}{rgb}{0.000000,0.000000,0.000000}%
\pgfsetfillcolor{currentfill}%
\pgfsetlinewidth{0.501875pt}%
\definecolor{currentstroke}{rgb}{0.000000,0.000000,0.000000}%
\pgfsetstrokecolor{currentstroke}%
\pgfsetdash{}{0pt}%
\pgfsys@defobject{currentmarker}{\pgfqpoint{-0.069444in}{0.000000in}}{\pgfqpoint{0.000000in}{0.000000in}}{%
\pgfpathmoveto{\pgfqpoint{0.000000in}{0.000000in}}%
\pgfpathlineto{\pgfqpoint{-0.069444in}{0.000000in}}%
\pgfusepath{stroke,fill}%
}%
\begin{pgfscope}%
\pgfsys@transformshift{2.561650in}{1.327070in}%
\pgfsys@useobject{currentmarker}{}%
\end{pgfscope}%
\end{pgfscope}%
\begin{pgfscope}%
\pgftext[x=0.297396in,y=1.327070in,right,]{\rmfamily\fontsize{8.000000}{9.600000}\selectfont 0.12}%
\end{pgfscope}%
\begin{pgfscope}%
\pgfsetbuttcap%
\pgfsetroundjoin%
\definecolor{currentfill}{rgb}{0.000000,0.000000,0.000000}%
\pgfsetfillcolor{currentfill}%
\pgfsetlinewidth{0.501875pt}%
\definecolor{currentstroke}{rgb}{0.000000,0.000000,0.000000}%
\pgfsetstrokecolor{currentstroke}%
\pgfsetdash{}{0pt}%
\pgfsys@defobject{currentmarker}{\pgfqpoint{0.000000in}{0.000000in}}{\pgfqpoint{0.069444in}{0.000000in}}{%
\pgfpathmoveto{\pgfqpoint{0.000000in}{0.000000in}}%
\pgfpathlineto{\pgfqpoint{0.069444in}{0.000000in}}%
\pgfusepath{stroke,fill}%
}%
\begin{pgfscope}%
\pgfsys@transformshift{0.366840in}{1.473251in}%
\pgfsys@useobject{currentmarker}{}%
\end{pgfscope}%
\end{pgfscope}%
\begin{pgfscope}%
\pgfsetbuttcap%
\pgfsetroundjoin%
\definecolor{currentfill}{rgb}{0.000000,0.000000,0.000000}%
\pgfsetfillcolor{currentfill}%
\pgfsetlinewidth{0.501875pt}%
\definecolor{currentstroke}{rgb}{0.000000,0.000000,0.000000}%
\pgfsetstrokecolor{currentstroke}%
\pgfsetdash{}{0pt}%
\pgfsys@defobject{currentmarker}{\pgfqpoint{-0.069444in}{0.000000in}}{\pgfqpoint{0.000000in}{0.000000in}}{%
\pgfpathmoveto{\pgfqpoint{0.000000in}{0.000000in}}%
\pgfpathlineto{\pgfqpoint{-0.069444in}{0.000000in}}%
\pgfusepath{stroke,fill}%
}%
\begin{pgfscope}%
\pgfsys@transformshift{2.561650in}{1.473251in}%
\pgfsys@useobject{currentmarker}{}%
\end{pgfscope}%
\end{pgfscope}%
\begin{pgfscope}%
\pgftext[x=0.297396in,y=1.473251in,right,]{\rmfamily\fontsize{8.000000}{9.600000}\selectfont 0.14}%
\end{pgfscope}%
\begin{pgfscope}%
\pgfsetbuttcap%
\pgfsetroundjoin%
\definecolor{currentfill}{rgb}{0.000000,0.000000,0.000000}%
\pgfsetfillcolor{currentfill}%
\pgfsetlinewidth{0.501875pt}%
\definecolor{currentstroke}{rgb}{0.000000,0.000000,0.000000}%
\pgfsetstrokecolor{currentstroke}%
\pgfsetdash{}{0pt}%
\pgfsys@defobject{currentmarker}{\pgfqpoint{0.000000in}{0.000000in}}{\pgfqpoint{0.069444in}{0.000000in}}{%
\pgfpathmoveto{\pgfqpoint{0.000000in}{0.000000in}}%
\pgfpathlineto{\pgfqpoint{0.069444in}{0.000000in}}%
\pgfusepath{stroke,fill}%
}%
\begin{pgfscope}%
\pgfsys@transformshift{0.366840in}{1.619432in}%
\pgfsys@useobject{currentmarker}{}%
\end{pgfscope}%
\end{pgfscope}%
\begin{pgfscope}%
\pgfsetbuttcap%
\pgfsetroundjoin%
\definecolor{currentfill}{rgb}{0.000000,0.000000,0.000000}%
\pgfsetfillcolor{currentfill}%
\pgfsetlinewidth{0.501875pt}%
\definecolor{currentstroke}{rgb}{0.000000,0.000000,0.000000}%
\pgfsetstrokecolor{currentstroke}%
\pgfsetdash{}{0pt}%
\pgfsys@defobject{currentmarker}{\pgfqpoint{-0.069444in}{0.000000in}}{\pgfqpoint{0.000000in}{0.000000in}}{%
\pgfpathmoveto{\pgfqpoint{0.000000in}{0.000000in}}%
\pgfpathlineto{\pgfqpoint{-0.069444in}{0.000000in}}%
\pgfusepath{stroke,fill}%
}%
\begin{pgfscope}%
\pgfsys@transformshift{2.561650in}{1.619432in}%
\pgfsys@useobject{currentmarker}{}%
\end{pgfscope}%
\end{pgfscope}%
\begin{pgfscope}%
\pgftext[x=0.297396in,y=1.619432in,right,]{\rmfamily\fontsize{8.000000}{9.600000}\selectfont 0.16}%
\end{pgfscope}%
\end{pgfpicture}%
\makeatother%
\endgroup%

	\end{subfigure}
	\begin{subfigure}[t]{0.49\textwidth}
		\centering
    %\includegraphics[width=\textwidth]{store/variables/DATA_MC_REWEIGHTED_muplus_PIDmu.pdf}
    %% Creator: Matplotlib, PGF backend
%%
%% To include the figure in your LaTeX document, write
%%   \input{<filename>.pgf}
%%
%% Make sure the required packages are loaded in your preamble
%%   \usepackage{pgf}
%%
%% Figures using additional raster images can only be included by \input if
%% they are in the same directory as the main LaTeX file. For loading figures
%% from other directories you can use the `import` package
%%   \usepackage{import}
%% and then include the figures with
%%   \import{<path to file>}{<filename>.pgf}
%%
%% Matplotlib used the following preamble
%%   \usepackage{fontspec}
%%   \setmainfont{DejaVu Serif}
%%   \setsansfont{DejaVu Sans}
%%   \setmonofont{DejaVu Sans Mono}
%%
\begingroup%
\makeatletter%
\begin{pgfpicture}%
\pgfpathrectangle{\pgfpointorigin}{\pgfqpoint{2.682342in}{1.719349in}}%
\pgfusepath{use as bounding box, clip}%
\begin{pgfscope}%
\pgfsetbuttcap%
\pgfsetmiterjoin%
\definecolor{currentfill}{rgb}{1.000000,1.000000,1.000000}%
\pgfsetfillcolor{currentfill}%
\pgfsetlinewidth{0.000000pt}%
\definecolor{currentstroke}{rgb}{1.000000,1.000000,1.000000}%
\pgfsetstrokecolor{currentstroke}%
\pgfsetdash{}{0pt}%
\pgfpathmoveto{\pgfqpoint{0.000000in}{0.000000in}}%
\pgfpathlineto{\pgfqpoint{2.682342in}{0.000000in}}%
\pgfpathlineto{\pgfqpoint{2.682342in}{1.719349in}}%
\pgfpathlineto{\pgfqpoint{0.000000in}{1.719349in}}%
\pgfpathclose%
\pgfusepath{fill}%
\end{pgfscope}%
\begin{pgfscope}%
\pgfsetbuttcap%
\pgfsetmiterjoin%
\definecolor{currentfill}{rgb}{1.000000,1.000000,1.000000}%
\pgfsetfillcolor{currentfill}%
\pgfsetlinewidth{0.000000pt}%
\definecolor{currentstroke}{rgb}{0.000000,0.000000,0.000000}%
\pgfsetstrokecolor{currentstroke}%
\pgfsetstrokeopacity{0.000000}%
\pgfsetdash{}{0pt}%
\pgfpathmoveto{\pgfqpoint{0.366840in}{0.449983in}}%
\pgfpathlineto{\pgfqpoint{2.561650in}{0.449983in}}%
\pgfpathlineto{\pgfqpoint{2.561650in}{1.615583in}}%
\pgfpathlineto{\pgfqpoint{0.366840in}{1.615583in}}%
\pgfpathclose%
\pgfusepath{fill}%
\end{pgfscope}%
\begin{pgfscope}%
\pgfpathrectangle{\pgfqpoint{0.366840in}{0.449983in}}{\pgfqpoint{2.194810in}{1.165600in}} %
\pgfusepath{clip}%
\pgfsetbuttcap%
\pgfsetmiterjoin%
\definecolor{currentfill}{rgb}{0.215686,0.470588,0.749020}%
\pgfsetfillcolor{currentfill}%
\pgfsetlinewidth{0.000000pt}%
\definecolor{currentstroke}{rgb}{0.000000,0.000000,0.000000}%
\pgfsetstrokecolor{currentstroke}%
\pgfsetdash{}{0pt}%
\pgfpathmoveto{\pgfqpoint{0.586376in}{0.449983in}}%
\pgfpathlineto{\pgfqpoint{0.586376in}{0.471459in}}%
\pgfpathlineto{\pgfqpoint{0.625295in}{0.471459in}}%
\pgfpathlineto{\pgfqpoint{0.625295in}{0.472858in}}%
\pgfpathlineto{\pgfqpoint{0.664213in}{0.472858in}}%
\pgfpathlineto{\pgfqpoint{0.664213in}{0.483394in}}%
\pgfpathlineto{\pgfqpoint{0.703132in}{0.483394in}}%
\pgfpathlineto{\pgfqpoint{0.703132in}{0.487096in}}%
\pgfpathlineto{\pgfqpoint{0.742051in}{0.487096in}}%
\pgfpathlineto{\pgfqpoint{0.742051in}{0.497927in}}%
\pgfpathlineto{\pgfqpoint{0.780970in}{0.497927in}}%
\pgfpathlineto{\pgfqpoint{0.780970in}{0.510827in}}%
\pgfpathlineto{\pgfqpoint{0.819888in}{0.510827in}}%
\pgfpathlineto{\pgfqpoint{0.819888in}{0.521530in}}%
\pgfpathlineto{\pgfqpoint{0.858807in}{0.521530in}}%
\pgfpathlineto{\pgfqpoint{0.858807in}{0.535099in}}%
\pgfpathlineto{\pgfqpoint{0.897726in}{0.535099in}}%
\pgfpathlineto{\pgfqpoint{0.897726in}{0.551245in}}%
\pgfpathlineto{\pgfqpoint{0.936644in}{0.551245in}}%
\pgfpathlineto{\pgfqpoint{0.936644in}{0.578576in}}%
\pgfpathlineto{\pgfqpoint{0.975563in}{0.578576in}}%
\pgfpathlineto{\pgfqpoint{0.975563in}{0.599928in}}%
\pgfpathlineto{\pgfqpoint{1.014482in}{0.599928in}}%
\pgfpathlineto{\pgfqpoint{1.014482in}{0.630169in}}%
\pgfpathlineto{\pgfqpoint{1.053401in}{0.630169in}}%
\pgfpathlineto{\pgfqpoint{1.053401in}{0.669929in}}%
\pgfpathlineto{\pgfqpoint{1.092319in}{0.669929in}}%
\pgfpathlineto{\pgfqpoint{1.092319in}{0.705965in}}%
\pgfpathlineto{\pgfqpoint{1.131238in}{0.705965in}}%
\pgfpathlineto{\pgfqpoint{1.131238in}{0.755781in}}%
\pgfpathlineto{\pgfqpoint{1.170157in}{0.755781in}}%
\pgfpathlineto{\pgfqpoint{1.170157in}{0.817247in}}%
\pgfpathlineto{\pgfqpoint{1.209075in}{0.817247in}}%
\pgfpathlineto{\pgfqpoint{1.209075in}{0.884065in}}%
\pgfpathlineto{\pgfqpoint{1.247994in}{0.884065in}}%
\pgfpathlineto{\pgfqpoint{1.247994in}{0.954243in}}%
\pgfpathlineto{\pgfqpoint{1.286913in}{0.954243in}}%
\pgfpathlineto{\pgfqpoint{1.286913in}{1.072746in}}%
\pgfpathlineto{\pgfqpoint{1.325831in}{1.072746in}}%
\pgfpathlineto{\pgfqpoint{1.325831in}{1.202751in}}%
\pgfpathlineto{\pgfqpoint{1.364750in}{1.202751in}}%
\pgfpathlineto{\pgfqpoint{1.364750in}{1.328946in}}%
\pgfpathlineto{\pgfqpoint{1.403669in}{1.328946in}}%
\pgfpathlineto{\pgfqpoint{1.403669in}{1.237280in}}%
\pgfpathlineto{\pgfqpoint{1.442588in}{1.237280in}}%
\pgfpathlineto{\pgfqpoint{1.442588in}{1.313171in}}%
\pgfpathlineto{\pgfqpoint{1.481506in}{1.313171in}}%
\pgfpathlineto{\pgfqpoint{1.481506in}{1.414183in}}%
\pgfpathlineto{\pgfqpoint{1.520425in}{1.414183in}}%
\pgfpathlineto{\pgfqpoint{1.520425in}{1.423289in}}%
\pgfpathlineto{\pgfqpoint{1.559344in}{1.423289in}}%
\pgfpathlineto{\pgfqpoint{1.559344in}{1.471274in}}%
\pgfpathlineto{\pgfqpoint{1.598262in}{1.471274in}}%
\pgfpathlineto{\pgfqpoint{1.598262in}{1.522923in}}%
\pgfpathlineto{\pgfqpoint{1.637181in}{1.522923in}}%
\pgfpathlineto{\pgfqpoint{1.637181in}{1.556066in}}%
\pgfpathlineto{\pgfqpoint{1.676100in}{1.556066in}}%
\pgfpathlineto{\pgfqpoint{1.676100in}{1.542132in}}%
\pgfpathlineto{\pgfqpoint{1.715019in}{1.542132in}}%
\pgfpathlineto{\pgfqpoint{1.715019in}{1.522696in}}%
\pgfpathlineto{\pgfqpoint{1.753937in}{1.522696in}}%
\pgfpathlineto{\pgfqpoint{1.753937in}{1.405420in}}%
\pgfpathlineto{\pgfqpoint{1.792856in}{1.405420in}}%
\pgfpathlineto{\pgfqpoint{1.792856in}{1.263111in}}%
\pgfpathlineto{\pgfqpoint{1.831775in}{1.263111in}}%
\pgfpathlineto{\pgfqpoint{1.831775in}{1.108506in}}%
\pgfpathlineto{\pgfqpoint{1.870693in}{1.108506in}}%
\pgfpathlineto{\pgfqpoint{1.870693in}{0.977022in}}%
\pgfpathlineto{\pgfqpoint{1.909612in}{0.977022in}}%
\pgfpathlineto{\pgfqpoint{1.909612in}{0.919043in}}%
\pgfpathlineto{\pgfqpoint{1.948531in}{0.919043in}}%
\pgfpathlineto{\pgfqpoint{1.948531in}{0.869647in}}%
\pgfpathlineto{\pgfqpoint{1.987449in}{0.869647in}}%
\pgfpathlineto{\pgfqpoint{1.987449in}{0.841328in}}%
\pgfpathlineto{\pgfqpoint{2.026368in}{0.841328in}}%
\pgfpathlineto{\pgfqpoint{2.026368in}{0.820259in}}%
\pgfpathlineto{\pgfqpoint{2.065287in}{0.820259in}}%
\pgfpathlineto{\pgfqpoint{2.065287in}{0.806290in}}%
\pgfpathlineto{\pgfqpoint{2.104206in}{0.806290in}}%
\pgfpathlineto{\pgfqpoint{2.104206in}{0.794412in}}%
\pgfpathlineto{\pgfqpoint{2.143124in}{0.794412in}}%
\pgfpathlineto{\pgfqpoint{2.143124in}{0.752594in}}%
\pgfpathlineto{\pgfqpoint{2.182043in}{0.752594in}}%
\pgfpathlineto{\pgfqpoint{2.182043in}{0.680148in}}%
\pgfpathlineto{\pgfqpoint{2.220962in}{0.680148in}}%
\pgfpathlineto{\pgfqpoint{2.220962in}{0.632460in}}%
\pgfpathlineto{\pgfqpoint{2.259880in}{0.632460in}}%
\pgfpathlineto{\pgfqpoint{2.259880in}{0.561868in}}%
\pgfpathlineto{\pgfqpoint{2.298799in}{0.561868in}}%
\pgfpathlineto{\pgfqpoint{2.298799in}{0.525715in}}%
\pgfpathlineto{\pgfqpoint{2.337718in}{0.525715in}}%
\pgfpathlineto{\pgfqpoint{2.337718in}{0.495907in}}%
\pgfpathlineto{\pgfqpoint{2.376637in}{0.495907in}}%
\pgfpathlineto{\pgfqpoint{2.376637in}{0.480641in}}%
\pgfpathlineto{\pgfqpoint{2.415555in}{0.480641in}}%
\pgfpathlineto{\pgfqpoint{2.415555in}{0.465672in}}%
\pgfpathlineto{\pgfqpoint{2.454474in}{0.465672in}}%
\pgfpathlineto{\pgfqpoint{2.454474in}{0.454640in}}%
\pgfpathlineto{\pgfqpoint{2.493393in}{0.454640in}}%
\pgfpathlineto{\pgfqpoint{2.493393in}{0.451468in}}%
\pgfpathlineto{\pgfqpoint{2.532311in}{0.451468in}}%
\pgfpathlineto{\pgfqpoint{2.532311in}{0.449983in}}%
\pgfpathlineto{\pgfqpoint{2.493393in}{0.449983in}}%
\pgfpathlineto{\pgfqpoint{2.493393in}{0.449983in}}%
\pgfpathlineto{\pgfqpoint{2.454474in}{0.449983in}}%
\pgfpathlineto{\pgfqpoint{2.454474in}{0.449983in}}%
\pgfpathlineto{\pgfqpoint{2.415555in}{0.449983in}}%
\pgfpathlineto{\pgfqpoint{2.415555in}{0.449983in}}%
\pgfpathlineto{\pgfqpoint{2.376637in}{0.449983in}}%
\pgfpathlineto{\pgfqpoint{2.376637in}{0.449983in}}%
\pgfpathlineto{\pgfqpoint{2.337718in}{0.449983in}}%
\pgfpathlineto{\pgfqpoint{2.337718in}{0.449983in}}%
\pgfpathlineto{\pgfqpoint{2.298799in}{0.449983in}}%
\pgfpathlineto{\pgfqpoint{2.298799in}{0.449983in}}%
\pgfpathlineto{\pgfqpoint{2.259880in}{0.449983in}}%
\pgfpathlineto{\pgfqpoint{2.259880in}{0.449983in}}%
\pgfpathlineto{\pgfqpoint{2.220962in}{0.449983in}}%
\pgfpathlineto{\pgfqpoint{2.220962in}{0.449983in}}%
\pgfpathlineto{\pgfqpoint{2.182043in}{0.449983in}}%
\pgfpathlineto{\pgfqpoint{2.182043in}{0.449983in}}%
\pgfpathlineto{\pgfqpoint{2.143124in}{0.449983in}}%
\pgfpathlineto{\pgfqpoint{2.143124in}{0.449983in}}%
\pgfpathlineto{\pgfqpoint{2.104206in}{0.449983in}}%
\pgfpathlineto{\pgfqpoint{2.104206in}{0.449983in}}%
\pgfpathlineto{\pgfqpoint{2.065287in}{0.449983in}}%
\pgfpathlineto{\pgfqpoint{2.065287in}{0.449983in}}%
\pgfpathlineto{\pgfqpoint{2.026368in}{0.449983in}}%
\pgfpathlineto{\pgfqpoint{2.026368in}{0.449983in}}%
\pgfpathlineto{\pgfqpoint{1.987449in}{0.449983in}}%
\pgfpathlineto{\pgfqpoint{1.987449in}{0.449983in}}%
\pgfpathlineto{\pgfqpoint{1.948531in}{0.449983in}}%
\pgfpathlineto{\pgfqpoint{1.948531in}{0.449983in}}%
\pgfpathlineto{\pgfqpoint{1.909612in}{0.449983in}}%
\pgfpathlineto{\pgfqpoint{1.909612in}{0.449983in}}%
\pgfpathlineto{\pgfqpoint{1.870693in}{0.449983in}}%
\pgfpathlineto{\pgfqpoint{1.870693in}{0.449983in}}%
\pgfpathlineto{\pgfqpoint{1.831775in}{0.449983in}}%
\pgfpathlineto{\pgfqpoint{1.831775in}{0.449983in}}%
\pgfpathlineto{\pgfqpoint{1.792856in}{0.449983in}}%
\pgfpathlineto{\pgfqpoint{1.792856in}{0.449983in}}%
\pgfpathlineto{\pgfqpoint{1.753937in}{0.449983in}}%
\pgfpathlineto{\pgfqpoint{1.753937in}{0.449983in}}%
\pgfpathlineto{\pgfqpoint{1.715019in}{0.449983in}}%
\pgfpathlineto{\pgfqpoint{1.715019in}{0.449983in}}%
\pgfpathlineto{\pgfqpoint{1.676100in}{0.449983in}}%
\pgfpathlineto{\pgfqpoint{1.676100in}{0.449983in}}%
\pgfpathlineto{\pgfqpoint{1.637181in}{0.449983in}}%
\pgfpathlineto{\pgfqpoint{1.637181in}{0.449983in}}%
\pgfpathlineto{\pgfqpoint{1.598262in}{0.449983in}}%
\pgfpathlineto{\pgfqpoint{1.598262in}{0.449983in}}%
\pgfpathlineto{\pgfqpoint{1.559344in}{0.449983in}}%
\pgfpathlineto{\pgfqpoint{1.559344in}{0.449983in}}%
\pgfpathlineto{\pgfqpoint{1.520425in}{0.449983in}}%
\pgfpathlineto{\pgfqpoint{1.520425in}{0.449983in}}%
\pgfpathlineto{\pgfqpoint{1.481506in}{0.449983in}}%
\pgfpathlineto{\pgfqpoint{1.481506in}{0.449983in}}%
\pgfpathlineto{\pgfqpoint{1.442588in}{0.449983in}}%
\pgfpathlineto{\pgfqpoint{1.442588in}{0.449983in}}%
\pgfpathlineto{\pgfqpoint{1.403669in}{0.449983in}}%
\pgfpathlineto{\pgfqpoint{1.403669in}{0.449983in}}%
\pgfpathlineto{\pgfqpoint{1.364750in}{0.449983in}}%
\pgfpathlineto{\pgfqpoint{1.364750in}{0.449983in}}%
\pgfpathlineto{\pgfqpoint{1.325831in}{0.449983in}}%
\pgfpathlineto{\pgfqpoint{1.325831in}{0.449983in}}%
\pgfpathlineto{\pgfqpoint{1.286913in}{0.449983in}}%
\pgfpathlineto{\pgfqpoint{1.286913in}{0.449983in}}%
\pgfpathlineto{\pgfqpoint{1.247994in}{0.449983in}}%
\pgfpathlineto{\pgfqpoint{1.247994in}{0.449983in}}%
\pgfpathlineto{\pgfqpoint{1.209075in}{0.449983in}}%
\pgfpathlineto{\pgfqpoint{1.209075in}{0.449983in}}%
\pgfpathlineto{\pgfqpoint{1.170157in}{0.449983in}}%
\pgfpathlineto{\pgfqpoint{1.170157in}{0.449983in}}%
\pgfpathlineto{\pgfqpoint{1.131238in}{0.449983in}}%
\pgfpathlineto{\pgfqpoint{1.131238in}{0.449983in}}%
\pgfpathlineto{\pgfqpoint{1.092319in}{0.449983in}}%
\pgfpathlineto{\pgfqpoint{1.092319in}{0.449983in}}%
\pgfpathlineto{\pgfqpoint{1.053401in}{0.449983in}}%
\pgfpathlineto{\pgfqpoint{1.053401in}{0.449983in}}%
\pgfpathlineto{\pgfqpoint{1.014482in}{0.449983in}}%
\pgfpathlineto{\pgfqpoint{1.014482in}{0.449983in}}%
\pgfpathlineto{\pgfqpoint{0.975563in}{0.449983in}}%
\pgfpathlineto{\pgfqpoint{0.975563in}{0.449983in}}%
\pgfpathlineto{\pgfqpoint{0.936644in}{0.449983in}}%
\pgfpathlineto{\pgfqpoint{0.936644in}{0.449983in}}%
\pgfpathlineto{\pgfqpoint{0.897726in}{0.449983in}}%
\pgfpathlineto{\pgfqpoint{0.897726in}{0.449983in}}%
\pgfpathlineto{\pgfqpoint{0.858807in}{0.449983in}}%
\pgfpathlineto{\pgfqpoint{0.858807in}{0.449983in}}%
\pgfpathlineto{\pgfqpoint{0.819888in}{0.449983in}}%
\pgfpathlineto{\pgfqpoint{0.819888in}{0.449983in}}%
\pgfpathlineto{\pgfqpoint{0.780970in}{0.449983in}}%
\pgfpathlineto{\pgfqpoint{0.780970in}{0.449983in}}%
\pgfpathlineto{\pgfqpoint{0.742051in}{0.449983in}}%
\pgfpathlineto{\pgfqpoint{0.742051in}{0.449983in}}%
\pgfpathlineto{\pgfqpoint{0.703132in}{0.449983in}}%
\pgfpathlineto{\pgfqpoint{0.703132in}{0.449983in}}%
\pgfpathlineto{\pgfqpoint{0.664213in}{0.449983in}}%
\pgfpathlineto{\pgfqpoint{0.664213in}{0.449983in}}%
\pgfpathlineto{\pgfqpoint{0.625295in}{0.449983in}}%
\pgfpathlineto{\pgfqpoint{0.625295in}{0.449983in}}%
\pgfpathlineto{\pgfqpoint{0.586376in}{0.449983in}}%
\pgfusepath{fill}%
\end{pgfscope}%
\begin{pgfscope}%
\pgfpathrectangle{\pgfqpoint{0.366840in}{0.449983in}}{\pgfqpoint{2.194810in}{1.165600in}} %
\pgfusepath{clip}%
\pgfsetbuttcap%
\pgfsetmiterjoin%
\pgfsetlinewidth{0.501875pt}%
\definecolor{currentstroke}{rgb}{1.000000,0.000000,0.000000}%
\pgfsetstrokecolor{currentstroke}%
\pgfsetdash{}{0pt}%
\pgfpathmoveto{\pgfqpoint{0.586376in}{0.449983in}}%
\pgfpathlineto{\pgfqpoint{0.586376in}{0.464877in}}%
\pgfpathlineto{\pgfqpoint{0.625295in}{0.464877in}}%
\pgfpathlineto{\pgfqpoint{0.625295in}{0.471112in}}%
\pgfpathlineto{\pgfqpoint{0.664213in}{0.471112in}}%
\pgfpathlineto{\pgfqpoint{0.664213in}{0.475441in}}%
\pgfpathlineto{\pgfqpoint{0.703132in}{0.475441in}}%
\pgfpathlineto{\pgfqpoint{0.703132in}{0.481058in}}%
\pgfpathlineto{\pgfqpoint{0.742051in}{0.481058in}}%
\pgfpathlineto{\pgfqpoint{0.742051in}{0.488221in}}%
\pgfpathlineto{\pgfqpoint{0.780970in}{0.488221in}}%
\pgfpathlineto{\pgfqpoint{0.780970in}{0.499816in}}%
\pgfpathlineto{\pgfqpoint{0.819888in}{0.499816in}}%
\pgfpathlineto{\pgfqpoint{0.819888in}{0.507907in}}%
\pgfpathlineto{\pgfqpoint{0.858807in}{0.507907in}}%
\pgfpathlineto{\pgfqpoint{0.858807in}{0.526407in}}%
\pgfpathlineto{\pgfqpoint{0.897726in}{0.526407in}}%
\pgfpathlineto{\pgfqpoint{0.897726in}{0.538878in}}%
\pgfpathlineto{\pgfqpoint{0.936644in}{0.538878in}}%
\pgfpathlineto{\pgfqpoint{0.936644in}{0.564387in}}%
\pgfpathlineto{\pgfqpoint{0.975563in}{0.564387in}}%
\pgfpathlineto{\pgfqpoint{0.975563in}{0.586597in}}%
\pgfpathlineto{\pgfqpoint{1.014482in}{0.586597in}}%
\pgfpathlineto{\pgfqpoint{1.014482in}{0.619682in}}%
\pgfpathlineto{\pgfqpoint{1.053401in}{0.619682in}}%
\pgfpathlineto{\pgfqpoint{1.053401in}{0.652972in}}%
\pgfpathlineto{\pgfqpoint{1.092319in}{0.652972in}}%
\pgfpathlineto{\pgfqpoint{1.092319in}{0.690643in}}%
\pgfpathlineto{\pgfqpoint{1.131238in}{0.690643in}}%
\pgfpathlineto{\pgfqpoint{1.131238in}{0.736765in}}%
\pgfpathlineto{\pgfqpoint{1.170157in}{0.736765in}}%
\pgfpathlineto{\pgfqpoint{1.170157in}{0.793245in}}%
\pgfpathlineto{\pgfqpoint{1.209075in}{0.793245in}}%
\pgfpathlineto{\pgfqpoint{1.209075in}{0.859413in}}%
\pgfpathlineto{\pgfqpoint{1.247994in}{0.859413in}}%
\pgfpathlineto{\pgfqpoint{1.247994in}{0.926045in}}%
\pgfpathlineto{\pgfqpoint{1.286913in}{0.926045in}}%
\pgfpathlineto{\pgfqpoint{1.286913in}{1.050394in}}%
\pgfpathlineto{\pgfqpoint{1.325831in}{1.050394in}}%
\pgfpathlineto{\pgfqpoint{1.325831in}{1.198500in}}%
\pgfpathlineto{\pgfqpoint{1.364750in}{1.198500in}}%
\pgfpathlineto{\pgfqpoint{1.364750in}{1.343308in}}%
\pgfpathlineto{\pgfqpoint{1.403669in}{1.343308in}}%
\pgfpathlineto{\pgfqpoint{1.403669in}{1.224061in}}%
\pgfpathlineto{\pgfqpoint{1.442588in}{1.224061in}}%
\pgfpathlineto{\pgfqpoint{1.442588in}{1.293682in}}%
\pgfpathlineto{\pgfqpoint{1.481506in}{1.293682in}}%
\pgfpathlineto{\pgfqpoint{1.481506in}{1.394532in}}%
\pgfpathlineto{\pgfqpoint{1.520425in}{1.394532in}}%
\pgfpathlineto{\pgfqpoint{1.520425in}{1.452919in}}%
\pgfpathlineto{\pgfqpoint{1.559344in}{1.452919in}}%
\pgfpathlineto{\pgfqpoint{1.559344in}{1.477758in}}%
\pgfpathlineto{\pgfqpoint{1.598262in}{1.477758in}}%
\pgfpathlineto{\pgfqpoint{1.598262in}{1.531661in}}%
\pgfpathlineto{\pgfqpoint{1.637181in}{1.531661in}}%
\pgfpathlineto{\pgfqpoint{1.637181in}{1.570002in}}%
\pgfpathlineto{\pgfqpoint{1.676100in}{1.570002in}}%
\pgfpathlineto{\pgfqpoint{1.676100in}{1.545266in}}%
\pgfpathlineto{\pgfqpoint{1.715019in}{1.545266in}}%
\pgfpathlineto{\pgfqpoint{1.715019in}{1.478891in}}%
\pgfpathlineto{\pgfqpoint{1.753937in}{1.478891in}}%
\pgfpathlineto{\pgfqpoint{1.753937in}{1.374846in}}%
\pgfpathlineto{\pgfqpoint{1.792856in}{1.374846in}}%
\pgfpathlineto{\pgfqpoint{1.792856in}{1.207003in}}%
\pgfpathlineto{\pgfqpoint{1.831775in}{1.207003in}}%
\pgfpathlineto{\pgfqpoint{1.831775in}{1.059413in}}%
\pgfpathlineto{\pgfqpoint{1.870693in}{1.059413in}}%
\pgfpathlineto{\pgfqpoint{1.870693in}{0.955213in}}%
\pgfpathlineto{\pgfqpoint{1.909612in}{0.955213in}}%
\pgfpathlineto{\pgfqpoint{1.909612in}{0.915636in}}%
\pgfpathlineto{\pgfqpoint{1.948531in}{0.915636in}}%
\pgfpathlineto{\pgfqpoint{1.948531in}{0.867040in}}%
\pgfpathlineto{\pgfqpoint{1.987449in}{0.867040in}}%
\pgfpathlineto{\pgfqpoint{1.987449in}{0.836945in}}%
\pgfpathlineto{\pgfqpoint{2.026368in}{0.836945in}}%
\pgfpathlineto{\pgfqpoint{2.026368in}{0.836275in}}%
\pgfpathlineto{\pgfqpoint{2.065287in}{0.836275in}}%
\pgfpathlineto{\pgfqpoint{2.065287in}{0.812003in}}%
\pgfpathlineto{\pgfqpoint{2.104206in}{0.812003in}}%
\pgfpathlineto{\pgfqpoint{2.104206in}{0.809014in}}%
\pgfpathlineto{\pgfqpoint{2.143124in}{0.809014in}}%
\pgfpathlineto{\pgfqpoint{2.143124in}{0.783454in}}%
\pgfpathlineto{\pgfqpoint{2.182043in}{0.783454in}}%
\pgfpathlineto{\pgfqpoint{2.182043in}{0.740475in}}%
\pgfpathlineto{\pgfqpoint{2.220962in}{0.740475in}}%
\pgfpathlineto{\pgfqpoint{2.220962in}{0.707391in}}%
\pgfpathlineto{\pgfqpoint{2.259880in}{0.707391in}}%
\pgfpathlineto{\pgfqpoint{2.259880in}{0.646324in}}%
\pgfpathlineto{\pgfqpoint{2.298799in}{0.646324in}}%
\pgfpathlineto{\pgfqpoint{2.298799in}{0.597626in}}%
\pgfpathlineto{\pgfqpoint{2.337718in}{0.597626in}}%
\pgfpathlineto{\pgfqpoint{2.337718in}{0.543464in}}%
\pgfpathlineto{\pgfqpoint{2.376637in}{0.543464in}}%
\pgfpathlineto{\pgfqpoint{2.376637in}{0.512596in}}%
\pgfpathlineto{\pgfqpoint{2.415555in}{0.512596in}}%
\pgfpathlineto{\pgfqpoint{2.415555in}{0.477966in}}%
\pgfpathlineto{\pgfqpoint{2.454474in}{0.477966in}}%
\pgfpathlineto{\pgfqpoint{2.454474in}{0.459878in}}%
\pgfpathlineto{\pgfqpoint{2.493393in}{0.459878in}}%
\pgfpathlineto{\pgfqpoint{2.493393in}{0.453952in}}%
\pgfpathlineto{\pgfqpoint{2.532311in}{0.453952in}}%
\pgfpathlineto{\pgfqpoint{2.532311in}{0.449983in}}%
\pgfusepath{stroke}%
\end{pgfscope}%
\begin{pgfscope}%
\pgfpathrectangle{\pgfqpoint{0.366840in}{0.449983in}}{\pgfqpoint{2.194810in}{1.165600in}} %
\pgfusepath{clip}%
\pgfsetbuttcap%
\pgfsetmiterjoin%
\pgfsetlinewidth{1.003750pt}%
\definecolor{currentstroke}{rgb}{1.000000,0.647059,0.000000}%
\pgfsetstrokecolor{currentstroke}%
\pgfsetdash{}{0pt}%
\pgfpathmoveto{\pgfqpoint{0.586376in}{0.449983in}}%
\pgfpathlineto{\pgfqpoint{0.586376in}{0.467647in}}%
\pgfpathlineto{\pgfqpoint{0.625295in}{0.467647in}}%
\pgfpathlineto{\pgfqpoint{0.625295in}{0.475426in}}%
\pgfpathlineto{\pgfqpoint{0.664213in}{0.475426in}}%
\pgfpathlineto{\pgfqpoint{0.664213in}{0.480517in}}%
\pgfpathlineto{\pgfqpoint{0.703132in}{0.480517in}}%
\pgfpathlineto{\pgfqpoint{0.703132in}{0.486520in}}%
\pgfpathlineto{\pgfqpoint{0.742051in}{0.486520in}}%
\pgfpathlineto{\pgfqpoint{0.742051in}{0.494136in}}%
\pgfpathlineto{\pgfqpoint{0.780970in}{0.494136in}}%
\pgfpathlineto{\pgfqpoint{0.780970in}{0.509836in}}%
\pgfpathlineto{\pgfqpoint{0.819888in}{0.509836in}}%
\pgfpathlineto{\pgfqpoint{0.819888in}{0.516396in}}%
\pgfpathlineto{\pgfqpoint{0.858807in}{0.516396in}}%
\pgfpathlineto{\pgfqpoint{0.858807in}{0.535467in}}%
\pgfpathlineto{\pgfqpoint{0.897726in}{0.535467in}}%
\pgfpathlineto{\pgfqpoint{0.897726in}{0.548821in}}%
\pgfpathlineto{\pgfqpoint{0.936644in}{0.548821in}}%
\pgfpathlineto{\pgfqpoint{0.936644in}{0.576423in}}%
\pgfpathlineto{\pgfqpoint{0.975563in}{0.576423in}}%
\pgfpathlineto{\pgfqpoint{0.975563in}{0.597644in}}%
\pgfpathlineto{\pgfqpoint{1.014482in}{0.597644in}}%
\pgfpathlineto{\pgfqpoint{1.014482in}{0.633669in}}%
\pgfpathlineto{\pgfqpoint{1.053401in}{0.633669in}}%
\pgfpathlineto{\pgfqpoint{1.053401in}{0.668659in}}%
\pgfpathlineto{\pgfqpoint{1.092319in}{0.668659in}}%
\pgfpathlineto{\pgfqpoint{1.092319in}{0.707298in}}%
\pgfpathlineto{\pgfqpoint{1.131238in}{0.707298in}}%
\pgfpathlineto{\pgfqpoint{1.131238in}{0.755445in}}%
\pgfpathlineto{\pgfqpoint{1.170157in}{0.755445in}}%
\pgfpathlineto{\pgfqpoint{1.170157in}{0.811155in}}%
\pgfpathlineto{\pgfqpoint{1.209075in}{0.811155in}}%
\pgfpathlineto{\pgfqpoint{1.209075in}{0.875587in}}%
\pgfpathlineto{\pgfqpoint{1.247994in}{0.875587in}}%
\pgfpathlineto{\pgfqpoint{1.247994in}{0.950488in}}%
\pgfpathlineto{\pgfqpoint{1.286913in}{0.950488in}}%
\pgfpathlineto{\pgfqpoint{1.286913in}{1.080970in}}%
\pgfpathlineto{\pgfqpoint{1.325831in}{1.080970in}}%
\pgfpathlineto{\pgfqpoint{1.325831in}{1.229111in}}%
\pgfpathlineto{\pgfqpoint{1.364750in}{1.229111in}}%
\pgfpathlineto{\pgfqpoint{1.364750in}{1.384790in}}%
\pgfpathlineto{\pgfqpoint{1.403669in}{1.384790in}}%
\pgfpathlineto{\pgfqpoint{1.403669in}{1.238410in}}%
\pgfpathlineto{\pgfqpoint{1.442588in}{1.238410in}}%
\pgfpathlineto{\pgfqpoint{1.442588in}{1.300509in}}%
\pgfpathlineto{\pgfqpoint{1.481506in}{1.300509in}}%
\pgfpathlineto{\pgfqpoint{1.481506in}{1.401482in}}%
\pgfpathlineto{\pgfqpoint{1.520425in}{1.401482in}}%
\pgfpathlineto{\pgfqpoint{1.520425in}{1.453471in}}%
\pgfpathlineto{\pgfqpoint{1.559344in}{1.453471in}}%
\pgfpathlineto{\pgfqpoint{1.559344in}{1.471661in}}%
\pgfpathlineto{\pgfqpoint{1.598262in}{1.471661in}}%
\pgfpathlineto{\pgfqpoint{1.598262in}{1.523806in}}%
\pgfpathlineto{\pgfqpoint{1.637181in}{1.523806in}}%
\pgfpathlineto{\pgfqpoint{1.637181in}{1.557059in}}%
\pgfpathlineto{\pgfqpoint{1.676100in}{1.557059in}}%
\pgfpathlineto{\pgfqpoint{1.676100in}{1.538950in}}%
\pgfpathlineto{\pgfqpoint{1.715019in}{1.538950in}}%
\pgfpathlineto{\pgfqpoint{1.715019in}{1.499363in}}%
\pgfpathlineto{\pgfqpoint{1.753937in}{1.499363in}}%
\pgfpathlineto{\pgfqpoint{1.753937in}{1.389296in}}%
\pgfpathlineto{\pgfqpoint{1.792856in}{1.389296in}}%
\pgfpathlineto{\pgfqpoint{1.792856in}{1.217474in}}%
\pgfpathlineto{\pgfqpoint{1.831775in}{1.217474in}}%
\pgfpathlineto{\pgfqpoint{1.831775in}{1.075837in}}%
\pgfpathlineto{\pgfqpoint{1.870693in}{1.075837in}}%
\pgfpathlineto{\pgfqpoint{1.870693in}{0.967034in}}%
\pgfpathlineto{\pgfqpoint{1.909612in}{0.967034in}}%
\pgfpathlineto{\pgfqpoint{1.909612in}{0.922533in}}%
\pgfpathlineto{\pgfqpoint{1.948531in}{0.922533in}}%
\pgfpathlineto{\pgfqpoint{1.948531in}{0.875357in}}%
\pgfpathlineto{\pgfqpoint{1.987449in}{0.875357in}}%
\pgfpathlineto{\pgfqpoint{1.987449in}{0.840275in}}%
\pgfpathlineto{\pgfqpoint{2.026368in}{0.840275in}}%
\pgfpathlineto{\pgfqpoint{2.026368in}{0.832094in}}%
\pgfpathlineto{\pgfqpoint{2.065287in}{0.832094in}}%
\pgfpathlineto{\pgfqpoint{2.065287in}{0.804767in}}%
\pgfpathlineto{\pgfqpoint{2.104206in}{0.804767in}}%
\pgfpathlineto{\pgfqpoint{2.104206in}{0.797194in}}%
\pgfpathlineto{\pgfqpoint{2.143124in}{0.797194in}}%
\pgfpathlineto{\pgfqpoint{2.143124in}{0.760408in}}%
\pgfpathlineto{\pgfqpoint{2.182043in}{0.760408in}}%
\pgfpathlineto{\pgfqpoint{2.182043in}{0.687925in}}%
\pgfpathlineto{\pgfqpoint{2.220962in}{0.687925in}}%
\pgfpathlineto{\pgfqpoint{2.220962in}{0.637834in}}%
\pgfpathlineto{\pgfqpoint{2.259880in}{0.637834in}}%
\pgfpathlineto{\pgfqpoint{2.259880in}{0.570003in}}%
\pgfpathlineto{\pgfqpoint{2.298799in}{0.570003in}}%
\pgfpathlineto{\pgfqpoint{2.298799in}{0.533895in}}%
\pgfpathlineto{\pgfqpoint{2.337718in}{0.533895in}}%
\pgfpathlineto{\pgfqpoint{2.337718in}{0.501692in}}%
\pgfpathlineto{\pgfqpoint{2.376637in}{0.501692in}}%
\pgfpathlineto{\pgfqpoint{2.376637in}{0.484450in}}%
\pgfpathlineto{\pgfqpoint{2.415555in}{0.484450in}}%
\pgfpathlineto{\pgfqpoint{2.415555in}{0.464854in}}%
\pgfpathlineto{\pgfqpoint{2.454474in}{0.464854in}}%
\pgfpathlineto{\pgfqpoint{2.454474in}{0.455190in}}%
\pgfpathlineto{\pgfqpoint{2.493393in}{0.455190in}}%
\pgfpathlineto{\pgfqpoint{2.493393in}{0.452118in}}%
\pgfpathlineto{\pgfqpoint{2.532311in}{0.452118in}}%
\pgfpathlineto{\pgfqpoint{2.532311in}{0.449983in}}%
\pgfusepath{stroke}%
\end{pgfscope}%
\begin{pgfscope}%
\pgfsetrectcap%
\pgfsetmiterjoin%
\pgfsetlinewidth{1.003750pt}%
\definecolor{currentstroke}{rgb}{0.000000,0.000000,0.000000}%
\pgfsetstrokecolor{currentstroke}%
\pgfsetdash{}{0pt}%
\pgfpathmoveto{\pgfqpoint{0.366840in}{1.615583in}}%
\pgfpathlineto{\pgfqpoint{2.561650in}{1.615583in}}%
\pgfusepath{stroke}%
\end{pgfscope}%
\begin{pgfscope}%
\pgfsetrectcap%
\pgfsetmiterjoin%
\pgfsetlinewidth{1.003750pt}%
\definecolor{currentstroke}{rgb}{0.000000,0.000000,0.000000}%
\pgfsetstrokecolor{currentstroke}%
\pgfsetdash{}{0pt}%
\pgfpathmoveto{\pgfqpoint{2.561650in}{0.449983in}}%
\pgfpathlineto{\pgfqpoint{2.561650in}{1.615583in}}%
\pgfusepath{stroke}%
\end{pgfscope}%
\begin{pgfscope}%
\pgfsetrectcap%
\pgfsetmiterjoin%
\pgfsetlinewidth{1.003750pt}%
\definecolor{currentstroke}{rgb}{0.000000,0.000000,0.000000}%
\pgfsetstrokecolor{currentstroke}%
\pgfsetdash{}{0pt}%
\pgfpathmoveto{\pgfqpoint{0.366840in}{0.449983in}}%
\pgfpathlineto{\pgfqpoint{2.561650in}{0.449983in}}%
\pgfusepath{stroke}%
\end{pgfscope}%
\begin{pgfscope}%
\pgfsetrectcap%
\pgfsetmiterjoin%
\pgfsetlinewidth{1.003750pt}%
\definecolor{currentstroke}{rgb}{0.000000,0.000000,0.000000}%
\pgfsetstrokecolor{currentstroke}%
\pgfsetdash{}{0pt}%
\pgfpathmoveto{\pgfqpoint{0.366840in}{0.449983in}}%
\pgfpathlineto{\pgfqpoint{0.366840in}{1.615583in}}%
\pgfusepath{stroke}%
\end{pgfscope}%
\begin{pgfscope}%
\pgfsetbuttcap%
\pgfsetroundjoin%
\definecolor{currentfill}{rgb}{0.000000,0.000000,0.000000}%
\pgfsetfillcolor{currentfill}%
\pgfsetlinewidth{0.501875pt}%
\definecolor{currentstroke}{rgb}{0.000000,0.000000,0.000000}%
\pgfsetstrokecolor{currentstroke}%
\pgfsetdash{}{0pt}%
\pgfsys@defobject{currentmarker}{\pgfqpoint{0.000000in}{0.000000in}}{\pgfqpoint{0.000000in}{0.069444in}}{%
\pgfpathmoveto{\pgfqpoint{0.000000in}{0.000000in}}%
\pgfpathlineto{\pgfqpoint{0.000000in}{0.069444in}}%
\pgfusepath{stroke,fill}%
}%
\begin{pgfscope}%
\pgfsys@transformshift{0.366840in}{0.449983in}%
\pgfsys@useobject{currentmarker}{}%
\end{pgfscope}%
\end{pgfscope}%
\begin{pgfscope}%
\pgfsetbuttcap%
\pgfsetroundjoin%
\definecolor{currentfill}{rgb}{0.000000,0.000000,0.000000}%
\pgfsetfillcolor{currentfill}%
\pgfsetlinewidth{0.501875pt}%
\definecolor{currentstroke}{rgb}{0.000000,0.000000,0.000000}%
\pgfsetstrokecolor{currentstroke}%
\pgfsetdash{}{0pt}%
\pgfsys@defobject{currentmarker}{\pgfqpoint{0.000000in}{-0.069444in}}{\pgfqpoint{0.000000in}{0.000000in}}{%
\pgfpathmoveto{\pgfqpoint{0.000000in}{0.000000in}}%
\pgfpathlineto{\pgfqpoint{0.000000in}{-0.069444in}}%
\pgfusepath{stroke,fill}%
}%
\begin{pgfscope}%
\pgfsys@transformshift{0.366840in}{1.615583in}%
\pgfsys@useobject{currentmarker}{}%
\end{pgfscope}%
\end{pgfscope}%
\begin{pgfscope}%
\pgftext[x=0.366840in,y=0.380539in,,top]{\rmfamily\fontsize{8.000000}{9.600000}\selectfont −5}%
\end{pgfscope}%
\begin{pgfscope}%
\pgfsetbuttcap%
\pgfsetroundjoin%
\definecolor{currentfill}{rgb}{0.000000,0.000000,0.000000}%
\pgfsetfillcolor{currentfill}%
\pgfsetlinewidth{0.501875pt}%
\definecolor{currentstroke}{rgb}{0.000000,0.000000,0.000000}%
\pgfsetstrokecolor{currentstroke}%
\pgfsetdash{}{0pt}%
\pgfsys@defobject{currentmarker}{\pgfqpoint{0.000000in}{0.000000in}}{\pgfqpoint{0.000000in}{0.069444in}}{%
\pgfpathmoveto{\pgfqpoint{0.000000in}{0.000000in}}%
\pgfpathlineto{\pgfqpoint{0.000000in}{0.069444in}}%
\pgfusepath{stroke,fill}%
}%
\begin{pgfscope}%
\pgfsys@transformshift{0.915543in}{0.449983in}%
\pgfsys@useobject{currentmarker}{}%
\end{pgfscope}%
\end{pgfscope}%
\begin{pgfscope}%
\pgfsetbuttcap%
\pgfsetroundjoin%
\definecolor{currentfill}{rgb}{0.000000,0.000000,0.000000}%
\pgfsetfillcolor{currentfill}%
\pgfsetlinewidth{0.501875pt}%
\definecolor{currentstroke}{rgb}{0.000000,0.000000,0.000000}%
\pgfsetstrokecolor{currentstroke}%
\pgfsetdash{}{0pt}%
\pgfsys@defobject{currentmarker}{\pgfqpoint{0.000000in}{-0.069444in}}{\pgfqpoint{0.000000in}{0.000000in}}{%
\pgfpathmoveto{\pgfqpoint{0.000000in}{0.000000in}}%
\pgfpathlineto{\pgfqpoint{0.000000in}{-0.069444in}}%
\pgfusepath{stroke,fill}%
}%
\begin{pgfscope}%
\pgfsys@transformshift{0.915543in}{1.615583in}%
\pgfsys@useobject{currentmarker}{}%
\end{pgfscope}%
\end{pgfscope}%
\begin{pgfscope}%
\pgftext[x=0.915543in,y=0.380539in,,top]{\rmfamily\fontsize{8.000000}{9.600000}\selectfont 0}%
\end{pgfscope}%
\begin{pgfscope}%
\pgfsetbuttcap%
\pgfsetroundjoin%
\definecolor{currentfill}{rgb}{0.000000,0.000000,0.000000}%
\pgfsetfillcolor{currentfill}%
\pgfsetlinewidth{0.501875pt}%
\definecolor{currentstroke}{rgb}{0.000000,0.000000,0.000000}%
\pgfsetstrokecolor{currentstroke}%
\pgfsetdash{}{0pt}%
\pgfsys@defobject{currentmarker}{\pgfqpoint{0.000000in}{0.000000in}}{\pgfqpoint{0.000000in}{0.069444in}}{%
\pgfpathmoveto{\pgfqpoint{0.000000in}{0.000000in}}%
\pgfpathlineto{\pgfqpoint{0.000000in}{0.069444in}}%
\pgfusepath{stroke,fill}%
}%
\begin{pgfscope}%
\pgfsys@transformshift{1.464245in}{0.449983in}%
\pgfsys@useobject{currentmarker}{}%
\end{pgfscope}%
\end{pgfscope}%
\begin{pgfscope}%
\pgfsetbuttcap%
\pgfsetroundjoin%
\definecolor{currentfill}{rgb}{0.000000,0.000000,0.000000}%
\pgfsetfillcolor{currentfill}%
\pgfsetlinewidth{0.501875pt}%
\definecolor{currentstroke}{rgb}{0.000000,0.000000,0.000000}%
\pgfsetstrokecolor{currentstroke}%
\pgfsetdash{}{0pt}%
\pgfsys@defobject{currentmarker}{\pgfqpoint{0.000000in}{-0.069444in}}{\pgfqpoint{0.000000in}{0.000000in}}{%
\pgfpathmoveto{\pgfqpoint{0.000000in}{0.000000in}}%
\pgfpathlineto{\pgfqpoint{0.000000in}{-0.069444in}}%
\pgfusepath{stroke,fill}%
}%
\begin{pgfscope}%
\pgfsys@transformshift{1.464245in}{1.615583in}%
\pgfsys@useobject{currentmarker}{}%
\end{pgfscope}%
\end{pgfscope}%
\begin{pgfscope}%
\pgftext[x=1.464245in,y=0.380539in,,top]{\rmfamily\fontsize{8.000000}{9.600000}\selectfont 5}%
\end{pgfscope}%
\begin{pgfscope}%
\pgfsetbuttcap%
\pgfsetroundjoin%
\definecolor{currentfill}{rgb}{0.000000,0.000000,0.000000}%
\pgfsetfillcolor{currentfill}%
\pgfsetlinewidth{0.501875pt}%
\definecolor{currentstroke}{rgb}{0.000000,0.000000,0.000000}%
\pgfsetstrokecolor{currentstroke}%
\pgfsetdash{}{0pt}%
\pgfsys@defobject{currentmarker}{\pgfqpoint{0.000000in}{0.000000in}}{\pgfqpoint{0.000000in}{0.069444in}}{%
\pgfpathmoveto{\pgfqpoint{0.000000in}{0.000000in}}%
\pgfpathlineto{\pgfqpoint{0.000000in}{0.069444in}}%
\pgfusepath{stroke,fill}%
}%
\begin{pgfscope}%
\pgfsys@transformshift{2.012947in}{0.449983in}%
\pgfsys@useobject{currentmarker}{}%
\end{pgfscope}%
\end{pgfscope}%
\begin{pgfscope}%
\pgfsetbuttcap%
\pgfsetroundjoin%
\definecolor{currentfill}{rgb}{0.000000,0.000000,0.000000}%
\pgfsetfillcolor{currentfill}%
\pgfsetlinewidth{0.501875pt}%
\definecolor{currentstroke}{rgb}{0.000000,0.000000,0.000000}%
\pgfsetstrokecolor{currentstroke}%
\pgfsetdash{}{0pt}%
\pgfsys@defobject{currentmarker}{\pgfqpoint{0.000000in}{-0.069444in}}{\pgfqpoint{0.000000in}{0.000000in}}{%
\pgfpathmoveto{\pgfqpoint{0.000000in}{0.000000in}}%
\pgfpathlineto{\pgfqpoint{0.000000in}{-0.069444in}}%
\pgfusepath{stroke,fill}%
}%
\begin{pgfscope}%
\pgfsys@transformshift{2.012947in}{1.615583in}%
\pgfsys@useobject{currentmarker}{}%
\end{pgfscope}%
\end{pgfscope}%
\begin{pgfscope}%
\pgftext[x=2.012947in,y=0.380539in,,top]{\rmfamily\fontsize{8.000000}{9.600000}\selectfont 10}%
\end{pgfscope}%
\begin{pgfscope}%
\pgfsetbuttcap%
\pgfsetroundjoin%
\definecolor{currentfill}{rgb}{0.000000,0.000000,0.000000}%
\pgfsetfillcolor{currentfill}%
\pgfsetlinewidth{0.501875pt}%
\definecolor{currentstroke}{rgb}{0.000000,0.000000,0.000000}%
\pgfsetstrokecolor{currentstroke}%
\pgfsetdash{}{0pt}%
\pgfsys@defobject{currentmarker}{\pgfqpoint{0.000000in}{0.000000in}}{\pgfqpoint{0.000000in}{0.069444in}}{%
\pgfpathmoveto{\pgfqpoint{0.000000in}{0.000000in}}%
\pgfpathlineto{\pgfqpoint{0.000000in}{0.069444in}}%
\pgfusepath{stroke,fill}%
}%
\begin{pgfscope}%
\pgfsys@transformshift{2.561650in}{0.449983in}%
\pgfsys@useobject{currentmarker}{}%
\end{pgfscope}%
\end{pgfscope}%
\begin{pgfscope}%
\pgfsetbuttcap%
\pgfsetroundjoin%
\definecolor{currentfill}{rgb}{0.000000,0.000000,0.000000}%
\pgfsetfillcolor{currentfill}%
\pgfsetlinewidth{0.501875pt}%
\definecolor{currentstroke}{rgb}{0.000000,0.000000,0.000000}%
\pgfsetstrokecolor{currentstroke}%
\pgfsetdash{}{0pt}%
\pgfsys@defobject{currentmarker}{\pgfqpoint{0.000000in}{-0.069444in}}{\pgfqpoint{0.000000in}{0.000000in}}{%
\pgfpathmoveto{\pgfqpoint{0.000000in}{0.000000in}}%
\pgfpathlineto{\pgfqpoint{0.000000in}{-0.069444in}}%
\pgfusepath{stroke,fill}%
}%
\begin{pgfscope}%
\pgfsys@transformshift{2.561650in}{1.615583in}%
\pgfsys@useobject{currentmarker}{}%
\end{pgfscope}%
\end{pgfscope}%
\begin{pgfscope}%
\pgftext[x=2.561650in,y=0.380539in,,top]{\rmfamily\fontsize{8.000000}{9.600000}\selectfont 15}%
\end{pgfscope}%
\begin{pgfscope}%
\pgftext[x=1.464245in,y=0.203564in,,top]{\rmfamily\fontsize{9.000000}{10.800000}\selectfont \(\displaystyle \mathrm{DLL}_{\mu/\pi}(\mu^+)\)}%
\end{pgfscope}%
\begin{pgfscope}%
\pgfsetbuttcap%
\pgfsetroundjoin%
\definecolor{currentfill}{rgb}{0.000000,0.000000,0.000000}%
\pgfsetfillcolor{currentfill}%
\pgfsetlinewidth{0.501875pt}%
\definecolor{currentstroke}{rgb}{0.000000,0.000000,0.000000}%
\pgfsetstrokecolor{currentstroke}%
\pgfsetdash{}{0pt}%
\pgfsys@defobject{currentmarker}{\pgfqpoint{0.000000in}{0.000000in}}{\pgfqpoint{0.069444in}{0.000000in}}{%
\pgfpathmoveto{\pgfqpoint{0.000000in}{0.000000in}}%
\pgfpathlineto{\pgfqpoint{0.069444in}{0.000000in}}%
\pgfusepath{stroke,fill}%
}%
\begin{pgfscope}%
\pgfsys@transformshift{0.366840in}{0.449983in}%
\pgfsys@useobject{currentmarker}{}%
\end{pgfscope}%
\end{pgfscope}%
\begin{pgfscope}%
\pgfsetbuttcap%
\pgfsetroundjoin%
\definecolor{currentfill}{rgb}{0.000000,0.000000,0.000000}%
\pgfsetfillcolor{currentfill}%
\pgfsetlinewidth{0.501875pt}%
\definecolor{currentstroke}{rgb}{0.000000,0.000000,0.000000}%
\pgfsetstrokecolor{currentstroke}%
\pgfsetdash{}{0pt}%
\pgfsys@defobject{currentmarker}{\pgfqpoint{-0.069444in}{0.000000in}}{\pgfqpoint{0.000000in}{0.000000in}}{%
\pgfpathmoveto{\pgfqpoint{0.000000in}{0.000000in}}%
\pgfpathlineto{\pgfqpoint{-0.069444in}{0.000000in}}%
\pgfusepath{stroke,fill}%
}%
\begin{pgfscope}%
\pgfsys@transformshift{2.561650in}{0.449983in}%
\pgfsys@useobject{currentmarker}{}%
\end{pgfscope}%
\end{pgfscope}%
\begin{pgfscope}%
\pgftext[x=0.297396in,y=0.449983in,right,]{\rmfamily\fontsize{8.000000}{9.600000}\selectfont 0.00}%
\end{pgfscope}%
\begin{pgfscope}%
\pgfsetbuttcap%
\pgfsetroundjoin%
\definecolor{currentfill}{rgb}{0.000000,0.000000,0.000000}%
\pgfsetfillcolor{currentfill}%
\pgfsetlinewidth{0.501875pt}%
\definecolor{currentstroke}{rgb}{0.000000,0.000000,0.000000}%
\pgfsetstrokecolor{currentstroke}%
\pgfsetdash{}{0pt}%
\pgfsys@defobject{currentmarker}{\pgfqpoint{0.000000in}{0.000000in}}{\pgfqpoint{0.069444in}{0.000000in}}{%
\pgfpathmoveto{\pgfqpoint{0.000000in}{0.000000in}}%
\pgfpathlineto{\pgfqpoint{0.069444in}{0.000000in}}%
\pgfusepath{stroke,fill}%
}%
\begin{pgfscope}%
\pgfsys@transformshift{0.366840in}{0.595683in}%
\pgfsys@useobject{currentmarker}{}%
\end{pgfscope}%
\end{pgfscope}%
\begin{pgfscope}%
\pgfsetbuttcap%
\pgfsetroundjoin%
\definecolor{currentfill}{rgb}{0.000000,0.000000,0.000000}%
\pgfsetfillcolor{currentfill}%
\pgfsetlinewidth{0.501875pt}%
\definecolor{currentstroke}{rgb}{0.000000,0.000000,0.000000}%
\pgfsetstrokecolor{currentstroke}%
\pgfsetdash{}{0pt}%
\pgfsys@defobject{currentmarker}{\pgfqpoint{-0.069444in}{0.000000in}}{\pgfqpoint{0.000000in}{0.000000in}}{%
\pgfpathmoveto{\pgfqpoint{0.000000in}{0.000000in}}%
\pgfpathlineto{\pgfqpoint{-0.069444in}{0.000000in}}%
\pgfusepath{stroke,fill}%
}%
\begin{pgfscope}%
\pgfsys@transformshift{2.561650in}{0.595683in}%
\pgfsys@useobject{currentmarker}{}%
\end{pgfscope}%
\end{pgfscope}%
\begin{pgfscope}%
\pgftext[x=0.297396in,y=0.595683in,right,]{\rmfamily\fontsize{8.000000}{9.600000}\selectfont 0.02}%
\end{pgfscope}%
\begin{pgfscope}%
\pgfsetbuttcap%
\pgfsetroundjoin%
\definecolor{currentfill}{rgb}{0.000000,0.000000,0.000000}%
\pgfsetfillcolor{currentfill}%
\pgfsetlinewidth{0.501875pt}%
\definecolor{currentstroke}{rgb}{0.000000,0.000000,0.000000}%
\pgfsetstrokecolor{currentstroke}%
\pgfsetdash{}{0pt}%
\pgfsys@defobject{currentmarker}{\pgfqpoint{0.000000in}{0.000000in}}{\pgfqpoint{0.069444in}{0.000000in}}{%
\pgfpathmoveto{\pgfqpoint{0.000000in}{0.000000in}}%
\pgfpathlineto{\pgfqpoint{0.069444in}{0.000000in}}%
\pgfusepath{stroke,fill}%
}%
\begin{pgfscope}%
\pgfsys@transformshift{0.366840in}{0.741383in}%
\pgfsys@useobject{currentmarker}{}%
\end{pgfscope}%
\end{pgfscope}%
\begin{pgfscope}%
\pgfsetbuttcap%
\pgfsetroundjoin%
\definecolor{currentfill}{rgb}{0.000000,0.000000,0.000000}%
\pgfsetfillcolor{currentfill}%
\pgfsetlinewidth{0.501875pt}%
\definecolor{currentstroke}{rgb}{0.000000,0.000000,0.000000}%
\pgfsetstrokecolor{currentstroke}%
\pgfsetdash{}{0pt}%
\pgfsys@defobject{currentmarker}{\pgfqpoint{-0.069444in}{0.000000in}}{\pgfqpoint{0.000000in}{0.000000in}}{%
\pgfpathmoveto{\pgfqpoint{0.000000in}{0.000000in}}%
\pgfpathlineto{\pgfqpoint{-0.069444in}{0.000000in}}%
\pgfusepath{stroke,fill}%
}%
\begin{pgfscope}%
\pgfsys@transformshift{2.561650in}{0.741383in}%
\pgfsys@useobject{currentmarker}{}%
\end{pgfscope}%
\end{pgfscope}%
\begin{pgfscope}%
\pgftext[x=0.297396in,y=0.741383in,right,]{\rmfamily\fontsize{8.000000}{9.600000}\selectfont 0.04}%
\end{pgfscope}%
\begin{pgfscope}%
\pgfsetbuttcap%
\pgfsetroundjoin%
\definecolor{currentfill}{rgb}{0.000000,0.000000,0.000000}%
\pgfsetfillcolor{currentfill}%
\pgfsetlinewidth{0.501875pt}%
\definecolor{currentstroke}{rgb}{0.000000,0.000000,0.000000}%
\pgfsetstrokecolor{currentstroke}%
\pgfsetdash{}{0pt}%
\pgfsys@defobject{currentmarker}{\pgfqpoint{0.000000in}{0.000000in}}{\pgfqpoint{0.069444in}{0.000000in}}{%
\pgfpathmoveto{\pgfqpoint{0.000000in}{0.000000in}}%
\pgfpathlineto{\pgfqpoint{0.069444in}{0.000000in}}%
\pgfusepath{stroke,fill}%
}%
\begin{pgfscope}%
\pgfsys@transformshift{0.366840in}{0.887083in}%
\pgfsys@useobject{currentmarker}{}%
\end{pgfscope}%
\end{pgfscope}%
\begin{pgfscope}%
\pgfsetbuttcap%
\pgfsetroundjoin%
\definecolor{currentfill}{rgb}{0.000000,0.000000,0.000000}%
\pgfsetfillcolor{currentfill}%
\pgfsetlinewidth{0.501875pt}%
\definecolor{currentstroke}{rgb}{0.000000,0.000000,0.000000}%
\pgfsetstrokecolor{currentstroke}%
\pgfsetdash{}{0pt}%
\pgfsys@defobject{currentmarker}{\pgfqpoint{-0.069444in}{0.000000in}}{\pgfqpoint{0.000000in}{0.000000in}}{%
\pgfpathmoveto{\pgfqpoint{0.000000in}{0.000000in}}%
\pgfpathlineto{\pgfqpoint{-0.069444in}{0.000000in}}%
\pgfusepath{stroke,fill}%
}%
\begin{pgfscope}%
\pgfsys@transformshift{2.561650in}{0.887083in}%
\pgfsys@useobject{currentmarker}{}%
\end{pgfscope}%
\end{pgfscope}%
\begin{pgfscope}%
\pgftext[x=0.297396in,y=0.887083in,right,]{\rmfamily\fontsize{8.000000}{9.600000}\selectfont 0.06}%
\end{pgfscope}%
\begin{pgfscope}%
\pgfsetbuttcap%
\pgfsetroundjoin%
\definecolor{currentfill}{rgb}{0.000000,0.000000,0.000000}%
\pgfsetfillcolor{currentfill}%
\pgfsetlinewidth{0.501875pt}%
\definecolor{currentstroke}{rgb}{0.000000,0.000000,0.000000}%
\pgfsetstrokecolor{currentstroke}%
\pgfsetdash{}{0pt}%
\pgfsys@defobject{currentmarker}{\pgfqpoint{0.000000in}{0.000000in}}{\pgfqpoint{0.069444in}{0.000000in}}{%
\pgfpathmoveto{\pgfqpoint{0.000000in}{0.000000in}}%
\pgfpathlineto{\pgfqpoint{0.069444in}{0.000000in}}%
\pgfusepath{stroke,fill}%
}%
\begin{pgfscope}%
\pgfsys@transformshift{0.366840in}{1.032783in}%
\pgfsys@useobject{currentmarker}{}%
\end{pgfscope}%
\end{pgfscope}%
\begin{pgfscope}%
\pgfsetbuttcap%
\pgfsetroundjoin%
\definecolor{currentfill}{rgb}{0.000000,0.000000,0.000000}%
\pgfsetfillcolor{currentfill}%
\pgfsetlinewidth{0.501875pt}%
\definecolor{currentstroke}{rgb}{0.000000,0.000000,0.000000}%
\pgfsetstrokecolor{currentstroke}%
\pgfsetdash{}{0pt}%
\pgfsys@defobject{currentmarker}{\pgfqpoint{-0.069444in}{0.000000in}}{\pgfqpoint{0.000000in}{0.000000in}}{%
\pgfpathmoveto{\pgfqpoint{0.000000in}{0.000000in}}%
\pgfpathlineto{\pgfqpoint{-0.069444in}{0.000000in}}%
\pgfusepath{stroke,fill}%
}%
\begin{pgfscope}%
\pgfsys@transformshift{2.561650in}{1.032783in}%
\pgfsys@useobject{currentmarker}{}%
\end{pgfscope}%
\end{pgfscope}%
\begin{pgfscope}%
\pgftext[x=0.297396in,y=1.032783in,right,]{\rmfamily\fontsize{8.000000}{9.600000}\selectfont 0.08}%
\end{pgfscope}%
\begin{pgfscope}%
\pgfsetbuttcap%
\pgfsetroundjoin%
\definecolor{currentfill}{rgb}{0.000000,0.000000,0.000000}%
\pgfsetfillcolor{currentfill}%
\pgfsetlinewidth{0.501875pt}%
\definecolor{currentstroke}{rgb}{0.000000,0.000000,0.000000}%
\pgfsetstrokecolor{currentstroke}%
\pgfsetdash{}{0pt}%
\pgfsys@defobject{currentmarker}{\pgfqpoint{0.000000in}{0.000000in}}{\pgfqpoint{0.069444in}{0.000000in}}{%
\pgfpathmoveto{\pgfqpoint{0.000000in}{0.000000in}}%
\pgfpathlineto{\pgfqpoint{0.069444in}{0.000000in}}%
\pgfusepath{stroke,fill}%
}%
\begin{pgfscope}%
\pgfsys@transformshift{0.366840in}{1.178483in}%
\pgfsys@useobject{currentmarker}{}%
\end{pgfscope}%
\end{pgfscope}%
\begin{pgfscope}%
\pgfsetbuttcap%
\pgfsetroundjoin%
\definecolor{currentfill}{rgb}{0.000000,0.000000,0.000000}%
\pgfsetfillcolor{currentfill}%
\pgfsetlinewidth{0.501875pt}%
\definecolor{currentstroke}{rgb}{0.000000,0.000000,0.000000}%
\pgfsetstrokecolor{currentstroke}%
\pgfsetdash{}{0pt}%
\pgfsys@defobject{currentmarker}{\pgfqpoint{-0.069444in}{0.000000in}}{\pgfqpoint{0.000000in}{0.000000in}}{%
\pgfpathmoveto{\pgfqpoint{0.000000in}{0.000000in}}%
\pgfpathlineto{\pgfqpoint{-0.069444in}{0.000000in}}%
\pgfusepath{stroke,fill}%
}%
\begin{pgfscope}%
\pgfsys@transformshift{2.561650in}{1.178483in}%
\pgfsys@useobject{currentmarker}{}%
\end{pgfscope}%
\end{pgfscope}%
\begin{pgfscope}%
\pgftext[x=0.297396in,y=1.178483in,right,]{\rmfamily\fontsize{8.000000}{9.600000}\selectfont 0.10}%
\end{pgfscope}%
\begin{pgfscope}%
\pgfsetbuttcap%
\pgfsetroundjoin%
\definecolor{currentfill}{rgb}{0.000000,0.000000,0.000000}%
\pgfsetfillcolor{currentfill}%
\pgfsetlinewidth{0.501875pt}%
\definecolor{currentstroke}{rgb}{0.000000,0.000000,0.000000}%
\pgfsetstrokecolor{currentstroke}%
\pgfsetdash{}{0pt}%
\pgfsys@defobject{currentmarker}{\pgfqpoint{0.000000in}{0.000000in}}{\pgfqpoint{0.069444in}{0.000000in}}{%
\pgfpathmoveto{\pgfqpoint{0.000000in}{0.000000in}}%
\pgfpathlineto{\pgfqpoint{0.069444in}{0.000000in}}%
\pgfusepath{stroke,fill}%
}%
\begin{pgfscope}%
\pgfsys@transformshift{0.366840in}{1.324183in}%
\pgfsys@useobject{currentmarker}{}%
\end{pgfscope}%
\end{pgfscope}%
\begin{pgfscope}%
\pgfsetbuttcap%
\pgfsetroundjoin%
\definecolor{currentfill}{rgb}{0.000000,0.000000,0.000000}%
\pgfsetfillcolor{currentfill}%
\pgfsetlinewidth{0.501875pt}%
\definecolor{currentstroke}{rgb}{0.000000,0.000000,0.000000}%
\pgfsetstrokecolor{currentstroke}%
\pgfsetdash{}{0pt}%
\pgfsys@defobject{currentmarker}{\pgfqpoint{-0.069444in}{0.000000in}}{\pgfqpoint{0.000000in}{0.000000in}}{%
\pgfpathmoveto{\pgfqpoint{0.000000in}{0.000000in}}%
\pgfpathlineto{\pgfqpoint{-0.069444in}{0.000000in}}%
\pgfusepath{stroke,fill}%
}%
\begin{pgfscope}%
\pgfsys@transformshift{2.561650in}{1.324183in}%
\pgfsys@useobject{currentmarker}{}%
\end{pgfscope}%
\end{pgfscope}%
\begin{pgfscope}%
\pgftext[x=0.297396in,y=1.324183in,right,]{\rmfamily\fontsize{8.000000}{9.600000}\selectfont 0.12}%
\end{pgfscope}%
\begin{pgfscope}%
\pgfsetbuttcap%
\pgfsetroundjoin%
\definecolor{currentfill}{rgb}{0.000000,0.000000,0.000000}%
\pgfsetfillcolor{currentfill}%
\pgfsetlinewidth{0.501875pt}%
\definecolor{currentstroke}{rgb}{0.000000,0.000000,0.000000}%
\pgfsetstrokecolor{currentstroke}%
\pgfsetdash{}{0pt}%
\pgfsys@defobject{currentmarker}{\pgfqpoint{0.000000in}{0.000000in}}{\pgfqpoint{0.069444in}{0.000000in}}{%
\pgfpathmoveto{\pgfqpoint{0.000000in}{0.000000in}}%
\pgfpathlineto{\pgfqpoint{0.069444in}{0.000000in}}%
\pgfusepath{stroke,fill}%
}%
\begin{pgfscope}%
\pgfsys@transformshift{0.366840in}{1.469883in}%
\pgfsys@useobject{currentmarker}{}%
\end{pgfscope}%
\end{pgfscope}%
\begin{pgfscope}%
\pgfsetbuttcap%
\pgfsetroundjoin%
\definecolor{currentfill}{rgb}{0.000000,0.000000,0.000000}%
\pgfsetfillcolor{currentfill}%
\pgfsetlinewidth{0.501875pt}%
\definecolor{currentstroke}{rgb}{0.000000,0.000000,0.000000}%
\pgfsetstrokecolor{currentstroke}%
\pgfsetdash{}{0pt}%
\pgfsys@defobject{currentmarker}{\pgfqpoint{-0.069444in}{0.000000in}}{\pgfqpoint{0.000000in}{0.000000in}}{%
\pgfpathmoveto{\pgfqpoint{0.000000in}{0.000000in}}%
\pgfpathlineto{\pgfqpoint{-0.069444in}{0.000000in}}%
\pgfusepath{stroke,fill}%
}%
\begin{pgfscope}%
\pgfsys@transformshift{2.561650in}{1.469883in}%
\pgfsys@useobject{currentmarker}{}%
\end{pgfscope}%
\end{pgfscope}%
\begin{pgfscope}%
\pgftext[x=0.297396in,y=1.469883in,right,]{\rmfamily\fontsize{8.000000}{9.600000}\selectfont 0.14}%
\end{pgfscope}%
\begin{pgfscope}%
\pgfsetbuttcap%
\pgfsetroundjoin%
\definecolor{currentfill}{rgb}{0.000000,0.000000,0.000000}%
\pgfsetfillcolor{currentfill}%
\pgfsetlinewidth{0.501875pt}%
\definecolor{currentstroke}{rgb}{0.000000,0.000000,0.000000}%
\pgfsetstrokecolor{currentstroke}%
\pgfsetdash{}{0pt}%
\pgfsys@defobject{currentmarker}{\pgfqpoint{0.000000in}{0.000000in}}{\pgfqpoint{0.069444in}{0.000000in}}{%
\pgfpathmoveto{\pgfqpoint{0.000000in}{0.000000in}}%
\pgfpathlineto{\pgfqpoint{0.069444in}{0.000000in}}%
\pgfusepath{stroke,fill}%
}%
\begin{pgfscope}%
\pgfsys@transformshift{0.366840in}{1.615583in}%
\pgfsys@useobject{currentmarker}{}%
\end{pgfscope}%
\end{pgfscope}%
\begin{pgfscope}%
\pgfsetbuttcap%
\pgfsetroundjoin%
\definecolor{currentfill}{rgb}{0.000000,0.000000,0.000000}%
\pgfsetfillcolor{currentfill}%
\pgfsetlinewidth{0.501875pt}%
\definecolor{currentstroke}{rgb}{0.000000,0.000000,0.000000}%
\pgfsetstrokecolor{currentstroke}%
\pgfsetdash{}{0pt}%
\pgfsys@defobject{currentmarker}{\pgfqpoint{-0.069444in}{0.000000in}}{\pgfqpoint{0.000000in}{0.000000in}}{%
\pgfpathmoveto{\pgfqpoint{0.000000in}{0.000000in}}%
\pgfpathlineto{\pgfqpoint{-0.069444in}{0.000000in}}%
\pgfusepath{stroke,fill}%
}%
\begin{pgfscope}%
\pgfsys@transformshift{2.561650in}{1.615583in}%
\pgfsys@useobject{currentmarker}{}%
\end{pgfscope}%
\end{pgfscope}%
\begin{pgfscope}%
\pgftext[x=0.297396in,y=1.615583in,right,]{\rmfamily\fontsize{8.000000}{9.600000}\selectfont 0.16}%
\end{pgfscope}%
\end{pgfpicture}%
\makeatother%
\endgroup%

	\end{subfigure}

	\caption{
    Input variables after the full, classifier-based reweighting (orange).
    The original simulated dataset (red) and data distribution (blue) are  given as a reference.
  }
  \thisfloatpagestyle{empty}
  \label{fig:mcfeaturesreweighted}
\end{figure}


\chapter{Determination of expected limit}

In order to estimate the number of observed signal decays, signal and background contributions to the reconstructed $B^0$ and $D^0$ masses are modelled statistically (see sections \ref{signalmodel} and \ref{backgroundmodel}).
The parameters of the statistical model, including the \emph{signal yield}, can then be inferred through a \emph{Maximum Likelihood Estimate} (see section \ref{mle}).
The model also allows for the construction of an upper confidence limit on the signal yield.
\emph{Nuisance parameters} are treated using the Profile Likelihood ratio method.
By calculating the normalization constant $\alpha$ (section \ref{normalization}), a measured signal yield can be converted to a branching fraction for the signal decay.
By repeatedly filling the blinded signal window with toy simulated candidates (using a background-only hypothesis) and calculating a limit on the signal branching fraction, an expected limit is produced.

\section{Maximum Likelihood estimation}
\label{mle}

In order to extract an estimate of the number of signal events from the dataset, a \gls{MLE} is performed.
The \gls{MLE} is a method to determine the parameters $θ$ of a probability distribution $p(x | θ)$, given the data $x$.
It is given by
\begin{equation}
  θ^* = \mathrm{argmax}\ \mathcal{L}(θ | x)
  \label{eq:mle}
\end{equation}
where $\mathcal{L}(θ | x)$ is the \textit{Likelihood}, which can be obtained by fixing the data $x$ in the probability distribution $p(x | θ)$ and varying the parameters $θ$.

If we are dealing with $N$ identically distributed and uncorrelated events $\vec{x}$, we can express the Likelihood $\mathcal{L}$ as
\begin{equation}
  \mathcal{L}(θ | \vec{x}) = \prod_i^N p(x_i | θ)\:.
\end{equation}

In high energy physics, statistical models often have to discriminate between different categories of events (\eg \textit{signal} or \textit{background}).
This can be realized as a \textit{mixture model} \begin{equation}
  \mathcal{L}(θ, \vec{f} | \vec{x}) = \prod_i^N \sum_j^K f_j p_j(x_i | θ)\:,
\end{equation}
where $K$ is the number of categories and $f_j$ is the weight of the $j$th category with
\begin{equation}
  \sum_j f_j = 1\:.
\end{equation}
The events originating from mixture category $j$ are distributed according to $p_j(x | θ)$.

From this, estimates $f_j^*$ for the mixture weights can be inferred via \eqref{eq:mle}.
In this case, we are not directly interested in estimating the $f_j^*$ , but rather in the number of events in a certain category $N_j$, \eg the number of signal events.
Such an estimate $N^*_j$ can in principle be obtained through
\begin{equation}
  N^*_j = f^*_j N\:,
\end{equation}
where $N$ has to be treated as the result of a Poisson experiment.

The Poissonian fluctuation of $N_\text{total}$ can also be directly included in the model through
\begin{equation}
  \mathcal{L}(θ, \vec{f}, n | \vec{x}, N) = \frac{n^N\mathrm{e}^{-n}}{N!}  \prod_i^N \sum_j^K f_j p_j(x_i | θ)\:,
\end{equation}
where $n$ is the expected value of $N$.

By performing the variable transformation $n f_j \to N_j$ and neglecting the constant factor $\frac{1}{N!}$, we obtain
\begin{equation}
  \mathcal{L}(θ, N_j | \vec{x}, N) = \mathrm{e}^{-\sum_j N_j}  \prod_i^N \sum_j^K N_j p_j(x_i | θ)\:.
\end{equation}
This model, which is referred to as the \textit{Extended Likelihood}\cite{Lyons1986}, allows for a direct estimate of $N_j$.

\section{Signal model}
\label{signalmodel}

The signal model consists of a two-dimensional probability distribution over $m(\PKplus\Ppiminus\APmuon\Pmuon)$ (corresponding to the \PBzero mass, referred to as $m_B$) and $m(\PKplus\Ppiminus)$ (corresponding to the \APDzero mass, referred to as $m_D$).
Describing the \APDzero mass in addition to the \PBzero mass allows one to separate background contributions containing resonant $\PKplus\Ppiminus$ pairs (peaking in the \APDzero mass) and non-resonant $\PKplus\Ppiminus$ pairs (flat in the \APDzero mass), both of which exist as part of the combinatorial background.

For both dimensions, a \emph{Double Gaussian} model
\begin{equation}
  p(x|\mu,\sigma_1,\sigma_2 f) = f\,\mathup{Normal}(x|\mu,\sigma_1) + (1-f) \mathup{Normal}(x|\mu, \sigma_2)
\end{equation}
is used.
Here, the normal distribution is defined by
\begin{equation}
  \mathup{Normal}(x|\mu,\sigma) = \frac{1}{\sigma \sqrt{2\pi}} \mathup{exp}\left(\frac{-(x-\mu)^2}{2\sigma^2}\right)\:.
\end{equation}
The use of two normal distributions with different widths accomodates for the fact that the resolution of the reconstructed mass differs on a per-event basis.

The total model for both dimensions is obtained as
\begin{equation}
  p(m_B, m_D) = p(m_B) \times p(m_D)\:.
\end{equation}

Given the low expected signal yield in the data sample, the parameters of the signal model cannot be reliably determined from data and are thus estimated from a fit to the simulated $\PBzero\to\APDzero\APmuon\Pmuon$ data sample.
Plots of the fitted model in both dimensions are given in figure \ref{fig:mcfitb}.
The estimated parameters are listed in table \ref{tab:mcfit}.

\begin{figure}
  \centering
  \begin{subfigure}[t]{0.49\textwidth}
    %% Creator: Matplotlib, PGF backend
%%
%% To include the figure in your LaTeX document, write
%%   \input{<filename>.pgf}
%%
%% Make sure the required packages are loaded in your preamble
%%   \usepackage{pgf}
%%
%% Figures using additional raster images can only be included by \input if
%% they are in the same directory as the main LaTeX file. For loading figures
%% from other directories you can use the `import` package
%%   \usepackage{import}
%% and then include the figures with
%%   \import{<path to file>}{<filename>.pgf}
%%
%% Matplotlib used the following preamble
%%   \usepackage{fontspec}
%%   \setmainfont{DejaVu Serif}
%%   \setsansfont{DejaVu Sans}
%%   \setmonofont{DejaVu Sans Mono}
%%
\begingroup%
\makeatletter%
\begin{pgfpicture}%
\pgfpathrectangle{\pgfpointorigin}{\pgfqpoint{2.619584in}{1.941634in}}%
\pgfusepath{use as bounding box, clip}%
\begin{pgfscope}%
\pgfsetbuttcap%
\pgfsetmiterjoin%
\definecolor{currentfill}{rgb}{1.000000,1.000000,1.000000}%
\pgfsetfillcolor{currentfill}%
\pgfsetlinewidth{0.000000pt}%
\definecolor{currentstroke}{rgb}{1.000000,1.000000,1.000000}%
\pgfsetstrokecolor{currentstroke}%
\pgfsetdash{}{0pt}%
\pgfpathmoveto{\pgfqpoint{-0.000000in}{-0.000000in}}%
\pgfpathlineto{\pgfqpoint{2.619584in}{-0.000000in}}%
\pgfpathlineto{\pgfqpoint{2.619584in}{1.941634in}}%
\pgfpathlineto{\pgfqpoint{-0.000000in}{1.941634in}}%
\pgfpathclose%
\pgfusepath{fill}%
\end{pgfscope}%
\begin{pgfscope}%
\pgfsetbuttcap%
\pgfsetmiterjoin%
\definecolor{currentfill}{rgb}{1.000000,1.000000,1.000000}%
\pgfsetfillcolor{currentfill}%
\pgfsetlinewidth{0.000000pt}%
\definecolor{currentstroke}{rgb}{0.000000,0.000000,0.000000}%
\pgfsetstrokecolor{currentstroke}%
\pgfsetstrokeopacity{0.000000}%
\pgfsetdash{}{0pt}%
\pgfpathmoveto{\pgfqpoint{0.636356in}{0.440955in}}%
\pgfpathlineto{\pgfqpoint{2.569584in}{0.440955in}}%
\pgfpathlineto{\pgfqpoint{2.569584in}{0.603387in}}%
\pgfpathlineto{\pgfqpoint{0.636356in}{0.603387in}}%
\pgfpathclose%
\pgfusepath{fill}%
\end{pgfscope}%
\begin{pgfscope}%
\pgfpathrectangle{\pgfqpoint{0.636356in}{0.440955in}}{\pgfqpoint{1.933229in}{0.162432in}} %
\pgfusepath{clip}%
\pgfsetbuttcap%
\pgfsetroundjoin%
\definecolor{currentfill}{rgb}{0.733333,0.733333,0.733333}%
\pgfsetfillcolor{currentfill}%
\pgfsetlinewidth{0.000000pt}%
\definecolor{currentstroke}{rgb}{0.733333,0.733333,0.733333}%
\pgfsetstrokecolor{currentstroke}%
\pgfsetdash{}{0pt}%
\pgfpathmoveto{\pgfqpoint{0.636356in}{0.549243in}}%
\pgfpathlineto{\pgfqpoint{0.636356in}{0.576315in}}%
\pgfpathlineto{\pgfqpoint{2.569584in}{0.576315in}}%
\pgfpathlineto{\pgfqpoint{2.569584in}{0.549243in}}%
\pgfpathlineto{\pgfqpoint{2.569584in}{0.549243in}}%
\pgfpathlineto{\pgfqpoint{0.636356in}{0.549243in}}%
\pgfpathlineto{\pgfqpoint{0.636356in}{0.549243in}}%
\pgfusepath{fill}%
\end{pgfscope}%
\begin{pgfscope}%
\pgfpathrectangle{\pgfqpoint{0.636356in}{0.440955in}}{\pgfqpoint{1.933229in}{0.162432in}} %
\pgfusepath{clip}%
\pgfsetbuttcap%
\pgfsetroundjoin%
\definecolor{currentfill}{rgb}{0.733333,0.733333,0.733333}%
\pgfsetfillcolor{currentfill}%
\pgfsetlinewidth{0.000000pt}%
\definecolor{currentstroke}{rgb}{0.733333,0.733333,0.733333}%
\pgfsetstrokecolor{currentstroke}%
\pgfsetdash{}{0pt}%
\pgfpathmoveto{\pgfqpoint{0.636356in}{0.495099in}}%
\pgfpathlineto{\pgfqpoint{0.636356in}{0.468027in}}%
\pgfpathlineto{\pgfqpoint{2.569584in}{0.468027in}}%
\pgfpathlineto{\pgfqpoint{2.569584in}{0.495099in}}%
\pgfpathlineto{\pgfqpoint{2.569584in}{0.495099in}}%
\pgfpathlineto{\pgfqpoint{0.636356in}{0.495099in}}%
\pgfpathlineto{\pgfqpoint{0.636356in}{0.495099in}}%
\pgfusepath{fill}%
\end{pgfscope}%
\begin{pgfscope}%
\pgfpathrectangle{\pgfqpoint{0.636356in}{0.440955in}}{\pgfqpoint{1.933229in}{0.162432in}} %
\pgfusepath{clip}%
\pgfsetbuttcap%
\pgfsetmiterjoin%
\definecolor{currentfill}{rgb}{0.333333,0.333333,0.333333}%
\pgfsetfillcolor{currentfill}%
\pgfsetlinewidth{0.501875pt}%
\definecolor{currentstroke}{rgb}{0.000000,0.000000,0.000000}%
\pgfsetstrokecolor{currentstroke}%
\pgfsetdash{}{0pt}%
\pgfpathmoveto{\pgfqpoint{0.636356in}{0.522171in}}%
\pgfpathlineto{\pgfqpoint{0.655688in}{0.522171in}}%
\pgfpathlineto{\pgfqpoint{0.655688in}{0.532679in}}%
\pgfpathlineto{\pgfqpoint{0.636356in}{0.532679in}}%
\pgfpathlineto{\pgfqpoint{0.636356in}{0.522171in}}%
\pgfusepath{stroke,fill}%
\end{pgfscope}%
\begin{pgfscope}%
\pgfpathrectangle{\pgfqpoint{0.636356in}{0.440955in}}{\pgfqpoint{1.933229in}{0.162432in}} %
\pgfusepath{clip}%
\pgfsetbuttcap%
\pgfsetmiterjoin%
\definecolor{currentfill}{rgb}{0.333333,0.333333,0.333333}%
\pgfsetfillcolor{currentfill}%
\pgfsetlinewidth{0.501875pt}%
\definecolor{currentstroke}{rgb}{0.000000,0.000000,0.000000}%
\pgfsetstrokecolor{currentstroke}%
\pgfsetdash{}{0pt}%
\pgfpathmoveto{\pgfqpoint{0.655688in}{0.522171in}}%
\pgfpathlineto{\pgfqpoint{0.675020in}{0.522171in}}%
\pgfpathlineto{\pgfqpoint{0.675020in}{0.546515in}}%
\pgfpathlineto{\pgfqpoint{0.655688in}{0.546515in}}%
\pgfpathlineto{\pgfqpoint{0.655688in}{0.522171in}}%
\pgfusepath{stroke,fill}%
\end{pgfscope}%
\begin{pgfscope}%
\pgfpathrectangle{\pgfqpoint{0.636356in}{0.440955in}}{\pgfqpoint{1.933229in}{0.162432in}} %
\pgfusepath{clip}%
\pgfsetbuttcap%
\pgfsetmiterjoin%
\definecolor{currentfill}{rgb}{0.333333,0.333333,0.333333}%
\pgfsetfillcolor{currentfill}%
\pgfsetlinewidth{0.501875pt}%
\definecolor{currentstroke}{rgb}{0.000000,0.000000,0.000000}%
\pgfsetstrokecolor{currentstroke}%
\pgfsetdash{}{0pt}%
\pgfpathmoveto{\pgfqpoint{0.675020in}{0.522171in}}%
\pgfpathlineto{\pgfqpoint{0.694353in}{0.522171in}}%
\pgfpathlineto{\pgfqpoint{0.694353in}{0.542006in}}%
\pgfpathlineto{\pgfqpoint{0.675020in}{0.542006in}}%
\pgfpathlineto{\pgfqpoint{0.675020in}{0.522171in}}%
\pgfusepath{stroke,fill}%
\end{pgfscope}%
\begin{pgfscope}%
\pgfpathrectangle{\pgfqpoint{0.636356in}{0.440955in}}{\pgfqpoint{1.933229in}{0.162432in}} %
\pgfusepath{clip}%
\pgfsetbuttcap%
\pgfsetmiterjoin%
\definecolor{currentfill}{rgb}{0.333333,0.333333,0.333333}%
\pgfsetfillcolor{currentfill}%
\pgfsetlinewidth{0.501875pt}%
\definecolor{currentstroke}{rgb}{0.000000,0.000000,0.000000}%
\pgfsetstrokecolor{currentstroke}%
\pgfsetdash{}{0pt}%
\pgfpathmoveto{\pgfqpoint{0.694353in}{0.520691in}}%
\pgfpathlineto{\pgfqpoint{0.713685in}{0.520691in}}%
\pgfpathlineto{\pgfqpoint{0.713685in}{0.522171in}}%
\pgfpathlineto{\pgfqpoint{0.694353in}{0.522171in}}%
\pgfpathlineto{\pgfqpoint{0.694353in}{0.520691in}}%
\pgfusepath{stroke,fill}%
\end{pgfscope}%
\begin{pgfscope}%
\pgfpathrectangle{\pgfqpoint{0.636356in}{0.440955in}}{\pgfqpoint{1.933229in}{0.162432in}} %
\pgfusepath{clip}%
\pgfsetbuttcap%
\pgfsetmiterjoin%
\definecolor{currentfill}{rgb}{0.333333,0.333333,0.333333}%
\pgfsetfillcolor{currentfill}%
\pgfsetlinewidth{0.501875pt}%
\definecolor{currentstroke}{rgb}{0.000000,0.000000,0.000000}%
\pgfsetstrokecolor{currentstroke}%
\pgfsetdash{}{0pt}%
\pgfpathmoveto{\pgfqpoint{0.713685in}{0.522171in}}%
\pgfpathlineto{\pgfqpoint{0.733017in}{0.522171in}}%
\pgfpathlineto{\pgfqpoint{0.733017in}{0.523979in}}%
\pgfpathlineto{\pgfqpoint{0.713685in}{0.523979in}}%
\pgfpathlineto{\pgfqpoint{0.713685in}{0.522171in}}%
\pgfusepath{stroke,fill}%
\end{pgfscope}%
\begin{pgfscope}%
\pgfpathrectangle{\pgfqpoint{0.636356in}{0.440955in}}{\pgfqpoint{1.933229in}{0.162432in}} %
\pgfusepath{clip}%
\pgfsetbuttcap%
\pgfsetmiterjoin%
\definecolor{currentfill}{rgb}{0.333333,0.333333,0.333333}%
\pgfsetfillcolor{currentfill}%
\pgfsetlinewidth{0.501875pt}%
\definecolor{currentstroke}{rgb}{0.000000,0.000000,0.000000}%
\pgfsetstrokecolor{currentstroke}%
\pgfsetdash{}{0pt}%
\pgfpathmoveto{\pgfqpoint{0.733017in}{0.522171in}}%
\pgfpathlineto{\pgfqpoint{0.752350in}{0.522171in}}%
\pgfpathlineto{\pgfqpoint{0.752350in}{0.534454in}}%
\pgfpathlineto{\pgfqpoint{0.733017in}{0.534454in}}%
\pgfpathlineto{\pgfqpoint{0.733017in}{0.522171in}}%
\pgfusepath{stroke,fill}%
\end{pgfscope}%
\begin{pgfscope}%
\pgfpathrectangle{\pgfqpoint{0.636356in}{0.440955in}}{\pgfqpoint{1.933229in}{0.162432in}} %
\pgfusepath{clip}%
\pgfsetbuttcap%
\pgfsetmiterjoin%
\definecolor{currentfill}{rgb}{0.333333,0.333333,0.333333}%
\pgfsetfillcolor{currentfill}%
\pgfsetlinewidth{0.501875pt}%
\definecolor{currentstroke}{rgb}{0.000000,0.000000,0.000000}%
\pgfsetstrokecolor{currentstroke}%
\pgfsetdash{}{0pt}%
\pgfpathmoveto{\pgfqpoint{0.752350in}{0.522171in}}%
\pgfpathlineto{\pgfqpoint{0.771682in}{0.522171in}}%
\pgfpathlineto{\pgfqpoint{0.771682in}{0.528763in}}%
\pgfpathlineto{\pgfqpoint{0.752350in}{0.528763in}}%
\pgfpathlineto{\pgfqpoint{0.752350in}{0.522171in}}%
\pgfusepath{stroke,fill}%
\end{pgfscope}%
\begin{pgfscope}%
\pgfpathrectangle{\pgfqpoint{0.636356in}{0.440955in}}{\pgfqpoint{1.933229in}{0.162432in}} %
\pgfusepath{clip}%
\pgfsetbuttcap%
\pgfsetmiterjoin%
\definecolor{currentfill}{rgb}{0.333333,0.333333,0.333333}%
\pgfsetfillcolor{currentfill}%
\pgfsetlinewidth{0.501875pt}%
\definecolor{currentstroke}{rgb}{0.000000,0.000000,0.000000}%
\pgfsetstrokecolor{currentstroke}%
\pgfsetdash{}{0pt}%
\pgfpathmoveto{\pgfqpoint{0.771682in}{0.522171in}}%
\pgfpathlineto{\pgfqpoint{0.791014in}{0.522171in}}%
\pgfpathlineto{\pgfqpoint{0.791014in}{0.545519in}}%
\pgfpathlineto{\pgfqpoint{0.771682in}{0.545519in}}%
\pgfpathlineto{\pgfqpoint{0.771682in}{0.522171in}}%
\pgfusepath{stroke,fill}%
\end{pgfscope}%
\begin{pgfscope}%
\pgfpathrectangle{\pgfqpoint{0.636356in}{0.440955in}}{\pgfqpoint{1.933229in}{0.162432in}} %
\pgfusepath{clip}%
\pgfsetbuttcap%
\pgfsetmiterjoin%
\definecolor{currentfill}{rgb}{0.333333,0.333333,0.333333}%
\pgfsetfillcolor{currentfill}%
\pgfsetlinewidth{0.501875pt}%
\definecolor{currentstroke}{rgb}{0.000000,0.000000,0.000000}%
\pgfsetstrokecolor{currentstroke}%
\pgfsetdash{}{0pt}%
\pgfpathmoveto{\pgfqpoint{0.791014in}{0.517373in}}%
\pgfpathlineto{\pgfqpoint{0.810346in}{0.517373in}}%
\pgfpathlineto{\pgfqpoint{0.810346in}{0.522171in}}%
\pgfpathlineto{\pgfqpoint{0.791014in}{0.522171in}}%
\pgfpathlineto{\pgfqpoint{0.791014in}{0.517373in}}%
\pgfusepath{stroke,fill}%
\end{pgfscope}%
\begin{pgfscope}%
\pgfpathrectangle{\pgfqpoint{0.636356in}{0.440955in}}{\pgfqpoint{1.933229in}{0.162432in}} %
\pgfusepath{clip}%
\pgfsetbuttcap%
\pgfsetmiterjoin%
\definecolor{currentfill}{rgb}{0.333333,0.333333,0.333333}%
\pgfsetfillcolor{currentfill}%
\pgfsetlinewidth{0.501875pt}%
\definecolor{currentstroke}{rgb}{0.000000,0.000000,0.000000}%
\pgfsetstrokecolor{currentstroke}%
\pgfsetdash{}{0pt}%
\pgfpathmoveto{\pgfqpoint{0.810346in}{0.522171in}}%
\pgfpathlineto{\pgfqpoint{0.829679in}{0.522171in}}%
\pgfpathlineto{\pgfqpoint{0.829679in}{0.602073in}}%
\pgfpathlineto{\pgfqpoint{0.810346in}{0.602073in}}%
\pgfpathlineto{\pgfqpoint{0.810346in}{0.522171in}}%
\pgfusepath{stroke,fill}%
\end{pgfscope}%
\begin{pgfscope}%
\pgfpathrectangle{\pgfqpoint{0.636356in}{0.440955in}}{\pgfqpoint{1.933229in}{0.162432in}} %
\pgfusepath{clip}%
\pgfsetbuttcap%
\pgfsetmiterjoin%
\definecolor{currentfill}{rgb}{0.333333,0.333333,0.333333}%
\pgfsetfillcolor{currentfill}%
\pgfsetlinewidth{0.501875pt}%
\definecolor{currentstroke}{rgb}{0.000000,0.000000,0.000000}%
\pgfsetstrokecolor{currentstroke}%
\pgfsetdash{}{0pt}%
\pgfpathmoveto{\pgfqpoint{0.829679in}{0.522171in}}%
\pgfpathlineto{\pgfqpoint{0.849011in}{0.522171in}}%
\pgfpathlineto{\pgfqpoint{0.849011in}{0.543783in}}%
\pgfpathlineto{\pgfqpoint{0.829679in}{0.543783in}}%
\pgfpathlineto{\pgfqpoint{0.829679in}{0.522171in}}%
\pgfusepath{stroke,fill}%
\end{pgfscope}%
\begin{pgfscope}%
\pgfpathrectangle{\pgfqpoint{0.636356in}{0.440955in}}{\pgfqpoint{1.933229in}{0.162432in}} %
\pgfusepath{clip}%
\pgfsetbuttcap%
\pgfsetmiterjoin%
\definecolor{currentfill}{rgb}{0.333333,0.333333,0.333333}%
\pgfsetfillcolor{currentfill}%
\pgfsetlinewidth{0.501875pt}%
\definecolor{currentstroke}{rgb}{0.000000,0.000000,0.000000}%
\pgfsetstrokecolor{currentstroke}%
\pgfsetdash{}{0pt}%
\pgfpathmoveto{\pgfqpoint{0.849011in}{0.522171in}}%
\pgfpathlineto{\pgfqpoint{0.868343in}{0.522171in}}%
\pgfpathlineto{\pgfqpoint{0.868343in}{0.556836in}}%
\pgfpathlineto{\pgfqpoint{0.849011in}{0.556836in}}%
\pgfpathlineto{\pgfqpoint{0.849011in}{0.522171in}}%
\pgfusepath{stroke,fill}%
\end{pgfscope}%
\begin{pgfscope}%
\pgfpathrectangle{\pgfqpoint{0.636356in}{0.440955in}}{\pgfqpoint{1.933229in}{0.162432in}} %
\pgfusepath{clip}%
\pgfsetbuttcap%
\pgfsetmiterjoin%
\definecolor{currentfill}{rgb}{0.333333,0.333333,0.333333}%
\pgfsetfillcolor{currentfill}%
\pgfsetlinewidth{0.501875pt}%
\definecolor{currentstroke}{rgb}{0.000000,0.000000,0.000000}%
\pgfsetstrokecolor{currentstroke}%
\pgfsetdash{}{0pt}%
\pgfpathmoveto{\pgfqpoint{0.868343in}{0.522171in}}%
\pgfpathlineto{\pgfqpoint{0.887676in}{0.522171in}}%
\pgfpathlineto{\pgfqpoint{0.887676in}{0.528611in}}%
\pgfpathlineto{\pgfqpoint{0.868343in}{0.528611in}}%
\pgfpathlineto{\pgfqpoint{0.868343in}{0.522171in}}%
\pgfusepath{stroke,fill}%
\end{pgfscope}%
\begin{pgfscope}%
\pgfpathrectangle{\pgfqpoint{0.636356in}{0.440955in}}{\pgfqpoint{1.933229in}{0.162432in}} %
\pgfusepath{clip}%
\pgfsetbuttcap%
\pgfsetmiterjoin%
\definecolor{currentfill}{rgb}{0.333333,0.333333,0.333333}%
\pgfsetfillcolor{currentfill}%
\pgfsetlinewidth{0.501875pt}%
\definecolor{currentstroke}{rgb}{0.000000,0.000000,0.000000}%
\pgfsetstrokecolor{currentstroke}%
\pgfsetdash{}{0pt}%
\pgfpathmoveto{\pgfqpoint{0.887676in}{0.522171in}}%
\pgfpathlineto{\pgfqpoint{0.907008in}{0.522171in}}%
\pgfpathlineto{\pgfqpoint{0.907008in}{0.533917in}}%
\pgfpathlineto{\pgfqpoint{0.887676in}{0.533917in}}%
\pgfpathlineto{\pgfqpoint{0.887676in}{0.522171in}}%
\pgfusepath{stroke,fill}%
\end{pgfscope}%
\begin{pgfscope}%
\pgfpathrectangle{\pgfqpoint{0.636356in}{0.440955in}}{\pgfqpoint{1.933229in}{0.162432in}} %
\pgfusepath{clip}%
\pgfsetbuttcap%
\pgfsetmiterjoin%
\definecolor{currentfill}{rgb}{0.333333,0.333333,0.333333}%
\pgfsetfillcolor{currentfill}%
\pgfsetlinewidth{0.501875pt}%
\definecolor{currentstroke}{rgb}{0.000000,0.000000,0.000000}%
\pgfsetstrokecolor{currentstroke}%
\pgfsetdash{}{0pt}%
\pgfpathmoveto{\pgfqpoint{0.907008in}{0.497889in}}%
\pgfpathlineto{\pgfqpoint{0.926340in}{0.497889in}}%
\pgfpathlineto{\pgfqpoint{0.926340in}{0.522171in}}%
\pgfpathlineto{\pgfqpoint{0.907008in}{0.522171in}}%
\pgfpathlineto{\pgfqpoint{0.907008in}{0.497889in}}%
\pgfusepath{stroke,fill}%
\end{pgfscope}%
\begin{pgfscope}%
\pgfpathrectangle{\pgfqpoint{0.636356in}{0.440955in}}{\pgfqpoint{1.933229in}{0.162432in}} %
\pgfusepath{clip}%
\pgfsetbuttcap%
\pgfsetmiterjoin%
\definecolor{currentfill}{rgb}{0.333333,0.333333,0.333333}%
\pgfsetfillcolor{currentfill}%
\pgfsetlinewidth{0.501875pt}%
\definecolor{currentstroke}{rgb}{0.000000,0.000000,0.000000}%
\pgfsetstrokecolor{currentstroke}%
\pgfsetdash{}{0pt}%
\pgfpathmoveto{\pgfqpoint{0.926340in}{0.511249in}}%
\pgfpathlineto{\pgfqpoint{0.945672in}{0.511249in}}%
\pgfpathlineto{\pgfqpoint{0.945672in}{0.522171in}}%
\pgfpathlineto{\pgfqpoint{0.926340in}{0.522171in}}%
\pgfpathlineto{\pgfqpoint{0.926340in}{0.511249in}}%
\pgfusepath{stroke,fill}%
\end{pgfscope}%
\begin{pgfscope}%
\pgfpathrectangle{\pgfqpoint{0.636356in}{0.440955in}}{\pgfqpoint{1.933229in}{0.162432in}} %
\pgfusepath{clip}%
\pgfsetbuttcap%
\pgfsetmiterjoin%
\definecolor{currentfill}{rgb}{0.333333,0.333333,0.333333}%
\pgfsetfillcolor{currentfill}%
\pgfsetlinewidth{0.501875pt}%
\definecolor{currentstroke}{rgb}{0.000000,0.000000,0.000000}%
\pgfsetstrokecolor{currentstroke}%
\pgfsetdash{}{0pt}%
\pgfpathmoveto{\pgfqpoint{0.945672in}{0.522171in}}%
\pgfpathlineto{\pgfqpoint{0.965005in}{0.522171in}}%
\pgfpathlineto{\pgfqpoint{0.965005in}{0.549477in}}%
\pgfpathlineto{\pgfqpoint{0.945672in}{0.549477in}}%
\pgfpathlineto{\pgfqpoint{0.945672in}{0.522171in}}%
\pgfusepath{stroke,fill}%
\end{pgfscope}%
\begin{pgfscope}%
\pgfpathrectangle{\pgfqpoint{0.636356in}{0.440955in}}{\pgfqpoint{1.933229in}{0.162432in}} %
\pgfusepath{clip}%
\pgfsetbuttcap%
\pgfsetmiterjoin%
\definecolor{currentfill}{rgb}{0.333333,0.333333,0.333333}%
\pgfsetfillcolor{currentfill}%
\pgfsetlinewidth{0.501875pt}%
\definecolor{currentstroke}{rgb}{0.000000,0.000000,0.000000}%
\pgfsetstrokecolor{currentstroke}%
\pgfsetdash{}{0pt}%
\pgfpathmoveto{\pgfqpoint{0.965005in}{0.521361in}}%
\pgfpathlineto{\pgfqpoint{0.984337in}{0.521361in}}%
\pgfpathlineto{\pgfqpoint{0.984337in}{0.522171in}}%
\pgfpathlineto{\pgfqpoint{0.965005in}{0.522171in}}%
\pgfpathlineto{\pgfqpoint{0.965005in}{0.521361in}}%
\pgfusepath{stroke,fill}%
\end{pgfscope}%
\begin{pgfscope}%
\pgfpathrectangle{\pgfqpoint{0.636356in}{0.440955in}}{\pgfqpoint{1.933229in}{0.162432in}} %
\pgfusepath{clip}%
\pgfsetbuttcap%
\pgfsetmiterjoin%
\definecolor{currentfill}{rgb}{0.333333,0.333333,0.333333}%
\pgfsetfillcolor{currentfill}%
\pgfsetlinewidth{0.501875pt}%
\definecolor{currentstroke}{rgb}{0.000000,0.000000,0.000000}%
\pgfsetstrokecolor{currentstroke}%
\pgfsetdash{}{0pt}%
\pgfpathmoveto{\pgfqpoint{0.984337in}{0.522171in}}%
\pgfpathlineto{\pgfqpoint{1.003669in}{0.522171in}}%
\pgfpathlineto{\pgfqpoint{1.003669in}{0.526091in}}%
\pgfpathlineto{\pgfqpoint{0.984337in}{0.526091in}}%
\pgfpathlineto{\pgfqpoint{0.984337in}{0.522171in}}%
\pgfusepath{stroke,fill}%
\end{pgfscope}%
\begin{pgfscope}%
\pgfpathrectangle{\pgfqpoint{0.636356in}{0.440955in}}{\pgfqpoint{1.933229in}{0.162432in}} %
\pgfusepath{clip}%
\pgfsetbuttcap%
\pgfsetmiterjoin%
\definecolor{currentfill}{rgb}{0.333333,0.333333,0.333333}%
\pgfsetfillcolor{currentfill}%
\pgfsetlinewidth{0.501875pt}%
\definecolor{currentstroke}{rgb}{0.000000,0.000000,0.000000}%
\pgfsetstrokecolor{currentstroke}%
\pgfsetdash{}{0pt}%
\pgfpathmoveto{\pgfqpoint{1.003669in}{0.522171in}}%
\pgfpathlineto{\pgfqpoint{1.023002in}{0.522171in}}%
\pgfpathlineto{\pgfqpoint{1.023002in}{0.563192in}}%
\pgfpathlineto{\pgfqpoint{1.003669in}{0.563192in}}%
\pgfpathlineto{\pgfqpoint{1.003669in}{0.522171in}}%
\pgfusepath{stroke,fill}%
\end{pgfscope}%
\begin{pgfscope}%
\pgfpathrectangle{\pgfqpoint{0.636356in}{0.440955in}}{\pgfqpoint{1.933229in}{0.162432in}} %
\pgfusepath{clip}%
\pgfsetbuttcap%
\pgfsetmiterjoin%
\definecolor{currentfill}{rgb}{0.333333,0.333333,0.333333}%
\pgfsetfillcolor{currentfill}%
\pgfsetlinewidth{0.501875pt}%
\definecolor{currentstroke}{rgb}{0.000000,0.000000,0.000000}%
\pgfsetstrokecolor{currentstroke}%
\pgfsetdash{}{0pt}%
\pgfpathmoveto{\pgfqpoint{1.023002in}{0.522171in}}%
\pgfpathlineto{\pgfqpoint{1.042334in}{0.522171in}}%
\pgfpathlineto{\pgfqpoint{1.042334in}{0.575631in}}%
\pgfpathlineto{\pgfqpoint{1.023002in}{0.575631in}}%
\pgfpathlineto{\pgfqpoint{1.023002in}{0.522171in}}%
\pgfusepath{stroke,fill}%
\end{pgfscope}%
\begin{pgfscope}%
\pgfpathrectangle{\pgfqpoint{0.636356in}{0.440955in}}{\pgfqpoint{1.933229in}{0.162432in}} %
\pgfusepath{clip}%
\pgfsetbuttcap%
\pgfsetmiterjoin%
\definecolor{currentfill}{rgb}{0.333333,0.333333,0.333333}%
\pgfsetfillcolor{currentfill}%
\pgfsetlinewidth{0.501875pt}%
\definecolor{currentstroke}{rgb}{0.000000,0.000000,0.000000}%
\pgfsetstrokecolor{currentstroke}%
\pgfsetdash{}{0pt}%
\pgfpathmoveto{\pgfqpoint{1.042334in}{0.522171in}}%
\pgfpathlineto{\pgfqpoint{1.061666in}{0.522171in}}%
\pgfpathlineto{\pgfqpoint{1.061666in}{0.551056in}}%
\pgfpathlineto{\pgfqpoint{1.042334in}{0.551056in}}%
\pgfpathlineto{\pgfqpoint{1.042334in}{0.522171in}}%
\pgfusepath{stroke,fill}%
\end{pgfscope}%
\begin{pgfscope}%
\pgfpathrectangle{\pgfqpoint{0.636356in}{0.440955in}}{\pgfqpoint{1.933229in}{0.162432in}} %
\pgfusepath{clip}%
\pgfsetbuttcap%
\pgfsetmiterjoin%
\definecolor{currentfill}{rgb}{0.333333,0.333333,0.333333}%
\pgfsetfillcolor{currentfill}%
\pgfsetlinewidth{0.501875pt}%
\definecolor{currentstroke}{rgb}{0.000000,0.000000,0.000000}%
\pgfsetstrokecolor{currentstroke}%
\pgfsetdash{}{0pt}%
\pgfpathmoveto{\pgfqpoint{1.061666in}{0.522171in}}%
\pgfpathlineto{\pgfqpoint{1.080998in}{0.522171in}}%
\pgfpathlineto{\pgfqpoint{1.080998in}{0.530070in}}%
\pgfpathlineto{\pgfqpoint{1.061666in}{0.530070in}}%
\pgfpathlineto{\pgfqpoint{1.061666in}{0.522171in}}%
\pgfusepath{stroke,fill}%
\end{pgfscope}%
\begin{pgfscope}%
\pgfpathrectangle{\pgfqpoint{0.636356in}{0.440955in}}{\pgfqpoint{1.933229in}{0.162432in}} %
\pgfusepath{clip}%
\pgfsetbuttcap%
\pgfsetmiterjoin%
\definecolor{currentfill}{rgb}{0.333333,0.333333,0.333333}%
\pgfsetfillcolor{currentfill}%
\pgfsetlinewidth{0.501875pt}%
\definecolor{currentstroke}{rgb}{0.000000,0.000000,0.000000}%
\pgfsetstrokecolor{currentstroke}%
\pgfsetdash{}{0pt}%
\pgfpathmoveto{\pgfqpoint{1.080998in}{0.522171in}}%
\pgfpathlineto{\pgfqpoint{1.100331in}{0.522171in}}%
\pgfpathlineto{\pgfqpoint{1.100331in}{0.531486in}}%
\pgfpathlineto{\pgfqpoint{1.080998in}{0.531486in}}%
\pgfpathlineto{\pgfqpoint{1.080998in}{0.522171in}}%
\pgfusepath{stroke,fill}%
\end{pgfscope}%
\begin{pgfscope}%
\pgfpathrectangle{\pgfqpoint{0.636356in}{0.440955in}}{\pgfqpoint{1.933229in}{0.162432in}} %
\pgfusepath{clip}%
\pgfsetbuttcap%
\pgfsetmiterjoin%
\definecolor{currentfill}{rgb}{0.333333,0.333333,0.333333}%
\pgfsetfillcolor{currentfill}%
\pgfsetlinewidth{0.501875pt}%
\definecolor{currentstroke}{rgb}{0.000000,0.000000,0.000000}%
\pgfsetstrokecolor{currentstroke}%
\pgfsetdash{}{0pt}%
\pgfpathmoveto{\pgfqpoint{1.100331in}{0.522171in}}%
\pgfpathlineto{\pgfqpoint{1.119663in}{0.522171in}}%
\pgfpathlineto{\pgfqpoint{1.119663in}{0.530927in}}%
\pgfpathlineto{\pgfqpoint{1.100331in}{0.530927in}}%
\pgfpathlineto{\pgfqpoint{1.100331in}{0.522171in}}%
\pgfusepath{stroke,fill}%
\end{pgfscope}%
\begin{pgfscope}%
\pgfpathrectangle{\pgfqpoint{0.636356in}{0.440955in}}{\pgfqpoint{1.933229in}{0.162432in}} %
\pgfusepath{clip}%
\pgfsetbuttcap%
\pgfsetmiterjoin%
\definecolor{currentfill}{rgb}{0.333333,0.333333,0.333333}%
\pgfsetfillcolor{currentfill}%
\pgfsetlinewidth{0.501875pt}%
\definecolor{currentstroke}{rgb}{0.000000,0.000000,0.000000}%
\pgfsetstrokecolor{currentstroke}%
\pgfsetdash{}{0pt}%
\pgfpathmoveto{\pgfqpoint{1.119663in}{0.522171in}}%
\pgfpathlineto{\pgfqpoint{1.138995in}{0.522171in}}%
\pgfpathlineto{\pgfqpoint{1.138995in}{0.523748in}}%
\pgfpathlineto{\pgfqpoint{1.119663in}{0.523748in}}%
\pgfpathlineto{\pgfqpoint{1.119663in}{0.522171in}}%
\pgfusepath{stroke,fill}%
\end{pgfscope}%
\begin{pgfscope}%
\pgfpathrectangle{\pgfqpoint{0.636356in}{0.440955in}}{\pgfqpoint{1.933229in}{0.162432in}} %
\pgfusepath{clip}%
\pgfsetbuttcap%
\pgfsetmiterjoin%
\definecolor{currentfill}{rgb}{0.333333,0.333333,0.333333}%
\pgfsetfillcolor{currentfill}%
\pgfsetlinewidth{0.501875pt}%
\definecolor{currentstroke}{rgb}{0.000000,0.000000,0.000000}%
\pgfsetstrokecolor{currentstroke}%
\pgfsetdash{}{0pt}%
\pgfpathmoveto{\pgfqpoint{1.138995in}{0.522171in}}%
\pgfpathlineto{\pgfqpoint{1.158328in}{0.522171in}}%
\pgfpathlineto{\pgfqpoint{1.158328in}{0.529511in}}%
\pgfpathlineto{\pgfqpoint{1.138995in}{0.529511in}}%
\pgfpathlineto{\pgfqpoint{1.138995in}{0.522171in}}%
\pgfusepath{stroke,fill}%
\end{pgfscope}%
\begin{pgfscope}%
\pgfpathrectangle{\pgfqpoint{0.636356in}{0.440955in}}{\pgfqpoint{1.933229in}{0.162432in}} %
\pgfusepath{clip}%
\pgfsetbuttcap%
\pgfsetmiterjoin%
\definecolor{currentfill}{rgb}{0.333333,0.333333,0.333333}%
\pgfsetfillcolor{currentfill}%
\pgfsetlinewidth{0.501875pt}%
\definecolor{currentstroke}{rgb}{0.000000,0.000000,0.000000}%
\pgfsetstrokecolor{currentstroke}%
\pgfsetdash{}{0pt}%
\pgfpathmoveto{\pgfqpoint{1.158328in}{0.502590in}}%
\pgfpathlineto{\pgfqpoint{1.177660in}{0.502590in}}%
\pgfpathlineto{\pgfqpoint{1.177660in}{0.522171in}}%
\pgfpathlineto{\pgfqpoint{1.158328in}{0.522171in}}%
\pgfpathlineto{\pgfqpoint{1.158328in}{0.502590in}}%
\pgfusepath{stroke,fill}%
\end{pgfscope}%
\begin{pgfscope}%
\pgfpathrectangle{\pgfqpoint{0.636356in}{0.440955in}}{\pgfqpoint{1.933229in}{0.162432in}} %
\pgfusepath{clip}%
\pgfsetbuttcap%
\pgfsetmiterjoin%
\definecolor{currentfill}{rgb}{0.333333,0.333333,0.333333}%
\pgfsetfillcolor{currentfill}%
\pgfsetlinewidth{0.501875pt}%
\definecolor{currentstroke}{rgb}{0.000000,0.000000,0.000000}%
\pgfsetstrokecolor{currentstroke}%
\pgfsetdash{}{0pt}%
\pgfpathmoveto{\pgfqpoint{1.177660in}{0.513033in}}%
\pgfpathlineto{\pgfqpoint{1.196992in}{0.513033in}}%
\pgfpathlineto{\pgfqpoint{1.196992in}{0.522171in}}%
\pgfpathlineto{\pgfqpoint{1.177660in}{0.522171in}}%
\pgfpathlineto{\pgfqpoint{1.177660in}{0.513033in}}%
\pgfusepath{stroke,fill}%
\end{pgfscope}%
\begin{pgfscope}%
\pgfpathrectangle{\pgfqpoint{0.636356in}{0.440955in}}{\pgfqpoint{1.933229in}{0.162432in}} %
\pgfusepath{clip}%
\pgfsetbuttcap%
\pgfsetmiterjoin%
\definecolor{currentfill}{rgb}{0.333333,0.333333,0.333333}%
\pgfsetfillcolor{currentfill}%
\pgfsetlinewidth{0.501875pt}%
\definecolor{currentstroke}{rgb}{0.000000,0.000000,0.000000}%
\pgfsetstrokecolor{currentstroke}%
\pgfsetdash{}{0pt}%
\pgfpathmoveto{\pgfqpoint{1.196992in}{0.522171in}}%
\pgfpathlineto{\pgfqpoint{1.216324in}{0.522171in}}%
\pgfpathlineto{\pgfqpoint{1.216324in}{0.552480in}}%
\pgfpathlineto{\pgfqpoint{1.196992in}{0.552480in}}%
\pgfpathlineto{\pgfqpoint{1.196992in}{0.522171in}}%
\pgfusepath{stroke,fill}%
\end{pgfscope}%
\begin{pgfscope}%
\pgfpathrectangle{\pgfqpoint{0.636356in}{0.440955in}}{\pgfqpoint{1.933229in}{0.162432in}} %
\pgfusepath{clip}%
\pgfsetbuttcap%
\pgfsetmiterjoin%
\definecolor{currentfill}{rgb}{0.333333,0.333333,0.333333}%
\pgfsetfillcolor{currentfill}%
\pgfsetlinewidth{0.501875pt}%
\definecolor{currentstroke}{rgb}{0.000000,0.000000,0.000000}%
\pgfsetstrokecolor{currentstroke}%
\pgfsetdash{}{0pt}%
\pgfpathmoveto{\pgfqpoint{1.216324in}{0.522171in}}%
\pgfpathlineto{\pgfqpoint{1.235657in}{0.522171in}}%
\pgfpathlineto{\pgfqpoint{1.235657in}{0.530867in}}%
\pgfpathlineto{\pgfqpoint{1.216324in}{0.530867in}}%
\pgfpathlineto{\pgfqpoint{1.216324in}{0.522171in}}%
\pgfusepath{stroke,fill}%
\end{pgfscope}%
\begin{pgfscope}%
\pgfpathrectangle{\pgfqpoint{0.636356in}{0.440955in}}{\pgfqpoint{1.933229in}{0.162432in}} %
\pgfusepath{clip}%
\pgfsetbuttcap%
\pgfsetmiterjoin%
\definecolor{currentfill}{rgb}{0.333333,0.333333,0.333333}%
\pgfsetfillcolor{currentfill}%
\pgfsetlinewidth{0.501875pt}%
\definecolor{currentstroke}{rgb}{0.000000,0.000000,0.000000}%
\pgfsetstrokecolor{currentstroke}%
\pgfsetdash{}{0pt}%
\pgfpathmoveto{\pgfqpoint{1.235657in}{0.522171in}}%
\pgfpathlineto{\pgfqpoint{1.254989in}{0.522171in}}%
\pgfpathlineto{\pgfqpoint{1.254989in}{0.538540in}}%
\pgfpathlineto{\pgfqpoint{1.235657in}{0.538540in}}%
\pgfpathlineto{\pgfqpoint{1.235657in}{0.522171in}}%
\pgfusepath{stroke,fill}%
\end{pgfscope}%
\begin{pgfscope}%
\pgfpathrectangle{\pgfqpoint{0.636356in}{0.440955in}}{\pgfqpoint{1.933229in}{0.162432in}} %
\pgfusepath{clip}%
\pgfsetbuttcap%
\pgfsetmiterjoin%
\definecolor{currentfill}{rgb}{0.333333,0.333333,0.333333}%
\pgfsetfillcolor{currentfill}%
\pgfsetlinewidth{0.501875pt}%
\definecolor{currentstroke}{rgb}{0.000000,0.000000,0.000000}%
\pgfsetstrokecolor{currentstroke}%
\pgfsetdash{}{0pt}%
\pgfpathmoveto{\pgfqpoint{1.254989in}{0.463880in}}%
\pgfpathlineto{\pgfqpoint{1.274321in}{0.463880in}}%
\pgfpathlineto{\pgfqpoint{1.274321in}{0.522171in}}%
\pgfpathlineto{\pgfqpoint{1.254989in}{0.522171in}}%
\pgfpathlineto{\pgfqpoint{1.254989in}{0.463880in}}%
\pgfusepath{stroke,fill}%
\end{pgfscope}%
\begin{pgfscope}%
\pgfpathrectangle{\pgfqpoint{0.636356in}{0.440955in}}{\pgfqpoint{1.933229in}{0.162432in}} %
\pgfusepath{clip}%
\pgfsetbuttcap%
\pgfsetmiterjoin%
\definecolor{currentfill}{rgb}{0.333333,0.333333,0.333333}%
\pgfsetfillcolor{currentfill}%
\pgfsetlinewidth{0.501875pt}%
\definecolor{currentstroke}{rgb}{0.000000,0.000000,0.000000}%
\pgfsetstrokecolor{currentstroke}%
\pgfsetdash{}{0pt}%
\pgfpathmoveto{\pgfqpoint{1.274321in}{0.492582in}}%
\pgfpathlineto{\pgfqpoint{1.293654in}{0.492582in}}%
\pgfpathlineto{\pgfqpoint{1.293654in}{0.522171in}}%
\pgfpathlineto{\pgfqpoint{1.274321in}{0.522171in}}%
\pgfpathlineto{\pgfqpoint{1.274321in}{0.492582in}}%
\pgfusepath{stroke,fill}%
\end{pgfscope}%
\begin{pgfscope}%
\pgfpathrectangle{\pgfqpoint{0.636356in}{0.440955in}}{\pgfqpoint{1.933229in}{0.162432in}} %
\pgfusepath{clip}%
\pgfsetbuttcap%
\pgfsetmiterjoin%
\definecolor{currentfill}{rgb}{0.333333,0.333333,0.333333}%
\pgfsetfillcolor{currentfill}%
\pgfsetlinewidth{0.501875pt}%
\definecolor{currentstroke}{rgb}{0.000000,0.000000,0.000000}%
\pgfsetstrokecolor{currentstroke}%
\pgfsetdash{}{0pt}%
\pgfpathmoveto{\pgfqpoint{1.293654in}{0.513816in}}%
\pgfpathlineto{\pgfqpoint{1.312986in}{0.513816in}}%
\pgfpathlineto{\pgfqpoint{1.312986in}{0.522171in}}%
\pgfpathlineto{\pgfqpoint{1.293654in}{0.522171in}}%
\pgfpathlineto{\pgfqpoint{1.293654in}{0.513816in}}%
\pgfusepath{stroke,fill}%
\end{pgfscope}%
\begin{pgfscope}%
\pgfpathrectangle{\pgfqpoint{0.636356in}{0.440955in}}{\pgfqpoint{1.933229in}{0.162432in}} %
\pgfusepath{clip}%
\pgfsetbuttcap%
\pgfsetmiterjoin%
\definecolor{currentfill}{rgb}{0.333333,0.333333,0.333333}%
\pgfsetfillcolor{currentfill}%
\pgfsetlinewidth{0.501875pt}%
\definecolor{currentstroke}{rgb}{0.000000,0.000000,0.000000}%
\pgfsetstrokecolor{currentstroke}%
\pgfsetdash{}{0pt}%
\pgfpathmoveto{\pgfqpoint{1.312986in}{0.522171in}}%
\pgfpathlineto{\pgfqpoint{1.332318in}{0.522171in}}%
\pgfpathlineto{\pgfqpoint{1.332318in}{0.525100in}}%
\pgfpathlineto{\pgfqpoint{1.312986in}{0.525100in}}%
\pgfpathlineto{\pgfqpoint{1.312986in}{0.522171in}}%
\pgfusepath{stroke,fill}%
\end{pgfscope}%
\begin{pgfscope}%
\pgfpathrectangle{\pgfqpoint{0.636356in}{0.440955in}}{\pgfqpoint{1.933229in}{0.162432in}} %
\pgfusepath{clip}%
\pgfsetbuttcap%
\pgfsetmiterjoin%
\definecolor{currentfill}{rgb}{0.333333,0.333333,0.333333}%
\pgfsetfillcolor{currentfill}%
\pgfsetlinewidth{0.501875pt}%
\definecolor{currentstroke}{rgb}{0.000000,0.000000,0.000000}%
\pgfsetstrokecolor{currentstroke}%
\pgfsetdash{}{0pt}%
\pgfpathmoveto{\pgfqpoint{1.332318in}{0.510887in}}%
\pgfpathlineto{\pgfqpoint{1.351650in}{0.510887in}}%
\pgfpathlineto{\pgfqpoint{1.351650in}{0.522171in}}%
\pgfpathlineto{\pgfqpoint{1.332318in}{0.522171in}}%
\pgfpathlineto{\pgfqpoint{1.332318in}{0.510887in}}%
\pgfusepath{stroke,fill}%
\end{pgfscope}%
\begin{pgfscope}%
\pgfpathrectangle{\pgfqpoint{0.636356in}{0.440955in}}{\pgfqpoint{1.933229in}{0.162432in}} %
\pgfusepath{clip}%
\pgfsetbuttcap%
\pgfsetmiterjoin%
\definecolor{currentfill}{rgb}{0.333333,0.333333,0.333333}%
\pgfsetfillcolor{currentfill}%
\pgfsetlinewidth{0.501875pt}%
\definecolor{currentstroke}{rgb}{0.000000,0.000000,0.000000}%
\pgfsetstrokecolor{currentstroke}%
\pgfsetdash{}{0pt}%
\pgfpathmoveto{\pgfqpoint{1.351650in}{0.509983in}}%
\pgfpathlineto{\pgfqpoint{1.370983in}{0.509983in}}%
\pgfpathlineto{\pgfqpoint{1.370983in}{0.522171in}}%
\pgfpathlineto{\pgfqpoint{1.351650in}{0.522171in}}%
\pgfpathlineto{\pgfqpoint{1.351650in}{0.509983in}}%
\pgfusepath{stroke,fill}%
\end{pgfscope}%
\begin{pgfscope}%
\pgfpathrectangle{\pgfqpoint{0.636356in}{0.440955in}}{\pgfqpoint{1.933229in}{0.162432in}} %
\pgfusepath{clip}%
\pgfsetbuttcap%
\pgfsetmiterjoin%
\definecolor{currentfill}{rgb}{0.333333,0.333333,0.333333}%
\pgfsetfillcolor{currentfill}%
\pgfsetlinewidth{0.501875pt}%
\definecolor{currentstroke}{rgb}{0.000000,0.000000,0.000000}%
\pgfsetstrokecolor{currentstroke}%
\pgfsetdash{}{0pt}%
\pgfpathmoveto{\pgfqpoint{1.370983in}{0.522171in}}%
\pgfpathlineto{\pgfqpoint{1.390315in}{0.522171in}}%
\pgfpathlineto{\pgfqpoint{1.390315in}{0.536372in}}%
\pgfpathlineto{\pgfqpoint{1.370983in}{0.536372in}}%
\pgfpathlineto{\pgfqpoint{1.370983in}{0.522171in}}%
\pgfusepath{stroke,fill}%
\end{pgfscope}%
\begin{pgfscope}%
\pgfpathrectangle{\pgfqpoint{0.636356in}{0.440955in}}{\pgfqpoint{1.933229in}{0.162432in}} %
\pgfusepath{clip}%
\pgfsetbuttcap%
\pgfsetmiterjoin%
\definecolor{currentfill}{rgb}{0.333333,0.333333,0.333333}%
\pgfsetfillcolor{currentfill}%
\pgfsetlinewidth{0.501875pt}%
\definecolor{currentstroke}{rgb}{0.000000,0.000000,0.000000}%
\pgfsetstrokecolor{currentstroke}%
\pgfsetdash{}{0pt}%
\pgfpathmoveto{\pgfqpoint{1.390315in}{0.461328in}}%
\pgfpathlineto{\pgfqpoint{1.409647in}{0.461328in}}%
\pgfpathlineto{\pgfqpoint{1.409647in}{0.522171in}}%
\pgfpathlineto{\pgfqpoint{1.390315in}{0.522171in}}%
\pgfpathlineto{\pgfqpoint{1.390315in}{0.461328in}}%
\pgfusepath{stroke,fill}%
\end{pgfscope}%
\begin{pgfscope}%
\pgfpathrectangle{\pgfqpoint{0.636356in}{0.440955in}}{\pgfqpoint{1.933229in}{0.162432in}} %
\pgfusepath{clip}%
\pgfsetbuttcap%
\pgfsetmiterjoin%
\definecolor{currentfill}{rgb}{0.333333,0.333333,0.333333}%
\pgfsetfillcolor{currentfill}%
\pgfsetlinewidth{0.501875pt}%
\definecolor{currentstroke}{rgb}{0.000000,0.000000,0.000000}%
\pgfsetstrokecolor{currentstroke}%
\pgfsetdash{}{0pt}%
\pgfpathmoveto{\pgfqpoint{1.409647in}{0.522171in}}%
\pgfpathlineto{\pgfqpoint{1.428980in}{0.522171in}}%
\pgfpathlineto{\pgfqpoint{1.428980in}{0.531244in}}%
\pgfpathlineto{\pgfqpoint{1.409647in}{0.531244in}}%
\pgfpathlineto{\pgfqpoint{1.409647in}{0.522171in}}%
\pgfusepath{stroke,fill}%
\end{pgfscope}%
\begin{pgfscope}%
\pgfpathrectangle{\pgfqpoint{0.636356in}{0.440955in}}{\pgfqpoint{1.933229in}{0.162432in}} %
\pgfusepath{clip}%
\pgfsetbuttcap%
\pgfsetmiterjoin%
\definecolor{currentfill}{rgb}{0.333333,0.333333,0.333333}%
\pgfsetfillcolor{currentfill}%
\pgfsetlinewidth{0.501875pt}%
\definecolor{currentstroke}{rgb}{0.000000,0.000000,0.000000}%
\pgfsetstrokecolor{currentstroke}%
\pgfsetdash{}{0pt}%
\pgfpathmoveto{\pgfqpoint{1.428980in}{0.505435in}}%
\pgfpathlineto{\pgfqpoint{1.448312in}{0.505435in}}%
\pgfpathlineto{\pgfqpoint{1.448312in}{0.522171in}}%
\pgfpathlineto{\pgfqpoint{1.428980in}{0.522171in}}%
\pgfpathlineto{\pgfqpoint{1.428980in}{0.505435in}}%
\pgfusepath{stroke,fill}%
\end{pgfscope}%
\begin{pgfscope}%
\pgfpathrectangle{\pgfqpoint{0.636356in}{0.440955in}}{\pgfqpoint{1.933229in}{0.162432in}} %
\pgfusepath{clip}%
\pgfsetbuttcap%
\pgfsetmiterjoin%
\definecolor{currentfill}{rgb}{0.333333,0.333333,0.333333}%
\pgfsetfillcolor{currentfill}%
\pgfsetlinewidth{0.501875pt}%
\definecolor{currentstroke}{rgb}{0.000000,0.000000,0.000000}%
\pgfsetstrokecolor{currentstroke}%
\pgfsetdash{}{0pt}%
\pgfpathmoveto{\pgfqpoint{1.448312in}{0.515359in}}%
\pgfpathlineto{\pgfqpoint{1.467644in}{0.515359in}}%
\pgfpathlineto{\pgfqpoint{1.467644in}{0.522171in}}%
\pgfpathlineto{\pgfqpoint{1.448312in}{0.522171in}}%
\pgfpathlineto{\pgfqpoint{1.448312in}{0.515359in}}%
\pgfusepath{stroke,fill}%
\end{pgfscope}%
\begin{pgfscope}%
\pgfpathrectangle{\pgfqpoint{0.636356in}{0.440955in}}{\pgfqpoint{1.933229in}{0.162432in}} %
\pgfusepath{clip}%
\pgfsetbuttcap%
\pgfsetmiterjoin%
\definecolor{currentfill}{rgb}{0.333333,0.333333,0.333333}%
\pgfsetfillcolor{currentfill}%
\pgfsetlinewidth{0.501875pt}%
\definecolor{currentstroke}{rgb}{0.000000,0.000000,0.000000}%
\pgfsetstrokecolor{currentstroke}%
\pgfsetdash{}{0pt}%
\pgfpathmoveto{\pgfqpoint{1.467644in}{0.495990in}}%
\pgfpathlineto{\pgfqpoint{1.486976in}{0.495990in}}%
\pgfpathlineto{\pgfqpoint{1.486976in}{0.522171in}}%
\pgfpathlineto{\pgfqpoint{1.467644in}{0.522171in}}%
\pgfpathlineto{\pgfqpoint{1.467644in}{0.495990in}}%
\pgfusepath{stroke,fill}%
\end{pgfscope}%
\begin{pgfscope}%
\pgfpathrectangle{\pgfqpoint{0.636356in}{0.440955in}}{\pgfqpoint{1.933229in}{0.162432in}} %
\pgfusepath{clip}%
\pgfsetbuttcap%
\pgfsetmiterjoin%
\definecolor{currentfill}{rgb}{0.333333,0.333333,0.333333}%
\pgfsetfillcolor{currentfill}%
\pgfsetlinewidth{0.501875pt}%
\definecolor{currentstroke}{rgb}{0.000000,0.000000,0.000000}%
\pgfsetstrokecolor{currentstroke}%
\pgfsetdash{}{0pt}%
\pgfpathmoveto{\pgfqpoint{1.486976in}{0.496638in}}%
\pgfpathlineto{\pgfqpoint{1.506309in}{0.496638in}}%
\pgfpathlineto{\pgfqpoint{1.506309in}{0.522171in}}%
\pgfpathlineto{\pgfqpoint{1.486976in}{0.522171in}}%
\pgfpathlineto{\pgfqpoint{1.486976in}{0.496638in}}%
\pgfusepath{stroke,fill}%
\end{pgfscope}%
\begin{pgfscope}%
\pgfpathrectangle{\pgfqpoint{0.636356in}{0.440955in}}{\pgfqpoint{1.933229in}{0.162432in}} %
\pgfusepath{clip}%
\pgfsetbuttcap%
\pgfsetmiterjoin%
\definecolor{currentfill}{rgb}{0.333333,0.333333,0.333333}%
\pgfsetfillcolor{currentfill}%
\pgfsetlinewidth{0.501875pt}%
\definecolor{currentstroke}{rgb}{0.000000,0.000000,0.000000}%
\pgfsetstrokecolor{currentstroke}%
\pgfsetdash{}{0pt}%
\pgfpathmoveto{\pgfqpoint{1.506309in}{0.520221in}}%
\pgfpathlineto{\pgfqpoint{1.525641in}{0.520221in}}%
\pgfpathlineto{\pgfqpoint{1.525641in}{0.522171in}}%
\pgfpathlineto{\pgfqpoint{1.506309in}{0.522171in}}%
\pgfpathlineto{\pgfqpoint{1.506309in}{0.520221in}}%
\pgfusepath{stroke,fill}%
\end{pgfscope}%
\begin{pgfscope}%
\pgfpathrectangle{\pgfqpoint{0.636356in}{0.440955in}}{\pgfqpoint{1.933229in}{0.162432in}} %
\pgfusepath{clip}%
\pgfsetbuttcap%
\pgfsetmiterjoin%
\definecolor{currentfill}{rgb}{0.333333,0.333333,0.333333}%
\pgfsetfillcolor{currentfill}%
\pgfsetlinewidth{0.501875pt}%
\definecolor{currentstroke}{rgb}{0.000000,0.000000,0.000000}%
\pgfsetstrokecolor{currentstroke}%
\pgfsetdash{}{0pt}%
\pgfpathmoveto{\pgfqpoint{1.525641in}{0.479564in}}%
\pgfpathlineto{\pgfqpoint{1.544973in}{0.479564in}}%
\pgfpathlineto{\pgfqpoint{1.544973in}{0.522171in}}%
\pgfpathlineto{\pgfqpoint{1.525641in}{0.522171in}}%
\pgfpathlineto{\pgfqpoint{1.525641in}{0.479564in}}%
\pgfusepath{stroke,fill}%
\end{pgfscope}%
\begin{pgfscope}%
\pgfpathrectangle{\pgfqpoint{0.636356in}{0.440955in}}{\pgfqpoint{1.933229in}{0.162432in}} %
\pgfusepath{clip}%
\pgfsetbuttcap%
\pgfsetmiterjoin%
\definecolor{currentfill}{rgb}{0.333333,0.333333,0.333333}%
\pgfsetfillcolor{currentfill}%
\pgfsetlinewidth{0.501875pt}%
\definecolor{currentstroke}{rgb}{0.000000,0.000000,0.000000}%
\pgfsetstrokecolor{currentstroke}%
\pgfsetdash{}{0pt}%
\pgfpathmoveto{\pgfqpoint{1.544973in}{0.488001in}}%
\pgfpathlineto{\pgfqpoint{1.564306in}{0.488001in}}%
\pgfpathlineto{\pgfqpoint{1.564306in}{0.522171in}}%
\pgfpathlineto{\pgfqpoint{1.544973in}{0.522171in}}%
\pgfpathlineto{\pgfqpoint{1.544973in}{0.488001in}}%
\pgfusepath{stroke,fill}%
\end{pgfscope}%
\begin{pgfscope}%
\pgfpathrectangle{\pgfqpoint{0.636356in}{0.440955in}}{\pgfqpoint{1.933229in}{0.162432in}} %
\pgfusepath{clip}%
\pgfsetbuttcap%
\pgfsetmiterjoin%
\definecolor{currentfill}{rgb}{0.333333,0.333333,0.333333}%
\pgfsetfillcolor{currentfill}%
\pgfsetlinewidth{0.501875pt}%
\definecolor{currentstroke}{rgb}{0.000000,0.000000,0.000000}%
\pgfsetstrokecolor{currentstroke}%
\pgfsetdash{}{0pt}%
\pgfpathmoveto{\pgfqpoint{1.564306in}{0.521987in}}%
\pgfpathlineto{\pgfqpoint{1.583638in}{0.521987in}}%
\pgfpathlineto{\pgfqpoint{1.583638in}{0.522171in}}%
\pgfpathlineto{\pgfqpoint{1.564306in}{0.522171in}}%
\pgfpathlineto{\pgfqpoint{1.564306in}{0.521987in}}%
\pgfusepath{stroke,fill}%
\end{pgfscope}%
\begin{pgfscope}%
\pgfpathrectangle{\pgfqpoint{0.636356in}{0.440955in}}{\pgfqpoint{1.933229in}{0.162432in}} %
\pgfusepath{clip}%
\pgfsetbuttcap%
\pgfsetmiterjoin%
\definecolor{currentfill}{rgb}{0.333333,0.333333,0.333333}%
\pgfsetfillcolor{currentfill}%
\pgfsetlinewidth{0.501875pt}%
\definecolor{currentstroke}{rgb}{0.000000,0.000000,0.000000}%
\pgfsetstrokecolor{currentstroke}%
\pgfsetdash{}{0pt}%
\pgfpathmoveto{\pgfqpoint{1.583638in}{0.516871in}}%
\pgfpathlineto{\pgfqpoint{1.602970in}{0.516871in}}%
\pgfpathlineto{\pgfqpoint{1.602970in}{0.522171in}}%
\pgfpathlineto{\pgfqpoint{1.583638in}{0.522171in}}%
\pgfpathlineto{\pgfqpoint{1.583638in}{0.516871in}}%
\pgfusepath{stroke,fill}%
\end{pgfscope}%
\begin{pgfscope}%
\pgfpathrectangle{\pgfqpoint{0.636356in}{0.440955in}}{\pgfqpoint{1.933229in}{0.162432in}} %
\pgfusepath{clip}%
\pgfsetbuttcap%
\pgfsetmiterjoin%
\definecolor{currentfill}{rgb}{0.333333,0.333333,0.333333}%
\pgfsetfillcolor{currentfill}%
\pgfsetlinewidth{0.501875pt}%
\definecolor{currentstroke}{rgb}{0.000000,0.000000,0.000000}%
\pgfsetstrokecolor{currentstroke}%
\pgfsetdash{}{0pt}%
\pgfpathmoveto{\pgfqpoint{1.602970in}{0.522171in}}%
\pgfpathlineto{\pgfqpoint{1.622302in}{0.522171in}}%
\pgfpathlineto{\pgfqpoint{1.622302in}{0.528100in}}%
\pgfpathlineto{\pgfqpoint{1.602970in}{0.528100in}}%
\pgfpathlineto{\pgfqpoint{1.602970in}{0.522171in}}%
\pgfusepath{stroke,fill}%
\end{pgfscope}%
\begin{pgfscope}%
\pgfpathrectangle{\pgfqpoint{0.636356in}{0.440955in}}{\pgfqpoint{1.933229in}{0.162432in}} %
\pgfusepath{clip}%
\pgfsetbuttcap%
\pgfsetmiterjoin%
\definecolor{currentfill}{rgb}{0.333333,0.333333,0.333333}%
\pgfsetfillcolor{currentfill}%
\pgfsetlinewidth{0.501875pt}%
\definecolor{currentstroke}{rgb}{0.000000,0.000000,0.000000}%
\pgfsetstrokecolor{currentstroke}%
\pgfsetdash{}{0pt}%
\pgfpathmoveto{\pgfqpoint{1.622302in}{0.522171in}}%
\pgfpathlineto{\pgfqpoint{1.641635in}{0.522171in}}%
\pgfpathlineto{\pgfqpoint{1.641635in}{0.531427in}}%
\pgfpathlineto{\pgfqpoint{1.622302in}{0.531427in}}%
\pgfpathlineto{\pgfqpoint{1.622302in}{0.522171in}}%
\pgfusepath{stroke,fill}%
\end{pgfscope}%
\begin{pgfscope}%
\pgfpathrectangle{\pgfqpoint{0.636356in}{0.440955in}}{\pgfqpoint{1.933229in}{0.162432in}} %
\pgfusepath{clip}%
\pgfsetbuttcap%
\pgfsetmiterjoin%
\definecolor{currentfill}{rgb}{0.333333,0.333333,0.333333}%
\pgfsetfillcolor{currentfill}%
\pgfsetlinewidth{0.501875pt}%
\definecolor{currentstroke}{rgb}{0.000000,0.000000,0.000000}%
\pgfsetstrokecolor{currentstroke}%
\pgfsetdash{}{0pt}%
\pgfpathmoveto{\pgfqpoint{1.641635in}{0.522171in}}%
\pgfpathlineto{\pgfqpoint{1.660967in}{0.522171in}}%
\pgfpathlineto{\pgfqpoint{1.660967in}{0.538048in}}%
\pgfpathlineto{\pgfqpoint{1.641635in}{0.538048in}}%
\pgfpathlineto{\pgfqpoint{1.641635in}{0.522171in}}%
\pgfusepath{stroke,fill}%
\end{pgfscope}%
\begin{pgfscope}%
\pgfpathrectangle{\pgfqpoint{0.636356in}{0.440955in}}{\pgfqpoint{1.933229in}{0.162432in}} %
\pgfusepath{clip}%
\pgfsetbuttcap%
\pgfsetmiterjoin%
\definecolor{currentfill}{rgb}{0.333333,0.333333,0.333333}%
\pgfsetfillcolor{currentfill}%
\pgfsetlinewidth{0.501875pt}%
\definecolor{currentstroke}{rgb}{0.000000,0.000000,0.000000}%
\pgfsetstrokecolor{currentstroke}%
\pgfsetdash{}{0pt}%
\pgfpathmoveto{\pgfqpoint{1.660967in}{0.514565in}}%
\pgfpathlineto{\pgfqpoint{1.680299in}{0.514565in}}%
\pgfpathlineto{\pgfqpoint{1.680299in}{0.522171in}}%
\pgfpathlineto{\pgfqpoint{1.660967in}{0.522171in}}%
\pgfpathlineto{\pgfqpoint{1.660967in}{0.514565in}}%
\pgfusepath{stroke,fill}%
\end{pgfscope}%
\begin{pgfscope}%
\pgfpathrectangle{\pgfqpoint{0.636356in}{0.440955in}}{\pgfqpoint{1.933229in}{0.162432in}} %
\pgfusepath{clip}%
\pgfsetbuttcap%
\pgfsetmiterjoin%
\definecolor{currentfill}{rgb}{0.333333,0.333333,0.333333}%
\pgfsetfillcolor{currentfill}%
\pgfsetlinewidth{0.501875pt}%
\definecolor{currentstroke}{rgb}{0.000000,0.000000,0.000000}%
\pgfsetstrokecolor{currentstroke}%
\pgfsetdash{}{0pt}%
\pgfpathmoveto{\pgfqpoint{1.680299in}{0.522171in}}%
\pgfpathlineto{\pgfqpoint{1.699632in}{0.522171in}}%
\pgfpathlineto{\pgfqpoint{1.699632in}{0.549975in}}%
\pgfpathlineto{\pgfqpoint{1.680299in}{0.549975in}}%
\pgfpathlineto{\pgfqpoint{1.680299in}{0.522171in}}%
\pgfusepath{stroke,fill}%
\end{pgfscope}%
\begin{pgfscope}%
\pgfpathrectangle{\pgfqpoint{0.636356in}{0.440955in}}{\pgfqpoint{1.933229in}{0.162432in}} %
\pgfusepath{clip}%
\pgfsetbuttcap%
\pgfsetmiterjoin%
\definecolor{currentfill}{rgb}{0.333333,0.333333,0.333333}%
\pgfsetfillcolor{currentfill}%
\pgfsetlinewidth{0.501875pt}%
\definecolor{currentstroke}{rgb}{0.000000,0.000000,0.000000}%
\pgfsetstrokecolor{currentstroke}%
\pgfsetdash{}{0pt}%
\pgfpathmoveto{\pgfqpoint{1.699632in}{0.520694in}}%
\pgfpathlineto{\pgfqpoint{1.718964in}{0.520694in}}%
\pgfpathlineto{\pgfqpoint{1.718964in}{0.522171in}}%
\pgfpathlineto{\pgfqpoint{1.699632in}{0.522171in}}%
\pgfpathlineto{\pgfqpoint{1.699632in}{0.520694in}}%
\pgfusepath{stroke,fill}%
\end{pgfscope}%
\begin{pgfscope}%
\pgfpathrectangle{\pgfqpoint{0.636356in}{0.440955in}}{\pgfqpoint{1.933229in}{0.162432in}} %
\pgfusepath{clip}%
\pgfsetbuttcap%
\pgfsetmiterjoin%
\definecolor{currentfill}{rgb}{0.333333,0.333333,0.333333}%
\pgfsetfillcolor{currentfill}%
\pgfsetlinewidth{0.501875pt}%
\definecolor{currentstroke}{rgb}{0.000000,0.000000,0.000000}%
\pgfsetstrokecolor{currentstroke}%
\pgfsetdash{}{0pt}%
\pgfpathmoveto{\pgfqpoint{1.718964in}{0.522171in}}%
\pgfpathlineto{\pgfqpoint{1.738296in}{0.522171in}}%
\pgfpathlineto{\pgfqpoint{1.738296in}{0.566211in}}%
\pgfpathlineto{\pgfqpoint{1.718964in}{0.566211in}}%
\pgfpathlineto{\pgfqpoint{1.718964in}{0.522171in}}%
\pgfusepath{stroke,fill}%
\end{pgfscope}%
\begin{pgfscope}%
\pgfpathrectangle{\pgfqpoint{0.636356in}{0.440955in}}{\pgfqpoint{1.933229in}{0.162432in}} %
\pgfusepath{clip}%
\pgfsetbuttcap%
\pgfsetmiterjoin%
\definecolor{currentfill}{rgb}{0.333333,0.333333,0.333333}%
\pgfsetfillcolor{currentfill}%
\pgfsetlinewidth{0.501875pt}%
\definecolor{currentstroke}{rgb}{0.000000,0.000000,0.000000}%
\pgfsetstrokecolor{currentstroke}%
\pgfsetdash{}{0pt}%
\pgfpathmoveto{\pgfqpoint{1.738296in}{0.522171in}}%
\pgfpathlineto{\pgfqpoint{1.757628in}{0.522171in}}%
\pgfpathlineto{\pgfqpoint{1.757628in}{0.583507in}}%
\pgfpathlineto{\pgfqpoint{1.738296in}{0.583507in}}%
\pgfpathlineto{\pgfqpoint{1.738296in}{0.522171in}}%
\pgfusepath{stroke,fill}%
\end{pgfscope}%
\begin{pgfscope}%
\pgfpathrectangle{\pgfqpoint{0.636356in}{0.440955in}}{\pgfqpoint{1.933229in}{0.162432in}} %
\pgfusepath{clip}%
\pgfsetbuttcap%
\pgfsetmiterjoin%
\definecolor{currentfill}{rgb}{0.333333,0.333333,0.333333}%
\pgfsetfillcolor{currentfill}%
\pgfsetlinewidth{0.501875pt}%
\definecolor{currentstroke}{rgb}{0.000000,0.000000,0.000000}%
\pgfsetstrokecolor{currentstroke}%
\pgfsetdash{}{0pt}%
\pgfpathmoveto{\pgfqpoint{1.757628in}{0.522171in}}%
\pgfpathlineto{\pgfqpoint{1.776961in}{0.522171in}}%
\pgfpathlineto{\pgfqpoint{1.776961in}{0.548638in}}%
\pgfpathlineto{\pgfqpoint{1.757628in}{0.548638in}}%
\pgfpathlineto{\pgfqpoint{1.757628in}{0.522171in}}%
\pgfusepath{stroke,fill}%
\end{pgfscope}%
\begin{pgfscope}%
\pgfpathrectangle{\pgfqpoint{0.636356in}{0.440955in}}{\pgfqpoint{1.933229in}{0.162432in}} %
\pgfusepath{clip}%
\pgfsetbuttcap%
\pgfsetmiterjoin%
\definecolor{currentfill}{rgb}{0.333333,0.333333,0.333333}%
\pgfsetfillcolor{currentfill}%
\pgfsetlinewidth{0.501875pt}%
\definecolor{currentstroke}{rgb}{0.000000,0.000000,0.000000}%
\pgfsetstrokecolor{currentstroke}%
\pgfsetdash{}{0pt}%
\pgfpathmoveto{\pgfqpoint{1.776961in}{0.507490in}}%
\pgfpathlineto{\pgfqpoint{1.796293in}{0.507490in}}%
\pgfpathlineto{\pgfqpoint{1.796293in}{0.522171in}}%
\pgfpathlineto{\pgfqpoint{1.776961in}{0.522171in}}%
\pgfpathlineto{\pgfqpoint{1.776961in}{0.507490in}}%
\pgfusepath{stroke,fill}%
\end{pgfscope}%
\begin{pgfscope}%
\pgfpathrectangle{\pgfqpoint{0.636356in}{0.440955in}}{\pgfqpoint{1.933229in}{0.162432in}} %
\pgfusepath{clip}%
\pgfsetbuttcap%
\pgfsetmiterjoin%
\definecolor{currentfill}{rgb}{0.333333,0.333333,0.333333}%
\pgfsetfillcolor{currentfill}%
\pgfsetlinewidth{0.501875pt}%
\definecolor{currentstroke}{rgb}{0.000000,0.000000,0.000000}%
\pgfsetstrokecolor{currentstroke}%
\pgfsetdash{}{0pt}%
\pgfpathmoveto{\pgfqpoint{1.796293in}{0.509426in}}%
\pgfpathlineto{\pgfqpoint{1.815625in}{0.509426in}}%
\pgfpathlineto{\pgfqpoint{1.815625in}{0.522171in}}%
\pgfpathlineto{\pgfqpoint{1.796293in}{0.522171in}}%
\pgfpathlineto{\pgfqpoint{1.796293in}{0.509426in}}%
\pgfusepath{stroke,fill}%
\end{pgfscope}%
\begin{pgfscope}%
\pgfpathrectangle{\pgfqpoint{0.636356in}{0.440955in}}{\pgfqpoint{1.933229in}{0.162432in}} %
\pgfusepath{clip}%
\pgfsetbuttcap%
\pgfsetmiterjoin%
\definecolor{currentfill}{rgb}{0.333333,0.333333,0.333333}%
\pgfsetfillcolor{currentfill}%
\pgfsetlinewidth{0.501875pt}%
\definecolor{currentstroke}{rgb}{0.000000,0.000000,0.000000}%
\pgfsetstrokecolor{currentstroke}%
\pgfsetdash{}{0pt}%
\pgfpathmoveto{\pgfqpoint{1.815625in}{0.522171in}}%
\pgfpathlineto{\pgfqpoint{1.834958in}{0.522171in}}%
\pgfpathlineto{\pgfqpoint{1.834958in}{0.563128in}}%
\pgfpathlineto{\pgfqpoint{1.815625in}{0.563128in}}%
\pgfpathlineto{\pgfqpoint{1.815625in}{0.522171in}}%
\pgfusepath{stroke,fill}%
\end{pgfscope}%
\begin{pgfscope}%
\pgfpathrectangle{\pgfqpoint{0.636356in}{0.440955in}}{\pgfqpoint{1.933229in}{0.162432in}} %
\pgfusepath{clip}%
\pgfsetbuttcap%
\pgfsetmiterjoin%
\definecolor{currentfill}{rgb}{0.333333,0.333333,0.333333}%
\pgfsetfillcolor{currentfill}%
\pgfsetlinewidth{0.501875pt}%
\definecolor{currentstroke}{rgb}{0.000000,0.000000,0.000000}%
\pgfsetstrokecolor{currentstroke}%
\pgfsetdash{}{0pt}%
\pgfpathmoveto{\pgfqpoint{1.834958in}{0.504066in}}%
\pgfpathlineto{\pgfqpoint{1.854290in}{0.504066in}}%
\pgfpathlineto{\pgfqpoint{1.854290in}{0.522171in}}%
\pgfpathlineto{\pgfqpoint{1.834958in}{0.522171in}}%
\pgfpathlineto{\pgfqpoint{1.834958in}{0.504066in}}%
\pgfusepath{stroke,fill}%
\end{pgfscope}%
\begin{pgfscope}%
\pgfpathrectangle{\pgfqpoint{0.636356in}{0.440955in}}{\pgfqpoint{1.933229in}{0.162432in}} %
\pgfusepath{clip}%
\pgfsetbuttcap%
\pgfsetmiterjoin%
\definecolor{currentfill}{rgb}{0.333333,0.333333,0.333333}%
\pgfsetfillcolor{currentfill}%
\pgfsetlinewidth{0.501875pt}%
\definecolor{currentstroke}{rgb}{0.000000,0.000000,0.000000}%
\pgfsetstrokecolor{currentstroke}%
\pgfsetdash{}{0pt}%
\pgfpathmoveto{\pgfqpoint{1.854290in}{0.522171in}}%
\pgfpathlineto{\pgfqpoint{1.873622in}{0.522171in}}%
\pgfpathlineto{\pgfqpoint{1.873622in}{0.526905in}}%
\pgfpathlineto{\pgfqpoint{1.854290in}{0.526905in}}%
\pgfpathlineto{\pgfqpoint{1.854290in}{0.522171in}}%
\pgfusepath{stroke,fill}%
\end{pgfscope}%
\begin{pgfscope}%
\pgfpathrectangle{\pgfqpoint{0.636356in}{0.440955in}}{\pgfqpoint{1.933229in}{0.162432in}} %
\pgfusepath{clip}%
\pgfsetbuttcap%
\pgfsetmiterjoin%
\definecolor{currentfill}{rgb}{0.333333,0.333333,0.333333}%
\pgfsetfillcolor{currentfill}%
\pgfsetlinewidth{0.501875pt}%
\definecolor{currentstroke}{rgb}{0.000000,0.000000,0.000000}%
\pgfsetstrokecolor{currentstroke}%
\pgfsetdash{}{0pt}%
\pgfpathmoveto{\pgfqpoint{1.873622in}{0.522171in}}%
\pgfpathlineto{\pgfqpoint{1.892954in}{0.522171in}}%
\pgfpathlineto{\pgfqpoint{1.892954in}{0.563788in}}%
\pgfpathlineto{\pgfqpoint{1.873622in}{0.563788in}}%
\pgfpathlineto{\pgfqpoint{1.873622in}{0.522171in}}%
\pgfusepath{stroke,fill}%
\end{pgfscope}%
\begin{pgfscope}%
\pgfpathrectangle{\pgfqpoint{0.636356in}{0.440955in}}{\pgfqpoint{1.933229in}{0.162432in}} %
\pgfusepath{clip}%
\pgfsetbuttcap%
\pgfsetmiterjoin%
\definecolor{currentfill}{rgb}{0.333333,0.333333,0.333333}%
\pgfsetfillcolor{currentfill}%
\pgfsetlinewidth{0.501875pt}%
\definecolor{currentstroke}{rgb}{0.000000,0.000000,0.000000}%
\pgfsetstrokecolor{currentstroke}%
\pgfsetdash{}{0pt}%
\pgfpathmoveto{\pgfqpoint{1.892954in}{0.522171in}}%
\pgfpathlineto{\pgfqpoint{1.912287in}{0.522171in}}%
\pgfpathlineto{\pgfqpoint{1.912287in}{0.527017in}}%
\pgfpathlineto{\pgfqpoint{1.892954in}{0.527017in}}%
\pgfpathlineto{\pgfqpoint{1.892954in}{0.522171in}}%
\pgfusepath{stroke,fill}%
\end{pgfscope}%
\begin{pgfscope}%
\pgfpathrectangle{\pgfqpoint{0.636356in}{0.440955in}}{\pgfqpoint{1.933229in}{0.162432in}} %
\pgfusepath{clip}%
\pgfsetbuttcap%
\pgfsetmiterjoin%
\definecolor{currentfill}{rgb}{0.333333,0.333333,0.333333}%
\pgfsetfillcolor{currentfill}%
\pgfsetlinewidth{0.501875pt}%
\definecolor{currentstroke}{rgb}{0.000000,0.000000,0.000000}%
\pgfsetstrokecolor{currentstroke}%
\pgfsetdash{}{0pt}%
\pgfpathmoveto{\pgfqpoint{1.912287in}{0.518681in}}%
\pgfpathlineto{\pgfqpoint{1.931619in}{0.518681in}}%
\pgfpathlineto{\pgfqpoint{1.931619in}{0.522171in}}%
\pgfpathlineto{\pgfqpoint{1.912287in}{0.522171in}}%
\pgfpathlineto{\pgfqpoint{1.912287in}{0.518681in}}%
\pgfusepath{stroke,fill}%
\end{pgfscope}%
\begin{pgfscope}%
\pgfpathrectangle{\pgfqpoint{0.636356in}{0.440955in}}{\pgfqpoint{1.933229in}{0.162432in}} %
\pgfusepath{clip}%
\pgfsetbuttcap%
\pgfsetmiterjoin%
\definecolor{currentfill}{rgb}{0.333333,0.333333,0.333333}%
\pgfsetfillcolor{currentfill}%
\pgfsetlinewidth{0.501875pt}%
\definecolor{currentstroke}{rgb}{0.000000,0.000000,0.000000}%
\pgfsetstrokecolor{currentstroke}%
\pgfsetdash{}{0pt}%
\pgfpathmoveto{\pgfqpoint{1.931619in}{0.522171in}}%
\pgfpathlineto{\pgfqpoint{1.950951in}{0.522171in}}%
\pgfpathlineto{\pgfqpoint{1.950951in}{0.570076in}}%
\pgfpathlineto{\pgfqpoint{1.931619in}{0.570076in}}%
\pgfpathlineto{\pgfqpoint{1.931619in}{0.522171in}}%
\pgfusepath{stroke,fill}%
\end{pgfscope}%
\begin{pgfscope}%
\pgfpathrectangle{\pgfqpoint{0.636356in}{0.440955in}}{\pgfqpoint{1.933229in}{0.162432in}} %
\pgfusepath{clip}%
\pgfsetbuttcap%
\pgfsetmiterjoin%
\definecolor{currentfill}{rgb}{0.333333,0.333333,0.333333}%
\pgfsetfillcolor{currentfill}%
\pgfsetlinewidth{0.501875pt}%
\definecolor{currentstroke}{rgb}{0.000000,0.000000,0.000000}%
\pgfsetstrokecolor{currentstroke}%
\pgfsetdash{}{0pt}%
\pgfpathmoveto{\pgfqpoint{1.950951in}{0.522171in}}%
\pgfpathlineto{\pgfqpoint{1.970284in}{0.522171in}}%
\pgfpathlineto{\pgfqpoint{1.970284in}{0.542711in}}%
\pgfpathlineto{\pgfqpoint{1.950951in}{0.542711in}}%
\pgfpathlineto{\pgfqpoint{1.950951in}{0.522171in}}%
\pgfusepath{stroke,fill}%
\end{pgfscope}%
\begin{pgfscope}%
\pgfpathrectangle{\pgfqpoint{0.636356in}{0.440955in}}{\pgfqpoint{1.933229in}{0.162432in}} %
\pgfusepath{clip}%
\pgfsetbuttcap%
\pgfsetmiterjoin%
\definecolor{currentfill}{rgb}{0.333333,0.333333,0.333333}%
\pgfsetfillcolor{currentfill}%
\pgfsetlinewidth{0.501875pt}%
\definecolor{currentstroke}{rgb}{0.000000,0.000000,0.000000}%
\pgfsetstrokecolor{currentstroke}%
\pgfsetdash{}{0pt}%
\pgfpathmoveto{\pgfqpoint{1.970284in}{0.510332in}}%
\pgfpathlineto{\pgfqpoint{1.989616in}{0.510332in}}%
\pgfpathlineto{\pgfqpoint{1.989616in}{0.522171in}}%
\pgfpathlineto{\pgfqpoint{1.970284in}{0.522171in}}%
\pgfpathlineto{\pgfqpoint{1.970284in}{0.510332in}}%
\pgfusepath{stroke,fill}%
\end{pgfscope}%
\begin{pgfscope}%
\pgfpathrectangle{\pgfqpoint{0.636356in}{0.440955in}}{\pgfqpoint{1.933229in}{0.162432in}} %
\pgfusepath{clip}%
\pgfsetbuttcap%
\pgfsetmiterjoin%
\definecolor{currentfill}{rgb}{0.333333,0.333333,0.333333}%
\pgfsetfillcolor{currentfill}%
\pgfsetlinewidth{0.501875pt}%
\definecolor{currentstroke}{rgb}{0.000000,0.000000,0.000000}%
\pgfsetstrokecolor{currentstroke}%
\pgfsetdash{}{0pt}%
\pgfpathmoveto{\pgfqpoint{1.989616in}{0.522171in}}%
\pgfpathlineto{\pgfqpoint{2.008948in}{0.522171in}}%
\pgfpathlineto{\pgfqpoint{2.008948in}{0.522192in}}%
\pgfpathlineto{\pgfqpoint{1.989616in}{0.522192in}}%
\pgfpathlineto{\pgfqpoint{1.989616in}{0.522171in}}%
\pgfusepath{stroke,fill}%
\end{pgfscope}%
\begin{pgfscope}%
\pgfpathrectangle{\pgfqpoint{0.636356in}{0.440955in}}{\pgfqpoint{1.933229in}{0.162432in}} %
\pgfusepath{clip}%
\pgfsetbuttcap%
\pgfsetmiterjoin%
\definecolor{currentfill}{rgb}{0.333333,0.333333,0.333333}%
\pgfsetfillcolor{currentfill}%
\pgfsetlinewidth{0.501875pt}%
\definecolor{currentstroke}{rgb}{0.000000,0.000000,0.000000}%
\pgfsetstrokecolor{currentstroke}%
\pgfsetdash{}{0pt}%
\pgfpathmoveto{\pgfqpoint{2.008948in}{0.508390in}}%
\pgfpathlineto{\pgfqpoint{2.028280in}{0.508390in}}%
\pgfpathlineto{\pgfqpoint{2.028280in}{0.522171in}}%
\pgfpathlineto{\pgfqpoint{2.008948in}{0.522171in}}%
\pgfpathlineto{\pgfqpoint{2.008948in}{0.508390in}}%
\pgfusepath{stroke,fill}%
\end{pgfscope}%
\begin{pgfscope}%
\pgfpathrectangle{\pgfqpoint{0.636356in}{0.440955in}}{\pgfqpoint{1.933229in}{0.162432in}} %
\pgfusepath{clip}%
\pgfsetbuttcap%
\pgfsetmiterjoin%
\definecolor{currentfill}{rgb}{0.333333,0.333333,0.333333}%
\pgfsetfillcolor{currentfill}%
\pgfsetlinewidth{0.501875pt}%
\definecolor{currentstroke}{rgb}{0.000000,0.000000,0.000000}%
\pgfsetstrokecolor{currentstroke}%
\pgfsetdash{}{0pt}%
\pgfpathmoveto{\pgfqpoint{2.028280in}{0.522171in}}%
\pgfpathlineto{\pgfqpoint{2.047613in}{0.522171in}}%
\pgfpathlineto{\pgfqpoint{2.047613in}{0.540403in}}%
\pgfpathlineto{\pgfqpoint{2.028280in}{0.540403in}}%
\pgfpathlineto{\pgfqpoint{2.028280in}{0.522171in}}%
\pgfusepath{stroke,fill}%
\end{pgfscope}%
\begin{pgfscope}%
\pgfpathrectangle{\pgfqpoint{0.636356in}{0.440955in}}{\pgfqpoint{1.933229in}{0.162432in}} %
\pgfusepath{clip}%
\pgfsetbuttcap%
\pgfsetmiterjoin%
\definecolor{currentfill}{rgb}{0.333333,0.333333,0.333333}%
\pgfsetfillcolor{currentfill}%
\pgfsetlinewidth{0.501875pt}%
\definecolor{currentstroke}{rgb}{0.000000,0.000000,0.000000}%
\pgfsetstrokecolor{currentstroke}%
\pgfsetdash{}{0pt}%
\pgfpathmoveto{\pgfqpoint{2.047613in}{0.507437in}}%
\pgfpathlineto{\pgfqpoint{2.066945in}{0.507437in}}%
\pgfpathlineto{\pgfqpoint{2.066945in}{0.522171in}}%
\pgfpathlineto{\pgfqpoint{2.047613in}{0.522171in}}%
\pgfpathlineto{\pgfqpoint{2.047613in}{0.507437in}}%
\pgfusepath{stroke,fill}%
\end{pgfscope}%
\begin{pgfscope}%
\pgfpathrectangle{\pgfqpoint{0.636356in}{0.440955in}}{\pgfqpoint{1.933229in}{0.162432in}} %
\pgfusepath{clip}%
\pgfsetbuttcap%
\pgfsetmiterjoin%
\definecolor{currentfill}{rgb}{0.333333,0.333333,0.333333}%
\pgfsetfillcolor{currentfill}%
\pgfsetlinewidth{0.501875pt}%
\definecolor{currentstroke}{rgb}{0.000000,0.000000,0.000000}%
\pgfsetstrokecolor{currentstroke}%
\pgfsetdash{}{0pt}%
\pgfpathmoveto{\pgfqpoint{2.066945in}{0.513903in}}%
\pgfpathlineto{\pgfqpoint{2.086277in}{0.513903in}}%
\pgfpathlineto{\pgfqpoint{2.086277in}{0.522171in}}%
\pgfpathlineto{\pgfqpoint{2.066945in}{0.522171in}}%
\pgfpathlineto{\pgfqpoint{2.066945in}{0.513903in}}%
\pgfusepath{stroke,fill}%
\end{pgfscope}%
\begin{pgfscope}%
\pgfpathrectangle{\pgfqpoint{0.636356in}{0.440955in}}{\pgfqpoint{1.933229in}{0.162432in}} %
\pgfusepath{clip}%
\pgfsetbuttcap%
\pgfsetmiterjoin%
\definecolor{currentfill}{rgb}{0.333333,0.333333,0.333333}%
\pgfsetfillcolor{currentfill}%
\pgfsetlinewidth{0.501875pt}%
\definecolor{currentstroke}{rgb}{0.000000,0.000000,0.000000}%
\pgfsetstrokecolor{currentstroke}%
\pgfsetdash{}{0pt}%
\pgfpathmoveto{\pgfqpoint{2.086277in}{0.522171in}}%
\pgfpathlineto{\pgfqpoint{2.105610in}{0.522171in}}%
\pgfpathlineto{\pgfqpoint{2.105610in}{0.540513in}}%
\pgfpathlineto{\pgfqpoint{2.086277in}{0.540513in}}%
\pgfpathlineto{\pgfqpoint{2.086277in}{0.522171in}}%
\pgfusepath{stroke,fill}%
\end{pgfscope}%
\begin{pgfscope}%
\pgfpathrectangle{\pgfqpoint{0.636356in}{0.440955in}}{\pgfqpoint{1.933229in}{0.162432in}} %
\pgfusepath{clip}%
\pgfsetbuttcap%
\pgfsetmiterjoin%
\definecolor{currentfill}{rgb}{0.333333,0.333333,0.333333}%
\pgfsetfillcolor{currentfill}%
\pgfsetlinewidth{0.501875pt}%
\definecolor{currentstroke}{rgb}{0.000000,0.000000,0.000000}%
\pgfsetstrokecolor{currentstroke}%
\pgfsetdash{}{0pt}%
\pgfpathmoveto{\pgfqpoint{2.105610in}{0.505513in}}%
\pgfpathlineto{\pgfqpoint{2.124942in}{0.505513in}}%
\pgfpathlineto{\pgfqpoint{2.124942in}{0.522171in}}%
\pgfpathlineto{\pgfqpoint{2.105610in}{0.522171in}}%
\pgfpathlineto{\pgfqpoint{2.105610in}{0.505513in}}%
\pgfusepath{stroke,fill}%
\end{pgfscope}%
\begin{pgfscope}%
\pgfpathrectangle{\pgfqpoint{0.636356in}{0.440955in}}{\pgfqpoint{1.933229in}{0.162432in}} %
\pgfusepath{clip}%
\pgfsetbuttcap%
\pgfsetmiterjoin%
\definecolor{currentfill}{rgb}{0.333333,0.333333,0.333333}%
\pgfsetfillcolor{currentfill}%
\pgfsetlinewidth{0.501875pt}%
\definecolor{currentstroke}{rgb}{0.000000,0.000000,0.000000}%
\pgfsetstrokecolor{currentstroke}%
\pgfsetdash{}{0pt}%
\pgfpathmoveto{\pgfqpoint{2.124942in}{0.459760in}}%
\pgfpathlineto{\pgfqpoint{2.144274in}{0.459760in}}%
\pgfpathlineto{\pgfqpoint{2.144274in}{0.522171in}}%
\pgfpathlineto{\pgfqpoint{2.124942in}{0.522171in}}%
\pgfpathlineto{\pgfqpoint{2.124942in}{0.459760in}}%
\pgfusepath{stroke,fill}%
\end{pgfscope}%
\begin{pgfscope}%
\pgfpathrectangle{\pgfqpoint{0.636356in}{0.440955in}}{\pgfqpoint{1.933229in}{0.162432in}} %
\pgfusepath{clip}%
\pgfsetbuttcap%
\pgfsetmiterjoin%
\definecolor{currentfill}{rgb}{0.333333,0.333333,0.333333}%
\pgfsetfillcolor{currentfill}%
\pgfsetlinewidth{0.501875pt}%
\definecolor{currentstroke}{rgb}{0.000000,0.000000,0.000000}%
\pgfsetstrokecolor{currentstroke}%
\pgfsetdash{}{0pt}%
\pgfpathmoveto{\pgfqpoint{2.144274in}{0.522171in}}%
\pgfpathlineto{\pgfqpoint{2.163606in}{0.522171in}}%
\pgfpathlineto{\pgfqpoint{2.163606in}{0.560936in}}%
\pgfpathlineto{\pgfqpoint{2.144274in}{0.560936in}}%
\pgfpathlineto{\pgfqpoint{2.144274in}{0.522171in}}%
\pgfusepath{stroke,fill}%
\end{pgfscope}%
\begin{pgfscope}%
\pgfpathrectangle{\pgfqpoint{0.636356in}{0.440955in}}{\pgfqpoint{1.933229in}{0.162432in}} %
\pgfusepath{clip}%
\pgfsetbuttcap%
\pgfsetmiterjoin%
\definecolor{currentfill}{rgb}{0.333333,0.333333,0.333333}%
\pgfsetfillcolor{currentfill}%
\pgfsetlinewidth{0.501875pt}%
\definecolor{currentstroke}{rgb}{0.000000,0.000000,0.000000}%
\pgfsetstrokecolor{currentstroke}%
\pgfsetdash{}{0pt}%
\pgfpathmoveto{\pgfqpoint{2.163606in}{0.499754in}}%
\pgfpathlineto{\pgfqpoint{2.182939in}{0.499754in}}%
\pgfpathlineto{\pgfqpoint{2.182939in}{0.522171in}}%
\pgfpathlineto{\pgfqpoint{2.163606in}{0.522171in}}%
\pgfpathlineto{\pgfqpoint{2.163606in}{0.499754in}}%
\pgfusepath{stroke,fill}%
\end{pgfscope}%
\begin{pgfscope}%
\pgfpathrectangle{\pgfqpoint{0.636356in}{0.440955in}}{\pgfqpoint{1.933229in}{0.162432in}} %
\pgfusepath{clip}%
\pgfsetbuttcap%
\pgfsetmiterjoin%
\definecolor{currentfill}{rgb}{0.333333,0.333333,0.333333}%
\pgfsetfillcolor{currentfill}%
\pgfsetlinewidth{0.501875pt}%
\definecolor{currentstroke}{rgb}{0.000000,0.000000,0.000000}%
\pgfsetstrokecolor{currentstroke}%
\pgfsetdash{}{0pt}%
\pgfpathmoveto{\pgfqpoint{2.182939in}{0.499243in}}%
\pgfpathlineto{\pgfqpoint{2.202271in}{0.499243in}}%
\pgfpathlineto{\pgfqpoint{2.202271in}{0.522171in}}%
\pgfpathlineto{\pgfqpoint{2.182939in}{0.522171in}}%
\pgfpathlineto{\pgfqpoint{2.182939in}{0.499243in}}%
\pgfusepath{stroke,fill}%
\end{pgfscope}%
\begin{pgfscope}%
\pgfpathrectangle{\pgfqpoint{0.636356in}{0.440955in}}{\pgfqpoint{1.933229in}{0.162432in}} %
\pgfusepath{clip}%
\pgfsetbuttcap%
\pgfsetmiterjoin%
\definecolor{currentfill}{rgb}{0.333333,0.333333,0.333333}%
\pgfsetfillcolor{currentfill}%
\pgfsetlinewidth{0.501875pt}%
\definecolor{currentstroke}{rgb}{0.000000,0.000000,0.000000}%
\pgfsetstrokecolor{currentstroke}%
\pgfsetdash{}{0pt}%
\pgfpathmoveto{\pgfqpoint{2.202271in}{0.504065in}}%
\pgfpathlineto{\pgfqpoint{2.221603in}{0.504065in}}%
\pgfpathlineto{\pgfqpoint{2.221603in}{0.522171in}}%
\pgfpathlineto{\pgfqpoint{2.202271in}{0.522171in}}%
\pgfpathlineto{\pgfqpoint{2.202271in}{0.504065in}}%
\pgfusepath{stroke,fill}%
\end{pgfscope}%
\begin{pgfscope}%
\pgfpathrectangle{\pgfqpoint{0.636356in}{0.440955in}}{\pgfqpoint{1.933229in}{0.162432in}} %
\pgfusepath{clip}%
\pgfsetbuttcap%
\pgfsetmiterjoin%
\definecolor{currentfill}{rgb}{0.333333,0.333333,0.333333}%
\pgfsetfillcolor{currentfill}%
\pgfsetlinewidth{0.501875pt}%
\definecolor{currentstroke}{rgb}{0.000000,0.000000,0.000000}%
\pgfsetstrokecolor{currentstroke}%
\pgfsetdash{}{0pt}%
\pgfpathmoveto{\pgfqpoint{2.221603in}{0.517575in}}%
\pgfpathlineto{\pgfqpoint{2.240936in}{0.517575in}}%
\pgfpathlineto{\pgfqpoint{2.240936in}{0.522171in}}%
\pgfpathlineto{\pgfqpoint{2.221603in}{0.522171in}}%
\pgfpathlineto{\pgfqpoint{2.221603in}{0.517575in}}%
\pgfusepath{stroke,fill}%
\end{pgfscope}%
\begin{pgfscope}%
\pgfpathrectangle{\pgfqpoint{0.636356in}{0.440955in}}{\pgfqpoint{1.933229in}{0.162432in}} %
\pgfusepath{clip}%
\pgfsetbuttcap%
\pgfsetmiterjoin%
\definecolor{currentfill}{rgb}{0.333333,0.333333,0.333333}%
\pgfsetfillcolor{currentfill}%
\pgfsetlinewidth{0.501875pt}%
\definecolor{currentstroke}{rgb}{0.000000,0.000000,0.000000}%
\pgfsetstrokecolor{currentstroke}%
\pgfsetdash{}{0pt}%
\pgfpathmoveto{\pgfqpoint{2.240936in}{0.497998in}}%
\pgfpathlineto{\pgfqpoint{2.260268in}{0.497998in}}%
\pgfpathlineto{\pgfqpoint{2.260268in}{0.522171in}}%
\pgfpathlineto{\pgfqpoint{2.240936in}{0.522171in}}%
\pgfpathlineto{\pgfqpoint{2.240936in}{0.497998in}}%
\pgfusepath{stroke,fill}%
\end{pgfscope}%
\begin{pgfscope}%
\pgfpathrectangle{\pgfqpoint{0.636356in}{0.440955in}}{\pgfqpoint{1.933229in}{0.162432in}} %
\pgfusepath{clip}%
\pgfsetbuttcap%
\pgfsetmiterjoin%
\definecolor{currentfill}{rgb}{0.333333,0.333333,0.333333}%
\pgfsetfillcolor{currentfill}%
\pgfsetlinewidth{0.501875pt}%
\definecolor{currentstroke}{rgb}{0.000000,0.000000,0.000000}%
\pgfsetstrokecolor{currentstroke}%
\pgfsetdash{}{0pt}%
\pgfpathmoveto{\pgfqpoint{2.260268in}{0.516036in}}%
\pgfpathlineto{\pgfqpoint{2.279600in}{0.516036in}}%
\pgfpathlineto{\pgfqpoint{2.279600in}{0.522171in}}%
\pgfpathlineto{\pgfqpoint{2.260268in}{0.522171in}}%
\pgfpathlineto{\pgfqpoint{2.260268in}{0.516036in}}%
\pgfusepath{stroke,fill}%
\end{pgfscope}%
\begin{pgfscope}%
\pgfpathrectangle{\pgfqpoint{0.636356in}{0.440955in}}{\pgfqpoint{1.933229in}{0.162432in}} %
\pgfusepath{clip}%
\pgfsetbuttcap%
\pgfsetmiterjoin%
\definecolor{currentfill}{rgb}{0.333333,0.333333,0.333333}%
\pgfsetfillcolor{currentfill}%
\pgfsetlinewidth{0.501875pt}%
\definecolor{currentstroke}{rgb}{0.000000,0.000000,0.000000}%
\pgfsetstrokecolor{currentstroke}%
\pgfsetdash{}{0pt}%
\pgfpathmoveto{\pgfqpoint{2.279600in}{0.508703in}}%
\pgfpathlineto{\pgfqpoint{2.298932in}{0.508703in}}%
\pgfpathlineto{\pgfqpoint{2.298932in}{0.522171in}}%
\pgfpathlineto{\pgfqpoint{2.279600in}{0.522171in}}%
\pgfpathlineto{\pgfqpoint{2.279600in}{0.508703in}}%
\pgfusepath{stroke,fill}%
\end{pgfscope}%
\begin{pgfscope}%
\pgfpathrectangle{\pgfqpoint{0.636356in}{0.440955in}}{\pgfqpoint{1.933229in}{0.162432in}} %
\pgfusepath{clip}%
\pgfsetbuttcap%
\pgfsetmiterjoin%
\definecolor{currentfill}{rgb}{0.333333,0.333333,0.333333}%
\pgfsetfillcolor{currentfill}%
\pgfsetlinewidth{0.501875pt}%
\definecolor{currentstroke}{rgb}{0.000000,0.000000,0.000000}%
\pgfsetstrokecolor{currentstroke}%
\pgfsetdash{}{0pt}%
\pgfpathmoveto{\pgfqpoint{2.298932in}{0.505574in}}%
\pgfpathlineto{\pgfqpoint{2.318265in}{0.505574in}}%
\pgfpathlineto{\pgfqpoint{2.318265in}{0.522171in}}%
\pgfpathlineto{\pgfqpoint{2.298932in}{0.522171in}}%
\pgfpathlineto{\pgfqpoint{2.298932in}{0.505574in}}%
\pgfusepath{stroke,fill}%
\end{pgfscope}%
\begin{pgfscope}%
\pgfpathrectangle{\pgfqpoint{0.636356in}{0.440955in}}{\pgfqpoint{1.933229in}{0.162432in}} %
\pgfusepath{clip}%
\pgfsetbuttcap%
\pgfsetmiterjoin%
\definecolor{currentfill}{rgb}{0.333333,0.333333,0.333333}%
\pgfsetfillcolor{currentfill}%
\pgfsetlinewidth{0.501875pt}%
\definecolor{currentstroke}{rgb}{0.000000,0.000000,0.000000}%
\pgfsetstrokecolor{currentstroke}%
\pgfsetdash{}{0pt}%
\pgfpathmoveto{\pgfqpoint{2.318265in}{0.491833in}}%
\pgfpathlineto{\pgfqpoint{2.337597in}{0.491833in}}%
\pgfpathlineto{\pgfqpoint{2.337597in}{0.522171in}}%
\pgfpathlineto{\pgfqpoint{2.318265in}{0.522171in}}%
\pgfpathlineto{\pgfqpoint{2.318265in}{0.491833in}}%
\pgfusepath{stroke,fill}%
\end{pgfscope}%
\begin{pgfscope}%
\pgfpathrectangle{\pgfqpoint{0.636356in}{0.440955in}}{\pgfqpoint{1.933229in}{0.162432in}} %
\pgfusepath{clip}%
\pgfsetbuttcap%
\pgfsetmiterjoin%
\definecolor{currentfill}{rgb}{0.333333,0.333333,0.333333}%
\pgfsetfillcolor{currentfill}%
\pgfsetlinewidth{0.501875pt}%
\definecolor{currentstroke}{rgb}{0.000000,0.000000,0.000000}%
\pgfsetstrokecolor{currentstroke}%
\pgfsetdash{}{0pt}%
\pgfpathmoveto{\pgfqpoint{2.337597in}{0.515386in}}%
\pgfpathlineto{\pgfqpoint{2.356929in}{0.515386in}}%
\pgfpathlineto{\pgfqpoint{2.356929in}{0.522171in}}%
\pgfpathlineto{\pgfqpoint{2.337597in}{0.522171in}}%
\pgfpathlineto{\pgfqpoint{2.337597in}{0.515386in}}%
\pgfusepath{stroke,fill}%
\end{pgfscope}%
\begin{pgfscope}%
\pgfpathrectangle{\pgfqpoint{0.636356in}{0.440955in}}{\pgfqpoint{1.933229in}{0.162432in}} %
\pgfusepath{clip}%
\pgfsetbuttcap%
\pgfsetmiterjoin%
\definecolor{currentfill}{rgb}{0.333333,0.333333,0.333333}%
\pgfsetfillcolor{currentfill}%
\pgfsetlinewidth{0.501875pt}%
\definecolor{currentstroke}{rgb}{0.000000,0.000000,0.000000}%
\pgfsetstrokecolor{currentstroke}%
\pgfsetdash{}{0pt}%
\pgfpathmoveto{\pgfqpoint{2.356929in}{0.500868in}}%
\pgfpathlineto{\pgfqpoint{2.376262in}{0.500868in}}%
\pgfpathlineto{\pgfqpoint{2.376262in}{0.522171in}}%
\pgfpathlineto{\pgfqpoint{2.356929in}{0.522171in}}%
\pgfpathlineto{\pgfqpoint{2.356929in}{0.500868in}}%
\pgfusepath{stroke,fill}%
\end{pgfscope}%
\begin{pgfscope}%
\pgfpathrectangle{\pgfqpoint{0.636356in}{0.440955in}}{\pgfqpoint{1.933229in}{0.162432in}} %
\pgfusepath{clip}%
\pgfsetbuttcap%
\pgfsetmiterjoin%
\definecolor{currentfill}{rgb}{0.333333,0.333333,0.333333}%
\pgfsetfillcolor{currentfill}%
\pgfsetlinewidth{0.501875pt}%
\definecolor{currentstroke}{rgb}{0.000000,0.000000,0.000000}%
\pgfsetstrokecolor{currentstroke}%
\pgfsetdash{}{0pt}%
\pgfpathmoveto{\pgfqpoint{2.376262in}{0.464358in}}%
\pgfpathlineto{\pgfqpoint{2.395594in}{0.464358in}}%
\pgfpathlineto{\pgfqpoint{2.395594in}{0.522171in}}%
\pgfpathlineto{\pgfqpoint{2.376262in}{0.522171in}}%
\pgfpathlineto{\pgfqpoint{2.376262in}{0.464358in}}%
\pgfusepath{stroke,fill}%
\end{pgfscope}%
\begin{pgfscope}%
\pgfpathrectangle{\pgfqpoint{0.636356in}{0.440955in}}{\pgfqpoint{1.933229in}{0.162432in}} %
\pgfusepath{clip}%
\pgfsetbuttcap%
\pgfsetmiterjoin%
\definecolor{currentfill}{rgb}{0.333333,0.333333,0.333333}%
\pgfsetfillcolor{currentfill}%
\pgfsetlinewidth{0.501875pt}%
\definecolor{currentstroke}{rgb}{0.000000,0.000000,0.000000}%
\pgfsetstrokecolor{currentstroke}%
\pgfsetdash{}{0pt}%
\pgfpathmoveto{\pgfqpoint{2.395594in}{0.520076in}}%
\pgfpathlineto{\pgfqpoint{2.414926in}{0.520076in}}%
\pgfpathlineto{\pgfqpoint{2.414926in}{0.522171in}}%
\pgfpathlineto{\pgfqpoint{2.395594in}{0.522171in}}%
\pgfpathlineto{\pgfqpoint{2.395594in}{0.520076in}}%
\pgfusepath{stroke,fill}%
\end{pgfscope}%
\begin{pgfscope}%
\pgfpathrectangle{\pgfqpoint{0.636356in}{0.440955in}}{\pgfqpoint{1.933229in}{0.162432in}} %
\pgfusepath{clip}%
\pgfsetbuttcap%
\pgfsetmiterjoin%
\definecolor{currentfill}{rgb}{0.333333,0.333333,0.333333}%
\pgfsetfillcolor{currentfill}%
\pgfsetlinewidth{0.501875pt}%
\definecolor{currentstroke}{rgb}{0.000000,0.000000,0.000000}%
\pgfsetstrokecolor{currentstroke}%
\pgfsetdash{}{0pt}%
\pgfpathmoveto{\pgfqpoint{2.414926in}{0.490313in}}%
\pgfpathlineto{\pgfqpoint{2.434258in}{0.490313in}}%
\pgfpathlineto{\pgfqpoint{2.434258in}{0.522171in}}%
\pgfpathlineto{\pgfqpoint{2.414926in}{0.522171in}}%
\pgfpathlineto{\pgfqpoint{2.414926in}{0.490313in}}%
\pgfusepath{stroke,fill}%
\end{pgfscope}%
\begin{pgfscope}%
\pgfpathrectangle{\pgfqpoint{0.636356in}{0.440955in}}{\pgfqpoint{1.933229in}{0.162432in}} %
\pgfusepath{clip}%
\pgfsetbuttcap%
\pgfsetmiterjoin%
\definecolor{currentfill}{rgb}{0.333333,0.333333,0.333333}%
\pgfsetfillcolor{currentfill}%
\pgfsetlinewidth{0.501875pt}%
\definecolor{currentstroke}{rgb}{0.000000,0.000000,0.000000}%
\pgfsetstrokecolor{currentstroke}%
\pgfsetdash{}{0pt}%
\pgfpathmoveto{\pgfqpoint{2.434258in}{0.522171in}}%
\pgfpathlineto{\pgfqpoint{2.453591in}{0.522171in}}%
\pgfpathlineto{\pgfqpoint{2.453591in}{0.529684in}}%
\pgfpathlineto{\pgfqpoint{2.434258in}{0.529684in}}%
\pgfpathlineto{\pgfqpoint{2.434258in}{0.522171in}}%
\pgfusepath{stroke,fill}%
\end{pgfscope}%
\begin{pgfscope}%
\pgfpathrectangle{\pgfqpoint{0.636356in}{0.440955in}}{\pgfqpoint{1.933229in}{0.162432in}} %
\pgfusepath{clip}%
\pgfsetbuttcap%
\pgfsetmiterjoin%
\definecolor{currentfill}{rgb}{0.333333,0.333333,0.333333}%
\pgfsetfillcolor{currentfill}%
\pgfsetlinewidth{0.501875pt}%
\definecolor{currentstroke}{rgb}{0.000000,0.000000,0.000000}%
\pgfsetstrokecolor{currentstroke}%
\pgfsetdash{}{0pt}%
\pgfpathmoveto{\pgfqpoint{2.453591in}{0.522171in}}%
\pgfpathlineto{\pgfqpoint{2.472923in}{0.522171in}}%
\pgfpathlineto{\pgfqpoint{2.472923in}{0.545750in}}%
\pgfpathlineto{\pgfqpoint{2.453591in}{0.545750in}}%
\pgfpathlineto{\pgfqpoint{2.453591in}{0.522171in}}%
\pgfusepath{stroke,fill}%
\end{pgfscope}%
\begin{pgfscope}%
\pgfpathrectangle{\pgfqpoint{0.636356in}{0.440955in}}{\pgfqpoint{1.933229in}{0.162432in}} %
\pgfusepath{clip}%
\pgfsetbuttcap%
\pgfsetmiterjoin%
\definecolor{currentfill}{rgb}{0.333333,0.333333,0.333333}%
\pgfsetfillcolor{currentfill}%
\pgfsetlinewidth{0.501875pt}%
\definecolor{currentstroke}{rgb}{0.000000,0.000000,0.000000}%
\pgfsetstrokecolor{currentstroke}%
\pgfsetdash{}{0pt}%
\pgfpathmoveto{\pgfqpoint{2.472923in}{0.481497in}}%
\pgfpathlineto{\pgfqpoint{2.492255in}{0.481497in}}%
\pgfpathlineto{\pgfqpoint{2.492255in}{0.522171in}}%
\pgfpathlineto{\pgfqpoint{2.472923in}{0.522171in}}%
\pgfpathlineto{\pgfqpoint{2.472923in}{0.481497in}}%
\pgfusepath{stroke,fill}%
\end{pgfscope}%
\begin{pgfscope}%
\pgfpathrectangle{\pgfqpoint{0.636356in}{0.440955in}}{\pgfqpoint{1.933229in}{0.162432in}} %
\pgfusepath{clip}%
\pgfsetbuttcap%
\pgfsetmiterjoin%
\definecolor{currentfill}{rgb}{0.333333,0.333333,0.333333}%
\pgfsetfillcolor{currentfill}%
\pgfsetlinewidth{0.501875pt}%
\definecolor{currentstroke}{rgb}{0.000000,0.000000,0.000000}%
\pgfsetstrokecolor{currentstroke}%
\pgfsetdash{}{0pt}%
\pgfpathmoveto{\pgfqpoint{2.492255in}{0.503978in}}%
\pgfpathlineto{\pgfqpoint{2.511588in}{0.503978in}}%
\pgfpathlineto{\pgfqpoint{2.511588in}{0.522171in}}%
\pgfpathlineto{\pgfqpoint{2.492255in}{0.522171in}}%
\pgfpathlineto{\pgfqpoint{2.492255in}{0.503978in}}%
\pgfusepath{stroke,fill}%
\end{pgfscope}%
\begin{pgfscope}%
\pgfpathrectangle{\pgfqpoint{0.636356in}{0.440955in}}{\pgfqpoint{1.933229in}{0.162432in}} %
\pgfusepath{clip}%
\pgfsetbuttcap%
\pgfsetmiterjoin%
\definecolor{currentfill}{rgb}{0.333333,0.333333,0.333333}%
\pgfsetfillcolor{currentfill}%
\pgfsetlinewidth{0.501875pt}%
\definecolor{currentstroke}{rgb}{0.000000,0.000000,0.000000}%
\pgfsetstrokecolor{currentstroke}%
\pgfsetdash{}{0pt}%
\pgfpathmoveto{\pgfqpoint{2.511588in}{0.483495in}}%
\pgfpathlineto{\pgfqpoint{2.530920in}{0.483495in}}%
\pgfpathlineto{\pgfqpoint{2.530920in}{0.522171in}}%
\pgfpathlineto{\pgfqpoint{2.511588in}{0.522171in}}%
\pgfpathlineto{\pgfqpoint{2.511588in}{0.483495in}}%
\pgfusepath{stroke,fill}%
\end{pgfscope}%
\begin{pgfscope}%
\pgfpathrectangle{\pgfqpoint{0.636356in}{0.440955in}}{\pgfqpoint{1.933229in}{0.162432in}} %
\pgfusepath{clip}%
\pgfsetbuttcap%
\pgfsetmiterjoin%
\definecolor{currentfill}{rgb}{0.333333,0.333333,0.333333}%
\pgfsetfillcolor{currentfill}%
\pgfsetlinewidth{0.501875pt}%
\definecolor{currentstroke}{rgb}{0.000000,0.000000,0.000000}%
\pgfsetstrokecolor{currentstroke}%
\pgfsetdash{}{0pt}%
\pgfpathmoveto{\pgfqpoint{2.530920in}{0.493562in}}%
\pgfpathlineto{\pgfqpoint{2.550252in}{0.493562in}}%
\pgfpathlineto{\pgfqpoint{2.550252in}{0.522171in}}%
\pgfpathlineto{\pgfqpoint{2.530920in}{0.522171in}}%
\pgfpathlineto{\pgfqpoint{2.530920in}{0.493562in}}%
\pgfusepath{stroke,fill}%
\end{pgfscope}%
\begin{pgfscope}%
\pgfpathrectangle{\pgfqpoint{0.636356in}{0.440955in}}{\pgfqpoint{1.933229in}{0.162432in}} %
\pgfusepath{clip}%
\pgfsetbuttcap%
\pgfsetmiterjoin%
\definecolor{currentfill}{rgb}{0.333333,0.333333,0.333333}%
\pgfsetfillcolor{currentfill}%
\pgfsetlinewidth{0.501875pt}%
\definecolor{currentstroke}{rgb}{0.000000,0.000000,0.000000}%
\pgfsetstrokecolor{currentstroke}%
\pgfsetdash{}{0pt}%
\pgfpathmoveto{\pgfqpoint{2.550252in}{0.518522in}}%
\pgfpathlineto{\pgfqpoint{2.569584in}{0.518522in}}%
\pgfpathlineto{\pgfqpoint{2.569584in}{0.522171in}}%
\pgfpathlineto{\pgfqpoint{2.550252in}{0.522171in}}%
\pgfpathlineto{\pgfqpoint{2.550252in}{0.518522in}}%
\pgfusepath{stroke,fill}%
\end{pgfscope}%
\begin{pgfscope}%
\pgfsetrectcap%
\pgfsetmiterjoin%
\pgfsetlinewidth{1.003750pt}%
\definecolor{currentstroke}{rgb}{0.000000,0.000000,0.000000}%
\pgfsetstrokecolor{currentstroke}%
\pgfsetdash{}{0pt}%
\pgfpathmoveto{\pgfqpoint{0.636356in}{0.603387in}}%
\pgfpathlineto{\pgfqpoint{2.569584in}{0.603387in}}%
\pgfusepath{stroke}%
\end{pgfscope}%
\begin{pgfscope}%
\pgfsetrectcap%
\pgfsetmiterjoin%
\pgfsetlinewidth{1.003750pt}%
\definecolor{currentstroke}{rgb}{0.000000,0.000000,0.000000}%
\pgfsetstrokecolor{currentstroke}%
\pgfsetdash{}{0pt}%
\pgfpathmoveto{\pgfqpoint{2.569584in}{0.440955in}}%
\pgfpathlineto{\pgfqpoint{2.569584in}{0.603387in}}%
\pgfusepath{stroke}%
\end{pgfscope}%
\begin{pgfscope}%
\pgfsetrectcap%
\pgfsetmiterjoin%
\pgfsetlinewidth{1.003750pt}%
\definecolor{currentstroke}{rgb}{0.000000,0.000000,0.000000}%
\pgfsetstrokecolor{currentstroke}%
\pgfsetdash{}{0pt}%
\pgfpathmoveto{\pgfqpoint{0.636356in}{0.440955in}}%
\pgfpathlineto{\pgfqpoint{2.569584in}{0.440955in}}%
\pgfusepath{stroke}%
\end{pgfscope}%
\begin{pgfscope}%
\pgfsetrectcap%
\pgfsetmiterjoin%
\pgfsetlinewidth{1.003750pt}%
\definecolor{currentstroke}{rgb}{0.000000,0.000000,0.000000}%
\pgfsetstrokecolor{currentstroke}%
\pgfsetdash{}{0pt}%
\pgfpathmoveto{\pgfqpoint{0.636356in}{0.440955in}}%
\pgfpathlineto{\pgfqpoint{0.636356in}{0.603387in}}%
\pgfusepath{stroke}%
\end{pgfscope}%
\begin{pgfscope}%
\pgfsetbuttcap%
\pgfsetroundjoin%
\definecolor{currentfill}{rgb}{0.000000,0.000000,0.000000}%
\pgfsetfillcolor{currentfill}%
\pgfsetlinewidth{0.501875pt}%
\definecolor{currentstroke}{rgb}{0.000000,0.000000,0.000000}%
\pgfsetstrokecolor{currentstroke}%
\pgfsetdash{}{0pt}%
\pgfsys@defobject{currentmarker}{\pgfqpoint{0.000000in}{0.000000in}}{\pgfqpoint{0.000000in}{0.069444in}}{%
\pgfpathmoveto{\pgfqpoint{0.000000in}{0.000000in}}%
\pgfpathlineto{\pgfqpoint{0.000000in}{0.069444in}}%
\pgfusepath{stroke,fill}%
}%
\begin{pgfscope}%
\pgfsys@transformshift{0.899978in}{0.440955in}%
\pgfsys@useobject{currentmarker}{}%
\end{pgfscope}%
\end{pgfscope}%
\begin{pgfscope}%
\pgfsetbuttcap%
\pgfsetroundjoin%
\definecolor{currentfill}{rgb}{0.000000,0.000000,0.000000}%
\pgfsetfillcolor{currentfill}%
\pgfsetlinewidth{0.501875pt}%
\definecolor{currentstroke}{rgb}{0.000000,0.000000,0.000000}%
\pgfsetstrokecolor{currentstroke}%
\pgfsetdash{}{0pt}%
\pgfsys@defobject{currentmarker}{\pgfqpoint{0.000000in}{-0.069444in}}{\pgfqpoint{0.000000in}{0.000000in}}{%
\pgfpathmoveto{\pgfqpoint{0.000000in}{0.000000in}}%
\pgfpathlineto{\pgfqpoint{0.000000in}{-0.069444in}}%
\pgfusepath{stroke,fill}%
}%
\begin{pgfscope}%
\pgfsys@transformshift{0.899978in}{0.603387in}%
\pgfsys@useobject{currentmarker}{}%
\end{pgfscope}%
\end{pgfscope}%
\begin{pgfscope}%
\pgftext[x=0.899978in,y=0.371511in,,top]{\rmfamily\fontsize{8.000000}{9.600000}\selectfont 5240}%
\end{pgfscope}%
\begin{pgfscope}%
\pgfsetbuttcap%
\pgfsetroundjoin%
\definecolor{currentfill}{rgb}{0.000000,0.000000,0.000000}%
\pgfsetfillcolor{currentfill}%
\pgfsetlinewidth{0.501875pt}%
\definecolor{currentstroke}{rgb}{0.000000,0.000000,0.000000}%
\pgfsetstrokecolor{currentstroke}%
\pgfsetdash{}{0pt}%
\pgfsys@defobject{currentmarker}{\pgfqpoint{0.000000in}{0.000000in}}{\pgfqpoint{0.000000in}{0.069444in}}{%
\pgfpathmoveto{\pgfqpoint{0.000000in}{0.000000in}}%
\pgfpathlineto{\pgfqpoint{0.000000in}{0.069444in}}%
\pgfusepath{stroke,fill}%
}%
\begin{pgfscope}%
\pgfsys@transformshift{1.251474in}{0.440955in}%
\pgfsys@useobject{currentmarker}{}%
\end{pgfscope}%
\end{pgfscope}%
\begin{pgfscope}%
\pgfsetbuttcap%
\pgfsetroundjoin%
\definecolor{currentfill}{rgb}{0.000000,0.000000,0.000000}%
\pgfsetfillcolor{currentfill}%
\pgfsetlinewidth{0.501875pt}%
\definecolor{currentstroke}{rgb}{0.000000,0.000000,0.000000}%
\pgfsetstrokecolor{currentstroke}%
\pgfsetdash{}{0pt}%
\pgfsys@defobject{currentmarker}{\pgfqpoint{0.000000in}{-0.069444in}}{\pgfqpoint{0.000000in}{0.000000in}}{%
\pgfpathmoveto{\pgfqpoint{0.000000in}{0.000000in}}%
\pgfpathlineto{\pgfqpoint{0.000000in}{-0.069444in}}%
\pgfusepath{stroke,fill}%
}%
\begin{pgfscope}%
\pgfsys@transformshift{1.251474in}{0.603387in}%
\pgfsys@useobject{currentmarker}{}%
\end{pgfscope}%
\end{pgfscope}%
\begin{pgfscope}%
\pgftext[x=1.251474in,y=0.371511in,,top]{\rmfamily\fontsize{8.000000}{9.600000}\selectfont 5260}%
\end{pgfscope}%
\begin{pgfscope}%
\pgfsetbuttcap%
\pgfsetroundjoin%
\definecolor{currentfill}{rgb}{0.000000,0.000000,0.000000}%
\pgfsetfillcolor{currentfill}%
\pgfsetlinewidth{0.501875pt}%
\definecolor{currentstroke}{rgb}{0.000000,0.000000,0.000000}%
\pgfsetstrokecolor{currentstroke}%
\pgfsetdash{}{0pt}%
\pgfsys@defobject{currentmarker}{\pgfqpoint{0.000000in}{0.000000in}}{\pgfqpoint{0.000000in}{0.069444in}}{%
\pgfpathmoveto{\pgfqpoint{0.000000in}{0.000000in}}%
\pgfpathlineto{\pgfqpoint{0.000000in}{0.069444in}}%
\pgfusepath{stroke,fill}%
}%
\begin{pgfscope}%
\pgfsys@transformshift{1.602970in}{0.440955in}%
\pgfsys@useobject{currentmarker}{}%
\end{pgfscope}%
\end{pgfscope}%
\begin{pgfscope}%
\pgfsetbuttcap%
\pgfsetroundjoin%
\definecolor{currentfill}{rgb}{0.000000,0.000000,0.000000}%
\pgfsetfillcolor{currentfill}%
\pgfsetlinewidth{0.501875pt}%
\definecolor{currentstroke}{rgb}{0.000000,0.000000,0.000000}%
\pgfsetstrokecolor{currentstroke}%
\pgfsetdash{}{0pt}%
\pgfsys@defobject{currentmarker}{\pgfqpoint{0.000000in}{-0.069444in}}{\pgfqpoint{0.000000in}{0.000000in}}{%
\pgfpathmoveto{\pgfqpoint{0.000000in}{0.000000in}}%
\pgfpathlineto{\pgfqpoint{0.000000in}{-0.069444in}}%
\pgfusepath{stroke,fill}%
}%
\begin{pgfscope}%
\pgfsys@transformshift{1.602970in}{0.603387in}%
\pgfsys@useobject{currentmarker}{}%
\end{pgfscope}%
\end{pgfscope}%
\begin{pgfscope}%
\pgftext[x=1.602970in,y=0.371511in,,top]{\rmfamily\fontsize{8.000000}{9.600000}\selectfont 5280}%
\end{pgfscope}%
\begin{pgfscope}%
\pgfsetbuttcap%
\pgfsetroundjoin%
\definecolor{currentfill}{rgb}{0.000000,0.000000,0.000000}%
\pgfsetfillcolor{currentfill}%
\pgfsetlinewidth{0.501875pt}%
\definecolor{currentstroke}{rgb}{0.000000,0.000000,0.000000}%
\pgfsetstrokecolor{currentstroke}%
\pgfsetdash{}{0pt}%
\pgfsys@defobject{currentmarker}{\pgfqpoint{0.000000in}{0.000000in}}{\pgfqpoint{0.000000in}{0.069444in}}{%
\pgfpathmoveto{\pgfqpoint{0.000000in}{0.000000in}}%
\pgfpathlineto{\pgfqpoint{0.000000in}{0.069444in}}%
\pgfusepath{stroke,fill}%
}%
\begin{pgfscope}%
\pgfsys@transformshift{1.954466in}{0.440955in}%
\pgfsys@useobject{currentmarker}{}%
\end{pgfscope}%
\end{pgfscope}%
\begin{pgfscope}%
\pgfsetbuttcap%
\pgfsetroundjoin%
\definecolor{currentfill}{rgb}{0.000000,0.000000,0.000000}%
\pgfsetfillcolor{currentfill}%
\pgfsetlinewidth{0.501875pt}%
\definecolor{currentstroke}{rgb}{0.000000,0.000000,0.000000}%
\pgfsetstrokecolor{currentstroke}%
\pgfsetdash{}{0pt}%
\pgfsys@defobject{currentmarker}{\pgfqpoint{0.000000in}{-0.069444in}}{\pgfqpoint{0.000000in}{0.000000in}}{%
\pgfpathmoveto{\pgfqpoint{0.000000in}{0.000000in}}%
\pgfpathlineto{\pgfqpoint{0.000000in}{-0.069444in}}%
\pgfusepath{stroke,fill}%
}%
\begin{pgfscope}%
\pgfsys@transformshift{1.954466in}{0.603387in}%
\pgfsys@useobject{currentmarker}{}%
\end{pgfscope}%
\end{pgfscope}%
\begin{pgfscope}%
\pgftext[x=1.954466in,y=0.371511in,,top]{\rmfamily\fontsize{8.000000}{9.600000}\selectfont 5300}%
\end{pgfscope}%
\begin{pgfscope}%
\pgfsetbuttcap%
\pgfsetroundjoin%
\definecolor{currentfill}{rgb}{0.000000,0.000000,0.000000}%
\pgfsetfillcolor{currentfill}%
\pgfsetlinewidth{0.501875pt}%
\definecolor{currentstroke}{rgb}{0.000000,0.000000,0.000000}%
\pgfsetstrokecolor{currentstroke}%
\pgfsetdash{}{0pt}%
\pgfsys@defobject{currentmarker}{\pgfqpoint{0.000000in}{0.000000in}}{\pgfqpoint{0.000000in}{0.069444in}}{%
\pgfpathmoveto{\pgfqpoint{0.000000in}{0.000000in}}%
\pgfpathlineto{\pgfqpoint{0.000000in}{0.069444in}}%
\pgfusepath{stroke,fill}%
}%
\begin{pgfscope}%
\pgfsys@transformshift{2.305962in}{0.440955in}%
\pgfsys@useobject{currentmarker}{}%
\end{pgfscope}%
\end{pgfscope}%
\begin{pgfscope}%
\pgfsetbuttcap%
\pgfsetroundjoin%
\definecolor{currentfill}{rgb}{0.000000,0.000000,0.000000}%
\pgfsetfillcolor{currentfill}%
\pgfsetlinewidth{0.501875pt}%
\definecolor{currentstroke}{rgb}{0.000000,0.000000,0.000000}%
\pgfsetstrokecolor{currentstroke}%
\pgfsetdash{}{0pt}%
\pgfsys@defobject{currentmarker}{\pgfqpoint{0.000000in}{-0.069444in}}{\pgfqpoint{0.000000in}{0.000000in}}{%
\pgfpathmoveto{\pgfqpoint{0.000000in}{0.000000in}}%
\pgfpathlineto{\pgfqpoint{0.000000in}{-0.069444in}}%
\pgfusepath{stroke,fill}%
}%
\begin{pgfscope}%
\pgfsys@transformshift{2.305962in}{0.603387in}%
\pgfsys@useobject{currentmarker}{}%
\end{pgfscope}%
\end{pgfscope}%
\begin{pgfscope}%
\pgftext[x=2.305962in,y=0.371511in,,top]{\rmfamily\fontsize{8.000000}{9.600000}\selectfont 5320}%
\end{pgfscope}%
\begin{pgfscope}%
\pgftext[x=1.602970in,y=0.194536in,,top]{\rmfamily\fontsize{9.000000}{10.800000}\selectfont \(\displaystyle m(K^+\!\pi^-\!\mu^+\!\mu^-)\)}%
\end{pgfscope}%
\begin{pgfscope}%
\pgfsetbuttcap%
\pgfsetroundjoin%
\definecolor{currentfill}{rgb}{0.000000,0.000000,0.000000}%
\pgfsetfillcolor{currentfill}%
\pgfsetlinewidth{0.501875pt}%
\definecolor{currentstroke}{rgb}{0.000000,0.000000,0.000000}%
\pgfsetstrokecolor{currentstroke}%
\pgfsetdash{}{0pt}%
\pgfsys@defobject{currentmarker}{\pgfqpoint{0.000000in}{0.000000in}}{\pgfqpoint{0.069444in}{0.000000in}}{%
\pgfpathmoveto{\pgfqpoint{0.000000in}{0.000000in}}%
\pgfpathlineto{\pgfqpoint{0.069444in}{0.000000in}}%
\pgfusepath{stroke,fill}%
}%
\begin{pgfscope}%
\pgfsys@transformshift{0.636356in}{0.440955in}%
\pgfsys@useobject{currentmarker}{}%
\end{pgfscope}%
\end{pgfscope}%
\begin{pgfscope}%
\pgfsetbuttcap%
\pgfsetroundjoin%
\definecolor{currentfill}{rgb}{0.000000,0.000000,0.000000}%
\pgfsetfillcolor{currentfill}%
\pgfsetlinewidth{0.501875pt}%
\definecolor{currentstroke}{rgb}{0.000000,0.000000,0.000000}%
\pgfsetstrokecolor{currentstroke}%
\pgfsetdash{}{0pt}%
\pgfsys@defobject{currentmarker}{\pgfqpoint{-0.069444in}{0.000000in}}{\pgfqpoint{0.000000in}{0.000000in}}{%
\pgfpathmoveto{\pgfqpoint{0.000000in}{0.000000in}}%
\pgfpathlineto{\pgfqpoint{-0.069444in}{0.000000in}}%
\pgfusepath{stroke,fill}%
}%
\begin{pgfscope}%
\pgfsys@transformshift{2.569584in}{0.440955in}%
\pgfsys@useobject{currentmarker}{}%
\end{pgfscope}%
\end{pgfscope}%
\begin{pgfscope}%
\pgftext[x=0.566911in,y=0.440955in,right,]{\rmfamily\fontsize{8.000000}{9.600000}\selectfont −3}%
\end{pgfscope}%
\begin{pgfscope}%
\pgfsetbuttcap%
\pgfsetroundjoin%
\definecolor{currentfill}{rgb}{0.000000,0.000000,0.000000}%
\pgfsetfillcolor{currentfill}%
\pgfsetlinewidth{0.501875pt}%
\definecolor{currentstroke}{rgb}{0.000000,0.000000,0.000000}%
\pgfsetstrokecolor{currentstroke}%
\pgfsetdash{}{0pt}%
\pgfsys@defobject{currentmarker}{\pgfqpoint{0.000000in}{0.000000in}}{\pgfqpoint{0.069444in}{0.000000in}}{%
\pgfpathmoveto{\pgfqpoint{0.000000in}{0.000000in}}%
\pgfpathlineto{\pgfqpoint{0.069444in}{0.000000in}}%
\pgfusepath{stroke,fill}%
}%
\begin{pgfscope}%
\pgfsys@transformshift{0.636356in}{0.522171in}%
\pgfsys@useobject{currentmarker}{}%
\end{pgfscope}%
\end{pgfscope}%
\begin{pgfscope}%
\pgfsetbuttcap%
\pgfsetroundjoin%
\definecolor{currentfill}{rgb}{0.000000,0.000000,0.000000}%
\pgfsetfillcolor{currentfill}%
\pgfsetlinewidth{0.501875pt}%
\definecolor{currentstroke}{rgb}{0.000000,0.000000,0.000000}%
\pgfsetstrokecolor{currentstroke}%
\pgfsetdash{}{0pt}%
\pgfsys@defobject{currentmarker}{\pgfqpoint{-0.069444in}{0.000000in}}{\pgfqpoint{0.000000in}{0.000000in}}{%
\pgfpathmoveto{\pgfqpoint{0.000000in}{0.000000in}}%
\pgfpathlineto{\pgfqpoint{-0.069444in}{0.000000in}}%
\pgfusepath{stroke,fill}%
}%
\begin{pgfscope}%
\pgfsys@transformshift{2.569584in}{0.522171in}%
\pgfsys@useobject{currentmarker}{}%
\end{pgfscope}%
\end{pgfscope}%
\begin{pgfscope}%
\pgftext[x=0.566911in,y=0.522171in,right,]{\rmfamily\fontsize{8.000000}{9.600000}\selectfont 0}%
\end{pgfscope}%
\begin{pgfscope}%
\pgfsetbuttcap%
\pgfsetroundjoin%
\definecolor{currentfill}{rgb}{0.000000,0.000000,0.000000}%
\pgfsetfillcolor{currentfill}%
\pgfsetlinewidth{0.501875pt}%
\definecolor{currentstroke}{rgb}{0.000000,0.000000,0.000000}%
\pgfsetstrokecolor{currentstroke}%
\pgfsetdash{}{0pt}%
\pgfsys@defobject{currentmarker}{\pgfqpoint{0.000000in}{0.000000in}}{\pgfqpoint{0.069444in}{0.000000in}}{%
\pgfpathmoveto{\pgfqpoint{0.000000in}{0.000000in}}%
\pgfpathlineto{\pgfqpoint{0.069444in}{0.000000in}}%
\pgfusepath{stroke,fill}%
}%
\begin{pgfscope}%
\pgfsys@transformshift{0.636356in}{0.603387in}%
\pgfsys@useobject{currentmarker}{}%
\end{pgfscope}%
\end{pgfscope}%
\begin{pgfscope}%
\pgfsetbuttcap%
\pgfsetroundjoin%
\definecolor{currentfill}{rgb}{0.000000,0.000000,0.000000}%
\pgfsetfillcolor{currentfill}%
\pgfsetlinewidth{0.501875pt}%
\definecolor{currentstroke}{rgb}{0.000000,0.000000,0.000000}%
\pgfsetstrokecolor{currentstroke}%
\pgfsetdash{}{0pt}%
\pgfsys@defobject{currentmarker}{\pgfqpoint{-0.069444in}{0.000000in}}{\pgfqpoint{0.000000in}{0.000000in}}{%
\pgfpathmoveto{\pgfqpoint{0.000000in}{0.000000in}}%
\pgfpathlineto{\pgfqpoint{-0.069444in}{0.000000in}}%
\pgfusepath{stroke,fill}%
}%
\begin{pgfscope}%
\pgfsys@transformshift{2.569584in}{0.603387in}%
\pgfsys@useobject{currentmarker}{}%
\end{pgfscope}%
\end{pgfscope}%
\begin{pgfscope}%
\pgftext[x=0.566911in,y=0.603387in,right,]{\rmfamily\fontsize{8.000000}{9.600000}\selectfont 3}%
\end{pgfscope}%
\begin{pgfscope}%
\pgftext[x=0.333676in,y=0.522171in,,bottom,rotate=90.000000]{\rmfamily\fontsize{9.000000}{10.800000}\selectfont \(\displaystyle \frac{\hat{n}_i -  n_i}{\sigma(n_i)}\)}%
\end{pgfscope}%
\begin{pgfscope}%
\pgfsetbuttcap%
\pgfsetmiterjoin%
\definecolor{currentfill}{rgb}{1.000000,1.000000,1.000000}%
\pgfsetfillcolor{currentfill}%
\pgfsetlinewidth{0.000000pt}%
\definecolor{currentstroke}{rgb}{0.000000,0.000000,0.000000}%
\pgfsetstrokecolor{currentstroke}%
\pgfsetstrokeopacity{0.000000}%
\pgfsetdash{}{0pt}%
\pgfpathmoveto{\pgfqpoint{0.636356in}{0.700846in}}%
\pgfpathlineto{\pgfqpoint{2.569584in}{0.700846in}}%
\pgfpathlineto{\pgfqpoint{2.569584in}{1.837869in}}%
\pgfpathlineto{\pgfqpoint{0.636356in}{1.837869in}}%
\pgfpathclose%
\pgfusepath{fill}%
\end{pgfscope}%
\begin{pgfscope}%
\pgfpathrectangle{\pgfqpoint{0.636356in}{0.700846in}}{\pgfqpoint{1.933229in}{1.137023in}} %
\pgfusepath{clip}%
\pgfsetbuttcap%
\pgfsetroundjoin%
\pgfsetlinewidth{1.003750pt}%
\definecolor{currentstroke}{rgb}{0.000000,0.000000,1.000000}%
\pgfsetstrokecolor{currentstroke}%
\pgfsetdash{{8.000000pt}{3.000000pt}}{0.000000pt}%
\pgfpathmoveto{\pgfqpoint{0.636356in}{0.710409in}}%
\pgfpathlineto{\pgfqpoint{0.714074in}{0.716928in}}%
\pgfpathlineto{\pgfqpoint{0.782077in}{0.725339in}}%
\pgfpathlineto{\pgfqpoint{0.830650in}{0.733204in}}%
\pgfpathlineto{\pgfqpoint{0.888938in}{0.745112in}}%
\pgfpathlineto{\pgfqpoint{0.947227in}{0.759957in}}%
\pgfpathlineto{\pgfqpoint{1.005515in}{0.777896in}}%
\pgfpathlineto{\pgfqpoint{1.063803in}{0.798883in}}%
\pgfpathlineto{\pgfqpoint{1.122092in}{0.822611in}}%
\pgfpathlineto{\pgfqpoint{1.190095in}{0.852940in}}%
\pgfpathlineto{\pgfqpoint{1.364960in}{0.932521in}}%
\pgfpathlineto{\pgfqpoint{1.413533in}{0.951418in}}%
\pgfpathlineto{\pgfqpoint{1.452392in}{0.964433in}}%
\pgfpathlineto{\pgfqpoint{1.491251in}{0.975166in}}%
\pgfpathlineto{\pgfqpoint{1.530110in}{0.983290in}}%
\pgfpathlineto{\pgfqpoint{1.568969in}{0.988554in}}%
\pgfpathlineto{\pgfqpoint{1.598113in}{0.990520in}}%
\pgfpathlineto{\pgfqpoint{1.627257in}{0.990747in}}%
\pgfpathlineto{\pgfqpoint{1.656401in}{0.989233in}}%
\pgfpathlineto{\pgfqpoint{1.695260in}{0.984551in}}%
\pgfpathlineto{\pgfqpoint{1.734119in}{0.976967in}}%
\pgfpathlineto{\pgfqpoint{1.772978in}{0.966720in}}%
\pgfpathlineto{\pgfqpoint{1.811837in}{0.954122in}}%
\pgfpathlineto{\pgfqpoint{1.860410in}{0.935633in}}%
\pgfpathlineto{\pgfqpoint{1.918698in}{0.910542in}}%
\pgfpathlineto{\pgfqpoint{2.015846in}{0.865475in}}%
\pgfpathlineto{\pgfqpoint{2.103278in}{0.825872in}}%
\pgfpathlineto{\pgfqpoint{2.161566in}{0.801825in}}%
\pgfpathlineto{\pgfqpoint{2.219855in}{0.780458in}}%
\pgfpathlineto{\pgfqpoint{2.278143in}{0.762114in}}%
\pgfpathlineto{\pgfqpoint{2.336431in}{0.746872in}}%
\pgfpathlineto{\pgfqpoint{2.394720in}{0.734596in}}%
\pgfpathlineto{\pgfqpoint{2.453008in}{0.725004in}}%
\pgfpathlineto{\pgfqpoint{2.521011in}{0.716733in}}%
\pgfpathlineto{\pgfqpoint{2.569584in}{0.712358in}}%
\pgfpathlineto{\pgfqpoint{2.569584in}{0.712358in}}%
\pgfusepath{stroke}%
\end{pgfscope}%
\begin{pgfscope}%
\pgfpathrectangle{\pgfqpoint{0.636356in}{0.700846in}}{\pgfqpoint{1.933229in}{1.137023in}} %
\pgfusepath{clip}%
\pgfsetbuttcap%
\pgfsetroundjoin%
\pgfsetlinewidth{1.003750pt}%
\definecolor{currentstroke}{rgb}{0.000000,0.500000,0.000000}%
\pgfsetstrokecolor{currentstroke}%
\pgfsetdash{{8.000000pt}{3.000000pt}}{0.000000pt}%
\pgfpathmoveto{\pgfqpoint{0.636356in}{0.700925in}}%
\pgfpathlineto{\pgfqpoint{0.801506in}{0.702204in}}%
\pgfpathlineto{\pgfqpoint{0.869509in}{0.704571in}}%
\pgfpathlineto{\pgfqpoint{0.927797in}{0.709148in}}%
\pgfpathlineto{\pgfqpoint{0.966656in}{0.714507in}}%
\pgfpathlineto{\pgfqpoint{1.005515in}{0.722683in}}%
\pgfpathlineto{\pgfqpoint{1.034659in}{0.731391in}}%
\pgfpathlineto{\pgfqpoint{1.063803in}{0.742648in}}%
\pgfpathlineto{\pgfqpoint{1.092948in}{0.757396in}}%
\pgfpathlineto{\pgfqpoint{1.112377in}{0.769290in}}%
\pgfpathlineto{\pgfqpoint{1.131806in}{0.783089in}}%
\pgfpathlineto{\pgfqpoint{1.151236in}{0.798959in}}%
\pgfpathlineto{\pgfqpoint{1.170665in}{0.817051in}}%
\pgfpathlineto{\pgfqpoint{1.190095in}{0.837489in}}%
\pgfpathlineto{\pgfqpoint{1.209524in}{0.860367in}}%
\pgfpathlineto{\pgfqpoint{1.228954in}{0.885735in}}%
\pgfpathlineto{\pgfqpoint{1.248383in}{0.913597in}}%
\pgfpathlineto{\pgfqpoint{1.277527in}{0.959829in}}%
\pgfpathlineto{\pgfqpoint{1.306671in}{1.011266in}}%
\pgfpathlineto{\pgfqpoint{1.335815in}{1.066729in}}%
\pgfpathlineto{\pgfqpoint{1.384389in}{1.165190in}}%
\pgfpathlineto{\pgfqpoint{1.442677in}{1.283077in}}%
\pgfpathlineto{\pgfqpoint{1.471821in}{1.337145in}}%
\pgfpathlineto{\pgfqpoint{1.491251in}{1.369870in}}%
\pgfpathlineto{\pgfqpoint{1.510680in}{1.399216in}}%
\pgfpathlineto{\pgfqpoint{1.530110in}{1.424602in}}%
\pgfpathlineto{\pgfqpoint{1.549539in}{1.445514in}}%
\pgfpathlineto{\pgfqpoint{1.568969in}{1.461516in}}%
\pgfpathlineto{\pgfqpoint{1.578683in}{1.467557in}}%
\pgfpathlineto{\pgfqpoint{1.588398in}{1.472271in}}%
\pgfpathlineto{\pgfqpoint{1.598113in}{1.475605in}}%
\pgfpathlineto{\pgfqpoint{1.607827in}{1.477550in}}%
\pgfpathlineto{\pgfqpoint{1.617542in}{1.478105in}}%
\pgfpathlineto{\pgfqpoint{1.627257in}{1.477237in}}%
\pgfpathlineto{\pgfqpoint{1.636972in}{1.475003in}}%
\pgfpathlineto{\pgfqpoint{1.646686in}{1.471342in}}%
\pgfpathlineto{\pgfqpoint{1.656401in}{1.466365in}}%
\pgfpathlineto{\pgfqpoint{1.675831in}{1.452376in}}%
\pgfpathlineto{\pgfqpoint{1.695260in}{1.433334in}}%
\pgfpathlineto{\pgfqpoint{1.714689in}{1.409637in}}%
\pgfpathlineto{\pgfqpoint{1.734119in}{1.381770in}}%
\pgfpathlineto{\pgfqpoint{1.753548in}{1.350291in}}%
\pgfpathlineto{\pgfqpoint{1.782692in}{1.297614in}}%
\pgfpathlineto{\pgfqpoint{1.821551in}{1.220706in}}%
\pgfpathlineto{\pgfqpoint{1.908984in}{1.043719in}}%
\pgfpathlineto{\pgfqpoint{1.938128in}{0.989847in}}%
\pgfpathlineto{\pgfqpoint{1.967272in}{0.940352in}}%
\pgfpathlineto{\pgfqpoint{1.996416in}{0.896321in}}%
\pgfpathlineto{\pgfqpoint{2.015846in}{0.869976in}}%
\pgfpathlineto{\pgfqpoint{2.035275in}{0.846130in}}%
\pgfpathlineto{\pgfqpoint{2.054704in}{0.824748in}}%
\pgfpathlineto{\pgfqpoint{2.074134in}{0.805754in}}%
\pgfpathlineto{\pgfqpoint{2.093563in}{0.789032in}}%
\pgfpathlineto{\pgfqpoint{2.112993in}{0.774443in}}%
\pgfpathlineto{\pgfqpoint{2.132422in}{0.761826in}}%
\pgfpathlineto{\pgfqpoint{2.151852in}{0.751008in}}%
\pgfpathlineto{\pgfqpoint{2.171281in}{0.741813in}}%
\pgfpathlineto{\pgfqpoint{2.200425in}{0.730588in}}%
\pgfpathlineto{\pgfqpoint{2.229569in}{0.722216in}}%
\pgfpathlineto{\pgfqpoint{2.258714in}{0.715849in}}%
\pgfpathlineto{\pgfqpoint{2.297572in}{0.710015in}}%
\pgfpathlineto{\pgfqpoint{2.346146in}{0.705632in}}%
\pgfpathlineto{\pgfqpoint{2.404434in}{0.702901in}}%
\pgfpathlineto{\pgfqpoint{2.482152in}{0.701448in}}%
\pgfpathlineto{\pgfqpoint{2.569584in}{0.700976in}}%
\pgfpathlineto{\pgfqpoint{2.569584in}{0.700976in}}%
\pgfusepath{stroke}%
\end{pgfscope}%
\begin{pgfscope}%
\pgfsetrectcap%
\pgfsetmiterjoin%
\pgfsetlinewidth{1.003750pt}%
\definecolor{currentstroke}{rgb}{0.000000,0.000000,0.000000}%
\pgfsetstrokecolor{currentstroke}%
\pgfsetdash{}{0pt}%
\pgfpathmoveto{\pgfqpoint{0.636356in}{1.837869in}}%
\pgfpathlineto{\pgfqpoint{2.569584in}{1.837869in}}%
\pgfusepath{stroke}%
\end{pgfscope}%
\begin{pgfscope}%
\pgfsetrectcap%
\pgfsetmiterjoin%
\pgfsetlinewidth{1.003750pt}%
\definecolor{currentstroke}{rgb}{0.000000,0.000000,0.000000}%
\pgfsetstrokecolor{currentstroke}%
\pgfsetdash{}{0pt}%
\pgfpathmoveto{\pgfqpoint{2.569584in}{0.700846in}}%
\pgfpathlineto{\pgfqpoint{2.569584in}{1.837869in}}%
\pgfusepath{stroke}%
\end{pgfscope}%
\begin{pgfscope}%
\pgfsetrectcap%
\pgfsetmiterjoin%
\pgfsetlinewidth{1.003750pt}%
\definecolor{currentstroke}{rgb}{0.000000,0.000000,0.000000}%
\pgfsetstrokecolor{currentstroke}%
\pgfsetdash{}{0pt}%
\pgfpathmoveto{\pgfqpoint{0.636356in}{0.700846in}}%
\pgfpathlineto{\pgfqpoint{2.569584in}{0.700846in}}%
\pgfusepath{stroke}%
\end{pgfscope}%
\begin{pgfscope}%
\pgfsetrectcap%
\pgfsetmiterjoin%
\pgfsetlinewidth{1.003750pt}%
\definecolor{currentstroke}{rgb}{0.000000,0.000000,0.000000}%
\pgfsetstrokecolor{currentstroke}%
\pgfsetdash{}{0pt}%
\pgfpathmoveto{\pgfqpoint{0.636356in}{0.700846in}}%
\pgfpathlineto{\pgfqpoint{0.636356in}{1.837869in}}%
\pgfusepath{stroke}%
\end{pgfscope}%
\begin{pgfscope}%
\pgfsetbuttcap%
\pgfsetroundjoin%
\definecolor{currentfill}{rgb}{0.000000,0.000000,0.000000}%
\pgfsetfillcolor{currentfill}%
\pgfsetlinewidth{0.501875pt}%
\definecolor{currentstroke}{rgb}{0.000000,0.000000,0.000000}%
\pgfsetstrokecolor{currentstroke}%
\pgfsetdash{}{0pt}%
\pgfsys@defobject{currentmarker}{\pgfqpoint{0.000000in}{0.000000in}}{\pgfqpoint{0.000000in}{0.069444in}}{%
\pgfpathmoveto{\pgfqpoint{0.000000in}{0.000000in}}%
\pgfpathlineto{\pgfqpoint{0.000000in}{0.069444in}}%
\pgfusepath{stroke,fill}%
}%
\begin{pgfscope}%
\pgfsys@transformshift{0.899978in}{0.700846in}%
\pgfsys@useobject{currentmarker}{}%
\end{pgfscope}%
\end{pgfscope}%
\begin{pgfscope}%
\pgfsetbuttcap%
\pgfsetroundjoin%
\definecolor{currentfill}{rgb}{0.000000,0.000000,0.000000}%
\pgfsetfillcolor{currentfill}%
\pgfsetlinewidth{0.501875pt}%
\definecolor{currentstroke}{rgb}{0.000000,0.000000,0.000000}%
\pgfsetstrokecolor{currentstroke}%
\pgfsetdash{}{0pt}%
\pgfsys@defobject{currentmarker}{\pgfqpoint{0.000000in}{-0.069444in}}{\pgfqpoint{0.000000in}{0.000000in}}{%
\pgfpathmoveto{\pgfqpoint{0.000000in}{0.000000in}}%
\pgfpathlineto{\pgfqpoint{0.000000in}{-0.069444in}}%
\pgfusepath{stroke,fill}%
}%
\begin{pgfscope}%
\pgfsys@transformshift{0.899978in}{1.837869in}%
\pgfsys@useobject{currentmarker}{}%
\end{pgfscope}%
\end{pgfscope}%
\begin{pgfscope}%
\pgfsetbuttcap%
\pgfsetroundjoin%
\definecolor{currentfill}{rgb}{0.000000,0.000000,0.000000}%
\pgfsetfillcolor{currentfill}%
\pgfsetlinewidth{0.501875pt}%
\definecolor{currentstroke}{rgb}{0.000000,0.000000,0.000000}%
\pgfsetstrokecolor{currentstroke}%
\pgfsetdash{}{0pt}%
\pgfsys@defobject{currentmarker}{\pgfqpoint{0.000000in}{0.000000in}}{\pgfqpoint{0.000000in}{0.069444in}}{%
\pgfpathmoveto{\pgfqpoint{0.000000in}{0.000000in}}%
\pgfpathlineto{\pgfqpoint{0.000000in}{0.069444in}}%
\pgfusepath{stroke,fill}%
}%
\begin{pgfscope}%
\pgfsys@transformshift{1.251474in}{0.700846in}%
\pgfsys@useobject{currentmarker}{}%
\end{pgfscope}%
\end{pgfscope}%
\begin{pgfscope}%
\pgfsetbuttcap%
\pgfsetroundjoin%
\definecolor{currentfill}{rgb}{0.000000,0.000000,0.000000}%
\pgfsetfillcolor{currentfill}%
\pgfsetlinewidth{0.501875pt}%
\definecolor{currentstroke}{rgb}{0.000000,0.000000,0.000000}%
\pgfsetstrokecolor{currentstroke}%
\pgfsetdash{}{0pt}%
\pgfsys@defobject{currentmarker}{\pgfqpoint{0.000000in}{-0.069444in}}{\pgfqpoint{0.000000in}{0.000000in}}{%
\pgfpathmoveto{\pgfqpoint{0.000000in}{0.000000in}}%
\pgfpathlineto{\pgfqpoint{0.000000in}{-0.069444in}}%
\pgfusepath{stroke,fill}%
}%
\begin{pgfscope}%
\pgfsys@transformshift{1.251474in}{1.837869in}%
\pgfsys@useobject{currentmarker}{}%
\end{pgfscope}%
\end{pgfscope}%
\begin{pgfscope}%
\pgfsetbuttcap%
\pgfsetroundjoin%
\definecolor{currentfill}{rgb}{0.000000,0.000000,0.000000}%
\pgfsetfillcolor{currentfill}%
\pgfsetlinewidth{0.501875pt}%
\definecolor{currentstroke}{rgb}{0.000000,0.000000,0.000000}%
\pgfsetstrokecolor{currentstroke}%
\pgfsetdash{}{0pt}%
\pgfsys@defobject{currentmarker}{\pgfqpoint{0.000000in}{0.000000in}}{\pgfqpoint{0.000000in}{0.069444in}}{%
\pgfpathmoveto{\pgfqpoint{0.000000in}{0.000000in}}%
\pgfpathlineto{\pgfqpoint{0.000000in}{0.069444in}}%
\pgfusepath{stroke,fill}%
}%
\begin{pgfscope}%
\pgfsys@transformshift{1.602970in}{0.700846in}%
\pgfsys@useobject{currentmarker}{}%
\end{pgfscope}%
\end{pgfscope}%
\begin{pgfscope}%
\pgfsetbuttcap%
\pgfsetroundjoin%
\definecolor{currentfill}{rgb}{0.000000,0.000000,0.000000}%
\pgfsetfillcolor{currentfill}%
\pgfsetlinewidth{0.501875pt}%
\definecolor{currentstroke}{rgb}{0.000000,0.000000,0.000000}%
\pgfsetstrokecolor{currentstroke}%
\pgfsetdash{}{0pt}%
\pgfsys@defobject{currentmarker}{\pgfqpoint{0.000000in}{-0.069444in}}{\pgfqpoint{0.000000in}{0.000000in}}{%
\pgfpathmoveto{\pgfqpoint{0.000000in}{0.000000in}}%
\pgfpathlineto{\pgfqpoint{0.000000in}{-0.069444in}}%
\pgfusepath{stroke,fill}%
}%
\begin{pgfscope}%
\pgfsys@transformshift{1.602970in}{1.837869in}%
\pgfsys@useobject{currentmarker}{}%
\end{pgfscope}%
\end{pgfscope}%
\begin{pgfscope}%
\pgfsetbuttcap%
\pgfsetroundjoin%
\definecolor{currentfill}{rgb}{0.000000,0.000000,0.000000}%
\pgfsetfillcolor{currentfill}%
\pgfsetlinewidth{0.501875pt}%
\definecolor{currentstroke}{rgb}{0.000000,0.000000,0.000000}%
\pgfsetstrokecolor{currentstroke}%
\pgfsetdash{}{0pt}%
\pgfsys@defobject{currentmarker}{\pgfqpoint{0.000000in}{0.000000in}}{\pgfqpoint{0.000000in}{0.069444in}}{%
\pgfpathmoveto{\pgfqpoint{0.000000in}{0.000000in}}%
\pgfpathlineto{\pgfqpoint{0.000000in}{0.069444in}}%
\pgfusepath{stroke,fill}%
}%
\begin{pgfscope}%
\pgfsys@transformshift{1.954466in}{0.700846in}%
\pgfsys@useobject{currentmarker}{}%
\end{pgfscope}%
\end{pgfscope}%
\begin{pgfscope}%
\pgfsetbuttcap%
\pgfsetroundjoin%
\definecolor{currentfill}{rgb}{0.000000,0.000000,0.000000}%
\pgfsetfillcolor{currentfill}%
\pgfsetlinewidth{0.501875pt}%
\definecolor{currentstroke}{rgb}{0.000000,0.000000,0.000000}%
\pgfsetstrokecolor{currentstroke}%
\pgfsetdash{}{0pt}%
\pgfsys@defobject{currentmarker}{\pgfqpoint{0.000000in}{-0.069444in}}{\pgfqpoint{0.000000in}{0.000000in}}{%
\pgfpathmoveto{\pgfqpoint{0.000000in}{0.000000in}}%
\pgfpathlineto{\pgfqpoint{0.000000in}{-0.069444in}}%
\pgfusepath{stroke,fill}%
}%
\begin{pgfscope}%
\pgfsys@transformshift{1.954466in}{1.837869in}%
\pgfsys@useobject{currentmarker}{}%
\end{pgfscope}%
\end{pgfscope}%
\begin{pgfscope}%
\pgfsetbuttcap%
\pgfsetroundjoin%
\definecolor{currentfill}{rgb}{0.000000,0.000000,0.000000}%
\pgfsetfillcolor{currentfill}%
\pgfsetlinewidth{0.501875pt}%
\definecolor{currentstroke}{rgb}{0.000000,0.000000,0.000000}%
\pgfsetstrokecolor{currentstroke}%
\pgfsetdash{}{0pt}%
\pgfsys@defobject{currentmarker}{\pgfqpoint{0.000000in}{0.000000in}}{\pgfqpoint{0.000000in}{0.069444in}}{%
\pgfpathmoveto{\pgfqpoint{0.000000in}{0.000000in}}%
\pgfpathlineto{\pgfqpoint{0.000000in}{0.069444in}}%
\pgfusepath{stroke,fill}%
}%
\begin{pgfscope}%
\pgfsys@transformshift{2.305962in}{0.700846in}%
\pgfsys@useobject{currentmarker}{}%
\end{pgfscope}%
\end{pgfscope}%
\begin{pgfscope}%
\pgfsetbuttcap%
\pgfsetroundjoin%
\definecolor{currentfill}{rgb}{0.000000,0.000000,0.000000}%
\pgfsetfillcolor{currentfill}%
\pgfsetlinewidth{0.501875pt}%
\definecolor{currentstroke}{rgb}{0.000000,0.000000,0.000000}%
\pgfsetstrokecolor{currentstroke}%
\pgfsetdash{}{0pt}%
\pgfsys@defobject{currentmarker}{\pgfqpoint{0.000000in}{-0.069444in}}{\pgfqpoint{0.000000in}{0.000000in}}{%
\pgfpathmoveto{\pgfqpoint{0.000000in}{0.000000in}}%
\pgfpathlineto{\pgfqpoint{0.000000in}{-0.069444in}}%
\pgfusepath{stroke,fill}%
}%
\begin{pgfscope}%
\pgfsys@transformshift{2.305962in}{1.837869in}%
\pgfsys@useobject{currentmarker}{}%
\end{pgfscope}%
\end{pgfscope}%
\begin{pgfscope}%
\pgfsetbuttcap%
\pgfsetroundjoin%
\definecolor{currentfill}{rgb}{0.000000,0.000000,0.000000}%
\pgfsetfillcolor{currentfill}%
\pgfsetlinewidth{0.501875pt}%
\definecolor{currentstroke}{rgb}{0.000000,0.000000,0.000000}%
\pgfsetstrokecolor{currentstroke}%
\pgfsetdash{}{0pt}%
\pgfsys@defobject{currentmarker}{\pgfqpoint{0.000000in}{0.000000in}}{\pgfqpoint{0.069444in}{0.000000in}}{%
\pgfpathmoveto{\pgfqpoint{0.000000in}{0.000000in}}%
\pgfpathlineto{\pgfqpoint{0.069444in}{0.000000in}}%
\pgfusepath{stroke,fill}%
}%
\begin{pgfscope}%
\pgfsys@transformshift{0.636356in}{0.700846in}%
\pgfsys@useobject{currentmarker}{}%
\end{pgfscope}%
\end{pgfscope}%
\begin{pgfscope}%
\pgfsetbuttcap%
\pgfsetroundjoin%
\definecolor{currentfill}{rgb}{0.000000,0.000000,0.000000}%
\pgfsetfillcolor{currentfill}%
\pgfsetlinewidth{0.501875pt}%
\definecolor{currentstroke}{rgb}{0.000000,0.000000,0.000000}%
\pgfsetstrokecolor{currentstroke}%
\pgfsetdash{}{0pt}%
\pgfsys@defobject{currentmarker}{\pgfqpoint{-0.069444in}{0.000000in}}{\pgfqpoint{0.000000in}{0.000000in}}{%
\pgfpathmoveto{\pgfqpoint{0.000000in}{0.000000in}}%
\pgfpathlineto{\pgfqpoint{-0.069444in}{0.000000in}}%
\pgfusepath{stroke,fill}%
}%
\begin{pgfscope}%
\pgfsys@transformshift{2.569584in}{0.700846in}%
\pgfsys@useobject{currentmarker}{}%
\end{pgfscope}%
\end{pgfscope}%
\begin{pgfscope}%
\pgftext[x=0.566911in,y=0.700846in,right,]{\rmfamily\fontsize{8.000000}{9.600000}\selectfont 0}%
\end{pgfscope}%
\begin{pgfscope}%
\pgfsetbuttcap%
\pgfsetroundjoin%
\definecolor{currentfill}{rgb}{0.000000,0.000000,0.000000}%
\pgfsetfillcolor{currentfill}%
\pgfsetlinewidth{0.501875pt}%
\definecolor{currentstroke}{rgb}{0.000000,0.000000,0.000000}%
\pgfsetstrokecolor{currentstroke}%
\pgfsetdash{}{0pt}%
\pgfsys@defobject{currentmarker}{\pgfqpoint{0.000000in}{0.000000in}}{\pgfqpoint{0.069444in}{0.000000in}}{%
\pgfpathmoveto{\pgfqpoint{0.000000in}{0.000000in}}%
\pgfpathlineto{\pgfqpoint{0.069444in}{0.000000in}}%
\pgfusepath{stroke,fill}%
}%
\begin{pgfscope}%
\pgfsys@transformshift{0.636356in}{0.827182in}%
\pgfsys@useobject{currentmarker}{}%
\end{pgfscope}%
\end{pgfscope}%
\begin{pgfscope}%
\pgfsetbuttcap%
\pgfsetroundjoin%
\definecolor{currentfill}{rgb}{0.000000,0.000000,0.000000}%
\pgfsetfillcolor{currentfill}%
\pgfsetlinewidth{0.501875pt}%
\definecolor{currentstroke}{rgb}{0.000000,0.000000,0.000000}%
\pgfsetstrokecolor{currentstroke}%
\pgfsetdash{}{0pt}%
\pgfsys@defobject{currentmarker}{\pgfqpoint{-0.069444in}{0.000000in}}{\pgfqpoint{0.000000in}{0.000000in}}{%
\pgfpathmoveto{\pgfqpoint{0.000000in}{0.000000in}}%
\pgfpathlineto{\pgfqpoint{-0.069444in}{0.000000in}}%
\pgfusepath{stroke,fill}%
}%
\begin{pgfscope}%
\pgfsys@transformshift{2.569584in}{0.827182in}%
\pgfsys@useobject{currentmarker}{}%
\end{pgfscope}%
\end{pgfscope}%
\begin{pgfscope}%
\pgftext[x=0.566911in,y=0.827182in,right,]{\rmfamily\fontsize{8.000000}{9.600000}\selectfont 200}%
\end{pgfscope}%
\begin{pgfscope}%
\pgfsetbuttcap%
\pgfsetroundjoin%
\definecolor{currentfill}{rgb}{0.000000,0.000000,0.000000}%
\pgfsetfillcolor{currentfill}%
\pgfsetlinewidth{0.501875pt}%
\definecolor{currentstroke}{rgb}{0.000000,0.000000,0.000000}%
\pgfsetstrokecolor{currentstroke}%
\pgfsetdash{}{0pt}%
\pgfsys@defobject{currentmarker}{\pgfqpoint{0.000000in}{0.000000in}}{\pgfqpoint{0.069444in}{0.000000in}}{%
\pgfpathmoveto{\pgfqpoint{0.000000in}{0.000000in}}%
\pgfpathlineto{\pgfqpoint{0.069444in}{0.000000in}}%
\pgfusepath{stroke,fill}%
}%
\begin{pgfscope}%
\pgfsys@transformshift{0.636356in}{0.953518in}%
\pgfsys@useobject{currentmarker}{}%
\end{pgfscope}%
\end{pgfscope}%
\begin{pgfscope}%
\pgfsetbuttcap%
\pgfsetroundjoin%
\definecolor{currentfill}{rgb}{0.000000,0.000000,0.000000}%
\pgfsetfillcolor{currentfill}%
\pgfsetlinewidth{0.501875pt}%
\definecolor{currentstroke}{rgb}{0.000000,0.000000,0.000000}%
\pgfsetstrokecolor{currentstroke}%
\pgfsetdash{}{0pt}%
\pgfsys@defobject{currentmarker}{\pgfqpoint{-0.069444in}{0.000000in}}{\pgfqpoint{0.000000in}{0.000000in}}{%
\pgfpathmoveto{\pgfqpoint{0.000000in}{0.000000in}}%
\pgfpathlineto{\pgfqpoint{-0.069444in}{0.000000in}}%
\pgfusepath{stroke,fill}%
}%
\begin{pgfscope}%
\pgfsys@transformshift{2.569584in}{0.953518in}%
\pgfsys@useobject{currentmarker}{}%
\end{pgfscope}%
\end{pgfscope}%
\begin{pgfscope}%
\pgftext[x=0.566911in,y=0.953518in,right,]{\rmfamily\fontsize{8.000000}{9.600000}\selectfont 400}%
\end{pgfscope}%
\begin{pgfscope}%
\pgfsetbuttcap%
\pgfsetroundjoin%
\definecolor{currentfill}{rgb}{0.000000,0.000000,0.000000}%
\pgfsetfillcolor{currentfill}%
\pgfsetlinewidth{0.501875pt}%
\definecolor{currentstroke}{rgb}{0.000000,0.000000,0.000000}%
\pgfsetstrokecolor{currentstroke}%
\pgfsetdash{}{0pt}%
\pgfsys@defobject{currentmarker}{\pgfqpoint{0.000000in}{0.000000in}}{\pgfqpoint{0.069444in}{0.000000in}}{%
\pgfpathmoveto{\pgfqpoint{0.000000in}{0.000000in}}%
\pgfpathlineto{\pgfqpoint{0.069444in}{0.000000in}}%
\pgfusepath{stroke,fill}%
}%
\begin{pgfscope}%
\pgfsys@transformshift{0.636356in}{1.079854in}%
\pgfsys@useobject{currentmarker}{}%
\end{pgfscope}%
\end{pgfscope}%
\begin{pgfscope}%
\pgfsetbuttcap%
\pgfsetroundjoin%
\definecolor{currentfill}{rgb}{0.000000,0.000000,0.000000}%
\pgfsetfillcolor{currentfill}%
\pgfsetlinewidth{0.501875pt}%
\definecolor{currentstroke}{rgb}{0.000000,0.000000,0.000000}%
\pgfsetstrokecolor{currentstroke}%
\pgfsetdash{}{0pt}%
\pgfsys@defobject{currentmarker}{\pgfqpoint{-0.069444in}{0.000000in}}{\pgfqpoint{0.000000in}{0.000000in}}{%
\pgfpathmoveto{\pgfqpoint{0.000000in}{0.000000in}}%
\pgfpathlineto{\pgfqpoint{-0.069444in}{0.000000in}}%
\pgfusepath{stroke,fill}%
}%
\begin{pgfscope}%
\pgfsys@transformshift{2.569584in}{1.079854in}%
\pgfsys@useobject{currentmarker}{}%
\end{pgfscope}%
\end{pgfscope}%
\begin{pgfscope}%
\pgftext[x=0.566911in,y=1.079854in,right,]{\rmfamily\fontsize{8.000000}{9.600000}\selectfont 600}%
\end{pgfscope}%
\begin{pgfscope}%
\pgfsetbuttcap%
\pgfsetroundjoin%
\definecolor{currentfill}{rgb}{0.000000,0.000000,0.000000}%
\pgfsetfillcolor{currentfill}%
\pgfsetlinewidth{0.501875pt}%
\definecolor{currentstroke}{rgb}{0.000000,0.000000,0.000000}%
\pgfsetstrokecolor{currentstroke}%
\pgfsetdash{}{0pt}%
\pgfsys@defobject{currentmarker}{\pgfqpoint{0.000000in}{0.000000in}}{\pgfqpoint{0.069444in}{0.000000in}}{%
\pgfpathmoveto{\pgfqpoint{0.000000in}{0.000000in}}%
\pgfpathlineto{\pgfqpoint{0.069444in}{0.000000in}}%
\pgfusepath{stroke,fill}%
}%
\begin{pgfscope}%
\pgfsys@transformshift{0.636356in}{1.206190in}%
\pgfsys@useobject{currentmarker}{}%
\end{pgfscope}%
\end{pgfscope}%
\begin{pgfscope}%
\pgfsetbuttcap%
\pgfsetroundjoin%
\definecolor{currentfill}{rgb}{0.000000,0.000000,0.000000}%
\pgfsetfillcolor{currentfill}%
\pgfsetlinewidth{0.501875pt}%
\definecolor{currentstroke}{rgb}{0.000000,0.000000,0.000000}%
\pgfsetstrokecolor{currentstroke}%
\pgfsetdash{}{0pt}%
\pgfsys@defobject{currentmarker}{\pgfqpoint{-0.069444in}{0.000000in}}{\pgfqpoint{0.000000in}{0.000000in}}{%
\pgfpathmoveto{\pgfqpoint{0.000000in}{0.000000in}}%
\pgfpathlineto{\pgfqpoint{-0.069444in}{0.000000in}}%
\pgfusepath{stroke,fill}%
}%
\begin{pgfscope}%
\pgfsys@transformshift{2.569584in}{1.206190in}%
\pgfsys@useobject{currentmarker}{}%
\end{pgfscope}%
\end{pgfscope}%
\begin{pgfscope}%
\pgftext[x=0.566911in,y=1.206190in,right,]{\rmfamily\fontsize{8.000000}{9.600000}\selectfont 800}%
\end{pgfscope}%
\begin{pgfscope}%
\pgfsetbuttcap%
\pgfsetroundjoin%
\definecolor{currentfill}{rgb}{0.000000,0.000000,0.000000}%
\pgfsetfillcolor{currentfill}%
\pgfsetlinewidth{0.501875pt}%
\definecolor{currentstroke}{rgb}{0.000000,0.000000,0.000000}%
\pgfsetstrokecolor{currentstroke}%
\pgfsetdash{}{0pt}%
\pgfsys@defobject{currentmarker}{\pgfqpoint{0.000000in}{0.000000in}}{\pgfqpoint{0.069444in}{0.000000in}}{%
\pgfpathmoveto{\pgfqpoint{0.000000in}{0.000000in}}%
\pgfpathlineto{\pgfqpoint{0.069444in}{0.000000in}}%
\pgfusepath{stroke,fill}%
}%
\begin{pgfscope}%
\pgfsys@transformshift{0.636356in}{1.332526in}%
\pgfsys@useobject{currentmarker}{}%
\end{pgfscope}%
\end{pgfscope}%
\begin{pgfscope}%
\pgfsetbuttcap%
\pgfsetroundjoin%
\definecolor{currentfill}{rgb}{0.000000,0.000000,0.000000}%
\pgfsetfillcolor{currentfill}%
\pgfsetlinewidth{0.501875pt}%
\definecolor{currentstroke}{rgb}{0.000000,0.000000,0.000000}%
\pgfsetstrokecolor{currentstroke}%
\pgfsetdash{}{0pt}%
\pgfsys@defobject{currentmarker}{\pgfqpoint{-0.069444in}{0.000000in}}{\pgfqpoint{0.000000in}{0.000000in}}{%
\pgfpathmoveto{\pgfqpoint{0.000000in}{0.000000in}}%
\pgfpathlineto{\pgfqpoint{-0.069444in}{0.000000in}}%
\pgfusepath{stroke,fill}%
}%
\begin{pgfscope}%
\pgfsys@transformshift{2.569584in}{1.332526in}%
\pgfsys@useobject{currentmarker}{}%
\end{pgfscope}%
\end{pgfscope}%
\begin{pgfscope}%
\pgftext[x=0.566911in,y=1.332526in,right,]{\rmfamily\fontsize{8.000000}{9.600000}\selectfont 1000}%
\end{pgfscope}%
\begin{pgfscope}%
\pgfsetbuttcap%
\pgfsetroundjoin%
\definecolor{currentfill}{rgb}{0.000000,0.000000,0.000000}%
\pgfsetfillcolor{currentfill}%
\pgfsetlinewidth{0.501875pt}%
\definecolor{currentstroke}{rgb}{0.000000,0.000000,0.000000}%
\pgfsetstrokecolor{currentstroke}%
\pgfsetdash{}{0pt}%
\pgfsys@defobject{currentmarker}{\pgfqpoint{0.000000in}{0.000000in}}{\pgfqpoint{0.069444in}{0.000000in}}{%
\pgfpathmoveto{\pgfqpoint{0.000000in}{0.000000in}}%
\pgfpathlineto{\pgfqpoint{0.069444in}{0.000000in}}%
\pgfusepath{stroke,fill}%
}%
\begin{pgfscope}%
\pgfsys@transformshift{0.636356in}{1.458861in}%
\pgfsys@useobject{currentmarker}{}%
\end{pgfscope}%
\end{pgfscope}%
\begin{pgfscope}%
\pgfsetbuttcap%
\pgfsetroundjoin%
\definecolor{currentfill}{rgb}{0.000000,0.000000,0.000000}%
\pgfsetfillcolor{currentfill}%
\pgfsetlinewidth{0.501875pt}%
\definecolor{currentstroke}{rgb}{0.000000,0.000000,0.000000}%
\pgfsetstrokecolor{currentstroke}%
\pgfsetdash{}{0pt}%
\pgfsys@defobject{currentmarker}{\pgfqpoint{-0.069444in}{0.000000in}}{\pgfqpoint{0.000000in}{0.000000in}}{%
\pgfpathmoveto{\pgfqpoint{0.000000in}{0.000000in}}%
\pgfpathlineto{\pgfqpoint{-0.069444in}{0.000000in}}%
\pgfusepath{stroke,fill}%
}%
\begin{pgfscope}%
\pgfsys@transformshift{2.569584in}{1.458861in}%
\pgfsys@useobject{currentmarker}{}%
\end{pgfscope}%
\end{pgfscope}%
\begin{pgfscope}%
\pgftext[x=0.566911in,y=1.458861in,right,]{\rmfamily\fontsize{8.000000}{9.600000}\selectfont 1200}%
\end{pgfscope}%
\begin{pgfscope}%
\pgfsetbuttcap%
\pgfsetroundjoin%
\definecolor{currentfill}{rgb}{0.000000,0.000000,0.000000}%
\pgfsetfillcolor{currentfill}%
\pgfsetlinewidth{0.501875pt}%
\definecolor{currentstroke}{rgb}{0.000000,0.000000,0.000000}%
\pgfsetstrokecolor{currentstroke}%
\pgfsetdash{}{0pt}%
\pgfsys@defobject{currentmarker}{\pgfqpoint{0.000000in}{0.000000in}}{\pgfqpoint{0.069444in}{0.000000in}}{%
\pgfpathmoveto{\pgfqpoint{0.000000in}{0.000000in}}%
\pgfpathlineto{\pgfqpoint{0.069444in}{0.000000in}}%
\pgfusepath{stroke,fill}%
}%
\begin{pgfscope}%
\pgfsys@transformshift{0.636356in}{1.585197in}%
\pgfsys@useobject{currentmarker}{}%
\end{pgfscope}%
\end{pgfscope}%
\begin{pgfscope}%
\pgfsetbuttcap%
\pgfsetroundjoin%
\definecolor{currentfill}{rgb}{0.000000,0.000000,0.000000}%
\pgfsetfillcolor{currentfill}%
\pgfsetlinewidth{0.501875pt}%
\definecolor{currentstroke}{rgb}{0.000000,0.000000,0.000000}%
\pgfsetstrokecolor{currentstroke}%
\pgfsetdash{}{0pt}%
\pgfsys@defobject{currentmarker}{\pgfqpoint{-0.069444in}{0.000000in}}{\pgfqpoint{0.000000in}{0.000000in}}{%
\pgfpathmoveto{\pgfqpoint{0.000000in}{0.000000in}}%
\pgfpathlineto{\pgfqpoint{-0.069444in}{0.000000in}}%
\pgfusepath{stroke,fill}%
}%
\begin{pgfscope}%
\pgfsys@transformshift{2.569584in}{1.585197in}%
\pgfsys@useobject{currentmarker}{}%
\end{pgfscope}%
\end{pgfscope}%
\begin{pgfscope}%
\pgftext[x=0.566911in,y=1.585197in,right,]{\rmfamily\fontsize{8.000000}{9.600000}\selectfont 1400}%
\end{pgfscope}%
\begin{pgfscope}%
\pgfsetbuttcap%
\pgfsetroundjoin%
\definecolor{currentfill}{rgb}{0.000000,0.000000,0.000000}%
\pgfsetfillcolor{currentfill}%
\pgfsetlinewidth{0.501875pt}%
\definecolor{currentstroke}{rgb}{0.000000,0.000000,0.000000}%
\pgfsetstrokecolor{currentstroke}%
\pgfsetdash{}{0pt}%
\pgfsys@defobject{currentmarker}{\pgfqpoint{0.000000in}{0.000000in}}{\pgfqpoint{0.069444in}{0.000000in}}{%
\pgfpathmoveto{\pgfqpoint{0.000000in}{0.000000in}}%
\pgfpathlineto{\pgfqpoint{0.069444in}{0.000000in}}%
\pgfusepath{stroke,fill}%
}%
\begin{pgfscope}%
\pgfsys@transformshift{0.636356in}{1.711533in}%
\pgfsys@useobject{currentmarker}{}%
\end{pgfscope}%
\end{pgfscope}%
\begin{pgfscope}%
\pgfsetbuttcap%
\pgfsetroundjoin%
\definecolor{currentfill}{rgb}{0.000000,0.000000,0.000000}%
\pgfsetfillcolor{currentfill}%
\pgfsetlinewidth{0.501875pt}%
\definecolor{currentstroke}{rgb}{0.000000,0.000000,0.000000}%
\pgfsetstrokecolor{currentstroke}%
\pgfsetdash{}{0pt}%
\pgfsys@defobject{currentmarker}{\pgfqpoint{-0.069444in}{0.000000in}}{\pgfqpoint{0.000000in}{0.000000in}}{%
\pgfpathmoveto{\pgfqpoint{0.000000in}{0.000000in}}%
\pgfpathlineto{\pgfqpoint{-0.069444in}{0.000000in}}%
\pgfusepath{stroke,fill}%
}%
\begin{pgfscope}%
\pgfsys@transformshift{2.569584in}{1.711533in}%
\pgfsys@useobject{currentmarker}{}%
\end{pgfscope}%
\end{pgfscope}%
\begin{pgfscope}%
\pgftext[x=0.566911in,y=1.711533in,right,]{\rmfamily\fontsize{8.000000}{9.600000}\selectfont 1600}%
\end{pgfscope}%
\begin{pgfscope}%
\pgfsetbuttcap%
\pgfsetroundjoin%
\definecolor{currentfill}{rgb}{0.000000,0.000000,0.000000}%
\pgfsetfillcolor{currentfill}%
\pgfsetlinewidth{0.501875pt}%
\definecolor{currentstroke}{rgb}{0.000000,0.000000,0.000000}%
\pgfsetstrokecolor{currentstroke}%
\pgfsetdash{}{0pt}%
\pgfsys@defobject{currentmarker}{\pgfqpoint{0.000000in}{0.000000in}}{\pgfqpoint{0.069444in}{0.000000in}}{%
\pgfpathmoveto{\pgfqpoint{0.000000in}{0.000000in}}%
\pgfpathlineto{\pgfqpoint{0.069444in}{0.000000in}}%
\pgfusepath{stroke,fill}%
}%
\begin{pgfscope}%
\pgfsys@transformshift{0.636356in}{1.837869in}%
\pgfsys@useobject{currentmarker}{}%
\end{pgfscope}%
\end{pgfscope}%
\begin{pgfscope}%
\pgfsetbuttcap%
\pgfsetroundjoin%
\definecolor{currentfill}{rgb}{0.000000,0.000000,0.000000}%
\pgfsetfillcolor{currentfill}%
\pgfsetlinewidth{0.501875pt}%
\definecolor{currentstroke}{rgb}{0.000000,0.000000,0.000000}%
\pgfsetstrokecolor{currentstroke}%
\pgfsetdash{}{0pt}%
\pgfsys@defobject{currentmarker}{\pgfqpoint{-0.069444in}{0.000000in}}{\pgfqpoint{0.000000in}{0.000000in}}{%
\pgfpathmoveto{\pgfqpoint{0.000000in}{0.000000in}}%
\pgfpathlineto{\pgfqpoint{-0.069444in}{0.000000in}}%
\pgfusepath{stroke,fill}%
}%
\begin{pgfscope}%
\pgfsys@transformshift{2.569584in}{1.837869in}%
\pgfsys@useobject{currentmarker}{}%
\end{pgfscope}%
\end{pgfscope}%
\begin{pgfscope}%
\pgftext[x=0.566911in,y=1.837869in,right,]{\rmfamily\fontsize{8.000000}{9.600000}\selectfont 1800}%
\end{pgfscope}%
\begin{pgfscope}%
\pgftext[x=0.214698in,y=1.269358in,,bottom,rotate=90.000000]{\rmfamily\fontsize{9.000000}{10.800000}\selectfont Candidates}%
\end{pgfscope}%
\begin{pgfscope}%
\pgfsetrectcap%
\pgfsetroundjoin%
\pgfsetlinewidth{1.003750pt}%
\definecolor{currentstroke}{rgb}{1.000000,0.000000,0.000000}%
\pgfsetstrokecolor{currentstroke}%
\pgfsetdash{}{0pt}%
\pgfpathmoveto{\pgfqpoint{0.636356in}{0.710487in}}%
\pgfpathlineto{\pgfqpoint{0.704359in}{0.716248in}}%
\pgfpathlineto{\pgfqpoint{0.752932in}{0.721984in}}%
\pgfpathlineto{\pgfqpoint{0.811221in}{0.731405in}}%
\pgfpathlineto{\pgfqpoint{0.859794in}{0.742097in}}%
\pgfpathlineto{\pgfqpoint{0.898653in}{0.753043in}}%
\pgfpathlineto{\pgfqpoint{0.927797in}{0.762970in}}%
\pgfpathlineto{\pgfqpoint{0.966656in}{0.779251in}}%
\pgfpathlineto{\pgfqpoint{0.995800in}{0.794257in}}%
\pgfpathlineto{\pgfqpoint{1.024944in}{0.811873in}}%
\pgfpathlineto{\pgfqpoint{1.054089in}{0.832983in}}%
\pgfpathlineto{\pgfqpoint{1.073518in}{0.849068in}}%
\pgfpathlineto{\pgfqpoint{1.092948in}{0.867004in}}%
\pgfpathlineto{\pgfqpoint{1.112377in}{0.886948in}}%
\pgfpathlineto{\pgfqpoint{1.131806in}{0.909051in}}%
\pgfpathlineto{\pgfqpoint{1.151236in}{0.933451in}}%
\pgfpathlineto{\pgfqpoint{1.170665in}{0.960265in}}%
\pgfpathlineto{\pgfqpoint{1.190095in}{0.989583in}}%
\pgfpathlineto{\pgfqpoint{1.209524in}{1.021459in}}%
\pgfpathlineto{\pgfqpoint{1.238668in}{1.073955in}}%
\pgfpathlineto{\pgfqpoint{1.267812in}{1.132254in}}%
\pgfpathlineto{\pgfqpoint{1.296957in}{1.195397in}}%
\pgfpathlineto{\pgfqpoint{1.335815in}{1.285999in}}%
\pgfpathlineto{\pgfqpoint{1.452392in}{1.565251in}}%
\pgfpathlineto{\pgfqpoint{1.481536in}{1.625723in}}%
\pgfpathlineto{\pgfqpoint{1.500966in}{1.661571in}}%
\pgfpathlineto{\pgfqpoint{1.520395in}{1.693075in}}%
\pgfpathlineto{\pgfqpoint{1.539824in}{1.719653in}}%
\pgfpathlineto{\pgfqpoint{1.559254in}{1.740801in}}%
\pgfpathlineto{\pgfqpoint{1.568969in}{1.749224in}}%
\pgfpathlineto{\pgfqpoint{1.578683in}{1.756110in}}%
\pgfpathlineto{\pgfqpoint{1.588398in}{1.761482in}}%
\pgfpathlineto{\pgfqpoint{1.598113in}{1.765279in}}%
\pgfpathlineto{\pgfqpoint{1.607827in}{1.767493in}}%
\pgfpathlineto{\pgfqpoint{1.617542in}{1.768126in}}%
\pgfpathlineto{\pgfqpoint{1.627257in}{1.767138in}}%
\pgfpathlineto{\pgfqpoint{1.636972in}{1.764593in}}%
\pgfpathlineto{\pgfqpoint{1.646686in}{1.760423in}}%
\pgfpathlineto{\pgfqpoint{1.656401in}{1.754752in}}%
\pgfpathlineto{\pgfqpoint{1.666116in}{1.747483in}}%
\pgfpathlineto{\pgfqpoint{1.675831in}{1.738796in}}%
\pgfpathlineto{\pgfqpoint{1.695260in}{1.717038in}}%
\pgfpathlineto{\pgfqpoint{1.714689in}{1.689899in}}%
\pgfpathlineto{\pgfqpoint{1.734119in}{1.657891in}}%
\pgfpathlineto{\pgfqpoint{1.753548in}{1.621603in}}%
\pgfpathlineto{\pgfqpoint{1.782692in}{1.560533in}}%
\pgfpathlineto{\pgfqpoint{1.811837in}{1.493693in}}%
\pgfpathlineto{\pgfqpoint{1.870125in}{1.351711in}}%
\pgfpathlineto{\pgfqpoint{1.918698in}{1.235055in}}%
\pgfpathlineto{\pgfqpoint{1.947843in}{1.169131in}}%
\pgfpathlineto{\pgfqpoint{1.976987in}{1.107969in}}%
\pgfpathlineto{\pgfqpoint{2.006131in}{1.051874in}}%
\pgfpathlineto{\pgfqpoint{2.035275in}{1.001726in}}%
\pgfpathlineto{\pgfqpoint{2.054704in}{0.971413in}}%
\pgfpathlineto{\pgfqpoint{2.074134in}{0.943630in}}%
\pgfpathlineto{\pgfqpoint{2.093563in}{0.918300in}}%
\pgfpathlineto{\pgfqpoint{2.112993in}{0.895315in}}%
\pgfpathlineto{\pgfqpoint{2.132422in}{0.874545in}}%
\pgfpathlineto{\pgfqpoint{2.151852in}{0.855844in}}%
\pgfpathlineto{\pgfqpoint{2.171281in}{0.839055in}}%
\pgfpathlineto{\pgfqpoint{2.190710in}{0.824020in}}%
\pgfpathlineto{\pgfqpoint{2.219855in}{0.804305in}}%
\pgfpathlineto{\pgfqpoint{2.248999in}{0.787882in}}%
\pgfpathlineto{\pgfqpoint{2.278143in}{0.773885in}}%
\pgfpathlineto{\pgfqpoint{2.317002in}{0.758703in}}%
\pgfpathlineto{\pgfqpoint{2.355861in}{0.746611in}}%
\pgfpathlineto{\pgfqpoint{2.404434in}{0.734905in}}%
\pgfpathlineto{\pgfqpoint{2.443293in}{0.727588in}}%
\pgfpathlineto{\pgfqpoint{2.491867in}{0.720425in}}%
\pgfpathlineto{\pgfqpoint{2.559870in}{0.713315in}}%
\pgfpathlineto{\pgfqpoint{2.569584in}{0.712488in}}%
\pgfpathlineto{\pgfqpoint{2.569584in}{0.712488in}}%
\pgfusepath{stroke}%
\end{pgfscope}%
\begin{pgfscope}%
\pgfpathrectangle{\pgfqpoint{0.636356in}{0.700846in}}{\pgfqpoint{1.933229in}{1.137023in}} %
\pgfusepath{clip}%
\pgfsetbuttcap%
\pgfsetroundjoin%
\pgfsetlinewidth{0.501875pt}%
\definecolor{currentstroke}{rgb}{0.000000,0.000000,0.000000}%
\pgfsetstrokecolor{currentstroke}%
\pgfsetdash{}{0pt}%
\pgfpathmoveto{\pgfqpoint{0.646022in}{0.709562in}}%
\pgfpathlineto{\pgfqpoint{0.646022in}{0.715578in}}%
\pgfusepath{stroke}%
\end{pgfscope}%
\begin{pgfscope}%
\pgfpathrectangle{\pgfqpoint{0.636356in}{0.700846in}}{\pgfqpoint{1.933229in}{1.137023in}} %
\pgfusepath{clip}%
\pgfsetbuttcap%
\pgfsetroundjoin%
\pgfsetlinewidth{0.501875pt}%
\definecolor{currentstroke}{rgb}{0.000000,0.000000,0.000000}%
\pgfsetstrokecolor{currentstroke}%
\pgfsetdash{}{0pt}%
\pgfpathmoveto{\pgfqpoint{0.665354in}{0.712368in}}%
\pgfpathlineto{\pgfqpoint{0.665354in}{0.719080in}}%
\pgfusepath{stroke}%
\end{pgfscope}%
\begin{pgfscope}%
\pgfpathrectangle{\pgfqpoint{0.636356in}{0.700846in}}{\pgfqpoint{1.933229in}{1.137023in}} %
\pgfusepath{clip}%
\pgfsetbuttcap%
\pgfsetroundjoin%
\pgfsetlinewidth{0.501875pt}%
\definecolor{currentstroke}{rgb}{0.000000,0.000000,0.000000}%
\pgfsetstrokecolor{currentstroke}%
\pgfsetdash{}{0pt}%
\pgfpathmoveto{\pgfqpoint{0.684687in}{0.713501in}}%
\pgfpathlineto{\pgfqpoint{0.684687in}{0.720470in}}%
\pgfusepath{stroke}%
\end{pgfscope}%
\begin{pgfscope}%
\pgfpathrectangle{\pgfqpoint{0.636356in}{0.700846in}}{\pgfqpoint{1.933229in}{1.137023in}} %
\pgfusepath{clip}%
\pgfsetbuttcap%
\pgfsetroundjoin%
\pgfsetlinewidth{0.501875pt}%
\definecolor{currentstroke}{rgb}{0.000000,0.000000,0.000000}%
\pgfsetstrokecolor{currentstroke}%
\pgfsetdash{}{0pt}%
\pgfpathmoveto{\pgfqpoint{0.704019in}{0.712934in}}%
\pgfpathlineto{\pgfqpoint{0.704019in}{0.719776in}}%
\pgfusepath{stroke}%
\end{pgfscope}%
\begin{pgfscope}%
\pgfpathrectangle{\pgfqpoint{0.636356in}{0.700846in}}{\pgfqpoint{1.933229in}{1.137023in}} %
\pgfusepath{clip}%
\pgfsetbuttcap%
\pgfsetroundjoin%
\pgfsetlinewidth{0.501875pt}%
\definecolor{currentstroke}{rgb}{0.000000,0.000000,0.000000}%
\pgfsetstrokecolor{currentstroke}%
\pgfsetdash{}{0pt}%
\pgfpathmoveto{\pgfqpoint{0.723351in}{0.715211in}}%
\pgfpathlineto{\pgfqpoint{0.723351in}{0.722547in}}%
\pgfusepath{stroke}%
\end{pgfscope}%
\begin{pgfscope}%
\pgfpathrectangle{\pgfqpoint{0.636356in}{0.700846in}}{\pgfqpoint{1.933229in}{1.137023in}} %
\pgfusepath{clip}%
\pgfsetbuttcap%
\pgfsetroundjoin%
\pgfsetlinewidth{0.501875pt}%
\definecolor{currentstroke}{rgb}{0.000000,0.000000,0.000000}%
\pgfsetstrokecolor{currentstroke}%
\pgfsetdash{}{0pt}%
\pgfpathmoveto{\pgfqpoint{0.742683in}{0.718658in}}%
\pgfpathlineto{\pgfqpoint{0.742683in}{0.726674in}}%
\pgfusepath{stroke}%
\end{pgfscope}%
\begin{pgfscope}%
\pgfpathrectangle{\pgfqpoint{0.636356in}{0.700846in}}{\pgfqpoint{1.933229in}{1.137023in}} %
\pgfusepath{clip}%
\pgfsetbuttcap%
\pgfsetroundjoin%
\pgfsetlinewidth{0.501875pt}%
\definecolor{currentstroke}{rgb}{0.000000,0.000000,0.000000}%
\pgfsetstrokecolor{currentstroke}%
\pgfsetdash{}{0pt}%
\pgfpathmoveto{\pgfqpoint{0.762016in}{0.720394in}}%
\pgfpathlineto{\pgfqpoint{0.762016in}{0.728727in}}%
\pgfusepath{stroke}%
\end{pgfscope}%
\begin{pgfscope}%
\pgfpathrectangle{\pgfqpoint{0.636356in}{0.700846in}}{\pgfqpoint{1.933229in}{1.137023in}} %
\pgfusepath{clip}%
\pgfsetbuttcap%
\pgfsetroundjoin%
\pgfsetlinewidth{0.501875pt}%
\definecolor{currentstroke}{rgb}{0.000000,0.000000,0.000000}%
\pgfsetstrokecolor{currentstroke}%
\pgfsetdash{}{0pt}%
\pgfpathmoveto{\pgfqpoint{0.781348in}{0.725635in}}%
\pgfpathlineto{\pgfqpoint{0.781348in}{0.734851in}}%
\pgfusepath{stroke}%
\end{pgfscope}%
\begin{pgfscope}%
\pgfpathrectangle{\pgfqpoint{0.636356in}{0.700846in}}{\pgfqpoint{1.933229in}{1.137023in}} %
\pgfusepath{clip}%
\pgfsetbuttcap%
\pgfsetroundjoin%
\pgfsetlinewidth{0.501875pt}%
\definecolor{currentstroke}{rgb}{0.000000,0.000000,0.000000}%
\pgfsetstrokecolor{currentstroke}%
\pgfsetdash{}{0pt}%
\pgfpathmoveto{\pgfqpoint{0.800680in}{0.724466in}}%
\pgfpathlineto{\pgfqpoint{0.800680in}{0.733494in}}%
\pgfusepath{stroke}%
\end{pgfscope}%
\begin{pgfscope}%
\pgfpathrectangle{\pgfqpoint{0.636356in}{0.700846in}}{\pgfqpoint{1.933229in}{1.137023in}} %
\pgfusepath{clip}%
\pgfsetbuttcap%
\pgfsetroundjoin%
\pgfsetlinewidth{0.501875pt}%
\definecolor{currentstroke}{rgb}{0.000000,0.000000,0.000000}%
\pgfsetstrokecolor{currentstroke}%
\pgfsetdash{}{0pt}%
\pgfpathmoveto{\pgfqpoint{0.820013in}{0.743955in}}%
\pgfpathlineto{\pgfqpoint{0.820013in}{0.755684in}}%
\pgfusepath{stroke}%
\end{pgfscope}%
\begin{pgfscope}%
\pgfpathrectangle{\pgfqpoint{0.636356in}{0.700846in}}{\pgfqpoint{1.933229in}{1.137023in}} %
\pgfusepath{clip}%
\pgfsetbuttcap%
\pgfsetroundjoin%
\pgfsetlinewidth{0.501875pt}%
\definecolor{currentstroke}{rgb}{0.000000,0.000000,0.000000}%
\pgfsetstrokecolor{currentstroke}%
\pgfsetdash{}{0pt}%
\pgfpathmoveto{\pgfqpoint{0.839345in}{0.736234in}}%
\pgfpathlineto{\pgfqpoint{0.839345in}{0.746985in}}%
\pgfusepath{stroke}%
\end{pgfscope}%
\begin{pgfscope}%
\pgfpathrectangle{\pgfqpoint{0.636356in}{0.700846in}}{\pgfqpoint{1.933229in}{1.137023in}} %
\pgfusepath{clip}%
\pgfsetbuttcap%
\pgfsetroundjoin%
\pgfsetlinewidth{0.501875pt}%
\definecolor{currentstroke}{rgb}{0.000000,0.000000,0.000000}%
\pgfsetstrokecolor{currentstroke}%
\pgfsetdash{}{0pt}%
\pgfpathmoveto{\pgfqpoint{0.858677in}{0.743359in}}%
\pgfpathlineto{\pgfqpoint{0.858677in}{0.755017in}}%
\pgfusepath{stroke}%
\end{pgfscope}%
\begin{pgfscope}%
\pgfpathrectangle{\pgfqpoint{0.636356in}{0.700846in}}{\pgfqpoint{1.933229in}{1.137023in}} %
\pgfusepath{clip}%
\pgfsetbuttcap%
\pgfsetroundjoin%
\pgfsetlinewidth{0.501875pt}%
\definecolor{currentstroke}{rgb}{0.000000,0.000000,0.000000}%
\pgfsetstrokecolor{currentstroke}%
\pgfsetdash{}{0pt}%
\pgfpathmoveto{\pgfqpoint{0.878009in}{0.742764in}}%
\pgfpathlineto{\pgfqpoint{0.878009in}{0.754349in}}%
\pgfusepath{stroke}%
\end{pgfscope}%
\begin{pgfscope}%
\pgfpathrectangle{\pgfqpoint{0.636356in}{0.700846in}}{\pgfqpoint{1.933229in}{1.137023in}} %
\pgfusepath{clip}%
\pgfsetbuttcap%
\pgfsetroundjoin%
\pgfsetlinewidth{0.501875pt}%
\definecolor{currentstroke}{rgb}{0.000000,0.000000,0.000000}%
\pgfsetstrokecolor{currentstroke}%
\pgfsetdash{}{0pt}%
\pgfpathmoveto{\pgfqpoint{0.897342in}{0.749324in}}%
\pgfpathlineto{\pgfqpoint{0.897342in}{0.761683in}}%
\pgfusepath{stroke}%
\end{pgfscope}%
\begin{pgfscope}%
\pgfpathrectangle{\pgfqpoint{0.636356in}{0.700846in}}{\pgfqpoint{1.933229in}{1.137023in}} %
\pgfusepath{clip}%
\pgfsetbuttcap%
\pgfsetroundjoin%
\pgfsetlinewidth{0.501875pt}%
\definecolor{currentstroke}{rgb}{0.000000,0.000000,0.000000}%
\pgfsetstrokecolor{currentstroke}%
\pgfsetdash{}{0pt}%
\pgfpathmoveto{\pgfqpoint{0.916674in}{0.747532in}}%
\pgfpathlineto{\pgfqpoint{0.916674in}{0.759685in}}%
\pgfusepath{stroke}%
\end{pgfscope}%
\begin{pgfscope}%
\pgfpathrectangle{\pgfqpoint{0.636356in}{0.700846in}}{\pgfqpoint{1.933229in}{1.137023in}} %
\pgfusepath{clip}%
\pgfsetbuttcap%
\pgfsetroundjoin%
\pgfsetlinewidth{0.501875pt}%
\definecolor{currentstroke}{rgb}{0.000000,0.000000,0.000000}%
\pgfsetstrokecolor{currentstroke}%
\pgfsetdash{}{0pt}%
\pgfpathmoveto{\pgfqpoint{0.936006in}{0.757108in}}%
\pgfpathlineto{\pgfqpoint{0.936006in}{0.770321in}}%
\pgfusepath{stroke}%
\end{pgfscope}%
\begin{pgfscope}%
\pgfpathrectangle{\pgfqpoint{0.636356in}{0.700846in}}{\pgfqpoint{1.933229in}{1.137023in}} %
\pgfusepath{clip}%
\pgfsetbuttcap%
\pgfsetroundjoin%
\pgfsetlinewidth{0.501875pt}%
\definecolor{currentstroke}{rgb}{0.000000,0.000000,0.000000}%
\pgfsetstrokecolor{currentstroke}%
\pgfsetdash{}{0pt}%
\pgfpathmoveto{\pgfqpoint{0.955339in}{0.774260in}}%
\pgfpathlineto{\pgfqpoint{0.955339in}{0.788511in}}%
\pgfusepath{stroke}%
\end{pgfscope}%
\begin{pgfscope}%
\pgfpathrectangle{\pgfqpoint{0.636356in}{0.700846in}}{\pgfqpoint{1.933229in}{1.137023in}} %
\pgfusepath{clip}%
\pgfsetbuttcap%
\pgfsetroundjoin%
\pgfsetlinewidth{0.501875pt}%
\definecolor{currentstroke}{rgb}{0.000000,0.000000,0.000000}%
\pgfsetstrokecolor{currentstroke}%
\pgfsetdash{}{0pt}%
\pgfpathmoveto{\pgfqpoint{0.974671in}{0.776071in}}%
\pgfpathlineto{\pgfqpoint{0.974671in}{0.790490in}}%
\pgfusepath{stroke}%
\end{pgfscope}%
\begin{pgfscope}%
\pgfpathrectangle{\pgfqpoint{0.636356in}{0.700846in}}{\pgfqpoint{1.933229in}{1.137023in}} %
\pgfusepath{clip}%
\pgfsetbuttcap%
\pgfsetroundjoin%
\pgfsetlinewidth{0.501875pt}%
\definecolor{currentstroke}{rgb}{0.000000,0.000000,0.000000}%
\pgfsetstrokecolor{currentstroke}%
\pgfsetdash{}{0pt}%
\pgfpathmoveto{\pgfqpoint{0.994003in}{0.786959in}}%
\pgfpathlineto{\pgfqpoint{0.994003in}{0.802342in}}%
\pgfusepath{stroke}%
\end{pgfscope}%
\begin{pgfscope}%
\pgfpathrectangle{\pgfqpoint{0.636356in}{0.700846in}}{\pgfqpoint{1.933229in}{1.137023in}} %
\pgfusepath{clip}%
\pgfsetbuttcap%
\pgfsetroundjoin%
\pgfsetlinewidth{0.501875pt}%
\definecolor{currentstroke}{rgb}{0.000000,0.000000,0.000000}%
\pgfsetstrokecolor{currentstroke}%
\pgfsetdash{}{0pt}%
\pgfpathmoveto{\pgfqpoint{1.013335in}{0.808817in}}%
\pgfpathlineto{\pgfqpoint{1.013335in}{0.825965in}}%
\pgfusepath{stroke}%
\end{pgfscope}%
\begin{pgfscope}%
\pgfpathrectangle{\pgfqpoint{0.636356in}{0.700846in}}{\pgfqpoint{1.933229in}{1.137023in}} %
\pgfusepath{clip}%
\pgfsetbuttcap%
\pgfsetroundjoin%
\pgfsetlinewidth{0.501875pt}%
\definecolor{currentstroke}{rgb}{0.000000,0.000000,0.000000}%
\pgfsetstrokecolor{currentstroke}%
\pgfsetdash{}{0pt}%
\pgfpathmoveto{\pgfqpoint{1.032668in}{0.825875in}}%
\pgfpathlineto{\pgfqpoint{1.032668in}{0.844281in}}%
\pgfusepath{stroke}%
\end{pgfscope}%
\begin{pgfscope}%
\pgfpathrectangle{\pgfqpoint{0.636356in}{0.700846in}}{\pgfqpoint{1.933229in}{1.137023in}} %
\pgfusepath{clip}%
\pgfsetbuttcap%
\pgfsetroundjoin%
\pgfsetlinewidth{0.501875pt}%
\definecolor{currentstroke}{rgb}{0.000000,0.000000,0.000000}%
\pgfsetstrokecolor{currentstroke}%
\pgfsetdash{}{0pt}%
\pgfpathmoveto{\pgfqpoint{1.052000in}{0.831978in}}%
\pgfpathlineto{\pgfqpoint{1.052000in}{0.850812in}}%
\pgfusepath{stroke}%
\end{pgfscope}%
\begin{pgfscope}%
\pgfpathrectangle{\pgfqpoint{0.636356in}{0.700846in}}{\pgfqpoint{1.933229in}{1.137023in}} %
\pgfusepath{clip}%
\pgfsetbuttcap%
\pgfsetroundjoin%
\pgfsetlinewidth{0.501875pt}%
\definecolor{currentstroke}{rgb}{0.000000,0.000000,0.000000}%
\pgfsetstrokecolor{currentstroke}%
\pgfsetdash{}{0pt}%
\pgfpathmoveto{\pgfqpoint{1.071332in}{0.840529in}}%
\pgfpathlineto{\pgfqpoint{1.071332in}{0.859948in}}%
\pgfusepath{stroke}%
\end{pgfscope}%
\begin{pgfscope}%
\pgfpathrectangle{\pgfqpoint{0.636356in}{0.700846in}}{\pgfqpoint{1.933229in}{1.137023in}} %
\pgfusepath{clip}%
\pgfsetbuttcap%
\pgfsetroundjoin%
\pgfsetlinewidth{0.501875pt}%
\definecolor{currentstroke}{rgb}{0.000000,0.000000,0.000000}%
\pgfsetstrokecolor{currentstroke}%
\pgfsetdash{}{0pt}%
\pgfpathmoveto{\pgfqpoint{1.090665in}{0.858269in}}%
\pgfpathlineto{\pgfqpoint{1.090665in}{0.878845in}}%
\pgfusepath{stroke}%
\end{pgfscope}%
\begin{pgfscope}%
\pgfpathrectangle{\pgfqpoint{0.636356in}{0.700846in}}{\pgfqpoint{1.933229in}{1.137023in}} %
\pgfusepath{clip}%
\pgfsetbuttcap%
\pgfsetroundjoin%
\pgfsetlinewidth{0.501875pt}%
\definecolor{currentstroke}{rgb}{0.000000,0.000000,0.000000}%
\pgfsetstrokecolor{currentstroke}%
\pgfsetdash{}{0pt}%
\pgfpathmoveto{\pgfqpoint{1.109997in}{0.877267in}}%
\pgfpathlineto{\pgfqpoint{1.109997in}{0.899012in}}%
\pgfusepath{stroke}%
\end{pgfscope}%
\begin{pgfscope}%
\pgfpathrectangle{\pgfqpoint{0.636356in}{0.700846in}}{\pgfqpoint{1.933229in}{1.137023in}} %
\pgfusepath{clip}%
\pgfsetbuttcap%
\pgfsetroundjoin%
\pgfsetlinewidth{0.501875pt}%
\definecolor{currentstroke}{rgb}{0.000000,0.000000,0.000000}%
\pgfsetstrokecolor{currentstroke}%
\pgfsetdash{}{0pt}%
\pgfpathmoveto{\pgfqpoint{1.129329in}{0.895680in}}%
\pgfpathlineto{\pgfqpoint{1.129329in}{0.918499in}}%
\pgfusepath{stroke}%
\end{pgfscope}%
\begin{pgfscope}%
\pgfpathrectangle{\pgfqpoint{0.636356in}{0.700846in}}{\pgfqpoint{1.933229in}{1.137023in}} %
\pgfusepath{clip}%
\pgfsetbuttcap%
\pgfsetroundjoin%
\pgfsetlinewidth{0.501875pt}%
\definecolor{currentstroke}{rgb}{0.000000,0.000000,0.000000}%
\pgfsetstrokecolor{currentstroke}%
\pgfsetdash{}{0pt}%
\pgfpathmoveto{\pgfqpoint{1.148661in}{0.921498in}}%
\pgfpathlineto{\pgfqpoint{1.148661in}{0.945742in}}%
\pgfusepath{stroke}%
\end{pgfscope}%
\begin{pgfscope}%
\pgfpathrectangle{\pgfqpoint{0.636356in}{0.700846in}}{\pgfqpoint{1.933229in}{1.137023in}} %
\pgfusepath{clip}%
\pgfsetbuttcap%
\pgfsetroundjoin%
\pgfsetlinewidth{0.501875pt}%
\definecolor{currentstroke}{rgb}{0.000000,0.000000,0.000000}%
\pgfsetstrokecolor{currentstroke}%
\pgfsetdash{}{0pt}%
\pgfpathmoveto{\pgfqpoint{1.167994in}{0.935038in}}%
\pgfpathlineto{\pgfqpoint{1.167994in}{0.959996in}}%
\pgfusepath{stroke}%
\end{pgfscope}%
\begin{pgfscope}%
\pgfpathrectangle{\pgfqpoint{0.636356in}{0.700846in}}{\pgfqpoint{1.933229in}{1.137023in}} %
\pgfusepath{clip}%
\pgfsetbuttcap%
\pgfsetroundjoin%
\pgfsetlinewidth{0.501875pt}%
\definecolor{currentstroke}{rgb}{0.000000,0.000000,0.000000}%
\pgfsetstrokecolor{currentstroke}%
\pgfsetdash{}{0pt}%
\pgfpathmoveto{\pgfqpoint{1.187326in}{0.967697in}}%
\pgfpathlineto{\pgfqpoint{1.187326in}{0.994295in}}%
\pgfusepath{stroke}%
\end{pgfscope}%
\begin{pgfscope}%
\pgfpathrectangle{\pgfqpoint{0.636356in}{0.700846in}}{\pgfqpoint{1.933229in}{1.137023in}} %
\pgfusepath{clip}%
\pgfsetbuttcap%
\pgfsetroundjoin%
\pgfsetlinewidth{0.501875pt}%
\definecolor{currentstroke}{rgb}{0.000000,0.000000,0.000000}%
\pgfsetstrokecolor{currentstroke}%
\pgfsetdash{}{0pt}%
\pgfpathmoveto{\pgfqpoint{1.206658in}{1.018317in}}%
\pgfpathlineto{\pgfqpoint{1.206658in}{1.047271in}}%
\pgfusepath{stroke}%
\end{pgfscope}%
\begin{pgfscope}%
\pgfpathrectangle{\pgfqpoint{0.636356in}{0.700846in}}{\pgfqpoint{1.933229in}{1.137023in}} %
\pgfusepath{clip}%
\pgfsetbuttcap%
\pgfsetroundjoin%
\pgfsetlinewidth{0.501875pt}%
\definecolor{currentstroke}{rgb}{0.000000,0.000000,0.000000}%
\pgfsetstrokecolor{currentstroke}%
\pgfsetdash{}{0pt}%
\pgfpathmoveto{\pgfqpoint{1.225991in}{1.040569in}}%
\pgfpathlineto{\pgfqpoint{1.225991in}{1.070499in}}%
\pgfusepath{stroke}%
\end{pgfscope}%
\begin{pgfscope}%
\pgfpathrectangle{\pgfqpoint{0.636356in}{0.700846in}}{\pgfqpoint{1.933229in}{1.137023in}} %
\pgfusepath{clip}%
\pgfsetbuttcap%
\pgfsetroundjoin%
\pgfsetlinewidth{0.501875pt}%
\definecolor{currentstroke}{rgb}{0.000000,0.000000,0.000000}%
\pgfsetstrokecolor{currentstroke}%
\pgfsetdash{}{0pt}%
\pgfpathmoveto{\pgfqpoint{1.245323in}{1.080786in}}%
\pgfpathlineto{\pgfqpoint{1.245323in}{1.112401in}}%
\pgfusepath{stroke}%
\end{pgfscope}%
\begin{pgfscope}%
\pgfpathrectangle{\pgfqpoint{0.636356in}{0.700846in}}{\pgfqpoint{1.933229in}{1.137023in}} %
\pgfusepath{clip}%
\pgfsetbuttcap%
\pgfsetroundjoin%
\pgfsetlinewidth{0.501875pt}%
\definecolor{currentstroke}{rgb}{0.000000,0.000000,0.000000}%
\pgfsetstrokecolor{currentstroke}%
\pgfsetdash{}{0pt}%
\pgfpathmoveto{\pgfqpoint{1.264655in}{1.075833in}}%
\pgfpathlineto{\pgfqpoint{1.264655in}{1.107246in}}%
\pgfusepath{stroke}%
\end{pgfscope}%
\begin{pgfscope}%
\pgfpathrectangle{\pgfqpoint{0.636356in}{0.700846in}}{\pgfqpoint{1.933229in}{1.137023in}} %
\pgfusepath{clip}%
\pgfsetbuttcap%
\pgfsetroundjoin%
\pgfsetlinewidth{0.501875pt}%
\definecolor{currentstroke}{rgb}{0.000000,0.000000,0.000000}%
\pgfsetstrokecolor{currentstroke}%
\pgfsetdash{}{0pt}%
\pgfpathmoveto{\pgfqpoint{1.283987in}{1.131580in}}%
\pgfpathlineto{\pgfqpoint{1.283987in}{1.165202in}}%
\pgfusepath{stroke}%
\end{pgfscope}%
\begin{pgfscope}%
\pgfpathrectangle{\pgfqpoint{0.636356in}{0.700846in}}{\pgfqpoint{1.933229in}{1.137023in}} %
\pgfusepath{clip}%
\pgfsetbuttcap%
\pgfsetroundjoin%
\pgfsetlinewidth{0.501875pt}%
\definecolor{currentstroke}{rgb}{0.000000,0.000000,0.000000}%
\pgfsetstrokecolor{currentstroke}%
\pgfsetdash{}{0pt}%
\pgfpathmoveto{\pgfqpoint{1.303320in}{1.186775in}}%
\pgfpathlineto{\pgfqpoint{1.303320in}{1.222446in}}%
\pgfusepath{stroke}%
\end{pgfscope}%
\begin{pgfscope}%
\pgfpathrectangle{\pgfqpoint{0.636356in}{0.700846in}}{\pgfqpoint{1.933229in}{1.137023in}} %
\pgfusepath{clip}%
\pgfsetbuttcap%
\pgfsetroundjoin%
\pgfsetlinewidth{0.501875pt}%
\definecolor{currentstroke}{rgb}{0.000000,0.000000,0.000000}%
\pgfsetstrokecolor{currentstroke}%
\pgfsetdash{}{0pt}%
\pgfpathmoveto{\pgfqpoint{1.322652in}{1.238299in}}%
\pgfpathlineto{\pgfqpoint{1.322652in}{1.275781in}}%
\pgfusepath{stroke}%
\end{pgfscope}%
\begin{pgfscope}%
\pgfpathrectangle{\pgfqpoint{0.636356in}{0.700846in}}{\pgfqpoint{1.933229in}{1.137023in}} %
\pgfusepath{clip}%
\pgfsetbuttcap%
\pgfsetroundjoin%
\pgfsetlinewidth{0.501875pt}%
\definecolor{currentstroke}{rgb}{0.000000,0.000000,0.000000}%
\pgfsetstrokecolor{currentstroke}%
\pgfsetdash{}{0pt}%
\pgfpathmoveto{\pgfqpoint{1.341984in}{1.273707in}}%
\pgfpathlineto{\pgfqpoint{1.341984in}{1.312384in}}%
\pgfusepath{stroke}%
\end{pgfscope}%
\begin{pgfscope}%
\pgfpathrectangle{\pgfqpoint{0.636356in}{0.700846in}}{\pgfqpoint{1.933229in}{1.137023in}} %
\pgfusepath{clip}%
\pgfsetbuttcap%
\pgfsetroundjoin%
\pgfsetlinewidth{0.501875pt}%
\definecolor{currentstroke}{rgb}{0.000000,0.000000,0.000000}%
\pgfsetstrokecolor{currentstroke}%
\pgfsetdash{}{0pt}%
\pgfpathmoveto{\pgfqpoint{1.361317in}{1.319081in}}%
\pgfpathlineto{\pgfqpoint{1.361317in}{1.359236in}}%
\pgfusepath{stroke}%
\end{pgfscope}%
\begin{pgfscope}%
\pgfpathrectangle{\pgfqpoint{0.636356in}{0.700846in}}{\pgfqpoint{1.933229in}{1.137023in}} %
\pgfusepath{clip}%
\pgfsetbuttcap%
\pgfsetroundjoin%
\pgfsetlinewidth{0.501875pt}%
\definecolor{currentstroke}{rgb}{0.000000,0.000000,0.000000}%
\pgfsetstrokecolor{currentstroke}%
\pgfsetdash{}{0pt}%
\pgfpathmoveto{\pgfqpoint{1.380649in}{1.385634in}}%
\pgfpathlineto{\pgfqpoint{1.380649in}{1.427862in}}%
\pgfusepath{stroke}%
\end{pgfscope}%
\begin{pgfscope}%
\pgfpathrectangle{\pgfqpoint{0.636356in}{0.700846in}}{\pgfqpoint{1.933229in}{1.137023in}} %
\pgfusepath{clip}%
\pgfsetbuttcap%
\pgfsetroundjoin%
\pgfsetlinewidth{0.501875pt}%
\definecolor{currentstroke}{rgb}{0.000000,0.000000,0.000000}%
\pgfsetstrokecolor{currentstroke}%
\pgfsetdash{}{0pt}%
\pgfpathmoveto{\pgfqpoint{1.399981in}{1.374434in}}%
\pgfpathlineto{\pgfqpoint{1.399981in}{1.416321in}}%
\pgfusepath{stroke}%
\end{pgfscope}%
\begin{pgfscope}%
\pgfpathrectangle{\pgfqpoint{0.636356in}{0.700846in}}{\pgfqpoint{1.933229in}{1.137023in}} %
\pgfusepath{clip}%
\pgfsetbuttcap%
\pgfsetroundjoin%
\pgfsetlinewidth{0.501875pt}%
\definecolor{currentstroke}{rgb}{0.000000,0.000000,0.000000}%
\pgfsetstrokecolor{currentstroke}%
\pgfsetdash{}{0pt}%
\pgfpathmoveto{\pgfqpoint{1.419313in}{1.474653in}}%
\pgfpathlineto{\pgfqpoint{1.419313in}{1.519503in}}%
\pgfusepath{stroke}%
\end{pgfscope}%
\begin{pgfscope}%
\pgfpathrectangle{\pgfqpoint{0.636356in}{0.700846in}}{\pgfqpoint{1.933229in}{1.137023in}} %
\pgfusepath{clip}%
\pgfsetbuttcap%
\pgfsetroundjoin%
\pgfsetlinewidth{0.501875pt}%
\definecolor{currentstroke}{rgb}{0.000000,0.000000,0.000000}%
\pgfsetstrokecolor{currentstroke}%
\pgfsetdash{}{0pt}%
\pgfpathmoveto{\pgfqpoint{1.438646in}{1.497699in}}%
\pgfpathlineto{\pgfqpoint{1.438646in}{1.543202in}}%
\pgfusepath{stroke}%
\end{pgfscope}%
\begin{pgfscope}%
\pgfpathrectangle{\pgfqpoint{0.636356in}{0.700846in}}{\pgfqpoint{1.933229in}{1.137023in}} %
\pgfusepath{clip}%
\pgfsetbuttcap%
\pgfsetroundjoin%
\pgfsetlinewidth{0.501875pt}%
\definecolor{currentstroke}{rgb}{0.000000,0.000000,0.000000}%
\pgfsetstrokecolor{currentstroke}%
\pgfsetdash{}{0pt}%
\pgfpathmoveto{\pgfqpoint{1.457978in}{1.548165in}}%
\pgfpathlineto{\pgfqpoint{1.457978in}{1.595067in}}%
\pgfusepath{stroke}%
\end{pgfscope}%
\begin{pgfscope}%
\pgfpathrectangle{\pgfqpoint{0.636356in}{0.700846in}}{\pgfqpoint{1.933229in}{1.137023in}} %
\pgfusepath{clip}%
\pgfsetbuttcap%
\pgfsetroundjoin%
\pgfsetlinewidth{0.501875pt}%
\definecolor{currentstroke}{rgb}{0.000000,0.000000,0.000000}%
\pgfsetstrokecolor{currentstroke}%
\pgfsetdash{}{0pt}%
\pgfpathmoveto{\pgfqpoint{1.477310in}{1.570601in}}%
\pgfpathlineto{\pgfqpoint{1.477310in}{1.618112in}}%
\pgfusepath{stroke}%
\end{pgfscope}%
\begin{pgfscope}%
\pgfpathrectangle{\pgfqpoint{0.636356in}{0.700846in}}{\pgfqpoint{1.933229in}{1.137023in}} %
\pgfusepath{clip}%
\pgfsetbuttcap%
\pgfsetroundjoin%
\pgfsetlinewidth{0.501875pt}%
\definecolor{currentstroke}{rgb}{0.000000,0.000000,0.000000}%
\pgfsetstrokecolor{currentstroke}%
\pgfsetdash{}{0pt}%
\pgfpathmoveto{\pgfqpoint{1.496643in}{1.606757in}}%
\pgfpathlineto{\pgfqpoint{1.496643in}{1.655232in}}%
\pgfusepath{stroke}%
\end{pgfscope}%
\begin{pgfscope}%
\pgfpathrectangle{\pgfqpoint{0.636356in}{0.700846in}}{\pgfqpoint{1.933229in}{1.137023in}} %
\pgfusepath{clip}%
\pgfsetbuttcap%
\pgfsetroundjoin%
\pgfsetlinewidth{0.501875pt}%
\definecolor{currentstroke}{rgb}{0.000000,0.000000,0.000000}%
\pgfsetstrokecolor{currentstroke}%
\pgfsetdash{}{0pt}%
\pgfpathmoveto{\pgfqpoint{1.515975in}{1.659759in}}%
\pgfpathlineto{\pgfqpoint{1.515975in}{1.709614in}}%
\pgfusepath{stroke}%
\end{pgfscope}%
\begin{pgfscope}%
\pgfpathrectangle{\pgfqpoint{0.636356in}{0.700846in}}{\pgfqpoint{1.933229in}{1.137023in}} %
\pgfusepath{clip}%
\pgfsetbuttcap%
\pgfsetroundjoin%
\pgfsetlinewidth{0.501875pt}%
\definecolor{currentstroke}{rgb}{0.000000,0.000000,0.000000}%
\pgfsetstrokecolor{currentstroke}%
\pgfsetdash{}{0pt}%
\pgfpathmoveto{\pgfqpoint{1.535307in}{1.649781in}}%
\pgfpathlineto{\pgfqpoint{1.535307in}{1.699379in}}%
\pgfusepath{stroke}%
\end{pgfscope}%
\begin{pgfscope}%
\pgfpathrectangle{\pgfqpoint{0.636356in}{0.700846in}}{\pgfqpoint{1.933229in}{1.137023in}} %
\pgfusepath{clip}%
\pgfsetbuttcap%
\pgfsetroundjoin%
\pgfsetlinewidth{0.501875pt}%
\definecolor{currentstroke}{rgb}{0.000000,0.000000,0.000000}%
\pgfsetstrokecolor{currentstroke}%
\pgfsetdash{}{0pt}%
\pgfpathmoveto{\pgfqpoint{1.554639in}{1.679095in}}%
\pgfpathlineto{\pgfqpoint{1.554639in}{1.729443in}}%
\pgfusepath{stroke}%
\end{pgfscope}%
\begin{pgfscope}%
\pgfpathrectangle{\pgfqpoint{0.636356in}{0.700846in}}{\pgfqpoint{1.933229in}{1.137023in}} %
\pgfusepath{clip}%
\pgfsetbuttcap%
\pgfsetroundjoin%
\pgfsetlinewidth{0.501875pt}%
\definecolor{currentstroke}{rgb}{0.000000,0.000000,0.000000}%
\pgfsetstrokecolor{currentstroke}%
\pgfsetdash{}{0pt}%
\pgfpathmoveto{\pgfqpoint{1.573972in}{1.727131in}}%
\pgfpathlineto{\pgfqpoint{1.573972in}{1.778685in}}%
\pgfusepath{stroke}%
\end{pgfscope}%
\begin{pgfscope}%
\pgfpathrectangle{\pgfqpoint{0.636356in}{0.700846in}}{\pgfqpoint{1.933229in}{1.137023in}} %
\pgfusepath{clip}%
\pgfsetbuttcap%
\pgfsetroundjoin%
\pgfsetlinewidth{0.501875pt}%
\definecolor{currentstroke}{rgb}{0.000000,0.000000,0.000000}%
\pgfsetstrokecolor{currentstroke}%
\pgfsetdash{}{0pt}%
\pgfpathmoveto{\pgfqpoint{1.593304in}{1.732746in}}%
\pgfpathlineto{\pgfqpoint{1.593304in}{1.784440in}}%
\pgfusepath{stroke}%
\end{pgfscope}%
\begin{pgfscope}%
\pgfpathrectangle{\pgfqpoint{0.636356in}{0.700846in}}{\pgfqpoint{1.933229in}{1.137023in}} %
\pgfusepath{clip}%
\pgfsetbuttcap%
\pgfsetroundjoin%
\pgfsetlinewidth{0.501875pt}%
\definecolor{currentstroke}{rgb}{0.000000,0.000000,0.000000}%
\pgfsetstrokecolor{currentstroke}%
\pgfsetdash{}{0pt}%
\pgfpathmoveto{\pgfqpoint{1.612636in}{1.747722in}}%
\pgfpathlineto{\pgfqpoint{1.612636in}{1.799785in}}%
\pgfusepath{stroke}%
\end{pgfscope}%
\begin{pgfscope}%
\pgfpathrectangle{\pgfqpoint{0.636356in}{0.700846in}}{\pgfqpoint{1.933229in}{1.137023in}} %
\pgfusepath{clip}%
\pgfsetbuttcap%
\pgfsetroundjoin%
\pgfsetlinewidth{0.501875pt}%
\definecolor{currentstroke}{rgb}{0.000000,0.000000,0.000000}%
\pgfsetstrokecolor{currentstroke}%
\pgfsetdash{}{0pt}%
\pgfpathmoveto{\pgfqpoint{1.631969in}{1.748970in}}%
\pgfpathlineto{\pgfqpoint{1.631969in}{1.801064in}}%
\pgfusepath{stroke}%
\end{pgfscope}%
\begin{pgfscope}%
\pgfpathrectangle{\pgfqpoint{0.636356in}{0.700846in}}{\pgfqpoint{1.933229in}{1.137023in}} %
\pgfusepath{clip}%
\pgfsetbuttcap%
\pgfsetroundjoin%
\pgfsetlinewidth{0.501875pt}%
\definecolor{currentstroke}{rgb}{0.000000,0.000000,0.000000}%
\pgfsetstrokecolor{currentstroke}%
\pgfsetdash{}{0pt}%
\pgfpathmoveto{\pgfqpoint{1.651301in}{1.747098in}}%
\pgfpathlineto{\pgfqpoint{1.651301in}{1.799146in}}%
\pgfusepath{stroke}%
\end{pgfscope}%
\begin{pgfscope}%
\pgfpathrectangle{\pgfqpoint{0.636356in}{0.700846in}}{\pgfqpoint{1.933229in}{1.137023in}} %
\pgfusepath{clip}%
\pgfsetbuttcap%
\pgfsetroundjoin%
\pgfsetlinewidth{0.501875pt}%
\definecolor{currentstroke}{rgb}{0.000000,0.000000,0.000000}%
\pgfsetstrokecolor{currentstroke}%
\pgfsetdash{}{0pt}%
\pgfpathmoveto{\pgfqpoint{1.670633in}{1.710909in}}%
\pgfpathlineto{\pgfqpoint{1.670633in}{1.762060in}}%
\pgfusepath{stroke}%
\end{pgfscope}%
\begin{pgfscope}%
\pgfpathrectangle{\pgfqpoint{0.636356in}{0.700846in}}{\pgfqpoint{1.933229in}{1.137023in}} %
\pgfusepath{clip}%
\pgfsetbuttcap%
\pgfsetroundjoin%
\pgfsetlinewidth{0.501875pt}%
\definecolor{currentstroke}{rgb}{0.000000,0.000000,0.000000}%
\pgfsetstrokecolor{currentstroke}%
\pgfsetdash{}{0pt}%
\pgfpathmoveto{\pgfqpoint{1.689965in}{1.724011in}}%
\pgfpathlineto{\pgfqpoint{1.689965in}{1.775488in}}%
\pgfusepath{stroke}%
\end{pgfscope}%
\begin{pgfscope}%
\pgfpathrectangle{\pgfqpoint{0.636356in}{0.700846in}}{\pgfqpoint{1.933229in}{1.137023in}} %
\pgfusepath{clip}%
\pgfsetbuttcap%
\pgfsetroundjoin%
\pgfsetlinewidth{0.501875pt}%
\definecolor{currentstroke}{rgb}{0.000000,0.000000,0.000000}%
\pgfsetstrokecolor{currentstroke}%
\pgfsetdash{}{0pt}%
\pgfpathmoveto{\pgfqpoint{1.709298in}{1.671610in}}%
\pgfpathlineto{\pgfqpoint{1.709298in}{1.721768in}}%
\pgfusepath{stroke}%
\end{pgfscope}%
\begin{pgfscope}%
\pgfpathrectangle{\pgfqpoint{0.636356in}{0.700846in}}{\pgfqpoint{1.933229in}{1.137023in}} %
\pgfusepath{clip}%
\pgfsetbuttcap%
\pgfsetroundjoin%
\pgfsetlinewidth{0.501875pt}%
\definecolor{currentstroke}{rgb}{0.000000,0.000000,0.000000}%
\pgfsetstrokecolor{currentstroke}%
\pgfsetdash{}{0pt}%
\pgfpathmoveto{\pgfqpoint{1.728630in}{1.682837in}}%
\pgfpathlineto{\pgfqpoint{1.728630in}{1.733281in}}%
\pgfusepath{stroke}%
\end{pgfscope}%
\begin{pgfscope}%
\pgfpathrectangle{\pgfqpoint{0.636356in}{0.700846in}}{\pgfqpoint{1.933229in}{1.137023in}} %
\pgfusepath{clip}%
\pgfsetbuttcap%
\pgfsetroundjoin%
\pgfsetlinewidth{0.501875pt}%
\definecolor{currentstroke}{rgb}{0.000000,0.000000,0.000000}%
\pgfsetstrokecolor{currentstroke}%
\pgfsetdash{}{0pt}%
\pgfpathmoveto{\pgfqpoint{1.747962in}{1.663502in}}%
\pgfpathlineto{\pgfqpoint{1.747962in}{1.713452in}}%
\pgfusepath{stroke}%
\end{pgfscope}%
\begin{pgfscope}%
\pgfpathrectangle{\pgfqpoint{0.636356in}{0.700846in}}{\pgfqpoint{1.933229in}{1.137023in}} %
\pgfusepath{clip}%
\pgfsetbuttcap%
\pgfsetroundjoin%
\pgfsetlinewidth{0.501875pt}%
\definecolor{currentstroke}{rgb}{0.000000,0.000000,0.000000}%
\pgfsetstrokecolor{currentstroke}%
\pgfsetdash{}{0pt}%
\pgfpathmoveto{\pgfqpoint{1.767295in}{1.593041in}}%
\pgfpathlineto{\pgfqpoint{1.767295in}{1.641153in}}%
\pgfusepath{stroke}%
\end{pgfscope}%
\begin{pgfscope}%
\pgfpathrectangle{\pgfqpoint{0.636356in}{0.700846in}}{\pgfqpoint{1.933229in}{1.137023in}} %
\pgfusepath{clip}%
\pgfsetbuttcap%
\pgfsetroundjoin%
\pgfsetlinewidth{0.501875pt}%
\definecolor{currentstroke}{rgb}{0.000000,0.000000,0.000000}%
\pgfsetstrokecolor{currentstroke}%
\pgfsetdash{}{0pt}%
\pgfpathmoveto{\pgfqpoint{1.786627in}{1.516388in}}%
\pgfpathlineto{\pgfqpoint{1.786627in}{1.562414in}}%
\pgfusepath{stroke}%
\end{pgfscope}%
\begin{pgfscope}%
\pgfpathrectangle{\pgfqpoint{0.636356in}{0.700846in}}{\pgfqpoint{1.933229in}{1.137023in}} %
\pgfusepath{clip}%
\pgfsetbuttcap%
\pgfsetroundjoin%
\pgfsetlinewidth{0.501875pt}%
\definecolor{currentstroke}{rgb}{0.000000,0.000000,0.000000}%
\pgfsetstrokecolor{currentstroke}%
\pgfsetdash{}{0pt}%
\pgfpathmoveto{\pgfqpoint{1.805959in}{1.474653in}}%
\pgfpathlineto{\pgfqpoint{1.805959in}{1.519503in}}%
\pgfusepath{stroke}%
\end{pgfscope}%
\begin{pgfscope}%
\pgfpathrectangle{\pgfqpoint{0.636356in}{0.700846in}}{\pgfqpoint{1.933229in}{1.137023in}} %
\pgfusepath{clip}%
\pgfsetbuttcap%
\pgfsetroundjoin%
\pgfsetlinewidth{0.501875pt}%
\definecolor{currentstroke}{rgb}{0.000000,0.000000,0.000000}%
\pgfsetstrokecolor{currentstroke}%
\pgfsetdash{}{0pt}%
\pgfpathmoveto{\pgfqpoint{1.825291in}{1.472785in}}%
\pgfpathlineto{\pgfqpoint{1.825291in}{1.517581in}}%
\pgfusepath{stroke}%
\end{pgfscope}%
\begin{pgfscope}%
\pgfpathrectangle{\pgfqpoint{0.636356in}{0.700846in}}{\pgfqpoint{1.933229in}{1.137023in}} %
\pgfusepath{clip}%
\pgfsetbuttcap%
\pgfsetroundjoin%
\pgfsetlinewidth{0.501875pt}%
\definecolor{currentstroke}{rgb}{0.000000,0.000000,0.000000}%
\pgfsetstrokecolor{currentstroke}%
\pgfsetdash{}{0pt}%
\pgfpathmoveto{\pgfqpoint{1.844624in}{1.379412in}}%
\pgfpathlineto{\pgfqpoint{1.844624in}{1.421450in}}%
\pgfusepath{stroke}%
\end{pgfscope}%
\begin{pgfscope}%
\pgfpathrectangle{\pgfqpoint{0.636356in}{0.700846in}}{\pgfqpoint{1.933229in}{1.137023in}} %
\pgfusepath{clip}%
\pgfsetbuttcap%
\pgfsetroundjoin%
\pgfsetlinewidth{0.501875pt}%
\definecolor{currentstroke}{rgb}{0.000000,0.000000,0.000000}%
\pgfsetstrokecolor{currentstroke}%
\pgfsetdash{}{0pt}%
\pgfpathmoveto{\pgfqpoint{1.863956in}{1.350174in}}%
\pgfpathlineto{\pgfqpoint{1.863956in}{1.391311in}}%
\pgfusepath{stroke}%
\end{pgfscope}%
\begin{pgfscope}%
\pgfpathrectangle{\pgfqpoint{0.636356in}{0.700846in}}{\pgfqpoint{1.933229in}{1.137023in}} %
\pgfusepath{clip}%
\pgfsetbuttcap%
\pgfsetroundjoin%
\pgfsetlinewidth{0.501875pt}%
\definecolor{currentstroke}{rgb}{0.000000,0.000000,0.000000}%
\pgfsetstrokecolor{currentstroke}%
\pgfsetdash{}{0pt}%
\pgfpathmoveto{\pgfqpoint{1.883288in}{1.330273in}}%
\pgfpathlineto{\pgfqpoint{1.883288in}{1.370784in}}%
\pgfusepath{stroke}%
\end{pgfscope}%
\begin{pgfscope}%
\pgfpathrectangle{\pgfqpoint{0.636356in}{0.700846in}}{\pgfqpoint{1.933229in}{1.137023in}} %
\pgfusepath{clip}%
\pgfsetbuttcap%
\pgfsetroundjoin%
\pgfsetlinewidth{0.501875pt}%
\definecolor{currentstroke}{rgb}{0.000000,0.000000,0.000000}%
\pgfsetstrokecolor{currentstroke}%
\pgfsetdash{}{0pt}%
\pgfpathmoveto{\pgfqpoint{1.902621in}{1.257554in}}%
\pgfpathlineto{\pgfqpoint{1.902621in}{1.295690in}}%
\pgfusepath{stroke}%
\end{pgfscope}%
\begin{pgfscope}%
\pgfpathrectangle{\pgfqpoint{0.636356in}{0.700846in}}{\pgfqpoint{1.933229in}{1.137023in}} %
\pgfusepath{clip}%
\pgfsetbuttcap%
\pgfsetroundjoin%
\pgfsetlinewidth{0.501875pt}%
\definecolor{currentstroke}{rgb}{0.000000,0.000000,0.000000}%
\pgfsetstrokecolor{currentstroke}%
\pgfsetdash{}{0pt}%
\pgfpathmoveto{\pgfqpoint{1.921953in}{1.207255in}}%
\pgfpathlineto{\pgfqpoint{1.921953in}{1.243657in}}%
\pgfusepath{stroke}%
\end{pgfscope}%
\begin{pgfscope}%
\pgfpathrectangle{\pgfqpoint{0.636356in}{0.700846in}}{\pgfqpoint{1.933229in}{1.137023in}} %
\pgfusepath{clip}%
\pgfsetbuttcap%
\pgfsetroundjoin%
\pgfsetlinewidth{0.501875pt}%
\definecolor{currentstroke}{rgb}{0.000000,0.000000,0.000000}%
\pgfsetstrokecolor{currentstroke}%
\pgfsetdash{}{0pt}%
\pgfpathmoveto{\pgfqpoint{1.941285in}{1.197324in}}%
\pgfpathlineto{\pgfqpoint{1.941285in}{1.233374in}}%
\pgfusepath{stroke}%
\end{pgfscope}%
\begin{pgfscope}%
\pgfpathrectangle{\pgfqpoint{0.636356in}{0.700846in}}{\pgfqpoint{1.933229in}{1.137023in}} %
\pgfusepath{clip}%
\pgfsetbuttcap%
\pgfsetroundjoin%
\pgfsetlinewidth{0.501875pt}%
\definecolor{currentstroke}{rgb}{0.000000,0.000000,0.000000}%
\pgfsetstrokecolor{currentstroke}%
\pgfsetdash{}{0pt}%
\pgfpathmoveto{\pgfqpoint{1.960617in}{1.137779in}}%
\pgfpathlineto{\pgfqpoint{1.960617in}{1.171637in}}%
\pgfusepath{stroke}%
\end{pgfscope}%
\begin{pgfscope}%
\pgfpathrectangle{\pgfqpoint{0.636356in}{0.700846in}}{\pgfqpoint{1.933229in}{1.137023in}} %
\pgfusepath{clip}%
\pgfsetbuttcap%
\pgfsetroundjoin%
\pgfsetlinewidth{0.501875pt}%
\definecolor{currentstroke}{rgb}{0.000000,0.000000,0.000000}%
\pgfsetstrokecolor{currentstroke}%
\pgfsetdash{}{0pt}%
\pgfpathmoveto{\pgfqpoint{1.979950in}{1.079548in}}%
\pgfpathlineto{\pgfqpoint{1.979950in}{1.111113in}}%
\pgfusepath{stroke}%
\end{pgfscope}%
\begin{pgfscope}%
\pgfpathrectangle{\pgfqpoint{0.636356in}{0.700846in}}{\pgfqpoint{1.933229in}{1.137023in}} %
\pgfusepath{clip}%
\pgfsetbuttcap%
\pgfsetroundjoin%
\pgfsetlinewidth{0.501875pt}%
\definecolor{currentstroke}{rgb}{0.000000,0.000000,0.000000}%
\pgfsetstrokecolor{currentstroke}%
\pgfsetdash{}{0pt}%
\pgfpathmoveto{\pgfqpoint{1.999282in}{1.049846in}}%
\pgfpathlineto{\pgfqpoint{1.999282in}{1.080173in}}%
\pgfusepath{stroke}%
\end{pgfscope}%
\begin{pgfscope}%
\pgfpathrectangle{\pgfqpoint{0.636356in}{0.700846in}}{\pgfqpoint{1.933229in}{1.137023in}} %
\pgfusepath{clip}%
\pgfsetbuttcap%
\pgfsetroundjoin%
\pgfsetlinewidth{0.501875pt}%
\definecolor{currentstroke}{rgb}{0.000000,0.000000,0.000000}%
\pgfsetstrokecolor{currentstroke}%
\pgfsetdash{}{0pt}%
\pgfpathmoveto{\pgfqpoint{2.018614in}{1.008432in}}%
\pgfpathlineto{\pgfqpoint{2.018614in}{1.036942in}}%
\pgfusepath{stroke}%
\end{pgfscope}%
\begin{pgfscope}%
\pgfpathrectangle{\pgfqpoint{0.636356in}{0.700846in}}{\pgfqpoint{1.933229in}{1.137023in}} %
\pgfusepath{clip}%
\pgfsetbuttcap%
\pgfsetroundjoin%
\pgfsetlinewidth{0.501875pt}%
\definecolor{currentstroke}{rgb}{0.000000,0.000000,0.000000}%
\pgfsetstrokecolor{currentstroke}%
\pgfsetdash{}{0pt}%
\pgfpathmoveto{\pgfqpoint{2.037947in}{0.992994in}}%
\pgfpathlineto{\pgfqpoint{2.037947in}{1.020795in}}%
\pgfusepath{stroke}%
\end{pgfscope}%
\begin{pgfscope}%
\pgfpathrectangle{\pgfqpoint{0.636356in}{0.700846in}}{\pgfqpoint{1.933229in}{1.137023in}} %
\pgfusepath{clip}%
\pgfsetbuttcap%
\pgfsetroundjoin%
\pgfsetlinewidth{0.501875pt}%
\definecolor{currentstroke}{rgb}{0.000000,0.000000,0.000000}%
\pgfsetstrokecolor{currentstroke}%
\pgfsetdash{}{0pt}%
\pgfpathmoveto{\pgfqpoint{2.057279in}{0.947972in}}%
\pgfpathlineto{\pgfqpoint{2.057279in}{0.973592in}}%
\pgfusepath{stroke}%
\end{pgfscope}%
\begin{pgfscope}%
\pgfpathrectangle{\pgfqpoint{0.636356in}{0.700846in}}{\pgfqpoint{1.933229in}{1.137023in}} %
\pgfusepath{clip}%
\pgfsetbuttcap%
\pgfsetroundjoin%
\pgfsetlinewidth{0.501875pt}%
\definecolor{currentstroke}{rgb}{0.000000,0.000000,0.000000}%
\pgfsetstrokecolor{currentstroke}%
\pgfsetdash{}{0pt}%
\pgfpathmoveto{\pgfqpoint{2.076611in}{0.924575in}}%
\pgfpathlineto{\pgfqpoint{2.076611in}{0.948982in}}%
\pgfusepath{stroke}%
\end{pgfscope}%
\begin{pgfscope}%
\pgfpathrectangle{\pgfqpoint{0.636356in}{0.700846in}}{\pgfqpoint{1.933229in}{1.137023in}} %
\pgfusepath{clip}%
\pgfsetbuttcap%
\pgfsetroundjoin%
\pgfsetlinewidth{0.501875pt}%
\definecolor{currentstroke}{rgb}{0.000000,0.000000,0.000000}%
\pgfsetstrokecolor{currentstroke}%
\pgfsetdash{}{0pt}%
\pgfpathmoveto{\pgfqpoint{2.095943in}{0.911658in}}%
\pgfpathlineto{\pgfqpoint{2.095943in}{0.935369in}}%
\pgfusepath{stroke}%
\end{pgfscope}%
\begin{pgfscope}%
\pgfpathrectangle{\pgfqpoint{0.636356in}{0.700846in}}{\pgfqpoint{1.933229in}{1.137023in}} %
\pgfusepath{clip}%
\pgfsetbuttcap%
\pgfsetroundjoin%
\pgfsetlinewidth{0.501875pt}%
\definecolor{currentstroke}{rgb}{0.000000,0.000000,0.000000}%
\pgfsetstrokecolor{currentstroke}%
\pgfsetdash{}{0pt}%
\pgfpathmoveto{\pgfqpoint{2.115276in}{0.875427in}}%
\pgfpathlineto{\pgfqpoint{2.115276in}{0.897061in}}%
\pgfusepath{stroke}%
\end{pgfscope}%
\begin{pgfscope}%
\pgfpathrectangle{\pgfqpoint{0.636356in}{0.700846in}}{\pgfqpoint{1.933229in}{1.137023in}} %
\pgfusepath{clip}%
\pgfsetbuttcap%
\pgfsetroundjoin%
\pgfsetlinewidth{0.501875pt}%
\definecolor{currentstroke}{rgb}{0.000000,0.000000,0.000000}%
\pgfsetstrokecolor{currentstroke}%
\pgfsetdash{}{0pt}%
\pgfpathmoveto{\pgfqpoint{2.134608in}{0.839918in}}%
\pgfpathlineto{\pgfqpoint{2.134608in}{0.859295in}}%
\pgfusepath{stroke}%
\end{pgfscope}%
\begin{pgfscope}%
\pgfpathrectangle{\pgfqpoint{0.636356in}{0.700846in}}{\pgfqpoint{1.933229in}{1.137023in}} %
\pgfusepath{clip}%
\pgfsetbuttcap%
\pgfsetroundjoin%
\pgfsetlinewidth{0.501875pt}%
\definecolor{currentstroke}{rgb}{0.000000,0.000000,0.000000}%
\pgfsetstrokecolor{currentstroke}%
\pgfsetdash{}{0pt}%
\pgfpathmoveto{\pgfqpoint{2.153940in}{0.858269in}}%
\pgfpathlineto{\pgfqpoint{2.153940in}{0.878845in}}%
\pgfusepath{stroke}%
\end{pgfscope}%
\begin{pgfscope}%
\pgfpathrectangle{\pgfqpoint{0.636356in}{0.700846in}}{\pgfqpoint{1.933229in}{1.137023in}} %
\pgfusepath{clip}%
\pgfsetbuttcap%
\pgfsetroundjoin%
\pgfsetlinewidth{0.501875pt}%
\definecolor{currentstroke}{rgb}{0.000000,0.000000,0.000000}%
\pgfsetstrokecolor{currentstroke}%
\pgfsetdash{}{0pt}%
\pgfpathmoveto{\pgfqpoint{2.173273in}{0.820997in}}%
\pgfpathlineto{\pgfqpoint{2.173273in}{0.839052in}}%
\pgfusepath{stroke}%
\end{pgfscope}%
\begin{pgfscope}%
\pgfpathrectangle{\pgfqpoint{0.636356in}{0.700846in}}{\pgfqpoint{1.933229in}{1.137023in}} %
\pgfusepath{clip}%
\pgfsetbuttcap%
\pgfsetroundjoin%
\pgfsetlinewidth{0.501875pt}%
\definecolor{currentstroke}{rgb}{0.000000,0.000000,0.000000}%
\pgfsetstrokecolor{currentstroke}%
\pgfsetdash{}{0pt}%
\pgfpathmoveto{\pgfqpoint{2.192605in}{0.806992in}}%
\pgfpathlineto{\pgfqpoint{2.192605in}{0.824000in}}%
\pgfusepath{stroke}%
\end{pgfscope}%
\begin{pgfscope}%
\pgfpathrectangle{\pgfqpoint{0.636356in}{0.700846in}}{\pgfqpoint{1.933229in}{1.137023in}} %
\pgfusepath{clip}%
\pgfsetbuttcap%
\pgfsetroundjoin%
\pgfsetlinewidth{0.501875pt}%
\definecolor{currentstroke}{rgb}{0.000000,0.000000,0.000000}%
\pgfsetstrokecolor{currentstroke}%
\pgfsetdash{}{0pt}%
\pgfpathmoveto{\pgfqpoint{2.211937in}{0.796055in}}%
\pgfpathlineto{\pgfqpoint{2.211937in}{0.812197in}}%
\pgfusepath{stroke}%
\end{pgfscope}%
\begin{pgfscope}%
\pgfpathrectangle{\pgfqpoint{0.636356in}{0.700846in}}{\pgfqpoint{1.933229in}{1.137023in}} %
\pgfusepath{clip}%
\pgfsetbuttcap%
\pgfsetroundjoin%
\pgfsetlinewidth{0.501875pt}%
\definecolor{currentstroke}{rgb}{0.000000,0.000000,0.000000}%
\pgfsetstrokecolor{currentstroke}%
\pgfsetdash{}{0pt}%
\pgfpathmoveto{\pgfqpoint{2.231269in}{0.788777in}}%
\pgfpathlineto{\pgfqpoint{2.231269in}{0.804314in}}%
\pgfusepath{stroke}%
\end{pgfscope}%
\begin{pgfscope}%
\pgfpathrectangle{\pgfqpoint{0.636356in}{0.700846in}}{\pgfqpoint{1.933229in}{1.137023in}} %
\pgfusepath{clip}%
\pgfsetbuttcap%
\pgfsetroundjoin%
\pgfsetlinewidth{0.501875pt}%
\definecolor{currentstroke}{rgb}{0.000000,0.000000,0.000000}%
\pgfsetstrokecolor{currentstroke}%
\pgfsetdash{}{0pt}%
\pgfpathmoveto{\pgfqpoint{2.250602in}{0.773656in}}%
\pgfpathlineto{\pgfqpoint{2.250602in}{0.787851in}}%
\pgfusepath{stroke}%
\end{pgfscope}%
\begin{pgfscope}%
\pgfpathrectangle{\pgfqpoint{0.636356in}{0.700846in}}{\pgfqpoint{1.933229in}{1.137023in}} %
\pgfusepath{clip}%
\pgfsetbuttcap%
\pgfsetroundjoin%
\pgfsetlinewidth{0.501875pt}%
\definecolor{currentstroke}{rgb}{0.000000,0.000000,0.000000}%
\pgfsetstrokecolor{currentstroke}%
\pgfsetdash{}{0pt}%
\pgfpathmoveto{\pgfqpoint{2.269934in}{0.769434in}}%
\pgfpathlineto{\pgfqpoint{2.269934in}{0.783230in}}%
\pgfusepath{stroke}%
\end{pgfscope}%
\begin{pgfscope}%
\pgfpathrectangle{\pgfqpoint{0.636356in}{0.700846in}}{\pgfqpoint{1.933229in}{1.137023in}} %
\pgfusepath{clip}%
\pgfsetbuttcap%
\pgfsetroundjoin%
\pgfsetlinewidth{0.501875pt}%
\definecolor{currentstroke}{rgb}{0.000000,0.000000,0.000000}%
\pgfsetstrokecolor{currentstroke}%
\pgfsetdash{}{0pt}%
\pgfpathmoveto{\pgfqpoint{2.289266in}{0.759806in}}%
\pgfpathlineto{\pgfqpoint{2.289266in}{0.772644in}}%
\pgfusepath{stroke}%
\end{pgfscope}%
\begin{pgfscope}%
\pgfpathrectangle{\pgfqpoint{0.636356in}{0.700846in}}{\pgfqpoint{1.933229in}{1.137023in}} %
\pgfusepath{clip}%
\pgfsetbuttcap%
\pgfsetroundjoin%
\pgfsetlinewidth{0.501875pt}%
\definecolor{currentstroke}{rgb}{0.000000,0.000000,0.000000}%
\pgfsetstrokecolor{currentstroke}%
\pgfsetdash{}{0pt}%
\pgfpathmoveto{\pgfqpoint{2.308599in}{0.751716in}}%
\pgfpathlineto{\pgfqpoint{2.308599in}{0.764344in}}%
\pgfusepath{stroke}%
\end{pgfscope}%
\begin{pgfscope}%
\pgfpathrectangle{\pgfqpoint{0.636356in}{0.700846in}}{\pgfqpoint{1.933229in}{1.137023in}} %
\pgfusepath{clip}%
\pgfsetbuttcap%
\pgfsetroundjoin%
\pgfsetlinewidth{0.501875pt}%
\definecolor{currentstroke}{rgb}{0.000000,0.000000,0.000000}%
\pgfsetstrokecolor{currentstroke}%
\pgfsetdash{}{0pt}%
\pgfpathmoveto{\pgfqpoint{2.327931in}{0.742764in}}%
\pgfpathlineto{\pgfqpoint{2.327931in}{0.754349in}}%
\pgfusepath{stroke}%
\end{pgfscope}%
\begin{pgfscope}%
\pgfpathrectangle{\pgfqpoint{0.636356in}{0.700846in}}{\pgfqpoint{1.933229in}{1.137023in}} %
\pgfusepath{clip}%
\pgfsetbuttcap%
\pgfsetroundjoin%
\pgfsetlinewidth{0.501875pt}%
\definecolor{currentstroke}{rgb}{0.000000,0.000000,0.000000}%
\pgfsetstrokecolor{currentstroke}%
\pgfsetdash{}{0pt}%
\pgfpathmoveto{\pgfqpoint{2.347263in}{0.742169in}}%
\pgfpathlineto{\pgfqpoint{2.347263in}{0.753681in}}%
\pgfusepath{stroke}%
\end{pgfscope}%
\begin{pgfscope}%
\pgfpathrectangle{\pgfqpoint{0.636356in}{0.700846in}}{\pgfqpoint{1.933229in}{1.137023in}} %
\pgfusepath{clip}%
\pgfsetbuttcap%
\pgfsetroundjoin%
\pgfsetlinewidth{0.501875pt}%
\definecolor{currentstroke}{rgb}{0.000000,0.000000,0.000000}%
\pgfsetstrokecolor{currentstroke}%
\pgfsetdash{}{0pt}%
\pgfpathmoveto{\pgfqpoint{2.366595in}{0.734459in}}%
\pgfpathlineto{\pgfqpoint{2.366595in}{0.744971in}}%
\pgfusepath{stroke}%
\end{pgfscope}%
\begin{pgfscope}%
\pgfpathrectangle{\pgfqpoint{0.636356in}{0.700846in}}{\pgfqpoint{1.933229in}{1.137023in}} %
\pgfusepath{clip}%
\pgfsetbuttcap%
\pgfsetroundjoin%
\pgfsetlinewidth{0.501875pt}%
\definecolor{currentstroke}{rgb}{0.000000,0.000000,0.000000}%
\pgfsetstrokecolor{currentstroke}%
\pgfsetdash{}{0pt}%
\pgfpathmoveto{\pgfqpoint{2.385928in}{0.724466in}}%
\pgfpathlineto{\pgfqpoint{2.385928in}{0.733494in}}%
\pgfusepath{stroke}%
\end{pgfscope}%
\begin{pgfscope}%
\pgfpathrectangle{\pgfqpoint{0.636356in}{0.700846in}}{\pgfqpoint{1.933229in}{1.137023in}} %
\pgfusepath{clip}%
\pgfsetbuttcap%
\pgfsetroundjoin%
\pgfsetlinewidth{0.501875pt}%
\definecolor{currentstroke}{rgb}{0.000000,0.000000,0.000000}%
\pgfsetstrokecolor{currentstroke}%
\pgfsetdash{}{0pt}%
\pgfpathmoveto{\pgfqpoint{2.405260in}{0.729741in}}%
\pgfpathlineto{\pgfqpoint{2.405260in}{0.739585in}}%
\pgfusepath{stroke}%
\end{pgfscope}%
\begin{pgfscope}%
\pgfpathrectangle{\pgfqpoint{0.636356in}{0.700846in}}{\pgfqpoint{1.933229in}{1.137023in}} %
\pgfusepath{clip}%
\pgfsetbuttcap%
\pgfsetroundjoin%
\pgfsetlinewidth{0.501875pt}%
\definecolor{currentstroke}{rgb}{0.000000,0.000000,0.000000}%
\pgfsetstrokecolor{currentstroke}%
\pgfsetdash{}{0pt}%
\pgfpathmoveto{\pgfqpoint{2.424592in}{0.721554in}}%
\pgfpathlineto{\pgfqpoint{2.424592in}{0.730092in}}%
\pgfusepath{stroke}%
\end{pgfscope}%
\begin{pgfscope}%
\pgfpathrectangle{\pgfqpoint{0.636356in}{0.700846in}}{\pgfqpoint{1.933229in}{1.137023in}} %
\pgfusepath{clip}%
\pgfsetbuttcap%
\pgfsetroundjoin%
\pgfsetlinewidth{0.501875pt}%
\definecolor{currentstroke}{rgb}{0.000000,0.000000,0.000000}%
\pgfsetstrokecolor{currentstroke}%
\pgfsetdash{}{0pt}%
\pgfpathmoveto{\pgfqpoint{2.443925in}{0.724466in}}%
\pgfpathlineto{\pgfqpoint{2.443925in}{0.733494in}}%
\pgfusepath{stroke}%
\end{pgfscope}%
\begin{pgfscope}%
\pgfpathrectangle{\pgfqpoint{0.636356in}{0.700846in}}{\pgfqpoint{1.933229in}{1.137023in}} %
\pgfusepath{clip}%
\pgfsetbuttcap%
\pgfsetroundjoin%
\pgfsetlinewidth{0.501875pt}%
\definecolor{currentstroke}{rgb}{0.000000,0.000000,0.000000}%
\pgfsetstrokecolor{currentstroke}%
\pgfsetdash{}{0pt}%
\pgfpathmoveto{\pgfqpoint{2.463257in}{0.723882in}}%
\pgfpathlineto{\pgfqpoint{2.463257in}{0.732815in}}%
\pgfusepath{stroke}%
\end{pgfscope}%
\begin{pgfscope}%
\pgfpathrectangle{\pgfqpoint{0.636356in}{0.700846in}}{\pgfqpoint{1.933229in}{1.137023in}} %
\pgfusepath{clip}%
\pgfsetbuttcap%
\pgfsetroundjoin%
\pgfsetlinewidth{0.501875pt}%
\definecolor{currentstroke}{rgb}{0.000000,0.000000,0.000000}%
\pgfsetstrokecolor{currentstroke}%
\pgfsetdash{}{0pt}%
\pgfpathmoveto{\pgfqpoint{2.482589in}{0.712934in}}%
\pgfpathlineto{\pgfqpoint{2.482589in}{0.719776in}}%
\pgfusepath{stroke}%
\end{pgfscope}%
\begin{pgfscope}%
\pgfpathrectangle{\pgfqpoint{0.636356in}{0.700846in}}{\pgfqpoint{1.933229in}{1.137023in}} %
\pgfusepath{clip}%
\pgfsetbuttcap%
\pgfsetroundjoin%
\pgfsetlinewidth{0.501875pt}%
\definecolor{currentstroke}{rgb}{0.000000,0.000000,0.000000}%
\pgfsetstrokecolor{currentstroke}%
\pgfsetdash{}{0pt}%
\pgfpathmoveto{\pgfqpoint{2.501921in}{0.713501in}}%
\pgfpathlineto{\pgfqpoint{2.501921in}{0.720470in}}%
\pgfusepath{stroke}%
\end{pgfscope}%
\begin{pgfscope}%
\pgfpathrectangle{\pgfqpoint{0.636356in}{0.700846in}}{\pgfqpoint{1.933229in}{1.137023in}} %
\pgfusepath{clip}%
\pgfsetbuttcap%
\pgfsetroundjoin%
\pgfsetlinewidth{0.501875pt}%
\definecolor{currentstroke}{rgb}{0.000000,0.000000,0.000000}%
\pgfsetstrokecolor{currentstroke}%
\pgfsetdash{}{0pt}%
\pgfpathmoveto{\pgfqpoint{2.521254in}{0.709562in}}%
\pgfpathlineto{\pgfqpoint{2.521254in}{0.715578in}}%
\pgfusepath{stroke}%
\end{pgfscope}%
\begin{pgfscope}%
\pgfpathrectangle{\pgfqpoint{0.636356in}{0.700846in}}{\pgfqpoint{1.933229in}{1.137023in}} %
\pgfusepath{clip}%
\pgfsetbuttcap%
\pgfsetroundjoin%
\pgfsetlinewidth{0.501875pt}%
\definecolor{currentstroke}{rgb}{0.000000,0.000000,0.000000}%
\pgfsetstrokecolor{currentstroke}%
\pgfsetdash{}{0pt}%
\pgfpathmoveto{\pgfqpoint{2.540586in}{0.709006in}}%
\pgfpathlineto{\pgfqpoint{2.540586in}{0.714872in}}%
\pgfusepath{stroke}%
\end{pgfscope}%
\begin{pgfscope}%
\pgfpathrectangle{\pgfqpoint{0.636356in}{0.700846in}}{\pgfqpoint{1.933229in}{1.137023in}} %
\pgfusepath{clip}%
\pgfsetbuttcap%
\pgfsetroundjoin%
\pgfsetlinewidth{0.501875pt}%
\definecolor{currentstroke}{rgb}{0.000000,0.000000,0.000000}%
\pgfsetstrokecolor{currentstroke}%
\pgfsetdash{}{0pt}%
\pgfpathmoveto{\pgfqpoint{2.559918in}{0.710119in}}%
\pgfpathlineto{\pgfqpoint{2.559918in}{0.716282in}}%
\pgfusepath{stroke}%
\end{pgfscope}%
\begin{pgfscope}%
\pgfsetbuttcap%
\pgfsetroundjoin%
\definecolor{currentfill}{rgb}{0.000000,0.000000,0.000000}%
\pgfsetfillcolor{currentfill}%
\pgfsetlinewidth{1.003750pt}%
\definecolor{currentstroke}{rgb}{0.000000,0.000000,0.000000}%
\pgfsetstrokecolor{currentstroke}%
\pgfsetdash{}{0pt}%
\pgfsys@defobject{currentmarker}{\pgfqpoint{-0.006944in}{-0.006944in}}{\pgfqpoint{0.006944in}{0.006944in}}{%
\pgfpathmoveto{\pgfqpoint{0.000000in}{-0.006944in}}%
\pgfpathcurveto{\pgfqpoint{0.001842in}{-0.006944in}}{\pgfqpoint{0.003608in}{-0.006213in}}{\pgfqpoint{0.004910in}{-0.004910in}}%
\pgfpathcurveto{\pgfqpoint{0.006213in}{-0.003608in}}{\pgfqpoint{0.006944in}{-0.001842in}}{\pgfqpoint{0.006944in}{0.000000in}}%
\pgfpathcurveto{\pgfqpoint{0.006944in}{0.001842in}}{\pgfqpoint{0.006213in}{0.003608in}}{\pgfqpoint{0.004910in}{0.004910in}}%
\pgfpathcurveto{\pgfqpoint{0.003608in}{0.006213in}}{\pgfqpoint{0.001842in}{0.006944in}}{\pgfqpoint{0.000000in}{0.006944in}}%
\pgfpathcurveto{\pgfqpoint{-0.001842in}{0.006944in}}{\pgfqpoint{-0.003608in}{0.006213in}}{\pgfqpoint{-0.004910in}{0.004910in}}%
\pgfpathcurveto{\pgfqpoint{-0.006213in}{0.003608in}}{\pgfqpoint{-0.006944in}{0.001842in}}{\pgfqpoint{-0.006944in}{0.000000in}}%
\pgfpathcurveto{\pgfqpoint{-0.006944in}{-0.001842in}}{\pgfqpoint{-0.006213in}{-0.003608in}}{\pgfqpoint{-0.004910in}{-0.004910in}}%
\pgfpathcurveto{\pgfqpoint{-0.003608in}{-0.006213in}}{\pgfqpoint{-0.001842in}{-0.006944in}}{\pgfqpoint{0.000000in}{-0.006944in}}%
\pgfpathclose%
\pgfusepath{stroke,fill}%
}%
\begin{pgfscope}%
\pgfsys@transformshift{0.646022in}{0.712217in}%
\pgfsys@useobject{currentmarker}{}%
\end{pgfscope}%
\begin{pgfscope}%
\pgfsys@transformshift{0.665354in}{0.715375in}%
\pgfsys@useobject{currentmarker}{}%
\end{pgfscope}%
\begin{pgfscope}%
\pgfsys@transformshift{0.684687in}{0.716638in}%
\pgfsys@useobject{currentmarker}{}%
\end{pgfscope}%
\begin{pgfscope}%
\pgfsys@transformshift{0.704019in}{0.716007in}%
\pgfsys@useobject{currentmarker}{}%
\end{pgfscope}%
\begin{pgfscope}%
\pgfsys@transformshift{0.723351in}{0.718533in}%
\pgfsys@useobject{currentmarker}{}%
\end{pgfscope}%
\begin{pgfscope}%
\pgfsys@transformshift{0.742683in}{0.722323in}%
\pgfsys@useobject{currentmarker}{}%
\end{pgfscope}%
\begin{pgfscope}%
\pgfsys@transformshift{0.762016in}{0.724218in}%
\pgfsys@useobject{currentmarker}{}%
\end{pgfscope}%
\begin{pgfscope}%
\pgfsys@transformshift{0.781348in}{0.729904in}%
\pgfsys@useobject{currentmarker}{}%
\end{pgfscope}%
\begin{pgfscope}%
\pgfsys@transformshift{0.800680in}{0.728640in}%
\pgfsys@useobject{currentmarker}{}%
\end{pgfscope}%
\begin{pgfscope}%
\pgfsys@transformshift{0.820013in}{0.749486in}%
\pgfsys@useobject{currentmarker}{}%
\end{pgfscope}%
\begin{pgfscope}%
\pgfsys@transformshift{0.839345in}{0.741274in}%
\pgfsys@useobject{currentmarker}{}%
\end{pgfscope}%
\begin{pgfscope}%
\pgfsys@transformshift{0.858677in}{0.748854in}%
\pgfsys@useobject{currentmarker}{}%
\end{pgfscope}%
\begin{pgfscope}%
\pgfsys@transformshift{0.878009in}{0.748222in}%
\pgfsys@useobject{currentmarker}{}%
\end{pgfscope}%
\begin{pgfscope}%
\pgfsys@transformshift{0.897342in}{0.755171in}%
\pgfsys@useobject{currentmarker}{}%
\end{pgfscope}%
\begin{pgfscope}%
\pgfsys@transformshift{0.916674in}{0.753276in}%
\pgfsys@useobject{currentmarker}{}%
\end{pgfscope}%
\begin{pgfscope}%
\pgfsys@transformshift{0.936006in}{0.763383in}%
\pgfsys@useobject{currentmarker}{}%
\end{pgfscope}%
\begin{pgfscope}%
\pgfsys@transformshift{0.955339in}{0.781070in}%
\pgfsys@useobject{currentmarker}{}%
\end{pgfscope}%
\begin{pgfscope}%
\pgfsys@transformshift{0.974671in}{0.782965in}%
\pgfsys@useobject{currentmarker}{}%
\end{pgfscope}%
\begin{pgfscope}%
\pgfsys@transformshift{0.994003in}{0.794335in}%
\pgfsys@useobject{currentmarker}{}%
\end{pgfscope}%
\begin{pgfscope}%
\pgfsys@transformshift{1.013335in}{0.817075in}%
\pgfsys@useobject{currentmarker}{}%
\end{pgfscope}%
\begin{pgfscope}%
\pgfsys@transformshift{1.032668in}{0.834762in}%
\pgfsys@useobject{currentmarker}{}%
\end{pgfscope}%
\begin{pgfscope}%
\pgfsys@transformshift{1.052000in}{0.841079in}%
\pgfsys@useobject{currentmarker}{}%
\end{pgfscope}%
\begin{pgfscope}%
\pgfsys@transformshift{1.071332in}{0.849923in}%
\pgfsys@useobject{currentmarker}{}%
\end{pgfscope}%
\begin{pgfscope}%
\pgfsys@transformshift{1.090665in}{0.868241in}%
\pgfsys@useobject{currentmarker}{}%
\end{pgfscope}%
\begin{pgfscope}%
\pgfsys@transformshift{1.109997in}{0.887823in}%
\pgfsys@useobject{currentmarker}{}%
\end{pgfscope}%
\begin{pgfscope}%
\pgfsys@transformshift{1.129329in}{0.906774in}%
\pgfsys@useobject{currentmarker}{}%
\end{pgfscope}%
\begin{pgfscope}%
\pgfsys@transformshift{1.148661in}{0.933304in}%
\pgfsys@useobject{currentmarker}{}%
\end{pgfscope}%
\begin{pgfscope}%
\pgfsys@transformshift{1.167994in}{0.947201in}%
\pgfsys@useobject{currentmarker}{}%
\end{pgfscope}%
\begin{pgfscope}%
\pgfsys@transformshift{1.187326in}{0.980680in}%
\pgfsys@useobject{currentmarker}{}%
\end{pgfscope}%
\begin{pgfscope}%
\pgfsys@transformshift{1.206658in}{1.032478in}%
\pgfsys@useobject{currentmarker}{}%
\end{pgfscope}%
\begin{pgfscope}%
\pgfsys@transformshift{1.225991in}{1.055218in}%
\pgfsys@useobject{currentmarker}{}%
\end{pgfscope}%
\begin{pgfscope}%
\pgfsys@transformshift{1.245323in}{1.096278in}%
\pgfsys@useobject{currentmarker}{}%
\end{pgfscope}%
\begin{pgfscope}%
\pgfsys@transformshift{1.264655in}{1.091224in}%
\pgfsys@useobject{currentmarker}{}%
\end{pgfscope}%
\begin{pgfscope}%
\pgfsys@transformshift{1.283987in}{1.148075in}%
\pgfsys@useobject{currentmarker}{}%
\end{pgfscope}%
\begin{pgfscope}%
\pgfsys@transformshift{1.303320in}{1.204295in}%
\pgfsys@useobject{currentmarker}{}%
\end{pgfscope}%
\begin{pgfscope}%
\pgfsys@transformshift{1.322652in}{1.256724in}%
\pgfsys@useobject{currentmarker}{}%
\end{pgfscope}%
\begin{pgfscope}%
\pgfsys@transformshift{1.341984in}{1.292730in}%
\pgfsys@useobject{currentmarker}{}%
\end{pgfscope}%
\begin{pgfscope}%
\pgfsys@transformshift{1.361317in}{1.338842in}%
\pgfsys@useobject{currentmarker}{}%
\end{pgfscope}%
\begin{pgfscope}%
\pgfsys@transformshift{1.380649in}{1.406432in}%
\pgfsys@useobject{currentmarker}{}%
\end{pgfscope}%
\begin{pgfscope}%
\pgfsys@transformshift{1.399981in}{1.395062in}%
\pgfsys@useobject{currentmarker}{}%
\end{pgfscope}%
\begin{pgfscope}%
\pgfsys@transformshift{1.419313in}{1.496762in}%
\pgfsys@useobject{currentmarker}{}%
\end{pgfscope}%
\begin{pgfscope}%
\pgfsys@transformshift{1.438646in}{1.520134in}%
\pgfsys@useobject{currentmarker}{}%
\end{pgfscope}%
\begin{pgfscope}%
\pgfsys@transformshift{1.457978in}{1.571300in}%
\pgfsys@useobject{currentmarker}{}%
\end{pgfscope}%
\begin{pgfscope}%
\pgfsys@transformshift{1.477310in}{1.594041in}%
\pgfsys@useobject{currentmarker}{}%
\end{pgfscope}%
\begin{pgfscope}%
\pgfsys@transformshift{1.496643in}{1.630678in}%
\pgfsys@useobject{currentmarker}{}%
\end{pgfscope}%
\begin{pgfscope}%
\pgfsys@transformshift{1.515975in}{1.684371in}%
\pgfsys@useobject{currentmarker}{}%
\end{pgfscope}%
\begin{pgfscope}%
\pgfsys@transformshift{1.535307in}{1.674264in}%
\pgfsys@useobject{currentmarker}{}%
\end{pgfscope}%
\begin{pgfscope}%
\pgfsys@transformshift{1.554639in}{1.703953in}%
\pgfsys@useobject{currentmarker}{}%
\end{pgfscope}%
\begin{pgfscope}%
\pgfsys@transformshift{1.573972in}{1.752592in}%
\pgfsys@useobject{currentmarker}{}%
\end{pgfscope}%
\begin{pgfscope}%
\pgfsys@transformshift{1.593304in}{1.758277in}%
\pgfsys@useobject{currentmarker}{}%
\end{pgfscope}%
\begin{pgfscope}%
\pgfsys@transformshift{1.612636in}{1.773438in}%
\pgfsys@useobject{currentmarker}{}%
\end{pgfscope}%
\begin{pgfscope}%
\pgfsys@transformshift{1.631969in}{1.774701in}%
\pgfsys@useobject{currentmarker}{}%
\end{pgfscope}%
\begin{pgfscope}%
\pgfsys@transformshift{1.651301in}{1.772806in}%
\pgfsys@useobject{currentmarker}{}%
\end{pgfscope}%
\begin{pgfscope}%
\pgfsys@transformshift{1.670633in}{1.736169in}%
\pgfsys@useobject{currentmarker}{}%
\end{pgfscope}%
\begin{pgfscope}%
\pgfsys@transformshift{1.689965in}{1.749434in}%
\pgfsys@useobject{currentmarker}{}%
\end{pgfscope}%
\begin{pgfscope}%
\pgfsys@transformshift{1.709298in}{1.696373in}%
\pgfsys@useobject{currentmarker}{}%
\end{pgfscope}%
\begin{pgfscope}%
\pgfsys@transformshift{1.728630in}{1.707743in}%
\pgfsys@useobject{currentmarker}{}%
\end{pgfscope}%
\begin{pgfscope}%
\pgfsys@transformshift{1.747962in}{1.688161in}%
\pgfsys@useobject{currentmarker}{}%
\end{pgfscope}%
\begin{pgfscope}%
\pgfsys@transformshift{1.767295in}{1.616781in}%
\pgfsys@useobject{currentmarker}{}%
\end{pgfscope}%
\begin{pgfscope}%
\pgfsys@transformshift{1.786627in}{1.539085in}%
\pgfsys@useobject{currentmarker}{}%
\end{pgfscope}%
\begin{pgfscope}%
\pgfsys@transformshift{1.805959in}{1.496762in}%
\pgfsys@useobject{currentmarker}{}%
\end{pgfscope}%
\begin{pgfscope}%
\pgfsys@transformshift{1.825291in}{1.494867in}%
\pgfsys@useobject{currentmarker}{}%
\end{pgfscope}%
\begin{pgfscope}%
\pgfsys@transformshift{1.844624in}{1.400115in}%
\pgfsys@useobject{currentmarker}{}%
\end{pgfscope}%
\begin{pgfscope}%
\pgfsys@transformshift{1.863956in}{1.370426in}%
\pgfsys@useobject{currentmarker}{}%
\end{pgfscope}%
\begin{pgfscope}%
\pgfsys@transformshift{1.883288in}{1.350213in}%
\pgfsys@useobject{currentmarker}{}%
\end{pgfscope}%
\begin{pgfscope}%
\pgfsys@transformshift{1.902621in}{1.276306in}%
\pgfsys@useobject{currentmarker}{}%
\end{pgfscope}%
\begin{pgfscope}%
\pgfsys@transformshift{1.921953in}{1.225140in}%
\pgfsys@useobject{currentmarker}{}%
\end{pgfscope}%
\begin{pgfscope}%
\pgfsys@transformshift{1.941285in}{1.215033in}%
\pgfsys@useobject{currentmarker}{}%
\end{pgfscope}%
\begin{pgfscope}%
\pgfsys@transformshift{1.960617in}{1.154392in}%
\pgfsys@useobject{currentmarker}{}%
\end{pgfscope}%
\begin{pgfscope}%
\pgfsys@transformshift{1.979950in}{1.095014in}%
\pgfsys@useobject{currentmarker}{}%
\end{pgfscope}%
\begin{pgfscope}%
\pgfsys@transformshift{1.999282in}{1.064694in}%
\pgfsys@useobject{currentmarker}{}%
\end{pgfscope}%
\begin{pgfscope}%
\pgfsys@transformshift{2.018614in}{1.022371in}%
\pgfsys@useobject{currentmarker}{}%
\end{pgfscope}%
\begin{pgfscope}%
\pgfsys@transformshift{2.037947in}{1.006579in}%
\pgfsys@useobject{currentmarker}{}%
\end{pgfscope}%
\begin{pgfscope}%
\pgfsys@transformshift{2.057279in}{0.960466in}%
\pgfsys@useobject{currentmarker}{}%
\end{pgfscope}%
\begin{pgfscope}%
\pgfsys@transformshift{2.076611in}{0.936463in}%
\pgfsys@useobject{currentmarker}{}%
\end{pgfscope}%
\begin{pgfscope}%
\pgfsys@transformshift{2.095943in}{0.923197in}%
\pgfsys@useobject{currentmarker}{}%
\end{pgfscope}%
\begin{pgfscope}%
\pgfsys@transformshift{2.115276in}{0.885928in}%
\pgfsys@useobject{currentmarker}{}%
\end{pgfscope}%
\begin{pgfscope}%
\pgfsys@transformshift{2.134608in}{0.849291in}%
\pgfsys@useobject{currentmarker}{}%
\end{pgfscope}%
\begin{pgfscope}%
\pgfsys@transformshift{2.153940in}{0.868241in}%
\pgfsys@useobject{currentmarker}{}%
\end{pgfscope}%
\begin{pgfscope}%
\pgfsys@transformshift{2.173273in}{0.829709in}%
\pgfsys@useobject{currentmarker}{}%
\end{pgfscope}%
\begin{pgfscope}%
\pgfsys@transformshift{2.192605in}{0.815180in}%
\pgfsys@useobject{currentmarker}{}%
\end{pgfscope}%
\begin{pgfscope}%
\pgfsys@transformshift{2.211937in}{0.803810in}%
\pgfsys@useobject{currentmarker}{}%
\end{pgfscope}%
\begin{pgfscope}%
\pgfsys@transformshift{2.231269in}{0.796230in}%
\pgfsys@useobject{currentmarker}{}%
\end{pgfscope}%
\begin{pgfscope}%
\pgfsys@transformshift{2.250602in}{0.780438in}%
\pgfsys@useobject{currentmarker}{}%
\end{pgfscope}%
\begin{pgfscope}%
\pgfsys@transformshift{2.269934in}{0.776016in}%
\pgfsys@useobject{currentmarker}{}%
\end{pgfscope}%
\begin{pgfscope}%
\pgfsys@transformshift{2.289266in}{0.765909in}%
\pgfsys@useobject{currentmarker}{}%
\end{pgfscope}%
\begin{pgfscope}%
\pgfsys@transformshift{2.308599in}{0.757697in}%
\pgfsys@useobject{currentmarker}{}%
\end{pgfscope}%
\begin{pgfscope}%
\pgfsys@transformshift{2.327931in}{0.748222in}%
\pgfsys@useobject{currentmarker}{}%
\end{pgfscope}%
\begin{pgfscope}%
\pgfsys@transformshift{2.347263in}{0.747591in}%
\pgfsys@useobject{currentmarker}{}%
\end{pgfscope}%
\begin{pgfscope}%
\pgfsys@transformshift{2.366595in}{0.739379in}%
\pgfsys@useobject{currentmarker}{}%
\end{pgfscope}%
\begin{pgfscope}%
\pgfsys@transformshift{2.385928in}{0.728640in}%
\pgfsys@useobject{currentmarker}{}%
\end{pgfscope}%
\begin{pgfscope}%
\pgfsys@transformshift{2.405260in}{0.734325in}%
\pgfsys@useobject{currentmarker}{}%
\end{pgfscope}%
\begin{pgfscope}%
\pgfsys@transformshift{2.424592in}{0.725482in}%
\pgfsys@useobject{currentmarker}{}%
\end{pgfscope}%
\begin{pgfscope}%
\pgfsys@transformshift{2.443925in}{0.728640in}%
\pgfsys@useobject{currentmarker}{}%
\end{pgfscope}%
\begin{pgfscope}%
\pgfsys@transformshift{2.463257in}{0.728009in}%
\pgfsys@useobject{currentmarker}{}%
\end{pgfscope}%
\begin{pgfscope}%
\pgfsys@transformshift{2.482589in}{0.716007in}%
\pgfsys@useobject{currentmarker}{}%
\end{pgfscope}%
\begin{pgfscope}%
\pgfsys@transformshift{2.501921in}{0.716638in}%
\pgfsys@useobject{currentmarker}{}%
\end{pgfscope}%
\begin{pgfscope}%
\pgfsys@transformshift{2.521254in}{0.712217in}%
\pgfsys@useobject{currentmarker}{}%
\end{pgfscope}%
\begin{pgfscope}%
\pgfsys@transformshift{2.540586in}{0.711585in}%
\pgfsys@useobject{currentmarker}{}%
\end{pgfscope}%
\begin{pgfscope}%
\pgfsys@transformshift{2.559918in}{0.712848in}%
\pgfsys@useobject{currentmarker}{}%
\end{pgfscope}%
\end{pgfscope}%
\end{pgfpicture}%
\makeatother%
\endgroup%

  \end{subfigure}
  \begin{subfigure}[t]{0.49\textwidth}
    %% Creator: Matplotlib, PGF backend
%%
%% To include the figure in your LaTeX document, write
%%   \input{<filename>.pgf}
%%
%% Make sure the required packages are loaded in your preamble
%%   \usepackage{pgf}
%%
%% Figures using additional raster images can only be included by \input if
%% they are in the same directory as the main LaTeX file. For loading figures
%% from other directories you can use the `import` package
%%   \usepackage{import}
%% and then include the figures with
%%   \import{<path to file>}{<filename>.pgf}
%%
%% Matplotlib used the following preamble
%%   \usepackage{fontspec}
%%   \setmainfont{DejaVu Serif}
%%   \setsansfont{DejaVu Sans}
%%   \setmonofont{DejaVu Sans Mono}
%%
\begingroup%
\makeatletter%
\begin{pgfpicture}%
\pgfpathrectangle{\pgfpointorigin}{\pgfqpoint{2.760969in}{1.941634in}}%
\pgfusepath{use as bounding box, clip}%
\begin{pgfscope}%
\pgfsetbuttcap%
\pgfsetmiterjoin%
\definecolor{currentfill}{rgb}{1.000000,1.000000,1.000000}%
\pgfsetfillcolor{currentfill}%
\pgfsetlinewidth{0.000000pt}%
\definecolor{currentstroke}{rgb}{1.000000,1.000000,1.000000}%
\pgfsetstrokecolor{currentstroke}%
\pgfsetdash{}{0pt}%
\pgfpathmoveto{\pgfqpoint{-0.000000in}{-0.000000in}}%
\pgfpathlineto{\pgfqpoint{2.760969in}{-0.000000in}}%
\pgfpathlineto{\pgfqpoint{2.760969in}{1.941634in}}%
\pgfpathlineto{\pgfqpoint{-0.000000in}{1.941634in}}%
\pgfpathclose%
\pgfusepath{fill}%
\end{pgfscope}%
\begin{pgfscope}%
\pgfsetbuttcap%
\pgfsetmiterjoin%
\definecolor{currentfill}{rgb}{1.000000,1.000000,1.000000}%
\pgfsetfillcolor{currentfill}%
\pgfsetlinewidth{0.000000pt}%
\definecolor{currentstroke}{rgb}{0.000000,0.000000,0.000000}%
\pgfsetstrokecolor{currentstroke}%
\pgfsetstrokeopacity{0.000000}%
\pgfsetdash{}{0pt}%
\pgfpathmoveto{\pgfqpoint{0.636356in}{0.440955in}}%
\pgfpathlineto{\pgfqpoint{2.569584in}{0.440955in}}%
\pgfpathlineto{\pgfqpoint{2.569584in}{0.603387in}}%
\pgfpathlineto{\pgfqpoint{0.636356in}{0.603387in}}%
\pgfpathclose%
\pgfusepath{fill}%
\end{pgfscope}%
\begin{pgfscope}%
\pgfpathrectangle{\pgfqpoint{0.636356in}{0.440955in}}{\pgfqpoint{1.933229in}{0.162432in}} %
\pgfusepath{clip}%
\pgfsetbuttcap%
\pgfsetroundjoin%
\definecolor{currentfill}{rgb}{0.733333,0.733333,0.733333}%
\pgfsetfillcolor{currentfill}%
\pgfsetlinewidth{0.000000pt}%
\definecolor{currentstroke}{rgb}{0.733333,0.733333,0.733333}%
\pgfsetstrokecolor{currentstroke}%
\pgfsetdash{}{0pt}%
\pgfpathmoveto{\pgfqpoint{0.636356in}{0.549243in}}%
\pgfpathlineto{\pgfqpoint{0.636356in}{0.576315in}}%
\pgfpathlineto{\pgfqpoint{2.569584in}{0.576315in}}%
\pgfpathlineto{\pgfqpoint{2.569584in}{0.549243in}}%
\pgfpathlineto{\pgfqpoint{2.569584in}{0.549243in}}%
\pgfpathlineto{\pgfqpoint{0.636356in}{0.549243in}}%
\pgfpathlineto{\pgfqpoint{0.636356in}{0.549243in}}%
\pgfusepath{fill}%
\end{pgfscope}%
\begin{pgfscope}%
\pgfpathrectangle{\pgfqpoint{0.636356in}{0.440955in}}{\pgfqpoint{1.933229in}{0.162432in}} %
\pgfusepath{clip}%
\pgfsetbuttcap%
\pgfsetroundjoin%
\definecolor{currentfill}{rgb}{0.733333,0.733333,0.733333}%
\pgfsetfillcolor{currentfill}%
\pgfsetlinewidth{0.000000pt}%
\definecolor{currentstroke}{rgb}{0.733333,0.733333,0.733333}%
\pgfsetstrokecolor{currentstroke}%
\pgfsetdash{}{0pt}%
\pgfpathmoveto{\pgfqpoint{0.636356in}{0.495099in}}%
\pgfpathlineto{\pgfqpoint{0.636356in}{0.468027in}}%
\pgfpathlineto{\pgfqpoint{2.569584in}{0.468027in}}%
\pgfpathlineto{\pgfqpoint{2.569584in}{0.495099in}}%
\pgfpathlineto{\pgfqpoint{2.569584in}{0.495099in}}%
\pgfpathlineto{\pgfqpoint{0.636356in}{0.495099in}}%
\pgfpathlineto{\pgfqpoint{0.636356in}{0.495099in}}%
\pgfusepath{fill}%
\end{pgfscope}%
\begin{pgfscope}%
\pgfpathrectangle{\pgfqpoint{0.636356in}{0.440955in}}{\pgfqpoint{1.933229in}{0.162432in}} %
\pgfusepath{clip}%
\pgfsetbuttcap%
\pgfsetmiterjoin%
\definecolor{currentfill}{rgb}{0.333333,0.333333,0.333333}%
\pgfsetfillcolor{currentfill}%
\pgfsetlinewidth{0.501875pt}%
\definecolor{currentstroke}{rgb}{0.000000,0.000000,0.000000}%
\pgfsetstrokecolor{currentstroke}%
\pgfsetdash{}{0pt}%
\pgfpathmoveto{\pgfqpoint{0.636356in}{0.522171in}}%
\pgfpathlineto{\pgfqpoint{0.655688in}{0.522171in}}%
\pgfpathlineto{\pgfqpoint{0.655688in}{0.564610in}}%
\pgfpathlineto{\pgfqpoint{0.636356in}{0.564610in}}%
\pgfpathlineto{\pgfqpoint{0.636356in}{0.522171in}}%
\pgfusepath{stroke,fill}%
\end{pgfscope}%
\begin{pgfscope}%
\pgfpathrectangle{\pgfqpoint{0.636356in}{0.440955in}}{\pgfqpoint{1.933229in}{0.162432in}} %
\pgfusepath{clip}%
\pgfsetbuttcap%
\pgfsetmiterjoin%
\definecolor{currentfill}{rgb}{0.333333,0.333333,0.333333}%
\pgfsetfillcolor{currentfill}%
\pgfsetlinewidth{0.501875pt}%
\definecolor{currentstroke}{rgb}{0.000000,0.000000,0.000000}%
\pgfsetstrokecolor{currentstroke}%
\pgfsetdash{}{0pt}%
\pgfpathmoveto{\pgfqpoint{0.655688in}{0.489929in}}%
\pgfpathlineto{\pgfqpoint{0.675020in}{0.489929in}}%
\pgfpathlineto{\pgfqpoint{0.675020in}{0.522171in}}%
\pgfpathlineto{\pgfqpoint{0.655688in}{0.522171in}}%
\pgfpathlineto{\pgfqpoint{0.655688in}{0.489929in}}%
\pgfusepath{stroke,fill}%
\end{pgfscope}%
\begin{pgfscope}%
\pgfpathrectangle{\pgfqpoint{0.636356in}{0.440955in}}{\pgfqpoint{1.933229in}{0.162432in}} %
\pgfusepath{clip}%
\pgfsetbuttcap%
\pgfsetmiterjoin%
\definecolor{currentfill}{rgb}{0.333333,0.333333,0.333333}%
\pgfsetfillcolor{currentfill}%
\pgfsetlinewidth{0.501875pt}%
\definecolor{currentstroke}{rgb}{0.000000,0.000000,0.000000}%
\pgfsetstrokecolor{currentstroke}%
\pgfsetdash{}{0pt}%
\pgfpathmoveto{\pgfqpoint{0.675020in}{0.522171in}}%
\pgfpathlineto{\pgfqpoint{0.694353in}{0.522171in}}%
\pgfpathlineto{\pgfqpoint{0.694353in}{0.586668in}}%
\pgfpathlineto{\pgfqpoint{0.675020in}{0.586668in}}%
\pgfpathlineto{\pgfqpoint{0.675020in}{0.522171in}}%
\pgfusepath{stroke,fill}%
\end{pgfscope}%
\begin{pgfscope}%
\pgfpathrectangle{\pgfqpoint{0.636356in}{0.440955in}}{\pgfqpoint{1.933229in}{0.162432in}} %
\pgfusepath{clip}%
\pgfsetbuttcap%
\pgfsetmiterjoin%
\definecolor{currentfill}{rgb}{0.333333,0.333333,0.333333}%
\pgfsetfillcolor{currentfill}%
\pgfsetlinewidth{0.501875pt}%
\definecolor{currentstroke}{rgb}{0.000000,0.000000,0.000000}%
\pgfsetstrokecolor{currentstroke}%
\pgfsetdash{}{0pt}%
\pgfpathmoveto{\pgfqpoint{0.694353in}{0.522171in}}%
\pgfpathlineto{\pgfqpoint{0.713685in}{0.522171in}}%
\pgfpathlineto{\pgfqpoint{0.713685in}{0.548037in}}%
\pgfpathlineto{\pgfqpoint{0.694353in}{0.548037in}}%
\pgfpathlineto{\pgfqpoint{0.694353in}{0.522171in}}%
\pgfusepath{stroke,fill}%
\end{pgfscope}%
\begin{pgfscope}%
\pgfpathrectangle{\pgfqpoint{0.636356in}{0.440955in}}{\pgfqpoint{1.933229in}{0.162432in}} %
\pgfusepath{clip}%
\pgfsetbuttcap%
\pgfsetmiterjoin%
\definecolor{currentfill}{rgb}{0.333333,0.333333,0.333333}%
\pgfsetfillcolor{currentfill}%
\pgfsetlinewidth{0.501875pt}%
\definecolor{currentstroke}{rgb}{0.000000,0.000000,0.000000}%
\pgfsetstrokecolor{currentstroke}%
\pgfsetdash{}{0pt}%
\pgfpathmoveto{\pgfqpoint{0.713685in}{0.522171in}}%
\pgfpathlineto{\pgfqpoint{0.733017in}{0.522171in}}%
\pgfpathlineto{\pgfqpoint{0.733017in}{0.580490in}}%
\pgfpathlineto{\pgfqpoint{0.713685in}{0.580490in}}%
\pgfpathlineto{\pgfqpoint{0.713685in}{0.522171in}}%
\pgfusepath{stroke,fill}%
\end{pgfscope}%
\begin{pgfscope}%
\pgfpathrectangle{\pgfqpoint{0.636356in}{0.440955in}}{\pgfqpoint{1.933229in}{0.162432in}} %
\pgfusepath{clip}%
\pgfsetbuttcap%
\pgfsetmiterjoin%
\definecolor{currentfill}{rgb}{0.333333,0.333333,0.333333}%
\pgfsetfillcolor{currentfill}%
\pgfsetlinewidth{0.501875pt}%
\definecolor{currentstroke}{rgb}{0.000000,0.000000,0.000000}%
\pgfsetstrokecolor{currentstroke}%
\pgfsetdash{}{0pt}%
\pgfpathmoveto{\pgfqpoint{0.733017in}{0.522171in}}%
\pgfpathlineto{\pgfqpoint{0.752350in}{0.522171in}}%
\pgfpathlineto{\pgfqpoint{0.752350in}{0.554552in}}%
\pgfpathlineto{\pgfqpoint{0.733017in}{0.554552in}}%
\pgfpathlineto{\pgfqpoint{0.733017in}{0.522171in}}%
\pgfusepath{stroke,fill}%
\end{pgfscope}%
\begin{pgfscope}%
\pgfpathrectangle{\pgfqpoint{0.636356in}{0.440955in}}{\pgfqpoint{1.933229in}{0.162432in}} %
\pgfusepath{clip}%
\pgfsetbuttcap%
\pgfsetmiterjoin%
\definecolor{currentfill}{rgb}{0.333333,0.333333,0.333333}%
\pgfsetfillcolor{currentfill}%
\pgfsetlinewidth{0.501875pt}%
\definecolor{currentstroke}{rgb}{0.000000,0.000000,0.000000}%
\pgfsetstrokecolor{currentstroke}%
\pgfsetdash{}{0pt}%
\pgfpathmoveto{\pgfqpoint{0.752350in}{0.519045in}}%
\pgfpathlineto{\pgfqpoint{0.771682in}{0.519045in}}%
\pgfpathlineto{\pgfqpoint{0.771682in}{0.522171in}}%
\pgfpathlineto{\pgfqpoint{0.752350in}{0.522171in}}%
\pgfpathlineto{\pgfqpoint{0.752350in}{0.519045in}}%
\pgfusepath{stroke,fill}%
\end{pgfscope}%
\begin{pgfscope}%
\pgfpathrectangle{\pgfqpoint{0.636356in}{0.440955in}}{\pgfqpoint{1.933229in}{0.162432in}} %
\pgfusepath{clip}%
\pgfsetbuttcap%
\pgfsetmiterjoin%
\definecolor{currentfill}{rgb}{0.333333,0.333333,0.333333}%
\pgfsetfillcolor{currentfill}%
\pgfsetlinewidth{0.501875pt}%
\definecolor{currentstroke}{rgb}{0.000000,0.000000,0.000000}%
\pgfsetstrokecolor{currentstroke}%
\pgfsetdash{}{0pt}%
\pgfpathmoveto{\pgfqpoint{0.771682in}{0.522171in}}%
\pgfpathlineto{\pgfqpoint{0.791014in}{0.522171in}}%
\pgfpathlineto{\pgfqpoint{0.791014in}{0.539448in}}%
\pgfpathlineto{\pgfqpoint{0.771682in}{0.539448in}}%
\pgfpathlineto{\pgfqpoint{0.771682in}{0.522171in}}%
\pgfusepath{stroke,fill}%
\end{pgfscope}%
\begin{pgfscope}%
\pgfpathrectangle{\pgfqpoint{0.636356in}{0.440955in}}{\pgfqpoint{1.933229in}{0.162432in}} %
\pgfusepath{clip}%
\pgfsetbuttcap%
\pgfsetmiterjoin%
\definecolor{currentfill}{rgb}{0.333333,0.333333,0.333333}%
\pgfsetfillcolor{currentfill}%
\pgfsetlinewidth{0.501875pt}%
\definecolor{currentstroke}{rgb}{0.000000,0.000000,0.000000}%
\pgfsetstrokecolor{currentstroke}%
\pgfsetdash{}{0pt}%
\pgfpathmoveto{\pgfqpoint{0.791014in}{0.500846in}}%
\pgfpathlineto{\pgfqpoint{0.810346in}{0.500846in}}%
\pgfpathlineto{\pgfqpoint{0.810346in}{0.522171in}}%
\pgfpathlineto{\pgfqpoint{0.791014in}{0.522171in}}%
\pgfpathlineto{\pgfqpoint{0.791014in}{0.500846in}}%
\pgfusepath{stroke,fill}%
\end{pgfscope}%
\begin{pgfscope}%
\pgfpathrectangle{\pgfqpoint{0.636356in}{0.440955in}}{\pgfqpoint{1.933229in}{0.162432in}} %
\pgfusepath{clip}%
\pgfsetbuttcap%
\pgfsetmiterjoin%
\definecolor{currentfill}{rgb}{0.333333,0.333333,0.333333}%
\pgfsetfillcolor{currentfill}%
\pgfsetlinewidth{0.501875pt}%
\definecolor{currentstroke}{rgb}{0.000000,0.000000,0.000000}%
\pgfsetstrokecolor{currentstroke}%
\pgfsetdash{}{0pt}%
\pgfpathmoveto{\pgfqpoint{0.810346in}{0.516970in}}%
\pgfpathlineto{\pgfqpoint{0.829679in}{0.516970in}}%
\pgfpathlineto{\pgfqpoint{0.829679in}{0.522171in}}%
\pgfpathlineto{\pgfqpoint{0.810346in}{0.522171in}}%
\pgfpathlineto{\pgfqpoint{0.810346in}{0.516970in}}%
\pgfusepath{stroke,fill}%
\end{pgfscope}%
\begin{pgfscope}%
\pgfpathrectangle{\pgfqpoint{0.636356in}{0.440955in}}{\pgfqpoint{1.933229in}{0.162432in}} %
\pgfusepath{clip}%
\pgfsetbuttcap%
\pgfsetmiterjoin%
\definecolor{currentfill}{rgb}{0.333333,0.333333,0.333333}%
\pgfsetfillcolor{currentfill}%
\pgfsetlinewidth{0.501875pt}%
\definecolor{currentstroke}{rgb}{0.000000,0.000000,0.000000}%
\pgfsetstrokecolor{currentstroke}%
\pgfsetdash{}{0pt}%
\pgfpathmoveto{\pgfqpoint{0.829679in}{0.480772in}}%
\pgfpathlineto{\pgfqpoint{0.849011in}{0.480772in}}%
\pgfpathlineto{\pgfqpoint{0.849011in}{0.522171in}}%
\pgfpathlineto{\pgfqpoint{0.829679in}{0.522171in}}%
\pgfpathlineto{\pgfqpoint{0.829679in}{0.480772in}}%
\pgfusepath{stroke,fill}%
\end{pgfscope}%
\begin{pgfscope}%
\pgfpathrectangle{\pgfqpoint{0.636356in}{0.440955in}}{\pgfqpoint{1.933229in}{0.162432in}} %
\pgfusepath{clip}%
\pgfsetbuttcap%
\pgfsetmiterjoin%
\definecolor{currentfill}{rgb}{0.333333,0.333333,0.333333}%
\pgfsetfillcolor{currentfill}%
\pgfsetlinewidth{0.501875pt}%
\definecolor{currentstroke}{rgb}{0.000000,0.000000,0.000000}%
\pgfsetstrokecolor{currentstroke}%
\pgfsetdash{}{0pt}%
\pgfpathmoveto{\pgfqpoint{0.849011in}{0.522171in}}%
\pgfpathlineto{\pgfqpoint{0.868343in}{0.522171in}}%
\pgfpathlineto{\pgfqpoint{0.868343in}{0.526636in}}%
\pgfpathlineto{\pgfqpoint{0.849011in}{0.526636in}}%
\pgfpathlineto{\pgfqpoint{0.849011in}{0.522171in}}%
\pgfusepath{stroke,fill}%
\end{pgfscope}%
\begin{pgfscope}%
\pgfpathrectangle{\pgfqpoint{0.636356in}{0.440955in}}{\pgfqpoint{1.933229in}{0.162432in}} %
\pgfusepath{clip}%
\pgfsetbuttcap%
\pgfsetmiterjoin%
\definecolor{currentfill}{rgb}{0.333333,0.333333,0.333333}%
\pgfsetfillcolor{currentfill}%
\pgfsetlinewidth{0.501875pt}%
\definecolor{currentstroke}{rgb}{0.000000,0.000000,0.000000}%
\pgfsetstrokecolor{currentstroke}%
\pgfsetdash{}{0pt}%
\pgfpathmoveto{\pgfqpoint{0.868343in}{0.522171in}}%
\pgfpathlineto{\pgfqpoint{0.887676in}{0.522171in}}%
\pgfpathlineto{\pgfqpoint{0.887676in}{0.540614in}}%
\pgfpathlineto{\pgfqpoint{0.868343in}{0.540614in}}%
\pgfpathlineto{\pgfqpoint{0.868343in}{0.522171in}}%
\pgfusepath{stroke,fill}%
\end{pgfscope}%
\begin{pgfscope}%
\pgfpathrectangle{\pgfqpoint{0.636356in}{0.440955in}}{\pgfqpoint{1.933229in}{0.162432in}} %
\pgfusepath{clip}%
\pgfsetbuttcap%
\pgfsetmiterjoin%
\definecolor{currentfill}{rgb}{0.333333,0.333333,0.333333}%
\pgfsetfillcolor{currentfill}%
\pgfsetlinewidth{0.501875pt}%
\definecolor{currentstroke}{rgb}{0.000000,0.000000,0.000000}%
\pgfsetstrokecolor{currentstroke}%
\pgfsetdash{}{0pt}%
\pgfpathmoveto{\pgfqpoint{0.887676in}{0.522171in}}%
\pgfpathlineto{\pgfqpoint{0.907008in}{0.522171in}}%
\pgfpathlineto{\pgfqpoint{0.907008in}{0.524860in}}%
\pgfpathlineto{\pgfqpoint{0.887676in}{0.524860in}}%
\pgfpathlineto{\pgfqpoint{0.887676in}{0.522171in}}%
\pgfusepath{stroke,fill}%
\end{pgfscope}%
\begin{pgfscope}%
\pgfpathrectangle{\pgfqpoint{0.636356in}{0.440955in}}{\pgfqpoint{1.933229in}{0.162432in}} %
\pgfusepath{clip}%
\pgfsetbuttcap%
\pgfsetmiterjoin%
\definecolor{currentfill}{rgb}{0.333333,0.333333,0.333333}%
\pgfsetfillcolor{currentfill}%
\pgfsetlinewidth{0.501875pt}%
\definecolor{currentstroke}{rgb}{0.000000,0.000000,0.000000}%
\pgfsetstrokecolor{currentstroke}%
\pgfsetdash{}{0pt}%
\pgfpathmoveto{\pgfqpoint{0.907008in}{0.522171in}}%
\pgfpathlineto{\pgfqpoint{0.926340in}{0.522171in}}%
\pgfpathlineto{\pgfqpoint{0.926340in}{0.540516in}}%
\pgfpathlineto{\pgfqpoint{0.907008in}{0.540516in}}%
\pgfpathlineto{\pgfqpoint{0.907008in}{0.522171in}}%
\pgfusepath{stroke,fill}%
\end{pgfscope}%
\begin{pgfscope}%
\pgfpathrectangle{\pgfqpoint{0.636356in}{0.440955in}}{\pgfqpoint{1.933229in}{0.162432in}} %
\pgfusepath{clip}%
\pgfsetbuttcap%
\pgfsetmiterjoin%
\definecolor{currentfill}{rgb}{0.333333,0.333333,0.333333}%
\pgfsetfillcolor{currentfill}%
\pgfsetlinewidth{0.501875pt}%
\definecolor{currentstroke}{rgb}{0.000000,0.000000,0.000000}%
\pgfsetstrokecolor{currentstroke}%
\pgfsetdash{}{0pt}%
\pgfpathmoveto{\pgfqpoint{0.926340in}{0.522171in}}%
\pgfpathlineto{\pgfqpoint{0.945672in}{0.522171in}}%
\pgfpathlineto{\pgfqpoint{0.945672in}{0.549472in}}%
\pgfpathlineto{\pgfqpoint{0.926340in}{0.549472in}}%
\pgfpathlineto{\pgfqpoint{0.926340in}{0.522171in}}%
\pgfusepath{stroke,fill}%
\end{pgfscope}%
\begin{pgfscope}%
\pgfpathrectangle{\pgfqpoint{0.636356in}{0.440955in}}{\pgfqpoint{1.933229in}{0.162432in}} %
\pgfusepath{clip}%
\pgfsetbuttcap%
\pgfsetmiterjoin%
\definecolor{currentfill}{rgb}{0.333333,0.333333,0.333333}%
\pgfsetfillcolor{currentfill}%
\pgfsetlinewidth{0.501875pt}%
\definecolor{currentstroke}{rgb}{0.000000,0.000000,0.000000}%
\pgfsetstrokecolor{currentstroke}%
\pgfsetdash{}{0pt}%
\pgfpathmoveto{\pgfqpoint{0.945672in}{0.522171in}}%
\pgfpathlineto{\pgfqpoint{0.965005in}{0.522171in}}%
\pgfpathlineto{\pgfqpoint{0.965005in}{0.550590in}}%
\pgfpathlineto{\pgfqpoint{0.945672in}{0.550590in}}%
\pgfpathlineto{\pgfqpoint{0.945672in}{0.522171in}}%
\pgfusepath{stroke,fill}%
\end{pgfscope}%
\begin{pgfscope}%
\pgfpathrectangle{\pgfqpoint{0.636356in}{0.440955in}}{\pgfqpoint{1.933229in}{0.162432in}} %
\pgfusepath{clip}%
\pgfsetbuttcap%
\pgfsetmiterjoin%
\definecolor{currentfill}{rgb}{0.333333,0.333333,0.333333}%
\pgfsetfillcolor{currentfill}%
\pgfsetlinewidth{0.501875pt}%
\definecolor{currentstroke}{rgb}{0.000000,0.000000,0.000000}%
\pgfsetstrokecolor{currentstroke}%
\pgfsetdash{}{0pt}%
\pgfpathmoveto{\pgfqpoint{0.965005in}{0.522171in}}%
\pgfpathlineto{\pgfqpoint{0.984337in}{0.522171in}}%
\pgfpathlineto{\pgfqpoint{0.984337in}{0.574619in}}%
\pgfpathlineto{\pgfqpoint{0.965005in}{0.574619in}}%
\pgfpathlineto{\pgfqpoint{0.965005in}{0.522171in}}%
\pgfusepath{stroke,fill}%
\end{pgfscope}%
\begin{pgfscope}%
\pgfpathrectangle{\pgfqpoint{0.636356in}{0.440955in}}{\pgfqpoint{1.933229in}{0.162432in}} %
\pgfusepath{clip}%
\pgfsetbuttcap%
\pgfsetmiterjoin%
\definecolor{currentfill}{rgb}{0.333333,0.333333,0.333333}%
\pgfsetfillcolor{currentfill}%
\pgfsetlinewidth{0.501875pt}%
\definecolor{currentstroke}{rgb}{0.000000,0.000000,0.000000}%
\pgfsetstrokecolor{currentstroke}%
\pgfsetdash{}{0pt}%
\pgfpathmoveto{\pgfqpoint{0.984337in}{0.494756in}}%
\pgfpathlineto{\pgfqpoint{1.003669in}{0.494756in}}%
\pgfpathlineto{\pgfqpoint{1.003669in}{0.522171in}}%
\pgfpathlineto{\pgfqpoint{0.984337in}{0.522171in}}%
\pgfpathlineto{\pgfqpoint{0.984337in}{0.494756in}}%
\pgfusepath{stroke,fill}%
\end{pgfscope}%
\begin{pgfscope}%
\pgfpathrectangle{\pgfqpoint{0.636356in}{0.440955in}}{\pgfqpoint{1.933229in}{0.162432in}} %
\pgfusepath{clip}%
\pgfsetbuttcap%
\pgfsetmiterjoin%
\definecolor{currentfill}{rgb}{0.333333,0.333333,0.333333}%
\pgfsetfillcolor{currentfill}%
\pgfsetlinewidth{0.501875pt}%
\definecolor{currentstroke}{rgb}{0.000000,0.000000,0.000000}%
\pgfsetstrokecolor{currentstroke}%
\pgfsetdash{}{0pt}%
\pgfpathmoveto{\pgfqpoint{1.003669in}{0.522171in}}%
\pgfpathlineto{\pgfqpoint{1.023002in}{0.522171in}}%
\pgfpathlineto{\pgfqpoint{1.023002in}{0.530314in}}%
\pgfpathlineto{\pgfqpoint{1.003669in}{0.530314in}}%
\pgfpathlineto{\pgfqpoint{1.003669in}{0.522171in}}%
\pgfusepath{stroke,fill}%
\end{pgfscope}%
\begin{pgfscope}%
\pgfpathrectangle{\pgfqpoint{0.636356in}{0.440955in}}{\pgfqpoint{1.933229in}{0.162432in}} %
\pgfusepath{clip}%
\pgfsetbuttcap%
\pgfsetmiterjoin%
\definecolor{currentfill}{rgb}{0.333333,0.333333,0.333333}%
\pgfsetfillcolor{currentfill}%
\pgfsetlinewidth{0.501875pt}%
\definecolor{currentstroke}{rgb}{0.000000,0.000000,0.000000}%
\pgfsetstrokecolor{currentstroke}%
\pgfsetdash{}{0pt}%
\pgfpathmoveto{\pgfqpoint{1.023002in}{0.483788in}}%
\pgfpathlineto{\pgfqpoint{1.042334in}{0.483788in}}%
\pgfpathlineto{\pgfqpoint{1.042334in}{0.522171in}}%
\pgfpathlineto{\pgfqpoint{1.023002in}{0.522171in}}%
\pgfpathlineto{\pgfqpoint{1.023002in}{0.483788in}}%
\pgfusepath{stroke,fill}%
\end{pgfscope}%
\begin{pgfscope}%
\pgfpathrectangle{\pgfqpoint{0.636356in}{0.440955in}}{\pgfqpoint{1.933229in}{0.162432in}} %
\pgfusepath{clip}%
\pgfsetbuttcap%
\pgfsetmiterjoin%
\definecolor{currentfill}{rgb}{0.333333,0.333333,0.333333}%
\pgfsetfillcolor{currentfill}%
\pgfsetlinewidth{0.501875pt}%
\definecolor{currentstroke}{rgb}{0.000000,0.000000,0.000000}%
\pgfsetstrokecolor{currentstroke}%
\pgfsetdash{}{0pt}%
\pgfpathmoveto{\pgfqpoint{1.042334in}{0.522171in}}%
\pgfpathlineto{\pgfqpoint{1.061666in}{0.522171in}}%
\pgfpathlineto{\pgfqpoint{1.061666in}{0.557309in}}%
\pgfpathlineto{\pgfqpoint{1.042334in}{0.557309in}}%
\pgfpathlineto{\pgfqpoint{1.042334in}{0.522171in}}%
\pgfusepath{stroke,fill}%
\end{pgfscope}%
\begin{pgfscope}%
\pgfpathrectangle{\pgfqpoint{0.636356in}{0.440955in}}{\pgfqpoint{1.933229in}{0.162432in}} %
\pgfusepath{clip}%
\pgfsetbuttcap%
\pgfsetmiterjoin%
\definecolor{currentfill}{rgb}{0.333333,0.333333,0.333333}%
\pgfsetfillcolor{currentfill}%
\pgfsetlinewidth{0.501875pt}%
\definecolor{currentstroke}{rgb}{0.000000,0.000000,0.000000}%
\pgfsetstrokecolor{currentstroke}%
\pgfsetdash{}{0pt}%
\pgfpathmoveto{\pgfqpoint{1.061666in}{0.520673in}}%
\pgfpathlineto{\pgfqpoint{1.080998in}{0.520673in}}%
\pgfpathlineto{\pgfqpoint{1.080998in}{0.522171in}}%
\pgfpathlineto{\pgfqpoint{1.061666in}{0.522171in}}%
\pgfpathlineto{\pgfqpoint{1.061666in}{0.520673in}}%
\pgfusepath{stroke,fill}%
\end{pgfscope}%
\begin{pgfscope}%
\pgfpathrectangle{\pgfqpoint{0.636356in}{0.440955in}}{\pgfqpoint{1.933229in}{0.162432in}} %
\pgfusepath{clip}%
\pgfsetbuttcap%
\pgfsetmiterjoin%
\definecolor{currentfill}{rgb}{0.333333,0.333333,0.333333}%
\pgfsetfillcolor{currentfill}%
\pgfsetlinewidth{0.501875pt}%
\definecolor{currentstroke}{rgb}{0.000000,0.000000,0.000000}%
\pgfsetstrokecolor{currentstroke}%
\pgfsetdash{}{0pt}%
\pgfpathmoveto{\pgfqpoint{1.080998in}{0.522171in}}%
\pgfpathlineto{\pgfqpoint{1.100331in}{0.522171in}}%
\pgfpathlineto{\pgfqpoint{1.100331in}{0.573002in}}%
\pgfpathlineto{\pgfqpoint{1.080998in}{0.573002in}}%
\pgfpathlineto{\pgfqpoint{1.080998in}{0.522171in}}%
\pgfusepath{stroke,fill}%
\end{pgfscope}%
\begin{pgfscope}%
\pgfpathrectangle{\pgfqpoint{0.636356in}{0.440955in}}{\pgfqpoint{1.933229in}{0.162432in}} %
\pgfusepath{clip}%
\pgfsetbuttcap%
\pgfsetmiterjoin%
\definecolor{currentfill}{rgb}{0.333333,0.333333,0.333333}%
\pgfsetfillcolor{currentfill}%
\pgfsetlinewidth{0.501875pt}%
\definecolor{currentstroke}{rgb}{0.000000,0.000000,0.000000}%
\pgfsetstrokecolor{currentstroke}%
\pgfsetdash{}{0pt}%
\pgfpathmoveto{\pgfqpoint{1.100331in}{0.522171in}}%
\pgfpathlineto{\pgfqpoint{1.119663in}{0.522171in}}%
\pgfpathlineto{\pgfqpoint{1.119663in}{0.578581in}}%
\pgfpathlineto{\pgfqpoint{1.100331in}{0.578581in}}%
\pgfpathlineto{\pgfqpoint{1.100331in}{0.522171in}}%
\pgfusepath{stroke,fill}%
\end{pgfscope}%
\begin{pgfscope}%
\pgfpathrectangle{\pgfqpoint{0.636356in}{0.440955in}}{\pgfqpoint{1.933229in}{0.162432in}} %
\pgfusepath{clip}%
\pgfsetbuttcap%
\pgfsetmiterjoin%
\definecolor{currentfill}{rgb}{0.333333,0.333333,0.333333}%
\pgfsetfillcolor{currentfill}%
\pgfsetlinewidth{0.501875pt}%
\definecolor{currentstroke}{rgb}{0.000000,0.000000,0.000000}%
\pgfsetstrokecolor{currentstroke}%
\pgfsetdash{}{0pt}%
\pgfpathmoveto{\pgfqpoint{1.119663in}{0.522171in}}%
\pgfpathlineto{\pgfqpoint{1.138995in}{0.522171in}}%
\pgfpathlineto{\pgfqpoint{1.138995in}{0.553362in}}%
\pgfpathlineto{\pgfqpoint{1.119663in}{0.553362in}}%
\pgfpathlineto{\pgfqpoint{1.119663in}{0.522171in}}%
\pgfusepath{stroke,fill}%
\end{pgfscope}%
\begin{pgfscope}%
\pgfpathrectangle{\pgfqpoint{0.636356in}{0.440955in}}{\pgfqpoint{1.933229in}{0.162432in}} %
\pgfusepath{clip}%
\pgfsetbuttcap%
\pgfsetmiterjoin%
\definecolor{currentfill}{rgb}{0.333333,0.333333,0.333333}%
\pgfsetfillcolor{currentfill}%
\pgfsetlinewidth{0.501875pt}%
\definecolor{currentstroke}{rgb}{0.000000,0.000000,0.000000}%
\pgfsetstrokecolor{currentstroke}%
\pgfsetdash{}{0pt}%
\pgfpathmoveto{\pgfqpoint{1.138995in}{0.489732in}}%
\pgfpathlineto{\pgfqpoint{1.158328in}{0.489732in}}%
\pgfpathlineto{\pgfqpoint{1.158328in}{0.522171in}}%
\pgfpathlineto{\pgfqpoint{1.138995in}{0.522171in}}%
\pgfpathlineto{\pgfqpoint{1.138995in}{0.489732in}}%
\pgfusepath{stroke,fill}%
\end{pgfscope}%
\begin{pgfscope}%
\pgfpathrectangle{\pgfqpoint{0.636356in}{0.440955in}}{\pgfqpoint{1.933229in}{0.162432in}} %
\pgfusepath{clip}%
\pgfsetbuttcap%
\pgfsetmiterjoin%
\definecolor{currentfill}{rgb}{0.333333,0.333333,0.333333}%
\pgfsetfillcolor{currentfill}%
\pgfsetlinewidth{0.501875pt}%
\definecolor{currentstroke}{rgb}{0.000000,0.000000,0.000000}%
\pgfsetstrokecolor{currentstroke}%
\pgfsetdash{}{0pt}%
\pgfpathmoveto{\pgfqpoint{1.158328in}{0.522171in}}%
\pgfpathlineto{\pgfqpoint{1.177660in}{0.522171in}}%
\pgfpathlineto{\pgfqpoint{1.177660in}{0.546413in}}%
\pgfpathlineto{\pgfqpoint{1.158328in}{0.546413in}}%
\pgfpathlineto{\pgfqpoint{1.158328in}{0.522171in}}%
\pgfusepath{stroke,fill}%
\end{pgfscope}%
\begin{pgfscope}%
\pgfpathrectangle{\pgfqpoint{0.636356in}{0.440955in}}{\pgfqpoint{1.933229in}{0.162432in}} %
\pgfusepath{clip}%
\pgfsetbuttcap%
\pgfsetmiterjoin%
\definecolor{currentfill}{rgb}{0.333333,0.333333,0.333333}%
\pgfsetfillcolor{currentfill}%
\pgfsetlinewidth{0.501875pt}%
\definecolor{currentstroke}{rgb}{0.000000,0.000000,0.000000}%
\pgfsetstrokecolor{currentstroke}%
\pgfsetdash{}{0pt}%
\pgfpathmoveto{\pgfqpoint{1.177660in}{0.483267in}}%
\pgfpathlineto{\pgfqpoint{1.196992in}{0.483267in}}%
\pgfpathlineto{\pgfqpoint{1.196992in}{0.522171in}}%
\pgfpathlineto{\pgfqpoint{1.177660in}{0.522171in}}%
\pgfpathlineto{\pgfqpoint{1.177660in}{0.483267in}}%
\pgfusepath{stroke,fill}%
\end{pgfscope}%
\begin{pgfscope}%
\pgfpathrectangle{\pgfqpoint{0.636356in}{0.440955in}}{\pgfqpoint{1.933229in}{0.162432in}} %
\pgfusepath{clip}%
\pgfsetbuttcap%
\pgfsetmiterjoin%
\definecolor{currentfill}{rgb}{0.333333,0.333333,0.333333}%
\pgfsetfillcolor{currentfill}%
\pgfsetlinewidth{0.501875pt}%
\definecolor{currentstroke}{rgb}{0.000000,0.000000,0.000000}%
\pgfsetstrokecolor{currentstroke}%
\pgfsetdash{}{0pt}%
\pgfpathmoveto{\pgfqpoint{1.196992in}{0.507439in}}%
\pgfpathlineto{\pgfqpoint{1.216324in}{0.507439in}}%
\pgfpathlineto{\pgfqpoint{1.216324in}{0.522171in}}%
\pgfpathlineto{\pgfqpoint{1.196992in}{0.522171in}}%
\pgfpathlineto{\pgfqpoint{1.196992in}{0.507439in}}%
\pgfusepath{stroke,fill}%
\end{pgfscope}%
\begin{pgfscope}%
\pgfpathrectangle{\pgfqpoint{0.636356in}{0.440955in}}{\pgfqpoint{1.933229in}{0.162432in}} %
\pgfusepath{clip}%
\pgfsetbuttcap%
\pgfsetmiterjoin%
\definecolor{currentfill}{rgb}{0.333333,0.333333,0.333333}%
\pgfsetfillcolor{currentfill}%
\pgfsetlinewidth{0.501875pt}%
\definecolor{currentstroke}{rgb}{0.000000,0.000000,0.000000}%
\pgfsetstrokecolor{currentstroke}%
\pgfsetdash{}{0pt}%
\pgfpathmoveto{\pgfqpoint{1.216324in}{0.522171in}}%
\pgfpathlineto{\pgfqpoint{1.235657in}{0.522171in}}%
\pgfpathlineto{\pgfqpoint{1.235657in}{0.543961in}}%
\pgfpathlineto{\pgfqpoint{1.216324in}{0.543961in}}%
\pgfpathlineto{\pgfqpoint{1.216324in}{0.522171in}}%
\pgfusepath{stroke,fill}%
\end{pgfscope}%
\begin{pgfscope}%
\pgfpathrectangle{\pgfqpoint{0.636356in}{0.440955in}}{\pgfqpoint{1.933229in}{0.162432in}} %
\pgfusepath{clip}%
\pgfsetbuttcap%
\pgfsetmiterjoin%
\definecolor{currentfill}{rgb}{0.333333,0.333333,0.333333}%
\pgfsetfillcolor{currentfill}%
\pgfsetlinewidth{0.501875pt}%
\definecolor{currentstroke}{rgb}{0.000000,0.000000,0.000000}%
\pgfsetstrokecolor{currentstroke}%
\pgfsetdash{}{0pt}%
\pgfpathmoveto{\pgfqpoint{1.235657in}{0.520751in}}%
\pgfpathlineto{\pgfqpoint{1.254989in}{0.520751in}}%
\pgfpathlineto{\pgfqpoint{1.254989in}{0.522171in}}%
\pgfpathlineto{\pgfqpoint{1.235657in}{0.522171in}}%
\pgfpathlineto{\pgfqpoint{1.235657in}{0.520751in}}%
\pgfusepath{stroke,fill}%
\end{pgfscope}%
\begin{pgfscope}%
\pgfpathrectangle{\pgfqpoint{0.636356in}{0.440955in}}{\pgfqpoint{1.933229in}{0.162432in}} %
\pgfusepath{clip}%
\pgfsetbuttcap%
\pgfsetmiterjoin%
\definecolor{currentfill}{rgb}{0.333333,0.333333,0.333333}%
\pgfsetfillcolor{currentfill}%
\pgfsetlinewidth{0.501875pt}%
\definecolor{currentstroke}{rgb}{0.000000,0.000000,0.000000}%
\pgfsetstrokecolor{currentstroke}%
\pgfsetdash{}{0pt}%
\pgfpathmoveto{\pgfqpoint{1.254989in}{0.444013in}}%
\pgfpathlineto{\pgfqpoint{1.274321in}{0.444013in}}%
\pgfpathlineto{\pgfqpoint{1.274321in}{0.522171in}}%
\pgfpathlineto{\pgfqpoint{1.254989in}{0.522171in}}%
\pgfpathlineto{\pgfqpoint{1.254989in}{0.444013in}}%
\pgfusepath{stroke,fill}%
\end{pgfscope}%
\begin{pgfscope}%
\pgfpathrectangle{\pgfqpoint{0.636356in}{0.440955in}}{\pgfqpoint{1.933229in}{0.162432in}} %
\pgfusepath{clip}%
\pgfsetbuttcap%
\pgfsetmiterjoin%
\definecolor{currentfill}{rgb}{0.333333,0.333333,0.333333}%
\pgfsetfillcolor{currentfill}%
\pgfsetlinewidth{0.501875pt}%
\definecolor{currentstroke}{rgb}{0.000000,0.000000,0.000000}%
\pgfsetstrokecolor{currentstroke}%
\pgfsetdash{}{0pt}%
\pgfpathmoveto{\pgfqpoint{1.274321in}{0.522171in}}%
\pgfpathlineto{\pgfqpoint{1.293654in}{0.522171in}}%
\pgfpathlineto{\pgfqpoint{1.293654in}{0.547480in}}%
\pgfpathlineto{\pgfqpoint{1.274321in}{0.547480in}}%
\pgfpathlineto{\pgfqpoint{1.274321in}{0.522171in}}%
\pgfusepath{stroke,fill}%
\end{pgfscope}%
\begin{pgfscope}%
\pgfpathrectangle{\pgfqpoint{0.636356in}{0.440955in}}{\pgfqpoint{1.933229in}{0.162432in}} %
\pgfusepath{clip}%
\pgfsetbuttcap%
\pgfsetmiterjoin%
\definecolor{currentfill}{rgb}{0.333333,0.333333,0.333333}%
\pgfsetfillcolor{currentfill}%
\pgfsetlinewidth{0.501875pt}%
\definecolor{currentstroke}{rgb}{0.000000,0.000000,0.000000}%
\pgfsetstrokecolor{currentstroke}%
\pgfsetdash{}{0pt}%
\pgfpathmoveto{\pgfqpoint{1.293654in}{0.502036in}}%
\pgfpathlineto{\pgfqpoint{1.312986in}{0.502036in}}%
\pgfpathlineto{\pgfqpoint{1.312986in}{0.522171in}}%
\pgfpathlineto{\pgfqpoint{1.293654in}{0.522171in}}%
\pgfpathlineto{\pgfqpoint{1.293654in}{0.502036in}}%
\pgfusepath{stroke,fill}%
\end{pgfscope}%
\begin{pgfscope}%
\pgfpathrectangle{\pgfqpoint{0.636356in}{0.440955in}}{\pgfqpoint{1.933229in}{0.162432in}} %
\pgfusepath{clip}%
\pgfsetbuttcap%
\pgfsetmiterjoin%
\definecolor{currentfill}{rgb}{0.333333,0.333333,0.333333}%
\pgfsetfillcolor{currentfill}%
\pgfsetlinewidth{0.501875pt}%
\definecolor{currentstroke}{rgb}{0.000000,0.000000,0.000000}%
\pgfsetstrokecolor{currentstroke}%
\pgfsetdash{}{0pt}%
\pgfpathmoveto{\pgfqpoint{1.312986in}{0.463830in}}%
\pgfpathlineto{\pgfqpoint{1.332318in}{0.463830in}}%
\pgfpathlineto{\pgfqpoint{1.332318in}{0.522171in}}%
\pgfpathlineto{\pgfqpoint{1.312986in}{0.522171in}}%
\pgfpathlineto{\pgfqpoint{1.312986in}{0.463830in}}%
\pgfusepath{stroke,fill}%
\end{pgfscope}%
\begin{pgfscope}%
\pgfpathrectangle{\pgfqpoint{0.636356in}{0.440955in}}{\pgfqpoint{1.933229in}{0.162432in}} %
\pgfusepath{clip}%
\pgfsetbuttcap%
\pgfsetmiterjoin%
\definecolor{currentfill}{rgb}{0.333333,0.333333,0.333333}%
\pgfsetfillcolor{currentfill}%
\pgfsetlinewidth{0.501875pt}%
\definecolor{currentstroke}{rgb}{0.000000,0.000000,0.000000}%
\pgfsetstrokecolor{currentstroke}%
\pgfsetdash{}{0pt}%
\pgfpathmoveto{\pgfqpoint{1.332318in}{0.522171in}}%
\pgfpathlineto{\pgfqpoint{1.351650in}{0.522171in}}%
\pgfpathlineto{\pgfqpoint{1.351650in}{0.550977in}}%
\pgfpathlineto{\pgfqpoint{1.332318in}{0.550977in}}%
\pgfpathlineto{\pgfqpoint{1.332318in}{0.522171in}}%
\pgfusepath{stroke,fill}%
\end{pgfscope}%
\begin{pgfscope}%
\pgfpathrectangle{\pgfqpoint{0.636356in}{0.440955in}}{\pgfqpoint{1.933229in}{0.162432in}} %
\pgfusepath{clip}%
\pgfsetbuttcap%
\pgfsetmiterjoin%
\definecolor{currentfill}{rgb}{0.333333,0.333333,0.333333}%
\pgfsetfillcolor{currentfill}%
\pgfsetlinewidth{0.501875pt}%
\definecolor{currentstroke}{rgb}{0.000000,0.000000,0.000000}%
\pgfsetstrokecolor{currentstroke}%
\pgfsetdash{}{0pt}%
\pgfpathmoveto{\pgfqpoint{1.351650in}{0.465779in}}%
\pgfpathlineto{\pgfqpoint{1.370983in}{0.465779in}}%
\pgfpathlineto{\pgfqpoint{1.370983in}{0.522171in}}%
\pgfpathlineto{\pgfqpoint{1.351650in}{0.522171in}}%
\pgfpathlineto{\pgfqpoint{1.351650in}{0.465779in}}%
\pgfusepath{stroke,fill}%
\end{pgfscope}%
\begin{pgfscope}%
\pgfpathrectangle{\pgfqpoint{0.636356in}{0.440955in}}{\pgfqpoint{1.933229in}{0.162432in}} %
\pgfusepath{clip}%
\pgfsetbuttcap%
\pgfsetmiterjoin%
\definecolor{currentfill}{rgb}{0.333333,0.333333,0.333333}%
\pgfsetfillcolor{currentfill}%
\pgfsetlinewidth{0.501875pt}%
\definecolor{currentstroke}{rgb}{0.000000,0.000000,0.000000}%
\pgfsetstrokecolor{currentstroke}%
\pgfsetdash{}{0pt}%
\pgfpathmoveto{\pgfqpoint{1.370983in}{0.513861in}}%
\pgfpathlineto{\pgfqpoint{1.390315in}{0.513861in}}%
\pgfpathlineto{\pgfqpoint{1.390315in}{0.522171in}}%
\pgfpathlineto{\pgfqpoint{1.370983in}{0.522171in}}%
\pgfpathlineto{\pgfqpoint{1.370983in}{0.513861in}}%
\pgfusepath{stroke,fill}%
\end{pgfscope}%
\begin{pgfscope}%
\pgfpathrectangle{\pgfqpoint{0.636356in}{0.440955in}}{\pgfqpoint{1.933229in}{0.162432in}} %
\pgfusepath{clip}%
\pgfsetbuttcap%
\pgfsetmiterjoin%
\definecolor{currentfill}{rgb}{0.333333,0.333333,0.333333}%
\pgfsetfillcolor{currentfill}%
\pgfsetlinewidth{0.501875pt}%
\definecolor{currentstroke}{rgb}{0.000000,0.000000,0.000000}%
\pgfsetstrokecolor{currentstroke}%
\pgfsetdash{}{0pt}%
\pgfpathmoveto{\pgfqpoint{1.390315in}{0.475234in}}%
\pgfpathlineto{\pgfqpoint{1.409647in}{0.475234in}}%
\pgfpathlineto{\pgfqpoint{1.409647in}{0.522171in}}%
\pgfpathlineto{\pgfqpoint{1.390315in}{0.522171in}}%
\pgfpathlineto{\pgfqpoint{1.390315in}{0.475234in}}%
\pgfusepath{stroke,fill}%
\end{pgfscope}%
\begin{pgfscope}%
\pgfpathrectangle{\pgfqpoint{0.636356in}{0.440955in}}{\pgfqpoint{1.933229in}{0.162432in}} %
\pgfusepath{clip}%
\pgfsetbuttcap%
\pgfsetmiterjoin%
\definecolor{currentfill}{rgb}{0.333333,0.333333,0.333333}%
\pgfsetfillcolor{currentfill}%
\pgfsetlinewidth{0.501875pt}%
\definecolor{currentstroke}{rgb}{0.000000,0.000000,0.000000}%
\pgfsetstrokecolor{currentstroke}%
\pgfsetdash{}{0pt}%
\pgfpathmoveto{\pgfqpoint{1.409647in}{0.522171in}}%
\pgfpathlineto{\pgfqpoint{1.428980in}{0.522171in}}%
\pgfpathlineto{\pgfqpoint{1.428980in}{0.540173in}}%
\pgfpathlineto{\pgfqpoint{1.409647in}{0.540173in}}%
\pgfpathlineto{\pgfqpoint{1.409647in}{0.522171in}}%
\pgfusepath{stroke,fill}%
\end{pgfscope}%
\begin{pgfscope}%
\pgfpathrectangle{\pgfqpoint{0.636356in}{0.440955in}}{\pgfqpoint{1.933229in}{0.162432in}} %
\pgfusepath{clip}%
\pgfsetbuttcap%
\pgfsetmiterjoin%
\definecolor{currentfill}{rgb}{0.333333,0.333333,0.333333}%
\pgfsetfillcolor{currentfill}%
\pgfsetlinewidth{0.501875pt}%
\definecolor{currentstroke}{rgb}{0.000000,0.000000,0.000000}%
\pgfsetstrokecolor{currentstroke}%
\pgfsetdash{}{0pt}%
\pgfpathmoveto{\pgfqpoint{1.428980in}{0.508350in}}%
\pgfpathlineto{\pgfqpoint{1.448312in}{0.508350in}}%
\pgfpathlineto{\pgfqpoint{1.448312in}{0.522171in}}%
\pgfpathlineto{\pgfqpoint{1.428980in}{0.522171in}}%
\pgfpathlineto{\pgfqpoint{1.428980in}{0.508350in}}%
\pgfusepath{stroke,fill}%
\end{pgfscope}%
\begin{pgfscope}%
\pgfpathrectangle{\pgfqpoint{0.636356in}{0.440955in}}{\pgfqpoint{1.933229in}{0.162432in}} %
\pgfusepath{clip}%
\pgfsetbuttcap%
\pgfsetmiterjoin%
\definecolor{currentfill}{rgb}{0.333333,0.333333,0.333333}%
\pgfsetfillcolor{currentfill}%
\pgfsetlinewidth{0.501875pt}%
\definecolor{currentstroke}{rgb}{0.000000,0.000000,0.000000}%
\pgfsetstrokecolor{currentstroke}%
\pgfsetdash{}{0pt}%
\pgfpathmoveto{\pgfqpoint{1.448312in}{0.514351in}}%
\pgfpathlineto{\pgfqpoint{1.467644in}{0.514351in}}%
\pgfpathlineto{\pgfqpoint{1.467644in}{0.522171in}}%
\pgfpathlineto{\pgfqpoint{1.448312in}{0.522171in}}%
\pgfpathlineto{\pgfqpoint{1.448312in}{0.514351in}}%
\pgfusepath{stroke,fill}%
\end{pgfscope}%
\begin{pgfscope}%
\pgfpathrectangle{\pgfqpoint{0.636356in}{0.440955in}}{\pgfqpoint{1.933229in}{0.162432in}} %
\pgfusepath{clip}%
\pgfsetbuttcap%
\pgfsetmiterjoin%
\definecolor{currentfill}{rgb}{0.333333,0.333333,0.333333}%
\pgfsetfillcolor{currentfill}%
\pgfsetlinewidth{0.501875pt}%
\definecolor{currentstroke}{rgb}{0.000000,0.000000,0.000000}%
\pgfsetstrokecolor{currentstroke}%
\pgfsetdash{}{0pt}%
\pgfpathmoveto{\pgfqpoint{1.467644in}{0.522171in}}%
\pgfpathlineto{\pgfqpoint{1.486976in}{0.522171in}}%
\pgfpathlineto{\pgfqpoint{1.486976in}{0.566000in}}%
\pgfpathlineto{\pgfqpoint{1.467644in}{0.566000in}}%
\pgfpathlineto{\pgfqpoint{1.467644in}{0.522171in}}%
\pgfusepath{stroke,fill}%
\end{pgfscope}%
\begin{pgfscope}%
\pgfpathrectangle{\pgfqpoint{0.636356in}{0.440955in}}{\pgfqpoint{1.933229in}{0.162432in}} %
\pgfusepath{clip}%
\pgfsetbuttcap%
\pgfsetmiterjoin%
\definecolor{currentfill}{rgb}{0.333333,0.333333,0.333333}%
\pgfsetfillcolor{currentfill}%
\pgfsetlinewidth{0.501875pt}%
\definecolor{currentstroke}{rgb}{0.000000,0.000000,0.000000}%
\pgfsetstrokecolor{currentstroke}%
\pgfsetdash{}{0pt}%
\pgfpathmoveto{\pgfqpoint{1.486976in}{0.512833in}}%
\pgfpathlineto{\pgfqpoint{1.506309in}{0.512833in}}%
\pgfpathlineto{\pgfqpoint{1.506309in}{0.522171in}}%
\pgfpathlineto{\pgfqpoint{1.486976in}{0.522171in}}%
\pgfpathlineto{\pgfqpoint{1.486976in}{0.512833in}}%
\pgfusepath{stroke,fill}%
\end{pgfscope}%
\begin{pgfscope}%
\pgfpathrectangle{\pgfqpoint{0.636356in}{0.440955in}}{\pgfqpoint{1.933229in}{0.162432in}} %
\pgfusepath{clip}%
\pgfsetbuttcap%
\pgfsetmiterjoin%
\definecolor{currentfill}{rgb}{0.333333,0.333333,0.333333}%
\pgfsetfillcolor{currentfill}%
\pgfsetlinewidth{0.501875pt}%
\definecolor{currentstroke}{rgb}{0.000000,0.000000,0.000000}%
\pgfsetstrokecolor{currentstroke}%
\pgfsetdash{}{0pt}%
\pgfpathmoveto{\pgfqpoint{1.506309in}{0.495296in}}%
\pgfpathlineto{\pgfqpoint{1.525641in}{0.495296in}}%
\pgfpathlineto{\pgfqpoint{1.525641in}{0.522171in}}%
\pgfpathlineto{\pgfqpoint{1.506309in}{0.522171in}}%
\pgfpathlineto{\pgfqpoint{1.506309in}{0.495296in}}%
\pgfusepath{stroke,fill}%
\end{pgfscope}%
\begin{pgfscope}%
\pgfpathrectangle{\pgfqpoint{0.636356in}{0.440955in}}{\pgfqpoint{1.933229in}{0.162432in}} %
\pgfusepath{clip}%
\pgfsetbuttcap%
\pgfsetmiterjoin%
\definecolor{currentfill}{rgb}{0.333333,0.333333,0.333333}%
\pgfsetfillcolor{currentfill}%
\pgfsetlinewidth{0.501875pt}%
\definecolor{currentstroke}{rgb}{0.000000,0.000000,0.000000}%
\pgfsetstrokecolor{currentstroke}%
\pgfsetdash{}{0pt}%
\pgfpathmoveto{\pgfqpoint{1.525641in}{0.516925in}}%
\pgfpathlineto{\pgfqpoint{1.544973in}{0.516925in}}%
\pgfpathlineto{\pgfqpoint{1.544973in}{0.522171in}}%
\pgfpathlineto{\pgfqpoint{1.525641in}{0.522171in}}%
\pgfpathlineto{\pgfqpoint{1.525641in}{0.516925in}}%
\pgfusepath{stroke,fill}%
\end{pgfscope}%
\begin{pgfscope}%
\pgfpathrectangle{\pgfqpoint{0.636356in}{0.440955in}}{\pgfqpoint{1.933229in}{0.162432in}} %
\pgfusepath{clip}%
\pgfsetbuttcap%
\pgfsetmiterjoin%
\definecolor{currentfill}{rgb}{0.333333,0.333333,0.333333}%
\pgfsetfillcolor{currentfill}%
\pgfsetlinewidth{0.501875pt}%
\definecolor{currentstroke}{rgb}{0.000000,0.000000,0.000000}%
\pgfsetstrokecolor{currentstroke}%
\pgfsetdash{}{0pt}%
\pgfpathmoveto{\pgfqpoint{1.544973in}{0.511506in}}%
\pgfpathlineto{\pgfqpoint{1.564306in}{0.511506in}}%
\pgfpathlineto{\pgfqpoint{1.564306in}{0.522171in}}%
\pgfpathlineto{\pgfqpoint{1.544973in}{0.522171in}}%
\pgfpathlineto{\pgfqpoint{1.544973in}{0.511506in}}%
\pgfusepath{stroke,fill}%
\end{pgfscope}%
\begin{pgfscope}%
\pgfpathrectangle{\pgfqpoint{0.636356in}{0.440955in}}{\pgfqpoint{1.933229in}{0.162432in}} %
\pgfusepath{clip}%
\pgfsetbuttcap%
\pgfsetmiterjoin%
\definecolor{currentfill}{rgb}{0.333333,0.333333,0.333333}%
\pgfsetfillcolor{currentfill}%
\pgfsetlinewidth{0.501875pt}%
\definecolor{currentstroke}{rgb}{0.000000,0.000000,0.000000}%
\pgfsetstrokecolor{currentstroke}%
\pgfsetdash{}{0pt}%
\pgfpathmoveto{\pgfqpoint{1.564306in}{0.517313in}}%
\pgfpathlineto{\pgfqpoint{1.583638in}{0.517313in}}%
\pgfpathlineto{\pgfqpoint{1.583638in}{0.522171in}}%
\pgfpathlineto{\pgfqpoint{1.564306in}{0.522171in}}%
\pgfpathlineto{\pgfqpoint{1.564306in}{0.517313in}}%
\pgfusepath{stroke,fill}%
\end{pgfscope}%
\begin{pgfscope}%
\pgfpathrectangle{\pgfqpoint{0.636356in}{0.440955in}}{\pgfqpoint{1.933229in}{0.162432in}} %
\pgfusepath{clip}%
\pgfsetbuttcap%
\pgfsetmiterjoin%
\definecolor{currentfill}{rgb}{0.333333,0.333333,0.333333}%
\pgfsetfillcolor{currentfill}%
\pgfsetlinewidth{0.501875pt}%
\definecolor{currentstroke}{rgb}{0.000000,0.000000,0.000000}%
\pgfsetstrokecolor{currentstroke}%
\pgfsetdash{}{0pt}%
\pgfpathmoveto{\pgfqpoint{1.583638in}{0.522171in}}%
\pgfpathlineto{\pgfqpoint{1.602970in}{0.522171in}}%
\pgfpathlineto{\pgfqpoint{1.602970in}{0.546613in}}%
\pgfpathlineto{\pgfqpoint{1.583638in}{0.546613in}}%
\pgfpathlineto{\pgfqpoint{1.583638in}{0.522171in}}%
\pgfusepath{stroke,fill}%
\end{pgfscope}%
\begin{pgfscope}%
\pgfpathrectangle{\pgfqpoint{0.636356in}{0.440955in}}{\pgfqpoint{1.933229in}{0.162432in}} %
\pgfusepath{clip}%
\pgfsetbuttcap%
\pgfsetmiterjoin%
\definecolor{currentfill}{rgb}{0.333333,0.333333,0.333333}%
\pgfsetfillcolor{currentfill}%
\pgfsetlinewidth{0.501875pt}%
\definecolor{currentstroke}{rgb}{0.000000,0.000000,0.000000}%
\pgfsetstrokecolor{currentstroke}%
\pgfsetdash{}{0pt}%
\pgfpathmoveto{\pgfqpoint{1.602970in}{0.505148in}}%
\pgfpathlineto{\pgfqpoint{1.622302in}{0.505148in}}%
\pgfpathlineto{\pgfqpoint{1.622302in}{0.522171in}}%
\pgfpathlineto{\pgfqpoint{1.602970in}{0.522171in}}%
\pgfpathlineto{\pgfqpoint{1.602970in}{0.505148in}}%
\pgfusepath{stroke,fill}%
\end{pgfscope}%
\begin{pgfscope}%
\pgfpathrectangle{\pgfqpoint{0.636356in}{0.440955in}}{\pgfqpoint{1.933229in}{0.162432in}} %
\pgfusepath{clip}%
\pgfsetbuttcap%
\pgfsetmiterjoin%
\definecolor{currentfill}{rgb}{0.333333,0.333333,0.333333}%
\pgfsetfillcolor{currentfill}%
\pgfsetlinewidth{0.501875pt}%
\definecolor{currentstroke}{rgb}{0.000000,0.000000,0.000000}%
\pgfsetstrokecolor{currentstroke}%
\pgfsetdash{}{0pt}%
\pgfpathmoveto{\pgfqpoint{1.622302in}{0.512927in}}%
\pgfpathlineto{\pgfqpoint{1.641635in}{0.512927in}}%
\pgfpathlineto{\pgfqpoint{1.641635in}{0.522171in}}%
\pgfpathlineto{\pgfqpoint{1.622302in}{0.522171in}}%
\pgfpathlineto{\pgfqpoint{1.622302in}{0.512927in}}%
\pgfusepath{stroke,fill}%
\end{pgfscope}%
\begin{pgfscope}%
\pgfpathrectangle{\pgfqpoint{0.636356in}{0.440955in}}{\pgfqpoint{1.933229in}{0.162432in}} %
\pgfusepath{clip}%
\pgfsetbuttcap%
\pgfsetmiterjoin%
\definecolor{currentfill}{rgb}{0.333333,0.333333,0.333333}%
\pgfsetfillcolor{currentfill}%
\pgfsetlinewidth{0.501875pt}%
\definecolor{currentstroke}{rgb}{0.000000,0.000000,0.000000}%
\pgfsetstrokecolor{currentstroke}%
\pgfsetdash{}{0pt}%
\pgfpathmoveto{\pgfqpoint{1.641635in}{0.522171in}}%
\pgfpathlineto{\pgfqpoint{1.660967in}{0.522171in}}%
\pgfpathlineto{\pgfqpoint{1.660967in}{0.567705in}}%
\pgfpathlineto{\pgfqpoint{1.641635in}{0.567705in}}%
\pgfpathlineto{\pgfqpoint{1.641635in}{0.522171in}}%
\pgfusepath{stroke,fill}%
\end{pgfscope}%
\begin{pgfscope}%
\pgfpathrectangle{\pgfqpoint{0.636356in}{0.440955in}}{\pgfqpoint{1.933229in}{0.162432in}} %
\pgfusepath{clip}%
\pgfsetbuttcap%
\pgfsetmiterjoin%
\definecolor{currentfill}{rgb}{0.333333,0.333333,0.333333}%
\pgfsetfillcolor{currentfill}%
\pgfsetlinewidth{0.501875pt}%
\definecolor{currentstroke}{rgb}{0.000000,0.000000,0.000000}%
\pgfsetstrokecolor{currentstroke}%
\pgfsetdash{}{0pt}%
\pgfpathmoveto{\pgfqpoint{1.660967in}{0.490596in}}%
\pgfpathlineto{\pgfqpoint{1.680299in}{0.490596in}}%
\pgfpathlineto{\pgfqpoint{1.680299in}{0.522171in}}%
\pgfpathlineto{\pgfqpoint{1.660967in}{0.522171in}}%
\pgfpathlineto{\pgfqpoint{1.660967in}{0.490596in}}%
\pgfusepath{stroke,fill}%
\end{pgfscope}%
\begin{pgfscope}%
\pgfpathrectangle{\pgfqpoint{0.636356in}{0.440955in}}{\pgfqpoint{1.933229in}{0.162432in}} %
\pgfusepath{clip}%
\pgfsetbuttcap%
\pgfsetmiterjoin%
\definecolor{currentfill}{rgb}{0.333333,0.333333,0.333333}%
\pgfsetfillcolor{currentfill}%
\pgfsetlinewidth{0.501875pt}%
\definecolor{currentstroke}{rgb}{0.000000,0.000000,0.000000}%
\pgfsetstrokecolor{currentstroke}%
\pgfsetdash{}{0pt}%
\pgfpathmoveto{\pgfqpoint{1.680299in}{0.522171in}}%
\pgfpathlineto{\pgfqpoint{1.699632in}{0.522171in}}%
\pgfpathlineto{\pgfqpoint{1.699632in}{0.587809in}}%
\pgfpathlineto{\pgfqpoint{1.680299in}{0.587809in}}%
\pgfpathlineto{\pgfqpoint{1.680299in}{0.522171in}}%
\pgfusepath{stroke,fill}%
\end{pgfscope}%
\begin{pgfscope}%
\pgfpathrectangle{\pgfqpoint{0.636356in}{0.440955in}}{\pgfqpoint{1.933229in}{0.162432in}} %
\pgfusepath{clip}%
\pgfsetbuttcap%
\pgfsetmiterjoin%
\definecolor{currentfill}{rgb}{0.333333,0.333333,0.333333}%
\pgfsetfillcolor{currentfill}%
\pgfsetlinewidth{0.501875pt}%
\definecolor{currentstroke}{rgb}{0.000000,0.000000,0.000000}%
\pgfsetstrokecolor{currentstroke}%
\pgfsetdash{}{0pt}%
\pgfpathmoveto{\pgfqpoint{1.699632in}{0.522171in}}%
\pgfpathlineto{\pgfqpoint{1.718964in}{0.522171in}}%
\pgfpathlineto{\pgfqpoint{1.718964in}{0.568120in}}%
\pgfpathlineto{\pgfqpoint{1.699632in}{0.568120in}}%
\pgfpathlineto{\pgfqpoint{1.699632in}{0.522171in}}%
\pgfusepath{stroke,fill}%
\end{pgfscope}%
\begin{pgfscope}%
\pgfpathrectangle{\pgfqpoint{0.636356in}{0.440955in}}{\pgfqpoint{1.933229in}{0.162432in}} %
\pgfusepath{clip}%
\pgfsetbuttcap%
\pgfsetmiterjoin%
\definecolor{currentfill}{rgb}{0.333333,0.333333,0.333333}%
\pgfsetfillcolor{currentfill}%
\pgfsetlinewidth{0.501875pt}%
\definecolor{currentstroke}{rgb}{0.000000,0.000000,0.000000}%
\pgfsetstrokecolor{currentstroke}%
\pgfsetdash{}{0pt}%
\pgfpathmoveto{\pgfqpoint{1.718964in}{0.492924in}}%
\pgfpathlineto{\pgfqpoint{1.738296in}{0.492924in}}%
\pgfpathlineto{\pgfqpoint{1.738296in}{0.522171in}}%
\pgfpathlineto{\pgfqpoint{1.718964in}{0.522171in}}%
\pgfpathlineto{\pgfqpoint{1.718964in}{0.492924in}}%
\pgfusepath{stroke,fill}%
\end{pgfscope}%
\begin{pgfscope}%
\pgfpathrectangle{\pgfqpoint{0.636356in}{0.440955in}}{\pgfqpoint{1.933229in}{0.162432in}} %
\pgfusepath{clip}%
\pgfsetbuttcap%
\pgfsetmiterjoin%
\definecolor{currentfill}{rgb}{0.333333,0.333333,0.333333}%
\pgfsetfillcolor{currentfill}%
\pgfsetlinewidth{0.501875pt}%
\definecolor{currentstroke}{rgb}{0.000000,0.000000,0.000000}%
\pgfsetstrokecolor{currentstroke}%
\pgfsetdash{}{0pt}%
\pgfpathmoveto{\pgfqpoint{1.738296in}{0.522171in}}%
\pgfpathlineto{\pgfqpoint{1.757628in}{0.522171in}}%
\pgfpathlineto{\pgfqpoint{1.757628in}{0.549228in}}%
\pgfpathlineto{\pgfqpoint{1.738296in}{0.549228in}}%
\pgfpathlineto{\pgfqpoint{1.738296in}{0.522171in}}%
\pgfusepath{stroke,fill}%
\end{pgfscope}%
\begin{pgfscope}%
\pgfpathrectangle{\pgfqpoint{0.636356in}{0.440955in}}{\pgfqpoint{1.933229in}{0.162432in}} %
\pgfusepath{clip}%
\pgfsetbuttcap%
\pgfsetmiterjoin%
\definecolor{currentfill}{rgb}{0.333333,0.333333,0.333333}%
\pgfsetfillcolor{currentfill}%
\pgfsetlinewidth{0.501875pt}%
\definecolor{currentstroke}{rgb}{0.000000,0.000000,0.000000}%
\pgfsetstrokecolor{currentstroke}%
\pgfsetdash{}{0pt}%
\pgfpathmoveto{\pgfqpoint{1.757628in}{0.464522in}}%
\pgfpathlineto{\pgfqpoint{1.776961in}{0.464522in}}%
\pgfpathlineto{\pgfqpoint{1.776961in}{0.522171in}}%
\pgfpathlineto{\pgfqpoint{1.757628in}{0.522171in}}%
\pgfpathlineto{\pgfqpoint{1.757628in}{0.464522in}}%
\pgfusepath{stroke,fill}%
\end{pgfscope}%
\begin{pgfscope}%
\pgfpathrectangle{\pgfqpoint{0.636356in}{0.440955in}}{\pgfqpoint{1.933229in}{0.162432in}} %
\pgfusepath{clip}%
\pgfsetbuttcap%
\pgfsetmiterjoin%
\definecolor{currentfill}{rgb}{0.333333,0.333333,0.333333}%
\pgfsetfillcolor{currentfill}%
\pgfsetlinewidth{0.501875pt}%
\definecolor{currentstroke}{rgb}{0.000000,0.000000,0.000000}%
\pgfsetstrokecolor{currentstroke}%
\pgfsetdash{}{0pt}%
\pgfpathmoveto{\pgfqpoint{1.776961in}{0.522171in}}%
\pgfpathlineto{\pgfqpoint{1.796293in}{0.522171in}}%
\pgfpathlineto{\pgfqpoint{1.796293in}{0.525095in}}%
\pgfpathlineto{\pgfqpoint{1.776961in}{0.525095in}}%
\pgfpathlineto{\pgfqpoint{1.776961in}{0.522171in}}%
\pgfusepath{stroke,fill}%
\end{pgfscope}%
\begin{pgfscope}%
\pgfpathrectangle{\pgfqpoint{0.636356in}{0.440955in}}{\pgfqpoint{1.933229in}{0.162432in}} %
\pgfusepath{clip}%
\pgfsetbuttcap%
\pgfsetmiterjoin%
\definecolor{currentfill}{rgb}{0.333333,0.333333,0.333333}%
\pgfsetfillcolor{currentfill}%
\pgfsetlinewidth{0.501875pt}%
\definecolor{currentstroke}{rgb}{0.000000,0.000000,0.000000}%
\pgfsetstrokecolor{currentstroke}%
\pgfsetdash{}{0pt}%
\pgfpathmoveto{\pgfqpoint{1.796293in}{0.522171in}}%
\pgfpathlineto{\pgfqpoint{1.815625in}{0.522171in}}%
\pgfpathlineto{\pgfqpoint{1.815625in}{0.554213in}}%
\pgfpathlineto{\pgfqpoint{1.796293in}{0.554213in}}%
\pgfpathlineto{\pgfqpoint{1.796293in}{0.522171in}}%
\pgfusepath{stroke,fill}%
\end{pgfscope}%
\begin{pgfscope}%
\pgfpathrectangle{\pgfqpoint{0.636356in}{0.440955in}}{\pgfqpoint{1.933229in}{0.162432in}} %
\pgfusepath{clip}%
\pgfsetbuttcap%
\pgfsetmiterjoin%
\definecolor{currentfill}{rgb}{0.333333,0.333333,0.333333}%
\pgfsetfillcolor{currentfill}%
\pgfsetlinewidth{0.501875pt}%
\definecolor{currentstroke}{rgb}{0.000000,0.000000,0.000000}%
\pgfsetstrokecolor{currentstroke}%
\pgfsetdash{}{0pt}%
\pgfpathmoveto{\pgfqpoint{1.815625in}{0.522171in}}%
\pgfpathlineto{\pgfqpoint{1.834958in}{0.522171in}}%
\pgfpathlineto{\pgfqpoint{1.834958in}{0.528606in}}%
\pgfpathlineto{\pgfqpoint{1.815625in}{0.528606in}}%
\pgfpathlineto{\pgfqpoint{1.815625in}{0.522171in}}%
\pgfusepath{stroke,fill}%
\end{pgfscope}%
\begin{pgfscope}%
\pgfpathrectangle{\pgfqpoint{0.636356in}{0.440955in}}{\pgfqpoint{1.933229in}{0.162432in}} %
\pgfusepath{clip}%
\pgfsetbuttcap%
\pgfsetmiterjoin%
\definecolor{currentfill}{rgb}{0.333333,0.333333,0.333333}%
\pgfsetfillcolor{currentfill}%
\pgfsetlinewidth{0.501875pt}%
\definecolor{currentstroke}{rgb}{0.000000,0.000000,0.000000}%
\pgfsetstrokecolor{currentstroke}%
\pgfsetdash{}{0pt}%
\pgfpathmoveto{\pgfqpoint{1.834958in}{0.514713in}}%
\pgfpathlineto{\pgfqpoint{1.854290in}{0.514713in}}%
\pgfpathlineto{\pgfqpoint{1.854290in}{0.522171in}}%
\pgfpathlineto{\pgfqpoint{1.834958in}{0.522171in}}%
\pgfpathlineto{\pgfqpoint{1.834958in}{0.514713in}}%
\pgfusepath{stroke,fill}%
\end{pgfscope}%
\begin{pgfscope}%
\pgfpathrectangle{\pgfqpoint{0.636356in}{0.440955in}}{\pgfqpoint{1.933229in}{0.162432in}} %
\pgfusepath{clip}%
\pgfsetbuttcap%
\pgfsetmiterjoin%
\definecolor{currentfill}{rgb}{0.333333,0.333333,0.333333}%
\pgfsetfillcolor{currentfill}%
\pgfsetlinewidth{0.501875pt}%
\definecolor{currentstroke}{rgb}{0.000000,0.000000,0.000000}%
\pgfsetstrokecolor{currentstroke}%
\pgfsetdash{}{0pt}%
\pgfpathmoveto{\pgfqpoint{1.854290in}{0.522171in}}%
\pgfpathlineto{\pgfqpoint{1.873622in}{0.522171in}}%
\pgfpathlineto{\pgfqpoint{1.873622in}{0.540040in}}%
\pgfpathlineto{\pgfqpoint{1.854290in}{0.540040in}}%
\pgfpathlineto{\pgfqpoint{1.854290in}{0.522171in}}%
\pgfusepath{stroke,fill}%
\end{pgfscope}%
\begin{pgfscope}%
\pgfpathrectangle{\pgfqpoint{0.636356in}{0.440955in}}{\pgfqpoint{1.933229in}{0.162432in}} %
\pgfusepath{clip}%
\pgfsetbuttcap%
\pgfsetmiterjoin%
\definecolor{currentfill}{rgb}{0.333333,0.333333,0.333333}%
\pgfsetfillcolor{currentfill}%
\pgfsetlinewidth{0.501875pt}%
\definecolor{currentstroke}{rgb}{0.000000,0.000000,0.000000}%
\pgfsetstrokecolor{currentstroke}%
\pgfsetdash{}{0pt}%
\pgfpathmoveto{\pgfqpoint{1.873622in}{0.522171in}}%
\pgfpathlineto{\pgfqpoint{1.892954in}{0.522171in}}%
\pgfpathlineto{\pgfqpoint{1.892954in}{0.550052in}}%
\pgfpathlineto{\pgfqpoint{1.873622in}{0.550052in}}%
\pgfpathlineto{\pgfqpoint{1.873622in}{0.522171in}}%
\pgfusepath{stroke,fill}%
\end{pgfscope}%
\begin{pgfscope}%
\pgfpathrectangle{\pgfqpoint{0.636356in}{0.440955in}}{\pgfqpoint{1.933229in}{0.162432in}} %
\pgfusepath{clip}%
\pgfsetbuttcap%
\pgfsetmiterjoin%
\definecolor{currentfill}{rgb}{0.333333,0.333333,0.333333}%
\pgfsetfillcolor{currentfill}%
\pgfsetlinewidth{0.501875pt}%
\definecolor{currentstroke}{rgb}{0.000000,0.000000,0.000000}%
\pgfsetstrokecolor{currentstroke}%
\pgfsetdash{}{0pt}%
\pgfpathmoveto{\pgfqpoint{1.892954in}{0.505327in}}%
\pgfpathlineto{\pgfqpoint{1.912287in}{0.505327in}}%
\pgfpathlineto{\pgfqpoint{1.912287in}{0.522171in}}%
\pgfpathlineto{\pgfqpoint{1.892954in}{0.522171in}}%
\pgfpathlineto{\pgfqpoint{1.892954in}{0.505327in}}%
\pgfusepath{stroke,fill}%
\end{pgfscope}%
\begin{pgfscope}%
\pgfpathrectangle{\pgfqpoint{0.636356in}{0.440955in}}{\pgfqpoint{1.933229in}{0.162432in}} %
\pgfusepath{clip}%
\pgfsetbuttcap%
\pgfsetmiterjoin%
\definecolor{currentfill}{rgb}{0.333333,0.333333,0.333333}%
\pgfsetfillcolor{currentfill}%
\pgfsetlinewidth{0.501875pt}%
\definecolor{currentstroke}{rgb}{0.000000,0.000000,0.000000}%
\pgfsetstrokecolor{currentstroke}%
\pgfsetdash{}{0pt}%
\pgfpathmoveto{\pgfqpoint{1.912287in}{0.522171in}}%
\pgfpathlineto{\pgfqpoint{1.931619in}{0.522171in}}%
\pgfpathlineto{\pgfqpoint{1.931619in}{0.529883in}}%
\pgfpathlineto{\pgfqpoint{1.912287in}{0.529883in}}%
\pgfpathlineto{\pgfqpoint{1.912287in}{0.522171in}}%
\pgfusepath{stroke,fill}%
\end{pgfscope}%
\begin{pgfscope}%
\pgfpathrectangle{\pgfqpoint{0.636356in}{0.440955in}}{\pgfqpoint{1.933229in}{0.162432in}} %
\pgfusepath{clip}%
\pgfsetbuttcap%
\pgfsetmiterjoin%
\definecolor{currentfill}{rgb}{0.333333,0.333333,0.333333}%
\pgfsetfillcolor{currentfill}%
\pgfsetlinewidth{0.501875pt}%
\definecolor{currentstroke}{rgb}{0.000000,0.000000,0.000000}%
\pgfsetstrokecolor{currentstroke}%
\pgfsetdash{}{0pt}%
\pgfpathmoveto{\pgfqpoint{1.931619in}{0.522171in}}%
\pgfpathlineto{\pgfqpoint{1.950951in}{0.522171in}}%
\pgfpathlineto{\pgfqpoint{1.950951in}{0.525356in}}%
\pgfpathlineto{\pgfqpoint{1.931619in}{0.525356in}}%
\pgfpathlineto{\pgfqpoint{1.931619in}{0.522171in}}%
\pgfusepath{stroke,fill}%
\end{pgfscope}%
\begin{pgfscope}%
\pgfpathrectangle{\pgfqpoint{0.636356in}{0.440955in}}{\pgfqpoint{1.933229in}{0.162432in}} %
\pgfusepath{clip}%
\pgfsetbuttcap%
\pgfsetmiterjoin%
\definecolor{currentfill}{rgb}{0.333333,0.333333,0.333333}%
\pgfsetfillcolor{currentfill}%
\pgfsetlinewidth{0.501875pt}%
\definecolor{currentstroke}{rgb}{0.000000,0.000000,0.000000}%
\pgfsetstrokecolor{currentstroke}%
\pgfsetdash{}{0pt}%
\pgfpathmoveto{\pgfqpoint{1.950951in}{0.518440in}}%
\pgfpathlineto{\pgfqpoint{1.970284in}{0.518440in}}%
\pgfpathlineto{\pgfqpoint{1.970284in}{0.522171in}}%
\pgfpathlineto{\pgfqpoint{1.950951in}{0.522171in}}%
\pgfpathlineto{\pgfqpoint{1.950951in}{0.518440in}}%
\pgfusepath{stroke,fill}%
\end{pgfscope}%
\begin{pgfscope}%
\pgfpathrectangle{\pgfqpoint{0.636356in}{0.440955in}}{\pgfqpoint{1.933229in}{0.162432in}} %
\pgfusepath{clip}%
\pgfsetbuttcap%
\pgfsetmiterjoin%
\definecolor{currentfill}{rgb}{0.333333,0.333333,0.333333}%
\pgfsetfillcolor{currentfill}%
\pgfsetlinewidth{0.501875pt}%
\definecolor{currentstroke}{rgb}{0.000000,0.000000,0.000000}%
\pgfsetstrokecolor{currentstroke}%
\pgfsetdash{}{0pt}%
\pgfpathmoveto{\pgfqpoint{1.970284in}{0.503008in}}%
\pgfpathlineto{\pgfqpoint{1.989616in}{0.503008in}}%
\pgfpathlineto{\pgfqpoint{1.989616in}{0.522171in}}%
\pgfpathlineto{\pgfqpoint{1.970284in}{0.522171in}}%
\pgfpathlineto{\pgfqpoint{1.970284in}{0.503008in}}%
\pgfusepath{stroke,fill}%
\end{pgfscope}%
\begin{pgfscope}%
\pgfpathrectangle{\pgfqpoint{0.636356in}{0.440955in}}{\pgfqpoint{1.933229in}{0.162432in}} %
\pgfusepath{clip}%
\pgfsetbuttcap%
\pgfsetmiterjoin%
\definecolor{currentfill}{rgb}{0.333333,0.333333,0.333333}%
\pgfsetfillcolor{currentfill}%
\pgfsetlinewidth{0.501875pt}%
\definecolor{currentstroke}{rgb}{0.000000,0.000000,0.000000}%
\pgfsetstrokecolor{currentstroke}%
\pgfsetdash{}{0pt}%
\pgfpathmoveto{\pgfqpoint{1.989616in}{0.521447in}}%
\pgfpathlineto{\pgfqpoint{2.008948in}{0.521447in}}%
\pgfpathlineto{\pgfqpoint{2.008948in}{0.522171in}}%
\pgfpathlineto{\pgfqpoint{1.989616in}{0.522171in}}%
\pgfpathlineto{\pgfqpoint{1.989616in}{0.521447in}}%
\pgfusepath{stroke,fill}%
\end{pgfscope}%
\begin{pgfscope}%
\pgfpathrectangle{\pgfqpoint{0.636356in}{0.440955in}}{\pgfqpoint{1.933229in}{0.162432in}} %
\pgfusepath{clip}%
\pgfsetbuttcap%
\pgfsetmiterjoin%
\definecolor{currentfill}{rgb}{0.333333,0.333333,0.333333}%
\pgfsetfillcolor{currentfill}%
\pgfsetlinewidth{0.501875pt}%
\definecolor{currentstroke}{rgb}{0.000000,0.000000,0.000000}%
\pgfsetstrokecolor{currentstroke}%
\pgfsetdash{}{0pt}%
\pgfpathmoveto{\pgfqpoint{2.008948in}{0.522171in}}%
\pgfpathlineto{\pgfqpoint{2.028280in}{0.522171in}}%
\pgfpathlineto{\pgfqpoint{2.028280in}{0.552645in}}%
\pgfpathlineto{\pgfqpoint{2.008948in}{0.552645in}}%
\pgfpathlineto{\pgfqpoint{2.008948in}{0.522171in}}%
\pgfusepath{stroke,fill}%
\end{pgfscope}%
\begin{pgfscope}%
\pgfpathrectangle{\pgfqpoint{0.636356in}{0.440955in}}{\pgfqpoint{1.933229in}{0.162432in}} %
\pgfusepath{clip}%
\pgfsetbuttcap%
\pgfsetmiterjoin%
\definecolor{currentfill}{rgb}{0.333333,0.333333,0.333333}%
\pgfsetfillcolor{currentfill}%
\pgfsetlinewidth{0.501875pt}%
\definecolor{currentstroke}{rgb}{0.000000,0.000000,0.000000}%
\pgfsetstrokecolor{currentstroke}%
\pgfsetdash{}{0pt}%
\pgfpathmoveto{\pgfqpoint{2.028280in}{0.522171in}}%
\pgfpathlineto{\pgfqpoint{2.047613in}{0.522171in}}%
\pgfpathlineto{\pgfqpoint{2.047613in}{0.524065in}}%
\pgfpathlineto{\pgfqpoint{2.028280in}{0.524065in}}%
\pgfpathlineto{\pgfqpoint{2.028280in}{0.522171in}}%
\pgfusepath{stroke,fill}%
\end{pgfscope}%
\begin{pgfscope}%
\pgfpathrectangle{\pgfqpoint{0.636356in}{0.440955in}}{\pgfqpoint{1.933229in}{0.162432in}} %
\pgfusepath{clip}%
\pgfsetbuttcap%
\pgfsetmiterjoin%
\definecolor{currentfill}{rgb}{0.333333,0.333333,0.333333}%
\pgfsetfillcolor{currentfill}%
\pgfsetlinewidth{0.501875pt}%
\definecolor{currentstroke}{rgb}{0.000000,0.000000,0.000000}%
\pgfsetstrokecolor{currentstroke}%
\pgfsetdash{}{0pt}%
\pgfpathmoveto{\pgfqpoint{2.047613in}{0.522171in}}%
\pgfpathlineto{\pgfqpoint{2.066945in}{0.522171in}}%
\pgfpathlineto{\pgfqpoint{2.066945in}{0.534570in}}%
\pgfpathlineto{\pgfqpoint{2.047613in}{0.534570in}}%
\pgfpathlineto{\pgfqpoint{2.047613in}{0.522171in}}%
\pgfusepath{stroke,fill}%
\end{pgfscope}%
\begin{pgfscope}%
\pgfpathrectangle{\pgfqpoint{0.636356in}{0.440955in}}{\pgfqpoint{1.933229in}{0.162432in}} %
\pgfusepath{clip}%
\pgfsetbuttcap%
\pgfsetmiterjoin%
\definecolor{currentfill}{rgb}{0.333333,0.333333,0.333333}%
\pgfsetfillcolor{currentfill}%
\pgfsetlinewidth{0.501875pt}%
\definecolor{currentstroke}{rgb}{0.000000,0.000000,0.000000}%
\pgfsetstrokecolor{currentstroke}%
\pgfsetdash{}{0pt}%
\pgfpathmoveto{\pgfqpoint{2.066945in}{0.522171in}}%
\pgfpathlineto{\pgfqpoint{2.086277in}{0.522171in}}%
\pgfpathlineto{\pgfqpoint{2.086277in}{0.523015in}}%
\pgfpathlineto{\pgfqpoint{2.066945in}{0.523015in}}%
\pgfpathlineto{\pgfqpoint{2.066945in}{0.522171in}}%
\pgfusepath{stroke,fill}%
\end{pgfscope}%
\begin{pgfscope}%
\pgfpathrectangle{\pgfqpoint{0.636356in}{0.440955in}}{\pgfqpoint{1.933229in}{0.162432in}} %
\pgfusepath{clip}%
\pgfsetbuttcap%
\pgfsetmiterjoin%
\definecolor{currentfill}{rgb}{0.333333,0.333333,0.333333}%
\pgfsetfillcolor{currentfill}%
\pgfsetlinewidth{0.501875pt}%
\definecolor{currentstroke}{rgb}{0.000000,0.000000,0.000000}%
\pgfsetstrokecolor{currentstroke}%
\pgfsetdash{}{0pt}%
\pgfpathmoveto{\pgfqpoint{2.086277in}{0.522171in}}%
\pgfpathlineto{\pgfqpoint{2.105610in}{0.522171in}}%
\pgfpathlineto{\pgfqpoint{2.105610in}{0.584997in}}%
\pgfpathlineto{\pgfqpoint{2.086277in}{0.584997in}}%
\pgfpathlineto{\pgfqpoint{2.086277in}{0.522171in}}%
\pgfusepath{stroke,fill}%
\end{pgfscope}%
\begin{pgfscope}%
\pgfpathrectangle{\pgfqpoint{0.636356in}{0.440955in}}{\pgfqpoint{1.933229in}{0.162432in}} %
\pgfusepath{clip}%
\pgfsetbuttcap%
\pgfsetmiterjoin%
\definecolor{currentfill}{rgb}{0.333333,0.333333,0.333333}%
\pgfsetfillcolor{currentfill}%
\pgfsetlinewidth{0.501875pt}%
\definecolor{currentstroke}{rgb}{0.000000,0.000000,0.000000}%
\pgfsetstrokecolor{currentstroke}%
\pgfsetdash{}{0pt}%
\pgfpathmoveto{\pgfqpoint{2.105610in}{0.500508in}}%
\pgfpathlineto{\pgfqpoint{2.124942in}{0.500508in}}%
\pgfpathlineto{\pgfqpoint{2.124942in}{0.522171in}}%
\pgfpathlineto{\pgfqpoint{2.105610in}{0.522171in}}%
\pgfpathlineto{\pgfqpoint{2.105610in}{0.500508in}}%
\pgfusepath{stroke,fill}%
\end{pgfscope}%
\begin{pgfscope}%
\pgfpathrectangle{\pgfqpoint{0.636356in}{0.440955in}}{\pgfqpoint{1.933229in}{0.162432in}} %
\pgfusepath{clip}%
\pgfsetbuttcap%
\pgfsetmiterjoin%
\definecolor{currentfill}{rgb}{0.333333,0.333333,0.333333}%
\pgfsetfillcolor{currentfill}%
\pgfsetlinewidth{0.501875pt}%
\definecolor{currentstroke}{rgb}{0.000000,0.000000,0.000000}%
\pgfsetstrokecolor{currentstroke}%
\pgfsetdash{}{0pt}%
\pgfpathmoveto{\pgfqpoint{2.124942in}{0.522171in}}%
\pgfpathlineto{\pgfqpoint{2.144274in}{0.522171in}}%
\pgfpathlineto{\pgfqpoint{2.144274in}{0.529836in}}%
\pgfpathlineto{\pgfqpoint{2.124942in}{0.529836in}}%
\pgfpathlineto{\pgfqpoint{2.124942in}{0.522171in}}%
\pgfusepath{stroke,fill}%
\end{pgfscope}%
\begin{pgfscope}%
\pgfpathrectangle{\pgfqpoint{0.636356in}{0.440955in}}{\pgfqpoint{1.933229in}{0.162432in}} %
\pgfusepath{clip}%
\pgfsetbuttcap%
\pgfsetmiterjoin%
\definecolor{currentfill}{rgb}{0.333333,0.333333,0.333333}%
\pgfsetfillcolor{currentfill}%
\pgfsetlinewidth{0.501875pt}%
\definecolor{currentstroke}{rgb}{0.000000,0.000000,0.000000}%
\pgfsetstrokecolor{currentstroke}%
\pgfsetdash{}{0pt}%
\pgfpathmoveto{\pgfqpoint{2.144274in}{0.522171in}}%
\pgfpathlineto{\pgfqpoint{2.163606in}{0.522171in}}%
\pgfpathlineto{\pgfqpoint{2.163606in}{0.535662in}}%
\pgfpathlineto{\pgfqpoint{2.144274in}{0.535662in}}%
\pgfpathlineto{\pgfqpoint{2.144274in}{0.522171in}}%
\pgfusepath{stroke,fill}%
\end{pgfscope}%
\begin{pgfscope}%
\pgfpathrectangle{\pgfqpoint{0.636356in}{0.440955in}}{\pgfqpoint{1.933229in}{0.162432in}} %
\pgfusepath{clip}%
\pgfsetbuttcap%
\pgfsetmiterjoin%
\definecolor{currentfill}{rgb}{0.333333,0.333333,0.333333}%
\pgfsetfillcolor{currentfill}%
\pgfsetlinewidth{0.501875pt}%
\definecolor{currentstroke}{rgb}{0.000000,0.000000,0.000000}%
\pgfsetstrokecolor{currentstroke}%
\pgfsetdash{}{0pt}%
\pgfpathmoveto{\pgfqpoint{2.163606in}{0.489666in}}%
\pgfpathlineto{\pgfqpoint{2.182939in}{0.489666in}}%
\pgfpathlineto{\pgfqpoint{2.182939in}{0.522171in}}%
\pgfpathlineto{\pgfqpoint{2.163606in}{0.522171in}}%
\pgfpathlineto{\pgfqpoint{2.163606in}{0.489666in}}%
\pgfusepath{stroke,fill}%
\end{pgfscope}%
\begin{pgfscope}%
\pgfpathrectangle{\pgfqpoint{0.636356in}{0.440955in}}{\pgfqpoint{1.933229in}{0.162432in}} %
\pgfusepath{clip}%
\pgfsetbuttcap%
\pgfsetmiterjoin%
\definecolor{currentfill}{rgb}{0.333333,0.333333,0.333333}%
\pgfsetfillcolor{currentfill}%
\pgfsetlinewidth{0.501875pt}%
\definecolor{currentstroke}{rgb}{0.000000,0.000000,0.000000}%
\pgfsetstrokecolor{currentstroke}%
\pgfsetdash{}{0pt}%
\pgfpathmoveto{\pgfqpoint{2.182939in}{0.522171in}}%
\pgfpathlineto{\pgfqpoint{2.202271in}{0.522171in}}%
\pgfpathlineto{\pgfqpoint{2.202271in}{0.536488in}}%
\pgfpathlineto{\pgfqpoint{2.182939in}{0.536488in}}%
\pgfpathlineto{\pgfqpoint{2.182939in}{0.522171in}}%
\pgfusepath{stroke,fill}%
\end{pgfscope}%
\begin{pgfscope}%
\pgfpathrectangle{\pgfqpoint{0.636356in}{0.440955in}}{\pgfqpoint{1.933229in}{0.162432in}} %
\pgfusepath{clip}%
\pgfsetbuttcap%
\pgfsetmiterjoin%
\definecolor{currentfill}{rgb}{0.333333,0.333333,0.333333}%
\pgfsetfillcolor{currentfill}%
\pgfsetlinewidth{0.501875pt}%
\definecolor{currentstroke}{rgb}{0.000000,0.000000,0.000000}%
\pgfsetstrokecolor{currentstroke}%
\pgfsetdash{}{0pt}%
\pgfpathmoveto{\pgfqpoint{2.202271in}{0.506233in}}%
\pgfpathlineto{\pgfqpoint{2.221603in}{0.506233in}}%
\pgfpathlineto{\pgfqpoint{2.221603in}{0.522171in}}%
\pgfpathlineto{\pgfqpoint{2.202271in}{0.522171in}}%
\pgfpathlineto{\pgfqpoint{2.202271in}{0.506233in}}%
\pgfusepath{stroke,fill}%
\end{pgfscope}%
\begin{pgfscope}%
\pgfpathrectangle{\pgfqpoint{0.636356in}{0.440955in}}{\pgfqpoint{1.933229in}{0.162432in}} %
\pgfusepath{clip}%
\pgfsetbuttcap%
\pgfsetmiterjoin%
\definecolor{currentfill}{rgb}{0.333333,0.333333,0.333333}%
\pgfsetfillcolor{currentfill}%
\pgfsetlinewidth{0.501875pt}%
\definecolor{currentstroke}{rgb}{0.000000,0.000000,0.000000}%
\pgfsetstrokecolor{currentstroke}%
\pgfsetdash{}{0pt}%
\pgfpathmoveto{\pgfqpoint{2.221603in}{0.521327in}}%
\pgfpathlineto{\pgfqpoint{2.240936in}{0.521327in}}%
\pgfpathlineto{\pgfqpoint{2.240936in}{0.522171in}}%
\pgfpathlineto{\pgfqpoint{2.221603in}{0.522171in}}%
\pgfpathlineto{\pgfqpoint{2.221603in}{0.521327in}}%
\pgfusepath{stroke,fill}%
\end{pgfscope}%
\begin{pgfscope}%
\pgfpathrectangle{\pgfqpoint{0.636356in}{0.440955in}}{\pgfqpoint{1.933229in}{0.162432in}} %
\pgfusepath{clip}%
\pgfsetbuttcap%
\pgfsetmiterjoin%
\definecolor{currentfill}{rgb}{0.333333,0.333333,0.333333}%
\pgfsetfillcolor{currentfill}%
\pgfsetlinewidth{0.501875pt}%
\definecolor{currentstroke}{rgb}{0.000000,0.000000,0.000000}%
\pgfsetstrokecolor{currentstroke}%
\pgfsetdash{}{0pt}%
\pgfpathmoveto{\pgfqpoint{2.240936in}{0.478236in}}%
\pgfpathlineto{\pgfqpoint{2.260268in}{0.478236in}}%
\pgfpathlineto{\pgfqpoint{2.260268in}{0.522171in}}%
\pgfpathlineto{\pgfqpoint{2.240936in}{0.522171in}}%
\pgfpathlineto{\pgfqpoint{2.240936in}{0.478236in}}%
\pgfusepath{stroke,fill}%
\end{pgfscope}%
\begin{pgfscope}%
\pgfpathrectangle{\pgfqpoint{0.636356in}{0.440955in}}{\pgfqpoint{1.933229in}{0.162432in}} %
\pgfusepath{clip}%
\pgfsetbuttcap%
\pgfsetmiterjoin%
\definecolor{currentfill}{rgb}{0.333333,0.333333,0.333333}%
\pgfsetfillcolor{currentfill}%
\pgfsetlinewidth{0.501875pt}%
\definecolor{currentstroke}{rgb}{0.000000,0.000000,0.000000}%
\pgfsetstrokecolor{currentstroke}%
\pgfsetdash{}{0pt}%
\pgfpathmoveto{\pgfqpoint{2.260268in}{0.489456in}}%
\pgfpathlineto{\pgfqpoint{2.279600in}{0.489456in}}%
\pgfpathlineto{\pgfqpoint{2.279600in}{0.522171in}}%
\pgfpathlineto{\pgfqpoint{2.260268in}{0.522171in}}%
\pgfpathlineto{\pgfqpoint{2.260268in}{0.489456in}}%
\pgfusepath{stroke,fill}%
\end{pgfscope}%
\begin{pgfscope}%
\pgfpathrectangle{\pgfqpoint{0.636356in}{0.440955in}}{\pgfqpoint{1.933229in}{0.162432in}} %
\pgfusepath{clip}%
\pgfsetbuttcap%
\pgfsetmiterjoin%
\definecolor{currentfill}{rgb}{0.333333,0.333333,0.333333}%
\pgfsetfillcolor{currentfill}%
\pgfsetlinewidth{0.501875pt}%
\definecolor{currentstroke}{rgb}{0.000000,0.000000,0.000000}%
\pgfsetstrokecolor{currentstroke}%
\pgfsetdash{}{0pt}%
\pgfpathmoveto{\pgfqpoint{2.279600in}{0.522171in}}%
\pgfpathlineto{\pgfqpoint{2.298932in}{0.522171in}}%
\pgfpathlineto{\pgfqpoint{2.298932in}{0.540182in}}%
\pgfpathlineto{\pgfqpoint{2.279600in}{0.540182in}}%
\pgfpathlineto{\pgfqpoint{2.279600in}{0.522171in}}%
\pgfusepath{stroke,fill}%
\end{pgfscope}%
\begin{pgfscope}%
\pgfpathrectangle{\pgfqpoint{0.636356in}{0.440955in}}{\pgfqpoint{1.933229in}{0.162432in}} %
\pgfusepath{clip}%
\pgfsetbuttcap%
\pgfsetmiterjoin%
\definecolor{currentfill}{rgb}{0.333333,0.333333,0.333333}%
\pgfsetfillcolor{currentfill}%
\pgfsetlinewidth{0.501875pt}%
\definecolor{currentstroke}{rgb}{0.000000,0.000000,0.000000}%
\pgfsetstrokecolor{currentstroke}%
\pgfsetdash{}{0pt}%
\pgfpathmoveto{\pgfqpoint{2.298932in}{0.522171in}}%
\pgfpathlineto{\pgfqpoint{2.318265in}{0.522171in}}%
\pgfpathlineto{\pgfqpoint{2.318265in}{0.533754in}}%
\pgfpathlineto{\pgfqpoint{2.298932in}{0.533754in}}%
\pgfpathlineto{\pgfqpoint{2.298932in}{0.522171in}}%
\pgfusepath{stroke,fill}%
\end{pgfscope}%
\begin{pgfscope}%
\pgfpathrectangle{\pgfqpoint{0.636356in}{0.440955in}}{\pgfqpoint{1.933229in}{0.162432in}} %
\pgfusepath{clip}%
\pgfsetbuttcap%
\pgfsetmiterjoin%
\definecolor{currentfill}{rgb}{0.333333,0.333333,0.333333}%
\pgfsetfillcolor{currentfill}%
\pgfsetlinewidth{0.501875pt}%
\definecolor{currentstroke}{rgb}{0.000000,0.000000,0.000000}%
\pgfsetstrokecolor{currentstroke}%
\pgfsetdash{}{0pt}%
\pgfpathmoveto{\pgfqpoint{2.318265in}{0.491593in}}%
\pgfpathlineto{\pgfqpoint{2.337597in}{0.491593in}}%
\pgfpathlineto{\pgfqpoint{2.337597in}{0.522171in}}%
\pgfpathlineto{\pgfqpoint{2.318265in}{0.522171in}}%
\pgfpathlineto{\pgfqpoint{2.318265in}{0.491593in}}%
\pgfusepath{stroke,fill}%
\end{pgfscope}%
\begin{pgfscope}%
\pgfpathrectangle{\pgfqpoint{0.636356in}{0.440955in}}{\pgfqpoint{1.933229in}{0.162432in}} %
\pgfusepath{clip}%
\pgfsetbuttcap%
\pgfsetmiterjoin%
\definecolor{currentfill}{rgb}{0.333333,0.333333,0.333333}%
\pgfsetfillcolor{currentfill}%
\pgfsetlinewidth{0.501875pt}%
\definecolor{currentstroke}{rgb}{0.000000,0.000000,0.000000}%
\pgfsetstrokecolor{currentstroke}%
\pgfsetdash{}{0pt}%
\pgfpathmoveto{\pgfqpoint{2.337597in}{0.489934in}}%
\pgfpathlineto{\pgfqpoint{2.356929in}{0.489934in}}%
\pgfpathlineto{\pgfqpoint{2.356929in}{0.522171in}}%
\pgfpathlineto{\pgfqpoint{2.337597in}{0.522171in}}%
\pgfpathlineto{\pgfqpoint{2.337597in}{0.489934in}}%
\pgfusepath{stroke,fill}%
\end{pgfscope}%
\begin{pgfscope}%
\pgfpathrectangle{\pgfqpoint{0.636356in}{0.440955in}}{\pgfqpoint{1.933229in}{0.162432in}} %
\pgfusepath{clip}%
\pgfsetbuttcap%
\pgfsetmiterjoin%
\definecolor{currentfill}{rgb}{0.333333,0.333333,0.333333}%
\pgfsetfillcolor{currentfill}%
\pgfsetlinewidth{0.501875pt}%
\definecolor{currentstroke}{rgb}{0.000000,0.000000,0.000000}%
\pgfsetstrokecolor{currentstroke}%
\pgfsetdash{}{0pt}%
\pgfpathmoveto{\pgfqpoint{2.356929in}{0.482724in}}%
\pgfpathlineto{\pgfqpoint{2.376262in}{0.482724in}}%
\pgfpathlineto{\pgfqpoint{2.376262in}{0.522171in}}%
\pgfpathlineto{\pgfqpoint{2.356929in}{0.522171in}}%
\pgfpathlineto{\pgfqpoint{2.356929in}{0.482724in}}%
\pgfusepath{stroke,fill}%
\end{pgfscope}%
\begin{pgfscope}%
\pgfpathrectangle{\pgfqpoint{0.636356in}{0.440955in}}{\pgfqpoint{1.933229in}{0.162432in}} %
\pgfusepath{clip}%
\pgfsetbuttcap%
\pgfsetmiterjoin%
\definecolor{currentfill}{rgb}{0.333333,0.333333,0.333333}%
\pgfsetfillcolor{currentfill}%
\pgfsetlinewidth{0.501875pt}%
\definecolor{currentstroke}{rgb}{0.000000,0.000000,0.000000}%
\pgfsetstrokecolor{currentstroke}%
\pgfsetdash{}{0pt}%
\pgfpathmoveto{\pgfqpoint{2.376262in}{0.499629in}}%
\pgfpathlineto{\pgfqpoint{2.395594in}{0.499629in}}%
\pgfpathlineto{\pgfqpoint{2.395594in}{0.522171in}}%
\pgfpathlineto{\pgfqpoint{2.376262in}{0.522171in}}%
\pgfpathlineto{\pgfqpoint{2.376262in}{0.499629in}}%
\pgfusepath{stroke,fill}%
\end{pgfscope}%
\begin{pgfscope}%
\pgfpathrectangle{\pgfqpoint{0.636356in}{0.440955in}}{\pgfqpoint{1.933229in}{0.162432in}} %
\pgfusepath{clip}%
\pgfsetbuttcap%
\pgfsetmiterjoin%
\definecolor{currentfill}{rgb}{0.333333,0.333333,0.333333}%
\pgfsetfillcolor{currentfill}%
\pgfsetlinewidth{0.501875pt}%
\definecolor{currentstroke}{rgb}{0.000000,0.000000,0.000000}%
\pgfsetstrokecolor{currentstroke}%
\pgfsetdash{}{0pt}%
\pgfpathmoveto{\pgfqpoint{2.395594in}{0.453683in}}%
\pgfpathlineto{\pgfqpoint{2.414926in}{0.453683in}}%
\pgfpathlineto{\pgfqpoint{2.414926in}{0.522171in}}%
\pgfpathlineto{\pgfqpoint{2.395594in}{0.522171in}}%
\pgfpathlineto{\pgfqpoint{2.395594in}{0.453683in}}%
\pgfusepath{stroke,fill}%
\end{pgfscope}%
\begin{pgfscope}%
\pgfpathrectangle{\pgfqpoint{0.636356in}{0.440955in}}{\pgfqpoint{1.933229in}{0.162432in}} %
\pgfusepath{clip}%
\pgfsetbuttcap%
\pgfsetmiterjoin%
\definecolor{currentfill}{rgb}{0.333333,0.333333,0.333333}%
\pgfsetfillcolor{currentfill}%
\pgfsetlinewidth{0.501875pt}%
\definecolor{currentstroke}{rgb}{0.000000,0.000000,0.000000}%
\pgfsetstrokecolor{currentstroke}%
\pgfsetdash{}{0pt}%
\pgfpathmoveto{\pgfqpoint{2.414926in}{0.502397in}}%
\pgfpathlineto{\pgfqpoint{2.434258in}{0.502397in}}%
\pgfpathlineto{\pgfqpoint{2.434258in}{0.522171in}}%
\pgfpathlineto{\pgfqpoint{2.414926in}{0.522171in}}%
\pgfpathlineto{\pgfqpoint{2.414926in}{0.502397in}}%
\pgfusepath{stroke,fill}%
\end{pgfscope}%
\begin{pgfscope}%
\pgfpathrectangle{\pgfqpoint{0.636356in}{0.440955in}}{\pgfqpoint{1.933229in}{0.162432in}} %
\pgfusepath{clip}%
\pgfsetbuttcap%
\pgfsetmiterjoin%
\definecolor{currentfill}{rgb}{0.333333,0.333333,0.333333}%
\pgfsetfillcolor{currentfill}%
\pgfsetlinewidth{0.501875pt}%
\definecolor{currentstroke}{rgb}{0.000000,0.000000,0.000000}%
\pgfsetstrokecolor{currentstroke}%
\pgfsetdash{}{0pt}%
\pgfpathmoveto{\pgfqpoint{2.434258in}{0.512376in}}%
\pgfpathlineto{\pgfqpoint{2.453591in}{0.512376in}}%
\pgfpathlineto{\pgfqpoint{2.453591in}{0.522171in}}%
\pgfpathlineto{\pgfqpoint{2.434258in}{0.522171in}}%
\pgfpathlineto{\pgfqpoint{2.434258in}{0.512376in}}%
\pgfusepath{stroke,fill}%
\end{pgfscope}%
\begin{pgfscope}%
\pgfpathrectangle{\pgfqpoint{0.636356in}{0.440955in}}{\pgfqpoint{1.933229in}{0.162432in}} %
\pgfusepath{clip}%
\pgfsetbuttcap%
\pgfsetmiterjoin%
\definecolor{currentfill}{rgb}{0.333333,0.333333,0.333333}%
\pgfsetfillcolor{currentfill}%
\pgfsetlinewidth{0.501875pt}%
\definecolor{currentstroke}{rgb}{0.000000,0.000000,0.000000}%
\pgfsetstrokecolor{currentstroke}%
\pgfsetdash{}{0pt}%
\pgfpathmoveto{\pgfqpoint{2.453591in}{0.494426in}}%
\pgfpathlineto{\pgfqpoint{2.472923in}{0.494426in}}%
\pgfpathlineto{\pgfqpoint{2.472923in}{0.522171in}}%
\pgfpathlineto{\pgfqpoint{2.453591in}{0.522171in}}%
\pgfpathlineto{\pgfqpoint{2.453591in}{0.494426in}}%
\pgfusepath{stroke,fill}%
\end{pgfscope}%
\begin{pgfscope}%
\pgfpathrectangle{\pgfqpoint{0.636356in}{0.440955in}}{\pgfqpoint{1.933229in}{0.162432in}} %
\pgfusepath{clip}%
\pgfsetbuttcap%
\pgfsetmiterjoin%
\definecolor{currentfill}{rgb}{0.333333,0.333333,0.333333}%
\pgfsetfillcolor{currentfill}%
\pgfsetlinewidth{0.501875pt}%
\definecolor{currentstroke}{rgb}{0.000000,0.000000,0.000000}%
\pgfsetstrokecolor{currentstroke}%
\pgfsetdash{}{0pt}%
\pgfpathmoveto{\pgfqpoint{2.472923in}{0.512462in}}%
\pgfpathlineto{\pgfqpoint{2.492255in}{0.512462in}}%
\pgfpathlineto{\pgfqpoint{2.492255in}{0.522171in}}%
\pgfpathlineto{\pgfqpoint{2.472923in}{0.522171in}}%
\pgfpathlineto{\pgfqpoint{2.472923in}{0.512462in}}%
\pgfusepath{stroke,fill}%
\end{pgfscope}%
\begin{pgfscope}%
\pgfpathrectangle{\pgfqpoint{0.636356in}{0.440955in}}{\pgfqpoint{1.933229in}{0.162432in}} %
\pgfusepath{clip}%
\pgfsetbuttcap%
\pgfsetmiterjoin%
\definecolor{currentfill}{rgb}{0.333333,0.333333,0.333333}%
\pgfsetfillcolor{currentfill}%
\pgfsetlinewidth{0.501875pt}%
\definecolor{currentstroke}{rgb}{0.000000,0.000000,0.000000}%
\pgfsetstrokecolor{currentstroke}%
\pgfsetdash{}{0pt}%
\pgfpathmoveto{\pgfqpoint{2.492255in}{0.518952in}}%
\pgfpathlineto{\pgfqpoint{2.511588in}{0.518952in}}%
\pgfpathlineto{\pgfqpoint{2.511588in}{0.522171in}}%
\pgfpathlineto{\pgfqpoint{2.492255in}{0.522171in}}%
\pgfpathlineto{\pgfqpoint{2.492255in}{0.518952in}}%
\pgfusepath{stroke,fill}%
\end{pgfscope}%
\begin{pgfscope}%
\pgfpathrectangle{\pgfqpoint{0.636356in}{0.440955in}}{\pgfqpoint{1.933229in}{0.162432in}} %
\pgfusepath{clip}%
\pgfsetbuttcap%
\pgfsetmiterjoin%
\definecolor{currentfill}{rgb}{0.333333,0.333333,0.333333}%
\pgfsetfillcolor{currentfill}%
\pgfsetlinewidth{0.501875pt}%
\definecolor{currentstroke}{rgb}{0.000000,0.000000,0.000000}%
\pgfsetstrokecolor{currentstroke}%
\pgfsetdash{}{0pt}%
\pgfpathmoveto{\pgfqpoint{2.511588in}{0.505957in}}%
\pgfpathlineto{\pgfqpoint{2.530920in}{0.505957in}}%
\pgfpathlineto{\pgfqpoint{2.530920in}{0.522171in}}%
\pgfpathlineto{\pgfqpoint{2.511588in}{0.522171in}}%
\pgfpathlineto{\pgfqpoint{2.511588in}{0.505957in}}%
\pgfusepath{stroke,fill}%
\end{pgfscope}%
\begin{pgfscope}%
\pgfpathrectangle{\pgfqpoint{0.636356in}{0.440955in}}{\pgfqpoint{1.933229in}{0.162432in}} %
\pgfusepath{clip}%
\pgfsetbuttcap%
\pgfsetmiterjoin%
\definecolor{currentfill}{rgb}{0.333333,0.333333,0.333333}%
\pgfsetfillcolor{currentfill}%
\pgfsetlinewidth{0.501875pt}%
\definecolor{currentstroke}{rgb}{0.000000,0.000000,0.000000}%
\pgfsetstrokecolor{currentstroke}%
\pgfsetdash{}{0pt}%
\pgfpathmoveto{\pgfqpoint{2.530920in}{0.497593in}}%
\pgfpathlineto{\pgfqpoint{2.550252in}{0.497593in}}%
\pgfpathlineto{\pgfqpoint{2.550252in}{0.522171in}}%
\pgfpathlineto{\pgfqpoint{2.530920in}{0.522171in}}%
\pgfpathlineto{\pgfqpoint{2.530920in}{0.497593in}}%
\pgfusepath{stroke,fill}%
\end{pgfscope}%
\begin{pgfscope}%
\pgfpathrectangle{\pgfqpoint{0.636356in}{0.440955in}}{\pgfqpoint{1.933229in}{0.162432in}} %
\pgfusepath{clip}%
\pgfsetbuttcap%
\pgfsetmiterjoin%
\definecolor{currentfill}{rgb}{0.333333,0.333333,0.333333}%
\pgfsetfillcolor{currentfill}%
\pgfsetlinewidth{0.501875pt}%
\definecolor{currentstroke}{rgb}{0.000000,0.000000,0.000000}%
\pgfsetstrokecolor{currentstroke}%
\pgfsetdash{}{0pt}%
\pgfpathmoveto{\pgfqpoint{2.550252in}{0.522171in}}%
\pgfpathlineto{\pgfqpoint{2.569584in}{0.522171in}}%
\pgfpathlineto{\pgfqpoint{2.569584in}{0.525219in}}%
\pgfpathlineto{\pgfqpoint{2.550252in}{0.525219in}}%
\pgfpathlineto{\pgfqpoint{2.550252in}{0.522171in}}%
\pgfusepath{stroke,fill}%
\end{pgfscope}%
\begin{pgfscope}%
\pgfsetrectcap%
\pgfsetmiterjoin%
\pgfsetlinewidth{1.003750pt}%
\definecolor{currentstroke}{rgb}{0.000000,0.000000,0.000000}%
\pgfsetstrokecolor{currentstroke}%
\pgfsetdash{}{0pt}%
\pgfpathmoveto{\pgfqpoint{0.636356in}{0.603387in}}%
\pgfpathlineto{\pgfqpoint{2.569584in}{0.603387in}}%
\pgfusepath{stroke}%
\end{pgfscope}%
\begin{pgfscope}%
\pgfsetrectcap%
\pgfsetmiterjoin%
\pgfsetlinewidth{1.003750pt}%
\definecolor{currentstroke}{rgb}{0.000000,0.000000,0.000000}%
\pgfsetstrokecolor{currentstroke}%
\pgfsetdash{}{0pt}%
\pgfpathmoveto{\pgfqpoint{2.569584in}{0.440955in}}%
\pgfpathlineto{\pgfqpoint{2.569584in}{0.603387in}}%
\pgfusepath{stroke}%
\end{pgfscope}%
\begin{pgfscope}%
\pgfsetrectcap%
\pgfsetmiterjoin%
\pgfsetlinewidth{1.003750pt}%
\definecolor{currentstroke}{rgb}{0.000000,0.000000,0.000000}%
\pgfsetstrokecolor{currentstroke}%
\pgfsetdash{}{0pt}%
\pgfpathmoveto{\pgfqpoint{0.636356in}{0.440955in}}%
\pgfpathlineto{\pgfqpoint{2.569584in}{0.440955in}}%
\pgfusepath{stroke}%
\end{pgfscope}%
\begin{pgfscope}%
\pgfsetrectcap%
\pgfsetmiterjoin%
\pgfsetlinewidth{1.003750pt}%
\definecolor{currentstroke}{rgb}{0.000000,0.000000,0.000000}%
\pgfsetstrokecolor{currentstroke}%
\pgfsetdash{}{0pt}%
\pgfpathmoveto{\pgfqpoint{0.636356in}{0.440955in}}%
\pgfpathlineto{\pgfqpoint{0.636356in}{0.603387in}}%
\pgfusepath{stroke}%
\end{pgfscope}%
\begin{pgfscope}%
\pgfsetbuttcap%
\pgfsetroundjoin%
\definecolor{currentfill}{rgb}{0.000000,0.000000,0.000000}%
\pgfsetfillcolor{currentfill}%
\pgfsetlinewidth{0.501875pt}%
\definecolor{currentstroke}{rgb}{0.000000,0.000000,0.000000}%
\pgfsetstrokecolor{currentstroke}%
\pgfsetdash{}{0pt}%
\pgfsys@defobject{currentmarker}{\pgfqpoint{0.000000in}{0.000000in}}{\pgfqpoint{0.000000in}{0.069444in}}{%
\pgfpathmoveto{\pgfqpoint{0.000000in}{0.000000in}}%
\pgfpathlineto{\pgfqpoint{0.000000in}{0.069444in}}%
\pgfusepath{stroke,fill}%
}%
\begin{pgfscope}%
\pgfsys@transformshift{0.636356in}{0.440955in}%
\pgfsys@useobject{currentmarker}{}%
\end{pgfscope}%
\end{pgfscope}%
\begin{pgfscope}%
\pgfsetbuttcap%
\pgfsetroundjoin%
\definecolor{currentfill}{rgb}{0.000000,0.000000,0.000000}%
\pgfsetfillcolor{currentfill}%
\pgfsetlinewidth{0.501875pt}%
\definecolor{currentstroke}{rgb}{0.000000,0.000000,0.000000}%
\pgfsetstrokecolor{currentstroke}%
\pgfsetdash{}{0pt}%
\pgfsys@defobject{currentmarker}{\pgfqpoint{0.000000in}{-0.069444in}}{\pgfqpoint{0.000000in}{0.000000in}}{%
\pgfpathmoveto{\pgfqpoint{0.000000in}{0.000000in}}%
\pgfpathlineto{\pgfqpoint{0.000000in}{-0.069444in}}%
\pgfusepath{stroke,fill}%
}%
\begin{pgfscope}%
\pgfsys@transformshift{0.636356in}{0.603387in}%
\pgfsys@useobject{currentmarker}{}%
\end{pgfscope}%
\end{pgfscope}%
\begin{pgfscope}%
\pgftext[x=0.636356in,y=0.371511in,,top]{\rmfamily\fontsize{8.000000}{9.600000}\selectfont 1840}%
\end{pgfscope}%
\begin{pgfscope}%
\pgfsetbuttcap%
\pgfsetroundjoin%
\definecolor{currentfill}{rgb}{0.000000,0.000000,0.000000}%
\pgfsetfillcolor{currentfill}%
\pgfsetlinewidth{0.501875pt}%
\definecolor{currentstroke}{rgb}{0.000000,0.000000,0.000000}%
\pgfsetstrokecolor{currentstroke}%
\pgfsetdash{}{0pt}%
\pgfsys@defobject{currentmarker}{\pgfqpoint{0.000000in}{0.000000in}}{\pgfqpoint{0.000000in}{0.069444in}}{%
\pgfpathmoveto{\pgfqpoint{0.000000in}{0.000000in}}%
\pgfpathlineto{\pgfqpoint{0.000000in}{0.069444in}}%
\pgfusepath{stroke,fill}%
}%
\begin{pgfscope}%
\pgfsys@transformshift{1.023002in}{0.440955in}%
\pgfsys@useobject{currentmarker}{}%
\end{pgfscope}%
\end{pgfscope}%
\begin{pgfscope}%
\pgfsetbuttcap%
\pgfsetroundjoin%
\definecolor{currentfill}{rgb}{0.000000,0.000000,0.000000}%
\pgfsetfillcolor{currentfill}%
\pgfsetlinewidth{0.501875pt}%
\definecolor{currentstroke}{rgb}{0.000000,0.000000,0.000000}%
\pgfsetstrokecolor{currentstroke}%
\pgfsetdash{}{0pt}%
\pgfsys@defobject{currentmarker}{\pgfqpoint{0.000000in}{-0.069444in}}{\pgfqpoint{0.000000in}{0.000000in}}{%
\pgfpathmoveto{\pgfqpoint{0.000000in}{0.000000in}}%
\pgfpathlineto{\pgfqpoint{0.000000in}{-0.069444in}}%
\pgfusepath{stroke,fill}%
}%
\begin{pgfscope}%
\pgfsys@transformshift{1.023002in}{0.603387in}%
\pgfsys@useobject{currentmarker}{}%
\end{pgfscope}%
\end{pgfscope}%
\begin{pgfscope}%
\pgftext[x=1.023002in,y=0.371511in,,top]{\rmfamily\fontsize{8.000000}{9.600000}\selectfont 1850}%
\end{pgfscope}%
\begin{pgfscope}%
\pgfsetbuttcap%
\pgfsetroundjoin%
\definecolor{currentfill}{rgb}{0.000000,0.000000,0.000000}%
\pgfsetfillcolor{currentfill}%
\pgfsetlinewidth{0.501875pt}%
\definecolor{currentstroke}{rgb}{0.000000,0.000000,0.000000}%
\pgfsetstrokecolor{currentstroke}%
\pgfsetdash{}{0pt}%
\pgfsys@defobject{currentmarker}{\pgfqpoint{0.000000in}{0.000000in}}{\pgfqpoint{0.000000in}{0.069444in}}{%
\pgfpathmoveto{\pgfqpoint{0.000000in}{0.000000in}}%
\pgfpathlineto{\pgfqpoint{0.000000in}{0.069444in}}%
\pgfusepath{stroke,fill}%
}%
\begin{pgfscope}%
\pgfsys@transformshift{1.409647in}{0.440955in}%
\pgfsys@useobject{currentmarker}{}%
\end{pgfscope}%
\end{pgfscope}%
\begin{pgfscope}%
\pgfsetbuttcap%
\pgfsetroundjoin%
\definecolor{currentfill}{rgb}{0.000000,0.000000,0.000000}%
\pgfsetfillcolor{currentfill}%
\pgfsetlinewidth{0.501875pt}%
\definecolor{currentstroke}{rgb}{0.000000,0.000000,0.000000}%
\pgfsetstrokecolor{currentstroke}%
\pgfsetdash{}{0pt}%
\pgfsys@defobject{currentmarker}{\pgfqpoint{0.000000in}{-0.069444in}}{\pgfqpoint{0.000000in}{0.000000in}}{%
\pgfpathmoveto{\pgfqpoint{0.000000in}{0.000000in}}%
\pgfpathlineto{\pgfqpoint{0.000000in}{-0.069444in}}%
\pgfusepath{stroke,fill}%
}%
\begin{pgfscope}%
\pgfsys@transformshift{1.409647in}{0.603387in}%
\pgfsys@useobject{currentmarker}{}%
\end{pgfscope}%
\end{pgfscope}%
\begin{pgfscope}%
\pgftext[x=1.409647in,y=0.371511in,,top]{\rmfamily\fontsize{8.000000}{9.600000}\selectfont 1860}%
\end{pgfscope}%
\begin{pgfscope}%
\pgfsetbuttcap%
\pgfsetroundjoin%
\definecolor{currentfill}{rgb}{0.000000,0.000000,0.000000}%
\pgfsetfillcolor{currentfill}%
\pgfsetlinewidth{0.501875pt}%
\definecolor{currentstroke}{rgb}{0.000000,0.000000,0.000000}%
\pgfsetstrokecolor{currentstroke}%
\pgfsetdash{}{0pt}%
\pgfsys@defobject{currentmarker}{\pgfqpoint{0.000000in}{0.000000in}}{\pgfqpoint{0.000000in}{0.069444in}}{%
\pgfpathmoveto{\pgfqpoint{0.000000in}{0.000000in}}%
\pgfpathlineto{\pgfqpoint{0.000000in}{0.069444in}}%
\pgfusepath{stroke,fill}%
}%
\begin{pgfscope}%
\pgfsys@transformshift{1.796293in}{0.440955in}%
\pgfsys@useobject{currentmarker}{}%
\end{pgfscope}%
\end{pgfscope}%
\begin{pgfscope}%
\pgfsetbuttcap%
\pgfsetroundjoin%
\definecolor{currentfill}{rgb}{0.000000,0.000000,0.000000}%
\pgfsetfillcolor{currentfill}%
\pgfsetlinewidth{0.501875pt}%
\definecolor{currentstroke}{rgb}{0.000000,0.000000,0.000000}%
\pgfsetstrokecolor{currentstroke}%
\pgfsetdash{}{0pt}%
\pgfsys@defobject{currentmarker}{\pgfqpoint{0.000000in}{-0.069444in}}{\pgfqpoint{0.000000in}{0.000000in}}{%
\pgfpathmoveto{\pgfqpoint{0.000000in}{0.000000in}}%
\pgfpathlineto{\pgfqpoint{0.000000in}{-0.069444in}}%
\pgfusepath{stroke,fill}%
}%
\begin{pgfscope}%
\pgfsys@transformshift{1.796293in}{0.603387in}%
\pgfsys@useobject{currentmarker}{}%
\end{pgfscope}%
\end{pgfscope}%
\begin{pgfscope}%
\pgftext[x=1.796293in,y=0.371511in,,top]{\rmfamily\fontsize{8.000000}{9.600000}\selectfont 1870}%
\end{pgfscope}%
\begin{pgfscope}%
\pgfsetbuttcap%
\pgfsetroundjoin%
\definecolor{currentfill}{rgb}{0.000000,0.000000,0.000000}%
\pgfsetfillcolor{currentfill}%
\pgfsetlinewidth{0.501875pt}%
\definecolor{currentstroke}{rgb}{0.000000,0.000000,0.000000}%
\pgfsetstrokecolor{currentstroke}%
\pgfsetdash{}{0pt}%
\pgfsys@defobject{currentmarker}{\pgfqpoint{0.000000in}{0.000000in}}{\pgfqpoint{0.000000in}{0.069444in}}{%
\pgfpathmoveto{\pgfqpoint{0.000000in}{0.000000in}}%
\pgfpathlineto{\pgfqpoint{0.000000in}{0.069444in}}%
\pgfusepath{stroke,fill}%
}%
\begin{pgfscope}%
\pgfsys@transformshift{2.182939in}{0.440955in}%
\pgfsys@useobject{currentmarker}{}%
\end{pgfscope}%
\end{pgfscope}%
\begin{pgfscope}%
\pgfsetbuttcap%
\pgfsetroundjoin%
\definecolor{currentfill}{rgb}{0.000000,0.000000,0.000000}%
\pgfsetfillcolor{currentfill}%
\pgfsetlinewidth{0.501875pt}%
\definecolor{currentstroke}{rgb}{0.000000,0.000000,0.000000}%
\pgfsetstrokecolor{currentstroke}%
\pgfsetdash{}{0pt}%
\pgfsys@defobject{currentmarker}{\pgfqpoint{0.000000in}{-0.069444in}}{\pgfqpoint{0.000000in}{0.000000in}}{%
\pgfpathmoveto{\pgfqpoint{0.000000in}{0.000000in}}%
\pgfpathlineto{\pgfqpoint{0.000000in}{-0.069444in}}%
\pgfusepath{stroke,fill}%
}%
\begin{pgfscope}%
\pgfsys@transformshift{2.182939in}{0.603387in}%
\pgfsys@useobject{currentmarker}{}%
\end{pgfscope}%
\end{pgfscope}%
\begin{pgfscope}%
\pgftext[x=2.182939in,y=0.371511in,,top]{\rmfamily\fontsize{8.000000}{9.600000}\selectfont 1880}%
\end{pgfscope}%
\begin{pgfscope}%
\pgfsetbuttcap%
\pgfsetroundjoin%
\definecolor{currentfill}{rgb}{0.000000,0.000000,0.000000}%
\pgfsetfillcolor{currentfill}%
\pgfsetlinewidth{0.501875pt}%
\definecolor{currentstroke}{rgb}{0.000000,0.000000,0.000000}%
\pgfsetstrokecolor{currentstroke}%
\pgfsetdash{}{0pt}%
\pgfsys@defobject{currentmarker}{\pgfqpoint{0.000000in}{0.000000in}}{\pgfqpoint{0.000000in}{0.069444in}}{%
\pgfpathmoveto{\pgfqpoint{0.000000in}{0.000000in}}%
\pgfpathlineto{\pgfqpoint{0.000000in}{0.069444in}}%
\pgfusepath{stroke,fill}%
}%
\begin{pgfscope}%
\pgfsys@transformshift{2.569584in}{0.440955in}%
\pgfsys@useobject{currentmarker}{}%
\end{pgfscope}%
\end{pgfscope}%
\begin{pgfscope}%
\pgfsetbuttcap%
\pgfsetroundjoin%
\definecolor{currentfill}{rgb}{0.000000,0.000000,0.000000}%
\pgfsetfillcolor{currentfill}%
\pgfsetlinewidth{0.501875pt}%
\definecolor{currentstroke}{rgb}{0.000000,0.000000,0.000000}%
\pgfsetstrokecolor{currentstroke}%
\pgfsetdash{}{0pt}%
\pgfsys@defobject{currentmarker}{\pgfqpoint{0.000000in}{-0.069444in}}{\pgfqpoint{0.000000in}{0.000000in}}{%
\pgfpathmoveto{\pgfqpoint{0.000000in}{0.000000in}}%
\pgfpathlineto{\pgfqpoint{0.000000in}{-0.069444in}}%
\pgfusepath{stroke,fill}%
}%
\begin{pgfscope}%
\pgfsys@transformshift{2.569584in}{0.603387in}%
\pgfsys@useobject{currentmarker}{}%
\end{pgfscope}%
\end{pgfscope}%
\begin{pgfscope}%
\pgftext[x=2.569584in,y=0.371511in,,top]{\rmfamily\fontsize{8.000000}{9.600000}\selectfont 1890}%
\end{pgfscope}%
\begin{pgfscope}%
\pgftext[x=1.602970in,y=0.194536in,,top]{\rmfamily\fontsize{9.000000}{10.800000}\selectfont \(\displaystyle m(K^+\!\pi^-)\)}%
\end{pgfscope}%
\begin{pgfscope}%
\pgfsetbuttcap%
\pgfsetroundjoin%
\definecolor{currentfill}{rgb}{0.000000,0.000000,0.000000}%
\pgfsetfillcolor{currentfill}%
\pgfsetlinewidth{0.501875pt}%
\definecolor{currentstroke}{rgb}{0.000000,0.000000,0.000000}%
\pgfsetstrokecolor{currentstroke}%
\pgfsetdash{}{0pt}%
\pgfsys@defobject{currentmarker}{\pgfqpoint{0.000000in}{0.000000in}}{\pgfqpoint{0.069444in}{0.000000in}}{%
\pgfpathmoveto{\pgfqpoint{0.000000in}{0.000000in}}%
\pgfpathlineto{\pgfqpoint{0.069444in}{0.000000in}}%
\pgfusepath{stroke,fill}%
}%
\begin{pgfscope}%
\pgfsys@transformshift{0.636356in}{0.440955in}%
\pgfsys@useobject{currentmarker}{}%
\end{pgfscope}%
\end{pgfscope}%
\begin{pgfscope}%
\pgfsetbuttcap%
\pgfsetroundjoin%
\definecolor{currentfill}{rgb}{0.000000,0.000000,0.000000}%
\pgfsetfillcolor{currentfill}%
\pgfsetlinewidth{0.501875pt}%
\definecolor{currentstroke}{rgb}{0.000000,0.000000,0.000000}%
\pgfsetstrokecolor{currentstroke}%
\pgfsetdash{}{0pt}%
\pgfsys@defobject{currentmarker}{\pgfqpoint{-0.069444in}{0.000000in}}{\pgfqpoint{0.000000in}{0.000000in}}{%
\pgfpathmoveto{\pgfqpoint{0.000000in}{0.000000in}}%
\pgfpathlineto{\pgfqpoint{-0.069444in}{0.000000in}}%
\pgfusepath{stroke,fill}%
}%
\begin{pgfscope}%
\pgfsys@transformshift{2.569584in}{0.440955in}%
\pgfsys@useobject{currentmarker}{}%
\end{pgfscope}%
\end{pgfscope}%
\begin{pgfscope}%
\pgftext[x=0.566911in,y=0.440955in,right,]{\rmfamily\fontsize{8.000000}{9.600000}\selectfont −3}%
\end{pgfscope}%
\begin{pgfscope}%
\pgfsetbuttcap%
\pgfsetroundjoin%
\definecolor{currentfill}{rgb}{0.000000,0.000000,0.000000}%
\pgfsetfillcolor{currentfill}%
\pgfsetlinewidth{0.501875pt}%
\definecolor{currentstroke}{rgb}{0.000000,0.000000,0.000000}%
\pgfsetstrokecolor{currentstroke}%
\pgfsetdash{}{0pt}%
\pgfsys@defobject{currentmarker}{\pgfqpoint{0.000000in}{0.000000in}}{\pgfqpoint{0.069444in}{0.000000in}}{%
\pgfpathmoveto{\pgfqpoint{0.000000in}{0.000000in}}%
\pgfpathlineto{\pgfqpoint{0.069444in}{0.000000in}}%
\pgfusepath{stroke,fill}%
}%
\begin{pgfscope}%
\pgfsys@transformshift{0.636356in}{0.522171in}%
\pgfsys@useobject{currentmarker}{}%
\end{pgfscope}%
\end{pgfscope}%
\begin{pgfscope}%
\pgfsetbuttcap%
\pgfsetroundjoin%
\definecolor{currentfill}{rgb}{0.000000,0.000000,0.000000}%
\pgfsetfillcolor{currentfill}%
\pgfsetlinewidth{0.501875pt}%
\definecolor{currentstroke}{rgb}{0.000000,0.000000,0.000000}%
\pgfsetstrokecolor{currentstroke}%
\pgfsetdash{}{0pt}%
\pgfsys@defobject{currentmarker}{\pgfqpoint{-0.069444in}{0.000000in}}{\pgfqpoint{0.000000in}{0.000000in}}{%
\pgfpathmoveto{\pgfqpoint{0.000000in}{0.000000in}}%
\pgfpathlineto{\pgfqpoint{-0.069444in}{0.000000in}}%
\pgfusepath{stroke,fill}%
}%
\begin{pgfscope}%
\pgfsys@transformshift{2.569584in}{0.522171in}%
\pgfsys@useobject{currentmarker}{}%
\end{pgfscope}%
\end{pgfscope}%
\begin{pgfscope}%
\pgftext[x=0.566911in,y=0.522171in,right,]{\rmfamily\fontsize{8.000000}{9.600000}\selectfont 0}%
\end{pgfscope}%
\begin{pgfscope}%
\pgfsetbuttcap%
\pgfsetroundjoin%
\definecolor{currentfill}{rgb}{0.000000,0.000000,0.000000}%
\pgfsetfillcolor{currentfill}%
\pgfsetlinewidth{0.501875pt}%
\definecolor{currentstroke}{rgb}{0.000000,0.000000,0.000000}%
\pgfsetstrokecolor{currentstroke}%
\pgfsetdash{}{0pt}%
\pgfsys@defobject{currentmarker}{\pgfqpoint{0.000000in}{0.000000in}}{\pgfqpoint{0.069444in}{0.000000in}}{%
\pgfpathmoveto{\pgfqpoint{0.000000in}{0.000000in}}%
\pgfpathlineto{\pgfqpoint{0.069444in}{0.000000in}}%
\pgfusepath{stroke,fill}%
}%
\begin{pgfscope}%
\pgfsys@transformshift{0.636356in}{0.603387in}%
\pgfsys@useobject{currentmarker}{}%
\end{pgfscope}%
\end{pgfscope}%
\begin{pgfscope}%
\pgfsetbuttcap%
\pgfsetroundjoin%
\definecolor{currentfill}{rgb}{0.000000,0.000000,0.000000}%
\pgfsetfillcolor{currentfill}%
\pgfsetlinewidth{0.501875pt}%
\definecolor{currentstroke}{rgb}{0.000000,0.000000,0.000000}%
\pgfsetstrokecolor{currentstroke}%
\pgfsetdash{}{0pt}%
\pgfsys@defobject{currentmarker}{\pgfqpoint{-0.069444in}{0.000000in}}{\pgfqpoint{0.000000in}{0.000000in}}{%
\pgfpathmoveto{\pgfqpoint{0.000000in}{0.000000in}}%
\pgfpathlineto{\pgfqpoint{-0.069444in}{0.000000in}}%
\pgfusepath{stroke,fill}%
}%
\begin{pgfscope}%
\pgfsys@transformshift{2.569584in}{0.603387in}%
\pgfsys@useobject{currentmarker}{}%
\end{pgfscope}%
\end{pgfscope}%
\begin{pgfscope}%
\pgftext[x=0.566911in,y=0.603387in,right,]{\rmfamily\fontsize{8.000000}{9.600000}\selectfont 3}%
\end{pgfscope}%
\begin{pgfscope}%
\pgftext[x=0.333676in,y=0.522171in,,bottom,rotate=90.000000]{\rmfamily\fontsize{9.000000}{10.800000}\selectfont \(\displaystyle \frac{\hat{n}_i -  n_i}{\sigma(n_i)}\)}%
\end{pgfscope}%
\begin{pgfscope}%
\pgfsetbuttcap%
\pgfsetmiterjoin%
\definecolor{currentfill}{rgb}{1.000000,1.000000,1.000000}%
\pgfsetfillcolor{currentfill}%
\pgfsetlinewidth{0.000000pt}%
\definecolor{currentstroke}{rgb}{0.000000,0.000000,0.000000}%
\pgfsetstrokecolor{currentstroke}%
\pgfsetstrokeopacity{0.000000}%
\pgfsetdash{}{0pt}%
\pgfpathmoveto{\pgfqpoint{0.636356in}{0.700846in}}%
\pgfpathlineto{\pgfqpoint{2.569584in}{0.700846in}}%
\pgfpathlineto{\pgfqpoint{2.569584in}{1.837869in}}%
\pgfpathlineto{\pgfqpoint{0.636356in}{1.837869in}}%
\pgfpathclose%
\pgfusepath{fill}%
\end{pgfscope}%
\begin{pgfscope}%
\pgfpathrectangle{\pgfqpoint{0.636356in}{0.700846in}}{\pgfqpoint{1.933229in}{1.137023in}} %
\pgfusepath{clip}%
\pgfsetbuttcap%
\pgfsetroundjoin%
\pgfsetlinewidth{1.003750pt}%
\definecolor{currentstroke}{rgb}{0.000000,0.000000,1.000000}%
\pgfsetstrokecolor{currentstroke}%
\pgfsetdash{{8.000000pt}{3.000000pt}}{0.000000pt}%
\pgfpathmoveto{\pgfqpoint{0.636356in}{0.700920in}}%
\pgfpathlineto{\pgfqpoint{0.801506in}{0.702137in}}%
\pgfpathlineto{\pgfqpoint{0.869509in}{0.704407in}}%
\pgfpathlineto{\pgfqpoint{0.927797in}{0.708821in}}%
\pgfpathlineto{\pgfqpoint{0.966656in}{0.714007in}}%
\pgfpathlineto{\pgfqpoint{1.005515in}{0.721940in}}%
\pgfpathlineto{\pgfqpoint{1.034659in}{0.730407in}}%
\pgfpathlineto{\pgfqpoint{1.063803in}{0.741370in}}%
\pgfpathlineto{\pgfqpoint{1.092948in}{0.755755in}}%
\pgfpathlineto{\pgfqpoint{1.112377in}{0.767368in}}%
\pgfpathlineto{\pgfqpoint{1.131806in}{0.780853in}}%
\pgfpathlineto{\pgfqpoint{1.151236in}{0.796375in}}%
\pgfpathlineto{\pgfqpoint{1.170665in}{0.814080in}}%
\pgfpathlineto{\pgfqpoint{1.190095in}{0.834095in}}%
\pgfpathlineto{\pgfqpoint{1.209524in}{0.856510in}}%
\pgfpathlineto{\pgfqpoint{1.228954in}{0.881378in}}%
\pgfpathlineto{\pgfqpoint{1.248383in}{0.908702in}}%
\pgfpathlineto{\pgfqpoint{1.277527in}{0.954056in}}%
\pgfpathlineto{\pgfqpoint{1.306671in}{1.004532in}}%
\pgfpathlineto{\pgfqpoint{1.335815in}{1.058962in}}%
\pgfpathlineto{\pgfqpoint{1.384389in}{1.155560in}}%
\pgfpathlineto{\pgfqpoint{1.442677in}{1.271069in}}%
\pgfpathlineto{\pgfqpoint{1.471821in}{1.323935in}}%
\pgfpathlineto{\pgfqpoint{1.491251in}{1.355873in}}%
\pgfpathlineto{\pgfqpoint{1.510680in}{1.384451in}}%
\pgfpathlineto{\pgfqpoint{1.530110in}{1.409100in}}%
\pgfpathlineto{\pgfqpoint{1.549539in}{1.429315in}}%
\pgfpathlineto{\pgfqpoint{1.568969in}{1.444671in}}%
\pgfpathlineto{\pgfqpoint{1.578683in}{1.450411in}}%
\pgfpathlineto{\pgfqpoint{1.588398in}{1.454842in}}%
\pgfpathlineto{\pgfqpoint{1.598113in}{1.457910in}}%
\pgfpathlineto{\pgfqpoint{1.607827in}{1.459608in}}%
\pgfpathlineto{\pgfqpoint{1.617542in}{1.459937in}}%
\pgfpathlineto{\pgfqpoint{1.627257in}{1.458866in}}%
\pgfpathlineto{\pgfqpoint{1.636972in}{1.456450in}}%
\pgfpathlineto{\pgfqpoint{1.646686in}{1.452633in}}%
\pgfpathlineto{\pgfqpoint{1.656401in}{1.447523in}}%
\pgfpathlineto{\pgfqpoint{1.675831in}{1.433349in}}%
\pgfpathlineto{\pgfqpoint{1.695260in}{1.414231in}}%
\pgfpathlineto{\pgfqpoint{1.714689in}{1.390572in}}%
\pgfpathlineto{\pgfqpoint{1.734119in}{1.362860in}}%
\pgfpathlineto{\pgfqpoint{1.753548in}{1.331651in}}%
\pgfpathlineto{\pgfqpoint{1.782692in}{1.279592in}}%
\pgfpathlineto{\pgfqpoint{1.821551in}{1.203865in}}%
\pgfpathlineto{\pgfqpoint{1.899269in}{1.049069in}}%
\pgfpathlineto{\pgfqpoint{1.928413in}{0.995150in}}%
\pgfpathlineto{\pgfqpoint{1.957557in}{0.945687in}}%
\pgfpathlineto{\pgfqpoint{1.986701in}{0.901049in}}%
\pgfpathlineto{\pgfqpoint{2.015846in}{0.862087in}}%
\pgfpathlineto{\pgfqpoint{2.035275in}{0.839103in}}%
\pgfpathlineto{\pgfqpoint{2.054704in}{0.818536in}}%
\pgfpathlineto{\pgfqpoint{2.074134in}{0.800302in}}%
\pgfpathlineto{\pgfqpoint{2.093563in}{0.784284in}}%
\pgfpathlineto{\pgfqpoint{2.112993in}{0.770338in}}%
\pgfpathlineto{\pgfqpoint{2.132422in}{0.758304in}}%
\pgfpathlineto{\pgfqpoint{2.151852in}{0.748010in}}%
\pgfpathlineto{\pgfqpoint{2.171281in}{0.739279in}}%
\pgfpathlineto{\pgfqpoint{2.200425in}{0.728651in}}%
\pgfpathlineto{\pgfqpoint{2.229569in}{0.720754in}}%
\pgfpathlineto{\pgfqpoint{2.258714in}{0.714770in}}%
\pgfpathlineto{\pgfqpoint{2.297572in}{0.709312in}}%
\pgfpathlineto{\pgfqpoint{2.346146in}{0.705236in}}%
\pgfpathlineto{\pgfqpoint{2.404434in}{0.702714in}}%
\pgfpathlineto{\pgfqpoint{2.491867in}{0.701300in}}%
\pgfpathlineto{\pgfqpoint{2.569584in}{0.700961in}}%
\pgfpathlineto{\pgfqpoint{2.569584in}{0.700961in}}%
\pgfusepath{stroke}%
\end{pgfscope}%
\begin{pgfscope}%
\pgfpathrectangle{\pgfqpoint{0.636356in}{0.700846in}}{\pgfqpoint{1.933229in}{1.137023in}} %
\pgfusepath{clip}%
\pgfsetbuttcap%
\pgfsetroundjoin%
\pgfsetlinewidth{1.003750pt}%
\definecolor{currentstroke}{rgb}{0.000000,0.500000,0.000000}%
\pgfsetstrokecolor{currentstroke}%
\pgfsetdash{{8.000000pt}{3.000000pt}}{0.000000pt}%
\pgfpathmoveto{\pgfqpoint{0.636356in}{0.719315in}}%
\pgfpathlineto{\pgfqpoint{0.704359in}{0.727379in}}%
\pgfpathlineto{\pgfqpoint{0.762647in}{0.736265in}}%
\pgfpathlineto{\pgfqpoint{0.820935in}{0.747232in}}%
\pgfpathlineto{\pgfqpoint{0.879224in}{0.760446in}}%
\pgfpathlineto{\pgfqpoint{0.937512in}{0.775974in}}%
\pgfpathlineto{\pgfqpoint{0.995800in}{0.793758in}}%
\pgfpathlineto{\pgfqpoint{1.054089in}{0.813577in}}%
\pgfpathlineto{\pgfqpoint{1.131806in}{0.842458in}}%
\pgfpathlineto{\pgfqpoint{1.335815in}{0.920118in}}%
\pgfpathlineto{\pgfqpoint{1.394104in}{0.938840in}}%
\pgfpathlineto{\pgfqpoint{1.442677in}{0.951957in}}%
\pgfpathlineto{\pgfqpoint{1.481536in}{0.960486in}}%
\pgfpathlineto{\pgfqpoint{1.520395in}{0.967034in}}%
\pgfpathlineto{\pgfqpoint{1.559254in}{0.971441in}}%
\pgfpathlineto{\pgfqpoint{1.598113in}{0.973595in}}%
\pgfpathlineto{\pgfqpoint{1.636972in}{0.973441in}}%
\pgfpathlineto{\pgfqpoint{1.675831in}{0.970985in}}%
\pgfpathlineto{\pgfqpoint{1.714689in}{0.966287in}}%
\pgfpathlineto{\pgfqpoint{1.753548in}{0.959466in}}%
\pgfpathlineto{\pgfqpoint{1.802122in}{0.948203in}}%
\pgfpathlineto{\pgfqpoint{1.850695in}{0.934331in}}%
\pgfpathlineto{\pgfqpoint{1.908984in}{0.914915in}}%
\pgfpathlineto{\pgfqpoint{1.976987in}{0.889674in}}%
\pgfpathlineto{\pgfqpoint{2.190710in}{0.808335in}}%
\pgfpathlineto{\pgfqpoint{2.248999in}{0.789006in}}%
\pgfpathlineto{\pgfqpoint{2.307287in}{0.771785in}}%
\pgfpathlineto{\pgfqpoint{2.365575in}{0.756848in}}%
\pgfpathlineto{\pgfqpoint{2.423864in}{0.744220in}}%
\pgfpathlineto{\pgfqpoint{2.482152in}{0.733803in}}%
\pgfpathlineto{\pgfqpoint{2.540440in}{0.725415in}}%
\pgfpathlineto{\pgfqpoint{2.569584in}{0.721885in}}%
\pgfpathlineto{\pgfqpoint{2.569584in}{0.721885in}}%
\pgfusepath{stroke}%
\end{pgfscope}%
\begin{pgfscope}%
\pgfsetrectcap%
\pgfsetmiterjoin%
\pgfsetlinewidth{1.003750pt}%
\definecolor{currentstroke}{rgb}{0.000000,0.000000,0.000000}%
\pgfsetstrokecolor{currentstroke}%
\pgfsetdash{}{0pt}%
\pgfpathmoveto{\pgfqpoint{0.636356in}{1.837869in}}%
\pgfpathlineto{\pgfqpoint{2.569584in}{1.837869in}}%
\pgfusepath{stroke}%
\end{pgfscope}%
\begin{pgfscope}%
\pgfsetrectcap%
\pgfsetmiterjoin%
\pgfsetlinewidth{1.003750pt}%
\definecolor{currentstroke}{rgb}{0.000000,0.000000,0.000000}%
\pgfsetstrokecolor{currentstroke}%
\pgfsetdash{}{0pt}%
\pgfpathmoveto{\pgfqpoint{2.569584in}{0.700846in}}%
\pgfpathlineto{\pgfqpoint{2.569584in}{1.837869in}}%
\pgfusepath{stroke}%
\end{pgfscope}%
\begin{pgfscope}%
\pgfsetrectcap%
\pgfsetmiterjoin%
\pgfsetlinewidth{1.003750pt}%
\definecolor{currentstroke}{rgb}{0.000000,0.000000,0.000000}%
\pgfsetstrokecolor{currentstroke}%
\pgfsetdash{}{0pt}%
\pgfpathmoveto{\pgfqpoint{0.636356in}{0.700846in}}%
\pgfpathlineto{\pgfqpoint{2.569584in}{0.700846in}}%
\pgfusepath{stroke}%
\end{pgfscope}%
\begin{pgfscope}%
\pgfsetrectcap%
\pgfsetmiterjoin%
\pgfsetlinewidth{1.003750pt}%
\definecolor{currentstroke}{rgb}{0.000000,0.000000,0.000000}%
\pgfsetstrokecolor{currentstroke}%
\pgfsetdash{}{0pt}%
\pgfpathmoveto{\pgfqpoint{0.636356in}{0.700846in}}%
\pgfpathlineto{\pgfqpoint{0.636356in}{1.837869in}}%
\pgfusepath{stroke}%
\end{pgfscope}%
\begin{pgfscope}%
\pgfsetbuttcap%
\pgfsetroundjoin%
\definecolor{currentfill}{rgb}{0.000000,0.000000,0.000000}%
\pgfsetfillcolor{currentfill}%
\pgfsetlinewidth{0.501875pt}%
\definecolor{currentstroke}{rgb}{0.000000,0.000000,0.000000}%
\pgfsetstrokecolor{currentstroke}%
\pgfsetdash{}{0pt}%
\pgfsys@defobject{currentmarker}{\pgfqpoint{0.000000in}{0.000000in}}{\pgfqpoint{0.000000in}{0.069444in}}{%
\pgfpathmoveto{\pgfqpoint{0.000000in}{0.000000in}}%
\pgfpathlineto{\pgfqpoint{0.000000in}{0.069444in}}%
\pgfusepath{stroke,fill}%
}%
\begin{pgfscope}%
\pgfsys@transformshift{0.636356in}{0.700846in}%
\pgfsys@useobject{currentmarker}{}%
\end{pgfscope}%
\end{pgfscope}%
\begin{pgfscope}%
\pgfsetbuttcap%
\pgfsetroundjoin%
\definecolor{currentfill}{rgb}{0.000000,0.000000,0.000000}%
\pgfsetfillcolor{currentfill}%
\pgfsetlinewidth{0.501875pt}%
\definecolor{currentstroke}{rgb}{0.000000,0.000000,0.000000}%
\pgfsetstrokecolor{currentstroke}%
\pgfsetdash{}{0pt}%
\pgfsys@defobject{currentmarker}{\pgfqpoint{0.000000in}{-0.069444in}}{\pgfqpoint{0.000000in}{0.000000in}}{%
\pgfpathmoveto{\pgfqpoint{0.000000in}{0.000000in}}%
\pgfpathlineto{\pgfqpoint{0.000000in}{-0.069444in}}%
\pgfusepath{stroke,fill}%
}%
\begin{pgfscope}%
\pgfsys@transformshift{0.636356in}{1.837869in}%
\pgfsys@useobject{currentmarker}{}%
\end{pgfscope}%
\end{pgfscope}%
\begin{pgfscope}%
\pgfsetbuttcap%
\pgfsetroundjoin%
\definecolor{currentfill}{rgb}{0.000000,0.000000,0.000000}%
\pgfsetfillcolor{currentfill}%
\pgfsetlinewidth{0.501875pt}%
\definecolor{currentstroke}{rgb}{0.000000,0.000000,0.000000}%
\pgfsetstrokecolor{currentstroke}%
\pgfsetdash{}{0pt}%
\pgfsys@defobject{currentmarker}{\pgfqpoint{0.000000in}{0.000000in}}{\pgfqpoint{0.000000in}{0.069444in}}{%
\pgfpathmoveto{\pgfqpoint{0.000000in}{0.000000in}}%
\pgfpathlineto{\pgfqpoint{0.000000in}{0.069444in}}%
\pgfusepath{stroke,fill}%
}%
\begin{pgfscope}%
\pgfsys@transformshift{1.023002in}{0.700846in}%
\pgfsys@useobject{currentmarker}{}%
\end{pgfscope}%
\end{pgfscope}%
\begin{pgfscope}%
\pgfsetbuttcap%
\pgfsetroundjoin%
\definecolor{currentfill}{rgb}{0.000000,0.000000,0.000000}%
\pgfsetfillcolor{currentfill}%
\pgfsetlinewidth{0.501875pt}%
\definecolor{currentstroke}{rgb}{0.000000,0.000000,0.000000}%
\pgfsetstrokecolor{currentstroke}%
\pgfsetdash{}{0pt}%
\pgfsys@defobject{currentmarker}{\pgfqpoint{0.000000in}{-0.069444in}}{\pgfqpoint{0.000000in}{0.000000in}}{%
\pgfpathmoveto{\pgfqpoint{0.000000in}{0.000000in}}%
\pgfpathlineto{\pgfqpoint{0.000000in}{-0.069444in}}%
\pgfusepath{stroke,fill}%
}%
\begin{pgfscope}%
\pgfsys@transformshift{1.023002in}{1.837869in}%
\pgfsys@useobject{currentmarker}{}%
\end{pgfscope}%
\end{pgfscope}%
\begin{pgfscope}%
\pgfsetbuttcap%
\pgfsetroundjoin%
\definecolor{currentfill}{rgb}{0.000000,0.000000,0.000000}%
\pgfsetfillcolor{currentfill}%
\pgfsetlinewidth{0.501875pt}%
\definecolor{currentstroke}{rgb}{0.000000,0.000000,0.000000}%
\pgfsetstrokecolor{currentstroke}%
\pgfsetdash{}{0pt}%
\pgfsys@defobject{currentmarker}{\pgfqpoint{0.000000in}{0.000000in}}{\pgfqpoint{0.000000in}{0.069444in}}{%
\pgfpathmoveto{\pgfqpoint{0.000000in}{0.000000in}}%
\pgfpathlineto{\pgfqpoint{0.000000in}{0.069444in}}%
\pgfusepath{stroke,fill}%
}%
\begin{pgfscope}%
\pgfsys@transformshift{1.409647in}{0.700846in}%
\pgfsys@useobject{currentmarker}{}%
\end{pgfscope}%
\end{pgfscope}%
\begin{pgfscope}%
\pgfsetbuttcap%
\pgfsetroundjoin%
\definecolor{currentfill}{rgb}{0.000000,0.000000,0.000000}%
\pgfsetfillcolor{currentfill}%
\pgfsetlinewidth{0.501875pt}%
\definecolor{currentstroke}{rgb}{0.000000,0.000000,0.000000}%
\pgfsetstrokecolor{currentstroke}%
\pgfsetdash{}{0pt}%
\pgfsys@defobject{currentmarker}{\pgfqpoint{0.000000in}{-0.069444in}}{\pgfqpoint{0.000000in}{0.000000in}}{%
\pgfpathmoveto{\pgfqpoint{0.000000in}{0.000000in}}%
\pgfpathlineto{\pgfqpoint{0.000000in}{-0.069444in}}%
\pgfusepath{stroke,fill}%
}%
\begin{pgfscope}%
\pgfsys@transformshift{1.409647in}{1.837869in}%
\pgfsys@useobject{currentmarker}{}%
\end{pgfscope}%
\end{pgfscope}%
\begin{pgfscope}%
\pgfsetbuttcap%
\pgfsetroundjoin%
\definecolor{currentfill}{rgb}{0.000000,0.000000,0.000000}%
\pgfsetfillcolor{currentfill}%
\pgfsetlinewidth{0.501875pt}%
\definecolor{currentstroke}{rgb}{0.000000,0.000000,0.000000}%
\pgfsetstrokecolor{currentstroke}%
\pgfsetdash{}{0pt}%
\pgfsys@defobject{currentmarker}{\pgfqpoint{0.000000in}{0.000000in}}{\pgfqpoint{0.000000in}{0.069444in}}{%
\pgfpathmoveto{\pgfqpoint{0.000000in}{0.000000in}}%
\pgfpathlineto{\pgfqpoint{0.000000in}{0.069444in}}%
\pgfusepath{stroke,fill}%
}%
\begin{pgfscope}%
\pgfsys@transformshift{1.796293in}{0.700846in}%
\pgfsys@useobject{currentmarker}{}%
\end{pgfscope}%
\end{pgfscope}%
\begin{pgfscope}%
\pgfsetbuttcap%
\pgfsetroundjoin%
\definecolor{currentfill}{rgb}{0.000000,0.000000,0.000000}%
\pgfsetfillcolor{currentfill}%
\pgfsetlinewidth{0.501875pt}%
\definecolor{currentstroke}{rgb}{0.000000,0.000000,0.000000}%
\pgfsetstrokecolor{currentstroke}%
\pgfsetdash{}{0pt}%
\pgfsys@defobject{currentmarker}{\pgfqpoint{0.000000in}{-0.069444in}}{\pgfqpoint{0.000000in}{0.000000in}}{%
\pgfpathmoveto{\pgfqpoint{0.000000in}{0.000000in}}%
\pgfpathlineto{\pgfqpoint{0.000000in}{-0.069444in}}%
\pgfusepath{stroke,fill}%
}%
\begin{pgfscope}%
\pgfsys@transformshift{1.796293in}{1.837869in}%
\pgfsys@useobject{currentmarker}{}%
\end{pgfscope}%
\end{pgfscope}%
\begin{pgfscope}%
\pgfsetbuttcap%
\pgfsetroundjoin%
\definecolor{currentfill}{rgb}{0.000000,0.000000,0.000000}%
\pgfsetfillcolor{currentfill}%
\pgfsetlinewidth{0.501875pt}%
\definecolor{currentstroke}{rgb}{0.000000,0.000000,0.000000}%
\pgfsetstrokecolor{currentstroke}%
\pgfsetdash{}{0pt}%
\pgfsys@defobject{currentmarker}{\pgfqpoint{0.000000in}{0.000000in}}{\pgfqpoint{0.000000in}{0.069444in}}{%
\pgfpathmoveto{\pgfqpoint{0.000000in}{0.000000in}}%
\pgfpathlineto{\pgfqpoint{0.000000in}{0.069444in}}%
\pgfusepath{stroke,fill}%
}%
\begin{pgfscope}%
\pgfsys@transformshift{2.182939in}{0.700846in}%
\pgfsys@useobject{currentmarker}{}%
\end{pgfscope}%
\end{pgfscope}%
\begin{pgfscope}%
\pgfsetbuttcap%
\pgfsetroundjoin%
\definecolor{currentfill}{rgb}{0.000000,0.000000,0.000000}%
\pgfsetfillcolor{currentfill}%
\pgfsetlinewidth{0.501875pt}%
\definecolor{currentstroke}{rgb}{0.000000,0.000000,0.000000}%
\pgfsetstrokecolor{currentstroke}%
\pgfsetdash{}{0pt}%
\pgfsys@defobject{currentmarker}{\pgfqpoint{0.000000in}{-0.069444in}}{\pgfqpoint{0.000000in}{0.000000in}}{%
\pgfpathmoveto{\pgfqpoint{0.000000in}{0.000000in}}%
\pgfpathlineto{\pgfqpoint{0.000000in}{-0.069444in}}%
\pgfusepath{stroke,fill}%
}%
\begin{pgfscope}%
\pgfsys@transformshift{2.182939in}{1.837869in}%
\pgfsys@useobject{currentmarker}{}%
\end{pgfscope}%
\end{pgfscope}%
\begin{pgfscope}%
\pgfsetbuttcap%
\pgfsetroundjoin%
\definecolor{currentfill}{rgb}{0.000000,0.000000,0.000000}%
\pgfsetfillcolor{currentfill}%
\pgfsetlinewidth{0.501875pt}%
\definecolor{currentstroke}{rgb}{0.000000,0.000000,0.000000}%
\pgfsetstrokecolor{currentstroke}%
\pgfsetdash{}{0pt}%
\pgfsys@defobject{currentmarker}{\pgfqpoint{0.000000in}{0.000000in}}{\pgfqpoint{0.000000in}{0.069444in}}{%
\pgfpathmoveto{\pgfqpoint{0.000000in}{0.000000in}}%
\pgfpathlineto{\pgfqpoint{0.000000in}{0.069444in}}%
\pgfusepath{stroke,fill}%
}%
\begin{pgfscope}%
\pgfsys@transformshift{2.569584in}{0.700846in}%
\pgfsys@useobject{currentmarker}{}%
\end{pgfscope}%
\end{pgfscope}%
\begin{pgfscope}%
\pgfsetbuttcap%
\pgfsetroundjoin%
\definecolor{currentfill}{rgb}{0.000000,0.000000,0.000000}%
\pgfsetfillcolor{currentfill}%
\pgfsetlinewidth{0.501875pt}%
\definecolor{currentstroke}{rgb}{0.000000,0.000000,0.000000}%
\pgfsetstrokecolor{currentstroke}%
\pgfsetdash{}{0pt}%
\pgfsys@defobject{currentmarker}{\pgfqpoint{0.000000in}{-0.069444in}}{\pgfqpoint{0.000000in}{0.000000in}}{%
\pgfpathmoveto{\pgfqpoint{0.000000in}{0.000000in}}%
\pgfpathlineto{\pgfqpoint{0.000000in}{-0.069444in}}%
\pgfusepath{stroke,fill}%
}%
\begin{pgfscope}%
\pgfsys@transformshift{2.569584in}{1.837869in}%
\pgfsys@useobject{currentmarker}{}%
\end{pgfscope}%
\end{pgfscope}%
\begin{pgfscope}%
\pgfsetbuttcap%
\pgfsetroundjoin%
\definecolor{currentfill}{rgb}{0.000000,0.000000,0.000000}%
\pgfsetfillcolor{currentfill}%
\pgfsetlinewidth{0.501875pt}%
\definecolor{currentstroke}{rgb}{0.000000,0.000000,0.000000}%
\pgfsetstrokecolor{currentstroke}%
\pgfsetdash{}{0pt}%
\pgfsys@defobject{currentmarker}{\pgfqpoint{0.000000in}{0.000000in}}{\pgfqpoint{0.069444in}{0.000000in}}{%
\pgfpathmoveto{\pgfqpoint{0.000000in}{0.000000in}}%
\pgfpathlineto{\pgfqpoint{0.069444in}{0.000000in}}%
\pgfusepath{stroke,fill}%
}%
\begin{pgfscope}%
\pgfsys@transformshift{0.636356in}{0.700846in}%
\pgfsys@useobject{currentmarker}{}%
\end{pgfscope}%
\end{pgfscope}%
\begin{pgfscope}%
\pgfsetbuttcap%
\pgfsetroundjoin%
\definecolor{currentfill}{rgb}{0.000000,0.000000,0.000000}%
\pgfsetfillcolor{currentfill}%
\pgfsetlinewidth{0.501875pt}%
\definecolor{currentstroke}{rgb}{0.000000,0.000000,0.000000}%
\pgfsetstrokecolor{currentstroke}%
\pgfsetdash{}{0pt}%
\pgfsys@defobject{currentmarker}{\pgfqpoint{-0.069444in}{0.000000in}}{\pgfqpoint{0.000000in}{0.000000in}}{%
\pgfpathmoveto{\pgfqpoint{0.000000in}{0.000000in}}%
\pgfpathlineto{\pgfqpoint{-0.069444in}{0.000000in}}%
\pgfusepath{stroke,fill}%
}%
\begin{pgfscope}%
\pgfsys@transformshift{2.569584in}{0.700846in}%
\pgfsys@useobject{currentmarker}{}%
\end{pgfscope}%
\end{pgfscope}%
\begin{pgfscope}%
\pgftext[x=0.566911in,y=0.700846in,right,]{\rmfamily\fontsize{8.000000}{9.600000}\selectfont 0}%
\end{pgfscope}%
\begin{pgfscope}%
\pgfsetbuttcap%
\pgfsetroundjoin%
\definecolor{currentfill}{rgb}{0.000000,0.000000,0.000000}%
\pgfsetfillcolor{currentfill}%
\pgfsetlinewidth{0.501875pt}%
\definecolor{currentstroke}{rgb}{0.000000,0.000000,0.000000}%
\pgfsetstrokecolor{currentstroke}%
\pgfsetdash{}{0pt}%
\pgfsys@defobject{currentmarker}{\pgfqpoint{0.000000in}{0.000000in}}{\pgfqpoint{0.069444in}{0.000000in}}{%
\pgfpathmoveto{\pgfqpoint{0.000000in}{0.000000in}}%
\pgfpathlineto{\pgfqpoint{0.069444in}{0.000000in}}%
\pgfusepath{stroke,fill}%
}%
\begin{pgfscope}%
\pgfsys@transformshift{0.636356in}{0.827182in}%
\pgfsys@useobject{currentmarker}{}%
\end{pgfscope}%
\end{pgfscope}%
\begin{pgfscope}%
\pgfsetbuttcap%
\pgfsetroundjoin%
\definecolor{currentfill}{rgb}{0.000000,0.000000,0.000000}%
\pgfsetfillcolor{currentfill}%
\pgfsetlinewidth{0.501875pt}%
\definecolor{currentstroke}{rgb}{0.000000,0.000000,0.000000}%
\pgfsetstrokecolor{currentstroke}%
\pgfsetdash{}{0pt}%
\pgfsys@defobject{currentmarker}{\pgfqpoint{-0.069444in}{0.000000in}}{\pgfqpoint{0.000000in}{0.000000in}}{%
\pgfpathmoveto{\pgfqpoint{0.000000in}{0.000000in}}%
\pgfpathlineto{\pgfqpoint{-0.069444in}{0.000000in}}%
\pgfusepath{stroke,fill}%
}%
\begin{pgfscope}%
\pgfsys@transformshift{2.569584in}{0.827182in}%
\pgfsys@useobject{currentmarker}{}%
\end{pgfscope}%
\end{pgfscope}%
\begin{pgfscope}%
\pgftext[x=0.566911in,y=0.827182in,right,]{\rmfamily\fontsize{8.000000}{9.600000}\selectfont 200}%
\end{pgfscope}%
\begin{pgfscope}%
\pgfsetbuttcap%
\pgfsetroundjoin%
\definecolor{currentfill}{rgb}{0.000000,0.000000,0.000000}%
\pgfsetfillcolor{currentfill}%
\pgfsetlinewidth{0.501875pt}%
\definecolor{currentstroke}{rgb}{0.000000,0.000000,0.000000}%
\pgfsetstrokecolor{currentstroke}%
\pgfsetdash{}{0pt}%
\pgfsys@defobject{currentmarker}{\pgfqpoint{0.000000in}{0.000000in}}{\pgfqpoint{0.069444in}{0.000000in}}{%
\pgfpathmoveto{\pgfqpoint{0.000000in}{0.000000in}}%
\pgfpathlineto{\pgfqpoint{0.069444in}{0.000000in}}%
\pgfusepath{stroke,fill}%
}%
\begin{pgfscope}%
\pgfsys@transformshift{0.636356in}{0.953518in}%
\pgfsys@useobject{currentmarker}{}%
\end{pgfscope}%
\end{pgfscope}%
\begin{pgfscope}%
\pgfsetbuttcap%
\pgfsetroundjoin%
\definecolor{currentfill}{rgb}{0.000000,0.000000,0.000000}%
\pgfsetfillcolor{currentfill}%
\pgfsetlinewidth{0.501875pt}%
\definecolor{currentstroke}{rgb}{0.000000,0.000000,0.000000}%
\pgfsetstrokecolor{currentstroke}%
\pgfsetdash{}{0pt}%
\pgfsys@defobject{currentmarker}{\pgfqpoint{-0.069444in}{0.000000in}}{\pgfqpoint{0.000000in}{0.000000in}}{%
\pgfpathmoveto{\pgfqpoint{0.000000in}{0.000000in}}%
\pgfpathlineto{\pgfqpoint{-0.069444in}{0.000000in}}%
\pgfusepath{stroke,fill}%
}%
\begin{pgfscope}%
\pgfsys@transformshift{2.569584in}{0.953518in}%
\pgfsys@useobject{currentmarker}{}%
\end{pgfscope}%
\end{pgfscope}%
\begin{pgfscope}%
\pgftext[x=0.566911in,y=0.953518in,right,]{\rmfamily\fontsize{8.000000}{9.600000}\selectfont 400}%
\end{pgfscope}%
\begin{pgfscope}%
\pgfsetbuttcap%
\pgfsetroundjoin%
\definecolor{currentfill}{rgb}{0.000000,0.000000,0.000000}%
\pgfsetfillcolor{currentfill}%
\pgfsetlinewidth{0.501875pt}%
\definecolor{currentstroke}{rgb}{0.000000,0.000000,0.000000}%
\pgfsetstrokecolor{currentstroke}%
\pgfsetdash{}{0pt}%
\pgfsys@defobject{currentmarker}{\pgfqpoint{0.000000in}{0.000000in}}{\pgfqpoint{0.069444in}{0.000000in}}{%
\pgfpathmoveto{\pgfqpoint{0.000000in}{0.000000in}}%
\pgfpathlineto{\pgfqpoint{0.069444in}{0.000000in}}%
\pgfusepath{stroke,fill}%
}%
\begin{pgfscope}%
\pgfsys@transformshift{0.636356in}{1.079854in}%
\pgfsys@useobject{currentmarker}{}%
\end{pgfscope}%
\end{pgfscope}%
\begin{pgfscope}%
\pgfsetbuttcap%
\pgfsetroundjoin%
\definecolor{currentfill}{rgb}{0.000000,0.000000,0.000000}%
\pgfsetfillcolor{currentfill}%
\pgfsetlinewidth{0.501875pt}%
\definecolor{currentstroke}{rgb}{0.000000,0.000000,0.000000}%
\pgfsetstrokecolor{currentstroke}%
\pgfsetdash{}{0pt}%
\pgfsys@defobject{currentmarker}{\pgfqpoint{-0.069444in}{0.000000in}}{\pgfqpoint{0.000000in}{0.000000in}}{%
\pgfpathmoveto{\pgfqpoint{0.000000in}{0.000000in}}%
\pgfpathlineto{\pgfqpoint{-0.069444in}{0.000000in}}%
\pgfusepath{stroke,fill}%
}%
\begin{pgfscope}%
\pgfsys@transformshift{2.569584in}{1.079854in}%
\pgfsys@useobject{currentmarker}{}%
\end{pgfscope}%
\end{pgfscope}%
\begin{pgfscope}%
\pgftext[x=0.566911in,y=1.079854in,right,]{\rmfamily\fontsize{8.000000}{9.600000}\selectfont 600}%
\end{pgfscope}%
\begin{pgfscope}%
\pgfsetbuttcap%
\pgfsetroundjoin%
\definecolor{currentfill}{rgb}{0.000000,0.000000,0.000000}%
\pgfsetfillcolor{currentfill}%
\pgfsetlinewidth{0.501875pt}%
\definecolor{currentstroke}{rgb}{0.000000,0.000000,0.000000}%
\pgfsetstrokecolor{currentstroke}%
\pgfsetdash{}{0pt}%
\pgfsys@defobject{currentmarker}{\pgfqpoint{0.000000in}{0.000000in}}{\pgfqpoint{0.069444in}{0.000000in}}{%
\pgfpathmoveto{\pgfqpoint{0.000000in}{0.000000in}}%
\pgfpathlineto{\pgfqpoint{0.069444in}{0.000000in}}%
\pgfusepath{stroke,fill}%
}%
\begin{pgfscope}%
\pgfsys@transformshift{0.636356in}{1.206190in}%
\pgfsys@useobject{currentmarker}{}%
\end{pgfscope}%
\end{pgfscope}%
\begin{pgfscope}%
\pgfsetbuttcap%
\pgfsetroundjoin%
\definecolor{currentfill}{rgb}{0.000000,0.000000,0.000000}%
\pgfsetfillcolor{currentfill}%
\pgfsetlinewidth{0.501875pt}%
\definecolor{currentstroke}{rgb}{0.000000,0.000000,0.000000}%
\pgfsetstrokecolor{currentstroke}%
\pgfsetdash{}{0pt}%
\pgfsys@defobject{currentmarker}{\pgfqpoint{-0.069444in}{0.000000in}}{\pgfqpoint{0.000000in}{0.000000in}}{%
\pgfpathmoveto{\pgfqpoint{0.000000in}{0.000000in}}%
\pgfpathlineto{\pgfqpoint{-0.069444in}{0.000000in}}%
\pgfusepath{stroke,fill}%
}%
\begin{pgfscope}%
\pgfsys@transformshift{2.569584in}{1.206190in}%
\pgfsys@useobject{currentmarker}{}%
\end{pgfscope}%
\end{pgfscope}%
\begin{pgfscope}%
\pgftext[x=0.566911in,y=1.206190in,right,]{\rmfamily\fontsize{8.000000}{9.600000}\selectfont 800}%
\end{pgfscope}%
\begin{pgfscope}%
\pgfsetbuttcap%
\pgfsetroundjoin%
\definecolor{currentfill}{rgb}{0.000000,0.000000,0.000000}%
\pgfsetfillcolor{currentfill}%
\pgfsetlinewidth{0.501875pt}%
\definecolor{currentstroke}{rgb}{0.000000,0.000000,0.000000}%
\pgfsetstrokecolor{currentstroke}%
\pgfsetdash{}{0pt}%
\pgfsys@defobject{currentmarker}{\pgfqpoint{0.000000in}{0.000000in}}{\pgfqpoint{0.069444in}{0.000000in}}{%
\pgfpathmoveto{\pgfqpoint{0.000000in}{0.000000in}}%
\pgfpathlineto{\pgfqpoint{0.069444in}{0.000000in}}%
\pgfusepath{stroke,fill}%
}%
\begin{pgfscope}%
\pgfsys@transformshift{0.636356in}{1.332526in}%
\pgfsys@useobject{currentmarker}{}%
\end{pgfscope}%
\end{pgfscope}%
\begin{pgfscope}%
\pgfsetbuttcap%
\pgfsetroundjoin%
\definecolor{currentfill}{rgb}{0.000000,0.000000,0.000000}%
\pgfsetfillcolor{currentfill}%
\pgfsetlinewidth{0.501875pt}%
\definecolor{currentstroke}{rgb}{0.000000,0.000000,0.000000}%
\pgfsetstrokecolor{currentstroke}%
\pgfsetdash{}{0pt}%
\pgfsys@defobject{currentmarker}{\pgfqpoint{-0.069444in}{0.000000in}}{\pgfqpoint{0.000000in}{0.000000in}}{%
\pgfpathmoveto{\pgfqpoint{0.000000in}{0.000000in}}%
\pgfpathlineto{\pgfqpoint{-0.069444in}{0.000000in}}%
\pgfusepath{stroke,fill}%
}%
\begin{pgfscope}%
\pgfsys@transformshift{2.569584in}{1.332526in}%
\pgfsys@useobject{currentmarker}{}%
\end{pgfscope}%
\end{pgfscope}%
\begin{pgfscope}%
\pgftext[x=0.566911in,y=1.332526in,right,]{\rmfamily\fontsize{8.000000}{9.600000}\selectfont 1000}%
\end{pgfscope}%
\begin{pgfscope}%
\pgfsetbuttcap%
\pgfsetroundjoin%
\definecolor{currentfill}{rgb}{0.000000,0.000000,0.000000}%
\pgfsetfillcolor{currentfill}%
\pgfsetlinewidth{0.501875pt}%
\definecolor{currentstroke}{rgb}{0.000000,0.000000,0.000000}%
\pgfsetstrokecolor{currentstroke}%
\pgfsetdash{}{0pt}%
\pgfsys@defobject{currentmarker}{\pgfqpoint{0.000000in}{0.000000in}}{\pgfqpoint{0.069444in}{0.000000in}}{%
\pgfpathmoveto{\pgfqpoint{0.000000in}{0.000000in}}%
\pgfpathlineto{\pgfqpoint{0.069444in}{0.000000in}}%
\pgfusepath{stroke,fill}%
}%
\begin{pgfscope}%
\pgfsys@transformshift{0.636356in}{1.458861in}%
\pgfsys@useobject{currentmarker}{}%
\end{pgfscope}%
\end{pgfscope}%
\begin{pgfscope}%
\pgfsetbuttcap%
\pgfsetroundjoin%
\definecolor{currentfill}{rgb}{0.000000,0.000000,0.000000}%
\pgfsetfillcolor{currentfill}%
\pgfsetlinewidth{0.501875pt}%
\definecolor{currentstroke}{rgb}{0.000000,0.000000,0.000000}%
\pgfsetstrokecolor{currentstroke}%
\pgfsetdash{}{0pt}%
\pgfsys@defobject{currentmarker}{\pgfqpoint{-0.069444in}{0.000000in}}{\pgfqpoint{0.000000in}{0.000000in}}{%
\pgfpathmoveto{\pgfqpoint{0.000000in}{0.000000in}}%
\pgfpathlineto{\pgfqpoint{-0.069444in}{0.000000in}}%
\pgfusepath{stroke,fill}%
}%
\begin{pgfscope}%
\pgfsys@transformshift{2.569584in}{1.458861in}%
\pgfsys@useobject{currentmarker}{}%
\end{pgfscope}%
\end{pgfscope}%
\begin{pgfscope}%
\pgftext[x=0.566911in,y=1.458861in,right,]{\rmfamily\fontsize{8.000000}{9.600000}\selectfont 1200}%
\end{pgfscope}%
\begin{pgfscope}%
\pgfsetbuttcap%
\pgfsetroundjoin%
\definecolor{currentfill}{rgb}{0.000000,0.000000,0.000000}%
\pgfsetfillcolor{currentfill}%
\pgfsetlinewidth{0.501875pt}%
\definecolor{currentstroke}{rgb}{0.000000,0.000000,0.000000}%
\pgfsetstrokecolor{currentstroke}%
\pgfsetdash{}{0pt}%
\pgfsys@defobject{currentmarker}{\pgfqpoint{0.000000in}{0.000000in}}{\pgfqpoint{0.069444in}{0.000000in}}{%
\pgfpathmoveto{\pgfqpoint{0.000000in}{0.000000in}}%
\pgfpathlineto{\pgfqpoint{0.069444in}{0.000000in}}%
\pgfusepath{stroke,fill}%
}%
\begin{pgfscope}%
\pgfsys@transformshift{0.636356in}{1.585197in}%
\pgfsys@useobject{currentmarker}{}%
\end{pgfscope}%
\end{pgfscope}%
\begin{pgfscope}%
\pgfsetbuttcap%
\pgfsetroundjoin%
\definecolor{currentfill}{rgb}{0.000000,0.000000,0.000000}%
\pgfsetfillcolor{currentfill}%
\pgfsetlinewidth{0.501875pt}%
\definecolor{currentstroke}{rgb}{0.000000,0.000000,0.000000}%
\pgfsetstrokecolor{currentstroke}%
\pgfsetdash{}{0pt}%
\pgfsys@defobject{currentmarker}{\pgfqpoint{-0.069444in}{0.000000in}}{\pgfqpoint{0.000000in}{0.000000in}}{%
\pgfpathmoveto{\pgfqpoint{0.000000in}{0.000000in}}%
\pgfpathlineto{\pgfqpoint{-0.069444in}{0.000000in}}%
\pgfusepath{stroke,fill}%
}%
\begin{pgfscope}%
\pgfsys@transformshift{2.569584in}{1.585197in}%
\pgfsys@useobject{currentmarker}{}%
\end{pgfscope}%
\end{pgfscope}%
\begin{pgfscope}%
\pgftext[x=0.566911in,y=1.585197in,right,]{\rmfamily\fontsize{8.000000}{9.600000}\selectfont 1400}%
\end{pgfscope}%
\begin{pgfscope}%
\pgfsetbuttcap%
\pgfsetroundjoin%
\definecolor{currentfill}{rgb}{0.000000,0.000000,0.000000}%
\pgfsetfillcolor{currentfill}%
\pgfsetlinewidth{0.501875pt}%
\definecolor{currentstroke}{rgb}{0.000000,0.000000,0.000000}%
\pgfsetstrokecolor{currentstroke}%
\pgfsetdash{}{0pt}%
\pgfsys@defobject{currentmarker}{\pgfqpoint{0.000000in}{0.000000in}}{\pgfqpoint{0.069444in}{0.000000in}}{%
\pgfpathmoveto{\pgfqpoint{0.000000in}{0.000000in}}%
\pgfpathlineto{\pgfqpoint{0.069444in}{0.000000in}}%
\pgfusepath{stroke,fill}%
}%
\begin{pgfscope}%
\pgfsys@transformshift{0.636356in}{1.711533in}%
\pgfsys@useobject{currentmarker}{}%
\end{pgfscope}%
\end{pgfscope}%
\begin{pgfscope}%
\pgfsetbuttcap%
\pgfsetroundjoin%
\definecolor{currentfill}{rgb}{0.000000,0.000000,0.000000}%
\pgfsetfillcolor{currentfill}%
\pgfsetlinewidth{0.501875pt}%
\definecolor{currentstroke}{rgb}{0.000000,0.000000,0.000000}%
\pgfsetstrokecolor{currentstroke}%
\pgfsetdash{}{0pt}%
\pgfsys@defobject{currentmarker}{\pgfqpoint{-0.069444in}{0.000000in}}{\pgfqpoint{0.000000in}{0.000000in}}{%
\pgfpathmoveto{\pgfqpoint{0.000000in}{0.000000in}}%
\pgfpathlineto{\pgfqpoint{-0.069444in}{0.000000in}}%
\pgfusepath{stroke,fill}%
}%
\begin{pgfscope}%
\pgfsys@transformshift{2.569584in}{1.711533in}%
\pgfsys@useobject{currentmarker}{}%
\end{pgfscope}%
\end{pgfscope}%
\begin{pgfscope}%
\pgftext[x=0.566911in,y=1.711533in,right,]{\rmfamily\fontsize{8.000000}{9.600000}\selectfont 1600}%
\end{pgfscope}%
\begin{pgfscope}%
\pgfsetbuttcap%
\pgfsetroundjoin%
\definecolor{currentfill}{rgb}{0.000000,0.000000,0.000000}%
\pgfsetfillcolor{currentfill}%
\pgfsetlinewidth{0.501875pt}%
\definecolor{currentstroke}{rgb}{0.000000,0.000000,0.000000}%
\pgfsetstrokecolor{currentstroke}%
\pgfsetdash{}{0pt}%
\pgfsys@defobject{currentmarker}{\pgfqpoint{0.000000in}{0.000000in}}{\pgfqpoint{0.069444in}{0.000000in}}{%
\pgfpathmoveto{\pgfqpoint{0.000000in}{0.000000in}}%
\pgfpathlineto{\pgfqpoint{0.069444in}{0.000000in}}%
\pgfusepath{stroke,fill}%
}%
\begin{pgfscope}%
\pgfsys@transformshift{0.636356in}{1.837869in}%
\pgfsys@useobject{currentmarker}{}%
\end{pgfscope}%
\end{pgfscope}%
\begin{pgfscope}%
\pgfsetbuttcap%
\pgfsetroundjoin%
\definecolor{currentfill}{rgb}{0.000000,0.000000,0.000000}%
\pgfsetfillcolor{currentfill}%
\pgfsetlinewidth{0.501875pt}%
\definecolor{currentstroke}{rgb}{0.000000,0.000000,0.000000}%
\pgfsetstrokecolor{currentstroke}%
\pgfsetdash{}{0pt}%
\pgfsys@defobject{currentmarker}{\pgfqpoint{-0.069444in}{0.000000in}}{\pgfqpoint{0.000000in}{0.000000in}}{%
\pgfpathmoveto{\pgfqpoint{0.000000in}{0.000000in}}%
\pgfpathlineto{\pgfqpoint{-0.069444in}{0.000000in}}%
\pgfusepath{stroke,fill}%
}%
\begin{pgfscope}%
\pgfsys@transformshift{2.569584in}{1.837869in}%
\pgfsys@useobject{currentmarker}{}%
\end{pgfscope}%
\end{pgfscope}%
\begin{pgfscope}%
\pgftext[x=0.566911in,y=1.837869in,right,]{\rmfamily\fontsize{8.000000}{9.600000}\selectfont 1800}%
\end{pgfscope}%
\begin{pgfscope}%
\pgftext[x=0.214698in,y=1.269358in,,bottom,rotate=90.000000]{\rmfamily\fontsize{9.000000}{10.800000}\selectfont Candidates}%
\end{pgfscope}%
\begin{pgfscope}%
\pgfsetrectcap%
\pgfsetroundjoin%
\pgfsetlinewidth{1.003750pt}%
\definecolor{currentstroke}{rgb}{1.000000,0.000000,0.000000}%
\pgfsetstrokecolor{currentstroke}%
\pgfsetdash{}{0pt}%
\pgfpathmoveto{\pgfqpoint{0.636356in}{0.719388in}}%
\pgfpathlineto{\pgfqpoint{0.704359in}{0.727636in}}%
\pgfpathlineto{\pgfqpoint{0.752932in}{0.735203in}}%
\pgfpathlineto{\pgfqpoint{0.811221in}{0.746715in}}%
\pgfpathlineto{\pgfqpoint{0.859794in}{0.758899in}}%
\pgfpathlineto{\pgfqpoint{0.898653in}{0.770768in}}%
\pgfpathlineto{\pgfqpoint{0.927797in}{0.781179in}}%
\pgfpathlineto{\pgfqpoint{0.966656in}{0.797735in}}%
\pgfpathlineto{\pgfqpoint{1.005515in}{0.818005in}}%
\pgfpathlineto{\pgfqpoint{1.034659in}{0.836326in}}%
\pgfpathlineto{\pgfqpoint{1.063803in}{0.857564in}}%
\pgfpathlineto{\pgfqpoint{1.092948in}{0.882641in}}%
\pgfpathlineto{\pgfqpoint{1.112377in}{0.901557in}}%
\pgfpathlineto{\pgfqpoint{1.131806in}{0.922465in}}%
\pgfpathlineto{\pgfqpoint{1.151236in}{0.945503in}}%
\pgfpathlineto{\pgfqpoint{1.170665in}{0.970791in}}%
\pgfpathlineto{\pgfqpoint{1.190095in}{0.998424in}}%
\pgfpathlineto{\pgfqpoint{1.209524in}{1.028464in}}%
\pgfpathlineto{\pgfqpoint{1.238668in}{1.077949in}}%
\pgfpathlineto{\pgfqpoint{1.267812in}{1.132947in}}%
\pgfpathlineto{\pgfqpoint{1.296957in}{1.192572in}}%
\pgfpathlineto{\pgfqpoint{1.335815in}{1.278233in}}%
\pgfpathlineto{\pgfqpoint{1.452392in}{1.542681in}}%
\pgfpathlineto{\pgfqpoint{1.481536in}{1.599897in}}%
\pgfpathlineto{\pgfqpoint{1.500966in}{1.633761in}}%
\pgfpathlineto{\pgfqpoint{1.520395in}{1.663459in}}%
\pgfpathlineto{\pgfqpoint{1.539824in}{1.688433in}}%
\pgfpathlineto{\pgfqpoint{1.559254in}{1.708199in}}%
\pgfpathlineto{\pgfqpoint{1.568969in}{1.716020in}}%
\pgfpathlineto{\pgfqpoint{1.578683in}{1.722369in}}%
\pgfpathlineto{\pgfqpoint{1.588398in}{1.727268in}}%
\pgfpathlineto{\pgfqpoint{1.598113in}{1.730658in}}%
\pgfpathlineto{\pgfqpoint{1.607827in}{1.732535in}}%
\pgfpathlineto{\pgfqpoint{1.617542in}{1.732898in}}%
\pgfpathlineto{\pgfqpoint{1.627257in}{1.731715in}}%
\pgfpathlineto{\pgfqpoint{1.636972in}{1.729045in}}%
\pgfpathlineto{\pgfqpoint{1.646686in}{1.724825in}}%
\pgfpathlineto{\pgfqpoint{1.656401in}{1.719175in}}%
\pgfpathlineto{\pgfqpoint{1.675831in}{1.703487in}}%
\pgfpathlineto{\pgfqpoint{1.695260in}{1.682295in}}%
\pgfpathlineto{\pgfqpoint{1.714689in}{1.656013in}}%
\pgfpathlineto{\pgfqpoint{1.734119in}{1.625146in}}%
\pgfpathlineto{\pgfqpoint{1.753548in}{1.590270in}}%
\pgfpathlineto{\pgfqpoint{1.782692in}{1.531791in}}%
\pgfpathlineto{\pgfqpoint{1.811837in}{1.468029in}}%
\pgfpathlineto{\pgfqpoint{1.879840in}{1.310886in}}%
\pgfpathlineto{\pgfqpoint{1.918698in}{1.223348in}}%
\pgfpathlineto{\pgfqpoint{1.947843in}{1.161470in}}%
\pgfpathlineto{\pgfqpoint{1.976987in}{1.104246in}}%
\pgfpathlineto{\pgfqpoint{2.006131in}{1.051901in}}%
\pgfpathlineto{\pgfqpoint{2.035275in}{1.005210in}}%
\pgfpathlineto{\pgfqpoint{2.054704in}{0.977020in}}%
\pgfpathlineto{\pgfqpoint{2.074134in}{0.951192in}}%
\pgfpathlineto{\pgfqpoint{2.093563in}{0.927640in}}%
\pgfpathlineto{\pgfqpoint{2.112993in}{0.906248in}}%
\pgfpathlineto{\pgfqpoint{2.132422in}{0.886880in}}%
\pgfpathlineto{\pgfqpoint{2.151852in}{0.869390in}}%
\pgfpathlineto{\pgfqpoint{2.180996in}{0.846231in}}%
\pgfpathlineto{\pgfqpoint{2.210140in}{0.826645in}}%
\pgfpathlineto{\pgfqpoint{2.239284in}{0.809731in}}%
\pgfpathlineto{\pgfqpoint{2.278143in}{0.790992in}}%
\pgfpathlineto{\pgfqpoint{2.317002in}{0.775640in}}%
\pgfpathlineto{\pgfqpoint{2.355861in}{0.762950in}}%
\pgfpathlineto{\pgfqpoint{2.404434in}{0.750044in}}%
\pgfpathlineto{\pgfqpoint{2.453008in}{0.739583in}}%
\pgfpathlineto{\pgfqpoint{2.511296in}{0.729670in}}%
\pgfpathlineto{\pgfqpoint{2.569584in}{0.721999in}}%
\pgfpathlineto{\pgfqpoint{2.569584in}{0.721999in}}%
\pgfusepath{stroke}%
\end{pgfscope}%
\begin{pgfscope}%
\pgfpathrectangle{\pgfqpoint{0.636356in}{0.700846in}}{\pgfqpoint{1.933229in}{1.137023in}} %
\pgfusepath{clip}%
\pgfsetbuttcap%
\pgfsetroundjoin%
\pgfsetlinewidth{0.501875pt}%
\definecolor{currentstroke}{rgb}{0.000000,0.000000,0.000000}%
\pgfsetstrokecolor{currentstroke}%
\pgfsetdash{}{0pt}%
\pgfpathmoveto{\pgfqpoint{0.646022in}{0.722717in}}%
\pgfpathlineto{\pgfqpoint{0.646022in}{0.731454in}}%
\pgfusepath{stroke}%
\end{pgfscope}%
\begin{pgfscope}%
\pgfpathrectangle{\pgfqpoint{0.636356in}{0.700846in}}{\pgfqpoint{1.933229in}{1.137023in}} %
\pgfusepath{clip}%
\pgfsetbuttcap%
\pgfsetroundjoin%
\pgfsetlinewidth{0.501875pt}%
\definecolor{currentstroke}{rgb}{0.000000,0.000000,0.000000}%
\pgfsetstrokecolor{currentstroke}%
\pgfsetdash{}{0pt}%
\pgfpathmoveto{\pgfqpoint{0.665354in}{0.714640in}}%
\pgfpathlineto{\pgfqpoint{0.665354in}{0.721856in}}%
\pgfusepath{stroke}%
\end{pgfscope}%
\begin{pgfscope}%
\pgfpathrectangle{\pgfqpoint{0.636356in}{0.700846in}}{\pgfqpoint{1.933229in}{1.137023in}} %
\pgfusepath{clip}%
\pgfsetbuttcap%
\pgfsetroundjoin%
\pgfsetlinewidth{0.501875pt}%
\definecolor{currentstroke}{rgb}{0.000000,0.000000,0.000000}%
\pgfsetstrokecolor{currentstroke}%
\pgfsetdash{}{0pt}%
\pgfpathmoveto{\pgfqpoint{0.684687in}{0.731507in}}%
\pgfpathlineto{\pgfqpoint{0.684687in}{0.741607in}}%
\pgfusepath{stroke}%
\end{pgfscope}%
\begin{pgfscope}%
\pgfpathrectangle{\pgfqpoint{0.636356in}{0.700846in}}{\pgfqpoint{1.933229in}{1.137023in}} %
\pgfusepath{clip}%
\pgfsetbuttcap%
\pgfsetroundjoin%
\pgfsetlinewidth{0.501875pt}%
\definecolor{currentstroke}{rgb}{0.000000,0.000000,0.000000}%
\pgfsetstrokecolor{currentstroke}%
\pgfsetdash{}{0pt}%
\pgfpathmoveto{\pgfqpoint{0.704019in}{0.727392in}}%
\pgfpathlineto{\pgfqpoint{0.704019in}{0.736882in}}%
\pgfusepath{stroke}%
\end{pgfscope}%
\begin{pgfscope}%
\pgfpathrectangle{\pgfqpoint{0.636356in}{0.700846in}}{\pgfqpoint{1.933229in}{1.137023in}} %
\pgfusepath{clip}%
\pgfsetbuttcap%
\pgfsetroundjoin%
\pgfsetlinewidth{0.501875pt}%
\definecolor{currentstroke}{rgb}{0.000000,0.000000,0.000000}%
\pgfsetstrokecolor{currentstroke}%
\pgfsetdash{}{0pt}%
\pgfpathmoveto{\pgfqpoint{0.723351in}{0.736234in}}%
\pgfpathlineto{\pgfqpoint{0.723351in}{0.746985in}}%
\pgfusepath{stroke}%
\end{pgfscope}%
\begin{pgfscope}%
\pgfpathrectangle{\pgfqpoint{0.636356in}{0.700846in}}{\pgfqpoint{1.933229in}{1.137023in}} %
\pgfusepath{clip}%
\pgfsetbuttcap%
\pgfsetroundjoin%
\pgfsetlinewidth{0.501875pt}%
\definecolor{currentstroke}{rgb}{0.000000,0.000000,0.000000}%
\pgfsetstrokecolor{currentstroke}%
\pgfsetdash{}{0pt}%
\pgfpathmoveto{\pgfqpoint{0.742683in}{0.734459in}}%
\pgfpathlineto{\pgfqpoint{0.742683in}{0.744971in}}%
\pgfusepath{stroke}%
\end{pgfscope}%
\begin{pgfscope}%
\pgfpathrectangle{\pgfqpoint{0.636356in}{0.700846in}}{\pgfqpoint{1.933229in}{1.137023in}} %
\pgfusepath{clip}%
\pgfsetbuttcap%
\pgfsetroundjoin%
\pgfsetlinewidth{0.501875pt}%
\definecolor{currentstroke}{rgb}{0.000000,0.000000,0.000000}%
\pgfsetstrokecolor{currentstroke}%
\pgfsetdash{}{0pt}%
\pgfpathmoveto{\pgfqpoint{0.762016in}{0.731507in}}%
\pgfpathlineto{\pgfqpoint{0.762016in}{0.741607in}}%
\pgfusepath{stroke}%
\end{pgfscope}%
\begin{pgfscope}%
\pgfpathrectangle{\pgfqpoint{0.636356in}{0.700846in}}{\pgfqpoint{1.933229in}{1.137023in}} %
\pgfusepath{clip}%
\pgfsetbuttcap%
\pgfsetroundjoin%
\pgfsetlinewidth{0.501875pt}%
\definecolor{currentstroke}{rgb}{0.000000,0.000000,0.000000}%
\pgfsetstrokecolor{currentstroke}%
\pgfsetdash{}{0pt}%
\pgfpathmoveto{\pgfqpoint{0.781348in}{0.738604in}}%
\pgfpathlineto{\pgfqpoint{0.781348in}{0.749667in}}%
\pgfusepath{stroke}%
\end{pgfscope}%
\begin{pgfscope}%
\pgfpathrectangle{\pgfqpoint{0.636356in}{0.700846in}}{\pgfqpoint{1.933229in}{1.137023in}} %
\pgfusepath{clip}%
\pgfsetbuttcap%
\pgfsetroundjoin%
\pgfsetlinewidth{0.501875pt}%
\definecolor{currentstroke}{rgb}{0.000000,0.000000,0.000000}%
\pgfsetstrokecolor{currentstroke}%
\pgfsetdash{}{0pt}%
\pgfpathmoveto{\pgfqpoint{0.800680in}{0.735050in}}%
\pgfpathlineto{\pgfqpoint{0.800680in}{0.745643in}}%
\pgfusepath{stroke}%
\end{pgfscope}%
\begin{pgfscope}%
\pgfpathrectangle{\pgfqpoint{0.636356in}{0.700846in}}{\pgfqpoint{1.933229in}{1.137023in}} %
\pgfusepath{clip}%
\pgfsetbuttcap%
\pgfsetroundjoin%
\pgfsetlinewidth{0.501875pt}%
\definecolor{currentstroke}{rgb}{0.000000,0.000000,0.000000}%
\pgfsetstrokecolor{currentstroke}%
\pgfsetdash{}{0pt}%
\pgfpathmoveto{\pgfqpoint{0.820013in}{0.742169in}}%
\pgfpathlineto{\pgfqpoint{0.820013in}{0.753681in}}%
\pgfusepath{stroke}%
\end{pgfscope}%
\begin{pgfscope}%
\pgfpathrectangle{\pgfqpoint{0.636356in}{0.700846in}}{\pgfqpoint{1.933229in}{1.137023in}} %
\pgfusepath{clip}%
\pgfsetbuttcap%
\pgfsetroundjoin%
\pgfsetlinewidth{0.501875pt}%
\definecolor{currentstroke}{rgb}{0.000000,0.000000,0.000000}%
\pgfsetstrokecolor{currentstroke}%
\pgfsetdash{}{0pt}%
\pgfpathmoveto{\pgfqpoint{0.839345in}{0.739198in}}%
\pgfpathlineto{\pgfqpoint{0.839345in}{0.750336in}}%
\pgfusepath{stroke}%
\end{pgfscope}%
\begin{pgfscope}%
\pgfpathrectangle{\pgfqpoint{0.636356in}{0.700846in}}{\pgfqpoint{1.933229in}{1.137023in}} %
\pgfusepath{clip}%
\pgfsetbuttcap%
\pgfsetroundjoin%
\pgfsetlinewidth{0.501875pt}%
\definecolor{currentstroke}{rgb}{0.000000,0.000000,0.000000}%
\pgfsetstrokecolor{currentstroke}%
\pgfsetdash{}{0pt}%
\pgfpathmoveto{\pgfqpoint{0.858677in}{0.753512in}}%
\pgfpathlineto{\pgfqpoint{0.858677in}{0.766338in}}%
\pgfusepath{stroke}%
\end{pgfscope}%
\begin{pgfscope}%
\pgfpathrectangle{\pgfqpoint{0.636356in}{0.700846in}}{\pgfqpoint{1.933229in}{1.137023in}} %
\pgfusepath{clip}%
\pgfsetbuttcap%
\pgfsetroundjoin%
\pgfsetlinewidth{0.501875pt}%
\definecolor{currentstroke}{rgb}{0.000000,0.000000,0.000000}%
\pgfsetstrokecolor{currentstroke}%
\pgfsetdash{}{0pt}%
\pgfpathmoveto{\pgfqpoint{0.878009in}{0.762210in}}%
\pgfpathlineto{\pgfqpoint{0.878009in}{0.775294in}}%
\pgfusepath{stroke}%
\end{pgfscope}%
\begin{pgfscope}%
\pgfpathrectangle{\pgfqpoint{0.636356in}{0.700846in}}{\pgfqpoint{1.933229in}{1.137023in}} %
\pgfusepath{clip}%
\pgfsetbuttcap%
\pgfsetroundjoin%
\pgfsetlinewidth{0.501875pt}%
\definecolor{currentstroke}{rgb}{0.000000,0.000000,0.000000}%
\pgfsetstrokecolor{currentstroke}%
\pgfsetdash{}{0pt}%
\pgfpathmoveto{\pgfqpoint{0.897342in}{0.764616in}}%
\pgfpathlineto{\pgfqpoint{0.897342in}{0.777941in}}%
\pgfusepath{stroke}%
\end{pgfscope}%
\begin{pgfscope}%
\pgfpathrectangle{\pgfqpoint{0.636356in}{0.700846in}}{\pgfqpoint{1.933229in}{1.137023in}} %
\pgfusepath{clip}%
\pgfsetbuttcap%
\pgfsetroundjoin%
\pgfsetlinewidth{0.501875pt}%
\definecolor{currentstroke}{rgb}{0.000000,0.000000,0.000000}%
\pgfsetstrokecolor{currentstroke}%
\pgfsetdash{}{0pt}%
\pgfpathmoveto{\pgfqpoint{0.916674in}{0.774863in}}%
\pgfpathlineto{\pgfqpoint{0.916674in}{0.789171in}}%
\pgfusepath{stroke}%
\end{pgfscope}%
\begin{pgfscope}%
\pgfpathrectangle{\pgfqpoint{0.636356in}{0.700846in}}{\pgfqpoint{1.933229in}{1.137023in}} %
\pgfusepath{clip}%
\pgfsetbuttcap%
\pgfsetroundjoin%
\pgfsetlinewidth{0.501875pt}%
\definecolor{currentstroke}{rgb}{0.000000,0.000000,0.000000}%
\pgfsetstrokecolor{currentstroke}%
\pgfsetdash{}{0pt}%
\pgfpathmoveto{\pgfqpoint{0.936006in}{0.784537in}}%
\pgfpathlineto{\pgfqpoint{0.936006in}{0.799711in}}%
\pgfusepath{stroke}%
\end{pgfscope}%
\begin{pgfscope}%
\pgfpathrectangle{\pgfqpoint{0.636356in}{0.700846in}}{\pgfqpoint{1.933229in}{1.137023in}} %
\pgfusepath{clip}%
\pgfsetbuttcap%
\pgfsetroundjoin%
\pgfsetlinewidth{0.501875pt}%
\definecolor{currentstroke}{rgb}{0.000000,0.000000,0.000000}%
\pgfsetstrokecolor{currentstroke}%
\pgfsetdash{}{0pt}%
\pgfpathmoveto{\pgfqpoint{0.955339in}{0.793021in}}%
\pgfpathlineto{\pgfqpoint{0.955339in}{0.808914in}}%
\pgfusepath{stroke}%
\end{pgfscope}%
\begin{pgfscope}%
\pgfpathrectangle{\pgfqpoint{0.636356in}{0.700846in}}{\pgfqpoint{1.933229in}{1.137023in}} %
\pgfusepath{clip}%
\pgfsetbuttcap%
\pgfsetroundjoin%
\pgfsetlinewidth{0.501875pt}%
\definecolor{currentstroke}{rgb}{0.000000,0.000000,0.000000}%
\pgfsetstrokecolor{currentstroke}%
\pgfsetdash{}{0pt}%
\pgfpathmoveto{\pgfqpoint{0.974671in}{0.809425in}}%
\pgfpathlineto{\pgfqpoint{0.974671in}{0.826620in}}%
\pgfusepath{stroke}%
\end{pgfscope}%
\begin{pgfscope}%
\pgfpathrectangle{\pgfqpoint{0.636356in}{0.700846in}}{\pgfqpoint{1.933229in}{1.137023in}} %
\pgfusepath{clip}%
\pgfsetbuttcap%
\pgfsetroundjoin%
\pgfsetlinewidth{0.501875pt}%
\definecolor{currentstroke}{rgb}{0.000000,0.000000,0.000000}%
\pgfsetstrokecolor{currentstroke}%
\pgfsetdash{}{0pt}%
\pgfpathmoveto{\pgfqpoint{0.994003in}{0.795448in}}%
\pgfpathlineto{\pgfqpoint{0.994003in}{0.811540in}}%
\pgfusepath{stroke}%
\end{pgfscope}%
\begin{pgfscope}%
\pgfpathrectangle{\pgfqpoint{0.636356in}{0.700846in}}{\pgfqpoint{1.933229in}{1.137023in}} %
\pgfusepath{clip}%
\pgfsetbuttcap%
\pgfsetroundjoin%
\pgfsetlinewidth{0.501875pt}%
\definecolor{currentstroke}{rgb}{0.000000,0.000000,0.000000}%
\pgfsetstrokecolor{currentstroke}%
\pgfsetdash{}{0pt}%
\pgfpathmoveto{\pgfqpoint{1.013335in}{0.816731in}}%
\pgfpathlineto{\pgfqpoint{1.013335in}{0.834475in}}%
\pgfusepath{stroke}%
\end{pgfscope}%
\begin{pgfscope}%
\pgfpathrectangle{\pgfqpoint{0.636356in}{0.700846in}}{\pgfqpoint{1.933229in}{1.137023in}} %
\pgfusepath{clip}%
\pgfsetbuttcap%
\pgfsetroundjoin%
\pgfsetlinewidth{0.501875pt}%
\definecolor{currentstroke}{rgb}{0.000000,0.000000,0.000000}%
\pgfsetstrokecolor{currentstroke}%
\pgfsetdash{}{0pt}%
\pgfpathmoveto{\pgfqpoint{1.032668in}{0.813686in}}%
\pgfpathlineto{\pgfqpoint{1.032668in}{0.831203in}}%
\pgfusepath{stroke}%
\end{pgfscope}%
\begin{pgfscope}%
\pgfpathrectangle{\pgfqpoint{0.636356in}{0.700846in}}{\pgfqpoint{1.933229in}{1.137023in}} %
\pgfusepath{clip}%
\pgfsetbuttcap%
\pgfsetroundjoin%
\pgfsetlinewidth{0.501875pt}%
\definecolor{currentstroke}{rgb}{0.000000,0.000000,0.000000}%
\pgfsetstrokecolor{currentstroke}%
\pgfsetdash{}{0pt}%
\pgfpathmoveto{\pgfqpoint{1.052000in}{0.851536in}}%
\pgfpathlineto{\pgfqpoint{1.052000in}{0.871681in}}%
\pgfusepath{stroke}%
\end{pgfscope}%
\begin{pgfscope}%
\pgfpathrectangle{\pgfqpoint{0.636356in}{0.700846in}}{\pgfqpoint{1.933229in}{1.137023in}} %
\pgfusepath{clip}%
\pgfsetbuttcap%
\pgfsetroundjoin%
\pgfsetlinewidth{0.501875pt}%
\definecolor{currentstroke}{rgb}{0.000000,0.000000,0.000000}%
\pgfsetstrokecolor{currentstroke}%
\pgfsetdash{}{0pt}%
\pgfpathmoveto{\pgfqpoint{1.071332in}{0.853372in}}%
\pgfpathlineto{\pgfqpoint{1.071332in}{0.873635in}}%
\pgfusepath{stroke}%
\end{pgfscope}%
\begin{pgfscope}%
\pgfpathrectangle{\pgfqpoint{0.636356in}{0.700846in}}{\pgfqpoint{1.933229in}{1.137023in}} %
\pgfusepath{clip}%
\pgfsetbuttcap%
\pgfsetroundjoin%
\pgfsetlinewidth{0.501875pt}%
\definecolor{currentstroke}{rgb}{0.000000,0.000000,0.000000}%
\pgfsetstrokecolor{currentstroke}%
\pgfsetdash{}{0pt}%
\pgfpathmoveto{\pgfqpoint{1.090665in}{0.890153in}}%
\pgfpathlineto{\pgfqpoint{1.090665in}{0.912656in}}%
\pgfusepath{stroke}%
\end{pgfscope}%
\begin{pgfscope}%
\pgfpathrectangle{\pgfqpoint{0.636356in}{0.700846in}}{\pgfqpoint{1.933229in}{1.137023in}} %
\pgfusepath{clip}%
\pgfsetbuttcap%
\pgfsetroundjoin%
\pgfsetlinewidth{0.501875pt}%
\definecolor{currentstroke}{rgb}{0.000000,0.000000,0.000000}%
\pgfsetstrokecolor{currentstroke}%
\pgfsetdash{}{0pt}%
\pgfpathmoveto{\pgfqpoint{1.109997in}{0.911658in}}%
\pgfpathlineto{\pgfqpoint{1.109997in}{0.935369in}}%
\pgfusepath{stroke}%
\end{pgfscope}%
\begin{pgfscope}%
\pgfpathrectangle{\pgfqpoint{0.636356in}{0.700846in}}{\pgfqpoint{1.933229in}{1.137023in}} %
\pgfusepath{clip}%
\pgfsetbuttcap%
\pgfsetroundjoin%
\pgfsetlinewidth{0.501875pt}%
\definecolor{currentstroke}{rgb}{0.000000,0.000000,0.000000}%
\pgfsetstrokecolor{currentstroke}%
\pgfsetdash{}{0pt}%
\pgfpathmoveto{\pgfqpoint{1.129329in}{0.921498in}}%
\pgfpathlineto{\pgfqpoint{1.129329in}{0.945742in}}%
\pgfusepath{stroke}%
\end{pgfscope}%
\begin{pgfscope}%
\pgfpathrectangle{\pgfqpoint{0.636356in}{0.700846in}}{\pgfqpoint{1.933229in}{1.137023in}} %
\pgfusepath{clip}%
\pgfsetbuttcap%
\pgfsetroundjoin%
\pgfsetlinewidth{0.501875pt}%
\definecolor{currentstroke}{rgb}{0.000000,0.000000,0.000000}%
\pgfsetstrokecolor{currentstroke}%
\pgfsetdash{}{0pt}%
\pgfpathmoveto{\pgfqpoint{1.148661in}{0.915962in}}%
\pgfpathlineto{\pgfqpoint{1.148661in}{0.939908in}}%
\pgfusepath{stroke}%
\end{pgfscope}%
\begin{pgfscope}%
\pgfpathrectangle{\pgfqpoint{0.636356in}{0.700846in}}{\pgfqpoint{1.933229in}{1.137023in}} %
\pgfusepath{clip}%
\pgfsetbuttcap%
\pgfsetroundjoin%
\pgfsetlinewidth{0.501875pt}%
\definecolor{currentstroke}{rgb}{0.000000,0.000000,0.000000}%
\pgfsetstrokecolor{currentstroke}%
\pgfsetdash{}{0pt}%
\pgfpathmoveto{\pgfqpoint{1.167994in}{0.965847in}}%
\pgfpathlineto{\pgfqpoint{1.167994in}{0.992355in}}%
\pgfusepath{stroke}%
\end{pgfscope}%
\begin{pgfscope}%
\pgfpathrectangle{\pgfqpoint{0.636356in}{0.700846in}}{\pgfqpoint{1.933229in}{1.137023in}} %
\pgfusepath{clip}%
\pgfsetbuttcap%
\pgfsetroundjoin%
\pgfsetlinewidth{0.501875pt}%
\definecolor{currentstroke}{rgb}{0.000000,0.000000,0.000000}%
\pgfsetstrokecolor{currentstroke}%
\pgfsetdash{}{0pt}%
\pgfpathmoveto{\pgfqpoint{1.187326in}{0.962148in}}%
\pgfpathlineto{\pgfqpoint{1.187326in}{0.988474in}}%
\pgfusepath{stroke}%
\end{pgfscope}%
\begin{pgfscope}%
\pgfpathrectangle{\pgfqpoint{0.636356in}{0.700846in}}{\pgfqpoint{1.933229in}{1.137023in}} %
\pgfusepath{clip}%
\pgfsetbuttcap%
\pgfsetroundjoin%
\pgfsetlinewidth{0.501875pt}%
\definecolor{currentstroke}{rgb}{0.000000,0.000000,0.000000}%
\pgfsetstrokecolor{currentstroke}%
\pgfsetdash{}{0pt}%
\pgfpathmoveto{\pgfqpoint{1.206658in}{1.002256in}}%
\pgfpathlineto{\pgfqpoint{1.206658in}{1.030484in}}%
\pgfusepath{stroke}%
\end{pgfscope}%
\begin{pgfscope}%
\pgfpathrectangle{\pgfqpoint{0.636356in}{0.700846in}}{\pgfqpoint{1.933229in}{1.137023in}} %
\pgfusepath{clip}%
\pgfsetbuttcap%
\pgfsetroundjoin%
\pgfsetlinewidth{0.501875pt}%
\definecolor{currentstroke}{rgb}{0.000000,0.000000,0.000000}%
\pgfsetstrokecolor{currentstroke}%
\pgfsetdash{}{0pt}%
\pgfpathmoveto{\pgfqpoint{1.225991in}{1.052939in}}%
\pgfpathlineto{\pgfqpoint{1.225991in}{1.083397in}}%
\pgfusepath{stroke}%
\end{pgfscope}%
\begin{pgfscope}%
\pgfpathrectangle{\pgfqpoint{0.636356in}{0.700846in}}{\pgfqpoint{1.933229in}{1.137023in}} %
\pgfusepath{clip}%
\pgfsetbuttcap%
\pgfsetroundjoin%
\pgfsetlinewidth{0.501875pt}%
\definecolor{currentstroke}{rgb}{0.000000,0.000000,0.000000}%
\pgfsetstrokecolor{currentstroke}%
\pgfsetdash{}{0pt}%
\pgfpathmoveto{\pgfqpoint{1.245323in}{1.073977in}}%
\pgfpathlineto{\pgfqpoint{1.245323in}{1.105313in}}%
\pgfusepath{stroke}%
\end{pgfscope}%
\begin{pgfscope}%
\pgfpathrectangle{\pgfqpoint{0.636356in}{0.700846in}}{\pgfqpoint{1.933229in}{1.137023in}} %
\pgfusepath{clip}%
\pgfsetbuttcap%
\pgfsetroundjoin%
\pgfsetlinewidth{0.501875pt}%
\definecolor{currentstroke}{rgb}{0.000000,0.000000,0.000000}%
\pgfsetstrokecolor{currentstroke}%
\pgfsetdash{}{0pt}%
\pgfpathmoveto{\pgfqpoint{1.264655in}{1.065931in}}%
\pgfpathlineto{\pgfqpoint{1.264655in}{1.096935in}}%
\pgfusepath{stroke}%
\end{pgfscope}%
\begin{pgfscope}%
\pgfpathrectangle{\pgfqpoint{0.636356in}{0.700846in}}{\pgfqpoint{1.933229in}{1.137023in}} %
\pgfusepath{clip}%
\pgfsetbuttcap%
\pgfsetroundjoin%
\pgfsetlinewidth{0.501875pt}%
\definecolor{currentstroke}{rgb}{0.000000,0.000000,0.000000}%
\pgfsetstrokecolor{currentstroke}%
\pgfsetdash{}{0pt}%
\pgfpathmoveto{\pgfqpoint{1.283987in}{1.164442in}}%
\pgfpathlineto{\pgfqpoint{1.283987in}{1.199299in}}%
\pgfusepath{stroke}%
\end{pgfscope}%
\begin{pgfscope}%
\pgfpathrectangle{\pgfqpoint{0.636356in}{0.700846in}}{\pgfqpoint{1.933229in}{1.137023in}} %
\pgfusepath{clip}%
\pgfsetbuttcap%
\pgfsetroundjoin%
\pgfsetlinewidth{0.501875pt}%
\definecolor{currentstroke}{rgb}{0.000000,0.000000,0.000000}%
\pgfsetstrokecolor{currentstroke}%
\pgfsetdash{}{0pt}%
\pgfpathmoveto{\pgfqpoint{1.303320in}{1.175607in}}%
\pgfpathlineto{\pgfqpoint{1.303320in}{1.210874in}}%
\pgfusepath{stroke}%
\end{pgfscope}%
\begin{pgfscope}%
\pgfpathrectangle{\pgfqpoint{0.636356in}{0.700846in}}{\pgfqpoint{1.933229in}{1.137023in}} %
\pgfusepath{clip}%
\pgfsetbuttcap%
\pgfsetroundjoin%
\pgfsetlinewidth{0.501875pt}%
\definecolor{currentstroke}{rgb}{0.000000,0.000000,0.000000}%
\pgfsetstrokecolor{currentstroke}%
\pgfsetdash{}{0pt}%
\pgfpathmoveto{\pgfqpoint{1.322652in}{1.191739in}}%
\pgfpathlineto{\pgfqpoint{1.322652in}{1.227589in}}%
\pgfusepath{stroke}%
\end{pgfscope}%
\begin{pgfscope}%
\pgfpathrectangle{\pgfqpoint{0.636356in}{0.700846in}}{\pgfqpoint{1.933229in}{1.137023in}} %
\pgfusepath{clip}%
\pgfsetbuttcap%
\pgfsetroundjoin%
\pgfsetlinewidth{0.501875pt}%
\definecolor{currentstroke}{rgb}{0.000000,0.000000,0.000000}%
\pgfsetstrokecolor{currentstroke}%
\pgfsetdash{}{0pt}%
\pgfpathmoveto{\pgfqpoint{1.341984in}{1.293593in}}%
\pgfpathlineto{\pgfqpoint{1.341984in}{1.332925in}}%
\pgfusepath{stroke}%
\end{pgfscope}%
\begin{pgfscope}%
\pgfpathrectangle{\pgfqpoint{0.636356in}{0.700846in}}{\pgfqpoint{1.933229in}{1.137023in}} %
\pgfusepath{clip}%
\pgfsetbuttcap%
\pgfsetroundjoin%
\pgfsetlinewidth{0.501875pt}%
\definecolor{currentstroke}{rgb}{0.000000,0.000000,0.000000}%
\pgfsetstrokecolor{currentstroke}%
\pgfsetdash{}{0pt}%
\pgfpathmoveto{\pgfqpoint{1.361317in}{1.276814in}}%
\pgfpathlineto{\pgfqpoint{1.361317in}{1.315594in}}%
\pgfusepath{stroke}%
\end{pgfscope}%
\begin{pgfscope}%
\pgfpathrectangle{\pgfqpoint{0.636356in}{0.700846in}}{\pgfqpoint{1.933229in}{1.137023in}} %
\pgfusepath{clip}%
\pgfsetbuttcap%
\pgfsetroundjoin%
\pgfsetlinewidth{0.501875pt}%
\definecolor{currentstroke}{rgb}{0.000000,0.000000,0.000000}%
\pgfsetstrokecolor{currentstroke}%
\pgfsetdash{}{0pt}%
\pgfpathmoveto{\pgfqpoint{1.380649in}{1.355150in}}%
\pgfpathlineto{\pgfqpoint{1.380649in}{1.396441in}}%
\pgfusepath{stroke}%
\end{pgfscope}%
\begin{pgfscope}%
\pgfpathrectangle{\pgfqpoint{0.636356in}{0.700846in}}{\pgfqpoint{1.933229in}{1.137023in}} %
\pgfusepath{clip}%
\pgfsetbuttcap%
\pgfsetroundjoin%
\pgfsetlinewidth{0.501875pt}%
\definecolor{currentstroke}{rgb}{0.000000,0.000000,0.000000}%
\pgfsetstrokecolor{currentstroke}%
\pgfsetdash{}{0pt}%
\pgfpathmoveto{\pgfqpoint{1.399981in}{1.369457in}}%
\pgfpathlineto{\pgfqpoint{1.399981in}{1.411191in}}%
\pgfusepath{stroke}%
\end{pgfscope}%
\begin{pgfscope}%
\pgfpathrectangle{\pgfqpoint{0.636356in}{0.700846in}}{\pgfqpoint{1.933229in}{1.137023in}} %
\pgfusepath{clip}%
\pgfsetbuttcap%
\pgfsetroundjoin%
\pgfsetlinewidth{0.501875pt}%
\definecolor{currentstroke}{rgb}{0.000000,0.000000,0.000000}%
\pgfsetstrokecolor{currentstroke}%
\pgfsetdash{}{0pt}%
\pgfpathmoveto{\pgfqpoint{1.419313in}{1.463444in}}%
\pgfpathlineto{\pgfqpoint{1.419313in}{1.507972in}}%
\pgfusepath{stroke}%
\end{pgfscope}%
\begin{pgfscope}%
\pgfpathrectangle{\pgfqpoint{0.636356in}{0.700846in}}{\pgfqpoint{1.933229in}{1.137023in}} %
\pgfusepath{clip}%
\pgfsetbuttcap%
\pgfsetroundjoin%
\pgfsetlinewidth{0.501875pt}%
\definecolor{currentstroke}{rgb}{0.000000,0.000000,0.000000}%
\pgfsetstrokecolor{currentstroke}%
\pgfsetdash{}{0pt}%
\pgfpathmoveto{\pgfqpoint{1.438646in}{1.479636in}}%
\pgfpathlineto{\pgfqpoint{1.438646in}{1.524627in}}%
\pgfusepath{stroke}%
\end{pgfscope}%
\begin{pgfscope}%
\pgfpathrectangle{\pgfqpoint{0.636356in}{0.700846in}}{\pgfqpoint{1.933229in}{1.137023in}} %
\pgfusepath{clip}%
\pgfsetbuttcap%
\pgfsetroundjoin%
\pgfsetlinewidth{0.501875pt}%
\definecolor{currentstroke}{rgb}{0.000000,0.000000,0.000000}%
\pgfsetstrokecolor{currentstroke}%
\pgfsetdash{}{0pt}%
\pgfpathmoveto{\pgfqpoint{1.457978in}{1.524487in}}%
\pgfpathlineto{\pgfqpoint{1.457978in}{1.570738in}}%
\pgfusepath{stroke}%
\end{pgfscope}%
\begin{pgfscope}%
\pgfpathrectangle{\pgfqpoint{0.636356in}{0.700846in}}{\pgfqpoint{1.933229in}{1.137023in}} %
\pgfusepath{clip}%
\pgfsetbuttcap%
\pgfsetroundjoin%
\pgfsetlinewidth{0.501875pt}%
\definecolor{currentstroke}{rgb}{0.000000,0.000000,0.000000}%
\pgfsetstrokecolor{currentstroke}%
\pgfsetdash{}{0pt}%
\pgfpathmoveto{\pgfqpoint{1.477310in}{1.606757in}}%
\pgfpathlineto{\pgfqpoint{1.477310in}{1.655232in}}%
\pgfusepath{stroke}%
\end{pgfscope}%
\begin{pgfscope}%
\pgfpathrectangle{\pgfqpoint{0.636356in}{0.700846in}}{\pgfqpoint{1.933229in}{1.137023in}} %
\pgfusepath{clip}%
\pgfsetbuttcap%
\pgfsetroundjoin%
\pgfsetlinewidth{0.501875pt}%
\definecolor{currentstroke}{rgb}{0.000000,0.000000,0.000000}%
\pgfsetstrokecolor{currentstroke}%
\pgfsetdash{}{0pt}%
\pgfpathmoveto{\pgfqpoint{1.496643in}{1.594288in}}%
\pgfpathlineto{\pgfqpoint{1.496643in}{1.642433in}}%
\pgfusepath{stroke}%
\end{pgfscope}%
\begin{pgfscope}%
\pgfpathrectangle{\pgfqpoint{0.636356in}{0.700846in}}{\pgfqpoint{1.933229in}{1.137023in}} %
\pgfusepath{clip}%
\pgfsetbuttcap%
\pgfsetroundjoin%
\pgfsetlinewidth{0.501875pt}%
\definecolor{currentstroke}{rgb}{0.000000,0.000000,0.000000}%
\pgfsetstrokecolor{currentstroke}%
\pgfsetdash{}{0pt}%
\pgfpathmoveto{\pgfqpoint{1.515975in}{1.608627in}}%
\pgfpathlineto{\pgfqpoint{1.515975in}{1.657151in}}%
\pgfusepath{stroke}%
\end{pgfscope}%
\begin{pgfscope}%
\pgfpathrectangle{\pgfqpoint{0.636356in}{0.700846in}}{\pgfqpoint{1.933229in}{1.137023in}} %
\pgfusepath{clip}%
\pgfsetbuttcap%
\pgfsetroundjoin%
\pgfsetlinewidth{0.501875pt}%
\definecolor{currentstroke}{rgb}{0.000000,0.000000,0.000000}%
\pgfsetstrokecolor{currentstroke}%
\pgfsetdash{}{0pt}%
\pgfpathmoveto{\pgfqpoint{1.535307in}{1.653523in}}%
\pgfpathlineto{\pgfqpoint{1.535307in}{1.703217in}}%
\pgfusepath{stroke}%
\end{pgfscope}%
\begin{pgfscope}%
\pgfpathrectangle{\pgfqpoint{0.636356in}{0.700846in}}{\pgfqpoint{1.933229in}{1.137023in}} %
\pgfusepath{clip}%
\pgfsetbuttcap%
\pgfsetroundjoin%
\pgfsetlinewidth{0.501875pt}%
\definecolor{currentstroke}{rgb}{0.000000,0.000000,0.000000}%
\pgfsetstrokecolor{currentstroke}%
\pgfsetdash{}{0pt}%
\pgfpathmoveto{\pgfqpoint{1.554639in}{1.669115in}}%
\pgfpathlineto{\pgfqpoint{1.554639in}{1.719209in}}%
\pgfusepath{stroke}%
\end{pgfscope}%
\begin{pgfscope}%
\pgfpathrectangle{\pgfqpoint{0.636356in}{0.700846in}}{\pgfqpoint{1.933229in}{1.137023in}} %
\pgfusepath{clip}%
\pgfsetbuttcap%
\pgfsetroundjoin%
\pgfsetlinewidth{0.501875pt}%
\definecolor{currentstroke}{rgb}{0.000000,0.000000,0.000000}%
\pgfsetstrokecolor{currentstroke}%
\pgfsetdash{}{0pt}%
\pgfpathmoveto{\pgfqpoint{1.573972in}{1.689699in}}%
\pgfpathlineto{\pgfqpoint{1.573972in}{1.740316in}}%
\pgfusepath{stroke}%
\end{pgfscope}%
\begin{pgfscope}%
\pgfpathrectangle{\pgfqpoint{0.636356in}{0.700846in}}{\pgfqpoint{1.933229in}{1.137023in}} %
\pgfusepath{clip}%
\pgfsetbuttcap%
\pgfsetroundjoin%
\pgfsetlinewidth{0.501875pt}%
\definecolor{currentstroke}{rgb}{0.000000,0.000000,0.000000}%
\pgfsetstrokecolor{currentstroke}%
\pgfsetdash{}{0pt}%
\pgfpathmoveto{\pgfqpoint{1.593304in}{1.726507in}}%
\pgfpathlineto{\pgfqpoint{1.593304in}{1.778046in}}%
\pgfusepath{stroke}%
\end{pgfscope}%
\begin{pgfscope}%
\pgfpathrectangle{\pgfqpoint{0.636356in}{0.700846in}}{\pgfqpoint{1.933229in}{1.137023in}} %
\pgfusepath{clip}%
\pgfsetbuttcap%
\pgfsetroundjoin%
\pgfsetlinewidth{0.501875pt}%
\definecolor{currentstroke}{rgb}{0.000000,0.000000,0.000000}%
\pgfsetstrokecolor{currentstroke}%
\pgfsetdash{}{0pt}%
\pgfpathmoveto{\pgfqpoint{1.612636in}{1.691570in}}%
\pgfpathlineto{\pgfqpoint{1.612636in}{1.742235in}}%
\pgfusepath{stroke}%
\end{pgfscope}%
\begin{pgfscope}%
\pgfpathrectangle{\pgfqpoint{0.636356in}{0.700846in}}{\pgfqpoint{1.933229in}{1.137023in}} %
\pgfusepath{clip}%
\pgfsetbuttcap%
\pgfsetroundjoin%
\pgfsetlinewidth{0.501875pt}%
\definecolor{currentstroke}{rgb}{0.000000,0.000000,0.000000}%
\pgfsetstrokecolor{currentstroke}%
\pgfsetdash{}{0pt}%
\pgfpathmoveto{\pgfqpoint{1.631969in}{1.696561in}}%
\pgfpathlineto{\pgfqpoint{1.631969in}{1.747351in}}%
\pgfusepath{stroke}%
\end{pgfscope}%
\begin{pgfscope}%
\pgfpathrectangle{\pgfqpoint{0.636356in}{0.700846in}}{\pgfqpoint{1.933229in}{1.137023in}} %
\pgfusepath{clip}%
\pgfsetbuttcap%
\pgfsetroundjoin%
\pgfsetlinewidth{0.501875pt}%
\definecolor{currentstroke}{rgb}{0.000000,0.000000,0.000000}%
\pgfsetstrokecolor{currentstroke}%
\pgfsetdash{}{0pt}%
\pgfpathmoveto{\pgfqpoint{1.651301in}{1.739610in}}%
\pgfpathlineto{\pgfqpoint{1.651301in}{1.791473in}}%
\pgfusepath{stroke}%
\end{pgfscope}%
\begin{pgfscope}%
\pgfpathrectangle{\pgfqpoint{0.636356in}{0.700846in}}{\pgfqpoint{1.933229in}{1.137023in}} %
\pgfusepath{clip}%
\pgfsetbuttcap%
\pgfsetroundjoin%
\pgfsetlinewidth{0.501875pt}%
\definecolor{currentstroke}{rgb}{0.000000,0.000000,0.000000}%
\pgfsetstrokecolor{currentstroke}%
\pgfsetdash{}{0pt}%
\pgfpathmoveto{\pgfqpoint{1.670633in}{1.654146in}}%
\pgfpathlineto{\pgfqpoint{1.670633in}{1.703857in}}%
\pgfusepath{stroke}%
\end{pgfscope}%
\begin{pgfscope}%
\pgfpathrectangle{\pgfqpoint{0.636356in}{0.700846in}}{\pgfqpoint{1.933229in}{1.137023in}} %
\pgfusepath{clip}%
\pgfsetbuttcap%
\pgfsetroundjoin%
\pgfsetlinewidth{0.501875pt}%
\definecolor{currentstroke}{rgb}{0.000000,0.000000,0.000000}%
\pgfsetstrokecolor{currentstroke}%
\pgfsetdash{}{0pt}%
\pgfpathmoveto{\pgfqpoint{1.689965in}{1.724635in}}%
\pgfpathlineto{\pgfqpoint{1.689965in}{1.776128in}}%
\pgfusepath{stroke}%
\end{pgfscope}%
\begin{pgfscope}%
\pgfpathrectangle{\pgfqpoint{0.636356in}{0.700846in}}{\pgfqpoint{1.933229in}{1.137023in}} %
\pgfusepath{clip}%
\pgfsetbuttcap%
\pgfsetroundjoin%
\pgfsetlinewidth{0.501875pt}%
\definecolor{currentstroke}{rgb}{0.000000,0.000000,0.000000}%
\pgfsetstrokecolor{currentstroke}%
\pgfsetdash{}{0pt}%
\pgfpathmoveto{\pgfqpoint{1.709298in}{1.680966in}}%
\pgfpathlineto{\pgfqpoint{1.709298in}{1.731362in}}%
\pgfusepath{stroke}%
\end{pgfscope}%
\begin{pgfscope}%
\pgfpathrectangle{\pgfqpoint{0.636356in}{0.700846in}}{\pgfqpoint{1.933229in}{1.137023in}} %
\pgfusepath{clip}%
\pgfsetbuttcap%
\pgfsetroundjoin%
\pgfsetlinewidth{0.501875pt}%
\definecolor{currentstroke}{rgb}{0.000000,0.000000,0.000000}%
\pgfsetstrokecolor{currentstroke}%
\pgfsetdash{}{0pt}%
\pgfpathmoveto{\pgfqpoint{1.728630in}{1.584314in}}%
\pgfpathlineto{\pgfqpoint{1.728630in}{1.632193in}}%
\pgfusepath{stroke}%
\end{pgfscope}%
\begin{pgfscope}%
\pgfpathrectangle{\pgfqpoint{0.636356in}{0.700846in}}{\pgfqpoint{1.933229in}{1.137023in}} %
\pgfusepath{clip}%
\pgfsetbuttcap%
\pgfsetroundjoin%
\pgfsetlinewidth{0.501875pt}%
\definecolor{currentstroke}{rgb}{0.000000,0.000000,0.000000}%
\pgfsetstrokecolor{currentstroke}%
\pgfsetdash{}{0pt}%
\pgfpathmoveto{\pgfqpoint{1.747962in}{1.600522in}}%
\pgfpathlineto{\pgfqpoint{1.747962in}{1.648832in}}%
\pgfusepath{stroke}%
\end{pgfscope}%
\begin{pgfscope}%
\pgfpathrectangle{\pgfqpoint{0.636356in}{0.700846in}}{\pgfqpoint{1.933229in}{1.137023in}} %
\pgfusepath{clip}%
\pgfsetbuttcap%
\pgfsetroundjoin%
\pgfsetlinewidth{0.501875pt}%
\definecolor{currentstroke}{rgb}{0.000000,0.000000,0.000000}%
\pgfsetstrokecolor{currentstroke}%
\pgfsetdash{}{0pt}%
\pgfpathmoveto{\pgfqpoint{1.767295in}{1.492093in}}%
\pgfpathlineto{\pgfqpoint{1.767295in}{1.537437in}}%
\pgfusepath{stroke}%
\end{pgfscope}%
\begin{pgfscope}%
\pgfpathrectangle{\pgfqpoint{0.636356in}{0.700846in}}{\pgfqpoint{1.933229in}{1.137023in}} %
\pgfusepath{clip}%
\pgfsetbuttcap%
\pgfsetroundjoin%
\pgfsetlinewidth{0.501875pt}%
\definecolor{currentstroke}{rgb}{0.000000,0.000000,0.000000}%
\pgfsetstrokecolor{currentstroke}%
\pgfsetdash{}{0pt}%
\pgfpathmoveto{\pgfqpoint{1.786627in}{1.503305in}}%
\pgfpathlineto{\pgfqpoint{1.786627in}{1.548965in}}%
\pgfusepath{stroke}%
\end{pgfscope}%
\begin{pgfscope}%
\pgfpathrectangle{\pgfqpoint{0.636356in}{0.700846in}}{\pgfqpoint{1.933229in}{1.137023in}} %
\pgfusepath{clip}%
\pgfsetbuttcap%
\pgfsetroundjoin%
\pgfsetlinewidth{0.501875pt}%
\definecolor{currentstroke}{rgb}{0.000000,0.000000,0.000000}%
\pgfsetstrokecolor{currentstroke}%
\pgfsetdash{}{0pt}%
\pgfpathmoveto{\pgfqpoint{1.805959in}{1.485241in}}%
\pgfpathlineto{\pgfqpoint{1.805959in}{1.530392in}}%
\pgfusepath{stroke}%
\end{pgfscope}%
\begin{pgfscope}%
\pgfpathrectangle{\pgfqpoint{0.636356in}{0.700846in}}{\pgfqpoint{1.933229in}{1.137023in}} %
\pgfusepath{clip}%
\pgfsetbuttcap%
\pgfsetroundjoin%
\pgfsetlinewidth{0.501875pt}%
\definecolor{currentstroke}{rgb}{0.000000,0.000000,0.000000}%
\pgfsetstrokecolor{currentstroke}%
\pgfsetdash{}{0pt}%
\pgfpathmoveto{\pgfqpoint{1.825291in}{1.421108in}}%
\pgfpathlineto{\pgfqpoint{1.825291in}{1.464400in}}%
\pgfusepath{stroke}%
\end{pgfscope}%
\begin{pgfscope}%
\pgfpathrectangle{\pgfqpoint{0.636356in}{0.700846in}}{\pgfqpoint{1.933229in}{1.137023in}} %
\pgfusepath{clip}%
\pgfsetbuttcap%
\pgfsetroundjoin%
\pgfsetlinewidth{0.501875pt}%
\definecolor{currentstroke}{rgb}{0.000000,0.000000,0.000000}%
\pgfsetstrokecolor{currentstroke}%
\pgfsetdash{}{0pt}%
\pgfpathmoveto{\pgfqpoint{1.844624in}{1.366347in}}%
\pgfpathlineto{\pgfqpoint{1.844624in}{1.407985in}}%
\pgfusepath{stroke}%
\end{pgfscope}%
\begin{pgfscope}%
\pgfpathrectangle{\pgfqpoint{0.636356in}{0.700846in}}{\pgfqpoint{1.933229in}{1.137023in}} %
\pgfusepath{clip}%
\pgfsetbuttcap%
\pgfsetroundjoin%
\pgfsetlinewidth{0.501875pt}%
\definecolor{currentstroke}{rgb}{0.000000,0.000000,0.000000}%
\pgfsetstrokecolor{currentstroke}%
\pgfsetdash{}{0pt}%
\pgfpathmoveto{\pgfqpoint{1.863956in}{1.340845in}}%
\pgfpathlineto{\pgfqpoint{1.863956in}{1.381689in}}%
\pgfusepath{stroke}%
\end{pgfscope}%
\begin{pgfscope}%
\pgfpathrectangle{\pgfqpoint{0.636356in}{0.700846in}}{\pgfqpoint{1.933229in}{1.137023in}} %
\pgfusepath{clip}%
\pgfsetbuttcap%
\pgfsetroundjoin%
\pgfsetlinewidth{0.501875pt}%
\definecolor{currentstroke}{rgb}{0.000000,0.000000,0.000000}%
\pgfsetstrokecolor{currentstroke}%
\pgfsetdash{}{0pt}%
\pgfpathmoveto{\pgfqpoint{1.883288in}{1.303539in}}%
\pgfpathlineto{\pgfqpoint{1.883288in}{1.343194in}}%
\pgfusepath{stroke}%
\end{pgfscope}%
\begin{pgfscope}%
\pgfpathrectangle{\pgfqpoint{0.636356in}{0.700846in}}{\pgfqpoint{1.933229in}{1.137023in}} %
\pgfusepath{clip}%
\pgfsetbuttcap%
\pgfsetroundjoin%
\pgfsetlinewidth{0.501875pt}%
\definecolor{currentstroke}{rgb}{0.000000,0.000000,0.000000}%
\pgfsetstrokecolor{currentstroke}%
\pgfsetdash{}{0pt}%
\pgfpathmoveto{\pgfqpoint{1.902621in}{1.228984in}}%
\pgfpathlineto{\pgfqpoint{1.902621in}{1.266146in}}%
\pgfusepath{stroke}%
\end{pgfscope}%
\begin{pgfscope}%
\pgfpathrectangle{\pgfqpoint{0.636356in}{0.700846in}}{\pgfqpoint{1.933229in}{1.137023in}} %
\pgfusepath{clip}%
\pgfsetbuttcap%
\pgfsetroundjoin%
\pgfsetlinewidth{0.501875pt}%
\definecolor{currentstroke}{rgb}{0.000000,0.000000,0.000000}%
\pgfsetstrokecolor{currentstroke}%
\pgfsetdash{}{0pt}%
\pgfpathmoveto{\pgfqpoint{1.921953in}{1.203531in}}%
\pgfpathlineto{\pgfqpoint{1.921953in}{1.239801in}}%
\pgfusepath{stroke}%
\end{pgfscope}%
\begin{pgfscope}%
\pgfpathrectangle{\pgfqpoint{0.636356in}{0.700846in}}{\pgfqpoint{1.933229in}{1.137023in}} %
\pgfusepath{clip}%
\pgfsetbuttcap%
\pgfsetroundjoin%
\pgfsetlinewidth{0.501875pt}%
\definecolor{currentstroke}{rgb}{0.000000,0.000000,0.000000}%
\pgfsetstrokecolor{currentstroke}%
\pgfsetdash{}{0pt}%
\pgfpathmoveto{\pgfqpoint{1.941285in}{1.160100in}}%
\pgfpathlineto{\pgfqpoint{1.941285in}{1.194797in}}%
\pgfusepath{stroke}%
\end{pgfscope}%
\begin{pgfscope}%
\pgfpathrectangle{\pgfqpoint{0.636356in}{0.700846in}}{\pgfqpoint{1.933229in}{1.137023in}} %
\pgfusepath{clip}%
\pgfsetbuttcap%
\pgfsetroundjoin%
\pgfsetlinewidth{0.501875pt}%
\definecolor{currentstroke}{rgb}{0.000000,0.000000,0.000000}%
\pgfsetstrokecolor{currentstroke}%
\pgfsetdash{}{0pt}%
\pgfpathmoveto{\pgfqpoint{1.960617in}{1.117327in}}%
\pgfpathlineto{\pgfqpoint{1.960617in}{1.150398in}}%
\pgfusepath{stroke}%
\end{pgfscope}%
\begin{pgfscope}%
\pgfpathrectangle{\pgfqpoint{0.636356in}{0.700846in}}{\pgfqpoint{1.933229in}{1.137023in}} %
\pgfusepath{clip}%
\pgfsetbuttcap%
\pgfsetroundjoin%
\pgfsetlinewidth{0.501875pt}%
\definecolor{currentstroke}{rgb}{0.000000,0.000000,0.000000}%
\pgfsetstrokecolor{currentstroke}%
\pgfsetdash{}{0pt}%
\pgfpathmoveto{\pgfqpoint{1.979950in}{1.072120in}}%
\pgfpathlineto{\pgfqpoint{1.979950in}{1.103380in}}%
\pgfusepath{stroke}%
\end{pgfscope}%
\begin{pgfscope}%
\pgfpathrectangle{\pgfqpoint{0.636356in}{0.700846in}}{\pgfqpoint{1.933229in}{1.137023in}} %
\pgfusepath{clip}%
\pgfsetbuttcap%
\pgfsetroundjoin%
\pgfsetlinewidth{0.501875pt}%
\definecolor{currentstroke}{rgb}{0.000000,0.000000,0.000000}%
\pgfsetstrokecolor{currentstroke}%
\pgfsetdash{}{0pt}%
\pgfpathmoveto{\pgfqpoint{1.999282in}{1.048609in}}%
\pgfpathlineto{\pgfqpoint{1.999282in}{1.078883in}}%
\pgfusepath{stroke}%
\end{pgfscope}%
\begin{pgfscope}%
\pgfpathrectangle{\pgfqpoint{0.636356in}{0.700846in}}{\pgfqpoint{1.933229in}{1.137023in}} %
\pgfusepath{clip}%
\pgfsetbuttcap%
\pgfsetroundjoin%
\pgfsetlinewidth{0.501875pt}%
\definecolor{currentstroke}{rgb}{0.000000,0.000000,0.000000}%
\pgfsetstrokecolor{currentstroke}%
\pgfsetdash{}{0pt}%
\pgfpathmoveto{\pgfqpoint{2.018614in}{1.033150in}}%
\pgfpathlineto{\pgfqpoint{2.018614in}{1.062758in}}%
\pgfusepath{stroke}%
\end{pgfscope}%
\begin{pgfscope}%
\pgfpathrectangle{\pgfqpoint{0.636356in}{0.700846in}}{\pgfqpoint{1.933229in}{1.137023in}} %
\pgfusepath{clip}%
\pgfsetbuttcap%
\pgfsetroundjoin%
\pgfsetlinewidth{0.501875pt}%
\definecolor{currentstroke}{rgb}{0.000000,0.000000,0.000000}%
\pgfsetstrokecolor{currentstroke}%
\pgfsetdash{}{0pt}%
\pgfpathmoveto{\pgfqpoint{2.037947in}{0.988673in}}%
\pgfpathlineto{\pgfqpoint{2.037947in}{1.016273in}}%
\pgfusepath{stroke}%
\end{pgfscope}%
\begin{pgfscope}%
\pgfpathrectangle{\pgfqpoint{0.636356in}{0.700846in}}{\pgfqpoint{1.933229in}{1.137023in}} %
\pgfusepath{clip}%
\pgfsetbuttcap%
\pgfsetroundjoin%
\pgfsetlinewidth{0.501875pt}%
\definecolor{currentstroke}{rgb}{0.000000,0.000000,0.000000}%
\pgfsetstrokecolor{currentstroke}%
\pgfsetdash{}{0pt}%
\pgfpathmoveto{\pgfqpoint{2.057279in}{0.966464in}}%
\pgfpathlineto{\pgfqpoint{2.057279in}{0.993002in}}%
\pgfusepath{stroke}%
\end{pgfscope}%
\begin{pgfscope}%
\pgfpathrectangle{\pgfqpoint{0.636356in}{0.700846in}}{\pgfqpoint{1.933229in}{1.137023in}} %
\pgfusepath{clip}%
\pgfsetbuttcap%
\pgfsetroundjoin%
\pgfsetlinewidth{0.501875pt}%
\definecolor{currentstroke}{rgb}{0.000000,0.000000,0.000000}%
\pgfsetstrokecolor{currentstroke}%
\pgfsetdash{}{0pt}%
\pgfpathmoveto{\pgfqpoint{2.076611in}{0.936270in}}%
\pgfpathlineto{\pgfqpoint{2.076611in}{0.961291in}}%
\pgfusepath{stroke}%
\end{pgfscope}%
\begin{pgfscope}%
\pgfpathrectangle{\pgfqpoint{0.636356in}{0.700846in}}{\pgfqpoint{1.933229in}{1.137023in}} %
\pgfusepath{clip}%
\pgfsetbuttcap%
\pgfsetroundjoin%
\pgfsetlinewidth{0.501875pt}%
\definecolor{currentstroke}{rgb}{0.000000,0.000000,0.000000}%
\pgfsetstrokecolor{currentstroke}%
\pgfsetdash{}{0pt}%
\pgfpathmoveto{\pgfqpoint{2.095943in}{0.941196in}}%
\pgfpathlineto{\pgfqpoint{2.095943in}{0.966471in}}%
\pgfusepath{stroke}%
\end{pgfscope}%
\begin{pgfscope}%
\pgfpathrectangle{\pgfqpoint{0.636356in}{0.700846in}}{\pgfqpoint{1.933229in}{1.137023in}} %
\pgfusepath{clip}%
\pgfsetbuttcap%
\pgfsetroundjoin%
\pgfsetlinewidth{0.501875pt}%
\definecolor{currentstroke}{rgb}{0.000000,0.000000,0.000000}%
\pgfsetstrokecolor{currentstroke}%
\pgfsetdash{}{0pt}%
\pgfpathmoveto{\pgfqpoint{2.115276in}{0.884015in}}%
\pgfpathlineto{\pgfqpoint{2.115276in}{0.906160in}}%
\pgfusepath{stroke}%
\end{pgfscope}%
\begin{pgfscope}%
\pgfpathrectangle{\pgfqpoint{0.636356in}{0.700846in}}{\pgfqpoint{1.933229in}{1.137023in}} %
\pgfusepath{clip}%
\pgfsetbuttcap%
\pgfsetroundjoin%
\pgfsetlinewidth{0.501875pt}%
\definecolor{currentstroke}{rgb}{0.000000,0.000000,0.000000}%
\pgfsetstrokecolor{currentstroke}%
\pgfsetdash{}{0pt}%
\pgfpathmoveto{\pgfqpoint{2.134608in}{0.877267in}}%
\pgfpathlineto{\pgfqpoint{2.134608in}{0.899012in}}%
\pgfusepath{stroke}%
\end{pgfscope}%
\begin{pgfscope}%
\pgfpathrectangle{\pgfqpoint{0.636356in}{0.700846in}}{\pgfqpoint{1.933229in}{1.137023in}} %
\pgfusepath{clip}%
\pgfsetbuttcap%
\pgfsetroundjoin%
\pgfsetlinewidth{0.501875pt}%
\definecolor{currentstroke}{rgb}{0.000000,0.000000,0.000000}%
\pgfsetstrokecolor{currentstroke}%
\pgfsetdash{}{0pt}%
\pgfpathmoveto{\pgfqpoint{2.153940in}{0.862556in}}%
\pgfpathlineto{\pgfqpoint{2.153940in}{0.883402in}}%
\pgfusepath{stroke}%
\end{pgfscope}%
\begin{pgfscope}%
\pgfpathrectangle{\pgfqpoint{0.636356in}{0.700846in}}{\pgfqpoint{1.933229in}{1.137023in}} %
\pgfusepath{clip}%
\pgfsetbuttcap%
\pgfsetroundjoin%
\pgfsetlinewidth{0.501875pt}%
\definecolor{currentstroke}{rgb}{0.000000,0.000000,0.000000}%
\pgfsetstrokecolor{currentstroke}%
\pgfsetdash{}{0pt}%
\pgfpathmoveto{\pgfqpoint{2.173273in}{0.831367in}}%
\pgfpathlineto{\pgfqpoint{2.173273in}{0.850159in}}%
\pgfusepath{stroke}%
\end{pgfscope}%
\begin{pgfscope}%
\pgfpathrectangle{\pgfqpoint{0.636356in}{0.700846in}}{\pgfqpoint{1.933229in}{1.137023in}} %
\pgfusepath{clip}%
\pgfsetbuttcap%
\pgfsetroundjoin%
\pgfsetlinewidth{0.501875pt}%
\definecolor{currentstroke}{rgb}{0.000000,0.000000,0.000000}%
\pgfsetstrokecolor{currentstroke}%
\pgfsetdash{}{0pt}%
\pgfpathmoveto{\pgfqpoint{2.192605in}{0.833810in}}%
\pgfpathlineto{\pgfqpoint{2.192605in}{0.852770in}}%
\pgfusepath{stroke}%
\end{pgfscope}%
\begin{pgfscope}%
\pgfpathrectangle{\pgfqpoint{0.636356in}{0.700846in}}{\pgfqpoint{1.933229in}{1.137023in}} %
\pgfusepath{clip}%
\pgfsetbuttcap%
\pgfsetroundjoin%
\pgfsetlinewidth{0.501875pt}%
\definecolor{currentstroke}{rgb}{0.000000,0.000000,0.000000}%
\pgfsetstrokecolor{currentstroke}%
\pgfsetdash{}{0pt}%
\pgfpathmoveto{\pgfqpoint{2.211937in}{0.811860in}}%
\pgfpathlineto{\pgfqpoint{2.211937in}{0.829239in}}%
\pgfusepath{stroke}%
\end{pgfscope}%
\begin{pgfscope}%
\pgfpathrectangle{\pgfqpoint{0.636356in}{0.700846in}}{\pgfqpoint{1.933229in}{1.137023in}} %
\pgfusepath{clip}%
\pgfsetbuttcap%
\pgfsetroundjoin%
\pgfsetlinewidth{0.501875pt}%
\definecolor{currentstroke}{rgb}{0.000000,0.000000,0.000000}%
\pgfsetstrokecolor{currentstroke}%
\pgfsetdash{}{0pt}%
\pgfpathmoveto{\pgfqpoint{2.231269in}{0.805776in}}%
\pgfpathlineto{\pgfqpoint{2.231269in}{0.822690in}}%
\pgfusepath{stroke}%
\end{pgfscope}%
\begin{pgfscope}%
\pgfpathrectangle{\pgfqpoint{0.636356in}{0.700846in}}{\pgfqpoint{1.933229in}{1.137023in}} %
\pgfusepath{clip}%
\pgfsetbuttcap%
\pgfsetroundjoin%
\pgfsetlinewidth{0.501875pt}%
\definecolor{currentstroke}{rgb}{0.000000,0.000000,0.000000}%
\pgfsetstrokecolor{currentstroke}%
\pgfsetdash{}{0pt}%
\pgfpathmoveto{\pgfqpoint{2.250602in}{0.783932in}}%
\pgfpathlineto{\pgfqpoint{2.250602in}{0.799053in}}%
\pgfusepath{stroke}%
\end{pgfscope}%
\begin{pgfscope}%
\pgfpathrectangle{\pgfqpoint{0.636356in}{0.700846in}}{\pgfqpoint{1.933229in}{1.137023in}} %
\pgfusepath{clip}%
\pgfsetbuttcap%
\pgfsetroundjoin%
\pgfsetlinewidth{0.501875pt}%
\definecolor{currentstroke}{rgb}{0.000000,0.000000,0.000000}%
\pgfsetstrokecolor{currentstroke}%
\pgfsetdash{}{0pt}%
\pgfpathmoveto{\pgfqpoint{2.269934in}{0.778488in}}%
\pgfpathlineto{\pgfqpoint{2.269934in}{0.793126in}}%
\pgfusepath{stroke}%
\end{pgfscope}%
\begin{pgfscope}%
\pgfpathrectangle{\pgfqpoint{0.636356in}{0.700846in}}{\pgfqpoint{1.933229in}{1.137023in}} %
\pgfusepath{clip}%
\pgfsetbuttcap%
\pgfsetroundjoin%
\pgfsetlinewidth{0.501875pt}%
\definecolor{currentstroke}{rgb}{0.000000,0.000000,0.000000}%
\pgfsetstrokecolor{currentstroke}%
\pgfsetdash{}{0pt}%
\pgfpathmoveto{\pgfqpoint{2.289266in}{0.783932in}}%
\pgfpathlineto{\pgfqpoint{2.289266in}{0.799053in}}%
\pgfusepath{stroke}%
\end{pgfscope}%
\begin{pgfscope}%
\pgfpathrectangle{\pgfqpoint{0.636356in}{0.700846in}}{\pgfqpoint{1.933229in}{1.137023in}} %
\pgfusepath{clip}%
\pgfsetbuttcap%
\pgfsetroundjoin%
\pgfsetlinewidth{0.501875pt}%
\definecolor{currentstroke}{rgb}{0.000000,0.000000,0.000000}%
\pgfsetstrokecolor{currentstroke}%
\pgfsetdash{}{0pt}%
\pgfpathmoveto{\pgfqpoint{2.308599in}{0.774863in}}%
\pgfpathlineto{\pgfqpoint{2.308599in}{0.789171in}}%
\pgfusepath{stroke}%
\end{pgfscope}%
\begin{pgfscope}%
\pgfpathrectangle{\pgfqpoint{0.636356in}{0.700846in}}{\pgfqpoint{1.933229in}{1.137023in}} %
\pgfusepath{clip}%
\pgfsetbuttcap%
\pgfsetroundjoin%
\pgfsetlinewidth{0.501875pt}%
\definecolor{currentstroke}{rgb}{0.000000,0.000000,0.000000}%
\pgfsetstrokecolor{currentstroke}%
\pgfsetdash{}{0pt}%
\pgfpathmoveto{\pgfqpoint{2.327931in}{0.757708in}}%
\pgfpathlineto{\pgfqpoint{2.327931in}{0.770984in}}%
\pgfusepath{stroke}%
\end{pgfscope}%
\begin{pgfscope}%
\pgfpathrectangle{\pgfqpoint{0.636356in}{0.700846in}}{\pgfqpoint{1.933229in}{1.137023in}} %
\pgfusepath{clip}%
\pgfsetbuttcap%
\pgfsetroundjoin%
\pgfsetlinewidth{0.501875pt}%
\definecolor{currentstroke}{rgb}{0.000000,0.000000,0.000000}%
\pgfsetstrokecolor{currentstroke}%
\pgfsetdash{}{0pt}%
\pgfpathmoveto{\pgfqpoint{2.347263in}{0.751716in}}%
\pgfpathlineto{\pgfqpoint{2.347263in}{0.764344in}}%
\pgfusepath{stroke}%
\end{pgfscope}%
\begin{pgfscope}%
\pgfpathrectangle{\pgfqpoint{0.636356in}{0.700846in}}{\pgfqpoint{1.933229in}{1.137023in}} %
\pgfusepath{clip}%
\pgfsetbuttcap%
\pgfsetroundjoin%
\pgfsetlinewidth{0.501875pt}%
\definecolor{currentstroke}{rgb}{0.000000,0.000000,0.000000}%
\pgfsetstrokecolor{currentstroke}%
\pgfsetdash{}{0pt}%
\pgfpathmoveto{\pgfqpoint{2.366595in}{0.745146in}}%
\pgfpathlineto{\pgfqpoint{2.366595in}{0.757019in}}%
\pgfusepath{stroke}%
\end{pgfscope}%
\begin{pgfscope}%
\pgfpathrectangle{\pgfqpoint{0.636356in}{0.700846in}}{\pgfqpoint{1.933229in}{1.137023in}} %
\pgfusepath{clip}%
\pgfsetbuttcap%
\pgfsetroundjoin%
\pgfsetlinewidth{0.501875pt}%
\definecolor{currentstroke}{rgb}{0.000000,0.000000,0.000000}%
\pgfsetstrokecolor{currentstroke}%
\pgfsetdash{}{0pt}%
\pgfpathmoveto{\pgfqpoint{2.385928in}{0.743955in}}%
\pgfpathlineto{\pgfqpoint{2.385928in}{0.755684in}}%
\pgfusepath{stroke}%
\end{pgfscope}%
\begin{pgfscope}%
\pgfpathrectangle{\pgfqpoint{0.636356in}{0.700846in}}{\pgfqpoint{1.933229in}{1.137023in}} %
\pgfusepath{clip}%
\pgfsetbuttcap%
\pgfsetroundjoin%
\pgfsetlinewidth{0.501875pt}%
\definecolor{currentstroke}{rgb}{0.000000,0.000000,0.000000}%
\pgfsetstrokecolor{currentstroke}%
\pgfsetdash{}{0pt}%
\pgfpathmoveto{\pgfqpoint{2.405260in}{0.731507in}}%
\pgfpathlineto{\pgfqpoint{2.405260in}{0.741607in}}%
\pgfusepath{stroke}%
\end{pgfscope}%
\begin{pgfscope}%
\pgfpathrectangle{\pgfqpoint{0.636356in}{0.700846in}}{\pgfqpoint{1.933229in}{1.137023in}} %
\pgfusepath{clip}%
\pgfsetbuttcap%
\pgfsetroundjoin%
\pgfsetlinewidth{0.501875pt}%
\definecolor{currentstroke}{rgb}{0.000000,0.000000,0.000000}%
\pgfsetstrokecolor{currentstroke}%
\pgfsetdash{}{0pt}%
\pgfpathmoveto{\pgfqpoint{2.424592in}{0.736234in}}%
\pgfpathlineto{\pgfqpoint{2.424592in}{0.746985in}}%
\pgfusepath{stroke}%
\end{pgfscope}%
\begin{pgfscope}%
\pgfpathrectangle{\pgfqpoint{0.636356in}{0.700846in}}{\pgfqpoint{1.933229in}{1.137023in}} %
\pgfusepath{clip}%
\pgfsetbuttcap%
\pgfsetroundjoin%
\pgfsetlinewidth{0.501875pt}%
\definecolor{currentstroke}{rgb}{0.000000,0.000000,0.000000}%
\pgfsetstrokecolor{currentstroke}%
\pgfsetdash{}{0pt}%
\pgfpathmoveto{\pgfqpoint{2.443925in}{0.734459in}}%
\pgfpathlineto{\pgfqpoint{2.443925in}{0.744971in}}%
\pgfusepath{stroke}%
\end{pgfscope}%
\begin{pgfscope}%
\pgfpathrectangle{\pgfqpoint{0.636356in}{0.700846in}}{\pgfqpoint{1.933229in}{1.137023in}} %
\pgfusepath{clip}%
\pgfsetbuttcap%
\pgfsetroundjoin%
\pgfsetlinewidth{0.501875pt}%
\definecolor{currentstroke}{rgb}{0.000000,0.000000,0.000000}%
\pgfsetstrokecolor{currentstroke}%
\pgfsetdash{}{0pt}%
\pgfpathmoveto{\pgfqpoint{2.463257in}{0.727979in}}%
\pgfpathlineto{\pgfqpoint{2.463257in}{0.737558in}}%
\pgfusepath{stroke}%
\end{pgfscope}%
\begin{pgfscope}%
\pgfpathrectangle{\pgfqpoint{0.636356in}{0.700846in}}{\pgfqpoint{1.933229in}{1.137023in}} %
\pgfusepath{clip}%
\pgfsetbuttcap%
\pgfsetroundjoin%
\pgfsetlinewidth{0.501875pt}%
\definecolor{currentstroke}{rgb}{0.000000,0.000000,0.000000}%
\pgfsetstrokecolor{currentstroke}%
\pgfsetdash{}{0pt}%
\pgfpathmoveto{\pgfqpoint{2.482589in}{0.727979in}}%
\pgfpathlineto{\pgfqpoint{2.482589in}{0.737558in}}%
\pgfusepath{stroke}%
\end{pgfscope}%
\begin{pgfscope}%
\pgfpathrectangle{\pgfqpoint{0.636356in}{0.700846in}}{\pgfqpoint{1.933229in}{1.137023in}} %
\pgfusepath{clip}%
\pgfsetbuttcap%
\pgfsetroundjoin%
\pgfsetlinewidth{0.501875pt}%
\definecolor{currentstroke}{rgb}{0.000000,0.000000,0.000000}%
\pgfsetstrokecolor{currentstroke}%
\pgfsetdash{}{0pt}%
\pgfpathmoveto{\pgfqpoint{2.501921in}{0.726220in}}%
\pgfpathlineto{\pgfqpoint{2.501921in}{0.735528in}}%
\pgfusepath{stroke}%
\end{pgfscope}%
\begin{pgfscope}%
\pgfpathrectangle{\pgfqpoint{0.636356in}{0.700846in}}{\pgfqpoint{1.933229in}{1.137023in}} %
\pgfusepath{clip}%
\pgfsetbuttcap%
\pgfsetroundjoin%
\pgfsetlinewidth{0.501875pt}%
\definecolor{currentstroke}{rgb}{0.000000,0.000000,0.000000}%
\pgfsetstrokecolor{currentstroke}%
\pgfsetdash{}{0pt}%
\pgfpathmoveto{\pgfqpoint{2.521254in}{0.721554in}}%
\pgfpathlineto{\pgfqpoint{2.521254in}{0.730092in}}%
\pgfusepath{stroke}%
\end{pgfscope}%
\begin{pgfscope}%
\pgfpathrectangle{\pgfqpoint{0.636356in}{0.700846in}}{\pgfqpoint{1.933229in}{1.137023in}} %
\pgfusepath{clip}%
\pgfsetbuttcap%
\pgfsetroundjoin%
\pgfsetlinewidth{0.501875pt}%
\definecolor{currentstroke}{rgb}{0.000000,0.000000,0.000000}%
\pgfsetstrokecolor{currentstroke}%
\pgfsetdash{}{0pt}%
\pgfpathmoveto{\pgfqpoint{2.540586in}{0.718081in}}%
\pgfpathlineto{\pgfqpoint{2.540586in}{0.725989in}}%
\pgfusepath{stroke}%
\end{pgfscope}%
\begin{pgfscope}%
\pgfpathrectangle{\pgfqpoint{0.636356in}{0.700846in}}{\pgfqpoint{1.933229in}{1.137023in}} %
\pgfusepath{clip}%
\pgfsetbuttcap%
\pgfsetroundjoin%
\pgfsetlinewidth{0.501875pt}%
\definecolor{currentstroke}{rgb}{0.000000,0.000000,0.000000}%
\pgfsetstrokecolor{currentstroke}%
\pgfsetdash{}{0pt}%
\pgfpathmoveto{\pgfqpoint{2.559918in}{0.719814in}}%
\pgfpathlineto{\pgfqpoint{2.559918in}{0.728044in}}%
\pgfusepath{stroke}%
\end{pgfscope}%
\begin{pgfscope}%
\pgfsetbuttcap%
\pgfsetroundjoin%
\definecolor{currentfill}{rgb}{0.000000,0.000000,0.000000}%
\pgfsetfillcolor{currentfill}%
\pgfsetlinewidth{1.003750pt}%
\definecolor{currentstroke}{rgb}{0.000000,0.000000,0.000000}%
\pgfsetstrokecolor{currentstroke}%
\pgfsetdash{}{0pt}%
\pgfsys@defobject{currentmarker}{\pgfqpoint{-0.006944in}{-0.006944in}}{\pgfqpoint{0.006944in}{0.006944in}}{%
\pgfpathmoveto{\pgfqpoint{0.000000in}{-0.006944in}}%
\pgfpathcurveto{\pgfqpoint{0.001842in}{-0.006944in}}{\pgfqpoint{0.003608in}{-0.006213in}}{\pgfqpoint{0.004910in}{-0.004910in}}%
\pgfpathcurveto{\pgfqpoint{0.006213in}{-0.003608in}}{\pgfqpoint{0.006944in}{-0.001842in}}{\pgfqpoint{0.006944in}{0.000000in}}%
\pgfpathcurveto{\pgfqpoint{0.006944in}{0.001842in}}{\pgfqpoint{0.006213in}{0.003608in}}{\pgfqpoint{0.004910in}{0.004910in}}%
\pgfpathcurveto{\pgfqpoint{0.003608in}{0.006213in}}{\pgfqpoint{0.001842in}{0.006944in}}{\pgfqpoint{0.000000in}{0.006944in}}%
\pgfpathcurveto{\pgfqpoint{-0.001842in}{0.006944in}}{\pgfqpoint{-0.003608in}{0.006213in}}{\pgfqpoint{-0.004910in}{0.004910in}}%
\pgfpathcurveto{\pgfqpoint{-0.006213in}{0.003608in}}{\pgfqpoint{-0.006944in}{0.001842in}}{\pgfqpoint{-0.006944in}{0.000000in}}%
\pgfpathcurveto{\pgfqpoint{-0.006944in}{-0.001842in}}{\pgfqpoint{-0.006213in}{-0.003608in}}{\pgfqpoint{-0.004910in}{-0.004910in}}%
\pgfpathcurveto{\pgfqpoint{-0.003608in}{-0.006213in}}{\pgfqpoint{-0.001842in}{-0.006944in}}{\pgfqpoint{0.000000in}{-0.006944in}}%
\pgfpathclose%
\pgfusepath{stroke,fill}%
}%
\begin{pgfscope}%
\pgfsys@transformshift{0.646022in}{0.726745in}%
\pgfsys@useobject{currentmarker}{}%
\end{pgfscope}%
\begin{pgfscope}%
\pgfsys@transformshift{0.665354in}{0.717902in}%
\pgfsys@useobject{currentmarker}{}%
\end{pgfscope}%
\begin{pgfscope}%
\pgfsys@transformshift{0.684687in}{0.736220in}%
\pgfsys@useobject{currentmarker}{}%
\end{pgfscope}%
\begin{pgfscope}%
\pgfsys@transformshift{0.704019in}{0.731799in}%
\pgfsys@useobject{currentmarker}{}%
\end{pgfscope}%
\begin{pgfscope}%
\pgfsys@transformshift{0.723351in}{0.741274in}%
\pgfsys@useobject{currentmarker}{}%
\end{pgfscope}%
\begin{pgfscope}%
\pgfsys@transformshift{0.742683in}{0.739379in}%
\pgfsys@useobject{currentmarker}{}%
\end{pgfscope}%
\begin{pgfscope}%
\pgfsys@transformshift{0.762016in}{0.736220in}%
\pgfsys@useobject{currentmarker}{}%
\end{pgfscope}%
\begin{pgfscope}%
\pgfsys@transformshift{0.781348in}{0.743801in}%
\pgfsys@useobject{currentmarker}{}%
\end{pgfscope}%
\begin{pgfscope}%
\pgfsys@transformshift{0.800680in}{0.740010in}%
\pgfsys@useobject{currentmarker}{}%
\end{pgfscope}%
\begin{pgfscope}%
\pgfsys@transformshift{0.820013in}{0.747591in}%
\pgfsys@useobject{currentmarker}{}%
\end{pgfscope}%
\begin{pgfscope}%
\pgfsys@transformshift{0.839345in}{0.744432in}%
\pgfsys@useobject{currentmarker}{}%
\end{pgfscope}%
\begin{pgfscope}%
\pgfsys@transformshift{0.858677in}{0.759592in}%
\pgfsys@useobject{currentmarker}{}%
\end{pgfscope}%
\begin{pgfscope}%
\pgfsys@transformshift{0.878009in}{0.768436in}%
\pgfsys@useobject{currentmarker}{}%
\end{pgfscope}%
\begin{pgfscope}%
\pgfsys@transformshift{0.897342in}{0.770963in}%
\pgfsys@useobject{currentmarker}{}%
\end{pgfscope}%
\begin{pgfscope}%
\pgfsys@transformshift{0.916674in}{0.781701in}%
\pgfsys@useobject{currentmarker}{}%
\end{pgfscope}%
\begin{pgfscope}%
\pgfsys@transformshift{0.936006in}{0.791808in}%
\pgfsys@useobject{currentmarker}{}%
\end{pgfscope}%
\begin{pgfscope}%
\pgfsys@transformshift{0.955339in}{0.800652in}%
\pgfsys@useobject{currentmarker}{}%
\end{pgfscope}%
\begin{pgfscope}%
\pgfsys@transformshift{0.974671in}{0.817707in}%
\pgfsys@useobject{currentmarker}{}%
\end{pgfscope}%
\begin{pgfscope}%
\pgfsys@transformshift{0.994003in}{0.803178in}%
\pgfsys@useobject{currentmarker}{}%
\end{pgfscope}%
\begin{pgfscope}%
\pgfsys@transformshift{1.013335in}{0.825287in}%
\pgfsys@useobject{currentmarker}{}%
\end{pgfscope}%
\begin{pgfscope}%
\pgfsys@transformshift{1.032668in}{0.822129in}%
\pgfsys@useobject{currentmarker}{}%
\end{pgfscope}%
\begin{pgfscope}%
\pgfsys@transformshift{1.052000in}{0.861293in}%
\pgfsys@useobject{currentmarker}{}%
\end{pgfscope}%
\begin{pgfscope}%
\pgfsys@transformshift{1.071332in}{0.863188in}%
\pgfsys@useobject{currentmarker}{}%
\end{pgfscope}%
\begin{pgfscope}%
\pgfsys@transformshift{1.090665in}{0.901089in}%
\pgfsys@useobject{currentmarker}{}%
\end{pgfscope}%
\begin{pgfscope}%
\pgfsys@transformshift{1.109997in}{0.923197in}%
\pgfsys@useobject{currentmarker}{}%
\end{pgfscope}%
\begin{pgfscope}%
\pgfsys@transformshift{1.129329in}{0.933304in}%
\pgfsys@useobject{currentmarker}{}%
\end{pgfscope}%
\begin{pgfscope}%
\pgfsys@transformshift{1.148661in}{0.927619in}%
\pgfsys@useobject{currentmarker}{}%
\end{pgfscope}%
\begin{pgfscope}%
\pgfsys@transformshift{1.167994in}{0.978785in}%
\pgfsys@useobject{currentmarker}{}%
\end{pgfscope}%
\begin{pgfscope}%
\pgfsys@transformshift{1.187326in}{0.974995in}%
\pgfsys@useobject{currentmarker}{}%
\end{pgfscope}%
\begin{pgfscope}%
\pgfsys@transformshift{1.206658in}{1.016054in}%
\pgfsys@useobject{currentmarker}{}%
\end{pgfscope}%
\begin{pgfscope}%
\pgfsys@transformshift{1.225991in}{1.067852in}%
\pgfsys@useobject{currentmarker}{}%
\end{pgfscope}%
\begin{pgfscope}%
\pgfsys@transformshift{1.245323in}{1.089329in}%
\pgfsys@useobject{currentmarker}{}%
\end{pgfscope}%
\begin{pgfscope}%
\pgfsys@transformshift{1.264655in}{1.081117in}%
\pgfsys@useobject{currentmarker}{}%
\end{pgfscope}%
\begin{pgfscope}%
\pgfsys@transformshift{1.283987in}{1.181554in}%
\pgfsys@useobject{currentmarker}{}%
\end{pgfscope}%
\begin{pgfscope}%
\pgfsys@transformshift{1.303320in}{1.192924in}%
\pgfsys@useobject{currentmarker}{}%
\end{pgfscope}%
\begin{pgfscope}%
\pgfsys@transformshift{1.322652in}{1.209348in}%
\pgfsys@useobject{currentmarker}{}%
\end{pgfscope}%
\begin{pgfscope}%
\pgfsys@transformshift{1.341984in}{1.312944in}%
\pgfsys@useobject{currentmarker}{}%
\end{pgfscope}%
\begin{pgfscope}%
\pgfsys@transformshift{1.361317in}{1.295888in}%
\pgfsys@useobject{currentmarker}{}%
\end{pgfscope}%
\begin{pgfscope}%
\pgfsys@transformshift{1.380649in}{1.375480in}%
\pgfsys@useobject{currentmarker}{}%
\end{pgfscope}%
\begin{pgfscope}%
\pgfsys@transformshift{1.399981in}{1.390008in}%
\pgfsys@useobject{currentmarker}{}%
\end{pgfscope}%
\begin{pgfscope}%
\pgfsys@transformshift{1.419313in}{1.485392in}%
\pgfsys@useobject{currentmarker}{}%
\end{pgfscope}%
\begin{pgfscope}%
\pgfsys@transformshift{1.438646in}{1.501816in}%
\pgfsys@useobject{currentmarker}{}%
\end{pgfscope}%
\begin{pgfscope}%
\pgfsys@transformshift{1.457978in}{1.547297in}%
\pgfsys@useobject{currentmarker}{}%
\end{pgfscope}%
\begin{pgfscope}%
\pgfsys@transformshift{1.477310in}{1.630678in}%
\pgfsys@useobject{currentmarker}{}%
\end{pgfscope}%
\begin{pgfscope}%
\pgfsys@transformshift{1.496643in}{1.618045in}%
\pgfsys@useobject{currentmarker}{}%
\end{pgfscope}%
\begin{pgfscope}%
\pgfsys@transformshift{1.515975in}{1.632573in}%
\pgfsys@useobject{currentmarker}{}%
\end{pgfscope}%
\begin{pgfscope}%
\pgfsys@transformshift{1.535307in}{1.678054in}%
\pgfsys@useobject{currentmarker}{}%
\end{pgfscope}%
\begin{pgfscope}%
\pgfsys@transformshift{1.554639in}{1.693846in}%
\pgfsys@useobject{currentmarker}{}%
\end{pgfscope}%
\begin{pgfscope}%
\pgfsys@transformshift{1.573972in}{1.714692in}%
\pgfsys@useobject{currentmarker}{}%
\end{pgfscope}%
\begin{pgfscope}%
\pgfsys@transformshift{1.593304in}{1.751961in}%
\pgfsys@useobject{currentmarker}{}%
\end{pgfscope}%
\begin{pgfscope}%
\pgfsys@transformshift{1.612636in}{1.716587in}%
\pgfsys@useobject{currentmarker}{}%
\end{pgfscope}%
\begin{pgfscope}%
\pgfsys@transformshift{1.631969in}{1.721640in}%
\pgfsys@useobject{currentmarker}{}%
\end{pgfscope}%
\begin{pgfscope}%
\pgfsys@transformshift{1.651301in}{1.765226in}%
\pgfsys@useobject{currentmarker}{}%
\end{pgfscope}%
\begin{pgfscope}%
\pgfsys@transformshift{1.670633in}{1.678686in}%
\pgfsys@useobject{currentmarker}{}%
\end{pgfscope}%
\begin{pgfscope}%
\pgfsys@transformshift{1.689965in}{1.750066in}%
\pgfsys@useobject{currentmarker}{}%
\end{pgfscope}%
\begin{pgfscope}%
\pgfsys@transformshift{1.709298in}{1.705848in}%
\pgfsys@useobject{currentmarker}{}%
\end{pgfscope}%
\begin{pgfscope}%
\pgfsys@transformshift{1.728630in}{1.607938in}%
\pgfsys@useobject{currentmarker}{}%
\end{pgfscope}%
\begin{pgfscope}%
\pgfsys@transformshift{1.747962in}{1.624361in}%
\pgfsys@useobject{currentmarker}{}%
\end{pgfscope}%
\begin{pgfscope}%
\pgfsys@transformshift{1.767295in}{1.514449in}%
\pgfsys@useobject{currentmarker}{}%
\end{pgfscope}%
\begin{pgfscope}%
\pgfsys@transformshift{1.786627in}{1.525819in}%
\pgfsys@useobject{currentmarker}{}%
\end{pgfscope}%
\begin{pgfscope}%
\pgfsys@transformshift{1.805959in}{1.507501in}%
\pgfsys@useobject{currentmarker}{}%
\end{pgfscope}%
\begin{pgfscope}%
\pgfsys@transformshift{1.825291in}{1.442438in}%
\pgfsys@useobject{currentmarker}{}%
\end{pgfscope}%
\begin{pgfscope}%
\pgfsys@transformshift{1.844624in}{1.386850in}%
\pgfsys@useobject{currentmarker}{}%
\end{pgfscope}%
\begin{pgfscope}%
\pgfsys@transformshift{1.863956in}{1.360951in}%
\pgfsys@useobject{currentmarker}{}%
\end{pgfscope}%
\begin{pgfscope}%
\pgfsys@transformshift{1.883288in}{1.323050in}%
\pgfsys@useobject{currentmarker}{}%
\end{pgfscope}%
\begin{pgfscope}%
\pgfsys@transformshift{1.902621in}{1.247249in}%
\pgfsys@useobject{currentmarker}{}%
\end{pgfscope}%
\begin{pgfscope}%
\pgfsys@transformshift{1.921953in}{1.221350in}%
\pgfsys@useobject{currentmarker}{}%
\end{pgfscope}%
\begin{pgfscope}%
\pgfsys@transformshift{1.941285in}{1.177132in}%
\pgfsys@useobject{currentmarker}{}%
\end{pgfscope}%
\begin{pgfscope}%
\pgfsys@transformshift{1.960617in}{1.133547in}%
\pgfsys@useobject{currentmarker}{}%
\end{pgfscope}%
\begin{pgfscope}%
\pgfsys@transformshift{1.979950in}{1.087434in}%
\pgfsys@useobject{currentmarker}{}%
\end{pgfscope}%
\begin{pgfscope}%
\pgfsys@transformshift{1.999282in}{1.063430in}%
\pgfsys@useobject{currentmarker}{}%
\end{pgfscope}%
\begin{pgfscope}%
\pgfsys@transformshift{2.018614in}{1.047638in}%
\pgfsys@useobject{currentmarker}{}%
\end{pgfscope}%
\begin{pgfscope}%
\pgfsys@transformshift{2.037947in}{1.002157in}%
\pgfsys@useobject{currentmarker}{}%
\end{pgfscope}%
\begin{pgfscope}%
\pgfsys@transformshift{2.057279in}{0.979417in}%
\pgfsys@useobject{currentmarker}{}%
\end{pgfscope}%
\begin{pgfscope}%
\pgfsys@transformshift{2.076611in}{0.948465in}%
\pgfsys@useobject{currentmarker}{}%
\end{pgfscope}%
\begin{pgfscope}%
\pgfsys@transformshift{2.095943in}{0.953518in}%
\pgfsys@useobject{currentmarker}{}%
\end{pgfscope}%
\begin{pgfscope}%
\pgfsys@transformshift{2.115276in}{0.894772in}%
\pgfsys@useobject{currentmarker}{}%
\end{pgfscope}%
\begin{pgfscope}%
\pgfsys@transformshift{2.134608in}{0.887823in}%
\pgfsys@useobject{currentmarker}{}%
\end{pgfscope}%
\begin{pgfscope}%
\pgfsys@transformshift{2.153940in}{0.872663in}%
\pgfsys@useobject{currentmarker}{}%
\end{pgfscope}%
\begin{pgfscope}%
\pgfsys@transformshift{2.173273in}{0.840447in}%
\pgfsys@useobject{currentmarker}{}%
\end{pgfscope}%
\begin{pgfscope}%
\pgfsys@transformshift{2.192605in}{0.842974in}%
\pgfsys@useobject{currentmarker}{}%
\end{pgfscope}%
\begin{pgfscope}%
\pgfsys@transformshift{2.211937in}{0.820234in}%
\pgfsys@useobject{currentmarker}{}%
\end{pgfscope}%
\begin{pgfscope}%
\pgfsys@transformshift{2.231269in}{0.813917in}%
\pgfsys@useobject{currentmarker}{}%
\end{pgfscope}%
\begin{pgfscope}%
\pgfsys@transformshift{2.250602in}{0.791176in}%
\pgfsys@useobject{currentmarker}{}%
\end{pgfscope}%
\begin{pgfscope}%
\pgfsys@transformshift{2.269934in}{0.785491in}%
\pgfsys@useobject{currentmarker}{}%
\end{pgfscope}%
\begin{pgfscope}%
\pgfsys@transformshift{2.289266in}{0.791176in}%
\pgfsys@useobject{currentmarker}{}%
\end{pgfscope}%
\begin{pgfscope}%
\pgfsys@transformshift{2.308599in}{0.781701in}%
\pgfsys@useobject{currentmarker}{}%
\end{pgfscope}%
\begin{pgfscope}%
\pgfsys@transformshift{2.327931in}{0.764014in}%
\pgfsys@useobject{currentmarker}{}%
\end{pgfscope}%
\begin{pgfscope}%
\pgfsys@transformshift{2.347263in}{0.757697in}%
\pgfsys@useobject{currentmarker}{}%
\end{pgfscope}%
\begin{pgfscope}%
\pgfsys@transformshift{2.366595in}{0.750749in}%
\pgfsys@useobject{currentmarker}{}%
\end{pgfscope}%
\begin{pgfscope}%
\pgfsys@transformshift{2.385928in}{0.749486in}%
\pgfsys@useobject{currentmarker}{}%
\end{pgfscope}%
\begin{pgfscope}%
\pgfsys@transformshift{2.405260in}{0.736220in}%
\pgfsys@useobject{currentmarker}{}%
\end{pgfscope}%
\begin{pgfscope}%
\pgfsys@transformshift{2.424592in}{0.741274in}%
\pgfsys@useobject{currentmarker}{}%
\end{pgfscope}%
\begin{pgfscope}%
\pgfsys@transformshift{2.443925in}{0.739379in}%
\pgfsys@useobject{currentmarker}{}%
\end{pgfscope}%
\begin{pgfscope}%
\pgfsys@transformshift{2.463257in}{0.732430in}%
\pgfsys@useobject{currentmarker}{}%
\end{pgfscope}%
\begin{pgfscope}%
\pgfsys@transformshift{2.482589in}{0.732430in}%
\pgfsys@useobject{currentmarker}{}%
\end{pgfscope}%
\begin{pgfscope}%
\pgfsys@transformshift{2.501921in}{0.730535in}%
\pgfsys@useobject{currentmarker}{}%
\end{pgfscope}%
\begin{pgfscope}%
\pgfsys@transformshift{2.521254in}{0.725482in}%
\pgfsys@useobject{currentmarker}{}%
\end{pgfscope}%
\begin{pgfscope}%
\pgfsys@transformshift{2.540586in}{0.721692in}%
\pgfsys@useobject{currentmarker}{}%
\end{pgfscope}%
\begin{pgfscope}%
\pgfsys@transformshift{2.559918in}{0.723587in}%
\pgfsys@useobject{currentmarker}{}%
\end{pgfscope}%
\end{pgfscope}%
\end{pgfpicture}%
\makeatother%
\endgroup%

  \end{subfigure}
  \caption{
    Fit to the reconstructed \PBzero (left) and \APDzero (right) masses of simulated $B^0\to\APDzero\APmuon\Pmuon$ decays.
    The two sub-components of the model are shown as blue and green dashed lines, while the total signal model is shown as a solid red line.
  }
  \label{fig:mcfitb}
\end{figure}

\begin{table}
  \centering
  \caption{MC fit results}
  \begin{tabular}{l l S[table-format=4.2,table-figures-uncertainty=1]}
    \toprule
    Dim. & Parameter & {Estimate} \\
    \midrule
    $B^0$ & $f$ & 0.37 \pm 0.05 \\
    & $\mu$ & 5280.8 \pm 0.07 \\
    & $\sigma_1$ & 21.3 \pm 0.7 \\
    & $\sigma_2$ & 13.0 \pm 0.3 \\
    \midrule
    $\overline{D}^0$ & $f$ & 0.61 \pm 0.04 \\
    & $\mu$ & 1865.3 \pm 0.03 \\
    & $\sigma_1$ & 5.88 \pm 0.13 \\
    & $\sigma_2$ & 10.91 \pm 0.35 \\
    \bottomrule
    % TODO fix number of figures
  \end{tabular}
  \label{tab:mcfit}
\end{table}

\section{Background models}
\label{backgroundmodel}

The background model takes into account the presence of combinatorial backgrounds with resonant and non-resonant behaviour in the reconstructed $\overline{D}^0$ mass.

The \emph{flat} (non-resonant) component is defined as a product of two exponential distributions
\begin{equation}
  p_\text{flat}(m_B,m_D|λ_B,λ_D) = \mathup{Exp}(m_B|λ_B) \times \mathup{Exp}(m_D|λ_D)\:.
\end{equation}
Here, the $\mathup{Exp}$ distribution is defined by
\begin{equation}
  \mathup{Exp}(x | λ) = \frac{\mathrm{exp}(-λx)}{\int_\text{mass range} \mathup{exp}(-λx) dx}\:.
\end{equation}

The \emph{peaking} (resonant) component is represented by an exponential function in the $m_B$ dimension and a normal function in the $m_D$ dimension:
\begin{equation}
  \begin{split}
    p_\text{peaking}(m_B,m_D|λ_B,\mu_D,\sigma_D) =  \\ \mathup{Exp}(m_B|λ_B) \times\mathup{Normal}(m_D|\mu_D,\sigma_D)\:.
  \end{split}
\end{equation}

The two components are added using weights $f$ and $(1-f)$ to obtain the total background model.
A fit to the left and right sidebands of the $\PBzero\to\APDzero\APmuon\Pmuon$ distribution is carried out (see figure \ref{fig:bkgfitb}).
The blinded signal region is excluded from the fit by correcting the likelihood by the value of the integral of the blinded region.
The resulting parameter estimates are given in table \ref{tab:bkgfit}.

\begin{figure}
  \centering
  \begin{subfigure}[t]{0.8\textwidth}
    \centering
    %% Creator: Matplotlib, PGF backend
%%
%% To include the figure in your LaTeX document, write
%%   \input{<filename>.pgf}
%%
%% Make sure the required packages are loaded in your preamble
%%   \usepackage{pgf}
%%
%% Figures using additional raster images can only be included by \input if
%% they are in the same directory as the main LaTeX file. For loading figures
%% from other directories you can use the `import` package
%%   \usepackage{import}
%% and then include the figures with
%%   \import{<path to file>}{<filename>.pgf}
%%
%% Matplotlib used the following preamble
%%   \usepackage{fontspec}
%%   \setmainfont{DejaVu Serif}
%%   \setsansfont{DejaVu Sans}
%%   \setmonofont{DejaVu Sans Mono}
%%
\begingroup%
\makeatletter%
\begin{pgfpicture}%
\pgfpathrectangle{\pgfpointorigin}{\pgfqpoint{4.049788in}{2.872910in}}%
\pgfusepath{use as bounding box, clip}%
\begin{pgfscope}%
\pgfsetbuttcap%
\pgfsetmiterjoin%
\definecolor{currentfill}{rgb}{1.000000,1.000000,1.000000}%
\pgfsetfillcolor{currentfill}%
\pgfsetlinewidth{0.000000pt}%
\definecolor{currentstroke}{rgb}{1.000000,1.000000,1.000000}%
\pgfsetstrokecolor{currentstroke}%
\pgfsetdash{}{0pt}%
\pgfpathmoveto{\pgfqpoint{0.000000in}{0.000000in}}%
\pgfpathlineto{\pgfqpoint{4.049788in}{0.000000in}}%
\pgfpathlineto{\pgfqpoint{4.049788in}{2.872910in}}%
\pgfpathlineto{\pgfqpoint{0.000000in}{2.872910in}}%
\pgfpathclose%
\pgfusepath{fill}%
\end{pgfscope}%
\begin{pgfscope}%
\pgfsetbuttcap%
\pgfsetmiterjoin%
\definecolor{currentfill}{rgb}{1.000000,1.000000,1.000000}%
\pgfsetfillcolor{currentfill}%
\pgfsetlinewidth{0.000000pt}%
\definecolor{currentstroke}{rgb}{0.000000,0.000000,0.000000}%
\pgfsetstrokecolor{currentstroke}%
\pgfsetstrokeopacity{0.000000}%
\pgfsetdash{}{0pt}%
\pgfpathmoveto{\pgfqpoint{0.636356in}{0.440955in}}%
\pgfpathlineto{\pgfqpoint{3.858404in}{0.440955in}}%
\pgfpathlineto{\pgfqpoint{3.858404in}{0.711675in}}%
\pgfpathlineto{\pgfqpoint{0.636356in}{0.711675in}}%
\pgfpathclose%
\pgfusepath{fill}%
\end{pgfscope}%
\begin{pgfscope}%
\pgfpathrectangle{\pgfqpoint{0.636356in}{0.440955in}}{\pgfqpoint{3.222048in}{0.270720in}} %
\pgfusepath{clip}%
\pgfsetbuttcap%
\pgfsetroundjoin%
\definecolor{currentfill}{rgb}{0.733333,0.733333,0.733333}%
\pgfsetfillcolor{currentfill}%
\pgfsetlinewidth{0.000000pt}%
\definecolor{currentstroke}{rgb}{0.733333,0.733333,0.733333}%
\pgfsetstrokecolor{currentstroke}%
\pgfsetdash{}{0pt}%
\pgfpathmoveto{\pgfqpoint{0.636356in}{0.621435in}}%
\pgfpathlineto{\pgfqpoint{0.636356in}{0.666555in}}%
\pgfpathlineto{\pgfqpoint{3.858404in}{0.666555in}}%
\pgfpathlineto{\pgfqpoint{3.858404in}{0.621435in}}%
\pgfpathlineto{\pgfqpoint{3.858404in}{0.621435in}}%
\pgfpathlineto{\pgfqpoint{0.636356in}{0.621435in}}%
\pgfpathlineto{\pgfqpoint{0.636356in}{0.621435in}}%
\pgfusepath{fill}%
\end{pgfscope}%
\begin{pgfscope}%
\pgfpathrectangle{\pgfqpoint{0.636356in}{0.440955in}}{\pgfqpoint{3.222048in}{0.270720in}} %
\pgfusepath{clip}%
\pgfsetbuttcap%
\pgfsetroundjoin%
\definecolor{currentfill}{rgb}{0.733333,0.733333,0.733333}%
\pgfsetfillcolor{currentfill}%
\pgfsetlinewidth{0.000000pt}%
\definecolor{currentstroke}{rgb}{0.733333,0.733333,0.733333}%
\pgfsetstrokecolor{currentstroke}%
\pgfsetdash{}{0pt}%
\pgfpathmoveto{\pgfqpoint{0.636356in}{0.531195in}}%
\pgfpathlineto{\pgfqpoint{0.636356in}{0.486075in}}%
\pgfpathlineto{\pgfqpoint{3.858404in}{0.486075in}}%
\pgfpathlineto{\pgfqpoint{3.858404in}{0.531195in}}%
\pgfpathlineto{\pgfqpoint{3.858404in}{0.531195in}}%
\pgfpathlineto{\pgfqpoint{0.636356in}{0.531195in}}%
\pgfpathlineto{\pgfqpoint{0.636356in}{0.531195in}}%
\pgfusepath{fill}%
\end{pgfscope}%
\begin{pgfscope}%
\pgfpathrectangle{\pgfqpoint{0.636356in}{0.440955in}}{\pgfqpoint{3.222048in}{0.270720in}} %
\pgfusepath{clip}%
\pgfsetbuttcap%
\pgfsetmiterjoin%
\definecolor{currentfill}{rgb}{0.333333,0.333333,0.333333}%
\pgfsetfillcolor{currentfill}%
\pgfsetlinewidth{0.501875pt}%
\definecolor{currentstroke}{rgb}{0.000000,0.000000,0.000000}%
\pgfsetstrokecolor{currentstroke}%
\pgfsetdash{}{0pt}%
\pgfpathmoveto{\pgfqpoint{0.636356in}{0.555714in}}%
\pgfpathlineto{\pgfqpoint{0.676631in}{0.555714in}}%
\pgfpathlineto{\pgfqpoint{0.676631in}{0.576315in}}%
\pgfpathlineto{\pgfqpoint{0.636356in}{0.576315in}}%
\pgfpathlineto{\pgfqpoint{0.636356in}{0.555714in}}%
\pgfusepath{stroke,fill}%
\end{pgfscope}%
\begin{pgfscope}%
\pgfpathrectangle{\pgfqpoint{0.636356in}{0.440955in}}{\pgfqpoint{3.222048in}{0.270720in}} %
\pgfusepath{clip}%
\pgfsetbuttcap%
\pgfsetmiterjoin%
\definecolor{currentfill}{rgb}{0.333333,0.333333,0.333333}%
\pgfsetfillcolor{currentfill}%
\pgfsetlinewidth{0.501875pt}%
\definecolor{currentstroke}{rgb}{0.000000,0.000000,0.000000}%
\pgfsetstrokecolor{currentstroke}%
\pgfsetdash{}{0pt}%
\pgfpathmoveto{\pgfqpoint{0.676631in}{0.576315in}}%
\pgfpathlineto{\pgfqpoint{0.716907in}{0.576315in}}%
\pgfpathlineto{\pgfqpoint{0.716907in}{0.590703in}}%
\pgfpathlineto{\pgfqpoint{0.676631in}{0.590703in}}%
\pgfpathlineto{\pgfqpoint{0.676631in}{0.576315in}}%
\pgfusepath{stroke,fill}%
\end{pgfscope}%
\begin{pgfscope}%
\pgfpathrectangle{\pgfqpoint{0.636356in}{0.440955in}}{\pgfqpoint{3.222048in}{0.270720in}} %
\pgfusepath{clip}%
\pgfsetbuttcap%
\pgfsetmiterjoin%
\definecolor{currentfill}{rgb}{0.333333,0.333333,0.333333}%
\pgfsetfillcolor{currentfill}%
\pgfsetlinewidth{0.501875pt}%
\definecolor{currentstroke}{rgb}{0.000000,0.000000,0.000000}%
\pgfsetstrokecolor{currentstroke}%
\pgfsetdash{}{0pt}%
\pgfpathmoveto{\pgfqpoint{0.716907in}{0.513196in}}%
\pgfpathlineto{\pgfqpoint{0.757183in}{0.513196in}}%
\pgfpathlineto{\pgfqpoint{0.757183in}{0.576315in}}%
\pgfpathlineto{\pgfqpoint{0.716907in}{0.576315in}}%
\pgfpathlineto{\pgfqpoint{0.716907in}{0.513196in}}%
\pgfusepath{stroke,fill}%
\end{pgfscope}%
\begin{pgfscope}%
\pgfpathrectangle{\pgfqpoint{0.636356in}{0.440955in}}{\pgfqpoint{3.222048in}{0.270720in}} %
\pgfusepath{clip}%
\pgfsetbuttcap%
\pgfsetmiterjoin%
\definecolor{currentfill}{rgb}{0.333333,0.333333,0.333333}%
\pgfsetfillcolor{currentfill}%
\pgfsetlinewidth{0.501875pt}%
\definecolor{currentstroke}{rgb}{0.000000,0.000000,0.000000}%
\pgfsetstrokecolor{currentstroke}%
\pgfsetdash{}{0pt}%
\pgfpathmoveto{\pgfqpoint{0.757183in}{0.540506in}}%
\pgfpathlineto{\pgfqpoint{0.797458in}{0.540506in}}%
\pgfpathlineto{\pgfqpoint{0.797458in}{0.576315in}}%
\pgfpathlineto{\pgfqpoint{0.757183in}{0.576315in}}%
\pgfpathlineto{\pgfqpoint{0.757183in}{0.540506in}}%
\pgfusepath{stroke,fill}%
\end{pgfscope}%
\begin{pgfscope}%
\pgfpathrectangle{\pgfqpoint{0.636356in}{0.440955in}}{\pgfqpoint{3.222048in}{0.270720in}} %
\pgfusepath{clip}%
\pgfsetbuttcap%
\pgfsetmiterjoin%
\definecolor{currentfill}{rgb}{0.333333,0.333333,0.333333}%
\pgfsetfillcolor{currentfill}%
\pgfsetlinewidth{0.501875pt}%
\definecolor{currentstroke}{rgb}{0.000000,0.000000,0.000000}%
\pgfsetstrokecolor{currentstroke}%
\pgfsetdash{}{0pt}%
\pgfpathmoveto{\pgfqpoint{0.797458in}{0.576315in}}%
\pgfpathlineto{\pgfqpoint{0.837734in}{0.576315in}}%
\pgfpathlineto{\pgfqpoint{0.837734in}{0.623439in}}%
\pgfpathlineto{\pgfqpoint{0.797458in}{0.623439in}}%
\pgfpathlineto{\pgfqpoint{0.797458in}{0.576315in}}%
\pgfusepath{stroke,fill}%
\end{pgfscope}%
\begin{pgfscope}%
\pgfpathrectangle{\pgfqpoint{0.636356in}{0.440955in}}{\pgfqpoint{3.222048in}{0.270720in}} %
\pgfusepath{clip}%
\pgfsetbuttcap%
\pgfsetmiterjoin%
\definecolor{currentfill}{rgb}{0.333333,0.333333,0.333333}%
\pgfsetfillcolor{currentfill}%
\pgfsetlinewidth{0.501875pt}%
\definecolor{currentstroke}{rgb}{0.000000,0.000000,0.000000}%
\pgfsetstrokecolor{currentstroke}%
\pgfsetdash{}{0pt}%
\pgfpathmoveto{\pgfqpoint{0.837734in}{0.576315in}}%
\pgfpathlineto{\pgfqpoint{0.878009in}{0.576315in}}%
\pgfpathlineto{\pgfqpoint{0.878009in}{0.620039in}}%
\pgfpathlineto{\pgfqpoint{0.837734in}{0.620039in}}%
\pgfpathlineto{\pgfqpoint{0.837734in}{0.576315in}}%
\pgfusepath{stroke,fill}%
\end{pgfscope}%
\begin{pgfscope}%
\pgfpathrectangle{\pgfqpoint{0.636356in}{0.440955in}}{\pgfqpoint{3.222048in}{0.270720in}} %
\pgfusepath{clip}%
\pgfsetbuttcap%
\pgfsetmiterjoin%
\definecolor{currentfill}{rgb}{0.333333,0.333333,0.333333}%
\pgfsetfillcolor{currentfill}%
\pgfsetlinewidth{0.501875pt}%
\definecolor{currentstroke}{rgb}{0.000000,0.000000,0.000000}%
\pgfsetstrokecolor{currentstroke}%
\pgfsetdash{}{0pt}%
\pgfpathmoveto{\pgfqpoint{0.878009in}{0.576315in}}%
\pgfpathlineto{\pgfqpoint{0.918285in}{0.576315in}}%
\pgfpathlineto{\pgfqpoint{0.918285in}{0.614957in}}%
\pgfpathlineto{\pgfqpoint{0.878009in}{0.614957in}}%
\pgfpathlineto{\pgfqpoint{0.878009in}{0.576315in}}%
\pgfusepath{stroke,fill}%
\end{pgfscope}%
\begin{pgfscope}%
\pgfpathrectangle{\pgfqpoint{0.636356in}{0.440955in}}{\pgfqpoint{3.222048in}{0.270720in}} %
\pgfusepath{clip}%
\pgfsetbuttcap%
\pgfsetmiterjoin%
\definecolor{currentfill}{rgb}{0.333333,0.333333,0.333333}%
\pgfsetfillcolor{currentfill}%
\pgfsetlinewidth{0.501875pt}%
\definecolor{currentstroke}{rgb}{0.000000,0.000000,0.000000}%
\pgfsetstrokecolor{currentstroke}%
\pgfsetdash{}{0pt}%
\pgfpathmoveto{\pgfqpoint{0.918285in}{0.562912in}}%
\pgfpathlineto{\pgfqpoint{0.958561in}{0.562912in}}%
\pgfpathlineto{\pgfqpoint{0.958561in}{0.576315in}}%
\pgfpathlineto{\pgfqpoint{0.918285in}{0.576315in}}%
\pgfpathlineto{\pgfqpoint{0.918285in}{0.562912in}}%
\pgfusepath{stroke,fill}%
\end{pgfscope}%
\begin{pgfscope}%
\pgfpathrectangle{\pgfqpoint{0.636356in}{0.440955in}}{\pgfqpoint{3.222048in}{0.270720in}} %
\pgfusepath{clip}%
\pgfsetbuttcap%
\pgfsetmiterjoin%
\definecolor{currentfill}{rgb}{0.333333,0.333333,0.333333}%
\pgfsetfillcolor{currentfill}%
\pgfsetlinewidth{0.501875pt}%
\definecolor{currentstroke}{rgb}{0.000000,0.000000,0.000000}%
\pgfsetstrokecolor{currentstroke}%
\pgfsetdash{}{0pt}%
\pgfpathmoveto{\pgfqpoint{0.958561in}{0.573998in}}%
\pgfpathlineto{\pgfqpoint{0.998836in}{0.573998in}}%
\pgfpathlineto{\pgfqpoint{0.998836in}{0.576315in}}%
\pgfpathlineto{\pgfqpoint{0.958561in}{0.576315in}}%
\pgfpathlineto{\pgfqpoint{0.958561in}{0.573998in}}%
\pgfusepath{stroke,fill}%
\end{pgfscope}%
\begin{pgfscope}%
\pgfpathrectangle{\pgfqpoint{0.636356in}{0.440955in}}{\pgfqpoint{3.222048in}{0.270720in}} %
\pgfusepath{clip}%
\pgfsetbuttcap%
\pgfsetmiterjoin%
\definecolor{currentfill}{rgb}{0.333333,0.333333,0.333333}%
\pgfsetfillcolor{currentfill}%
\pgfsetlinewidth{0.501875pt}%
\definecolor{currentstroke}{rgb}{0.000000,0.000000,0.000000}%
\pgfsetstrokecolor{currentstroke}%
\pgfsetdash{}{0pt}%
\pgfpathmoveto{\pgfqpoint{0.998836in}{0.573789in}}%
\pgfpathlineto{\pgfqpoint{1.039112in}{0.573789in}}%
\pgfpathlineto{\pgfqpoint{1.039112in}{0.576315in}}%
\pgfpathlineto{\pgfqpoint{0.998836in}{0.576315in}}%
\pgfpathlineto{\pgfqpoint{0.998836in}{0.573789in}}%
\pgfusepath{stroke,fill}%
\end{pgfscope}%
\begin{pgfscope}%
\pgfpathrectangle{\pgfqpoint{0.636356in}{0.440955in}}{\pgfqpoint{3.222048in}{0.270720in}} %
\pgfusepath{clip}%
\pgfsetbuttcap%
\pgfsetmiterjoin%
\definecolor{currentfill}{rgb}{0.333333,0.333333,0.333333}%
\pgfsetfillcolor{currentfill}%
\pgfsetlinewidth{0.501875pt}%
\definecolor{currentstroke}{rgb}{0.000000,0.000000,0.000000}%
\pgfsetstrokecolor{currentstroke}%
\pgfsetdash{}{0pt}%
\pgfpathmoveto{\pgfqpoint{1.039112in}{0.576315in}}%
\pgfpathlineto{\pgfqpoint{1.079387in}{0.576315in}}%
\pgfpathlineto{\pgfqpoint{1.079387in}{0.611467in}}%
\pgfpathlineto{\pgfqpoint{1.039112in}{0.611467in}}%
\pgfpathlineto{\pgfqpoint{1.039112in}{0.576315in}}%
\pgfusepath{stroke,fill}%
\end{pgfscope}%
\begin{pgfscope}%
\pgfpathrectangle{\pgfqpoint{0.636356in}{0.440955in}}{\pgfqpoint{3.222048in}{0.270720in}} %
\pgfusepath{clip}%
\pgfsetbuttcap%
\pgfsetmiterjoin%
\definecolor{currentfill}{rgb}{0.333333,0.333333,0.333333}%
\pgfsetfillcolor{currentfill}%
\pgfsetlinewidth{0.501875pt}%
\definecolor{currentstroke}{rgb}{0.000000,0.000000,0.000000}%
\pgfsetstrokecolor{currentstroke}%
\pgfsetdash{}{0pt}%
\pgfpathmoveto{\pgfqpoint{1.079387in}{0.529361in}}%
\pgfpathlineto{\pgfqpoint{1.119663in}{0.529361in}}%
\pgfpathlineto{\pgfqpoint{1.119663in}{0.576315in}}%
\pgfpathlineto{\pgfqpoint{1.079387in}{0.576315in}}%
\pgfpathlineto{\pgfqpoint{1.079387in}{0.529361in}}%
\pgfusepath{stroke,fill}%
\end{pgfscope}%
\begin{pgfscope}%
\pgfpathrectangle{\pgfqpoint{0.636356in}{0.440955in}}{\pgfqpoint{3.222048in}{0.270720in}} %
\pgfusepath{clip}%
\pgfsetbuttcap%
\pgfsetmiterjoin%
\definecolor{currentfill}{rgb}{0.333333,0.333333,0.333333}%
\pgfsetfillcolor{currentfill}%
\pgfsetlinewidth{0.501875pt}%
\definecolor{currentstroke}{rgb}{0.000000,0.000000,0.000000}%
\pgfsetstrokecolor{currentstroke}%
\pgfsetdash{}{0pt}%
\pgfpathmoveto{\pgfqpoint{1.119663in}{0.553983in}}%
\pgfpathlineto{\pgfqpoint{1.159939in}{0.553983in}}%
\pgfpathlineto{\pgfqpoint{1.159939in}{0.576315in}}%
\pgfpathlineto{\pgfqpoint{1.119663in}{0.576315in}}%
\pgfpathlineto{\pgfqpoint{1.119663in}{0.553983in}}%
\pgfusepath{stroke,fill}%
\end{pgfscope}%
\begin{pgfscope}%
\pgfpathrectangle{\pgfqpoint{0.636356in}{0.440955in}}{\pgfqpoint{3.222048in}{0.270720in}} %
\pgfusepath{clip}%
\pgfsetbuttcap%
\pgfsetmiterjoin%
\definecolor{currentfill}{rgb}{0.333333,0.333333,0.333333}%
\pgfsetfillcolor{currentfill}%
\pgfsetlinewidth{0.501875pt}%
\definecolor{currentstroke}{rgb}{0.000000,0.000000,0.000000}%
\pgfsetstrokecolor{currentstroke}%
\pgfsetdash{}{0pt}%
\pgfusepath{stroke,fill}%
\end{pgfscope}%
\begin{pgfscope}%
\pgfpathrectangle{\pgfqpoint{0.636356in}{0.440955in}}{\pgfqpoint{3.222048in}{0.270720in}} %
\pgfusepath{clip}%
\pgfsetbuttcap%
\pgfsetmiterjoin%
\definecolor{currentfill}{rgb}{0.333333,0.333333,0.333333}%
\pgfsetfillcolor{currentfill}%
\pgfsetlinewidth{0.501875pt}%
\definecolor{currentstroke}{rgb}{0.000000,0.000000,0.000000}%
\pgfsetstrokecolor{currentstroke}%
\pgfsetdash{}{0pt}%
\pgfusepath{stroke,fill}%
\end{pgfscope}%
\begin{pgfscope}%
\pgfpathrectangle{\pgfqpoint{0.636356in}{0.440955in}}{\pgfqpoint{3.222048in}{0.270720in}} %
\pgfusepath{clip}%
\pgfsetbuttcap%
\pgfsetmiterjoin%
\definecolor{currentfill}{rgb}{0.333333,0.333333,0.333333}%
\pgfsetfillcolor{currentfill}%
\pgfsetlinewidth{0.501875pt}%
\definecolor{currentstroke}{rgb}{0.000000,0.000000,0.000000}%
\pgfsetstrokecolor{currentstroke}%
\pgfsetdash{}{0pt}%
\pgfusepath{stroke,fill}%
\end{pgfscope}%
\begin{pgfscope}%
\pgfpathrectangle{\pgfqpoint{0.636356in}{0.440955in}}{\pgfqpoint{3.222048in}{0.270720in}} %
\pgfusepath{clip}%
\pgfsetbuttcap%
\pgfsetmiterjoin%
\definecolor{currentfill}{rgb}{0.333333,0.333333,0.333333}%
\pgfsetfillcolor{currentfill}%
\pgfsetlinewidth{0.501875pt}%
\definecolor{currentstroke}{rgb}{0.000000,0.000000,0.000000}%
\pgfsetstrokecolor{currentstroke}%
\pgfsetdash{}{0pt}%
\pgfpathmoveto{\pgfqpoint{1.280765in}{0.518205in}}%
\pgfpathlineto{\pgfqpoint{1.321041in}{0.518205in}}%
\pgfpathlineto{\pgfqpoint{1.321041in}{0.576315in}}%
\pgfpathlineto{\pgfqpoint{1.280765in}{0.576315in}}%
\pgfpathlineto{\pgfqpoint{1.280765in}{0.518205in}}%
\pgfusepath{stroke,fill}%
\end{pgfscope}%
\begin{pgfscope}%
\pgfpathrectangle{\pgfqpoint{0.636356in}{0.440955in}}{\pgfqpoint{3.222048in}{0.270720in}} %
\pgfusepath{clip}%
\pgfsetbuttcap%
\pgfsetmiterjoin%
\definecolor{currentfill}{rgb}{0.333333,0.333333,0.333333}%
\pgfsetfillcolor{currentfill}%
\pgfsetlinewidth{0.501875pt}%
\definecolor{currentstroke}{rgb}{0.000000,0.000000,0.000000}%
\pgfsetstrokecolor{currentstroke}%
\pgfsetdash{}{0pt}%
\pgfpathmoveto{\pgfqpoint{1.321041in}{0.576315in}}%
\pgfpathlineto{\pgfqpoint{1.361317in}{0.576315in}}%
\pgfpathlineto{\pgfqpoint{1.361317in}{0.626383in}}%
\pgfpathlineto{\pgfqpoint{1.321041in}{0.626383in}}%
\pgfpathlineto{\pgfqpoint{1.321041in}{0.576315in}}%
\pgfusepath{stroke,fill}%
\end{pgfscope}%
\begin{pgfscope}%
\pgfpathrectangle{\pgfqpoint{0.636356in}{0.440955in}}{\pgfqpoint{3.222048in}{0.270720in}} %
\pgfusepath{clip}%
\pgfsetbuttcap%
\pgfsetmiterjoin%
\definecolor{currentfill}{rgb}{0.333333,0.333333,0.333333}%
\pgfsetfillcolor{currentfill}%
\pgfsetlinewidth{0.501875pt}%
\definecolor{currentstroke}{rgb}{0.000000,0.000000,0.000000}%
\pgfsetstrokecolor{currentstroke}%
\pgfsetdash{}{0pt}%
\pgfpathmoveto{\pgfqpoint{1.361317in}{0.576315in}}%
\pgfpathlineto{\pgfqpoint{1.401592in}{0.576315in}}%
\pgfpathlineto{\pgfqpoint{1.401592in}{0.586737in}}%
\pgfpathlineto{\pgfqpoint{1.361317in}{0.586737in}}%
\pgfpathlineto{\pgfqpoint{1.361317in}{0.576315in}}%
\pgfusepath{stroke,fill}%
\end{pgfscope}%
\begin{pgfscope}%
\pgfpathrectangle{\pgfqpoint{0.636356in}{0.440955in}}{\pgfqpoint{3.222048in}{0.270720in}} %
\pgfusepath{clip}%
\pgfsetbuttcap%
\pgfsetmiterjoin%
\definecolor{currentfill}{rgb}{0.333333,0.333333,0.333333}%
\pgfsetfillcolor{currentfill}%
\pgfsetlinewidth{0.501875pt}%
\definecolor{currentstroke}{rgb}{0.000000,0.000000,0.000000}%
\pgfsetstrokecolor{currentstroke}%
\pgfsetdash{}{0pt}%
\pgfpathmoveto{\pgfqpoint{1.401592in}{0.559155in}}%
\pgfpathlineto{\pgfqpoint{1.441868in}{0.559155in}}%
\pgfpathlineto{\pgfqpoint{1.441868in}{0.576315in}}%
\pgfpathlineto{\pgfqpoint{1.401592in}{0.576315in}}%
\pgfpathlineto{\pgfqpoint{1.401592in}{0.559155in}}%
\pgfusepath{stroke,fill}%
\end{pgfscope}%
\begin{pgfscope}%
\pgfpathrectangle{\pgfqpoint{0.636356in}{0.440955in}}{\pgfqpoint{3.222048in}{0.270720in}} %
\pgfusepath{clip}%
\pgfsetbuttcap%
\pgfsetmiterjoin%
\definecolor{currentfill}{rgb}{0.333333,0.333333,0.333333}%
\pgfsetfillcolor{currentfill}%
\pgfsetlinewidth{0.501875pt}%
\definecolor{currentstroke}{rgb}{0.000000,0.000000,0.000000}%
\pgfsetstrokecolor{currentstroke}%
\pgfsetdash{}{0pt}%
\pgfpathmoveto{\pgfqpoint{1.441868in}{0.576315in}}%
\pgfpathlineto{\pgfqpoint{1.482143in}{0.576315in}}%
\pgfpathlineto{\pgfqpoint{1.482143in}{0.602309in}}%
\pgfpathlineto{\pgfqpoint{1.441868in}{0.602309in}}%
\pgfpathlineto{\pgfqpoint{1.441868in}{0.576315in}}%
\pgfusepath{stroke,fill}%
\end{pgfscope}%
\begin{pgfscope}%
\pgfpathrectangle{\pgfqpoint{0.636356in}{0.440955in}}{\pgfqpoint{3.222048in}{0.270720in}} %
\pgfusepath{clip}%
\pgfsetbuttcap%
\pgfsetmiterjoin%
\definecolor{currentfill}{rgb}{0.333333,0.333333,0.333333}%
\pgfsetfillcolor{currentfill}%
\pgfsetlinewidth{0.501875pt}%
\definecolor{currentstroke}{rgb}{0.000000,0.000000,0.000000}%
\pgfsetstrokecolor{currentstroke}%
\pgfsetdash{}{0pt}%
\pgfpathmoveto{\pgfqpoint{1.482143in}{0.441467in}}%
\pgfpathlineto{\pgfqpoint{1.522419in}{0.441467in}}%
\pgfpathlineto{\pgfqpoint{1.522419in}{0.576315in}}%
\pgfpathlineto{\pgfqpoint{1.482143in}{0.576315in}}%
\pgfpathlineto{\pgfqpoint{1.482143in}{0.441467in}}%
\pgfusepath{stroke,fill}%
\end{pgfscope}%
\begin{pgfscope}%
\pgfpathrectangle{\pgfqpoint{0.636356in}{0.440955in}}{\pgfqpoint{3.222048in}{0.270720in}} %
\pgfusepath{clip}%
\pgfsetbuttcap%
\pgfsetmiterjoin%
\definecolor{currentfill}{rgb}{0.333333,0.333333,0.333333}%
\pgfsetfillcolor{currentfill}%
\pgfsetlinewidth{0.501875pt}%
\definecolor{currentstroke}{rgb}{0.000000,0.000000,0.000000}%
\pgfsetstrokecolor{currentstroke}%
\pgfsetdash{}{0pt}%
\pgfpathmoveto{\pgfqpoint{1.522419in}{0.574219in}}%
\pgfpathlineto{\pgfqpoint{1.562695in}{0.574219in}}%
\pgfpathlineto{\pgfqpoint{1.562695in}{0.576315in}}%
\pgfpathlineto{\pgfqpoint{1.522419in}{0.576315in}}%
\pgfpathlineto{\pgfqpoint{1.522419in}{0.574219in}}%
\pgfusepath{stroke,fill}%
\end{pgfscope}%
\begin{pgfscope}%
\pgfpathrectangle{\pgfqpoint{0.636356in}{0.440955in}}{\pgfqpoint{3.222048in}{0.270720in}} %
\pgfusepath{clip}%
\pgfsetbuttcap%
\pgfsetmiterjoin%
\definecolor{currentfill}{rgb}{0.333333,0.333333,0.333333}%
\pgfsetfillcolor{currentfill}%
\pgfsetlinewidth{0.501875pt}%
\definecolor{currentstroke}{rgb}{0.000000,0.000000,0.000000}%
\pgfsetstrokecolor{currentstroke}%
\pgfsetdash{}{0pt}%
\pgfpathmoveto{\pgfqpoint{1.562695in}{0.561510in}}%
\pgfpathlineto{\pgfqpoint{1.602970in}{0.561510in}}%
\pgfpathlineto{\pgfqpoint{1.602970in}{0.576315in}}%
\pgfpathlineto{\pgfqpoint{1.562695in}{0.576315in}}%
\pgfpathlineto{\pgfqpoint{1.562695in}{0.561510in}}%
\pgfusepath{stroke,fill}%
\end{pgfscope}%
\begin{pgfscope}%
\pgfpathrectangle{\pgfqpoint{0.636356in}{0.440955in}}{\pgfqpoint{3.222048in}{0.270720in}} %
\pgfusepath{clip}%
\pgfsetbuttcap%
\pgfsetmiterjoin%
\definecolor{currentfill}{rgb}{0.333333,0.333333,0.333333}%
\pgfsetfillcolor{currentfill}%
\pgfsetlinewidth{0.501875pt}%
\definecolor{currentstroke}{rgb}{0.000000,0.000000,0.000000}%
\pgfsetstrokecolor{currentstroke}%
\pgfsetdash{}{0pt}%
\pgfpathmoveto{\pgfqpoint{1.602970in}{0.565723in}}%
\pgfpathlineto{\pgfqpoint{1.643246in}{0.565723in}}%
\pgfpathlineto{\pgfqpoint{1.643246in}{0.576315in}}%
\pgfpathlineto{\pgfqpoint{1.602970in}{0.576315in}}%
\pgfpathlineto{\pgfqpoint{1.602970in}{0.565723in}}%
\pgfusepath{stroke,fill}%
\end{pgfscope}%
\begin{pgfscope}%
\pgfpathrectangle{\pgfqpoint{0.636356in}{0.440955in}}{\pgfqpoint{3.222048in}{0.270720in}} %
\pgfusepath{clip}%
\pgfsetbuttcap%
\pgfsetmiterjoin%
\definecolor{currentfill}{rgb}{0.333333,0.333333,0.333333}%
\pgfsetfillcolor{currentfill}%
\pgfsetlinewidth{0.501875pt}%
\definecolor{currentstroke}{rgb}{0.000000,0.000000,0.000000}%
\pgfsetstrokecolor{currentstroke}%
\pgfsetdash{}{0pt}%
\pgfpathmoveto{\pgfqpoint{1.643246in}{0.576315in}}%
\pgfpathlineto{\pgfqpoint{1.683521in}{0.576315in}}%
\pgfpathlineto{\pgfqpoint{1.683521in}{0.630741in}}%
\pgfpathlineto{\pgfqpoint{1.643246in}{0.630741in}}%
\pgfpathlineto{\pgfqpoint{1.643246in}{0.576315in}}%
\pgfusepath{stroke,fill}%
\end{pgfscope}%
\begin{pgfscope}%
\pgfpathrectangle{\pgfqpoint{0.636356in}{0.440955in}}{\pgfqpoint{3.222048in}{0.270720in}} %
\pgfusepath{clip}%
\pgfsetbuttcap%
\pgfsetmiterjoin%
\definecolor{currentfill}{rgb}{0.333333,0.333333,0.333333}%
\pgfsetfillcolor{currentfill}%
\pgfsetlinewidth{0.501875pt}%
\definecolor{currentstroke}{rgb}{0.000000,0.000000,0.000000}%
\pgfsetstrokecolor{currentstroke}%
\pgfsetdash{}{0pt}%
\pgfpathmoveto{\pgfqpoint{1.683521in}{0.576315in}}%
\pgfpathlineto{\pgfqpoint{1.723797in}{0.576315in}}%
\pgfpathlineto{\pgfqpoint{1.723797in}{0.618997in}}%
\pgfpathlineto{\pgfqpoint{1.683521in}{0.618997in}}%
\pgfpathlineto{\pgfqpoint{1.683521in}{0.576315in}}%
\pgfusepath{stroke,fill}%
\end{pgfscope}%
\begin{pgfscope}%
\pgfpathrectangle{\pgfqpoint{0.636356in}{0.440955in}}{\pgfqpoint{3.222048in}{0.270720in}} %
\pgfusepath{clip}%
\pgfsetbuttcap%
\pgfsetmiterjoin%
\definecolor{currentfill}{rgb}{0.333333,0.333333,0.333333}%
\pgfsetfillcolor{currentfill}%
\pgfsetlinewidth{0.501875pt}%
\definecolor{currentstroke}{rgb}{0.000000,0.000000,0.000000}%
\pgfsetstrokecolor{currentstroke}%
\pgfsetdash{}{0pt}%
\pgfpathmoveto{\pgfqpoint{1.723797in}{0.556642in}}%
\pgfpathlineto{\pgfqpoint{1.764073in}{0.556642in}}%
\pgfpathlineto{\pgfqpoint{1.764073in}{0.576315in}}%
\pgfpathlineto{\pgfqpoint{1.723797in}{0.576315in}}%
\pgfpathlineto{\pgfqpoint{1.723797in}{0.556642in}}%
\pgfusepath{stroke,fill}%
\end{pgfscope}%
\begin{pgfscope}%
\pgfpathrectangle{\pgfqpoint{0.636356in}{0.440955in}}{\pgfqpoint{3.222048in}{0.270720in}} %
\pgfusepath{clip}%
\pgfsetbuttcap%
\pgfsetmiterjoin%
\definecolor{currentfill}{rgb}{0.333333,0.333333,0.333333}%
\pgfsetfillcolor{currentfill}%
\pgfsetlinewidth{0.501875pt}%
\definecolor{currentstroke}{rgb}{0.000000,0.000000,0.000000}%
\pgfsetstrokecolor{currentstroke}%
\pgfsetdash{}{0pt}%
\pgfpathmoveto{\pgfqpoint{1.764073in}{0.576315in}}%
\pgfpathlineto{\pgfqpoint{1.804348in}{0.576315in}}%
\pgfpathlineto{\pgfqpoint{1.804348in}{0.627433in}}%
\pgfpathlineto{\pgfqpoint{1.764073in}{0.627433in}}%
\pgfpathlineto{\pgfqpoint{1.764073in}{0.576315in}}%
\pgfusepath{stroke,fill}%
\end{pgfscope}%
\begin{pgfscope}%
\pgfpathrectangle{\pgfqpoint{0.636356in}{0.440955in}}{\pgfqpoint{3.222048in}{0.270720in}} %
\pgfusepath{clip}%
\pgfsetbuttcap%
\pgfsetmiterjoin%
\definecolor{currentfill}{rgb}{0.333333,0.333333,0.333333}%
\pgfsetfillcolor{currentfill}%
\pgfsetlinewidth{0.501875pt}%
\definecolor{currentstroke}{rgb}{0.000000,0.000000,0.000000}%
\pgfsetstrokecolor{currentstroke}%
\pgfsetdash{}{0pt}%
\pgfpathmoveto{\pgfqpoint{1.804348in}{0.563060in}}%
\pgfpathlineto{\pgfqpoint{1.844624in}{0.563060in}}%
\pgfpathlineto{\pgfqpoint{1.844624in}{0.576315in}}%
\pgfpathlineto{\pgfqpoint{1.804348in}{0.576315in}}%
\pgfpathlineto{\pgfqpoint{1.804348in}{0.563060in}}%
\pgfusepath{stroke,fill}%
\end{pgfscope}%
\begin{pgfscope}%
\pgfpathrectangle{\pgfqpoint{0.636356in}{0.440955in}}{\pgfqpoint{3.222048in}{0.270720in}} %
\pgfusepath{clip}%
\pgfsetbuttcap%
\pgfsetmiterjoin%
\definecolor{currentfill}{rgb}{0.333333,0.333333,0.333333}%
\pgfsetfillcolor{currentfill}%
\pgfsetlinewidth{0.501875pt}%
\definecolor{currentstroke}{rgb}{0.000000,0.000000,0.000000}%
\pgfsetstrokecolor{currentstroke}%
\pgfsetdash{}{0pt}%
\pgfpathmoveto{\pgfqpoint{1.844624in}{0.565859in}}%
\pgfpathlineto{\pgfqpoint{1.884899in}{0.565859in}}%
\pgfpathlineto{\pgfqpoint{1.884899in}{0.576315in}}%
\pgfpathlineto{\pgfqpoint{1.844624in}{0.576315in}}%
\pgfpathlineto{\pgfqpoint{1.844624in}{0.565859in}}%
\pgfusepath{stroke,fill}%
\end{pgfscope}%
\begin{pgfscope}%
\pgfpathrectangle{\pgfqpoint{0.636356in}{0.440955in}}{\pgfqpoint{3.222048in}{0.270720in}} %
\pgfusepath{clip}%
\pgfsetbuttcap%
\pgfsetmiterjoin%
\definecolor{currentfill}{rgb}{0.333333,0.333333,0.333333}%
\pgfsetfillcolor{currentfill}%
\pgfsetlinewidth{0.501875pt}%
\definecolor{currentstroke}{rgb}{0.000000,0.000000,0.000000}%
\pgfsetstrokecolor{currentstroke}%
\pgfsetdash{}{0pt}%
\pgfpathmoveto{\pgfqpoint{1.884899in}{0.568415in}}%
\pgfpathlineto{\pgfqpoint{1.925175in}{0.568415in}}%
\pgfpathlineto{\pgfqpoint{1.925175in}{0.576315in}}%
\pgfpathlineto{\pgfqpoint{1.884899in}{0.576315in}}%
\pgfpathlineto{\pgfqpoint{1.884899in}{0.568415in}}%
\pgfusepath{stroke,fill}%
\end{pgfscope}%
\begin{pgfscope}%
\pgfpathrectangle{\pgfqpoint{0.636356in}{0.440955in}}{\pgfqpoint{3.222048in}{0.270720in}} %
\pgfusepath{clip}%
\pgfsetbuttcap%
\pgfsetmiterjoin%
\definecolor{currentfill}{rgb}{0.333333,0.333333,0.333333}%
\pgfsetfillcolor{currentfill}%
\pgfsetlinewidth{0.501875pt}%
\definecolor{currentstroke}{rgb}{0.000000,0.000000,0.000000}%
\pgfsetstrokecolor{currentstroke}%
\pgfsetdash{}{0pt}%
\pgfpathmoveto{\pgfqpoint{1.925175in}{0.576315in}}%
\pgfpathlineto{\pgfqpoint{1.965451in}{0.576315in}}%
\pgfpathlineto{\pgfqpoint{1.965451in}{0.601329in}}%
\pgfpathlineto{\pgfqpoint{1.925175in}{0.601329in}}%
\pgfpathlineto{\pgfqpoint{1.925175in}{0.576315in}}%
\pgfusepath{stroke,fill}%
\end{pgfscope}%
\begin{pgfscope}%
\pgfpathrectangle{\pgfqpoint{0.636356in}{0.440955in}}{\pgfqpoint{3.222048in}{0.270720in}} %
\pgfusepath{clip}%
\pgfsetbuttcap%
\pgfsetmiterjoin%
\definecolor{currentfill}{rgb}{0.333333,0.333333,0.333333}%
\pgfsetfillcolor{currentfill}%
\pgfsetlinewidth{0.501875pt}%
\definecolor{currentstroke}{rgb}{0.000000,0.000000,0.000000}%
\pgfsetstrokecolor{currentstroke}%
\pgfsetdash{}{0pt}%
\pgfpathmoveto{\pgfqpoint{1.965451in}{0.530137in}}%
\pgfpathlineto{\pgfqpoint{2.005726in}{0.530137in}}%
\pgfpathlineto{\pgfqpoint{2.005726in}{0.576315in}}%
\pgfpathlineto{\pgfqpoint{1.965451in}{0.576315in}}%
\pgfpathlineto{\pgfqpoint{1.965451in}{0.530137in}}%
\pgfusepath{stroke,fill}%
\end{pgfscope}%
\begin{pgfscope}%
\pgfpathrectangle{\pgfqpoint{0.636356in}{0.440955in}}{\pgfqpoint{3.222048in}{0.270720in}} %
\pgfusepath{clip}%
\pgfsetbuttcap%
\pgfsetmiterjoin%
\definecolor{currentfill}{rgb}{0.333333,0.333333,0.333333}%
\pgfsetfillcolor{currentfill}%
\pgfsetlinewidth{0.501875pt}%
\definecolor{currentstroke}{rgb}{0.000000,0.000000,0.000000}%
\pgfsetstrokecolor{currentstroke}%
\pgfsetdash{}{0pt}%
\pgfpathmoveto{\pgfqpoint{2.005726in}{0.576315in}}%
\pgfpathlineto{\pgfqpoint{2.046002in}{0.576315in}}%
\pgfpathlineto{\pgfqpoint{2.046002in}{0.608593in}}%
\pgfpathlineto{\pgfqpoint{2.005726in}{0.608593in}}%
\pgfpathlineto{\pgfqpoint{2.005726in}{0.576315in}}%
\pgfusepath{stroke,fill}%
\end{pgfscope}%
\begin{pgfscope}%
\pgfpathrectangle{\pgfqpoint{0.636356in}{0.440955in}}{\pgfqpoint{3.222048in}{0.270720in}} %
\pgfusepath{clip}%
\pgfsetbuttcap%
\pgfsetmiterjoin%
\definecolor{currentfill}{rgb}{0.333333,0.333333,0.333333}%
\pgfsetfillcolor{currentfill}%
\pgfsetlinewidth{0.501875pt}%
\definecolor{currentstroke}{rgb}{0.000000,0.000000,0.000000}%
\pgfsetstrokecolor{currentstroke}%
\pgfsetdash{}{0pt}%
\pgfpathmoveto{\pgfqpoint{2.046002in}{0.576315in}}%
\pgfpathlineto{\pgfqpoint{2.086277in}{0.576315in}}%
\pgfpathlineto{\pgfqpoint{2.086277in}{0.577143in}}%
\pgfpathlineto{\pgfqpoint{2.046002in}{0.577143in}}%
\pgfpathlineto{\pgfqpoint{2.046002in}{0.576315in}}%
\pgfusepath{stroke,fill}%
\end{pgfscope}%
\begin{pgfscope}%
\pgfpathrectangle{\pgfqpoint{0.636356in}{0.440955in}}{\pgfqpoint{3.222048in}{0.270720in}} %
\pgfusepath{clip}%
\pgfsetbuttcap%
\pgfsetmiterjoin%
\definecolor{currentfill}{rgb}{0.333333,0.333333,0.333333}%
\pgfsetfillcolor{currentfill}%
\pgfsetlinewidth{0.501875pt}%
\definecolor{currentstroke}{rgb}{0.000000,0.000000,0.000000}%
\pgfsetstrokecolor{currentstroke}%
\pgfsetdash{}{0pt}%
\pgfpathmoveto{\pgfqpoint{2.086277in}{0.576315in}}%
\pgfpathlineto{\pgfqpoint{2.126553in}{0.576315in}}%
\pgfpathlineto{\pgfqpoint{2.126553in}{0.581657in}}%
\pgfpathlineto{\pgfqpoint{2.086277in}{0.581657in}}%
\pgfpathlineto{\pgfqpoint{2.086277in}{0.576315in}}%
\pgfusepath{stroke,fill}%
\end{pgfscope}%
\begin{pgfscope}%
\pgfpathrectangle{\pgfqpoint{0.636356in}{0.440955in}}{\pgfqpoint{3.222048in}{0.270720in}} %
\pgfusepath{clip}%
\pgfsetbuttcap%
\pgfsetmiterjoin%
\definecolor{currentfill}{rgb}{0.333333,0.333333,0.333333}%
\pgfsetfillcolor{currentfill}%
\pgfsetlinewidth{0.501875pt}%
\definecolor{currentstroke}{rgb}{0.000000,0.000000,0.000000}%
\pgfsetstrokecolor{currentstroke}%
\pgfsetdash{}{0pt}%
\pgfpathmoveto{\pgfqpoint{2.126553in}{0.543842in}}%
\pgfpathlineto{\pgfqpoint{2.166828in}{0.543842in}}%
\pgfpathlineto{\pgfqpoint{2.166828in}{0.576315in}}%
\pgfpathlineto{\pgfqpoint{2.126553in}{0.576315in}}%
\pgfpathlineto{\pgfqpoint{2.126553in}{0.543842in}}%
\pgfusepath{stroke,fill}%
\end{pgfscope}%
\begin{pgfscope}%
\pgfpathrectangle{\pgfqpoint{0.636356in}{0.440955in}}{\pgfqpoint{3.222048in}{0.270720in}} %
\pgfusepath{clip}%
\pgfsetbuttcap%
\pgfsetmiterjoin%
\definecolor{currentfill}{rgb}{0.333333,0.333333,0.333333}%
\pgfsetfillcolor{currentfill}%
\pgfsetlinewidth{0.501875pt}%
\definecolor{currentstroke}{rgb}{0.000000,0.000000,0.000000}%
\pgfsetstrokecolor{currentstroke}%
\pgfsetdash{}{0pt}%
\pgfpathmoveto{\pgfqpoint{2.166828in}{0.576315in}}%
\pgfpathlineto{\pgfqpoint{2.207104in}{0.576315in}}%
\pgfpathlineto{\pgfqpoint{2.207104in}{0.589583in}}%
\pgfpathlineto{\pgfqpoint{2.166828in}{0.589583in}}%
\pgfpathlineto{\pgfqpoint{2.166828in}{0.576315in}}%
\pgfusepath{stroke,fill}%
\end{pgfscope}%
\begin{pgfscope}%
\pgfpathrectangle{\pgfqpoint{0.636356in}{0.440955in}}{\pgfqpoint{3.222048in}{0.270720in}} %
\pgfusepath{clip}%
\pgfsetbuttcap%
\pgfsetmiterjoin%
\definecolor{currentfill}{rgb}{0.333333,0.333333,0.333333}%
\pgfsetfillcolor{currentfill}%
\pgfsetlinewidth{0.501875pt}%
\definecolor{currentstroke}{rgb}{0.000000,0.000000,0.000000}%
\pgfsetstrokecolor{currentstroke}%
\pgfsetdash{}{0pt}%
\pgfpathmoveto{\pgfqpoint{2.207104in}{0.549060in}}%
\pgfpathlineto{\pgfqpoint{2.247380in}{0.549060in}}%
\pgfpathlineto{\pgfqpoint{2.247380in}{0.576315in}}%
\pgfpathlineto{\pgfqpoint{2.207104in}{0.576315in}}%
\pgfpathlineto{\pgfqpoint{2.207104in}{0.549060in}}%
\pgfusepath{stroke,fill}%
\end{pgfscope}%
\begin{pgfscope}%
\pgfpathrectangle{\pgfqpoint{0.636356in}{0.440955in}}{\pgfqpoint{3.222048in}{0.270720in}} %
\pgfusepath{clip}%
\pgfsetbuttcap%
\pgfsetmiterjoin%
\definecolor{currentfill}{rgb}{0.333333,0.333333,0.333333}%
\pgfsetfillcolor{currentfill}%
\pgfsetlinewidth{0.501875pt}%
\definecolor{currentstroke}{rgb}{0.000000,0.000000,0.000000}%
\pgfsetstrokecolor{currentstroke}%
\pgfsetdash{}{0pt}%
\pgfpathmoveto{\pgfqpoint{2.247380in}{0.551349in}}%
\pgfpathlineto{\pgfqpoint{2.287655in}{0.551349in}}%
\pgfpathlineto{\pgfqpoint{2.287655in}{0.576315in}}%
\pgfpathlineto{\pgfqpoint{2.247380in}{0.576315in}}%
\pgfpathlineto{\pgfqpoint{2.247380in}{0.551349in}}%
\pgfusepath{stroke,fill}%
\end{pgfscope}%
\begin{pgfscope}%
\pgfpathrectangle{\pgfqpoint{0.636356in}{0.440955in}}{\pgfqpoint{3.222048in}{0.270720in}} %
\pgfusepath{clip}%
\pgfsetbuttcap%
\pgfsetmiterjoin%
\definecolor{currentfill}{rgb}{0.333333,0.333333,0.333333}%
\pgfsetfillcolor{currentfill}%
\pgfsetlinewidth{0.501875pt}%
\definecolor{currentstroke}{rgb}{0.000000,0.000000,0.000000}%
\pgfsetstrokecolor{currentstroke}%
\pgfsetdash{}{0pt}%
\pgfpathmoveto{\pgfqpoint{2.287655in}{0.553428in}}%
\pgfpathlineto{\pgfqpoint{2.327931in}{0.553428in}}%
\pgfpathlineto{\pgfqpoint{2.327931in}{0.576315in}}%
\pgfpathlineto{\pgfqpoint{2.287655in}{0.576315in}}%
\pgfpathlineto{\pgfqpoint{2.287655in}{0.553428in}}%
\pgfusepath{stroke,fill}%
\end{pgfscope}%
\begin{pgfscope}%
\pgfpathrectangle{\pgfqpoint{0.636356in}{0.440955in}}{\pgfqpoint{3.222048in}{0.270720in}} %
\pgfusepath{clip}%
\pgfsetbuttcap%
\pgfsetmiterjoin%
\definecolor{currentfill}{rgb}{0.333333,0.333333,0.333333}%
\pgfsetfillcolor{currentfill}%
\pgfsetlinewidth{0.501875pt}%
\definecolor{currentstroke}{rgb}{0.000000,0.000000,0.000000}%
\pgfsetstrokecolor{currentstroke}%
\pgfsetdash{}{0pt}%
\pgfpathmoveto{\pgfqpoint{2.327931in}{0.576315in}}%
\pgfpathlineto{\pgfqpoint{2.368206in}{0.576315in}}%
\pgfpathlineto{\pgfqpoint{2.368206in}{0.601756in}}%
\pgfpathlineto{\pgfqpoint{2.327931in}{0.601756in}}%
\pgfpathlineto{\pgfqpoint{2.327931in}{0.576315in}}%
\pgfusepath{stroke,fill}%
\end{pgfscope}%
\begin{pgfscope}%
\pgfpathrectangle{\pgfqpoint{0.636356in}{0.440955in}}{\pgfqpoint{3.222048in}{0.270720in}} %
\pgfusepath{clip}%
\pgfsetbuttcap%
\pgfsetmiterjoin%
\definecolor{currentfill}{rgb}{0.333333,0.333333,0.333333}%
\pgfsetfillcolor{currentfill}%
\pgfsetlinewidth{0.501875pt}%
\definecolor{currentstroke}{rgb}{0.000000,0.000000,0.000000}%
\pgfsetstrokecolor{currentstroke}%
\pgfsetdash{}{0pt}%
\pgfpathmoveto{\pgfqpoint{2.368206in}{0.557085in}}%
\pgfpathlineto{\pgfqpoint{2.408482in}{0.557085in}}%
\pgfpathlineto{\pgfqpoint{2.408482in}{0.576315in}}%
\pgfpathlineto{\pgfqpoint{2.368206in}{0.576315in}}%
\pgfpathlineto{\pgfqpoint{2.368206in}{0.557085in}}%
\pgfusepath{stroke,fill}%
\end{pgfscope}%
\begin{pgfscope}%
\pgfpathrectangle{\pgfqpoint{0.636356in}{0.440955in}}{\pgfqpoint{3.222048in}{0.270720in}} %
\pgfusepath{clip}%
\pgfsetbuttcap%
\pgfsetmiterjoin%
\definecolor{currentfill}{rgb}{0.333333,0.333333,0.333333}%
\pgfsetfillcolor{currentfill}%
\pgfsetlinewidth{0.501875pt}%
\definecolor{currentstroke}{rgb}{0.000000,0.000000,0.000000}%
\pgfsetstrokecolor{currentstroke}%
\pgfsetdash{}{0pt}%
\pgfpathmoveto{\pgfqpoint{2.408482in}{0.558686in}}%
\pgfpathlineto{\pgfqpoint{2.448758in}{0.558686in}}%
\pgfpathlineto{\pgfqpoint{2.448758in}{0.576315in}}%
\pgfpathlineto{\pgfqpoint{2.408482in}{0.576315in}}%
\pgfpathlineto{\pgfqpoint{2.408482in}{0.558686in}}%
\pgfusepath{stroke,fill}%
\end{pgfscope}%
\begin{pgfscope}%
\pgfpathrectangle{\pgfqpoint{0.636356in}{0.440955in}}{\pgfqpoint{3.222048in}{0.270720in}} %
\pgfusepath{clip}%
\pgfsetbuttcap%
\pgfsetmiterjoin%
\definecolor{currentfill}{rgb}{0.333333,0.333333,0.333333}%
\pgfsetfillcolor{currentfill}%
\pgfsetlinewidth{0.501875pt}%
\definecolor{currentstroke}{rgb}{0.000000,0.000000,0.000000}%
\pgfsetstrokecolor{currentstroke}%
\pgfsetdash{}{0pt}%
\pgfpathmoveto{\pgfqpoint{2.448758in}{0.576315in}}%
\pgfpathlineto{\pgfqpoint{2.489033in}{0.576315in}}%
\pgfpathlineto{\pgfqpoint{2.489033in}{0.608427in}}%
\pgfpathlineto{\pgfqpoint{2.448758in}{0.608427in}}%
\pgfpathlineto{\pgfqpoint{2.448758in}{0.576315in}}%
\pgfusepath{stroke,fill}%
\end{pgfscope}%
\begin{pgfscope}%
\pgfpathrectangle{\pgfqpoint{0.636356in}{0.440955in}}{\pgfqpoint{3.222048in}{0.270720in}} %
\pgfusepath{clip}%
\pgfsetbuttcap%
\pgfsetmiterjoin%
\definecolor{currentfill}{rgb}{0.333333,0.333333,0.333333}%
\pgfsetfillcolor{currentfill}%
\pgfsetlinewidth{0.501875pt}%
\definecolor{currentstroke}{rgb}{0.000000,0.000000,0.000000}%
\pgfsetstrokecolor{currentstroke}%
\pgfsetdash{}{0pt}%
\pgfpathmoveto{\pgfqpoint{2.489033in}{0.561488in}}%
\pgfpathlineto{\pgfqpoint{2.529309in}{0.561488in}}%
\pgfpathlineto{\pgfqpoint{2.529309in}{0.576315in}}%
\pgfpathlineto{\pgfqpoint{2.489033in}{0.576315in}}%
\pgfpathlineto{\pgfqpoint{2.489033in}{0.561488in}}%
\pgfusepath{stroke,fill}%
\end{pgfscope}%
\begin{pgfscope}%
\pgfpathrectangle{\pgfqpoint{0.636356in}{0.440955in}}{\pgfqpoint{3.222048in}{0.270720in}} %
\pgfusepath{clip}%
\pgfsetbuttcap%
\pgfsetmiterjoin%
\definecolor{currentfill}{rgb}{0.333333,0.333333,0.333333}%
\pgfsetfillcolor{currentfill}%
\pgfsetlinewidth{0.501875pt}%
\definecolor{currentstroke}{rgb}{0.000000,0.000000,0.000000}%
\pgfsetstrokecolor{currentstroke}%
\pgfsetdash{}{0pt}%
\pgfpathmoveto{\pgfqpoint{2.529309in}{0.576315in}}%
\pgfpathlineto{\pgfqpoint{2.569584in}{0.576315in}}%
\pgfpathlineto{\pgfqpoint{2.569584in}{0.611992in}}%
\pgfpathlineto{\pgfqpoint{2.529309in}{0.611992in}}%
\pgfpathlineto{\pgfqpoint{2.529309in}{0.576315in}}%
\pgfusepath{stroke,fill}%
\end{pgfscope}%
\begin{pgfscope}%
\pgfpathrectangle{\pgfqpoint{0.636356in}{0.440955in}}{\pgfqpoint{3.222048in}{0.270720in}} %
\pgfusepath{clip}%
\pgfsetbuttcap%
\pgfsetmiterjoin%
\definecolor{currentfill}{rgb}{0.333333,0.333333,0.333333}%
\pgfsetfillcolor{currentfill}%
\pgfsetlinewidth{0.501875pt}%
\definecolor{currentstroke}{rgb}{0.000000,0.000000,0.000000}%
\pgfsetstrokecolor{currentstroke}%
\pgfsetdash{}{0pt}%
\pgfpathmoveto{\pgfqpoint{2.569584in}{0.576315in}}%
\pgfpathlineto{\pgfqpoint{2.609860in}{0.576315in}}%
\pgfpathlineto{\pgfqpoint{2.609860in}{0.613546in}}%
\pgfpathlineto{\pgfqpoint{2.569584in}{0.613546in}}%
\pgfpathlineto{\pgfqpoint{2.569584in}{0.576315in}}%
\pgfusepath{stroke,fill}%
\end{pgfscope}%
\begin{pgfscope}%
\pgfpathrectangle{\pgfqpoint{0.636356in}{0.440955in}}{\pgfqpoint{3.222048in}{0.270720in}} %
\pgfusepath{clip}%
\pgfsetbuttcap%
\pgfsetmiterjoin%
\definecolor{currentfill}{rgb}{0.333333,0.333333,0.333333}%
\pgfsetfillcolor{currentfill}%
\pgfsetlinewidth{0.501875pt}%
\definecolor{currentstroke}{rgb}{0.000000,0.000000,0.000000}%
\pgfsetstrokecolor{currentstroke}%
\pgfsetdash{}{0pt}%
\pgfpathmoveto{\pgfqpoint{2.609860in}{0.564871in}}%
\pgfpathlineto{\pgfqpoint{2.650136in}{0.564871in}}%
\pgfpathlineto{\pgfqpoint{2.650136in}{0.576315in}}%
\pgfpathlineto{\pgfqpoint{2.609860in}{0.576315in}}%
\pgfpathlineto{\pgfqpoint{2.609860in}{0.564871in}}%
\pgfusepath{stroke,fill}%
\end{pgfscope}%
\begin{pgfscope}%
\pgfpathrectangle{\pgfqpoint{0.636356in}{0.440955in}}{\pgfqpoint{3.222048in}{0.270720in}} %
\pgfusepath{clip}%
\pgfsetbuttcap%
\pgfsetmiterjoin%
\definecolor{currentfill}{rgb}{0.333333,0.333333,0.333333}%
\pgfsetfillcolor{currentfill}%
\pgfsetlinewidth{0.501875pt}%
\definecolor{currentstroke}{rgb}{0.000000,0.000000,0.000000}%
\pgfsetstrokecolor{currentstroke}%
\pgfsetdash{}{0pt}%
\pgfpathmoveto{\pgfqpoint{2.650136in}{0.565812in}}%
\pgfpathlineto{\pgfqpoint{2.690411in}{0.565812in}}%
\pgfpathlineto{\pgfqpoint{2.690411in}{0.576315in}}%
\pgfpathlineto{\pgfqpoint{2.650136in}{0.576315in}}%
\pgfpathlineto{\pgfqpoint{2.650136in}{0.565812in}}%
\pgfusepath{stroke,fill}%
\end{pgfscope}%
\begin{pgfscope}%
\pgfpathrectangle{\pgfqpoint{0.636356in}{0.440955in}}{\pgfqpoint{3.222048in}{0.270720in}} %
\pgfusepath{clip}%
\pgfsetbuttcap%
\pgfsetmiterjoin%
\definecolor{currentfill}{rgb}{0.333333,0.333333,0.333333}%
\pgfsetfillcolor{currentfill}%
\pgfsetlinewidth{0.501875pt}%
\definecolor{currentstroke}{rgb}{0.000000,0.000000,0.000000}%
\pgfsetstrokecolor{currentstroke}%
\pgfsetdash{}{0pt}%
\pgfpathmoveto{\pgfqpoint{2.690411in}{0.566676in}}%
\pgfpathlineto{\pgfqpoint{2.730687in}{0.566676in}}%
\pgfpathlineto{\pgfqpoint{2.730687in}{0.576315in}}%
\pgfpathlineto{\pgfqpoint{2.690411in}{0.576315in}}%
\pgfpathlineto{\pgfqpoint{2.690411in}{0.566676in}}%
\pgfusepath{stroke,fill}%
\end{pgfscope}%
\begin{pgfscope}%
\pgfpathrectangle{\pgfqpoint{0.636356in}{0.440955in}}{\pgfqpoint{3.222048in}{0.270720in}} %
\pgfusepath{clip}%
\pgfsetbuttcap%
\pgfsetmiterjoin%
\definecolor{currentfill}{rgb}{0.333333,0.333333,0.333333}%
\pgfsetfillcolor{currentfill}%
\pgfsetlinewidth{0.501875pt}%
\definecolor{currentstroke}{rgb}{0.000000,0.000000,0.000000}%
\pgfsetstrokecolor{currentstroke}%
\pgfsetdash{}{0pt}%
\pgfpathmoveto{\pgfqpoint{2.730687in}{0.576315in}}%
\pgfpathlineto{\pgfqpoint{2.770962in}{0.576315in}}%
\pgfpathlineto{\pgfqpoint{2.770962in}{0.638309in}}%
\pgfpathlineto{\pgfqpoint{2.730687in}{0.638309in}}%
\pgfpathlineto{\pgfqpoint{2.730687in}{0.576315in}}%
\pgfusepath{stroke,fill}%
\end{pgfscope}%
\begin{pgfscope}%
\pgfpathrectangle{\pgfqpoint{0.636356in}{0.440955in}}{\pgfqpoint{3.222048in}{0.270720in}} %
\pgfusepath{clip}%
\pgfsetbuttcap%
\pgfsetmiterjoin%
\definecolor{currentfill}{rgb}{0.333333,0.333333,0.333333}%
\pgfsetfillcolor{currentfill}%
\pgfsetlinewidth{0.501875pt}%
\definecolor{currentstroke}{rgb}{0.000000,0.000000,0.000000}%
\pgfsetstrokecolor{currentstroke}%
\pgfsetdash{}{0pt}%
\pgfpathmoveto{\pgfqpoint{2.770962in}{0.568193in}}%
\pgfpathlineto{\pgfqpoint{2.811238in}{0.568193in}}%
\pgfpathlineto{\pgfqpoint{2.811238in}{0.576315in}}%
\pgfpathlineto{\pgfqpoint{2.770962in}{0.576315in}}%
\pgfpathlineto{\pgfqpoint{2.770962in}{0.568193in}}%
\pgfusepath{stroke,fill}%
\end{pgfscope}%
\begin{pgfscope}%
\pgfpathrectangle{\pgfqpoint{0.636356in}{0.440955in}}{\pgfqpoint{3.222048in}{0.270720in}} %
\pgfusepath{clip}%
\pgfsetbuttcap%
\pgfsetmiterjoin%
\definecolor{currentfill}{rgb}{0.333333,0.333333,0.333333}%
\pgfsetfillcolor{currentfill}%
\pgfsetlinewidth{0.501875pt}%
\definecolor{currentstroke}{rgb}{0.000000,0.000000,0.000000}%
\pgfsetstrokecolor{currentstroke}%
\pgfsetdash{}{0pt}%
\pgfpathmoveto{\pgfqpoint{2.811238in}{0.576315in}}%
\pgfpathlineto{\pgfqpoint{2.851514in}{0.576315in}}%
\pgfpathlineto{\pgfqpoint{2.851514in}{0.620513in}}%
\pgfpathlineto{\pgfqpoint{2.811238in}{0.620513in}}%
\pgfpathlineto{\pgfqpoint{2.811238in}{0.576315in}}%
\pgfusepath{stroke,fill}%
\end{pgfscope}%
\begin{pgfscope}%
\pgfpathrectangle{\pgfqpoint{0.636356in}{0.440955in}}{\pgfqpoint{3.222048in}{0.270720in}} %
\pgfusepath{clip}%
\pgfsetbuttcap%
\pgfsetmiterjoin%
\definecolor{currentfill}{rgb}{0.333333,0.333333,0.333333}%
\pgfsetfillcolor{currentfill}%
\pgfsetlinewidth{0.501875pt}%
\definecolor{currentstroke}{rgb}{0.000000,0.000000,0.000000}%
\pgfsetstrokecolor{currentstroke}%
\pgfsetdash{}{0pt}%
\pgfpathmoveto{\pgfqpoint{2.851514in}{0.569467in}}%
\pgfpathlineto{\pgfqpoint{2.891789in}{0.569467in}}%
\pgfpathlineto{\pgfqpoint{2.891789in}{0.576315in}}%
\pgfpathlineto{\pgfqpoint{2.851514in}{0.576315in}}%
\pgfpathlineto{\pgfqpoint{2.851514in}{0.569467in}}%
\pgfusepath{stroke,fill}%
\end{pgfscope}%
\begin{pgfscope}%
\pgfpathrectangle{\pgfqpoint{0.636356in}{0.440955in}}{\pgfqpoint{3.222048in}{0.270720in}} %
\pgfusepath{clip}%
\pgfsetbuttcap%
\pgfsetmiterjoin%
\definecolor{currentfill}{rgb}{0.333333,0.333333,0.333333}%
\pgfsetfillcolor{currentfill}%
\pgfsetlinewidth{0.501875pt}%
\definecolor{currentstroke}{rgb}{0.000000,0.000000,0.000000}%
\pgfsetstrokecolor{currentstroke}%
\pgfsetdash{}{0pt}%
\pgfpathmoveto{\pgfqpoint{2.891789in}{0.570028in}}%
\pgfpathlineto{\pgfqpoint{2.932065in}{0.570028in}}%
\pgfpathlineto{\pgfqpoint{2.932065in}{0.576315in}}%
\pgfpathlineto{\pgfqpoint{2.891789in}{0.576315in}}%
\pgfpathlineto{\pgfqpoint{2.891789in}{0.570028in}}%
\pgfusepath{stroke,fill}%
\end{pgfscope}%
\begin{pgfscope}%
\pgfpathrectangle{\pgfqpoint{0.636356in}{0.440955in}}{\pgfqpoint{3.222048in}{0.270720in}} %
\pgfusepath{clip}%
\pgfsetbuttcap%
\pgfsetmiterjoin%
\definecolor{currentfill}{rgb}{0.333333,0.333333,0.333333}%
\pgfsetfillcolor{currentfill}%
\pgfsetlinewidth{0.501875pt}%
\definecolor{currentstroke}{rgb}{0.000000,0.000000,0.000000}%
\pgfsetstrokecolor{currentstroke}%
\pgfsetdash{}{0pt}%
\pgfpathmoveto{\pgfqpoint{2.932065in}{0.570539in}}%
\pgfpathlineto{\pgfqpoint{2.972340in}{0.570539in}}%
\pgfpathlineto{\pgfqpoint{2.972340in}{0.576315in}}%
\pgfpathlineto{\pgfqpoint{2.932065in}{0.576315in}}%
\pgfpathlineto{\pgfqpoint{2.932065in}{0.570539in}}%
\pgfusepath{stroke,fill}%
\end{pgfscope}%
\begin{pgfscope}%
\pgfpathrectangle{\pgfqpoint{0.636356in}{0.440955in}}{\pgfqpoint{3.222048in}{0.270720in}} %
\pgfusepath{clip}%
\pgfsetbuttcap%
\pgfsetmiterjoin%
\definecolor{currentfill}{rgb}{0.333333,0.333333,0.333333}%
\pgfsetfillcolor{currentfill}%
\pgfsetlinewidth{0.501875pt}%
\definecolor{currentstroke}{rgb}{0.000000,0.000000,0.000000}%
\pgfsetstrokecolor{currentstroke}%
\pgfsetdash{}{0pt}%
\pgfpathmoveto{\pgfqpoint{2.972340in}{0.571010in}}%
\pgfpathlineto{\pgfqpoint{3.012616in}{0.571010in}}%
\pgfpathlineto{\pgfqpoint{3.012616in}{0.576315in}}%
\pgfpathlineto{\pgfqpoint{2.972340in}{0.576315in}}%
\pgfpathlineto{\pgfqpoint{2.972340in}{0.571010in}}%
\pgfusepath{stroke,fill}%
\end{pgfscope}%
\begin{pgfscope}%
\pgfpathrectangle{\pgfqpoint{0.636356in}{0.440955in}}{\pgfqpoint{3.222048in}{0.270720in}} %
\pgfusepath{clip}%
\pgfsetbuttcap%
\pgfsetmiterjoin%
\definecolor{currentfill}{rgb}{0.333333,0.333333,0.333333}%
\pgfsetfillcolor{currentfill}%
\pgfsetlinewidth{0.501875pt}%
\definecolor{currentstroke}{rgb}{0.000000,0.000000,0.000000}%
\pgfsetstrokecolor{currentstroke}%
\pgfsetdash{}{0pt}%
\pgfpathmoveto{\pgfqpoint{3.012616in}{0.571441in}}%
\pgfpathlineto{\pgfqpoint{3.052892in}{0.571441in}}%
\pgfpathlineto{\pgfqpoint{3.052892in}{0.576315in}}%
\pgfpathlineto{\pgfqpoint{3.012616in}{0.576315in}}%
\pgfpathlineto{\pgfqpoint{3.012616in}{0.571441in}}%
\pgfusepath{stroke,fill}%
\end{pgfscope}%
\begin{pgfscope}%
\pgfpathrectangle{\pgfqpoint{0.636356in}{0.440955in}}{\pgfqpoint{3.222048in}{0.270720in}} %
\pgfusepath{clip}%
\pgfsetbuttcap%
\pgfsetmiterjoin%
\definecolor{currentfill}{rgb}{0.333333,0.333333,0.333333}%
\pgfsetfillcolor{currentfill}%
\pgfsetlinewidth{0.501875pt}%
\definecolor{currentstroke}{rgb}{0.000000,0.000000,0.000000}%
\pgfsetstrokecolor{currentstroke}%
\pgfsetdash{}{0pt}%
\pgfpathmoveto{\pgfqpoint{3.052892in}{0.571836in}}%
\pgfpathlineto{\pgfqpoint{3.093167in}{0.571836in}}%
\pgfpathlineto{\pgfqpoint{3.093167in}{0.576315in}}%
\pgfpathlineto{\pgfqpoint{3.052892in}{0.576315in}}%
\pgfpathlineto{\pgfqpoint{3.052892in}{0.571836in}}%
\pgfusepath{stroke,fill}%
\end{pgfscope}%
\begin{pgfscope}%
\pgfpathrectangle{\pgfqpoint{0.636356in}{0.440955in}}{\pgfqpoint{3.222048in}{0.270720in}} %
\pgfusepath{clip}%
\pgfsetbuttcap%
\pgfsetmiterjoin%
\definecolor{currentfill}{rgb}{0.333333,0.333333,0.333333}%
\pgfsetfillcolor{currentfill}%
\pgfsetlinewidth{0.501875pt}%
\definecolor{currentstroke}{rgb}{0.000000,0.000000,0.000000}%
\pgfsetstrokecolor{currentstroke}%
\pgfsetdash{}{0pt}%
\pgfpathmoveto{\pgfqpoint{3.093167in}{0.572200in}}%
\pgfpathlineto{\pgfqpoint{3.133443in}{0.572200in}}%
\pgfpathlineto{\pgfqpoint{3.133443in}{0.576315in}}%
\pgfpathlineto{\pgfqpoint{3.093167in}{0.576315in}}%
\pgfpathlineto{\pgfqpoint{3.093167in}{0.572200in}}%
\pgfusepath{stroke,fill}%
\end{pgfscope}%
\begin{pgfscope}%
\pgfpathrectangle{\pgfqpoint{0.636356in}{0.440955in}}{\pgfqpoint{3.222048in}{0.270720in}} %
\pgfusepath{clip}%
\pgfsetbuttcap%
\pgfsetmiterjoin%
\definecolor{currentfill}{rgb}{0.333333,0.333333,0.333333}%
\pgfsetfillcolor{currentfill}%
\pgfsetlinewidth{0.501875pt}%
\definecolor{currentstroke}{rgb}{0.000000,0.000000,0.000000}%
\pgfsetstrokecolor{currentstroke}%
\pgfsetdash{}{0pt}%
\pgfpathmoveto{\pgfqpoint{3.133443in}{0.572532in}}%
\pgfpathlineto{\pgfqpoint{3.173718in}{0.572532in}}%
\pgfpathlineto{\pgfqpoint{3.173718in}{0.576315in}}%
\pgfpathlineto{\pgfqpoint{3.133443in}{0.576315in}}%
\pgfpathlineto{\pgfqpoint{3.133443in}{0.572532in}}%
\pgfusepath{stroke,fill}%
\end{pgfscope}%
\begin{pgfscope}%
\pgfpathrectangle{\pgfqpoint{0.636356in}{0.440955in}}{\pgfqpoint{3.222048in}{0.270720in}} %
\pgfusepath{clip}%
\pgfsetbuttcap%
\pgfsetmiterjoin%
\definecolor{currentfill}{rgb}{0.333333,0.333333,0.333333}%
\pgfsetfillcolor{currentfill}%
\pgfsetlinewidth{0.501875pt}%
\definecolor{currentstroke}{rgb}{0.000000,0.000000,0.000000}%
\pgfsetstrokecolor{currentstroke}%
\pgfsetdash{}{0pt}%
\pgfpathmoveto{\pgfqpoint{3.173718in}{0.572839in}}%
\pgfpathlineto{\pgfqpoint{3.213994in}{0.572839in}}%
\pgfpathlineto{\pgfqpoint{3.213994in}{0.576315in}}%
\pgfpathlineto{\pgfqpoint{3.173718in}{0.576315in}}%
\pgfpathlineto{\pgfqpoint{3.173718in}{0.572839in}}%
\pgfusepath{stroke,fill}%
\end{pgfscope}%
\begin{pgfscope}%
\pgfpathrectangle{\pgfqpoint{0.636356in}{0.440955in}}{\pgfqpoint{3.222048in}{0.270720in}} %
\pgfusepath{clip}%
\pgfsetbuttcap%
\pgfsetmiterjoin%
\definecolor{currentfill}{rgb}{0.333333,0.333333,0.333333}%
\pgfsetfillcolor{currentfill}%
\pgfsetlinewidth{0.501875pt}%
\definecolor{currentstroke}{rgb}{0.000000,0.000000,0.000000}%
\pgfsetstrokecolor{currentstroke}%
\pgfsetdash{}{0pt}%
\pgfpathmoveto{\pgfqpoint{3.213994in}{0.573119in}}%
\pgfpathlineto{\pgfqpoint{3.254270in}{0.573119in}}%
\pgfpathlineto{\pgfqpoint{3.254270in}{0.576315in}}%
\pgfpathlineto{\pgfqpoint{3.213994in}{0.576315in}}%
\pgfpathlineto{\pgfqpoint{3.213994in}{0.573119in}}%
\pgfusepath{stroke,fill}%
\end{pgfscope}%
\begin{pgfscope}%
\pgfpathrectangle{\pgfqpoint{0.636356in}{0.440955in}}{\pgfqpoint{3.222048in}{0.270720in}} %
\pgfusepath{clip}%
\pgfsetbuttcap%
\pgfsetmiterjoin%
\definecolor{currentfill}{rgb}{0.333333,0.333333,0.333333}%
\pgfsetfillcolor{currentfill}%
\pgfsetlinewidth{0.501875pt}%
\definecolor{currentstroke}{rgb}{0.000000,0.000000,0.000000}%
\pgfsetstrokecolor{currentstroke}%
\pgfsetdash{}{0pt}%
\pgfpathmoveto{\pgfqpoint{3.254270in}{0.573377in}}%
\pgfpathlineto{\pgfqpoint{3.294545in}{0.573377in}}%
\pgfpathlineto{\pgfqpoint{3.294545in}{0.576315in}}%
\pgfpathlineto{\pgfqpoint{3.254270in}{0.576315in}}%
\pgfpathlineto{\pgfqpoint{3.254270in}{0.573377in}}%
\pgfusepath{stroke,fill}%
\end{pgfscope}%
\begin{pgfscope}%
\pgfpathrectangle{\pgfqpoint{0.636356in}{0.440955in}}{\pgfqpoint{3.222048in}{0.270720in}} %
\pgfusepath{clip}%
\pgfsetbuttcap%
\pgfsetmiterjoin%
\definecolor{currentfill}{rgb}{0.333333,0.333333,0.333333}%
\pgfsetfillcolor{currentfill}%
\pgfsetlinewidth{0.501875pt}%
\definecolor{currentstroke}{rgb}{0.000000,0.000000,0.000000}%
\pgfsetstrokecolor{currentstroke}%
\pgfsetdash{}{0pt}%
\pgfpathmoveto{\pgfqpoint{3.294545in}{0.573613in}}%
\pgfpathlineto{\pgfqpoint{3.334821in}{0.573613in}}%
\pgfpathlineto{\pgfqpoint{3.334821in}{0.576315in}}%
\pgfpathlineto{\pgfqpoint{3.294545in}{0.576315in}}%
\pgfpathlineto{\pgfqpoint{3.294545in}{0.573613in}}%
\pgfusepath{stroke,fill}%
\end{pgfscope}%
\begin{pgfscope}%
\pgfpathrectangle{\pgfqpoint{0.636356in}{0.440955in}}{\pgfqpoint{3.222048in}{0.270720in}} %
\pgfusepath{clip}%
\pgfsetbuttcap%
\pgfsetmiterjoin%
\definecolor{currentfill}{rgb}{0.333333,0.333333,0.333333}%
\pgfsetfillcolor{currentfill}%
\pgfsetlinewidth{0.501875pt}%
\definecolor{currentstroke}{rgb}{0.000000,0.000000,0.000000}%
\pgfsetstrokecolor{currentstroke}%
\pgfsetdash{}{0pt}%
\pgfpathmoveto{\pgfqpoint{3.334821in}{0.573830in}}%
\pgfpathlineto{\pgfqpoint{3.375096in}{0.573830in}}%
\pgfpathlineto{\pgfqpoint{3.375096in}{0.576315in}}%
\pgfpathlineto{\pgfqpoint{3.334821in}{0.576315in}}%
\pgfpathlineto{\pgfqpoint{3.334821in}{0.573830in}}%
\pgfusepath{stroke,fill}%
\end{pgfscope}%
\begin{pgfscope}%
\pgfpathrectangle{\pgfqpoint{0.636356in}{0.440955in}}{\pgfqpoint{3.222048in}{0.270720in}} %
\pgfusepath{clip}%
\pgfsetbuttcap%
\pgfsetmiterjoin%
\definecolor{currentfill}{rgb}{0.333333,0.333333,0.333333}%
\pgfsetfillcolor{currentfill}%
\pgfsetlinewidth{0.501875pt}%
\definecolor{currentstroke}{rgb}{0.000000,0.000000,0.000000}%
\pgfsetstrokecolor{currentstroke}%
\pgfsetdash{}{0pt}%
\pgfpathmoveto{\pgfqpoint{3.375096in}{0.574030in}}%
\pgfpathlineto{\pgfqpoint{3.415372in}{0.574030in}}%
\pgfpathlineto{\pgfqpoint{3.415372in}{0.576315in}}%
\pgfpathlineto{\pgfqpoint{3.375096in}{0.576315in}}%
\pgfpathlineto{\pgfqpoint{3.375096in}{0.574030in}}%
\pgfusepath{stroke,fill}%
\end{pgfscope}%
\begin{pgfscope}%
\pgfpathrectangle{\pgfqpoint{0.636356in}{0.440955in}}{\pgfqpoint{3.222048in}{0.270720in}} %
\pgfusepath{clip}%
\pgfsetbuttcap%
\pgfsetmiterjoin%
\definecolor{currentfill}{rgb}{0.333333,0.333333,0.333333}%
\pgfsetfillcolor{currentfill}%
\pgfsetlinewidth{0.501875pt}%
\definecolor{currentstroke}{rgb}{0.000000,0.000000,0.000000}%
\pgfsetstrokecolor{currentstroke}%
\pgfsetdash{}{0pt}%
\pgfpathmoveto{\pgfqpoint{3.415372in}{0.574213in}}%
\pgfpathlineto{\pgfqpoint{3.455648in}{0.574213in}}%
\pgfpathlineto{\pgfqpoint{3.455648in}{0.576315in}}%
\pgfpathlineto{\pgfqpoint{3.415372in}{0.576315in}}%
\pgfpathlineto{\pgfqpoint{3.415372in}{0.574213in}}%
\pgfusepath{stroke,fill}%
\end{pgfscope}%
\begin{pgfscope}%
\pgfpathrectangle{\pgfqpoint{0.636356in}{0.440955in}}{\pgfqpoint{3.222048in}{0.270720in}} %
\pgfusepath{clip}%
\pgfsetbuttcap%
\pgfsetmiterjoin%
\definecolor{currentfill}{rgb}{0.333333,0.333333,0.333333}%
\pgfsetfillcolor{currentfill}%
\pgfsetlinewidth{0.501875pt}%
\definecolor{currentstroke}{rgb}{0.000000,0.000000,0.000000}%
\pgfsetstrokecolor{currentstroke}%
\pgfsetdash{}{0pt}%
\pgfpathmoveto{\pgfqpoint{3.455648in}{0.574382in}}%
\pgfpathlineto{\pgfqpoint{3.495923in}{0.574382in}}%
\pgfpathlineto{\pgfqpoint{3.495923in}{0.576315in}}%
\pgfpathlineto{\pgfqpoint{3.455648in}{0.576315in}}%
\pgfpathlineto{\pgfqpoint{3.455648in}{0.574382in}}%
\pgfusepath{stroke,fill}%
\end{pgfscope}%
\begin{pgfscope}%
\pgfpathrectangle{\pgfqpoint{0.636356in}{0.440955in}}{\pgfqpoint{3.222048in}{0.270720in}} %
\pgfusepath{clip}%
\pgfsetbuttcap%
\pgfsetmiterjoin%
\definecolor{currentfill}{rgb}{0.333333,0.333333,0.333333}%
\pgfsetfillcolor{currentfill}%
\pgfsetlinewidth{0.501875pt}%
\definecolor{currentstroke}{rgb}{0.000000,0.000000,0.000000}%
\pgfsetstrokecolor{currentstroke}%
\pgfsetdash{}{0pt}%
\pgfpathmoveto{\pgfqpoint{3.495923in}{0.574536in}}%
\pgfpathlineto{\pgfqpoint{3.536199in}{0.574536in}}%
\pgfpathlineto{\pgfqpoint{3.536199in}{0.576315in}}%
\pgfpathlineto{\pgfqpoint{3.495923in}{0.576315in}}%
\pgfpathlineto{\pgfqpoint{3.495923in}{0.574536in}}%
\pgfusepath{stroke,fill}%
\end{pgfscope}%
\begin{pgfscope}%
\pgfpathrectangle{\pgfqpoint{0.636356in}{0.440955in}}{\pgfqpoint{3.222048in}{0.270720in}} %
\pgfusepath{clip}%
\pgfsetbuttcap%
\pgfsetmiterjoin%
\definecolor{currentfill}{rgb}{0.333333,0.333333,0.333333}%
\pgfsetfillcolor{currentfill}%
\pgfsetlinewidth{0.501875pt}%
\definecolor{currentstroke}{rgb}{0.000000,0.000000,0.000000}%
\pgfsetstrokecolor{currentstroke}%
\pgfsetdash{}{0pt}%
\pgfpathmoveto{\pgfqpoint{3.536199in}{0.574678in}}%
\pgfpathlineto{\pgfqpoint{3.576474in}{0.574678in}}%
\pgfpathlineto{\pgfqpoint{3.576474in}{0.576315in}}%
\pgfpathlineto{\pgfqpoint{3.536199in}{0.576315in}}%
\pgfpathlineto{\pgfqpoint{3.536199in}{0.574678in}}%
\pgfusepath{stroke,fill}%
\end{pgfscope}%
\begin{pgfscope}%
\pgfpathrectangle{\pgfqpoint{0.636356in}{0.440955in}}{\pgfqpoint{3.222048in}{0.270720in}} %
\pgfusepath{clip}%
\pgfsetbuttcap%
\pgfsetmiterjoin%
\definecolor{currentfill}{rgb}{0.333333,0.333333,0.333333}%
\pgfsetfillcolor{currentfill}%
\pgfsetlinewidth{0.501875pt}%
\definecolor{currentstroke}{rgb}{0.000000,0.000000,0.000000}%
\pgfsetstrokecolor{currentstroke}%
\pgfsetdash{}{0pt}%
\pgfpathmoveto{\pgfqpoint{3.576474in}{0.574809in}}%
\pgfpathlineto{\pgfqpoint{3.616750in}{0.574809in}}%
\pgfpathlineto{\pgfqpoint{3.616750in}{0.576315in}}%
\pgfpathlineto{\pgfqpoint{3.576474in}{0.576315in}}%
\pgfpathlineto{\pgfqpoint{3.576474in}{0.574809in}}%
\pgfusepath{stroke,fill}%
\end{pgfscope}%
\begin{pgfscope}%
\pgfpathrectangle{\pgfqpoint{0.636356in}{0.440955in}}{\pgfqpoint{3.222048in}{0.270720in}} %
\pgfusepath{clip}%
\pgfsetbuttcap%
\pgfsetmiterjoin%
\definecolor{currentfill}{rgb}{0.333333,0.333333,0.333333}%
\pgfsetfillcolor{currentfill}%
\pgfsetlinewidth{0.501875pt}%
\definecolor{currentstroke}{rgb}{0.000000,0.000000,0.000000}%
\pgfsetstrokecolor{currentstroke}%
\pgfsetdash{}{0pt}%
\pgfpathmoveto{\pgfqpoint{3.616750in}{0.574929in}}%
\pgfpathlineto{\pgfqpoint{3.657026in}{0.574929in}}%
\pgfpathlineto{\pgfqpoint{3.657026in}{0.576315in}}%
\pgfpathlineto{\pgfqpoint{3.616750in}{0.576315in}}%
\pgfpathlineto{\pgfqpoint{3.616750in}{0.574929in}}%
\pgfusepath{stroke,fill}%
\end{pgfscope}%
\begin{pgfscope}%
\pgfpathrectangle{\pgfqpoint{0.636356in}{0.440955in}}{\pgfqpoint{3.222048in}{0.270720in}} %
\pgfusepath{clip}%
\pgfsetbuttcap%
\pgfsetmiterjoin%
\definecolor{currentfill}{rgb}{0.333333,0.333333,0.333333}%
\pgfsetfillcolor{currentfill}%
\pgfsetlinewidth{0.501875pt}%
\definecolor{currentstroke}{rgb}{0.000000,0.000000,0.000000}%
\pgfsetstrokecolor{currentstroke}%
\pgfsetdash{}{0pt}%
\pgfpathmoveto{\pgfqpoint{3.657026in}{0.575039in}}%
\pgfpathlineto{\pgfqpoint{3.697301in}{0.575039in}}%
\pgfpathlineto{\pgfqpoint{3.697301in}{0.576315in}}%
\pgfpathlineto{\pgfqpoint{3.657026in}{0.576315in}}%
\pgfpathlineto{\pgfqpoint{3.657026in}{0.575039in}}%
\pgfusepath{stroke,fill}%
\end{pgfscope}%
\begin{pgfscope}%
\pgfpathrectangle{\pgfqpoint{0.636356in}{0.440955in}}{\pgfqpoint{3.222048in}{0.270720in}} %
\pgfusepath{clip}%
\pgfsetbuttcap%
\pgfsetmiterjoin%
\definecolor{currentfill}{rgb}{0.333333,0.333333,0.333333}%
\pgfsetfillcolor{currentfill}%
\pgfsetlinewidth{0.501875pt}%
\definecolor{currentstroke}{rgb}{0.000000,0.000000,0.000000}%
\pgfsetstrokecolor{currentstroke}%
\pgfsetdash{}{0pt}%
\pgfpathmoveto{\pgfqpoint{3.697301in}{0.575140in}}%
\pgfpathlineto{\pgfqpoint{3.737577in}{0.575140in}}%
\pgfpathlineto{\pgfqpoint{3.737577in}{0.576315in}}%
\pgfpathlineto{\pgfqpoint{3.697301in}{0.576315in}}%
\pgfpathlineto{\pgfqpoint{3.697301in}{0.575140in}}%
\pgfusepath{stroke,fill}%
\end{pgfscope}%
\begin{pgfscope}%
\pgfpathrectangle{\pgfqpoint{0.636356in}{0.440955in}}{\pgfqpoint{3.222048in}{0.270720in}} %
\pgfusepath{clip}%
\pgfsetbuttcap%
\pgfsetmiterjoin%
\definecolor{currentfill}{rgb}{0.333333,0.333333,0.333333}%
\pgfsetfillcolor{currentfill}%
\pgfsetlinewidth{0.501875pt}%
\definecolor{currentstroke}{rgb}{0.000000,0.000000,0.000000}%
\pgfsetstrokecolor{currentstroke}%
\pgfsetdash{}{0pt}%
\pgfpathmoveto{\pgfqpoint{3.737577in}{0.575234in}}%
\pgfpathlineto{\pgfqpoint{3.777852in}{0.575234in}}%
\pgfpathlineto{\pgfqpoint{3.777852in}{0.576315in}}%
\pgfpathlineto{\pgfqpoint{3.737577in}{0.576315in}}%
\pgfpathlineto{\pgfqpoint{3.737577in}{0.575234in}}%
\pgfusepath{stroke,fill}%
\end{pgfscope}%
\begin{pgfscope}%
\pgfpathrectangle{\pgfqpoint{0.636356in}{0.440955in}}{\pgfqpoint{3.222048in}{0.270720in}} %
\pgfusepath{clip}%
\pgfsetbuttcap%
\pgfsetmiterjoin%
\definecolor{currentfill}{rgb}{0.333333,0.333333,0.333333}%
\pgfsetfillcolor{currentfill}%
\pgfsetlinewidth{0.501875pt}%
\definecolor{currentstroke}{rgb}{0.000000,0.000000,0.000000}%
\pgfsetstrokecolor{currentstroke}%
\pgfsetdash{}{0pt}%
\pgfpathmoveto{\pgfqpoint{3.777852in}{0.575319in}}%
\pgfpathlineto{\pgfqpoint{3.818128in}{0.575319in}}%
\pgfpathlineto{\pgfqpoint{3.818128in}{0.576315in}}%
\pgfpathlineto{\pgfqpoint{3.777852in}{0.576315in}}%
\pgfpathlineto{\pgfqpoint{3.777852in}{0.575319in}}%
\pgfusepath{stroke,fill}%
\end{pgfscope}%
\begin{pgfscope}%
\pgfpathrectangle{\pgfqpoint{0.636356in}{0.440955in}}{\pgfqpoint{3.222048in}{0.270720in}} %
\pgfusepath{clip}%
\pgfsetbuttcap%
\pgfsetmiterjoin%
\definecolor{currentfill}{rgb}{0.333333,0.333333,0.333333}%
\pgfsetfillcolor{currentfill}%
\pgfsetlinewidth{0.501875pt}%
\definecolor{currentstroke}{rgb}{0.000000,0.000000,0.000000}%
\pgfsetstrokecolor{currentstroke}%
\pgfsetdash{}{0pt}%
\pgfpathmoveto{\pgfqpoint{3.818128in}{0.575398in}}%
\pgfpathlineto{\pgfqpoint{3.858404in}{0.575398in}}%
\pgfpathlineto{\pgfqpoint{3.858404in}{0.576315in}}%
\pgfpathlineto{\pgfqpoint{3.818128in}{0.576315in}}%
\pgfpathlineto{\pgfqpoint{3.818128in}{0.575398in}}%
\pgfusepath{stroke,fill}%
\end{pgfscope}%
\begin{pgfscope}%
\pgfsetrectcap%
\pgfsetmiterjoin%
\pgfsetlinewidth{1.003750pt}%
\definecolor{currentstroke}{rgb}{0.000000,0.000000,0.000000}%
\pgfsetstrokecolor{currentstroke}%
\pgfsetdash{}{0pt}%
\pgfpathmoveto{\pgfqpoint{0.636356in}{0.711675in}}%
\pgfpathlineto{\pgfqpoint{3.858404in}{0.711675in}}%
\pgfusepath{stroke}%
\end{pgfscope}%
\begin{pgfscope}%
\pgfsetrectcap%
\pgfsetmiterjoin%
\pgfsetlinewidth{1.003750pt}%
\definecolor{currentstroke}{rgb}{0.000000,0.000000,0.000000}%
\pgfsetstrokecolor{currentstroke}%
\pgfsetdash{}{0pt}%
\pgfpathmoveto{\pgfqpoint{3.858404in}{0.440955in}}%
\pgfpathlineto{\pgfqpoint{3.858404in}{0.711675in}}%
\pgfusepath{stroke}%
\end{pgfscope}%
\begin{pgfscope}%
\pgfsetrectcap%
\pgfsetmiterjoin%
\pgfsetlinewidth{1.003750pt}%
\definecolor{currentstroke}{rgb}{0.000000,0.000000,0.000000}%
\pgfsetstrokecolor{currentstroke}%
\pgfsetdash{}{0pt}%
\pgfpathmoveto{\pgfqpoint{0.636356in}{0.440955in}}%
\pgfpathlineto{\pgfqpoint{3.858404in}{0.440955in}}%
\pgfusepath{stroke}%
\end{pgfscope}%
\begin{pgfscope}%
\pgfsetrectcap%
\pgfsetmiterjoin%
\pgfsetlinewidth{1.003750pt}%
\definecolor{currentstroke}{rgb}{0.000000,0.000000,0.000000}%
\pgfsetstrokecolor{currentstroke}%
\pgfsetdash{}{0pt}%
\pgfpathmoveto{\pgfqpoint{0.636356in}{0.440955in}}%
\pgfpathlineto{\pgfqpoint{0.636356in}{0.711675in}}%
\pgfusepath{stroke}%
\end{pgfscope}%
\begin{pgfscope}%
\pgfsetbuttcap%
\pgfsetroundjoin%
\definecolor{currentfill}{rgb}{0.000000,0.000000,0.000000}%
\pgfsetfillcolor{currentfill}%
\pgfsetlinewidth{0.501875pt}%
\definecolor{currentstroke}{rgb}{0.000000,0.000000,0.000000}%
\pgfsetstrokecolor{currentstroke}%
\pgfsetdash{}{0pt}%
\pgfsys@defobject{currentmarker}{\pgfqpoint{0.000000in}{0.000000in}}{\pgfqpoint{0.000000in}{0.069444in}}{%
\pgfpathmoveto{\pgfqpoint{0.000000in}{0.000000in}}%
\pgfpathlineto{\pgfqpoint{0.000000in}{0.069444in}}%
\pgfusepath{stroke,fill}%
}%
\begin{pgfscope}%
\pgfsys@transformshift{0.789787in}{0.440955in}%
\pgfsys@useobject{currentmarker}{}%
\end{pgfscope}%
\end{pgfscope}%
\begin{pgfscope}%
\pgfsetbuttcap%
\pgfsetroundjoin%
\definecolor{currentfill}{rgb}{0.000000,0.000000,0.000000}%
\pgfsetfillcolor{currentfill}%
\pgfsetlinewidth{0.501875pt}%
\definecolor{currentstroke}{rgb}{0.000000,0.000000,0.000000}%
\pgfsetstrokecolor{currentstroke}%
\pgfsetdash{}{0pt}%
\pgfsys@defobject{currentmarker}{\pgfqpoint{0.000000in}{-0.069444in}}{\pgfqpoint{0.000000in}{0.000000in}}{%
\pgfpathmoveto{\pgfqpoint{0.000000in}{0.000000in}}%
\pgfpathlineto{\pgfqpoint{0.000000in}{-0.069444in}}%
\pgfusepath{stroke,fill}%
}%
\begin{pgfscope}%
\pgfsys@transformshift{0.789787in}{0.711675in}%
\pgfsys@useobject{currentmarker}{}%
\end{pgfscope}%
\end{pgfscope}%
\begin{pgfscope}%
\pgftext[x=0.789787in,y=0.371511in,,top]{\rmfamily\fontsize{8.000000}{9.600000}\selectfont 5000}%
\end{pgfscope}%
\begin{pgfscope}%
\pgfsetbuttcap%
\pgfsetroundjoin%
\definecolor{currentfill}{rgb}{0.000000,0.000000,0.000000}%
\pgfsetfillcolor{currentfill}%
\pgfsetlinewidth{0.501875pt}%
\definecolor{currentstroke}{rgb}{0.000000,0.000000,0.000000}%
\pgfsetstrokecolor{currentstroke}%
\pgfsetdash{}{0pt}%
\pgfsys@defobject{currentmarker}{\pgfqpoint{0.000000in}{0.000000in}}{\pgfqpoint{0.000000in}{0.069444in}}{%
\pgfpathmoveto{\pgfqpoint{0.000000in}{0.000000in}}%
\pgfpathlineto{\pgfqpoint{0.000000in}{0.069444in}}%
\pgfusepath{stroke,fill}%
}%
\begin{pgfscope}%
\pgfsys@transformshift{1.556941in}{0.440955in}%
\pgfsys@useobject{currentmarker}{}%
\end{pgfscope}%
\end{pgfscope}%
\begin{pgfscope}%
\pgfsetbuttcap%
\pgfsetroundjoin%
\definecolor{currentfill}{rgb}{0.000000,0.000000,0.000000}%
\pgfsetfillcolor{currentfill}%
\pgfsetlinewidth{0.501875pt}%
\definecolor{currentstroke}{rgb}{0.000000,0.000000,0.000000}%
\pgfsetstrokecolor{currentstroke}%
\pgfsetdash{}{0pt}%
\pgfsys@defobject{currentmarker}{\pgfqpoint{0.000000in}{-0.069444in}}{\pgfqpoint{0.000000in}{0.000000in}}{%
\pgfpathmoveto{\pgfqpoint{0.000000in}{0.000000in}}%
\pgfpathlineto{\pgfqpoint{0.000000in}{-0.069444in}}%
\pgfusepath{stroke,fill}%
}%
\begin{pgfscope}%
\pgfsys@transformshift{1.556941in}{0.711675in}%
\pgfsys@useobject{currentmarker}{}%
\end{pgfscope}%
\end{pgfscope}%
\begin{pgfscope}%
\pgftext[x=1.556941in,y=0.371511in,,top]{\rmfamily\fontsize{8.000000}{9.600000}\selectfont 5500}%
\end{pgfscope}%
\begin{pgfscope}%
\pgfsetbuttcap%
\pgfsetroundjoin%
\definecolor{currentfill}{rgb}{0.000000,0.000000,0.000000}%
\pgfsetfillcolor{currentfill}%
\pgfsetlinewidth{0.501875pt}%
\definecolor{currentstroke}{rgb}{0.000000,0.000000,0.000000}%
\pgfsetstrokecolor{currentstroke}%
\pgfsetdash{}{0pt}%
\pgfsys@defobject{currentmarker}{\pgfqpoint{0.000000in}{0.000000in}}{\pgfqpoint{0.000000in}{0.069444in}}{%
\pgfpathmoveto{\pgfqpoint{0.000000in}{0.000000in}}%
\pgfpathlineto{\pgfqpoint{0.000000in}{0.069444in}}%
\pgfusepath{stroke,fill}%
}%
\begin{pgfscope}%
\pgfsys@transformshift{2.324095in}{0.440955in}%
\pgfsys@useobject{currentmarker}{}%
\end{pgfscope}%
\end{pgfscope}%
\begin{pgfscope}%
\pgfsetbuttcap%
\pgfsetroundjoin%
\definecolor{currentfill}{rgb}{0.000000,0.000000,0.000000}%
\pgfsetfillcolor{currentfill}%
\pgfsetlinewidth{0.501875pt}%
\definecolor{currentstroke}{rgb}{0.000000,0.000000,0.000000}%
\pgfsetstrokecolor{currentstroke}%
\pgfsetdash{}{0pt}%
\pgfsys@defobject{currentmarker}{\pgfqpoint{0.000000in}{-0.069444in}}{\pgfqpoint{0.000000in}{0.000000in}}{%
\pgfpathmoveto{\pgfqpoint{0.000000in}{0.000000in}}%
\pgfpathlineto{\pgfqpoint{0.000000in}{-0.069444in}}%
\pgfusepath{stroke,fill}%
}%
\begin{pgfscope}%
\pgfsys@transformshift{2.324095in}{0.711675in}%
\pgfsys@useobject{currentmarker}{}%
\end{pgfscope}%
\end{pgfscope}%
\begin{pgfscope}%
\pgftext[x=2.324095in,y=0.371511in,,top]{\rmfamily\fontsize{8.000000}{9.600000}\selectfont 6000}%
\end{pgfscope}%
\begin{pgfscope}%
\pgfsetbuttcap%
\pgfsetroundjoin%
\definecolor{currentfill}{rgb}{0.000000,0.000000,0.000000}%
\pgfsetfillcolor{currentfill}%
\pgfsetlinewidth{0.501875pt}%
\definecolor{currentstroke}{rgb}{0.000000,0.000000,0.000000}%
\pgfsetstrokecolor{currentstroke}%
\pgfsetdash{}{0pt}%
\pgfsys@defobject{currentmarker}{\pgfqpoint{0.000000in}{0.000000in}}{\pgfqpoint{0.000000in}{0.069444in}}{%
\pgfpathmoveto{\pgfqpoint{0.000000in}{0.000000in}}%
\pgfpathlineto{\pgfqpoint{0.000000in}{0.069444in}}%
\pgfusepath{stroke,fill}%
}%
\begin{pgfscope}%
\pgfsys@transformshift{3.091249in}{0.440955in}%
\pgfsys@useobject{currentmarker}{}%
\end{pgfscope}%
\end{pgfscope}%
\begin{pgfscope}%
\pgfsetbuttcap%
\pgfsetroundjoin%
\definecolor{currentfill}{rgb}{0.000000,0.000000,0.000000}%
\pgfsetfillcolor{currentfill}%
\pgfsetlinewidth{0.501875pt}%
\definecolor{currentstroke}{rgb}{0.000000,0.000000,0.000000}%
\pgfsetstrokecolor{currentstroke}%
\pgfsetdash{}{0pt}%
\pgfsys@defobject{currentmarker}{\pgfqpoint{0.000000in}{-0.069444in}}{\pgfqpoint{0.000000in}{0.000000in}}{%
\pgfpathmoveto{\pgfqpoint{0.000000in}{0.000000in}}%
\pgfpathlineto{\pgfqpoint{0.000000in}{-0.069444in}}%
\pgfusepath{stroke,fill}%
}%
\begin{pgfscope}%
\pgfsys@transformshift{3.091249in}{0.711675in}%
\pgfsys@useobject{currentmarker}{}%
\end{pgfscope}%
\end{pgfscope}%
\begin{pgfscope}%
\pgftext[x=3.091249in,y=0.371511in,,top]{\rmfamily\fontsize{8.000000}{9.600000}\selectfont 6500}%
\end{pgfscope}%
\begin{pgfscope}%
\pgfsetbuttcap%
\pgfsetroundjoin%
\definecolor{currentfill}{rgb}{0.000000,0.000000,0.000000}%
\pgfsetfillcolor{currentfill}%
\pgfsetlinewidth{0.501875pt}%
\definecolor{currentstroke}{rgb}{0.000000,0.000000,0.000000}%
\pgfsetstrokecolor{currentstroke}%
\pgfsetdash{}{0pt}%
\pgfsys@defobject{currentmarker}{\pgfqpoint{0.000000in}{0.000000in}}{\pgfqpoint{0.000000in}{0.069444in}}{%
\pgfpathmoveto{\pgfqpoint{0.000000in}{0.000000in}}%
\pgfpathlineto{\pgfqpoint{0.000000in}{0.069444in}}%
\pgfusepath{stroke,fill}%
}%
\begin{pgfscope}%
\pgfsys@transformshift{3.858404in}{0.440955in}%
\pgfsys@useobject{currentmarker}{}%
\end{pgfscope}%
\end{pgfscope}%
\begin{pgfscope}%
\pgfsetbuttcap%
\pgfsetroundjoin%
\definecolor{currentfill}{rgb}{0.000000,0.000000,0.000000}%
\pgfsetfillcolor{currentfill}%
\pgfsetlinewidth{0.501875pt}%
\definecolor{currentstroke}{rgb}{0.000000,0.000000,0.000000}%
\pgfsetstrokecolor{currentstroke}%
\pgfsetdash{}{0pt}%
\pgfsys@defobject{currentmarker}{\pgfqpoint{0.000000in}{-0.069444in}}{\pgfqpoint{0.000000in}{0.000000in}}{%
\pgfpathmoveto{\pgfqpoint{0.000000in}{0.000000in}}%
\pgfpathlineto{\pgfqpoint{0.000000in}{-0.069444in}}%
\pgfusepath{stroke,fill}%
}%
\begin{pgfscope}%
\pgfsys@transformshift{3.858404in}{0.711675in}%
\pgfsys@useobject{currentmarker}{}%
\end{pgfscope}%
\end{pgfscope}%
\begin{pgfscope}%
\pgftext[x=3.858404in,y=0.371511in,,top]{\rmfamily\fontsize{8.000000}{9.600000}\selectfont 7000}%
\end{pgfscope}%
\begin{pgfscope}%
\pgftext[x=2.247380in,y=0.194536in,,top]{\rmfamily\fontsize{9.000000}{10.800000}\selectfont \(\displaystyle m(K^+\!\pi^-\!\mu^+\!\mu^-)\)}%
\end{pgfscope}%
\begin{pgfscope}%
\pgfsetbuttcap%
\pgfsetroundjoin%
\definecolor{currentfill}{rgb}{0.000000,0.000000,0.000000}%
\pgfsetfillcolor{currentfill}%
\pgfsetlinewidth{0.501875pt}%
\definecolor{currentstroke}{rgb}{0.000000,0.000000,0.000000}%
\pgfsetstrokecolor{currentstroke}%
\pgfsetdash{}{0pt}%
\pgfsys@defobject{currentmarker}{\pgfqpoint{0.000000in}{0.000000in}}{\pgfqpoint{0.069444in}{0.000000in}}{%
\pgfpathmoveto{\pgfqpoint{0.000000in}{0.000000in}}%
\pgfpathlineto{\pgfqpoint{0.069444in}{0.000000in}}%
\pgfusepath{stroke,fill}%
}%
\begin{pgfscope}%
\pgfsys@transformshift{0.636356in}{0.440955in}%
\pgfsys@useobject{currentmarker}{}%
\end{pgfscope}%
\end{pgfscope}%
\begin{pgfscope}%
\pgfsetbuttcap%
\pgfsetroundjoin%
\definecolor{currentfill}{rgb}{0.000000,0.000000,0.000000}%
\pgfsetfillcolor{currentfill}%
\pgfsetlinewidth{0.501875pt}%
\definecolor{currentstroke}{rgb}{0.000000,0.000000,0.000000}%
\pgfsetstrokecolor{currentstroke}%
\pgfsetdash{}{0pt}%
\pgfsys@defobject{currentmarker}{\pgfqpoint{-0.069444in}{0.000000in}}{\pgfqpoint{0.000000in}{0.000000in}}{%
\pgfpathmoveto{\pgfqpoint{0.000000in}{0.000000in}}%
\pgfpathlineto{\pgfqpoint{-0.069444in}{0.000000in}}%
\pgfusepath{stroke,fill}%
}%
\begin{pgfscope}%
\pgfsys@transformshift{3.858404in}{0.440955in}%
\pgfsys@useobject{currentmarker}{}%
\end{pgfscope}%
\end{pgfscope}%
\begin{pgfscope}%
\pgftext[x=0.566911in,y=0.440955in,right,]{\rmfamily\fontsize{8.000000}{9.600000}\selectfont −3}%
\end{pgfscope}%
\begin{pgfscope}%
\pgfsetbuttcap%
\pgfsetroundjoin%
\definecolor{currentfill}{rgb}{0.000000,0.000000,0.000000}%
\pgfsetfillcolor{currentfill}%
\pgfsetlinewidth{0.501875pt}%
\definecolor{currentstroke}{rgb}{0.000000,0.000000,0.000000}%
\pgfsetstrokecolor{currentstroke}%
\pgfsetdash{}{0pt}%
\pgfsys@defobject{currentmarker}{\pgfqpoint{0.000000in}{0.000000in}}{\pgfqpoint{0.069444in}{0.000000in}}{%
\pgfpathmoveto{\pgfqpoint{0.000000in}{0.000000in}}%
\pgfpathlineto{\pgfqpoint{0.069444in}{0.000000in}}%
\pgfusepath{stroke,fill}%
}%
\begin{pgfscope}%
\pgfsys@transformshift{0.636356in}{0.576315in}%
\pgfsys@useobject{currentmarker}{}%
\end{pgfscope}%
\end{pgfscope}%
\begin{pgfscope}%
\pgfsetbuttcap%
\pgfsetroundjoin%
\definecolor{currentfill}{rgb}{0.000000,0.000000,0.000000}%
\pgfsetfillcolor{currentfill}%
\pgfsetlinewidth{0.501875pt}%
\definecolor{currentstroke}{rgb}{0.000000,0.000000,0.000000}%
\pgfsetstrokecolor{currentstroke}%
\pgfsetdash{}{0pt}%
\pgfsys@defobject{currentmarker}{\pgfqpoint{-0.069444in}{0.000000in}}{\pgfqpoint{0.000000in}{0.000000in}}{%
\pgfpathmoveto{\pgfqpoint{0.000000in}{0.000000in}}%
\pgfpathlineto{\pgfqpoint{-0.069444in}{0.000000in}}%
\pgfusepath{stroke,fill}%
}%
\begin{pgfscope}%
\pgfsys@transformshift{3.858404in}{0.576315in}%
\pgfsys@useobject{currentmarker}{}%
\end{pgfscope}%
\end{pgfscope}%
\begin{pgfscope}%
\pgftext[x=0.566911in,y=0.576315in,right,]{\rmfamily\fontsize{8.000000}{9.600000}\selectfont 0}%
\end{pgfscope}%
\begin{pgfscope}%
\pgfsetbuttcap%
\pgfsetroundjoin%
\definecolor{currentfill}{rgb}{0.000000,0.000000,0.000000}%
\pgfsetfillcolor{currentfill}%
\pgfsetlinewidth{0.501875pt}%
\definecolor{currentstroke}{rgb}{0.000000,0.000000,0.000000}%
\pgfsetstrokecolor{currentstroke}%
\pgfsetdash{}{0pt}%
\pgfsys@defobject{currentmarker}{\pgfqpoint{0.000000in}{0.000000in}}{\pgfqpoint{0.069444in}{0.000000in}}{%
\pgfpathmoveto{\pgfqpoint{0.000000in}{0.000000in}}%
\pgfpathlineto{\pgfqpoint{0.069444in}{0.000000in}}%
\pgfusepath{stroke,fill}%
}%
\begin{pgfscope}%
\pgfsys@transformshift{0.636356in}{0.711675in}%
\pgfsys@useobject{currentmarker}{}%
\end{pgfscope}%
\end{pgfscope}%
\begin{pgfscope}%
\pgfsetbuttcap%
\pgfsetroundjoin%
\definecolor{currentfill}{rgb}{0.000000,0.000000,0.000000}%
\pgfsetfillcolor{currentfill}%
\pgfsetlinewidth{0.501875pt}%
\definecolor{currentstroke}{rgb}{0.000000,0.000000,0.000000}%
\pgfsetstrokecolor{currentstroke}%
\pgfsetdash{}{0pt}%
\pgfsys@defobject{currentmarker}{\pgfqpoint{-0.069444in}{0.000000in}}{\pgfqpoint{0.000000in}{0.000000in}}{%
\pgfpathmoveto{\pgfqpoint{0.000000in}{0.000000in}}%
\pgfpathlineto{\pgfqpoint{-0.069444in}{0.000000in}}%
\pgfusepath{stroke,fill}%
}%
\begin{pgfscope}%
\pgfsys@transformshift{3.858404in}{0.711675in}%
\pgfsys@useobject{currentmarker}{}%
\end{pgfscope}%
\end{pgfscope}%
\begin{pgfscope}%
\pgftext[x=0.566911in,y=0.711675in,right,]{\rmfamily\fontsize{8.000000}{9.600000}\selectfont 3}%
\end{pgfscope}%
\begin{pgfscope}%
\pgftext[x=0.333676in,y=0.576315in,,bottom,rotate=90.000000]{\rmfamily\fontsize{9.000000}{10.800000}\selectfont \(\displaystyle \frac{\hat{n}_i -  n_i}{\sigma(n_i)}\)}%
\end{pgfscope}%
\begin{pgfscope}%
\pgfsetbuttcap%
\pgfsetmiterjoin%
\definecolor{currentfill}{rgb}{1.000000,1.000000,1.000000}%
\pgfsetfillcolor{currentfill}%
\pgfsetlinewidth{0.000000pt}%
\definecolor{currentstroke}{rgb}{0.000000,0.000000,0.000000}%
\pgfsetstrokecolor{currentstroke}%
\pgfsetstrokeopacity{0.000000}%
\pgfsetdash{}{0pt}%
\pgfpathmoveto{\pgfqpoint{0.636356in}{0.874107in}}%
\pgfpathlineto{\pgfqpoint{3.858404in}{0.874107in}}%
\pgfpathlineto{\pgfqpoint{3.858404in}{2.769145in}}%
\pgfpathlineto{\pgfqpoint{0.636356in}{2.769145in}}%
\pgfpathclose%
\pgfusepath{fill}%
\end{pgfscope}%
\begin{pgfscope}%
\pgfpathrectangle{\pgfqpoint{0.636356in}{0.874107in}}{\pgfqpoint{3.222048in}{1.895038in}} %
\pgfusepath{clip}%
\pgfsetbuttcap%
\pgfsetroundjoin%
\pgfsetlinewidth{1.003750pt}%
\definecolor{currentstroke}{rgb}{1.000000,0.000000,0.000000}%
\pgfsetstrokecolor{currentstroke}%
\pgfsetdash{{8.000000pt}{3.000000pt}}{0.000000pt}%
\pgfpathmoveto{\pgfqpoint{0.652547in}{1.414024in}}%
\pgfpathlineto{\pgfqpoint{0.701121in}{1.365862in}}%
\pgfpathlineto{\pgfqpoint{0.749694in}{1.321999in}}%
\pgfpathlineto{\pgfqpoint{0.798268in}{1.282043in}}%
\pgfpathlineto{\pgfqpoint{0.846841in}{1.245657in}}%
\pgfpathlineto{\pgfqpoint{0.895415in}{1.212515in}}%
\pgfpathlineto{\pgfqpoint{0.943989in}{1.182326in}}%
\pgfpathlineto{\pgfqpoint{0.992562in}{1.154835in}}%
\pgfpathlineto{\pgfqpoint{1.041136in}{1.129793in}}%
\pgfpathlineto{\pgfqpoint{1.089709in}{1.106985in}}%
\pgfpathlineto{\pgfqpoint{1.138283in}{1.086213in}}%
\pgfpathmoveto{\pgfqpoint{1.300195in}{1.029481in}}%
\pgfpathlineto{\pgfqpoint{1.364960in}{1.011286in}}%
\pgfpathlineto{\pgfqpoint{1.429724in}{0.995219in}}%
\pgfpathlineto{\pgfqpoint{1.494489in}{0.981029in}}%
\pgfpathlineto{\pgfqpoint{1.575445in}{0.965597in}}%
\pgfpathlineto{\pgfqpoint{1.672592in}{0.950016in}}%
\pgfpathlineto{\pgfqpoint{1.769739in}{0.937082in}}%
\pgfpathlineto{\pgfqpoint{1.866887in}{0.926346in}}%
\pgfpathlineto{\pgfqpoint{1.980225in}{0.916112in}}%
\pgfpathlineto{\pgfqpoint{2.109755in}{0.906847in}}%
\pgfpathlineto{\pgfqpoint{2.255475in}{0.898847in}}%
\pgfpathlineto{\pgfqpoint{2.417387in}{0.892224in}}%
\pgfpathlineto{\pgfqpoint{2.611682in}{0.886576in}}%
\pgfpathlineto{\pgfqpoint{2.838358in}{0.882169in}}%
\pgfpathlineto{\pgfqpoint{3.145991in}{0.878568in}}%
\pgfpathlineto{\pgfqpoint{3.566962in}{0.876092in}}%
\pgfpathlineto{\pgfqpoint{3.842212in}{0.875276in}}%
\pgfpathlineto{\pgfqpoint{3.842212in}{0.875276in}}%
\pgfusepath{stroke}%
\end{pgfscope}%
\begin{pgfscope}%
\pgfpathrectangle{\pgfqpoint{0.636356in}{0.874107in}}{\pgfqpoint{3.222048in}{1.895038in}} %
\pgfusepath{clip}%
\pgfsetbuttcap%
\pgfsetroundjoin%
\pgfsetlinewidth{1.003750pt}%
\definecolor{currentstroke}{rgb}{0.000000,0.000000,1.000000}%
\pgfsetstrokecolor{currentstroke}%
\pgfsetdash{{8.000000pt}{3.000000pt}}{0.000000pt}%
\pgfpathmoveto{\pgfqpoint{0.652547in}{1.837530in}}%
\pgfpathlineto{\pgfqpoint{0.668738in}{1.798729in}}%
\pgfpathlineto{\pgfqpoint{0.684929in}{1.761485in}}%
\pgfpathlineto{\pgfqpoint{0.717312in}{1.691424in}}%
\pgfpathlineto{\pgfqpoint{0.749694in}{1.626920in}}%
\pgfpathlineto{\pgfqpoint{0.782077in}{1.567488in}}%
\pgfpathlineto{\pgfqpoint{0.814459in}{1.512759in}}%
\pgfpathlineto{\pgfqpoint{0.846841in}{1.462345in}}%
\pgfpathlineto{\pgfqpoint{0.879224in}{1.415906in}}%
\pgfpathlineto{\pgfqpoint{0.911606in}{1.373144in}}%
\pgfpathlineto{\pgfqpoint{0.943989in}{1.333744in}}%
\pgfpathlineto{\pgfqpoint{0.976371in}{1.297468in}}%
\pgfpathlineto{\pgfqpoint{1.008753in}{1.264047in}}%
\pgfpathlineto{\pgfqpoint{1.041136in}{1.233265in}}%
\pgfpathlineto{\pgfqpoint{1.073518in}{1.204917in}}%
\pgfpathlineto{\pgfqpoint{1.105900in}{1.178798in}}%
\pgfpathlineto{\pgfqpoint{1.138283in}{1.154752in}}%
\pgfpathmoveto{\pgfqpoint{1.300195in}{1.060208in}}%
\pgfpathlineto{\pgfqpoint{1.348768in}{1.038589in}}%
\pgfpathlineto{\pgfqpoint{1.397342in}{1.019475in}}%
\pgfpathlineto{\pgfqpoint{1.462107in}{0.997494in}}%
\pgfpathlineto{\pgfqpoint{1.510680in}{0.983175in}}%
\pgfpathlineto{\pgfqpoint{1.559254in}{0.970513in}}%
\pgfpathlineto{\pgfqpoint{1.624019in}{0.955879in}}%
\pgfpathlineto{\pgfqpoint{1.688783in}{0.943494in}}%
\pgfpathlineto{\pgfqpoint{1.753548in}{0.932954in}}%
\pgfpathlineto{\pgfqpoint{1.834504in}{0.922016in}}%
\pgfpathlineto{\pgfqpoint{1.931651in}{0.911548in}}%
\pgfpathlineto{\pgfqpoint{2.028799in}{0.903369in}}%
\pgfpathlineto{\pgfqpoint{2.142137in}{0.896043in}}%
\pgfpathlineto{\pgfqpoint{2.287858in}{0.889263in}}%
\pgfpathlineto{\pgfqpoint{2.449770in}{0.884151in}}%
\pgfpathlineto{\pgfqpoint{2.660255in}{0.879994in}}%
\pgfpathlineto{\pgfqpoint{2.935505in}{0.877032in}}%
\pgfpathlineto{\pgfqpoint{3.356476in}{0.875112in}}%
\pgfpathlineto{\pgfqpoint{3.842212in}{0.874400in}}%
\pgfpathlineto{\pgfqpoint{3.842212in}{0.874400in}}%
\pgfusepath{stroke}%
\end{pgfscope}%
\begin{pgfscope}%
\pgfsetrectcap%
\pgfsetmiterjoin%
\pgfsetlinewidth{1.003750pt}%
\definecolor{currentstroke}{rgb}{0.000000,0.000000,0.000000}%
\pgfsetstrokecolor{currentstroke}%
\pgfsetdash{}{0pt}%
\pgfpathmoveto{\pgfqpoint{0.636356in}{2.769145in}}%
\pgfpathlineto{\pgfqpoint{3.858404in}{2.769145in}}%
\pgfusepath{stroke}%
\end{pgfscope}%
\begin{pgfscope}%
\pgfsetrectcap%
\pgfsetmiterjoin%
\pgfsetlinewidth{1.003750pt}%
\definecolor{currentstroke}{rgb}{0.000000,0.000000,0.000000}%
\pgfsetstrokecolor{currentstroke}%
\pgfsetdash{}{0pt}%
\pgfpathmoveto{\pgfqpoint{3.858404in}{0.874107in}}%
\pgfpathlineto{\pgfqpoint{3.858404in}{2.769145in}}%
\pgfusepath{stroke}%
\end{pgfscope}%
\begin{pgfscope}%
\pgfsetrectcap%
\pgfsetmiterjoin%
\pgfsetlinewidth{1.003750pt}%
\definecolor{currentstroke}{rgb}{0.000000,0.000000,0.000000}%
\pgfsetstrokecolor{currentstroke}%
\pgfsetdash{}{0pt}%
\pgfpathmoveto{\pgfqpoint{0.636356in}{0.874107in}}%
\pgfpathlineto{\pgfqpoint{3.858404in}{0.874107in}}%
\pgfusepath{stroke}%
\end{pgfscope}%
\begin{pgfscope}%
\pgfsetrectcap%
\pgfsetmiterjoin%
\pgfsetlinewidth{1.003750pt}%
\definecolor{currentstroke}{rgb}{0.000000,0.000000,0.000000}%
\pgfsetstrokecolor{currentstroke}%
\pgfsetdash{}{0pt}%
\pgfpathmoveto{\pgfqpoint{0.636356in}{0.874107in}}%
\pgfpathlineto{\pgfqpoint{0.636356in}{2.769145in}}%
\pgfusepath{stroke}%
\end{pgfscope}%
\begin{pgfscope}%
\pgfsetbuttcap%
\pgfsetroundjoin%
\definecolor{currentfill}{rgb}{0.000000,0.000000,0.000000}%
\pgfsetfillcolor{currentfill}%
\pgfsetlinewidth{0.501875pt}%
\definecolor{currentstroke}{rgb}{0.000000,0.000000,0.000000}%
\pgfsetstrokecolor{currentstroke}%
\pgfsetdash{}{0pt}%
\pgfsys@defobject{currentmarker}{\pgfqpoint{0.000000in}{0.000000in}}{\pgfqpoint{0.000000in}{0.069444in}}{%
\pgfpathmoveto{\pgfqpoint{0.000000in}{0.000000in}}%
\pgfpathlineto{\pgfqpoint{0.000000in}{0.069444in}}%
\pgfusepath{stroke,fill}%
}%
\begin{pgfscope}%
\pgfsys@transformshift{0.789787in}{0.874107in}%
\pgfsys@useobject{currentmarker}{}%
\end{pgfscope}%
\end{pgfscope}%
\begin{pgfscope}%
\pgfsetbuttcap%
\pgfsetroundjoin%
\definecolor{currentfill}{rgb}{0.000000,0.000000,0.000000}%
\pgfsetfillcolor{currentfill}%
\pgfsetlinewidth{0.501875pt}%
\definecolor{currentstroke}{rgb}{0.000000,0.000000,0.000000}%
\pgfsetstrokecolor{currentstroke}%
\pgfsetdash{}{0pt}%
\pgfsys@defobject{currentmarker}{\pgfqpoint{0.000000in}{-0.069444in}}{\pgfqpoint{0.000000in}{0.000000in}}{%
\pgfpathmoveto{\pgfqpoint{0.000000in}{0.000000in}}%
\pgfpathlineto{\pgfqpoint{0.000000in}{-0.069444in}}%
\pgfusepath{stroke,fill}%
}%
\begin{pgfscope}%
\pgfsys@transformshift{0.789787in}{2.769145in}%
\pgfsys@useobject{currentmarker}{}%
\end{pgfscope}%
\end{pgfscope}%
\begin{pgfscope}%
\pgfsetbuttcap%
\pgfsetroundjoin%
\definecolor{currentfill}{rgb}{0.000000,0.000000,0.000000}%
\pgfsetfillcolor{currentfill}%
\pgfsetlinewidth{0.501875pt}%
\definecolor{currentstroke}{rgb}{0.000000,0.000000,0.000000}%
\pgfsetstrokecolor{currentstroke}%
\pgfsetdash{}{0pt}%
\pgfsys@defobject{currentmarker}{\pgfqpoint{0.000000in}{0.000000in}}{\pgfqpoint{0.000000in}{0.069444in}}{%
\pgfpathmoveto{\pgfqpoint{0.000000in}{0.000000in}}%
\pgfpathlineto{\pgfqpoint{0.000000in}{0.069444in}}%
\pgfusepath{stroke,fill}%
}%
\begin{pgfscope}%
\pgfsys@transformshift{1.556941in}{0.874107in}%
\pgfsys@useobject{currentmarker}{}%
\end{pgfscope}%
\end{pgfscope}%
\begin{pgfscope}%
\pgfsetbuttcap%
\pgfsetroundjoin%
\definecolor{currentfill}{rgb}{0.000000,0.000000,0.000000}%
\pgfsetfillcolor{currentfill}%
\pgfsetlinewidth{0.501875pt}%
\definecolor{currentstroke}{rgb}{0.000000,0.000000,0.000000}%
\pgfsetstrokecolor{currentstroke}%
\pgfsetdash{}{0pt}%
\pgfsys@defobject{currentmarker}{\pgfqpoint{0.000000in}{-0.069444in}}{\pgfqpoint{0.000000in}{0.000000in}}{%
\pgfpathmoveto{\pgfqpoint{0.000000in}{0.000000in}}%
\pgfpathlineto{\pgfqpoint{0.000000in}{-0.069444in}}%
\pgfusepath{stroke,fill}%
}%
\begin{pgfscope}%
\pgfsys@transformshift{1.556941in}{2.769145in}%
\pgfsys@useobject{currentmarker}{}%
\end{pgfscope}%
\end{pgfscope}%
\begin{pgfscope}%
\pgfsetbuttcap%
\pgfsetroundjoin%
\definecolor{currentfill}{rgb}{0.000000,0.000000,0.000000}%
\pgfsetfillcolor{currentfill}%
\pgfsetlinewidth{0.501875pt}%
\definecolor{currentstroke}{rgb}{0.000000,0.000000,0.000000}%
\pgfsetstrokecolor{currentstroke}%
\pgfsetdash{}{0pt}%
\pgfsys@defobject{currentmarker}{\pgfqpoint{0.000000in}{0.000000in}}{\pgfqpoint{0.000000in}{0.069444in}}{%
\pgfpathmoveto{\pgfqpoint{0.000000in}{0.000000in}}%
\pgfpathlineto{\pgfqpoint{0.000000in}{0.069444in}}%
\pgfusepath{stroke,fill}%
}%
\begin{pgfscope}%
\pgfsys@transformshift{2.324095in}{0.874107in}%
\pgfsys@useobject{currentmarker}{}%
\end{pgfscope}%
\end{pgfscope}%
\begin{pgfscope}%
\pgfsetbuttcap%
\pgfsetroundjoin%
\definecolor{currentfill}{rgb}{0.000000,0.000000,0.000000}%
\pgfsetfillcolor{currentfill}%
\pgfsetlinewidth{0.501875pt}%
\definecolor{currentstroke}{rgb}{0.000000,0.000000,0.000000}%
\pgfsetstrokecolor{currentstroke}%
\pgfsetdash{}{0pt}%
\pgfsys@defobject{currentmarker}{\pgfqpoint{0.000000in}{-0.069444in}}{\pgfqpoint{0.000000in}{0.000000in}}{%
\pgfpathmoveto{\pgfqpoint{0.000000in}{0.000000in}}%
\pgfpathlineto{\pgfqpoint{0.000000in}{-0.069444in}}%
\pgfusepath{stroke,fill}%
}%
\begin{pgfscope}%
\pgfsys@transformshift{2.324095in}{2.769145in}%
\pgfsys@useobject{currentmarker}{}%
\end{pgfscope}%
\end{pgfscope}%
\begin{pgfscope}%
\pgfsetbuttcap%
\pgfsetroundjoin%
\definecolor{currentfill}{rgb}{0.000000,0.000000,0.000000}%
\pgfsetfillcolor{currentfill}%
\pgfsetlinewidth{0.501875pt}%
\definecolor{currentstroke}{rgb}{0.000000,0.000000,0.000000}%
\pgfsetstrokecolor{currentstroke}%
\pgfsetdash{}{0pt}%
\pgfsys@defobject{currentmarker}{\pgfqpoint{0.000000in}{0.000000in}}{\pgfqpoint{0.000000in}{0.069444in}}{%
\pgfpathmoveto{\pgfqpoint{0.000000in}{0.000000in}}%
\pgfpathlineto{\pgfqpoint{0.000000in}{0.069444in}}%
\pgfusepath{stroke,fill}%
}%
\begin{pgfscope}%
\pgfsys@transformshift{3.091249in}{0.874107in}%
\pgfsys@useobject{currentmarker}{}%
\end{pgfscope}%
\end{pgfscope}%
\begin{pgfscope}%
\pgfsetbuttcap%
\pgfsetroundjoin%
\definecolor{currentfill}{rgb}{0.000000,0.000000,0.000000}%
\pgfsetfillcolor{currentfill}%
\pgfsetlinewidth{0.501875pt}%
\definecolor{currentstroke}{rgb}{0.000000,0.000000,0.000000}%
\pgfsetstrokecolor{currentstroke}%
\pgfsetdash{}{0pt}%
\pgfsys@defobject{currentmarker}{\pgfqpoint{0.000000in}{-0.069444in}}{\pgfqpoint{0.000000in}{0.000000in}}{%
\pgfpathmoveto{\pgfqpoint{0.000000in}{0.000000in}}%
\pgfpathlineto{\pgfqpoint{0.000000in}{-0.069444in}}%
\pgfusepath{stroke,fill}%
}%
\begin{pgfscope}%
\pgfsys@transformshift{3.091249in}{2.769145in}%
\pgfsys@useobject{currentmarker}{}%
\end{pgfscope}%
\end{pgfscope}%
\begin{pgfscope}%
\pgfsetbuttcap%
\pgfsetroundjoin%
\definecolor{currentfill}{rgb}{0.000000,0.000000,0.000000}%
\pgfsetfillcolor{currentfill}%
\pgfsetlinewidth{0.501875pt}%
\definecolor{currentstroke}{rgb}{0.000000,0.000000,0.000000}%
\pgfsetstrokecolor{currentstroke}%
\pgfsetdash{}{0pt}%
\pgfsys@defobject{currentmarker}{\pgfqpoint{0.000000in}{0.000000in}}{\pgfqpoint{0.000000in}{0.069444in}}{%
\pgfpathmoveto{\pgfqpoint{0.000000in}{0.000000in}}%
\pgfpathlineto{\pgfqpoint{0.000000in}{0.069444in}}%
\pgfusepath{stroke,fill}%
}%
\begin{pgfscope}%
\pgfsys@transformshift{3.858404in}{0.874107in}%
\pgfsys@useobject{currentmarker}{}%
\end{pgfscope}%
\end{pgfscope}%
\begin{pgfscope}%
\pgfsetbuttcap%
\pgfsetroundjoin%
\definecolor{currentfill}{rgb}{0.000000,0.000000,0.000000}%
\pgfsetfillcolor{currentfill}%
\pgfsetlinewidth{0.501875pt}%
\definecolor{currentstroke}{rgb}{0.000000,0.000000,0.000000}%
\pgfsetstrokecolor{currentstroke}%
\pgfsetdash{}{0pt}%
\pgfsys@defobject{currentmarker}{\pgfqpoint{0.000000in}{-0.069444in}}{\pgfqpoint{0.000000in}{0.000000in}}{%
\pgfpathmoveto{\pgfqpoint{0.000000in}{0.000000in}}%
\pgfpathlineto{\pgfqpoint{0.000000in}{-0.069444in}}%
\pgfusepath{stroke,fill}%
}%
\begin{pgfscope}%
\pgfsys@transformshift{3.858404in}{2.769145in}%
\pgfsys@useobject{currentmarker}{}%
\end{pgfscope}%
\end{pgfscope}%
\begin{pgfscope}%
\pgfsetbuttcap%
\pgfsetroundjoin%
\definecolor{currentfill}{rgb}{0.000000,0.000000,0.000000}%
\pgfsetfillcolor{currentfill}%
\pgfsetlinewidth{0.501875pt}%
\definecolor{currentstroke}{rgb}{0.000000,0.000000,0.000000}%
\pgfsetstrokecolor{currentstroke}%
\pgfsetdash{}{0pt}%
\pgfsys@defobject{currentmarker}{\pgfqpoint{0.000000in}{0.000000in}}{\pgfqpoint{0.069444in}{0.000000in}}{%
\pgfpathmoveto{\pgfqpoint{0.000000in}{0.000000in}}%
\pgfpathlineto{\pgfqpoint{0.069444in}{0.000000in}}%
\pgfusepath{stroke,fill}%
}%
\begin{pgfscope}%
\pgfsys@transformshift{0.636356in}{0.874107in}%
\pgfsys@useobject{currentmarker}{}%
\end{pgfscope}%
\end{pgfscope}%
\begin{pgfscope}%
\pgfsetbuttcap%
\pgfsetroundjoin%
\definecolor{currentfill}{rgb}{0.000000,0.000000,0.000000}%
\pgfsetfillcolor{currentfill}%
\pgfsetlinewidth{0.501875pt}%
\definecolor{currentstroke}{rgb}{0.000000,0.000000,0.000000}%
\pgfsetstrokecolor{currentstroke}%
\pgfsetdash{}{0pt}%
\pgfsys@defobject{currentmarker}{\pgfqpoint{-0.069444in}{0.000000in}}{\pgfqpoint{0.000000in}{0.000000in}}{%
\pgfpathmoveto{\pgfqpoint{0.000000in}{0.000000in}}%
\pgfpathlineto{\pgfqpoint{-0.069444in}{0.000000in}}%
\pgfusepath{stroke,fill}%
}%
\begin{pgfscope}%
\pgfsys@transformshift{3.858404in}{0.874107in}%
\pgfsys@useobject{currentmarker}{}%
\end{pgfscope}%
\end{pgfscope}%
\begin{pgfscope}%
\pgftext[x=0.566911in,y=0.874107in,right,]{\rmfamily\fontsize{8.000000}{9.600000}\selectfont 0}%
\end{pgfscope}%
\begin{pgfscope}%
\pgfsetbuttcap%
\pgfsetroundjoin%
\definecolor{currentfill}{rgb}{0.000000,0.000000,0.000000}%
\pgfsetfillcolor{currentfill}%
\pgfsetlinewidth{0.501875pt}%
\definecolor{currentstroke}{rgb}{0.000000,0.000000,0.000000}%
\pgfsetstrokecolor{currentstroke}%
\pgfsetdash{}{0pt}%
\pgfsys@defobject{currentmarker}{\pgfqpoint{0.000000in}{0.000000in}}{\pgfqpoint{0.069444in}{0.000000in}}{%
\pgfpathmoveto{\pgfqpoint{0.000000in}{0.000000in}}%
\pgfpathlineto{\pgfqpoint{0.069444in}{0.000000in}}%
\pgfusepath{stroke,fill}%
}%
\begin{pgfscope}%
\pgfsys@transformshift{0.636356in}{1.189947in}%
\pgfsys@useobject{currentmarker}{}%
\end{pgfscope}%
\end{pgfscope}%
\begin{pgfscope}%
\pgfsetbuttcap%
\pgfsetroundjoin%
\definecolor{currentfill}{rgb}{0.000000,0.000000,0.000000}%
\pgfsetfillcolor{currentfill}%
\pgfsetlinewidth{0.501875pt}%
\definecolor{currentstroke}{rgb}{0.000000,0.000000,0.000000}%
\pgfsetstrokecolor{currentstroke}%
\pgfsetdash{}{0pt}%
\pgfsys@defobject{currentmarker}{\pgfqpoint{-0.069444in}{0.000000in}}{\pgfqpoint{0.000000in}{0.000000in}}{%
\pgfpathmoveto{\pgfqpoint{0.000000in}{0.000000in}}%
\pgfpathlineto{\pgfqpoint{-0.069444in}{0.000000in}}%
\pgfusepath{stroke,fill}%
}%
\begin{pgfscope}%
\pgfsys@transformshift{3.858404in}{1.189947in}%
\pgfsys@useobject{currentmarker}{}%
\end{pgfscope}%
\end{pgfscope}%
\begin{pgfscope}%
\pgftext[x=0.566911in,y=1.189947in,right,]{\rmfamily\fontsize{8.000000}{9.600000}\selectfont 5}%
\end{pgfscope}%
\begin{pgfscope}%
\pgfsetbuttcap%
\pgfsetroundjoin%
\definecolor{currentfill}{rgb}{0.000000,0.000000,0.000000}%
\pgfsetfillcolor{currentfill}%
\pgfsetlinewidth{0.501875pt}%
\definecolor{currentstroke}{rgb}{0.000000,0.000000,0.000000}%
\pgfsetstrokecolor{currentstroke}%
\pgfsetdash{}{0pt}%
\pgfsys@defobject{currentmarker}{\pgfqpoint{0.000000in}{0.000000in}}{\pgfqpoint{0.069444in}{0.000000in}}{%
\pgfpathmoveto{\pgfqpoint{0.000000in}{0.000000in}}%
\pgfpathlineto{\pgfqpoint{0.069444in}{0.000000in}}%
\pgfusepath{stroke,fill}%
}%
\begin{pgfscope}%
\pgfsys@transformshift{0.636356in}{1.505786in}%
\pgfsys@useobject{currentmarker}{}%
\end{pgfscope}%
\end{pgfscope}%
\begin{pgfscope}%
\pgfsetbuttcap%
\pgfsetroundjoin%
\definecolor{currentfill}{rgb}{0.000000,0.000000,0.000000}%
\pgfsetfillcolor{currentfill}%
\pgfsetlinewidth{0.501875pt}%
\definecolor{currentstroke}{rgb}{0.000000,0.000000,0.000000}%
\pgfsetstrokecolor{currentstroke}%
\pgfsetdash{}{0pt}%
\pgfsys@defobject{currentmarker}{\pgfqpoint{-0.069444in}{0.000000in}}{\pgfqpoint{0.000000in}{0.000000in}}{%
\pgfpathmoveto{\pgfqpoint{0.000000in}{0.000000in}}%
\pgfpathlineto{\pgfqpoint{-0.069444in}{0.000000in}}%
\pgfusepath{stroke,fill}%
}%
\begin{pgfscope}%
\pgfsys@transformshift{3.858404in}{1.505786in}%
\pgfsys@useobject{currentmarker}{}%
\end{pgfscope}%
\end{pgfscope}%
\begin{pgfscope}%
\pgftext[x=0.566911in,y=1.505786in,right,]{\rmfamily\fontsize{8.000000}{9.600000}\selectfont 10}%
\end{pgfscope}%
\begin{pgfscope}%
\pgfsetbuttcap%
\pgfsetroundjoin%
\definecolor{currentfill}{rgb}{0.000000,0.000000,0.000000}%
\pgfsetfillcolor{currentfill}%
\pgfsetlinewidth{0.501875pt}%
\definecolor{currentstroke}{rgb}{0.000000,0.000000,0.000000}%
\pgfsetstrokecolor{currentstroke}%
\pgfsetdash{}{0pt}%
\pgfsys@defobject{currentmarker}{\pgfqpoint{0.000000in}{0.000000in}}{\pgfqpoint{0.069444in}{0.000000in}}{%
\pgfpathmoveto{\pgfqpoint{0.000000in}{0.000000in}}%
\pgfpathlineto{\pgfqpoint{0.069444in}{0.000000in}}%
\pgfusepath{stroke,fill}%
}%
\begin{pgfscope}%
\pgfsys@transformshift{0.636356in}{1.821626in}%
\pgfsys@useobject{currentmarker}{}%
\end{pgfscope}%
\end{pgfscope}%
\begin{pgfscope}%
\pgfsetbuttcap%
\pgfsetroundjoin%
\definecolor{currentfill}{rgb}{0.000000,0.000000,0.000000}%
\pgfsetfillcolor{currentfill}%
\pgfsetlinewidth{0.501875pt}%
\definecolor{currentstroke}{rgb}{0.000000,0.000000,0.000000}%
\pgfsetstrokecolor{currentstroke}%
\pgfsetdash{}{0pt}%
\pgfsys@defobject{currentmarker}{\pgfqpoint{-0.069444in}{0.000000in}}{\pgfqpoint{0.000000in}{0.000000in}}{%
\pgfpathmoveto{\pgfqpoint{0.000000in}{0.000000in}}%
\pgfpathlineto{\pgfqpoint{-0.069444in}{0.000000in}}%
\pgfusepath{stroke,fill}%
}%
\begin{pgfscope}%
\pgfsys@transformshift{3.858404in}{1.821626in}%
\pgfsys@useobject{currentmarker}{}%
\end{pgfscope}%
\end{pgfscope}%
\begin{pgfscope}%
\pgftext[x=0.566911in,y=1.821626in,right,]{\rmfamily\fontsize{8.000000}{9.600000}\selectfont 15}%
\end{pgfscope}%
\begin{pgfscope}%
\pgfsetbuttcap%
\pgfsetroundjoin%
\definecolor{currentfill}{rgb}{0.000000,0.000000,0.000000}%
\pgfsetfillcolor{currentfill}%
\pgfsetlinewidth{0.501875pt}%
\definecolor{currentstroke}{rgb}{0.000000,0.000000,0.000000}%
\pgfsetstrokecolor{currentstroke}%
\pgfsetdash{}{0pt}%
\pgfsys@defobject{currentmarker}{\pgfqpoint{0.000000in}{0.000000in}}{\pgfqpoint{0.069444in}{0.000000in}}{%
\pgfpathmoveto{\pgfqpoint{0.000000in}{0.000000in}}%
\pgfpathlineto{\pgfqpoint{0.069444in}{0.000000in}}%
\pgfusepath{stroke,fill}%
}%
\begin{pgfscope}%
\pgfsys@transformshift{0.636356in}{2.137465in}%
\pgfsys@useobject{currentmarker}{}%
\end{pgfscope}%
\end{pgfscope}%
\begin{pgfscope}%
\pgfsetbuttcap%
\pgfsetroundjoin%
\definecolor{currentfill}{rgb}{0.000000,0.000000,0.000000}%
\pgfsetfillcolor{currentfill}%
\pgfsetlinewidth{0.501875pt}%
\definecolor{currentstroke}{rgb}{0.000000,0.000000,0.000000}%
\pgfsetstrokecolor{currentstroke}%
\pgfsetdash{}{0pt}%
\pgfsys@defobject{currentmarker}{\pgfqpoint{-0.069444in}{0.000000in}}{\pgfqpoint{0.000000in}{0.000000in}}{%
\pgfpathmoveto{\pgfqpoint{0.000000in}{0.000000in}}%
\pgfpathlineto{\pgfqpoint{-0.069444in}{0.000000in}}%
\pgfusepath{stroke,fill}%
}%
\begin{pgfscope}%
\pgfsys@transformshift{3.858404in}{2.137465in}%
\pgfsys@useobject{currentmarker}{}%
\end{pgfscope}%
\end{pgfscope}%
\begin{pgfscope}%
\pgftext[x=0.566911in,y=2.137465in,right,]{\rmfamily\fontsize{8.000000}{9.600000}\selectfont 20}%
\end{pgfscope}%
\begin{pgfscope}%
\pgfsetbuttcap%
\pgfsetroundjoin%
\definecolor{currentfill}{rgb}{0.000000,0.000000,0.000000}%
\pgfsetfillcolor{currentfill}%
\pgfsetlinewidth{0.501875pt}%
\definecolor{currentstroke}{rgb}{0.000000,0.000000,0.000000}%
\pgfsetstrokecolor{currentstroke}%
\pgfsetdash{}{0pt}%
\pgfsys@defobject{currentmarker}{\pgfqpoint{0.000000in}{0.000000in}}{\pgfqpoint{0.069444in}{0.000000in}}{%
\pgfpathmoveto{\pgfqpoint{0.000000in}{0.000000in}}%
\pgfpathlineto{\pgfqpoint{0.069444in}{0.000000in}}%
\pgfusepath{stroke,fill}%
}%
\begin{pgfscope}%
\pgfsys@transformshift{0.636356in}{2.453305in}%
\pgfsys@useobject{currentmarker}{}%
\end{pgfscope}%
\end{pgfscope}%
\begin{pgfscope}%
\pgfsetbuttcap%
\pgfsetroundjoin%
\definecolor{currentfill}{rgb}{0.000000,0.000000,0.000000}%
\pgfsetfillcolor{currentfill}%
\pgfsetlinewidth{0.501875pt}%
\definecolor{currentstroke}{rgb}{0.000000,0.000000,0.000000}%
\pgfsetstrokecolor{currentstroke}%
\pgfsetdash{}{0pt}%
\pgfsys@defobject{currentmarker}{\pgfqpoint{-0.069444in}{0.000000in}}{\pgfqpoint{0.000000in}{0.000000in}}{%
\pgfpathmoveto{\pgfqpoint{0.000000in}{0.000000in}}%
\pgfpathlineto{\pgfqpoint{-0.069444in}{0.000000in}}%
\pgfusepath{stroke,fill}%
}%
\begin{pgfscope}%
\pgfsys@transformshift{3.858404in}{2.453305in}%
\pgfsys@useobject{currentmarker}{}%
\end{pgfscope}%
\end{pgfscope}%
\begin{pgfscope}%
\pgftext[x=0.566911in,y=2.453305in,right,]{\rmfamily\fontsize{8.000000}{9.600000}\selectfont 25}%
\end{pgfscope}%
\begin{pgfscope}%
\pgfsetbuttcap%
\pgfsetroundjoin%
\definecolor{currentfill}{rgb}{0.000000,0.000000,0.000000}%
\pgfsetfillcolor{currentfill}%
\pgfsetlinewidth{0.501875pt}%
\definecolor{currentstroke}{rgb}{0.000000,0.000000,0.000000}%
\pgfsetstrokecolor{currentstroke}%
\pgfsetdash{}{0pt}%
\pgfsys@defobject{currentmarker}{\pgfqpoint{0.000000in}{0.000000in}}{\pgfqpoint{0.069444in}{0.000000in}}{%
\pgfpathmoveto{\pgfqpoint{0.000000in}{0.000000in}}%
\pgfpathlineto{\pgfqpoint{0.069444in}{0.000000in}}%
\pgfusepath{stroke,fill}%
}%
\begin{pgfscope}%
\pgfsys@transformshift{0.636356in}{2.769145in}%
\pgfsys@useobject{currentmarker}{}%
\end{pgfscope}%
\end{pgfscope}%
\begin{pgfscope}%
\pgfsetbuttcap%
\pgfsetroundjoin%
\definecolor{currentfill}{rgb}{0.000000,0.000000,0.000000}%
\pgfsetfillcolor{currentfill}%
\pgfsetlinewidth{0.501875pt}%
\definecolor{currentstroke}{rgb}{0.000000,0.000000,0.000000}%
\pgfsetstrokecolor{currentstroke}%
\pgfsetdash{}{0pt}%
\pgfsys@defobject{currentmarker}{\pgfqpoint{-0.069444in}{0.000000in}}{\pgfqpoint{0.000000in}{0.000000in}}{%
\pgfpathmoveto{\pgfqpoint{0.000000in}{0.000000in}}%
\pgfpathlineto{\pgfqpoint{-0.069444in}{0.000000in}}%
\pgfusepath{stroke,fill}%
}%
\begin{pgfscope}%
\pgfsys@transformshift{3.858404in}{2.769145in}%
\pgfsys@useobject{currentmarker}{}%
\end{pgfscope}%
\end{pgfscope}%
\begin{pgfscope}%
\pgftext[x=0.566911in,y=2.769145in,right,]{\rmfamily\fontsize{8.000000}{9.600000}\selectfont 30}%
\end{pgfscope}%
\begin{pgfscope}%
\pgftext[x=0.356082in,y=1.821626in,,bottom,rotate=90.000000]{\rmfamily\fontsize{9.000000}{10.800000}\selectfont Candidates}%
\end{pgfscope}%
\begin{pgfscope}%
\pgfsetrectcap%
\pgfsetroundjoin%
\pgfsetlinewidth{1.003750pt}%
\definecolor{currentstroke}{rgb}{1.000000,0.000000,0.000000}%
\pgfsetstrokecolor{currentstroke}%
\pgfsetdash{}{0pt}%
\pgfpathmoveto{\pgfqpoint{0.652547in}{2.377447in}}%
\pgfpathlineto{\pgfqpoint{0.668738in}{2.322092in}}%
\pgfpathlineto{\pgfqpoint{0.684929in}{2.268799in}}%
\pgfpathlineto{\pgfqpoint{0.717312in}{2.168098in}}%
\pgfpathlineto{\pgfqpoint{0.749694in}{2.074813in}}%
\pgfpathlineto{\pgfqpoint{0.782077in}{1.988330in}}%
\pgfpathlineto{\pgfqpoint{0.814459in}{1.908188in}}%
\pgfpathlineto{\pgfqpoint{0.846841in}{1.833896in}}%
\pgfpathlineto{\pgfqpoint{0.879224in}{1.765018in}}%
\pgfpathlineto{\pgfqpoint{0.911606in}{1.701175in}}%
\pgfpathlineto{\pgfqpoint{0.943989in}{1.641963in}}%
\pgfpathlineto{\pgfqpoint{0.976371in}{1.587076in}}%
\pgfpathlineto{\pgfqpoint{1.008753in}{1.536166in}}%
\pgfpathlineto{\pgfqpoint{1.041136in}{1.488951in}}%
\pgfpathlineto{\pgfqpoint{1.073518in}{1.445163in}}%
\pgfpathlineto{\pgfqpoint{1.105900in}{1.404533in}}%
\pgfpathlineto{\pgfqpoint{1.138283in}{1.366858in}}%
\pgfpathmoveto{\pgfqpoint{1.300195in}{1.215582in}}%
\pgfpathlineto{\pgfqpoint{1.348768in}{1.180092in}}%
\pgfpathlineto{\pgfqpoint{1.397342in}{1.148341in}}%
\pgfpathlineto{\pgfqpoint{1.445916in}{1.120011in}}%
\pgfpathlineto{\pgfqpoint{1.494489in}{1.094655in}}%
\pgfpathlineto{\pgfqpoint{1.543063in}{1.071949in}}%
\pgfpathlineto{\pgfqpoint{1.591636in}{1.051614in}}%
\pgfpathlineto{\pgfqpoint{1.640210in}{1.033399in}}%
\pgfpathlineto{\pgfqpoint{1.688783in}{1.017079in}}%
\pgfpathlineto{\pgfqpoint{1.737357in}{1.002456in}}%
\pgfpathlineto{\pgfqpoint{1.802122in}{0.985300in}}%
\pgfpathlineto{\pgfqpoint{1.866887in}{0.970489in}}%
\pgfpathlineto{\pgfqpoint{1.931651in}{0.957664in}}%
\pgfpathlineto{\pgfqpoint{2.012607in}{0.944043in}}%
\pgfpathlineto{\pgfqpoint{2.109755in}{0.930669in}}%
\pgfpathlineto{\pgfqpoint{2.206902in}{0.919887in}}%
\pgfpathlineto{\pgfqpoint{2.304049in}{0.911186in}}%
\pgfpathlineto{\pgfqpoint{2.417387in}{0.903132in}}%
\pgfpathlineto{\pgfqpoint{2.546917in}{0.896078in}}%
\pgfpathlineto{\pgfqpoint{2.708829in}{0.889653in}}%
\pgfpathlineto{\pgfqpoint{2.903123in}{0.884402in}}%
\pgfpathlineto{\pgfqpoint{3.145991in}{0.880283in}}%
\pgfpathlineto{\pgfqpoint{3.469815in}{0.877253in}}%
\pgfpathlineto{\pgfqpoint{3.842212in}{0.875569in}}%
\pgfpathlineto{\pgfqpoint{3.842212in}{0.875569in}}%
\pgfusepath{stroke}%
\end{pgfscope}%
\begin{pgfscope}%
\pgfpathrectangle{\pgfqpoint{0.636356in}{0.874107in}}{\pgfqpoint{3.222048in}{1.895038in}} %
\pgfusepath{clip}%
\pgfsetbuttcap%
\pgfsetroundjoin%
\pgfsetlinewidth{0.501875pt}%
\definecolor{currentstroke}{rgb}{0.000000,0.000000,0.000000}%
\pgfsetstrokecolor{currentstroke}%
\pgfsetdash{}{0pt}%
\pgfpathmoveto{\pgfqpoint{0.656494in}{1.913484in}}%
\pgfpathlineto{\pgfqpoint{0.656494in}{2.557859in}}%
\pgfusepath{stroke}%
\end{pgfscope}%
\begin{pgfscope}%
\pgfpathrectangle{\pgfqpoint{0.636356in}{0.874107in}}{\pgfqpoint{3.222048in}{1.895038in}} %
\pgfusepath{clip}%
\pgfsetbuttcap%
\pgfsetroundjoin%
\pgfsetlinewidth{0.501875pt}%
\definecolor{currentstroke}{rgb}{0.000000,0.000000,0.000000}%
\pgfsetstrokecolor{currentstroke}%
\pgfsetdash{}{0pt}%
\pgfpathmoveto{\pgfqpoint{0.696769in}{2.026245in}}%
\pgfpathlineto{\pgfqpoint{0.696769in}{2.697464in}}%
\pgfusepath{stroke}%
\end{pgfscope}%
\begin{pgfscope}%
\pgfpathrectangle{\pgfqpoint{0.636356in}{0.874107in}}{\pgfqpoint{3.222048in}{1.895038in}} %
\pgfusepath{clip}%
\pgfsetbuttcap%
\pgfsetroundjoin%
\pgfsetlinewidth{0.501875pt}%
\definecolor{currentstroke}{rgb}{0.000000,0.000000,0.000000}%
\pgfsetstrokecolor{currentstroke}%
\pgfsetdash{}{0pt}%
\pgfpathmoveto{\pgfqpoint{0.737045in}{1.470497in}}%
\pgfpathlineto{\pgfqpoint{0.737045in}{1.992026in}}%
\pgfusepath{stroke}%
\end{pgfscope}%
\begin{pgfscope}%
\pgfpathrectangle{\pgfqpoint{0.636356in}{0.874107in}}{\pgfqpoint{3.222048in}{1.895038in}} %
\pgfusepath{clip}%
\pgfsetbuttcap%
\pgfsetroundjoin%
\pgfsetlinewidth{0.501875pt}%
\definecolor{currentstroke}{rgb}{0.000000,0.000000,0.000000}%
\pgfsetstrokecolor{currentstroke}%
\pgfsetdash{}{0pt}%
\pgfpathmoveto{\pgfqpoint{0.777320in}{1.524958in}}%
\pgfpathlineto{\pgfqpoint{0.777320in}{2.063583in}}%
\pgfusepath{stroke}%
\end{pgfscope}%
\begin{pgfscope}%
\pgfpathrectangle{\pgfqpoint{0.636356in}{0.874107in}}{\pgfqpoint{3.222048in}{1.895038in}} %
\pgfusepath{clip}%
\pgfsetbuttcap%
\pgfsetroundjoin%
\pgfsetlinewidth{0.501875pt}%
\definecolor{currentstroke}{rgb}{0.000000,0.000000,0.000000}%
\pgfsetstrokecolor{currentstroke}%
\pgfsetdash{}{0pt}%
\pgfpathmoveto{\pgfqpoint{0.817596in}{1.913484in}}%
\pgfpathlineto{\pgfqpoint{0.817596in}{2.557859in}}%
\pgfusepath{stroke}%
\end{pgfscope}%
\begin{pgfscope}%
\pgfpathrectangle{\pgfqpoint{0.636356in}{0.874107in}}{\pgfqpoint{3.222048in}{1.895038in}} %
\pgfusepath{clip}%
\pgfsetbuttcap%
\pgfsetroundjoin%
\pgfsetlinewidth{0.501875pt}%
\definecolor{currentstroke}{rgb}{0.000000,0.000000,0.000000}%
\pgfsetstrokecolor{currentstroke}%
\pgfsetdash{}{0pt}%
\pgfpathmoveto{\pgfqpoint{0.857872in}{1.801398in}}%
\pgfpathlineto{\pgfqpoint{0.857872in}{2.417628in}}%
\pgfusepath{stroke}%
\end{pgfscope}%
\begin{pgfscope}%
\pgfpathrectangle{\pgfqpoint{0.636356in}{0.874107in}}{\pgfqpoint{3.222048in}{1.895038in}} %
\pgfusepath{clip}%
\pgfsetbuttcap%
\pgfsetroundjoin%
\pgfsetlinewidth{0.501875pt}%
\definecolor{currentstroke}{rgb}{0.000000,0.000000,0.000000}%
\pgfsetstrokecolor{currentstroke}%
\pgfsetdash{}{0pt}%
\pgfpathmoveto{\pgfqpoint{0.898147in}{1.690099in}}%
\pgfpathlineto{\pgfqpoint{0.898147in}{2.276670in}}%
\pgfusepath{stroke}%
\end{pgfscope}%
\begin{pgfscope}%
\pgfpathrectangle{\pgfqpoint{0.636356in}{0.874107in}}{\pgfqpoint{3.222048in}{1.895038in}} %
\pgfusepath{clip}%
\pgfsetbuttcap%
\pgfsetroundjoin%
\pgfsetlinewidth{0.501875pt}%
\definecolor{currentstroke}{rgb}{0.000000,0.000000,0.000000}%
\pgfsetstrokecolor{currentstroke}%
\pgfsetdash{}{0pt}%
\pgfpathmoveto{\pgfqpoint{0.938423in}{1.362674in}}%
\pgfpathlineto{\pgfqpoint{0.938423in}{1.847937in}}%
\pgfusepath{stroke}%
\end{pgfscope}%
\begin{pgfscope}%
\pgfpathrectangle{\pgfqpoint{0.636356in}{0.874107in}}{\pgfqpoint{3.222048in}{1.895038in}} %
\pgfusepath{clip}%
\pgfsetbuttcap%
\pgfsetroundjoin%
\pgfsetlinewidth{0.501875pt}%
\definecolor{currentstroke}{rgb}{0.000000,0.000000,0.000000}%
\pgfsetstrokecolor{currentstroke}%
\pgfsetdash{}{0pt}%
\pgfpathmoveto{\pgfqpoint{0.978698in}{1.362674in}}%
\pgfpathlineto{\pgfqpoint{0.978698in}{1.847937in}}%
\pgfusepath{stroke}%
\end{pgfscope}%
\begin{pgfscope}%
\pgfpathrectangle{\pgfqpoint{0.636356in}{0.874107in}}{\pgfqpoint{3.222048in}{1.895038in}} %
\pgfusepath{clip}%
\pgfsetbuttcap%
\pgfsetroundjoin%
\pgfsetlinewidth{0.501875pt}%
\definecolor{currentstroke}{rgb}{0.000000,0.000000,0.000000}%
\pgfsetstrokecolor{currentstroke}%
\pgfsetdash{}{0pt}%
\pgfpathmoveto{\pgfqpoint{1.018974in}{1.309416in}}%
\pgfpathlineto{\pgfqpoint{1.018974in}{1.775321in}}%
\pgfusepath{stroke}%
\end{pgfscope}%
\begin{pgfscope}%
\pgfpathrectangle{\pgfqpoint{0.636356in}{0.874107in}}{\pgfqpoint{3.222048in}{1.895038in}} %
\pgfusepath{clip}%
\pgfsetbuttcap%
\pgfsetroundjoin%
\pgfsetlinewidth{0.501875pt}%
\definecolor{currentstroke}{rgb}{0.000000,0.000000,0.000000}%
\pgfsetstrokecolor{currentstroke}%
\pgfsetdash{}{0pt}%
\pgfpathmoveto{\pgfqpoint{1.059250in}{1.416387in}}%
\pgfpathlineto{\pgfqpoint{1.059250in}{1.920156in}}%
\pgfusepath{stroke}%
\end{pgfscope}%
\begin{pgfscope}%
\pgfpathrectangle{\pgfqpoint{0.636356in}{0.874107in}}{\pgfqpoint{3.222048in}{1.895038in}} %
\pgfusepath{clip}%
\pgfsetbuttcap%
\pgfsetroundjoin%
\pgfsetlinewidth{0.501875pt}%
\definecolor{currentstroke}{rgb}{0.000000,0.000000,0.000000}%
\pgfsetstrokecolor{currentstroke}%
\pgfsetdash{}{0pt}%
\pgfpathmoveto{\pgfqpoint{1.099525in}{1.053523in}}%
\pgfpathlineto{\pgfqpoint{1.099525in}{1.403610in}}%
\pgfusepath{stroke}%
\end{pgfscope}%
\begin{pgfscope}%
\pgfpathrectangle{\pgfqpoint{0.636356in}{0.874107in}}{\pgfqpoint{3.222048in}{1.895038in}} %
\pgfusepath{clip}%
\pgfsetbuttcap%
\pgfsetroundjoin%
\pgfsetlinewidth{0.501875pt}%
\definecolor{currentstroke}{rgb}{0.000000,0.000000,0.000000}%
\pgfsetstrokecolor{currentstroke}%
\pgfsetdash{}{0pt}%
\pgfpathmoveto{\pgfqpoint{1.139801in}{1.102779in}}%
\pgfpathlineto{\pgfqpoint{1.139801in}{1.479486in}}%
\pgfusepath{stroke}%
\end{pgfscope}%
\begin{pgfscope}%
\pgfpathrectangle{\pgfqpoint{0.636356in}{0.874107in}}{\pgfqpoint{3.222048in}{1.895038in}} %
\pgfusepath{clip}%
\pgfsetbuttcap%
\pgfsetroundjoin%
\pgfsetlinewidth{0.501875pt}%
\definecolor{currentstroke}{rgb}{0.000000,0.000000,0.000000}%
\pgfsetstrokecolor{currentstroke}%
\pgfsetdash{}{0pt}%
\pgfpathmoveto{\pgfqpoint{1.180076in}{0.874107in}}%
\pgfpathlineto{\pgfqpoint{1.180076in}{0.946616in}}%
\pgfusepath{stroke}%
\end{pgfscope}%
\begin{pgfscope}%
\pgfpathrectangle{\pgfqpoint{0.636356in}{0.874107in}}{\pgfqpoint{3.222048in}{1.895038in}} %
\pgfusepath{clip}%
\pgfsetbuttcap%
\pgfsetroundjoin%
\pgfsetlinewidth{0.501875pt}%
\definecolor{currentstroke}{rgb}{0.000000,0.000000,0.000000}%
\pgfsetstrokecolor{currentstroke}%
\pgfsetdash{}{0pt}%
\pgfpathmoveto{\pgfqpoint{1.220352in}{0.874107in}}%
\pgfpathlineto{\pgfqpoint{1.220352in}{0.946616in}}%
\pgfusepath{stroke}%
\end{pgfscope}%
\begin{pgfscope}%
\pgfpathrectangle{\pgfqpoint{0.636356in}{0.874107in}}{\pgfqpoint{3.222048in}{1.895038in}} %
\pgfusepath{clip}%
\pgfsetbuttcap%
\pgfsetroundjoin%
\pgfsetlinewidth{0.501875pt}%
\definecolor{currentstroke}{rgb}{0.000000,0.000000,0.000000}%
\pgfsetstrokecolor{currentstroke}%
\pgfsetdash{}{0pt}%
\pgfpathmoveto{\pgfqpoint{1.260628in}{0.874107in}}%
\pgfpathlineto{\pgfqpoint{1.260628in}{0.946616in}}%
\pgfusepath{stroke}%
\end{pgfscope}%
\begin{pgfscope}%
\pgfpathrectangle{\pgfqpoint{0.636356in}{0.874107in}}{\pgfqpoint{3.222048in}{1.895038in}} %
\pgfusepath{clip}%
\pgfsetbuttcap%
\pgfsetroundjoin%
\pgfsetlinewidth{0.501875pt}%
\definecolor{currentstroke}{rgb}{0.000000,0.000000,0.000000}%
\pgfsetstrokecolor{currentstroke}%
\pgfsetdash{}{0pt}%
\pgfpathmoveto{\pgfqpoint{1.300903in}{0.918842in}}%
\pgfpathlineto{\pgfqpoint{1.300903in}{1.167071in}}%
\pgfusepath{stroke}%
\end{pgfscope}%
\begin{pgfscope}%
\pgfpathrectangle{\pgfqpoint{0.636356in}{0.874107in}}{\pgfqpoint{3.222048in}{1.895038in}} %
\pgfusepath{clip}%
\pgfsetbuttcap%
\pgfsetroundjoin%
\pgfsetlinewidth{0.501875pt}%
\definecolor{currentstroke}{rgb}{0.000000,0.000000,0.000000}%
\pgfsetstrokecolor{currentstroke}%
\pgfsetdash{}{0pt}%
\pgfpathmoveto{\pgfqpoint{1.341179in}{1.204577in}}%
\pgfpathlineto{\pgfqpoint{1.341179in}{1.628657in}}%
\pgfusepath{stroke}%
\end{pgfscope}%
\begin{pgfscope}%
\pgfpathrectangle{\pgfqpoint{0.636356in}{0.874107in}}{\pgfqpoint{3.222048in}{1.895038in}} %
\pgfusepath{clip}%
\pgfsetbuttcap%
\pgfsetroundjoin%
\pgfsetlinewidth{0.501875pt}%
\definecolor{currentstroke}{rgb}{0.000000,0.000000,0.000000}%
\pgfsetstrokecolor{currentstroke}%
\pgfsetdash{}{0pt}%
\pgfpathmoveto{\pgfqpoint{1.381454in}{1.053523in}}%
\pgfpathlineto{\pgfqpoint{1.381454in}{1.403610in}}%
\pgfusepath{stroke}%
\end{pgfscope}%
\begin{pgfscope}%
\pgfpathrectangle{\pgfqpoint{0.636356in}{0.874107in}}{\pgfqpoint{3.222048in}{1.895038in}} %
\pgfusepath{clip}%
\pgfsetbuttcap%
\pgfsetroundjoin%
\pgfsetlinewidth{0.501875pt}%
\definecolor{currentstroke}{rgb}{0.000000,0.000000,0.000000}%
\pgfsetstrokecolor{currentstroke}%
\pgfsetdash{}{0pt}%
\pgfpathmoveto{\pgfqpoint{1.421730in}{0.960476in}}%
\pgfpathlineto{\pgfqpoint{1.421730in}{1.247946in}}%
\pgfusepath{stroke}%
\end{pgfscope}%
\begin{pgfscope}%
\pgfpathrectangle{\pgfqpoint{0.636356in}{0.874107in}}{\pgfqpoint{3.222048in}{1.895038in}} %
\pgfusepath{clip}%
\pgfsetbuttcap%
\pgfsetroundjoin%
\pgfsetlinewidth{0.501875pt}%
\definecolor{currentstroke}{rgb}{0.000000,0.000000,0.000000}%
\pgfsetstrokecolor{currentstroke}%
\pgfsetdash{}{0pt}%
\pgfpathmoveto{\pgfqpoint{1.462006in}{1.053523in}}%
\pgfpathlineto{\pgfqpoint{1.462006in}{1.403610in}}%
\pgfusepath{stroke}%
\end{pgfscope}%
\begin{pgfscope}%
\pgfpathrectangle{\pgfqpoint{0.636356in}{0.874107in}}{\pgfqpoint{3.222048in}{1.895038in}} %
\pgfusepath{clip}%
\pgfsetbuttcap%
\pgfsetroundjoin%
\pgfsetlinewidth{0.501875pt}%
\definecolor{currentstroke}{rgb}{0.000000,0.000000,0.000000}%
\pgfsetstrokecolor{currentstroke}%
\pgfsetdash{}{0pt}%
\pgfpathmoveto{\pgfqpoint{1.502281in}{0.874107in}}%
\pgfpathlineto{\pgfqpoint{1.502281in}{0.946616in}}%
\pgfusepath{stroke}%
\end{pgfscope}%
\begin{pgfscope}%
\pgfpathrectangle{\pgfqpoint{0.636356in}{0.874107in}}{\pgfqpoint{3.222048in}{1.895038in}} %
\pgfusepath{clip}%
\pgfsetbuttcap%
\pgfsetroundjoin%
\pgfsetlinewidth{0.501875pt}%
\definecolor{currentstroke}{rgb}{0.000000,0.000000,0.000000}%
\pgfsetstrokecolor{currentstroke}%
\pgfsetdash{}{0pt}%
\pgfpathmoveto{\pgfqpoint{1.542557in}{0.960476in}}%
\pgfpathlineto{\pgfqpoint{1.542557in}{1.247946in}}%
\pgfusepath{stroke}%
\end{pgfscope}%
\begin{pgfscope}%
\pgfpathrectangle{\pgfqpoint{0.636356in}{0.874107in}}{\pgfqpoint{3.222048in}{1.895038in}} %
\pgfusepath{clip}%
\pgfsetbuttcap%
\pgfsetroundjoin%
\pgfsetlinewidth{0.501875pt}%
\definecolor{currentstroke}{rgb}{0.000000,0.000000,0.000000}%
\pgfsetstrokecolor{currentstroke}%
\pgfsetdash{}{0pt}%
\pgfpathmoveto{\pgfqpoint{1.582832in}{0.918842in}}%
\pgfpathlineto{\pgfqpoint{1.582832in}{1.167071in}}%
\pgfusepath{stroke}%
\end{pgfscope}%
\begin{pgfscope}%
\pgfpathrectangle{\pgfqpoint{0.636356in}{0.874107in}}{\pgfqpoint{3.222048in}{1.895038in}} %
\pgfusepath{clip}%
\pgfsetbuttcap%
\pgfsetroundjoin%
\pgfsetlinewidth{0.501875pt}%
\definecolor{currentstroke}{rgb}{0.000000,0.000000,0.000000}%
\pgfsetstrokecolor{currentstroke}%
\pgfsetdash{}{0pt}%
\pgfpathmoveto{\pgfqpoint{1.623108in}{0.918842in}}%
\pgfpathlineto{\pgfqpoint{1.623108in}{1.167071in}}%
\pgfusepath{stroke}%
\end{pgfscope}%
\begin{pgfscope}%
\pgfpathrectangle{\pgfqpoint{0.636356in}{0.874107in}}{\pgfqpoint{3.222048in}{1.895038in}} %
\pgfusepath{clip}%
\pgfsetbuttcap%
\pgfsetroundjoin%
\pgfsetlinewidth{0.501875pt}%
\definecolor{currentstroke}{rgb}{0.000000,0.000000,0.000000}%
\pgfsetstrokecolor{currentstroke}%
\pgfsetdash{}{0pt}%
\pgfpathmoveto{\pgfqpoint{1.663384in}{1.053523in}}%
\pgfpathlineto{\pgfqpoint{1.663384in}{1.403610in}}%
\pgfusepath{stroke}%
\end{pgfscope}%
\begin{pgfscope}%
\pgfpathrectangle{\pgfqpoint{0.636356in}{0.874107in}}{\pgfqpoint{3.222048in}{1.895038in}} %
\pgfusepath{clip}%
\pgfsetbuttcap%
\pgfsetroundjoin%
\pgfsetlinewidth{0.501875pt}%
\definecolor{currentstroke}{rgb}{0.000000,0.000000,0.000000}%
\pgfsetstrokecolor{currentstroke}%
\pgfsetdash{}{0pt}%
\pgfpathmoveto{\pgfqpoint{1.703659in}{1.005854in}}%
\pgfpathlineto{\pgfqpoint{1.703659in}{1.326563in}}%
\pgfusepath{stroke}%
\end{pgfscope}%
\begin{pgfscope}%
\pgfpathrectangle{\pgfqpoint{0.636356in}{0.874107in}}{\pgfqpoint{3.222048in}{1.895038in}} %
\pgfusepath{clip}%
\pgfsetbuttcap%
\pgfsetroundjoin%
\pgfsetlinewidth{0.501875pt}%
\definecolor{currentstroke}{rgb}{0.000000,0.000000,0.000000}%
\pgfsetstrokecolor{currentstroke}%
\pgfsetdash{}{0pt}%
\pgfpathmoveto{\pgfqpoint{1.743935in}{0.885019in}}%
\pgfpathlineto{\pgfqpoint{1.743935in}{1.082531in}}%
\pgfusepath{stroke}%
\end{pgfscope}%
\begin{pgfscope}%
\pgfpathrectangle{\pgfqpoint{0.636356in}{0.874107in}}{\pgfqpoint{3.222048in}{1.895038in}} %
\pgfusepath{clip}%
\pgfsetbuttcap%
\pgfsetroundjoin%
\pgfsetlinewidth{0.501875pt}%
\definecolor{currentstroke}{rgb}{0.000000,0.000000,0.000000}%
\pgfsetstrokecolor{currentstroke}%
\pgfsetdash{}{0pt}%
\pgfpathmoveto{\pgfqpoint{1.784210in}{1.005854in}}%
\pgfpathlineto{\pgfqpoint{1.784210in}{1.326563in}}%
\pgfusepath{stroke}%
\end{pgfscope}%
\begin{pgfscope}%
\pgfpathrectangle{\pgfqpoint{0.636356in}{0.874107in}}{\pgfqpoint{3.222048in}{1.895038in}} %
\pgfusepath{clip}%
\pgfsetbuttcap%
\pgfsetroundjoin%
\pgfsetlinewidth{0.501875pt}%
\definecolor{currentstroke}{rgb}{0.000000,0.000000,0.000000}%
\pgfsetstrokecolor{currentstroke}%
\pgfsetdash{}{0pt}%
\pgfpathmoveto{\pgfqpoint{1.824486in}{0.885019in}}%
\pgfpathlineto{\pgfqpoint{1.824486in}{1.082531in}}%
\pgfusepath{stroke}%
\end{pgfscope}%
\begin{pgfscope}%
\pgfpathrectangle{\pgfqpoint{0.636356in}{0.874107in}}{\pgfqpoint{3.222048in}{1.895038in}} %
\pgfusepath{clip}%
\pgfsetbuttcap%
\pgfsetroundjoin%
\pgfsetlinewidth{0.501875pt}%
\definecolor{currentstroke}{rgb}{0.000000,0.000000,0.000000}%
\pgfsetstrokecolor{currentstroke}%
\pgfsetdash{}{0pt}%
\pgfpathmoveto{\pgfqpoint{1.864762in}{0.885019in}}%
\pgfpathlineto{\pgfqpoint{1.864762in}{1.082531in}}%
\pgfusepath{stroke}%
\end{pgfscope}%
\begin{pgfscope}%
\pgfpathrectangle{\pgfqpoint{0.636356in}{0.874107in}}{\pgfqpoint{3.222048in}{1.895038in}} %
\pgfusepath{clip}%
\pgfsetbuttcap%
\pgfsetroundjoin%
\pgfsetlinewidth{0.501875pt}%
\definecolor{currentstroke}{rgb}{0.000000,0.000000,0.000000}%
\pgfsetstrokecolor{currentstroke}%
\pgfsetdash{}{0pt}%
\pgfpathmoveto{\pgfqpoint{1.905037in}{0.885019in}}%
\pgfpathlineto{\pgfqpoint{1.905037in}{1.082531in}}%
\pgfusepath{stroke}%
\end{pgfscope}%
\begin{pgfscope}%
\pgfpathrectangle{\pgfqpoint{0.636356in}{0.874107in}}{\pgfqpoint{3.222048in}{1.895038in}} %
\pgfusepath{clip}%
\pgfsetbuttcap%
\pgfsetroundjoin%
\pgfsetlinewidth{0.501875pt}%
\definecolor{currentstroke}{rgb}{0.000000,0.000000,0.000000}%
\pgfsetstrokecolor{currentstroke}%
\pgfsetdash{}{0pt}%
\pgfpathmoveto{\pgfqpoint{1.945313in}{0.918842in}}%
\pgfpathlineto{\pgfqpoint{1.945313in}{1.167071in}}%
\pgfusepath{stroke}%
\end{pgfscope}%
\begin{pgfscope}%
\pgfpathrectangle{\pgfqpoint{0.636356in}{0.874107in}}{\pgfqpoint{3.222048in}{1.895038in}} %
\pgfusepath{clip}%
\pgfsetbuttcap%
\pgfsetroundjoin%
\pgfsetlinewidth{0.501875pt}%
\definecolor{currentstroke}{rgb}{0.000000,0.000000,0.000000}%
\pgfsetstrokecolor{currentstroke}%
\pgfsetdash{}{0pt}%
\pgfpathmoveto{\pgfqpoint{1.985588in}{0.874107in}}%
\pgfpathlineto{\pgfqpoint{1.985588in}{0.946616in}}%
\pgfusepath{stroke}%
\end{pgfscope}%
\begin{pgfscope}%
\pgfpathrectangle{\pgfqpoint{0.636356in}{0.874107in}}{\pgfqpoint{3.222048in}{1.895038in}} %
\pgfusepath{clip}%
\pgfsetbuttcap%
\pgfsetroundjoin%
\pgfsetlinewidth{0.501875pt}%
\definecolor{currentstroke}{rgb}{0.000000,0.000000,0.000000}%
\pgfsetstrokecolor{currentstroke}%
\pgfsetdash{}{0pt}%
\pgfpathmoveto{\pgfqpoint{2.025864in}{0.918842in}}%
\pgfpathlineto{\pgfqpoint{2.025864in}{1.167071in}}%
\pgfusepath{stroke}%
\end{pgfscope}%
\begin{pgfscope}%
\pgfpathrectangle{\pgfqpoint{0.636356in}{0.874107in}}{\pgfqpoint{3.222048in}{1.895038in}} %
\pgfusepath{clip}%
\pgfsetbuttcap%
\pgfsetroundjoin%
\pgfsetlinewidth{0.501875pt}%
\definecolor{currentstroke}{rgb}{0.000000,0.000000,0.000000}%
\pgfsetstrokecolor{currentstroke}%
\pgfsetdash{}{0pt}%
\pgfpathmoveto{\pgfqpoint{2.066139in}{0.885019in}}%
\pgfpathlineto{\pgfqpoint{2.066139in}{1.082531in}}%
\pgfusepath{stroke}%
\end{pgfscope}%
\begin{pgfscope}%
\pgfpathrectangle{\pgfqpoint{0.636356in}{0.874107in}}{\pgfqpoint{3.222048in}{1.895038in}} %
\pgfusepath{clip}%
\pgfsetbuttcap%
\pgfsetroundjoin%
\pgfsetlinewidth{0.501875pt}%
\definecolor{currentstroke}{rgb}{0.000000,0.000000,0.000000}%
\pgfsetstrokecolor{currentstroke}%
\pgfsetdash{}{0pt}%
\pgfpathmoveto{\pgfqpoint{2.106415in}{0.885019in}}%
\pgfpathlineto{\pgfqpoint{2.106415in}{1.082531in}}%
\pgfusepath{stroke}%
\end{pgfscope}%
\begin{pgfscope}%
\pgfpathrectangle{\pgfqpoint{0.636356in}{0.874107in}}{\pgfqpoint{3.222048in}{1.895038in}} %
\pgfusepath{clip}%
\pgfsetbuttcap%
\pgfsetroundjoin%
\pgfsetlinewidth{0.501875pt}%
\definecolor{currentstroke}{rgb}{0.000000,0.000000,0.000000}%
\pgfsetstrokecolor{currentstroke}%
\pgfsetdash{}{0pt}%
\pgfpathmoveto{\pgfqpoint{2.146691in}{0.874107in}}%
\pgfpathlineto{\pgfqpoint{2.146691in}{0.946616in}}%
\pgfusepath{stroke}%
\end{pgfscope}%
\begin{pgfscope}%
\pgfpathrectangle{\pgfqpoint{0.636356in}{0.874107in}}{\pgfqpoint{3.222048in}{1.895038in}} %
\pgfusepath{clip}%
\pgfsetbuttcap%
\pgfsetroundjoin%
\pgfsetlinewidth{0.501875pt}%
\definecolor{currentstroke}{rgb}{0.000000,0.000000,0.000000}%
\pgfsetstrokecolor{currentstroke}%
\pgfsetdash{}{0pt}%
\pgfpathmoveto{\pgfqpoint{2.186966in}{0.885019in}}%
\pgfpathlineto{\pgfqpoint{2.186966in}{1.082531in}}%
\pgfusepath{stroke}%
\end{pgfscope}%
\begin{pgfscope}%
\pgfpathrectangle{\pgfqpoint{0.636356in}{0.874107in}}{\pgfqpoint{3.222048in}{1.895038in}} %
\pgfusepath{clip}%
\pgfsetbuttcap%
\pgfsetroundjoin%
\pgfsetlinewidth{0.501875pt}%
\definecolor{currentstroke}{rgb}{0.000000,0.000000,0.000000}%
\pgfsetstrokecolor{currentstroke}%
\pgfsetdash{}{0pt}%
\pgfpathmoveto{\pgfqpoint{2.227242in}{0.874107in}}%
\pgfpathlineto{\pgfqpoint{2.227242in}{0.946616in}}%
\pgfusepath{stroke}%
\end{pgfscope}%
\begin{pgfscope}%
\pgfpathrectangle{\pgfqpoint{0.636356in}{0.874107in}}{\pgfqpoint{3.222048in}{1.895038in}} %
\pgfusepath{clip}%
\pgfsetbuttcap%
\pgfsetroundjoin%
\pgfsetlinewidth{0.501875pt}%
\definecolor{currentstroke}{rgb}{0.000000,0.000000,0.000000}%
\pgfsetstrokecolor{currentstroke}%
\pgfsetdash{}{0pt}%
\pgfpathmoveto{\pgfqpoint{2.267517in}{0.874107in}}%
\pgfpathlineto{\pgfqpoint{2.267517in}{0.946616in}}%
\pgfusepath{stroke}%
\end{pgfscope}%
\begin{pgfscope}%
\pgfpathrectangle{\pgfqpoint{0.636356in}{0.874107in}}{\pgfqpoint{3.222048in}{1.895038in}} %
\pgfusepath{clip}%
\pgfsetbuttcap%
\pgfsetroundjoin%
\pgfsetlinewidth{0.501875pt}%
\definecolor{currentstroke}{rgb}{0.000000,0.000000,0.000000}%
\pgfsetstrokecolor{currentstroke}%
\pgfsetdash{}{0pt}%
\pgfpathmoveto{\pgfqpoint{2.307793in}{0.874107in}}%
\pgfpathlineto{\pgfqpoint{2.307793in}{0.946616in}}%
\pgfusepath{stroke}%
\end{pgfscope}%
\begin{pgfscope}%
\pgfpathrectangle{\pgfqpoint{0.636356in}{0.874107in}}{\pgfqpoint{3.222048in}{1.895038in}} %
\pgfusepath{clip}%
\pgfsetbuttcap%
\pgfsetroundjoin%
\pgfsetlinewidth{0.501875pt}%
\definecolor{currentstroke}{rgb}{0.000000,0.000000,0.000000}%
\pgfsetstrokecolor{currentstroke}%
\pgfsetdash{}{0pt}%
\pgfpathmoveto{\pgfqpoint{2.348069in}{0.885019in}}%
\pgfpathlineto{\pgfqpoint{2.348069in}{1.082531in}}%
\pgfusepath{stroke}%
\end{pgfscope}%
\begin{pgfscope}%
\pgfpathrectangle{\pgfqpoint{0.636356in}{0.874107in}}{\pgfqpoint{3.222048in}{1.895038in}} %
\pgfusepath{clip}%
\pgfsetbuttcap%
\pgfsetroundjoin%
\pgfsetlinewidth{0.501875pt}%
\definecolor{currentstroke}{rgb}{0.000000,0.000000,0.000000}%
\pgfsetstrokecolor{currentstroke}%
\pgfsetdash{}{0pt}%
\pgfpathmoveto{\pgfqpoint{2.388344in}{0.874107in}}%
\pgfpathlineto{\pgfqpoint{2.388344in}{0.946616in}}%
\pgfusepath{stroke}%
\end{pgfscope}%
\begin{pgfscope}%
\pgfpathrectangle{\pgfqpoint{0.636356in}{0.874107in}}{\pgfqpoint{3.222048in}{1.895038in}} %
\pgfusepath{clip}%
\pgfsetbuttcap%
\pgfsetroundjoin%
\pgfsetlinewidth{0.501875pt}%
\definecolor{currentstroke}{rgb}{0.000000,0.000000,0.000000}%
\pgfsetstrokecolor{currentstroke}%
\pgfsetdash{}{0pt}%
\pgfpathmoveto{\pgfqpoint{2.428620in}{0.874107in}}%
\pgfpathlineto{\pgfqpoint{2.428620in}{0.946616in}}%
\pgfusepath{stroke}%
\end{pgfscope}%
\begin{pgfscope}%
\pgfpathrectangle{\pgfqpoint{0.636356in}{0.874107in}}{\pgfqpoint{3.222048in}{1.895038in}} %
\pgfusepath{clip}%
\pgfsetbuttcap%
\pgfsetroundjoin%
\pgfsetlinewidth{0.501875pt}%
\definecolor{currentstroke}{rgb}{0.000000,0.000000,0.000000}%
\pgfsetstrokecolor{currentstroke}%
\pgfsetdash{}{0pt}%
\pgfpathmoveto{\pgfqpoint{2.468895in}{0.885019in}}%
\pgfpathlineto{\pgfqpoint{2.468895in}{1.082531in}}%
\pgfusepath{stroke}%
\end{pgfscope}%
\begin{pgfscope}%
\pgfpathrectangle{\pgfqpoint{0.636356in}{0.874107in}}{\pgfqpoint{3.222048in}{1.895038in}} %
\pgfusepath{clip}%
\pgfsetbuttcap%
\pgfsetroundjoin%
\pgfsetlinewidth{0.501875pt}%
\definecolor{currentstroke}{rgb}{0.000000,0.000000,0.000000}%
\pgfsetstrokecolor{currentstroke}%
\pgfsetdash{}{0pt}%
\pgfpathmoveto{\pgfqpoint{2.509171in}{0.874107in}}%
\pgfpathlineto{\pgfqpoint{2.509171in}{0.946616in}}%
\pgfusepath{stroke}%
\end{pgfscope}%
\begin{pgfscope}%
\pgfpathrectangle{\pgfqpoint{0.636356in}{0.874107in}}{\pgfqpoint{3.222048in}{1.895038in}} %
\pgfusepath{clip}%
\pgfsetbuttcap%
\pgfsetroundjoin%
\pgfsetlinewidth{0.501875pt}%
\definecolor{currentstroke}{rgb}{0.000000,0.000000,0.000000}%
\pgfsetstrokecolor{currentstroke}%
\pgfsetdash{}{0pt}%
\pgfpathmoveto{\pgfqpoint{2.549447in}{0.885019in}}%
\pgfpathlineto{\pgfqpoint{2.549447in}{1.082531in}}%
\pgfusepath{stroke}%
\end{pgfscope}%
\begin{pgfscope}%
\pgfpathrectangle{\pgfqpoint{0.636356in}{0.874107in}}{\pgfqpoint{3.222048in}{1.895038in}} %
\pgfusepath{clip}%
\pgfsetbuttcap%
\pgfsetroundjoin%
\pgfsetlinewidth{0.501875pt}%
\definecolor{currentstroke}{rgb}{0.000000,0.000000,0.000000}%
\pgfsetstrokecolor{currentstroke}%
\pgfsetdash{}{0pt}%
\pgfpathmoveto{\pgfqpoint{2.589722in}{0.885019in}}%
\pgfpathlineto{\pgfqpoint{2.589722in}{1.082531in}}%
\pgfusepath{stroke}%
\end{pgfscope}%
\begin{pgfscope}%
\pgfpathrectangle{\pgfqpoint{0.636356in}{0.874107in}}{\pgfqpoint{3.222048in}{1.895038in}} %
\pgfusepath{clip}%
\pgfsetbuttcap%
\pgfsetroundjoin%
\pgfsetlinewidth{0.501875pt}%
\definecolor{currentstroke}{rgb}{0.000000,0.000000,0.000000}%
\pgfsetstrokecolor{currentstroke}%
\pgfsetdash{}{0pt}%
\pgfpathmoveto{\pgfqpoint{2.629998in}{0.874107in}}%
\pgfpathlineto{\pgfqpoint{2.629998in}{0.946616in}}%
\pgfusepath{stroke}%
\end{pgfscope}%
\begin{pgfscope}%
\pgfpathrectangle{\pgfqpoint{0.636356in}{0.874107in}}{\pgfqpoint{3.222048in}{1.895038in}} %
\pgfusepath{clip}%
\pgfsetbuttcap%
\pgfsetroundjoin%
\pgfsetlinewidth{0.501875pt}%
\definecolor{currentstroke}{rgb}{0.000000,0.000000,0.000000}%
\pgfsetstrokecolor{currentstroke}%
\pgfsetdash{}{0pt}%
\pgfpathmoveto{\pgfqpoint{2.670273in}{0.874107in}}%
\pgfpathlineto{\pgfqpoint{2.670273in}{0.946616in}}%
\pgfusepath{stroke}%
\end{pgfscope}%
\begin{pgfscope}%
\pgfpathrectangle{\pgfqpoint{0.636356in}{0.874107in}}{\pgfqpoint{3.222048in}{1.895038in}} %
\pgfusepath{clip}%
\pgfsetbuttcap%
\pgfsetroundjoin%
\pgfsetlinewidth{0.501875pt}%
\definecolor{currentstroke}{rgb}{0.000000,0.000000,0.000000}%
\pgfsetstrokecolor{currentstroke}%
\pgfsetdash{}{0pt}%
\pgfpathmoveto{\pgfqpoint{2.710549in}{0.874107in}}%
\pgfpathlineto{\pgfqpoint{2.710549in}{0.946616in}}%
\pgfusepath{stroke}%
\end{pgfscope}%
\begin{pgfscope}%
\pgfpathrectangle{\pgfqpoint{0.636356in}{0.874107in}}{\pgfqpoint{3.222048in}{1.895038in}} %
\pgfusepath{clip}%
\pgfsetbuttcap%
\pgfsetroundjoin%
\pgfsetlinewidth{0.501875pt}%
\definecolor{currentstroke}{rgb}{0.000000,0.000000,0.000000}%
\pgfsetstrokecolor{currentstroke}%
\pgfsetdash{}{0pt}%
\pgfpathmoveto{\pgfqpoint{2.750825in}{0.918842in}}%
\pgfpathlineto{\pgfqpoint{2.750825in}{1.167071in}}%
\pgfusepath{stroke}%
\end{pgfscope}%
\begin{pgfscope}%
\pgfpathrectangle{\pgfqpoint{0.636356in}{0.874107in}}{\pgfqpoint{3.222048in}{1.895038in}} %
\pgfusepath{clip}%
\pgfsetbuttcap%
\pgfsetroundjoin%
\pgfsetlinewidth{0.501875pt}%
\definecolor{currentstroke}{rgb}{0.000000,0.000000,0.000000}%
\pgfsetstrokecolor{currentstroke}%
\pgfsetdash{}{0pt}%
\pgfpathmoveto{\pgfqpoint{2.791100in}{0.874107in}}%
\pgfpathlineto{\pgfqpoint{2.791100in}{0.946616in}}%
\pgfusepath{stroke}%
\end{pgfscope}%
\begin{pgfscope}%
\pgfpathrectangle{\pgfqpoint{0.636356in}{0.874107in}}{\pgfqpoint{3.222048in}{1.895038in}} %
\pgfusepath{clip}%
\pgfsetbuttcap%
\pgfsetroundjoin%
\pgfsetlinewidth{0.501875pt}%
\definecolor{currentstroke}{rgb}{0.000000,0.000000,0.000000}%
\pgfsetstrokecolor{currentstroke}%
\pgfsetdash{}{0pt}%
\pgfpathmoveto{\pgfqpoint{2.831376in}{0.885019in}}%
\pgfpathlineto{\pgfqpoint{2.831376in}{1.082531in}}%
\pgfusepath{stroke}%
\end{pgfscope}%
\begin{pgfscope}%
\pgfpathrectangle{\pgfqpoint{0.636356in}{0.874107in}}{\pgfqpoint{3.222048in}{1.895038in}} %
\pgfusepath{clip}%
\pgfsetbuttcap%
\pgfsetroundjoin%
\pgfsetlinewidth{0.501875pt}%
\definecolor{currentstroke}{rgb}{0.000000,0.000000,0.000000}%
\pgfsetstrokecolor{currentstroke}%
\pgfsetdash{}{0pt}%
\pgfpathmoveto{\pgfqpoint{2.871651in}{0.874107in}}%
\pgfpathlineto{\pgfqpoint{2.871651in}{0.946616in}}%
\pgfusepath{stroke}%
\end{pgfscope}%
\begin{pgfscope}%
\pgfpathrectangle{\pgfqpoint{0.636356in}{0.874107in}}{\pgfqpoint{3.222048in}{1.895038in}} %
\pgfusepath{clip}%
\pgfsetbuttcap%
\pgfsetroundjoin%
\pgfsetlinewidth{0.501875pt}%
\definecolor{currentstroke}{rgb}{0.000000,0.000000,0.000000}%
\pgfsetstrokecolor{currentstroke}%
\pgfsetdash{}{0pt}%
\pgfpathmoveto{\pgfqpoint{2.911927in}{0.874107in}}%
\pgfpathlineto{\pgfqpoint{2.911927in}{0.946616in}}%
\pgfusepath{stroke}%
\end{pgfscope}%
\begin{pgfscope}%
\pgfpathrectangle{\pgfqpoint{0.636356in}{0.874107in}}{\pgfqpoint{3.222048in}{1.895038in}} %
\pgfusepath{clip}%
\pgfsetbuttcap%
\pgfsetroundjoin%
\pgfsetlinewidth{0.501875pt}%
\definecolor{currentstroke}{rgb}{0.000000,0.000000,0.000000}%
\pgfsetstrokecolor{currentstroke}%
\pgfsetdash{}{0pt}%
\pgfpathmoveto{\pgfqpoint{2.952203in}{0.874107in}}%
\pgfpathlineto{\pgfqpoint{2.952203in}{0.946616in}}%
\pgfusepath{stroke}%
\end{pgfscope}%
\begin{pgfscope}%
\pgfpathrectangle{\pgfqpoint{0.636356in}{0.874107in}}{\pgfqpoint{3.222048in}{1.895038in}} %
\pgfusepath{clip}%
\pgfsetbuttcap%
\pgfsetroundjoin%
\pgfsetlinewidth{0.501875pt}%
\definecolor{currentstroke}{rgb}{0.000000,0.000000,0.000000}%
\pgfsetstrokecolor{currentstroke}%
\pgfsetdash{}{0pt}%
\pgfpathmoveto{\pgfqpoint{2.992478in}{0.874107in}}%
\pgfpathlineto{\pgfqpoint{2.992478in}{0.946616in}}%
\pgfusepath{stroke}%
\end{pgfscope}%
\begin{pgfscope}%
\pgfpathrectangle{\pgfqpoint{0.636356in}{0.874107in}}{\pgfqpoint{3.222048in}{1.895038in}} %
\pgfusepath{clip}%
\pgfsetbuttcap%
\pgfsetroundjoin%
\pgfsetlinewidth{0.501875pt}%
\definecolor{currentstroke}{rgb}{0.000000,0.000000,0.000000}%
\pgfsetstrokecolor{currentstroke}%
\pgfsetdash{}{0pt}%
\pgfpathmoveto{\pgfqpoint{3.032754in}{0.874107in}}%
\pgfpathlineto{\pgfqpoint{3.032754in}{0.946616in}}%
\pgfusepath{stroke}%
\end{pgfscope}%
\begin{pgfscope}%
\pgfpathrectangle{\pgfqpoint{0.636356in}{0.874107in}}{\pgfqpoint{3.222048in}{1.895038in}} %
\pgfusepath{clip}%
\pgfsetbuttcap%
\pgfsetroundjoin%
\pgfsetlinewidth{0.501875pt}%
\definecolor{currentstroke}{rgb}{0.000000,0.000000,0.000000}%
\pgfsetstrokecolor{currentstroke}%
\pgfsetdash{}{0pt}%
\pgfpathmoveto{\pgfqpoint{3.073029in}{0.874107in}}%
\pgfpathlineto{\pgfqpoint{3.073029in}{0.946616in}}%
\pgfusepath{stroke}%
\end{pgfscope}%
\begin{pgfscope}%
\pgfpathrectangle{\pgfqpoint{0.636356in}{0.874107in}}{\pgfqpoint{3.222048in}{1.895038in}} %
\pgfusepath{clip}%
\pgfsetbuttcap%
\pgfsetroundjoin%
\pgfsetlinewidth{0.501875pt}%
\definecolor{currentstroke}{rgb}{0.000000,0.000000,0.000000}%
\pgfsetstrokecolor{currentstroke}%
\pgfsetdash{}{0pt}%
\pgfpathmoveto{\pgfqpoint{3.113305in}{0.874107in}}%
\pgfpathlineto{\pgfqpoint{3.113305in}{0.946616in}}%
\pgfusepath{stroke}%
\end{pgfscope}%
\begin{pgfscope}%
\pgfpathrectangle{\pgfqpoint{0.636356in}{0.874107in}}{\pgfqpoint{3.222048in}{1.895038in}} %
\pgfusepath{clip}%
\pgfsetbuttcap%
\pgfsetroundjoin%
\pgfsetlinewidth{0.501875pt}%
\definecolor{currentstroke}{rgb}{0.000000,0.000000,0.000000}%
\pgfsetstrokecolor{currentstroke}%
\pgfsetdash{}{0pt}%
\pgfpathmoveto{\pgfqpoint{3.153581in}{0.874107in}}%
\pgfpathlineto{\pgfqpoint{3.153581in}{0.946616in}}%
\pgfusepath{stroke}%
\end{pgfscope}%
\begin{pgfscope}%
\pgfpathrectangle{\pgfqpoint{0.636356in}{0.874107in}}{\pgfqpoint{3.222048in}{1.895038in}} %
\pgfusepath{clip}%
\pgfsetbuttcap%
\pgfsetroundjoin%
\pgfsetlinewidth{0.501875pt}%
\definecolor{currentstroke}{rgb}{0.000000,0.000000,0.000000}%
\pgfsetstrokecolor{currentstroke}%
\pgfsetdash{}{0pt}%
\pgfpathmoveto{\pgfqpoint{3.193856in}{0.874107in}}%
\pgfpathlineto{\pgfqpoint{3.193856in}{0.946616in}}%
\pgfusepath{stroke}%
\end{pgfscope}%
\begin{pgfscope}%
\pgfpathrectangle{\pgfqpoint{0.636356in}{0.874107in}}{\pgfqpoint{3.222048in}{1.895038in}} %
\pgfusepath{clip}%
\pgfsetbuttcap%
\pgfsetroundjoin%
\pgfsetlinewidth{0.501875pt}%
\definecolor{currentstroke}{rgb}{0.000000,0.000000,0.000000}%
\pgfsetstrokecolor{currentstroke}%
\pgfsetdash{}{0pt}%
\pgfpathmoveto{\pgfqpoint{3.234132in}{0.874107in}}%
\pgfpathlineto{\pgfqpoint{3.234132in}{0.946616in}}%
\pgfusepath{stroke}%
\end{pgfscope}%
\begin{pgfscope}%
\pgfpathrectangle{\pgfqpoint{0.636356in}{0.874107in}}{\pgfqpoint{3.222048in}{1.895038in}} %
\pgfusepath{clip}%
\pgfsetbuttcap%
\pgfsetroundjoin%
\pgfsetlinewidth{0.501875pt}%
\definecolor{currentstroke}{rgb}{0.000000,0.000000,0.000000}%
\pgfsetstrokecolor{currentstroke}%
\pgfsetdash{}{0pt}%
\pgfpathmoveto{\pgfqpoint{3.274407in}{0.874107in}}%
\pgfpathlineto{\pgfqpoint{3.274407in}{0.946616in}}%
\pgfusepath{stroke}%
\end{pgfscope}%
\begin{pgfscope}%
\pgfpathrectangle{\pgfqpoint{0.636356in}{0.874107in}}{\pgfqpoint{3.222048in}{1.895038in}} %
\pgfusepath{clip}%
\pgfsetbuttcap%
\pgfsetroundjoin%
\pgfsetlinewidth{0.501875pt}%
\definecolor{currentstroke}{rgb}{0.000000,0.000000,0.000000}%
\pgfsetstrokecolor{currentstroke}%
\pgfsetdash{}{0pt}%
\pgfpathmoveto{\pgfqpoint{3.314683in}{0.874107in}}%
\pgfpathlineto{\pgfqpoint{3.314683in}{0.946616in}}%
\pgfusepath{stroke}%
\end{pgfscope}%
\begin{pgfscope}%
\pgfpathrectangle{\pgfqpoint{0.636356in}{0.874107in}}{\pgfqpoint{3.222048in}{1.895038in}} %
\pgfusepath{clip}%
\pgfsetbuttcap%
\pgfsetroundjoin%
\pgfsetlinewidth{0.501875pt}%
\definecolor{currentstroke}{rgb}{0.000000,0.000000,0.000000}%
\pgfsetstrokecolor{currentstroke}%
\pgfsetdash{}{0pt}%
\pgfpathmoveto{\pgfqpoint{3.354959in}{0.874107in}}%
\pgfpathlineto{\pgfqpoint{3.354959in}{0.946616in}}%
\pgfusepath{stroke}%
\end{pgfscope}%
\begin{pgfscope}%
\pgfpathrectangle{\pgfqpoint{0.636356in}{0.874107in}}{\pgfqpoint{3.222048in}{1.895038in}} %
\pgfusepath{clip}%
\pgfsetbuttcap%
\pgfsetroundjoin%
\pgfsetlinewidth{0.501875pt}%
\definecolor{currentstroke}{rgb}{0.000000,0.000000,0.000000}%
\pgfsetstrokecolor{currentstroke}%
\pgfsetdash{}{0pt}%
\pgfpathmoveto{\pgfqpoint{3.395234in}{0.874107in}}%
\pgfpathlineto{\pgfqpoint{3.395234in}{0.946616in}}%
\pgfusepath{stroke}%
\end{pgfscope}%
\begin{pgfscope}%
\pgfpathrectangle{\pgfqpoint{0.636356in}{0.874107in}}{\pgfqpoint{3.222048in}{1.895038in}} %
\pgfusepath{clip}%
\pgfsetbuttcap%
\pgfsetroundjoin%
\pgfsetlinewidth{0.501875pt}%
\definecolor{currentstroke}{rgb}{0.000000,0.000000,0.000000}%
\pgfsetstrokecolor{currentstroke}%
\pgfsetdash{}{0pt}%
\pgfpathmoveto{\pgfqpoint{3.435510in}{0.874107in}}%
\pgfpathlineto{\pgfqpoint{3.435510in}{0.946616in}}%
\pgfusepath{stroke}%
\end{pgfscope}%
\begin{pgfscope}%
\pgfpathrectangle{\pgfqpoint{0.636356in}{0.874107in}}{\pgfqpoint{3.222048in}{1.895038in}} %
\pgfusepath{clip}%
\pgfsetbuttcap%
\pgfsetroundjoin%
\pgfsetlinewidth{0.501875pt}%
\definecolor{currentstroke}{rgb}{0.000000,0.000000,0.000000}%
\pgfsetstrokecolor{currentstroke}%
\pgfsetdash{}{0pt}%
\pgfpathmoveto{\pgfqpoint{3.475785in}{0.874107in}}%
\pgfpathlineto{\pgfqpoint{3.475785in}{0.946616in}}%
\pgfusepath{stroke}%
\end{pgfscope}%
\begin{pgfscope}%
\pgfpathrectangle{\pgfqpoint{0.636356in}{0.874107in}}{\pgfqpoint{3.222048in}{1.895038in}} %
\pgfusepath{clip}%
\pgfsetbuttcap%
\pgfsetroundjoin%
\pgfsetlinewidth{0.501875pt}%
\definecolor{currentstroke}{rgb}{0.000000,0.000000,0.000000}%
\pgfsetstrokecolor{currentstroke}%
\pgfsetdash{}{0pt}%
\pgfpathmoveto{\pgfqpoint{3.516061in}{0.874107in}}%
\pgfpathlineto{\pgfqpoint{3.516061in}{0.946616in}}%
\pgfusepath{stroke}%
\end{pgfscope}%
\begin{pgfscope}%
\pgfpathrectangle{\pgfqpoint{0.636356in}{0.874107in}}{\pgfqpoint{3.222048in}{1.895038in}} %
\pgfusepath{clip}%
\pgfsetbuttcap%
\pgfsetroundjoin%
\pgfsetlinewidth{0.501875pt}%
\definecolor{currentstroke}{rgb}{0.000000,0.000000,0.000000}%
\pgfsetstrokecolor{currentstroke}%
\pgfsetdash{}{0pt}%
\pgfpathmoveto{\pgfqpoint{3.556337in}{0.874107in}}%
\pgfpathlineto{\pgfqpoint{3.556337in}{0.946616in}}%
\pgfusepath{stroke}%
\end{pgfscope}%
\begin{pgfscope}%
\pgfpathrectangle{\pgfqpoint{0.636356in}{0.874107in}}{\pgfqpoint{3.222048in}{1.895038in}} %
\pgfusepath{clip}%
\pgfsetbuttcap%
\pgfsetroundjoin%
\pgfsetlinewidth{0.501875pt}%
\definecolor{currentstroke}{rgb}{0.000000,0.000000,0.000000}%
\pgfsetstrokecolor{currentstroke}%
\pgfsetdash{}{0pt}%
\pgfpathmoveto{\pgfqpoint{3.596612in}{0.874107in}}%
\pgfpathlineto{\pgfqpoint{3.596612in}{0.946616in}}%
\pgfusepath{stroke}%
\end{pgfscope}%
\begin{pgfscope}%
\pgfpathrectangle{\pgfqpoint{0.636356in}{0.874107in}}{\pgfqpoint{3.222048in}{1.895038in}} %
\pgfusepath{clip}%
\pgfsetbuttcap%
\pgfsetroundjoin%
\pgfsetlinewidth{0.501875pt}%
\definecolor{currentstroke}{rgb}{0.000000,0.000000,0.000000}%
\pgfsetstrokecolor{currentstroke}%
\pgfsetdash{}{0pt}%
\pgfpathmoveto{\pgfqpoint{3.636888in}{0.874107in}}%
\pgfpathlineto{\pgfqpoint{3.636888in}{0.946616in}}%
\pgfusepath{stroke}%
\end{pgfscope}%
\begin{pgfscope}%
\pgfpathrectangle{\pgfqpoint{0.636356in}{0.874107in}}{\pgfqpoint{3.222048in}{1.895038in}} %
\pgfusepath{clip}%
\pgfsetbuttcap%
\pgfsetroundjoin%
\pgfsetlinewidth{0.501875pt}%
\definecolor{currentstroke}{rgb}{0.000000,0.000000,0.000000}%
\pgfsetstrokecolor{currentstroke}%
\pgfsetdash{}{0pt}%
\pgfpathmoveto{\pgfqpoint{3.677163in}{0.874107in}}%
\pgfpathlineto{\pgfqpoint{3.677163in}{0.946616in}}%
\pgfusepath{stroke}%
\end{pgfscope}%
\begin{pgfscope}%
\pgfpathrectangle{\pgfqpoint{0.636356in}{0.874107in}}{\pgfqpoint{3.222048in}{1.895038in}} %
\pgfusepath{clip}%
\pgfsetbuttcap%
\pgfsetroundjoin%
\pgfsetlinewidth{0.501875pt}%
\definecolor{currentstroke}{rgb}{0.000000,0.000000,0.000000}%
\pgfsetstrokecolor{currentstroke}%
\pgfsetdash{}{0pt}%
\pgfpathmoveto{\pgfqpoint{3.717439in}{0.874107in}}%
\pgfpathlineto{\pgfqpoint{3.717439in}{0.946616in}}%
\pgfusepath{stroke}%
\end{pgfscope}%
\begin{pgfscope}%
\pgfpathrectangle{\pgfqpoint{0.636356in}{0.874107in}}{\pgfqpoint{3.222048in}{1.895038in}} %
\pgfusepath{clip}%
\pgfsetbuttcap%
\pgfsetroundjoin%
\pgfsetlinewidth{0.501875pt}%
\definecolor{currentstroke}{rgb}{0.000000,0.000000,0.000000}%
\pgfsetstrokecolor{currentstroke}%
\pgfsetdash{}{0pt}%
\pgfpathmoveto{\pgfqpoint{3.757715in}{0.874107in}}%
\pgfpathlineto{\pgfqpoint{3.757715in}{0.946616in}}%
\pgfusepath{stroke}%
\end{pgfscope}%
\begin{pgfscope}%
\pgfpathrectangle{\pgfqpoint{0.636356in}{0.874107in}}{\pgfqpoint{3.222048in}{1.895038in}} %
\pgfusepath{clip}%
\pgfsetbuttcap%
\pgfsetroundjoin%
\pgfsetlinewidth{0.501875pt}%
\definecolor{currentstroke}{rgb}{0.000000,0.000000,0.000000}%
\pgfsetstrokecolor{currentstroke}%
\pgfsetdash{}{0pt}%
\pgfpathmoveto{\pgfqpoint{3.797990in}{0.874107in}}%
\pgfpathlineto{\pgfqpoint{3.797990in}{0.946616in}}%
\pgfusepath{stroke}%
\end{pgfscope}%
\begin{pgfscope}%
\pgfpathrectangle{\pgfqpoint{0.636356in}{0.874107in}}{\pgfqpoint{3.222048in}{1.895038in}} %
\pgfusepath{clip}%
\pgfsetbuttcap%
\pgfsetroundjoin%
\pgfsetlinewidth{0.501875pt}%
\definecolor{currentstroke}{rgb}{0.000000,0.000000,0.000000}%
\pgfsetstrokecolor{currentstroke}%
\pgfsetdash{}{0pt}%
\pgfpathmoveto{\pgfqpoint{3.838266in}{0.874107in}}%
\pgfpathlineto{\pgfqpoint{3.838266in}{0.946616in}}%
\pgfusepath{stroke}%
\end{pgfscope}%
\begin{pgfscope}%
\pgfsetbuttcap%
\pgfsetroundjoin%
\definecolor{currentfill}{rgb}{0.000000,0.000000,0.000000}%
\pgfsetfillcolor{currentfill}%
\pgfsetlinewidth{1.003750pt}%
\definecolor{currentstroke}{rgb}{0.000000,0.000000,0.000000}%
\pgfsetstrokecolor{currentstroke}%
\pgfsetdash{}{0pt}%
\pgfsys@defobject{currentmarker}{\pgfqpoint{-0.006944in}{-0.006944in}}{\pgfqpoint{0.006944in}{0.006944in}}{%
\pgfpathmoveto{\pgfqpoint{0.000000in}{-0.006944in}}%
\pgfpathcurveto{\pgfqpoint{0.001842in}{-0.006944in}}{\pgfqpoint{0.003608in}{-0.006213in}}{\pgfqpoint{0.004910in}{-0.004910in}}%
\pgfpathcurveto{\pgfqpoint{0.006213in}{-0.003608in}}{\pgfqpoint{0.006944in}{-0.001842in}}{\pgfqpoint{0.006944in}{0.000000in}}%
\pgfpathcurveto{\pgfqpoint{0.006944in}{0.001842in}}{\pgfqpoint{0.006213in}{0.003608in}}{\pgfqpoint{0.004910in}{0.004910in}}%
\pgfpathcurveto{\pgfqpoint{0.003608in}{0.006213in}}{\pgfqpoint{0.001842in}{0.006944in}}{\pgfqpoint{0.000000in}{0.006944in}}%
\pgfpathcurveto{\pgfqpoint{-0.001842in}{0.006944in}}{\pgfqpoint{-0.003608in}{0.006213in}}{\pgfqpoint{-0.004910in}{0.004910in}}%
\pgfpathcurveto{\pgfqpoint{-0.006213in}{0.003608in}}{\pgfqpoint{-0.006944in}{0.001842in}}{\pgfqpoint{-0.006944in}{0.000000in}}%
\pgfpathcurveto{\pgfqpoint{-0.006944in}{-0.001842in}}{\pgfqpoint{-0.006213in}{-0.003608in}}{\pgfqpoint{-0.004910in}{-0.004910in}}%
\pgfpathcurveto{\pgfqpoint{-0.003608in}{-0.006213in}}{\pgfqpoint{-0.001842in}{-0.006944in}}{\pgfqpoint{0.000000in}{-0.006944in}}%
\pgfpathclose%
\pgfusepath{stroke,fill}%
}%
\begin{pgfscope}%
\pgfsys@transformshift{0.656494in}{2.200633in}%
\pgfsys@useobject{currentmarker}{}%
\end{pgfscope}%
\begin{pgfscope}%
\pgfsys@transformshift{0.696769in}{2.326969in}%
\pgfsys@useobject{currentmarker}{}%
\end{pgfscope}%
\begin{pgfscope}%
\pgfsys@transformshift{0.737045in}{1.695290in}%
\pgfsys@useobject{currentmarker}{}%
\end{pgfscope}%
\begin{pgfscope}%
\pgfsys@transformshift{0.777320in}{1.758458in}%
\pgfsys@useobject{currentmarker}{}%
\end{pgfscope}%
\begin{pgfscope}%
\pgfsys@transformshift{0.817596in}{2.200633in}%
\pgfsys@useobject{currentmarker}{}%
\end{pgfscope}%
\begin{pgfscope}%
\pgfsys@transformshift{0.857872in}{2.074298in}%
\pgfsys@useobject{currentmarker}{}%
\end{pgfscope}%
\begin{pgfscope}%
\pgfsys@transformshift{0.898147in}{1.947962in}%
\pgfsys@useobject{currentmarker}{}%
\end{pgfscope}%
\begin{pgfscope}%
\pgfsys@transformshift{0.938423in}{1.568954in}%
\pgfsys@useobject{currentmarker}{}%
\end{pgfscope}%
\begin{pgfscope}%
\pgfsys@transformshift{0.978698in}{1.568954in}%
\pgfsys@useobject{currentmarker}{}%
\end{pgfscope}%
\begin{pgfscope}%
\pgfsys@transformshift{1.018974in}{1.505786in}%
\pgfsys@useobject{currentmarker}{}%
\end{pgfscope}%
\begin{pgfscope}%
\pgfsys@transformshift{1.059250in}{1.632122in}%
\pgfsys@useobject{currentmarker}{}%
\end{pgfscope}%
\begin{pgfscope}%
\pgfsys@transformshift{1.099525in}{1.189947in}%
\pgfsys@useobject{currentmarker}{}%
\end{pgfscope}%
\begin{pgfscope}%
\pgfsys@transformshift{1.139801in}{1.253114in}%
\pgfsys@useobject{currentmarker}{}%
\end{pgfscope}%
\begin{pgfscope}%
\pgfsys@transformshift{1.180076in}{0.874107in}%
\pgfsys@useobject{currentmarker}{}%
\end{pgfscope}%
\begin{pgfscope}%
\pgfsys@transformshift{1.220352in}{0.874107in}%
\pgfsys@useobject{currentmarker}{}%
\end{pgfscope}%
\begin{pgfscope}%
\pgfsys@transformshift{1.260628in}{0.874107in}%
\pgfsys@useobject{currentmarker}{}%
\end{pgfscope}%
\begin{pgfscope}%
\pgfsys@transformshift{1.300903in}{1.000443in}%
\pgfsys@useobject{currentmarker}{}%
\end{pgfscope}%
\begin{pgfscope}%
\pgfsys@transformshift{1.341179in}{1.379450in}%
\pgfsys@useobject{currentmarker}{}%
\end{pgfscope}%
\begin{pgfscope}%
\pgfsys@transformshift{1.381454in}{1.189947in}%
\pgfsys@useobject{currentmarker}{}%
\end{pgfscope}%
\begin{pgfscope}%
\pgfsys@transformshift{1.421730in}{1.063611in}%
\pgfsys@useobject{currentmarker}{}%
\end{pgfscope}%
\begin{pgfscope}%
\pgfsys@transformshift{1.462006in}{1.189947in}%
\pgfsys@useobject{currentmarker}{}%
\end{pgfscope}%
\begin{pgfscope}%
\pgfsys@transformshift{1.502281in}{0.874107in}%
\pgfsys@useobject{currentmarker}{}%
\end{pgfscope}%
\begin{pgfscope}%
\pgfsys@transformshift{1.542557in}{1.063611in}%
\pgfsys@useobject{currentmarker}{}%
\end{pgfscope}%
\begin{pgfscope}%
\pgfsys@transformshift{1.582832in}{1.000443in}%
\pgfsys@useobject{currentmarker}{}%
\end{pgfscope}%
\begin{pgfscope}%
\pgfsys@transformshift{1.623108in}{1.000443in}%
\pgfsys@useobject{currentmarker}{}%
\end{pgfscope}%
\begin{pgfscope}%
\pgfsys@transformshift{1.663384in}{1.189947in}%
\pgfsys@useobject{currentmarker}{}%
\end{pgfscope}%
\begin{pgfscope}%
\pgfsys@transformshift{1.703659in}{1.126779in}%
\pgfsys@useobject{currentmarker}{}%
\end{pgfscope}%
\begin{pgfscope}%
\pgfsys@transformshift{1.743935in}{0.937275in}%
\pgfsys@useobject{currentmarker}{}%
\end{pgfscope}%
\begin{pgfscope}%
\pgfsys@transformshift{1.784210in}{1.126779in}%
\pgfsys@useobject{currentmarker}{}%
\end{pgfscope}%
\begin{pgfscope}%
\pgfsys@transformshift{1.824486in}{0.937275in}%
\pgfsys@useobject{currentmarker}{}%
\end{pgfscope}%
\begin{pgfscope}%
\pgfsys@transformshift{1.864762in}{0.937275in}%
\pgfsys@useobject{currentmarker}{}%
\end{pgfscope}%
\begin{pgfscope}%
\pgfsys@transformshift{1.905037in}{0.937275in}%
\pgfsys@useobject{currentmarker}{}%
\end{pgfscope}%
\begin{pgfscope}%
\pgfsys@transformshift{1.945313in}{1.000443in}%
\pgfsys@useobject{currentmarker}{}%
\end{pgfscope}%
\begin{pgfscope}%
\pgfsys@transformshift{1.985588in}{0.874107in}%
\pgfsys@useobject{currentmarker}{}%
\end{pgfscope}%
\begin{pgfscope}%
\pgfsys@transformshift{2.025864in}{1.000443in}%
\pgfsys@useobject{currentmarker}{}%
\end{pgfscope}%
\begin{pgfscope}%
\pgfsys@transformshift{2.066139in}{0.937275in}%
\pgfsys@useobject{currentmarker}{}%
\end{pgfscope}%
\begin{pgfscope}%
\pgfsys@transformshift{2.106415in}{0.937275in}%
\pgfsys@useobject{currentmarker}{}%
\end{pgfscope}%
\begin{pgfscope}%
\pgfsys@transformshift{2.146691in}{0.874107in}%
\pgfsys@useobject{currentmarker}{}%
\end{pgfscope}%
\begin{pgfscope}%
\pgfsys@transformshift{2.186966in}{0.937275in}%
\pgfsys@useobject{currentmarker}{}%
\end{pgfscope}%
\begin{pgfscope}%
\pgfsys@transformshift{2.227242in}{0.874107in}%
\pgfsys@useobject{currentmarker}{}%
\end{pgfscope}%
\begin{pgfscope}%
\pgfsys@transformshift{2.267517in}{0.874107in}%
\pgfsys@useobject{currentmarker}{}%
\end{pgfscope}%
\begin{pgfscope}%
\pgfsys@transformshift{2.307793in}{0.874107in}%
\pgfsys@useobject{currentmarker}{}%
\end{pgfscope}%
\begin{pgfscope}%
\pgfsys@transformshift{2.348069in}{0.937275in}%
\pgfsys@useobject{currentmarker}{}%
\end{pgfscope}%
\begin{pgfscope}%
\pgfsys@transformshift{2.388344in}{0.874107in}%
\pgfsys@useobject{currentmarker}{}%
\end{pgfscope}%
\begin{pgfscope}%
\pgfsys@transformshift{2.428620in}{0.874107in}%
\pgfsys@useobject{currentmarker}{}%
\end{pgfscope}%
\begin{pgfscope}%
\pgfsys@transformshift{2.468895in}{0.937275in}%
\pgfsys@useobject{currentmarker}{}%
\end{pgfscope}%
\begin{pgfscope}%
\pgfsys@transformshift{2.509171in}{0.874107in}%
\pgfsys@useobject{currentmarker}{}%
\end{pgfscope}%
\begin{pgfscope}%
\pgfsys@transformshift{2.549447in}{0.937275in}%
\pgfsys@useobject{currentmarker}{}%
\end{pgfscope}%
\begin{pgfscope}%
\pgfsys@transformshift{2.589722in}{0.937275in}%
\pgfsys@useobject{currentmarker}{}%
\end{pgfscope}%
\begin{pgfscope}%
\pgfsys@transformshift{2.629998in}{0.874107in}%
\pgfsys@useobject{currentmarker}{}%
\end{pgfscope}%
\begin{pgfscope}%
\pgfsys@transformshift{2.670273in}{0.874107in}%
\pgfsys@useobject{currentmarker}{}%
\end{pgfscope}%
\begin{pgfscope}%
\pgfsys@transformshift{2.710549in}{0.874107in}%
\pgfsys@useobject{currentmarker}{}%
\end{pgfscope}%
\begin{pgfscope}%
\pgfsys@transformshift{2.750825in}{1.000443in}%
\pgfsys@useobject{currentmarker}{}%
\end{pgfscope}%
\begin{pgfscope}%
\pgfsys@transformshift{2.791100in}{0.874107in}%
\pgfsys@useobject{currentmarker}{}%
\end{pgfscope}%
\begin{pgfscope}%
\pgfsys@transformshift{2.831376in}{0.937275in}%
\pgfsys@useobject{currentmarker}{}%
\end{pgfscope}%
\begin{pgfscope}%
\pgfsys@transformshift{2.871651in}{0.874107in}%
\pgfsys@useobject{currentmarker}{}%
\end{pgfscope}%
\begin{pgfscope}%
\pgfsys@transformshift{2.911927in}{0.874107in}%
\pgfsys@useobject{currentmarker}{}%
\end{pgfscope}%
\begin{pgfscope}%
\pgfsys@transformshift{2.952203in}{0.874107in}%
\pgfsys@useobject{currentmarker}{}%
\end{pgfscope}%
\begin{pgfscope}%
\pgfsys@transformshift{2.992478in}{0.874107in}%
\pgfsys@useobject{currentmarker}{}%
\end{pgfscope}%
\begin{pgfscope}%
\pgfsys@transformshift{3.032754in}{0.874107in}%
\pgfsys@useobject{currentmarker}{}%
\end{pgfscope}%
\begin{pgfscope}%
\pgfsys@transformshift{3.073029in}{0.874107in}%
\pgfsys@useobject{currentmarker}{}%
\end{pgfscope}%
\begin{pgfscope}%
\pgfsys@transformshift{3.113305in}{0.874107in}%
\pgfsys@useobject{currentmarker}{}%
\end{pgfscope}%
\begin{pgfscope}%
\pgfsys@transformshift{3.153581in}{0.874107in}%
\pgfsys@useobject{currentmarker}{}%
\end{pgfscope}%
\begin{pgfscope}%
\pgfsys@transformshift{3.193856in}{0.874107in}%
\pgfsys@useobject{currentmarker}{}%
\end{pgfscope}%
\begin{pgfscope}%
\pgfsys@transformshift{3.234132in}{0.874107in}%
\pgfsys@useobject{currentmarker}{}%
\end{pgfscope}%
\begin{pgfscope}%
\pgfsys@transformshift{3.274407in}{0.874107in}%
\pgfsys@useobject{currentmarker}{}%
\end{pgfscope}%
\begin{pgfscope}%
\pgfsys@transformshift{3.314683in}{0.874107in}%
\pgfsys@useobject{currentmarker}{}%
\end{pgfscope}%
\begin{pgfscope}%
\pgfsys@transformshift{3.354959in}{0.874107in}%
\pgfsys@useobject{currentmarker}{}%
\end{pgfscope}%
\begin{pgfscope}%
\pgfsys@transformshift{3.395234in}{0.874107in}%
\pgfsys@useobject{currentmarker}{}%
\end{pgfscope}%
\begin{pgfscope}%
\pgfsys@transformshift{3.435510in}{0.874107in}%
\pgfsys@useobject{currentmarker}{}%
\end{pgfscope}%
\begin{pgfscope}%
\pgfsys@transformshift{3.475785in}{0.874107in}%
\pgfsys@useobject{currentmarker}{}%
\end{pgfscope}%
\begin{pgfscope}%
\pgfsys@transformshift{3.516061in}{0.874107in}%
\pgfsys@useobject{currentmarker}{}%
\end{pgfscope}%
\begin{pgfscope}%
\pgfsys@transformshift{3.556337in}{0.874107in}%
\pgfsys@useobject{currentmarker}{}%
\end{pgfscope}%
\begin{pgfscope}%
\pgfsys@transformshift{3.596612in}{0.874107in}%
\pgfsys@useobject{currentmarker}{}%
\end{pgfscope}%
\begin{pgfscope}%
\pgfsys@transformshift{3.636888in}{0.874107in}%
\pgfsys@useobject{currentmarker}{}%
\end{pgfscope}%
\begin{pgfscope}%
\pgfsys@transformshift{3.677163in}{0.874107in}%
\pgfsys@useobject{currentmarker}{}%
\end{pgfscope}%
\begin{pgfscope}%
\pgfsys@transformshift{3.717439in}{0.874107in}%
\pgfsys@useobject{currentmarker}{}%
\end{pgfscope}%
\begin{pgfscope}%
\pgfsys@transformshift{3.757715in}{0.874107in}%
\pgfsys@useobject{currentmarker}{}%
\end{pgfscope}%
\begin{pgfscope}%
\pgfsys@transformshift{3.797990in}{0.874107in}%
\pgfsys@useobject{currentmarker}{}%
\end{pgfscope}%
\begin{pgfscope}%
\pgfsys@transformshift{3.838266in}{0.874107in}%
\pgfsys@useobject{currentmarker}{}%
\end{pgfscope}%
\end{pgfscope}%
\end{pgfpicture}%
\makeatother%
\endgroup%

  \end{subfigure}
  \begin{subfigure}[t]{0.8\textwidth}
    \centering
    %% Creator: Matplotlib, PGF backend
%%
%% To include the figure in your LaTeX document, write
%%   \input{<filename>.pgf}
%%
%% Make sure the required packages are loaded in your preamble
%%   \usepackage{pgf}
%%
%% Figures using additional raster images can only be included by \input if
%% they are in the same directory as the main LaTeX file. For loading figures
%% from other directories you can use the `import` package
%%   \usepackage{import}
%% and then include the figures with
%%   \import{<path to file>}{<filename>.pgf}
%%
%% Matplotlib used the following preamble
%%   \usepackage{fontspec}
%%   \setmainfont{DejaVu Serif}
%%   \setsansfont{DejaVu Sans}
%%   \setmonofont{DejaVu Sans Mono}
%%
\begingroup%
\makeatletter%
\begin{pgfpicture}%
\pgfpathrectangle{\pgfpointorigin}{\pgfqpoint{3.908404in}{2.872910in}}%
\pgfusepath{use as bounding box, clip}%
\begin{pgfscope}%
\pgfsetbuttcap%
\pgfsetmiterjoin%
\definecolor{currentfill}{rgb}{1.000000,1.000000,1.000000}%
\pgfsetfillcolor{currentfill}%
\pgfsetlinewidth{0.000000pt}%
\definecolor{currentstroke}{rgb}{1.000000,1.000000,1.000000}%
\pgfsetstrokecolor{currentstroke}%
\pgfsetdash{}{0pt}%
\pgfpathmoveto{\pgfqpoint{0.000000in}{0.000000in}}%
\pgfpathlineto{\pgfqpoint{3.908404in}{0.000000in}}%
\pgfpathlineto{\pgfqpoint{3.908404in}{2.872910in}}%
\pgfpathlineto{\pgfqpoint{0.000000in}{2.872910in}}%
\pgfpathclose%
\pgfusepath{fill}%
\end{pgfscope}%
\begin{pgfscope}%
\pgfsetbuttcap%
\pgfsetmiterjoin%
\definecolor{currentfill}{rgb}{1.000000,1.000000,1.000000}%
\pgfsetfillcolor{currentfill}%
\pgfsetlinewidth{0.000000pt}%
\definecolor{currentstroke}{rgb}{0.000000,0.000000,0.000000}%
\pgfsetstrokecolor{currentstroke}%
\pgfsetstrokeopacity{0.000000}%
\pgfsetdash{}{0pt}%
\pgfpathmoveto{\pgfqpoint{0.636356in}{0.440955in}}%
\pgfpathlineto{\pgfqpoint{3.858404in}{0.440955in}}%
\pgfpathlineto{\pgfqpoint{3.858404in}{0.711675in}}%
\pgfpathlineto{\pgfqpoint{0.636356in}{0.711675in}}%
\pgfpathclose%
\pgfusepath{fill}%
\end{pgfscope}%
\begin{pgfscope}%
\pgfpathrectangle{\pgfqpoint{0.636356in}{0.440955in}}{\pgfqpoint{3.222048in}{0.270720in}} %
\pgfusepath{clip}%
\pgfsetbuttcap%
\pgfsetroundjoin%
\definecolor{currentfill}{rgb}{0.733333,0.733333,0.733333}%
\pgfsetfillcolor{currentfill}%
\pgfsetlinewidth{0.000000pt}%
\definecolor{currentstroke}{rgb}{0.733333,0.733333,0.733333}%
\pgfsetstrokecolor{currentstroke}%
\pgfsetdash{}{0pt}%
\pgfpathmoveto{\pgfqpoint{0.636356in}{0.621435in}}%
\pgfpathlineto{\pgfqpoint{0.636356in}{0.666555in}}%
\pgfpathlineto{\pgfqpoint{3.858404in}{0.666555in}}%
\pgfpathlineto{\pgfqpoint{3.858404in}{0.621435in}}%
\pgfpathlineto{\pgfqpoint{3.858404in}{0.621435in}}%
\pgfpathlineto{\pgfqpoint{0.636356in}{0.621435in}}%
\pgfpathlineto{\pgfqpoint{0.636356in}{0.621435in}}%
\pgfusepath{fill}%
\end{pgfscope}%
\begin{pgfscope}%
\pgfpathrectangle{\pgfqpoint{0.636356in}{0.440955in}}{\pgfqpoint{3.222048in}{0.270720in}} %
\pgfusepath{clip}%
\pgfsetbuttcap%
\pgfsetroundjoin%
\definecolor{currentfill}{rgb}{0.733333,0.733333,0.733333}%
\pgfsetfillcolor{currentfill}%
\pgfsetlinewidth{0.000000pt}%
\definecolor{currentstroke}{rgb}{0.733333,0.733333,0.733333}%
\pgfsetstrokecolor{currentstroke}%
\pgfsetdash{}{0pt}%
\pgfpathmoveto{\pgfqpoint{0.636356in}{0.531195in}}%
\pgfpathlineto{\pgfqpoint{0.636356in}{0.486075in}}%
\pgfpathlineto{\pgfqpoint{3.858404in}{0.486075in}}%
\pgfpathlineto{\pgfqpoint{3.858404in}{0.531195in}}%
\pgfpathlineto{\pgfqpoint{3.858404in}{0.531195in}}%
\pgfpathlineto{\pgfqpoint{0.636356in}{0.531195in}}%
\pgfpathlineto{\pgfqpoint{0.636356in}{0.531195in}}%
\pgfusepath{fill}%
\end{pgfscope}%
\begin{pgfscope}%
\pgfpathrectangle{\pgfqpoint{0.636356in}{0.440955in}}{\pgfqpoint{3.222048in}{0.270720in}} %
\pgfusepath{clip}%
\pgfsetbuttcap%
\pgfsetmiterjoin%
\definecolor{currentfill}{rgb}{0.333333,0.333333,0.333333}%
\pgfsetfillcolor{currentfill}%
\pgfsetlinewidth{0.501875pt}%
\definecolor{currentstroke}{rgb}{0.000000,0.000000,0.000000}%
\pgfsetstrokecolor{currentstroke}%
\pgfsetdash{}{0pt}%
\pgfpathmoveto{\pgfqpoint{0.636356in}{0.576315in}}%
\pgfpathlineto{\pgfqpoint{0.676631in}{0.576315in}}%
\pgfpathlineto{\pgfqpoint{0.676631in}{0.617904in}}%
\pgfpathlineto{\pgfqpoint{0.636356in}{0.617904in}}%
\pgfpathlineto{\pgfqpoint{0.636356in}{0.576315in}}%
\pgfusepath{stroke,fill}%
\end{pgfscope}%
\begin{pgfscope}%
\pgfpathrectangle{\pgfqpoint{0.636356in}{0.440955in}}{\pgfqpoint{3.222048in}{0.270720in}} %
\pgfusepath{clip}%
\pgfsetbuttcap%
\pgfsetmiterjoin%
\definecolor{currentfill}{rgb}{0.333333,0.333333,0.333333}%
\pgfsetfillcolor{currentfill}%
\pgfsetlinewidth{0.501875pt}%
\definecolor{currentstroke}{rgb}{0.000000,0.000000,0.000000}%
\pgfsetstrokecolor{currentstroke}%
\pgfsetdash{}{0pt}%
\pgfpathmoveto{\pgfqpoint{0.676631in}{0.576315in}}%
\pgfpathlineto{\pgfqpoint{0.716907in}{0.576315in}}%
\pgfpathlineto{\pgfqpoint{0.716907in}{0.594020in}}%
\pgfpathlineto{\pgfqpoint{0.676631in}{0.594020in}}%
\pgfpathlineto{\pgfqpoint{0.676631in}{0.576315in}}%
\pgfusepath{stroke,fill}%
\end{pgfscope}%
\begin{pgfscope}%
\pgfpathrectangle{\pgfqpoint{0.636356in}{0.440955in}}{\pgfqpoint{3.222048in}{0.270720in}} %
\pgfusepath{clip}%
\pgfsetbuttcap%
\pgfsetmiterjoin%
\definecolor{currentfill}{rgb}{0.333333,0.333333,0.333333}%
\pgfsetfillcolor{currentfill}%
\pgfsetlinewidth{0.501875pt}%
\definecolor{currentstroke}{rgb}{0.000000,0.000000,0.000000}%
\pgfsetstrokecolor{currentstroke}%
\pgfsetdash{}{0pt}%
\pgfpathmoveto{\pgfqpoint{0.716907in}{0.576315in}}%
\pgfpathlineto{\pgfqpoint{0.757183in}{0.576315in}}%
\pgfpathlineto{\pgfqpoint{0.757183in}{0.594088in}}%
\pgfpathlineto{\pgfqpoint{0.716907in}{0.594088in}}%
\pgfpathlineto{\pgfqpoint{0.716907in}{0.576315in}}%
\pgfusepath{stroke,fill}%
\end{pgfscope}%
\begin{pgfscope}%
\pgfpathrectangle{\pgfqpoint{0.636356in}{0.440955in}}{\pgfqpoint{3.222048in}{0.270720in}} %
\pgfusepath{clip}%
\pgfsetbuttcap%
\pgfsetmiterjoin%
\definecolor{currentfill}{rgb}{0.333333,0.333333,0.333333}%
\pgfsetfillcolor{currentfill}%
\pgfsetlinewidth{0.501875pt}%
\definecolor{currentstroke}{rgb}{0.000000,0.000000,0.000000}%
\pgfsetstrokecolor{currentstroke}%
\pgfsetdash{}{0pt}%
\pgfpathmoveto{\pgfqpoint{0.757183in}{0.566717in}}%
\pgfpathlineto{\pgfqpoint{0.797458in}{0.566717in}}%
\pgfpathlineto{\pgfqpoint{0.797458in}{0.576315in}}%
\pgfpathlineto{\pgfqpoint{0.757183in}{0.576315in}}%
\pgfpathlineto{\pgfqpoint{0.757183in}{0.566717in}}%
\pgfusepath{stroke,fill}%
\end{pgfscope}%
\begin{pgfscope}%
\pgfpathrectangle{\pgfqpoint{0.636356in}{0.440955in}}{\pgfqpoint{3.222048in}{0.270720in}} %
\pgfusepath{clip}%
\pgfsetbuttcap%
\pgfsetmiterjoin%
\definecolor{currentfill}{rgb}{0.333333,0.333333,0.333333}%
\pgfsetfillcolor{currentfill}%
\pgfsetlinewidth{0.501875pt}%
\definecolor{currentstroke}{rgb}{0.000000,0.000000,0.000000}%
\pgfsetstrokecolor{currentstroke}%
\pgfsetdash{}{0pt}%
\pgfpathmoveto{\pgfqpoint{0.797458in}{0.517857in}}%
\pgfpathlineto{\pgfqpoint{0.837734in}{0.517857in}}%
\pgfpathlineto{\pgfqpoint{0.837734in}{0.576315in}}%
\pgfpathlineto{\pgfqpoint{0.797458in}{0.576315in}}%
\pgfpathlineto{\pgfqpoint{0.797458in}{0.517857in}}%
\pgfusepath{stroke,fill}%
\end{pgfscope}%
\begin{pgfscope}%
\pgfpathrectangle{\pgfqpoint{0.636356in}{0.440955in}}{\pgfqpoint{3.222048in}{0.270720in}} %
\pgfusepath{clip}%
\pgfsetbuttcap%
\pgfsetmiterjoin%
\definecolor{currentfill}{rgb}{0.333333,0.333333,0.333333}%
\pgfsetfillcolor{currentfill}%
\pgfsetlinewidth{0.501875pt}%
\definecolor{currentstroke}{rgb}{0.000000,0.000000,0.000000}%
\pgfsetstrokecolor{currentstroke}%
\pgfsetdash{}{0pt}%
\pgfpathmoveto{\pgfqpoint{0.837734in}{0.517934in}}%
\pgfpathlineto{\pgfqpoint{0.878009in}{0.517934in}}%
\pgfpathlineto{\pgfqpoint{0.878009in}{0.576315in}}%
\pgfpathlineto{\pgfqpoint{0.837734in}{0.576315in}}%
\pgfpathlineto{\pgfqpoint{0.837734in}{0.517934in}}%
\pgfusepath{stroke,fill}%
\end{pgfscope}%
\begin{pgfscope}%
\pgfpathrectangle{\pgfqpoint{0.636356in}{0.440955in}}{\pgfqpoint{3.222048in}{0.270720in}} %
\pgfusepath{clip}%
\pgfsetbuttcap%
\pgfsetmiterjoin%
\definecolor{currentfill}{rgb}{0.333333,0.333333,0.333333}%
\pgfsetfillcolor{currentfill}%
\pgfsetlinewidth{0.501875pt}%
\definecolor{currentstroke}{rgb}{0.000000,0.000000,0.000000}%
\pgfsetstrokecolor{currentstroke}%
\pgfsetdash{}{0pt}%
\pgfpathmoveto{\pgfqpoint{0.878009in}{0.566832in}}%
\pgfpathlineto{\pgfqpoint{0.918285in}{0.566832in}}%
\pgfpathlineto{\pgfqpoint{0.918285in}{0.576315in}}%
\pgfpathlineto{\pgfqpoint{0.878009in}{0.576315in}}%
\pgfpathlineto{\pgfqpoint{0.878009in}{0.566832in}}%
\pgfusepath{stroke,fill}%
\end{pgfscope}%
\begin{pgfscope}%
\pgfpathrectangle{\pgfqpoint{0.636356in}{0.440955in}}{\pgfqpoint{3.222048in}{0.270720in}} %
\pgfusepath{clip}%
\pgfsetbuttcap%
\pgfsetmiterjoin%
\definecolor{currentfill}{rgb}{0.333333,0.333333,0.333333}%
\pgfsetfillcolor{currentfill}%
\pgfsetlinewidth{0.501875pt}%
\definecolor{currentstroke}{rgb}{0.000000,0.000000,0.000000}%
\pgfsetstrokecolor{currentstroke}%
\pgfsetdash{}{0pt}%
\pgfpathmoveto{\pgfqpoint{0.918285in}{0.576315in}}%
\pgfpathlineto{\pgfqpoint{0.958561in}{0.576315in}}%
\pgfpathlineto{\pgfqpoint{0.958561in}{0.594431in}}%
\pgfpathlineto{\pgfqpoint{0.918285in}{0.594431in}}%
\pgfpathlineto{\pgfqpoint{0.918285in}{0.576315in}}%
\pgfusepath{stroke,fill}%
\end{pgfscope}%
\begin{pgfscope}%
\pgfpathrectangle{\pgfqpoint{0.636356in}{0.440955in}}{\pgfqpoint{3.222048in}{0.270720in}} %
\pgfusepath{clip}%
\pgfsetbuttcap%
\pgfsetmiterjoin%
\definecolor{currentfill}{rgb}{0.333333,0.333333,0.333333}%
\pgfsetfillcolor{currentfill}%
\pgfsetlinewidth{0.501875pt}%
\definecolor{currentstroke}{rgb}{0.000000,0.000000,0.000000}%
\pgfsetstrokecolor{currentstroke}%
\pgfsetdash{}{0pt}%
\pgfpathmoveto{\pgfqpoint{0.958561in}{0.566909in}}%
\pgfpathlineto{\pgfqpoint{0.998836in}{0.566909in}}%
\pgfpathlineto{\pgfqpoint{0.998836in}{0.576315in}}%
\pgfpathlineto{\pgfqpoint{0.958561in}{0.576315in}}%
\pgfpathlineto{\pgfqpoint{0.958561in}{0.566909in}}%
\pgfusepath{stroke,fill}%
\end{pgfscope}%
\begin{pgfscope}%
\pgfpathrectangle{\pgfqpoint{0.636356in}{0.440955in}}{\pgfqpoint{3.222048in}{0.270720in}} %
\pgfusepath{clip}%
\pgfsetbuttcap%
\pgfsetmiterjoin%
\definecolor{currentfill}{rgb}{0.333333,0.333333,0.333333}%
\pgfsetfillcolor{currentfill}%
\pgfsetlinewidth{0.501875pt}%
\definecolor{currentstroke}{rgb}{0.000000,0.000000,0.000000}%
\pgfsetstrokecolor{currentstroke}%
\pgfsetdash{}{0pt}%
\pgfpathmoveto{\pgfqpoint{0.998836in}{0.518241in}}%
\pgfpathlineto{\pgfqpoint{1.039112in}{0.518241in}}%
\pgfpathlineto{\pgfqpoint{1.039112in}{0.576315in}}%
\pgfpathlineto{\pgfqpoint{0.998836in}{0.576315in}}%
\pgfpathlineto{\pgfqpoint{0.998836in}{0.518241in}}%
\pgfusepath{stroke,fill}%
\end{pgfscope}%
\begin{pgfscope}%
\pgfpathrectangle{\pgfqpoint{0.636356in}{0.440955in}}{\pgfqpoint{3.222048in}{0.270720in}} %
\pgfusepath{clip}%
\pgfsetbuttcap%
\pgfsetmiterjoin%
\definecolor{currentfill}{rgb}{0.333333,0.333333,0.333333}%
\pgfsetfillcolor{currentfill}%
\pgfsetlinewidth{0.501875pt}%
\definecolor{currentstroke}{rgb}{0.000000,0.000000,0.000000}%
\pgfsetstrokecolor{currentstroke}%
\pgfsetdash{}{0pt}%
\pgfpathmoveto{\pgfqpoint{1.039112in}{0.576315in}}%
\pgfpathlineto{\pgfqpoint{1.079387in}{0.576315in}}%
\pgfpathlineto{\pgfqpoint{1.079387in}{0.594635in}}%
\pgfpathlineto{\pgfqpoint{1.039112in}{0.594635in}}%
\pgfpathlineto{\pgfqpoint{1.039112in}{0.576315in}}%
\pgfusepath{stroke,fill}%
\end{pgfscope}%
\begin{pgfscope}%
\pgfpathrectangle{\pgfqpoint{0.636356in}{0.440955in}}{\pgfqpoint{3.222048in}{0.270720in}} %
\pgfusepath{clip}%
\pgfsetbuttcap%
\pgfsetmiterjoin%
\definecolor{currentfill}{rgb}{0.333333,0.333333,0.333333}%
\pgfsetfillcolor{currentfill}%
\pgfsetlinewidth{0.501875pt}%
\definecolor{currentstroke}{rgb}{0.000000,0.000000,0.000000}%
\pgfsetstrokecolor{currentstroke}%
\pgfsetdash{}{0pt}%
\pgfpathmoveto{\pgfqpoint{1.079387in}{0.576315in}}%
\pgfpathlineto{\pgfqpoint{1.119663in}{0.576315in}}%
\pgfpathlineto{\pgfqpoint{1.119663in}{0.594703in}}%
\pgfpathlineto{\pgfqpoint{1.079387in}{0.594703in}}%
\pgfpathlineto{\pgfqpoint{1.079387in}{0.576315in}}%
\pgfusepath{stroke,fill}%
\end{pgfscope}%
\begin{pgfscope}%
\pgfpathrectangle{\pgfqpoint{0.636356in}{0.440955in}}{\pgfqpoint{3.222048in}{0.270720in}} %
\pgfusepath{clip}%
\pgfsetbuttcap%
\pgfsetmiterjoin%
\definecolor{currentfill}{rgb}{0.333333,0.333333,0.333333}%
\pgfsetfillcolor{currentfill}%
\pgfsetlinewidth{0.501875pt}%
\definecolor{currentstroke}{rgb}{0.000000,0.000000,0.000000}%
\pgfsetstrokecolor{currentstroke}%
\pgfsetdash{}{0pt}%
\pgfpathmoveto{\pgfqpoint{1.119663in}{0.567061in}}%
\pgfpathlineto{\pgfqpoint{1.159939in}{0.567061in}}%
\pgfpathlineto{\pgfqpoint{1.159939in}{0.576315in}}%
\pgfpathlineto{\pgfqpoint{1.119663in}{0.576315in}}%
\pgfpathlineto{\pgfqpoint{1.119663in}{0.567061in}}%
\pgfusepath{stroke,fill}%
\end{pgfscope}%
\begin{pgfscope}%
\pgfpathrectangle{\pgfqpoint{0.636356in}{0.440955in}}{\pgfqpoint{3.222048in}{0.270720in}} %
\pgfusepath{clip}%
\pgfsetbuttcap%
\pgfsetmiterjoin%
\definecolor{currentfill}{rgb}{0.333333,0.333333,0.333333}%
\pgfsetfillcolor{currentfill}%
\pgfsetlinewidth{0.501875pt}%
\definecolor{currentstroke}{rgb}{0.000000,0.000000,0.000000}%
\pgfsetstrokecolor{currentstroke}%
\pgfsetdash{}{0pt}%
\pgfpathmoveto{\pgfqpoint{1.159939in}{0.567098in}}%
\pgfpathlineto{\pgfqpoint{1.200214in}{0.567098in}}%
\pgfpathlineto{\pgfqpoint{1.200214in}{0.576315in}}%
\pgfpathlineto{\pgfqpoint{1.159939in}{0.576315in}}%
\pgfpathlineto{\pgfqpoint{1.159939in}{0.567098in}}%
\pgfusepath{stroke,fill}%
\end{pgfscope}%
\begin{pgfscope}%
\pgfpathrectangle{\pgfqpoint{0.636356in}{0.440955in}}{\pgfqpoint{3.222048in}{0.270720in}} %
\pgfusepath{clip}%
\pgfsetbuttcap%
\pgfsetmiterjoin%
\definecolor{currentfill}{rgb}{0.333333,0.333333,0.333333}%
\pgfsetfillcolor{currentfill}%
\pgfsetlinewidth{0.501875pt}%
\definecolor{currentstroke}{rgb}{0.000000,0.000000,0.000000}%
\pgfsetstrokecolor{currentstroke}%
\pgfsetdash{}{0pt}%
\pgfpathmoveto{\pgfqpoint{1.200214in}{0.567134in}}%
\pgfpathlineto{\pgfqpoint{1.240490in}{0.567134in}}%
\pgfpathlineto{\pgfqpoint{1.240490in}{0.576315in}}%
\pgfpathlineto{\pgfqpoint{1.200214in}{0.576315in}}%
\pgfpathlineto{\pgfqpoint{1.200214in}{0.567134in}}%
\pgfusepath{stroke,fill}%
\end{pgfscope}%
\begin{pgfscope}%
\pgfpathrectangle{\pgfqpoint{0.636356in}{0.440955in}}{\pgfqpoint{3.222048in}{0.270720in}} %
\pgfusepath{clip}%
\pgfsetbuttcap%
\pgfsetmiterjoin%
\definecolor{currentfill}{rgb}{0.333333,0.333333,0.333333}%
\pgfsetfillcolor{currentfill}%
\pgfsetlinewidth{0.501875pt}%
\definecolor{currentstroke}{rgb}{0.000000,0.000000,0.000000}%
\pgfsetstrokecolor{currentstroke}%
\pgfsetdash{}{0pt}%
\pgfpathmoveto{\pgfqpoint{1.240490in}{0.576315in}}%
\pgfpathlineto{\pgfqpoint{1.280765in}{0.576315in}}%
\pgfpathlineto{\pgfqpoint{1.280765in}{0.594959in}}%
\pgfpathlineto{\pgfqpoint{1.240490in}{0.594959in}}%
\pgfpathlineto{\pgfqpoint{1.240490in}{0.576315in}}%
\pgfusepath{stroke,fill}%
\end{pgfscope}%
\begin{pgfscope}%
\pgfpathrectangle{\pgfqpoint{0.636356in}{0.440955in}}{\pgfqpoint{3.222048in}{0.270720in}} %
\pgfusepath{clip}%
\pgfsetbuttcap%
\pgfsetmiterjoin%
\definecolor{currentfill}{rgb}{0.333333,0.333333,0.333333}%
\pgfsetfillcolor{currentfill}%
\pgfsetlinewidth{0.501875pt}%
\definecolor{currentstroke}{rgb}{0.000000,0.000000,0.000000}%
\pgfsetstrokecolor{currentstroke}%
\pgfsetdash{}{0pt}%
\pgfpathmoveto{\pgfqpoint{1.280765in}{0.518739in}}%
\pgfpathlineto{\pgfqpoint{1.321041in}{0.518739in}}%
\pgfpathlineto{\pgfqpoint{1.321041in}{0.576315in}}%
\pgfpathlineto{\pgfqpoint{1.280765in}{0.576315in}}%
\pgfpathlineto{\pgfqpoint{1.280765in}{0.518739in}}%
\pgfusepath{stroke,fill}%
\end{pgfscope}%
\begin{pgfscope}%
\pgfpathrectangle{\pgfqpoint{0.636356in}{0.440955in}}{\pgfqpoint{3.222048in}{0.270720in}} %
\pgfusepath{clip}%
\pgfsetbuttcap%
\pgfsetmiterjoin%
\definecolor{currentfill}{rgb}{0.333333,0.333333,0.333333}%
\pgfsetfillcolor{currentfill}%
\pgfsetlinewidth{0.501875pt}%
\definecolor{currentstroke}{rgb}{0.000000,0.000000,0.000000}%
\pgfsetstrokecolor{currentstroke}%
\pgfsetdash{}{0pt}%
\pgfpathmoveto{\pgfqpoint{1.321041in}{0.576315in}}%
\pgfpathlineto{\pgfqpoint{1.361317in}{0.576315in}}%
\pgfpathlineto{\pgfqpoint{1.361317in}{0.595044in}}%
\pgfpathlineto{\pgfqpoint{1.321041in}{0.595044in}}%
\pgfpathlineto{\pgfqpoint{1.321041in}{0.576315in}}%
\pgfusepath{stroke,fill}%
\end{pgfscope}%
\begin{pgfscope}%
\pgfpathrectangle{\pgfqpoint{0.636356in}{0.440955in}}{\pgfqpoint{3.222048in}{0.270720in}} %
\pgfusepath{clip}%
\pgfsetbuttcap%
\pgfsetmiterjoin%
\definecolor{currentfill}{rgb}{0.333333,0.333333,0.333333}%
\pgfsetfillcolor{currentfill}%
\pgfsetlinewidth{0.501875pt}%
\definecolor{currentstroke}{rgb}{0.000000,0.000000,0.000000}%
\pgfsetstrokecolor{currentstroke}%
\pgfsetdash{}{0pt}%
\pgfpathmoveto{\pgfqpoint{1.361317in}{0.576315in}}%
\pgfpathlineto{\pgfqpoint{1.401592in}{0.576315in}}%
\pgfpathlineto{\pgfqpoint{1.401592in}{0.636092in}}%
\pgfpathlineto{\pgfqpoint{1.361317in}{0.636092in}}%
\pgfpathlineto{\pgfqpoint{1.361317in}{0.576315in}}%
\pgfusepath{stroke,fill}%
\end{pgfscope}%
\begin{pgfscope}%
\pgfpathrectangle{\pgfqpoint{0.636356in}{0.440955in}}{\pgfqpoint{3.222048in}{0.270720in}} %
\pgfusepath{clip}%
\pgfsetbuttcap%
\pgfsetmiterjoin%
\definecolor{currentfill}{rgb}{0.333333,0.333333,0.333333}%
\pgfsetfillcolor{currentfill}%
\pgfsetlinewidth{0.501875pt}%
\definecolor{currentstroke}{rgb}{0.000000,0.000000,0.000000}%
\pgfsetstrokecolor{currentstroke}%
\pgfsetdash{}{0pt}%
\pgfpathmoveto{\pgfqpoint{1.401592in}{0.567178in}}%
\pgfpathlineto{\pgfqpoint{1.441868in}{0.567178in}}%
\pgfpathlineto{\pgfqpoint{1.441868in}{0.576315in}}%
\pgfpathlineto{\pgfqpoint{1.401592in}{0.576315in}}%
\pgfpathlineto{\pgfqpoint{1.401592in}{0.567178in}}%
\pgfusepath{stroke,fill}%
\end{pgfscope}%
\begin{pgfscope}%
\pgfpathrectangle{\pgfqpoint{0.636356in}{0.440955in}}{\pgfqpoint{3.222048in}{0.270720in}} %
\pgfusepath{clip}%
\pgfsetbuttcap%
\pgfsetmiterjoin%
\definecolor{currentfill}{rgb}{0.333333,0.333333,0.333333}%
\pgfsetfillcolor{currentfill}%
\pgfsetlinewidth{0.501875pt}%
\definecolor{currentstroke}{rgb}{0.000000,0.000000,0.000000}%
\pgfsetstrokecolor{currentstroke}%
\pgfsetdash{}{0pt}%
\pgfpathmoveto{\pgfqpoint{1.441868in}{0.567081in}}%
\pgfpathlineto{\pgfqpoint{1.482143in}{0.567081in}}%
\pgfpathlineto{\pgfqpoint{1.482143in}{0.576315in}}%
\pgfpathlineto{\pgfqpoint{1.441868in}{0.576315in}}%
\pgfpathlineto{\pgfqpoint{1.441868in}{0.567081in}}%
\pgfusepath{stroke,fill}%
\end{pgfscope}%
\begin{pgfscope}%
\pgfpathrectangle{\pgfqpoint{0.636356in}{0.440955in}}{\pgfqpoint{3.222048in}{0.270720in}} %
\pgfusepath{clip}%
\pgfsetbuttcap%
\pgfsetmiterjoin%
\definecolor{currentfill}{rgb}{0.333333,0.333333,0.333333}%
\pgfsetfillcolor{currentfill}%
\pgfsetlinewidth{0.501875pt}%
\definecolor{currentstroke}{rgb}{0.000000,0.000000,0.000000}%
\pgfsetstrokecolor{currentstroke}%
\pgfsetdash{}{0pt}%
\pgfpathmoveto{\pgfqpoint{1.482143in}{0.576315in}}%
\pgfpathlineto{\pgfqpoint{1.522419in}{0.576315in}}%
\pgfpathlineto{\pgfqpoint{1.522419in}{0.594456in}}%
\pgfpathlineto{\pgfqpoint{1.482143in}{0.594456in}}%
\pgfpathlineto{\pgfqpoint{1.482143in}{0.576315in}}%
\pgfusepath{stroke,fill}%
\end{pgfscope}%
\begin{pgfscope}%
\pgfpathrectangle{\pgfqpoint{0.636356in}{0.440955in}}{\pgfqpoint{3.222048in}{0.270720in}} %
\pgfusepath{clip}%
\pgfsetbuttcap%
\pgfsetmiterjoin%
\definecolor{currentfill}{rgb}{0.333333,0.333333,0.333333}%
\pgfsetfillcolor{currentfill}%
\pgfsetlinewidth{0.501875pt}%
\definecolor{currentstroke}{rgb}{0.000000,0.000000,0.000000}%
\pgfsetstrokecolor{currentstroke}%
\pgfsetdash{}{0pt}%
\pgfpathmoveto{\pgfqpoint{1.522419in}{0.517360in}}%
\pgfpathlineto{\pgfqpoint{1.562695in}{0.517360in}}%
\pgfpathlineto{\pgfqpoint{1.562695in}{0.576315in}}%
\pgfpathlineto{\pgfqpoint{1.522419in}{0.576315in}}%
\pgfpathlineto{\pgfqpoint{1.522419in}{0.517360in}}%
\pgfusepath{stroke,fill}%
\end{pgfscope}%
\begin{pgfscope}%
\pgfpathrectangle{\pgfqpoint{0.636356in}{0.440955in}}{\pgfqpoint{3.222048in}{0.270720in}} %
\pgfusepath{clip}%
\pgfsetbuttcap%
\pgfsetmiterjoin%
\definecolor{currentfill}{rgb}{0.333333,0.333333,0.333333}%
\pgfsetfillcolor{currentfill}%
\pgfsetlinewidth{0.501875pt}%
\definecolor{currentstroke}{rgb}{0.000000,0.000000,0.000000}%
\pgfsetstrokecolor{currentstroke}%
\pgfsetdash{}{0pt}%
\pgfpathmoveto{\pgfqpoint{1.562695in}{0.516011in}}%
\pgfpathlineto{\pgfqpoint{1.602970in}{0.516011in}}%
\pgfpathlineto{\pgfqpoint{1.602970in}{0.576315in}}%
\pgfpathlineto{\pgfqpoint{1.562695in}{0.576315in}}%
\pgfpathlineto{\pgfqpoint{1.562695in}{0.516011in}}%
\pgfusepath{stroke,fill}%
\end{pgfscope}%
\begin{pgfscope}%
\pgfpathrectangle{\pgfqpoint{0.636356in}{0.440955in}}{\pgfqpoint{3.222048in}{0.270720in}} %
\pgfusepath{clip}%
\pgfsetbuttcap%
\pgfsetmiterjoin%
\definecolor{currentfill}{rgb}{0.333333,0.333333,0.333333}%
\pgfsetfillcolor{currentfill}%
\pgfsetlinewidth{0.501875pt}%
\definecolor{currentstroke}{rgb}{0.000000,0.000000,0.000000}%
\pgfsetstrokecolor{currentstroke}%
\pgfsetdash{}{0pt}%
\pgfpathmoveto{\pgfqpoint{1.602970in}{0.564739in}}%
\pgfpathlineto{\pgfqpoint{1.643246in}{0.564739in}}%
\pgfpathlineto{\pgfqpoint{1.643246in}{0.576315in}}%
\pgfpathlineto{\pgfqpoint{1.602970in}{0.576315in}}%
\pgfpathlineto{\pgfqpoint{1.602970in}{0.564739in}}%
\pgfusepath{stroke,fill}%
\end{pgfscope}%
\begin{pgfscope}%
\pgfpathrectangle{\pgfqpoint{0.636356in}{0.440955in}}{\pgfqpoint{3.222048in}{0.270720in}} %
\pgfusepath{clip}%
\pgfsetbuttcap%
\pgfsetmiterjoin%
\definecolor{currentfill}{rgb}{0.333333,0.333333,0.333333}%
\pgfsetfillcolor{currentfill}%
\pgfsetlinewidth{0.501875pt}%
\definecolor{currentstroke}{rgb}{0.000000,0.000000,0.000000}%
\pgfsetstrokecolor{currentstroke}%
\pgfsetdash{}{0pt}%
\pgfpathmoveto{\pgfqpoint{1.643246in}{0.510403in}}%
\pgfpathlineto{\pgfqpoint{1.683521in}{0.510403in}}%
\pgfpathlineto{\pgfqpoint{1.683521in}{0.576315in}}%
\pgfpathlineto{\pgfqpoint{1.643246in}{0.576315in}}%
\pgfpathlineto{\pgfqpoint{1.643246in}{0.510403in}}%
\pgfusepath{stroke,fill}%
\end{pgfscope}%
\begin{pgfscope}%
\pgfpathrectangle{\pgfqpoint{0.636356in}{0.440955in}}{\pgfqpoint{3.222048in}{0.270720in}} %
\pgfusepath{clip}%
\pgfsetbuttcap%
\pgfsetmiterjoin%
\definecolor{currentfill}{rgb}{0.333333,0.333333,0.333333}%
\pgfsetfillcolor{currentfill}%
\pgfsetlinewidth{0.501875pt}%
\definecolor{currentstroke}{rgb}{0.000000,0.000000,0.000000}%
\pgfsetstrokecolor{currentstroke}%
\pgfsetdash{}{0pt}%
\pgfpathmoveto{\pgfqpoint{1.683521in}{0.576315in}}%
\pgfpathlineto{\pgfqpoint{1.723797in}{0.576315in}}%
\pgfpathlineto{\pgfqpoint{1.723797in}{0.627886in}}%
\pgfpathlineto{\pgfqpoint{1.683521in}{0.627886in}}%
\pgfpathlineto{\pgfqpoint{1.683521in}{0.576315in}}%
\pgfusepath{stroke,fill}%
\end{pgfscope}%
\begin{pgfscope}%
\pgfpathrectangle{\pgfqpoint{0.636356in}{0.440955in}}{\pgfqpoint{3.222048in}{0.270720in}} %
\pgfusepath{clip}%
\pgfsetbuttcap%
\pgfsetmiterjoin%
\definecolor{currentfill}{rgb}{0.333333,0.333333,0.333333}%
\pgfsetfillcolor{currentfill}%
\pgfsetlinewidth{0.501875pt}%
\definecolor{currentstroke}{rgb}{0.000000,0.000000,0.000000}%
\pgfsetstrokecolor{currentstroke}%
\pgfsetdash{}{0pt}%
\pgfpathmoveto{\pgfqpoint{1.723797in}{0.556419in}}%
\pgfpathlineto{\pgfqpoint{1.764073in}{0.556419in}}%
\pgfpathlineto{\pgfqpoint{1.764073in}{0.576315in}}%
\pgfpathlineto{\pgfqpoint{1.723797in}{0.576315in}}%
\pgfpathlineto{\pgfqpoint{1.723797in}{0.556419in}}%
\pgfusepath{stroke,fill}%
\end{pgfscope}%
\begin{pgfscope}%
\pgfpathrectangle{\pgfqpoint{0.636356in}{0.440955in}}{\pgfqpoint{3.222048in}{0.270720in}} %
\pgfusepath{clip}%
\pgfsetbuttcap%
\pgfsetmiterjoin%
\definecolor{currentfill}{rgb}{0.333333,0.333333,0.333333}%
\pgfsetfillcolor{currentfill}%
\pgfsetlinewidth{0.501875pt}%
\definecolor{currentstroke}{rgb}{0.000000,0.000000,0.000000}%
\pgfsetstrokecolor{currentstroke}%
\pgfsetdash{}{0pt}%
\pgfpathmoveto{\pgfqpoint{1.764073in}{0.571236in}}%
\pgfpathlineto{\pgfqpoint{1.804348in}{0.571236in}}%
\pgfpathlineto{\pgfqpoint{1.804348in}{0.576315in}}%
\pgfpathlineto{\pgfqpoint{1.764073in}{0.576315in}}%
\pgfpathlineto{\pgfqpoint{1.764073in}{0.571236in}}%
\pgfusepath{stroke,fill}%
\end{pgfscope}%
\begin{pgfscope}%
\pgfpathrectangle{\pgfqpoint{0.636356in}{0.440955in}}{\pgfqpoint{3.222048in}{0.270720in}} %
\pgfusepath{clip}%
\pgfsetbuttcap%
\pgfsetmiterjoin%
\definecolor{currentfill}{rgb}{0.333333,0.333333,0.333333}%
\pgfsetfillcolor{currentfill}%
\pgfsetlinewidth{0.501875pt}%
\definecolor{currentstroke}{rgb}{0.000000,0.000000,0.000000}%
\pgfsetstrokecolor{currentstroke}%
\pgfsetdash{}{0pt}%
\pgfpathmoveto{\pgfqpoint{1.804348in}{0.576315in}}%
\pgfpathlineto{\pgfqpoint{1.844624in}{0.576315in}}%
\pgfpathlineto{\pgfqpoint{1.844624in}{0.607441in}}%
\pgfpathlineto{\pgfqpoint{1.804348in}{0.607441in}}%
\pgfpathlineto{\pgfqpoint{1.804348in}{0.576315in}}%
\pgfusepath{stroke,fill}%
\end{pgfscope}%
\begin{pgfscope}%
\pgfpathrectangle{\pgfqpoint{0.636356in}{0.440955in}}{\pgfqpoint{3.222048in}{0.270720in}} %
\pgfusepath{clip}%
\pgfsetbuttcap%
\pgfsetmiterjoin%
\definecolor{currentfill}{rgb}{0.333333,0.333333,0.333333}%
\pgfsetfillcolor{currentfill}%
\pgfsetlinewidth{0.501875pt}%
\definecolor{currentstroke}{rgb}{0.000000,0.000000,0.000000}%
\pgfsetstrokecolor{currentstroke}%
\pgfsetdash{}{0pt}%
\pgfpathmoveto{\pgfqpoint{1.844624in}{0.573423in}}%
\pgfpathlineto{\pgfqpoint{1.884899in}{0.573423in}}%
\pgfpathlineto{\pgfqpoint{1.884899in}{0.576315in}}%
\pgfpathlineto{\pgfqpoint{1.844624in}{0.576315in}}%
\pgfpathlineto{\pgfqpoint{1.844624in}{0.573423in}}%
\pgfusepath{stroke,fill}%
\end{pgfscope}%
\begin{pgfscope}%
\pgfpathrectangle{\pgfqpoint{0.636356in}{0.440955in}}{\pgfqpoint{3.222048in}{0.270720in}} %
\pgfusepath{clip}%
\pgfsetbuttcap%
\pgfsetmiterjoin%
\definecolor{currentfill}{rgb}{0.333333,0.333333,0.333333}%
\pgfsetfillcolor{currentfill}%
\pgfsetlinewidth{0.501875pt}%
\definecolor{currentstroke}{rgb}{0.000000,0.000000,0.000000}%
\pgfsetstrokecolor{currentstroke}%
\pgfsetdash{}{0pt}%
\pgfpathmoveto{\pgfqpoint{1.884899in}{0.576315in}}%
\pgfpathlineto{\pgfqpoint{1.925175in}{0.576315in}}%
\pgfpathlineto{\pgfqpoint{1.925175in}{0.600650in}}%
\pgfpathlineto{\pgfqpoint{1.884899in}{0.600650in}}%
\pgfpathlineto{\pgfqpoint{1.884899in}{0.576315in}}%
\pgfusepath{stroke,fill}%
\end{pgfscope}%
\begin{pgfscope}%
\pgfpathrectangle{\pgfqpoint{0.636356in}{0.440955in}}{\pgfqpoint{3.222048in}{0.270720in}} %
\pgfusepath{clip}%
\pgfsetbuttcap%
\pgfsetmiterjoin%
\definecolor{currentfill}{rgb}{0.333333,0.333333,0.333333}%
\pgfsetfillcolor{currentfill}%
\pgfsetlinewidth{0.501875pt}%
\definecolor{currentstroke}{rgb}{0.000000,0.000000,0.000000}%
\pgfsetstrokecolor{currentstroke}%
\pgfsetdash{}{0pt}%
\pgfpathmoveto{\pgfqpoint{1.925175in}{0.531380in}}%
\pgfpathlineto{\pgfqpoint{1.965451in}{0.531380in}}%
\pgfpathlineto{\pgfqpoint{1.965451in}{0.576315in}}%
\pgfpathlineto{\pgfqpoint{1.925175in}{0.576315in}}%
\pgfpathlineto{\pgfqpoint{1.925175in}{0.531380in}}%
\pgfusepath{stroke,fill}%
\end{pgfscope}%
\begin{pgfscope}%
\pgfpathrectangle{\pgfqpoint{0.636356in}{0.440955in}}{\pgfqpoint{3.222048in}{0.270720in}} %
\pgfusepath{clip}%
\pgfsetbuttcap%
\pgfsetmiterjoin%
\definecolor{currentfill}{rgb}{0.333333,0.333333,0.333333}%
\pgfsetfillcolor{currentfill}%
\pgfsetlinewidth{0.501875pt}%
\definecolor{currentstroke}{rgb}{0.000000,0.000000,0.000000}%
\pgfsetstrokecolor{currentstroke}%
\pgfsetdash{}{0pt}%
\pgfpathmoveto{\pgfqpoint{1.965451in}{0.568875in}}%
\pgfpathlineto{\pgfqpoint{2.005726in}{0.568875in}}%
\pgfpathlineto{\pgfqpoint{2.005726in}{0.576315in}}%
\pgfpathlineto{\pgfqpoint{1.965451in}{0.576315in}}%
\pgfpathlineto{\pgfqpoint{1.965451in}{0.568875in}}%
\pgfusepath{stroke,fill}%
\end{pgfscope}%
\begin{pgfscope}%
\pgfpathrectangle{\pgfqpoint{0.636356in}{0.440955in}}{\pgfqpoint{3.222048in}{0.270720in}} %
\pgfusepath{clip}%
\pgfsetbuttcap%
\pgfsetmiterjoin%
\definecolor{currentfill}{rgb}{0.333333,0.333333,0.333333}%
\pgfsetfillcolor{currentfill}%
\pgfsetlinewidth{0.501875pt}%
\definecolor{currentstroke}{rgb}{0.000000,0.000000,0.000000}%
\pgfsetstrokecolor{currentstroke}%
\pgfsetdash{}{0pt}%
\pgfpathmoveto{\pgfqpoint{2.005726in}{0.576315in}}%
\pgfpathlineto{\pgfqpoint{2.046002in}{0.576315in}}%
\pgfpathlineto{\pgfqpoint{2.046002in}{0.583355in}}%
\pgfpathlineto{\pgfqpoint{2.005726in}{0.583355in}}%
\pgfpathlineto{\pgfqpoint{2.005726in}{0.576315in}}%
\pgfusepath{stroke,fill}%
\end{pgfscope}%
\begin{pgfscope}%
\pgfpathrectangle{\pgfqpoint{0.636356in}{0.440955in}}{\pgfqpoint{3.222048in}{0.270720in}} %
\pgfusepath{clip}%
\pgfsetbuttcap%
\pgfsetmiterjoin%
\definecolor{currentfill}{rgb}{0.333333,0.333333,0.333333}%
\pgfsetfillcolor{currentfill}%
\pgfsetlinewidth{0.501875pt}%
\definecolor{currentstroke}{rgb}{0.000000,0.000000,0.000000}%
\pgfsetstrokecolor{currentstroke}%
\pgfsetdash{}{0pt}%
\pgfpathmoveto{\pgfqpoint{2.046002in}{0.478665in}}%
\pgfpathlineto{\pgfqpoint{2.086277in}{0.478665in}}%
\pgfpathlineto{\pgfqpoint{2.086277in}{0.576315in}}%
\pgfpathlineto{\pgfqpoint{2.046002in}{0.576315in}}%
\pgfpathlineto{\pgfqpoint{2.046002in}{0.478665in}}%
\pgfusepath{stroke,fill}%
\end{pgfscope}%
\begin{pgfscope}%
\pgfpathrectangle{\pgfqpoint{0.636356in}{0.440955in}}{\pgfqpoint{3.222048in}{0.270720in}} %
\pgfusepath{clip}%
\pgfsetbuttcap%
\pgfsetmiterjoin%
\definecolor{currentfill}{rgb}{0.333333,0.333333,0.333333}%
\pgfsetfillcolor{currentfill}%
\pgfsetlinewidth{0.501875pt}%
\definecolor{currentstroke}{rgb}{0.000000,0.000000,0.000000}%
\pgfsetstrokecolor{currentstroke}%
\pgfsetdash{}{0pt}%
\pgfpathmoveto{\pgfqpoint{2.086277in}{0.566810in}}%
\pgfpathlineto{\pgfqpoint{2.126553in}{0.566810in}}%
\pgfpathlineto{\pgfqpoint{2.126553in}{0.576315in}}%
\pgfpathlineto{\pgfqpoint{2.086277in}{0.576315in}}%
\pgfpathlineto{\pgfqpoint{2.086277in}{0.566810in}}%
\pgfusepath{stroke,fill}%
\end{pgfscope}%
\begin{pgfscope}%
\pgfpathrectangle{\pgfqpoint{0.636356in}{0.440955in}}{\pgfqpoint{3.222048in}{0.270720in}} %
\pgfusepath{clip}%
\pgfsetbuttcap%
\pgfsetmiterjoin%
\definecolor{currentfill}{rgb}{0.333333,0.333333,0.333333}%
\pgfsetfillcolor{currentfill}%
\pgfsetlinewidth{0.501875pt}%
\definecolor{currentstroke}{rgb}{0.000000,0.000000,0.000000}%
\pgfsetstrokecolor{currentstroke}%
\pgfsetdash{}{0pt}%
\pgfpathmoveto{\pgfqpoint{2.126553in}{0.566428in}}%
\pgfpathlineto{\pgfqpoint{2.166828in}{0.566428in}}%
\pgfpathlineto{\pgfqpoint{2.166828in}{0.576315in}}%
\pgfpathlineto{\pgfqpoint{2.126553in}{0.576315in}}%
\pgfpathlineto{\pgfqpoint{2.126553in}{0.566428in}}%
\pgfusepath{stroke,fill}%
\end{pgfscope}%
\begin{pgfscope}%
\pgfpathrectangle{\pgfqpoint{0.636356in}{0.440955in}}{\pgfqpoint{3.222048in}{0.270720in}} %
\pgfusepath{clip}%
\pgfsetbuttcap%
\pgfsetmiterjoin%
\definecolor{currentfill}{rgb}{0.333333,0.333333,0.333333}%
\pgfsetfillcolor{currentfill}%
\pgfsetlinewidth{0.501875pt}%
\definecolor{currentstroke}{rgb}{0.000000,0.000000,0.000000}%
\pgfsetstrokecolor{currentstroke}%
\pgfsetdash{}{0pt}%
\pgfpathmoveto{\pgfqpoint{2.166828in}{0.576315in}}%
\pgfpathlineto{\pgfqpoint{2.207104in}{0.576315in}}%
\pgfpathlineto{\pgfqpoint{2.207104in}{0.603728in}}%
\pgfpathlineto{\pgfqpoint{2.166828in}{0.603728in}}%
\pgfpathlineto{\pgfqpoint{2.166828in}{0.576315in}}%
\pgfusepath{stroke,fill}%
\end{pgfscope}%
\begin{pgfscope}%
\pgfpathrectangle{\pgfqpoint{0.636356in}{0.440955in}}{\pgfqpoint{3.222048in}{0.270720in}} %
\pgfusepath{clip}%
\pgfsetbuttcap%
\pgfsetmiterjoin%
\definecolor{currentfill}{rgb}{0.333333,0.333333,0.333333}%
\pgfsetfillcolor{currentfill}%
\pgfsetlinewidth{0.501875pt}%
\definecolor{currentstroke}{rgb}{0.000000,0.000000,0.000000}%
\pgfsetstrokecolor{currentstroke}%
\pgfsetdash{}{0pt}%
\pgfpathmoveto{\pgfqpoint{2.207104in}{0.548146in}}%
\pgfpathlineto{\pgfqpoint{2.247380in}{0.548146in}}%
\pgfpathlineto{\pgfqpoint{2.247380in}{0.576315in}}%
\pgfpathlineto{\pgfqpoint{2.207104in}{0.576315in}}%
\pgfpathlineto{\pgfqpoint{2.207104in}{0.548146in}}%
\pgfusepath{stroke,fill}%
\end{pgfscope}%
\begin{pgfscope}%
\pgfpathrectangle{\pgfqpoint{0.636356in}{0.440955in}}{\pgfqpoint{3.222048in}{0.270720in}} %
\pgfusepath{clip}%
\pgfsetbuttcap%
\pgfsetmiterjoin%
\definecolor{currentfill}{rgb}{0.333333,0.333333,0.333333}%
\pgfsetfillcolor{currentfill}%
\pgfsetlinewidth{0.501875pt}%
\definecolor{currentstroke}{rgb}{0.000000,0.000000,0.000000}%
\pgfsetstrokecolor{currentstroke}%
\pgfsetdash{}{0pt}%
\pgfpathmoveto{\pgfqpoint{2.247380in}{0.576315in}}%
\pgfpathlineto{\pgfqpoint{2.287655in}{0.576315in}}%
\pgfpathlineto{\pgfqpoint{2.287655in}{0.611356in}}%
\pgfpathlineto{\pgfqpoint{2.247380in}{0.611356in}}%
\pgfpathlineto{\pgfqpoint{2.247380in}{0.576315in}}%
\pgfusepath{stroke,fill}%
\end{pgfscope}%
\begin{pgfscope}%
\pgfpathrectangle{\pgfqpoint{0.636356in}{0.440955in}}{\pgfqpoint{3.222048in}{0.270720in}} %
\pgfusepath{clip}%
\pgfsetbuttcap%
\pgfsetmiterjoin%
\definecolor{currentfill}{rgb}{0.333333,0.333333,0.333333}%
\pgfsetfillcolor{currentfill}%
\pgfsetlinewidth{0.501875pt}%
\definecolor{currentstroke}{rgb}{0.000000,0.000000,0.000000}%
\pgfsetstrokecolor{currentstroke}%
\pgfsetdash{}{0pt}%
\pgfpathmoveto{\pgfqpoint{2.287655in}{0.564284in}}%
\pgfpathlineto{\pgfqpoint{2.327931in}{0.564284in}}%
\pgfpathlineto{\pgfqpoint{2.327931in}{0.576315in}}%
\pgfpathlineto{\pgfqpoint{2.287655in}{0.576315in}}%
\pgfpathlineto{\pgfqpoint{2.287655in}{0.564284in}}%
\pgfusepath{stroke,fill}%
\end{pgfscope}%
\begin{pgfscope}%
\pgfpathrectangle{\pgfqpoint{0.636356in}{0.440955in}}{\pgfqpoint{3.222048in}{0.270720in}} %
\pgfusepath{clip}%
\pgfsetbuttcap%
\pgfsetmiterjoin%
\definecolor{currentfill}{rgb}{0.333333,0.333333,0.333333}%
\pgfsetfillcolor{currentfill}%
\pgfsetlinewidth{0.501875pt}%
\definecolor{currentstroke}{rgb}{0.000000,0.000000,0.000000}%
\pgfsetstrokecolor{currentstroke}%
\pgfsetdash{}{0pt}%
\pgfpathmoveto{\pgfqpoint{2.327931in}{0.530485in}}%
\pgfpathlineto{\pgfqpoint{2.368206in}{0.530485in}}%
\pgfpathlineto{\pgfqpoint{2.368206in}{0.576315in}}%
\pgfpathlineto{\pgfqpoint{2.327931in}{0.576315in}}%
\pgfpathlineto{\pgfqpoint{2.327931in}{0.530485in}}%
\pgfusepath{stroke,fill}%
\end{pgfscope}%
\begin{pgfscope}%
\pgfpathrectangle{\pgfqpoint{0.636356in}{0.440955in}}{\pgfqpoint{3.222048in}{0.270720in}} %
\pgfusepath{clip}%
\pgfsetbuttcap%
\pgfsetmiterjoin%
\definecolor{currentfill}{rgb}{0.333333,0.333333,0.333333}%
\pgfsetfillcolor{currentfill}%
\pgfsetlinewidth{0.501875pt}%
\definecolor{currentstroke}{rgb}{0.000000,0.000000,0.000000}%
\pgfsetstrokecolor{currentstroke}%
\pgfsetdash{}{0pt}%
\pgfpathmoveto{\pgfqpoint{2.368206in}{0.570904in}}%
\pgfpathlineto{\pgfqpoint{2.408482in}{0.570904in}}%
\pgfpathlineto{\pgfqpoint{2.408482in}{0.576315in}}%
\pgfpathlineto{\pgfqpoint{2.368206in}{0.576315in}}%
\pgfpathlineto{\pgfqpoint{2.368206in}{0.570904in}}%
\pgfusepath{stroke,fill}%
\end{pgfscope}%
\begin{pgfscope}%
\pgfpathrectangle{\pgfqpoint{0.636356in}{0.440955in}}{\pgfqpoint{3.222048in}{0.270720in}} %
\pgfusepath{clip}%
\pgfsetbuttcap%
\pgfsetmiterjoin%
\definecolor{currentfill}{rgb}{0.333333,0.333333,0.333333}%
\pgfsetfillcolor{currentfill}%
\pgfsetlinewidth{0.501875pt}%
\definecolor{currentstroke}{rgb}{0.000000,0.000000,0.000000}%
\pgfsetstrokecolor{currentstroke}%
\pgfsetdash{}{0pt}%
\pgfpathmoveto{\pgfqpoint{2.408482in}{0.576315in}}%
\pgfpathlineto{\pgfqpoint{2.448758in}{0.576315in}}%
\pgfpathlineto{\pgfqpoint{2.448758in}{0.615538in}}%
\pgfpathlineto{\pgfqpoint{2.408482in}{0.615538in}}%
\pgfpathlineto{\pgfqpoint{2.408482in}{0.576315in}}%
\pgfusepath{stroke,fill}%
\end{pgfscope}%
\begin{pgfscope}%
\pgfpathrectangle{\pgfqpoint{0.636356in}{0.440955in}}{\pgfqpoint{3.222048in}{0.270720in}} %
\pgfusepath{clip}%
\pgfsetbuttcap%
\pgfsetmiterjoin%
\definecolor{currentfill}{rgb}{0.333333,0.333333,0.333333}%
\pgfsetfillcolor{currentfill}%
\pgfsetlinewidth{0.501875pt}%
\definecolor{currentstroke}{rgb}{0.000000,0.000000,0.000000}%
\pgfsetstrokecolor{currentstroke}%
\pgfsetdash{}{0pt}%
\pgfpathmoveto{\pgfqpoint{2.448758in}{0.493008in}}%
\pgfpathlineto{\pgfqpoint{2.489033in}{0.493008in}}%
\pgfpathlineto{\pgfqpoint{2.489033in}{0.576315in}}%
\pgfpathlineto{\pgfqpoint{2.448758in}{0.576315in}}%
\pgfpathlineto{\pgfqpoint{2.448758in}{0.493008in}}%
\pgfusepath{stroke,fill}%
\end{pgfscope}%
\begin{pgfscope}%
\pgfpathrectangle{\pgfqpoint{0.636356in}{0.440955in}}{\pgfqpoint{3.222048in}{0.270720in}} %
\pgfusepath{clip}%
\pgfsetbuttcap%
\pgfsetmiterjoin%
\definecolor{currentfill}{rgb}{0.333333,0.333333,0.333333}%
\pgfsetfillcolor{currentfill}%
\pgfsetlinewidth{0.501875pt}%
\definecolor{currentstroke}{rgb}{0.000000,0.000000,0.000000}%
\pgfsetstrokecolor{currentstroke}%
\pgfsetdash{}{0pt}%
\pgfpathmoveto{\pgfqpoint{2.489033in}{0.576315in}}%
\pgfpathlineto{\pgfqpoint{2.529309in}{0.576315in}}%
\pgfpathlineto{\pgfqpoint{2.529309in}{0.579959in}}%
\pgfpathlineto{\pgfqpoint{2.489033in}{0.579959in}}%
\pgfpathlineto{\pgfqpoint{2.489033in}{0.576315in}}%
\pgfusepath{stroke,fill}%
\end{pgfscope}%
\begin{pgfscope}%
\pgfpathrectangle{\pgfqpoint{0.636356in}{0.440955in}}{\pgfqpoint{3.222048in}{0.270720in}} %
\pgfusepath{clip}%
\pgfsetbuttcap%
\pgfsetmiterjoin%
\definecolor{currentfill}{rgb}{0.333333,0.333333,0.333333}%
\pgfsetfillcolor{currentfill}%
\pgfsetlinewidth{0.501875pt}%
\definecolor{currentstroke}{rgb}{0.000000,0.000000,0.000000}%
\pgfsetstrokecolor{currentstroke}%
\pgfsetdash{}{0pt}%
\pgfpathmoveto{\pgfqpoint{2.529309in}{0.576315in}}%
\pgfpathlineto{\pgfqpoint{2.569584in}{0.576315in}}%
\pgfpathlineto{\pgfqpoint{2.569584in}{0.622825in}}%
\pgfpathlineto{\pgfqpoint{2.529309in}{0.622825in}}%
\pgfpathlineto{\pgfqpoint{2.529309in}{0.576315in}}%
\pgfusepath{stroke,fill}%
\end{pgfscope}%
\begin{pgfscope}%
\pgfpathrectangle{\pgfqpoint{0.636356in}{0.440955in}}{\pgfqpoint{3.222048in}{0.270720in}} %
\pgfusepath{clip}%
\pgfsetbuttcap%
\pgfsetmiterjoin%
\definecolor{currentfill}{rgb}{0.333333,0.333333,0.333333}%
\pgfsetfillcolor{currentfill}%
\pgfsetlinewidth{0.501875pt}%
\definecolor{currentstroke}{rgb}{0.000000,0.000000,0.000000}%
\pgfsetstrokecolor{currentstroke}%
\pgfsetdash{}{0pt}%
\pgfpathmoveto{\pgfqpoint{2.569584in}{0.556001in}}%
\pgfpathlineto{\pgfqpoint{2.609860in}{0.556001in}}%
\pgfpathlineto{\pgfqpoint{2.609860in}{0.576315in}}%
\pgfpathlineto{\pgfqpoint{2.569584in}{0.576315in}}%
\pgfpathlineto{\pgfqpoint{2.569584in}{0.556001in}}%
\pgfusepath{stroke,fill}%
\end{pgfscope}%
\begin{pgfscope}%
\pgfpathrectangle{\pgfqpoint{0.636356in}{0.440955in}}{\pgfqpoint{3.222048in}{0.270720in}} %
\pgfusepath{clip}%
\pgfsetbuttcap%
\pgfsetmiterjoin%
\definecolor{currentfill}{rgb}{0.333333,0.333333,0.333333}%
\pgfsetfillcolor{currentfill}%
\pgfsetlinewidth{0.501875pt}%
\definecolor{currentstroke}{rgb}{0.000000,0.000000,0.000000}%
\pgfsetstrokecolor{currentstroke}%
\pgfsetdash{}{0pt}%
\pgfpathmoveto{\pgfqpoint{2.609860in}{0.537960in}}%
\pgfpathlineto{\pgfqpoint{2.650136in}{0.537960in}}%
\pgfpathlineto{\pgfqpoint{2.650136in}{0.576315in}}%
\pgfpathlineto{\pgfqpoint{2.609860in}{0.576315in}}%
\pgfpathlineto{\pgfqpoint{2.609860in}{0.537960in}}%
\pgfusepath{stroke,fill}%
\end{pgfscope}%
\begin{pgfscope}%
\pgfpathrectangle{\pgfqpoint{0.636356in}{0.440955in}}{\pgfqpoint{3.222048in}{0.270720in}} %
\pgfusepath{clip}%
\pgfsetbuttcap%
\pgfsetmiterjoin%
\definecolor{currentfill}{rgb}{0.333333,0.333333,0.333333}%
\pgfsetfillcolor{currentfill}%
\pgfsetlinewidth{0.501875pt}%
\definecolor{currentstroke}{rgb}{0.000000,0.000000,0.000000}%
\pgfsetstrokecolor{currentstroke}%
\pgfsetdash{}{0pt}%
\pgfpathmoveto{\pgfqpoint{2.650136in}{0.552307in}}%
\pgfpathlineto{\pgfqpoint{2.690411in}{0.552307in}}%
\pgfpathlineto{\pgfqpoint{2.690411in}{0.576315in}}%
\pgfpathlineto{\pgfqpoint{2.650136in}{0.576315in}}%
\pgfpathlineto{\pgfqpoint{2.650136in}{0.552307in}}%
\pgfusepath{stroke,fill}%
\end{pgfscope}%
\begin{pgfscope}%
\pgfpathrectangle{\pgfqpoint{0.636356in}{0.440955in}}{\pgfqpoint{3.222048in}{0.270720in}} %
\pgfusepath{clip}%
\pgfsetbuttcap%
\pgfsetmiterjoin%
\definecolor{currentfill}{rgb}{0.333333,0.333333,0.333333}%
\pgfsetfillcolor{currentfill}%
\pgfsetlinewidth{0.501875pt}%
\definecolor{currentstroke}{rgb}{0.000000,0.000000,0.000000}%
\pgfsetstrokecolor{currentstroke}%
\pgfsetdash{}{0pt}%
\pgfpathmoveto{\pgfqpoint{2.690411in}{0.564505in}}%
\pgfpathlineto{\pgfqpoint{2.730687in}{0.564505in}}%
\pgfpathlineto{\pgfqpoint{2.730687in}{0.576315in}}%
\pgfpathlineto{\pgfqpoint{2.690411in}{0.576315in}}%
\pgfpathlineto{\pgfqpoint{2.690411in}{0.564505in}}%
\pgfusepath{stroke,fill}%
\end{pgfscope}%
\begin{pgfscope}%
\pgfpathrectangle{\pgfqpoint{0.636356in}{0.440955in}}{\pgfqpoint{3.222048in}{0.270720in}} %
\pgfusepath{clip}%
\pgfsetbuttcap%
\pgfsetmiterjoin%
\definecolor{currentfill}{rgb}{0.333333,0.333333,0.333333}%
\pgfsetfillcolor{currentfill}%
\pgfsetlinewidth{0.501875pt}%
\definecolor{currentstroke}{rgb}{0.000000,0.000000,0.000000}%
\pgfsetstrokecolor{currentstroke}%
\pgfsetdash{}{0pt}%
\pgfpathmoveto{\pgfqpoint{2.730687in}{0.574480in}}%
\pgfpathlineto{\pgfqpoint{2.770962in}{0.574480in}}%
\pgfpathlineto{\pgfqpoint{2.770962in}{0.576315in}}%
\pgfpathlineto{\pgfqpoint{2.730687in}{0.576315in}}%
\pgfpathlineto{\pgfqpoint{2.730687in}{0.574480in}}%
\pgfusepath{stroke,fill}%
\end{pgfscope}%
\begin{pgfscope}%
\pgfpathrectangle{\pgfqpoint{0.636356in}{0.440955in}}{\pgfqpoint{3.222048in}{0.270720in}} %
\pgfusepath{clip}%
\pgfsetbuttcap%
\pgfsetmiterjoin%
\definecolor{currentfill}{rgb}{0.333333,0.333333,0.333333}%
\pgfsetfillcolor{currentfill}%
\pgfsetlinewidth{0.501875pt}%
\definecolor{currentstroke}{rgb}{0.000000,0.000000,0.000000}%
\pgfsetstrokecolor{currentstroke}%
\pgfsetdash{}{0pt}%
\pgfpathmoveto{\pgfqpoint{2.770962in}{0.544651in}}%
\pgfpathlineto{\pgfqpoint{2.811238in}{0.544651in}}%
\pgfpathlineto{\pgfqpoint{2.811238in}{0.576315in}}%
\pgfpathlineto{\pgfqpoint{2.770962in}{0.576315in}}%
\pgfpathlineto{\pgfqpoint{2.770962in}{0.544651in}}%
\pgfusepath{stroke,fill}%
\end{pgfscope}%
\begin{pgfscope}%
\pgfpathrectangle{\pgfqpoint{0.636356in}{0.440955in}}{\pgfqpoint{3.222048in}{0.270720in}} %
\pgfusepath{clip}%
\pgfsetbuttcap%
\pgfsetmiterjoin%
\definecolor{currentfill}{rgb}{0.333333,0.333333,0.333333}%
\pgfsetfillcolor{currentfill}%
\pgfsetlinewidth{0.501875pt}%
\definecolor{currentstroke}{rgb}{0.000000,0.000000,0.000000}%
\pgfsetstrokecolor{currentstroke}%
\pgfsetdash{}{0pt}%
\pgfpathmoveto{\pgfqpoint{2.811238in}{0.552210in}}%
\pgfpathlineto{\pgfqpoint{2.851514in}{0.552210in}}%
\pgfpathlineto{\pgfqpoint{2.851514in}{0.576315in}}%
\pgfpathlineto{\pgfqpoint{2.811238in}{0.576315in}}%
\pgfpathlineto{\pgfqpoint{2.811238in}{0.552210in}}%
\pgfusepath{stroke,fill}%
\end{pgfscope}%
\begin{pgfscope}%
\pgfpathrectangle{\pgfqpoint{0.636356in}{0.440955in}}{\pgfqpoint{3.222048in}{0.270720in}} %
\pgfusepath{clip}%
\pgfsetbuttcap%
\pgfsetmiterjoin%
\definecolor{currentfill}{rgb}{0.333333,0.333333,0.333333}%
\pgfsetfillcolor{currentfill}%
\pgfsetlinewidth{0.501875pt}%
\definecolor{currentstroke}{rgb}{0.000000,0.000000,0.000000}%
\pgfsetstrokecolor{currentstroke}%
\pgfsetdash{}{0pt}%
\pgfpathmoveto{\pgfqpoint{2.851514in}{0.576315in}}%
\pgfpathlineto{\pgfqpoint{2.891789in}{0.576315in}}%
\pgfpathlineto{\pgfqpoint{2.891789in}{0.605326in}}%
\pgfpathlineto{\pgfqpoint{2.851514in}{0.605326in}}%
\pgfpathlineto{\pgfqpoint{2.851514in}{0.576315in}}%
\pgfusepath{stroke,fill}%
\end{pgfscope}%
\begin{pgfscope}%
\pgfpathrectangle{\pgfqpoint{0.636356in}{0.440955in}}{\pgfqpoint{3.222048in}{0.270720in}} %
\pgfusepath{clip}%
\pgfsetbuttcap%
\pgfsetmiterjoin%
\definecolor{currentfill}{rgb}{0.333333,0.333333,0.333333}%
\pgfsetfillcolor{currentfill}%
\pgfsetlinewidth{0.501875pt}%
\definecolor{currentstroke}{rgb}{0.000000,0.000000,0.000000}%
\pgfsetstrokecolor{currentstroke}%
\pgfsetdash{}{0pt}%
\pgfpathmoveto{\pgfqpoint{2.891789in}{0.576315in}}%
\pgfpathlineto{\pgfqpoint{2.932065in}{0.576315in}}%
\pgfpathlineto{\pgfqpoint{2.932065in}{0.629332in}}%
\pgfpathlineto{\pgfqpoint{2.891789in}{0.629332in}}%
\pgfpathlineto{\pgfqpoint{2.891789in}{0.576315in}}%
\pgfusepath{stroke,fill}%
\end{pgfscope}%
\begin{pgfscope}%
\pgfpathrectangle{\pgfqpoint{0.636356in}{0.440955in}}{\pgfqpoint{3.222048in}{0.270720in}} %
\pgfusepath{clip}%
\pgfsetbuttcap%
\pgfsetmiterjoin%
\definecolor{currentfill}{rgb}{0.333333,0.333333,0.333333}%
\pgfsetfillcolor{currentfill}%
\pgfsetlinewidth{0.501875pt}%
\definecolor{currentstroke}{rgb}{0.000000,0.000000,0.000000}%
\pgfsetstrokecolor{currentstroke}%
\pgfsetdash{}{0pt}%
\pgfpathmoveto{\pgfqpoint{2.932065in}{0.576315in}}%
\pgfpathlineto{\pgfqpoint{2.972340in}{0.576315in}}%
\pgfpathlineto{\pgfqpoint{2.972340in}{0.589865in}}%
\pgfpathlineto{\pgfqpoint{2.932065in}{0.589865in}}%
\pgfpathlineto{\pgfqpoint{2.932065in}{0.576315in}}%
\pgfusepath{stroke,fill}%
\end{pgfscope}%
\begin{pgfscope}%
\pgfpathrectangle{\pgfqpoint{0.636356in}{0.440955in}}{\pgfqpoint{3.222048in}{0.270720in}} %
\pgfusepath{clip}%
\pgfsetbuttcap%
\pgfsetmiterjoin%
\definecolor{currentfill}{rgb}{0.333333,0.333333,0.333333}%
\pgfsetfillcolor{currentfill}%
\pgfsetlinewidth{0.501875pt}%
\definecolor{currentstroke}{rgb}{0.000000,0.000000,0.000000}%
\pgfsetstrokecolor{currentstroke}%
\pgfsetdash{}{0pt}%
\pgfpathmoveto{\pgfqpoint{2.972340in}{0.576315in}}%
\pgfpathlineto{\pgfqpoint{3.012616in}{0.576315in}}%
\pgfpathlineto{\pgfqpoint{3.012616in}{0.593019in}}%
\pgfpathlineto{\pgfqpoint{2.972340in}{0.593019in}}%
\pgfpathlineto{\pgfqpoint{2.972340in}{0.576315in}}%
\pgfusepath{stroke,fill}%
\end{pgfscope}%
\begin{pgfscope}%
\pgfpathrectangle{\pgfqpoint{0.636356in}{0.440955in}}{\pgfqpoint{3.222048in}{0.270720in}} %
\pgfusepath{clip}%
\pgfsetbuttcap%
\pgfsetmiterjoin%
\definecolor{currentfill}{rgb}{0.333333,0.333333,0.333333}%
\pgfsetfillcolor{currentfill}%
\pgfsetlinewidth{0.501875pt}%
\definecolor{currentstroke}{rgb}{0.000000,0.000000,0.000000}%
\pgfsetstrokecolor{currentstroke}%
\pgfsetdash{}{0pt}%
\pgfpathmoveto{\pgfqpoint{3.012616in}{0.567197in}}%
\pgfpathlineto{\pgfqpoint{3.052892in}{0.567197in}}%
\pgfpathlineto{\pgfqpoint{3.052892in}{0.576315in}}%
\pgfpathlineto{\pgfqpoint{3.012616in}{0.576315in}}%
\pgfpathlineto{\pgfqpoint{3.012616in}{0.567197in}}%
\pgfusepath{stroke,fill}%
\end{pgfscope}%
\begin{pgfscope}%
\pgfpathrectangle{\pgfqpoint{0.636356in}{0.440955in}}{\pgfqpoint{3.222048in}{0.270720in}} %
\pgfusepath{clip}%
\pgfsetbuttcap%
\pgfsetmiterjoin%
\definecolor{currentfill}{rgb}{0.333333,0.333333,0.333333}%
\pgfsetfillcolor{currentfill}%
\pgfsetlinewidth{0.501875pt}%
\definecolor{currentstroke}{rgb}{0.000000,0.000000,0.000000}%
\pgfsetstrokecolor{currentstroke}%
\pgfsetdash{}{0pt}%
\pgfpathmoveto{\pgfqpoint{3.052892in}{0.576315in}}%
\pgfpathlineto{\pgfqpoint{3.093167in}{0.576315in}}%
\pgfpathlineto{\pgfqpoint{3.093167in}{0.650939in}}%
\pgfpathlineto{\pgfqpoint{3.052892in}{0.650939in}}%
\pgfpathlineto{\pgfqpoint{3.052892in}{0.576315in}}%
\pgfusepath{stroke,fill}%
\end{pgfscope}%
\begin{pgfscope}%
\pgfpathrectangle{\pgfqpoint{0.636356in}{0.440955in}}{\pgfqpoint{3.222048in}{0.270720in}} %
\pgfusepath{clip}%
\pgfsetbuttcap%
\pgfsetmiterjoin%
\definecolor{currentfill}{rgb}{0.333333,0.333333,0.333333}%
\pgfsetfillcolor{currentfill}%
\pgfsetlinewidth{0.501875pt}%
\definecolor{currentstroke}{rgb}{0.000000,0.000000,0.000000}%
\pgfsetstrokecolor{currentstroke}%
\pgfsetdash{}{0pt}%
\pgfpathmoveto{\pgfqpoint{3.093167in}{0.576315in}}%
\pgfpathlineto{\pgfqpoint{3.133443in}{0.576315in}}%
\pgfpathlineto{\pgfqpoint{3.133443in}{0.597091in}}%
\pgfpathlineto{\pgfqpoint{3.093167in}{0.597091in}}%
\pgfpathlineto{\pgfqpoint{3.093167in}{0.576315in}}%
\pgfusepath{stroke,fill}%
\end{pgfscope}%
\begin{pgfscope}%
\pgfpathrectangle{\pgfqpoint{0.636356in}{0.440955in}}{\pgfqpoint{3.222048in}{0.270720in}} %
\pgfusepath{clip}%
\pgfsetbuttcap%
\pgfsetmiterjoin%
\definecolor{currentfill}{rgb}{0.333333,0.333333,0.333333}%
\pgfsetfillcolor{currentfill}%
\pgfsetlinewidth{0.501875pt}%
\definecolor{currentstroke}{rgb}{0.000000,0.000000,0.000000}%
\pgfsetstrokecolor{currentstroke}%
\pgfsetdash{}{0pt}%
\pgfpathmoveto{\pgfqpoint{3.133443in}{0.521620in}}%
\pgfpathlineto{\pgfqpoint{3.173718in}{0.521620in}}%
\pgfpathlineto{\pgfqpoint{3.173718in}{0.576315in}}%
\pgfpathlineto{\pgfqpoint{3.133443in}{0.576315in}}%
\pgfpathlineto{\pgfqpoint{3.133443in}{0.521620in}}%
\pgfusepath{stroke,fill}%
\end{pgfscope}%
\begin{pgfscope}%
\pgfpathrectangle{\pgfqpoint{0.636356in}{0.440955in}}{\pgfqpoint{3.222048in}{0.270720in}} %
\pgfusepath{clip}%
\pgfsetbuttcap%
\pgfsetmiterjoin%
\definecolor{currentfill}{rgb}{0.333333,0.333333,0.333333}%
\pgfsetfillcolor{currentfill}%
\pgfsetlinewidth{0.501875pt}%
\definecolor{currentstroke}{rgb}{0.000000,0.000000,0.000000}%
\pgfsetstrokecolor{currentstroke}%
\pgfsetdash{}{0pt}%
\pgfpathmoveto{\pgfqpoint{3.173718in}{0.521946in}}%
\pgfpathlineto{\pgfqpoint{3.213994in}{0.521946in}}%
\pgfpathlineto{\pgfqpoint{3.213994in}{0.576315in}}%
\pgfpathlineto{\pgfqpoint{3.173718in}{0.576315in}}%
\pgfpathlineto{\pgfqpoint{3.173718in}{0.521946in}}%
\pgfusepath{stroke,fill}%
\end{pgfscope}%
\begin{pgfscope}%
\pgfpathrectangle{\pgfqpoint{0.636356in}{0.440955in}}{\pgfqpoint{3.222048in}{0.270720in}} %
\pgfusepath{clip}%
\pgfsetbuttcap%
\pgfsetmiterjoin%
\definecolor{currentfill}{rgb}{0.333333,0.333333,0.333333}%
\pgfsetfillcolor{currentfill}%
\pgfsetlinewidth{0.501875pt}%
\definecolor{currentstroke}{rgb}{0.000000,0.000000,0.000000}%
\pgfsetstrokecolor{currentstroke}%
\pgfsetdash{}{0pt}%
\pgfpathmoveto{\pgfqpoint{3.213994in}{0.568903in}}%
\pgfpathlineto{\pgfqpoint{3.254270in}{0.568903in}}%
\pgfpathlineto{\pgfqpoint{3.254270in}{0.576315in}}%
\pgfpathlineto{\pgfqpoint{3.213994in}{0.576315in}}%
\pgfpathlineto{\pgfqpoint{3.213994in}{0.568903in}}%
\pgfusepath{stroke,fill}%
\end{pgfscope}%
\begin{pgfscope}%
\pgfpathrectangle{\pgfqpoint{0.636356in}{0.440955in}}{\pgfqpoint{3.222048in}{0.270720in}} %
\pgfusepath{clip}%
\pgfsetbuttcap%
\pgfsetmiterjoin%
\definecolor{currentfill}{rgb}{0.333333,0.333333,0.333333}%
\pgfsetfillcolor{currentfill}%
\pgfsetlinewidth{0.501875pt}%
\definecolor{currentstroke}{rgb}{0.000000,0.000000,0.000000}%
\pgfsetstrokecolor{currentstroke}%
\pgfsetdash{}{0pt}%
\pgfpathmoveto{\pgfqpoint{3.254270in}{0.568976in}}%
\pgfpathlineto{\pgfqpoint{3.294545in}{0.568976in}}%
\pgfpathlineto{\pgfqpoint{3.294545in}{0.576315in}}%
\pgfpathlineto{\pgfqpoint{3.254270in}{0.576315in}}%
\pgfpathlineto{\pgfqpoint{3.254270in}{0.568976in}}%
\pgfusepath{stroke,fill}%
\end{pgfscope}%
\begin{pgfscope}%
\pgfpathrectangle{\pgfqpoint{0.636356in}{0.440955in}}{\pgfqpoint{3.222048in}{0.270720in}} %
\pgfusepath{clip}%
\pgfsetbuttcap%
\pgfsetmiterjoin%
\definecolor{currentfill}{rgb}{0.333333,0.333333,0.333333}%
\pgfsetfillcolor{currentfill}%
\pgfsetlinewidth{0.501875pt}%
\definecolor{currentstroke}{rgb}{0.000000,0.000000,0.000000}%
\pgfsetstrokecolor{currentstroke}%
\pgfsetdash{}{0pt}%
\pgfpathmoveto{\pgfqpoint{3.294545in}{0.576315in}}%
\pgfpathlineto{\pgfqpoint{3.334821in}{0.576315in}}%
\pgfpathlineto{\pgfqpoint{3.334821in}{0.598275in}}%
\pgfpathlineto{\pgfqpoint{3.294545in}{0.598275in}}%
\pgfpathlineto{\pgfqpoint{3.294545in}{0.576315in}}%
\pgfusepath{stroke,fill}%
\end{pgfscope}%
\begin{pgfscope}%
\pgfpathrectangle{\pgfqpoint{0.636356in}{0.440955in}}{\pgfqpoint{3.222048in}{0.270720in}} %
\pgfusepath{clip}%
\pgfsetbuttcap%
\pgfsetmiterjoin%
\definecolor{currentfill}{rgb}{0.333333,0.333333,0.333333}%
\pgfsetfillcolor{currentfill}%
\pgfsetlinewidth{0.501875pt}%
\definecolor{currentstroke}{rgb}{0.000000,0.000000,0.000000}%
\pgfsetstrokecolor{currentstroke}%
\pgfsetdash{}{0pt}%
\pgfpathmoveto{\pgfqpoint{3.334821in}{0.522503in}}%
\pgfpathlineto{\pgfqpoint{3.375096in}{0.522503in}}%
\pgfpathlineto{\pgfqpoint{3.375096in}{0.576315in}}%
\pgfpathlineto{\pgfqpoint{3.334821in}{0.576315in}}%
\pgfpathlineto{\pgfqpoint{3.334821in}{0.522503in}}%
\pgfusepath{stroke,fill}%
\end{pgfscope}%
\begin{pgfscope}%
\pgfpathrectangle{\pgfqpoint{0.636356in}{0.440955in}}{\pgfqpoint{3.222048in}{0.270720in}} %
\pgfusepath{clip}%
\pgfsetbuttcap%
\pgfsetmiterjoin%
\definecolor{currentfill}{rgb}{0.333333,0.333333,0.333333}%
\pgfsetfillcolor{currentfill}%
\pgfsetlinewidth{0.501875pt}%
\definecolor{currentstroke}{rgb}{0.000000,0.000000,0.000000}%
\pgfsetstrokecolor{currentstroke}%
\pgfsetdash{}{0pt}%
\pgfpathmoveto{\pgfqpoint{3.375096in}{0.569115in}}%
\pgfpathlineto{\pgfqpoint{3.415372in}{0.569115in}}%
\pgfpathlineto{\pgfqpoint{3.415372in}{0.576315in}}%
\pgfpathlineto{\pgfqpoint{3.375096in}{0.576315in}}%
\pgfpathlineto{\pgfqpoint{3.375096in}{0.569115in}}%
\pgfusepath{stroke,fill}%
\end{pgfscope}%
\begin{pgfscope}%
\pgfpathrectangle{\pgfqpoint{0.636356in}{0.440955in}}{\pgfqpoint{3.222048in}{0.270720in}} %
\pgfusepath{clip}%
\pgfsetbuttcap%
\pgfsetmiterjoin%
\definecolor{currentfill}{rgb}{0.333333,0.333333,0.333333}%
\pgfsetfillcolor{currentfill}%
\pgfsetlinewidth{0.501875pt}%
\definecolor{currentstroke}{rgb}{0.000000,0.000000,0.000000}%
\pgfsetstrokecolor{currentstroke}%
\pgfsetdash{}{0pt}%
\pgfpathmoveto{\pgfqpoint{3.415372in}{0.522658in}}%
\pgfpathlineto{\pgfqpoint{3.455648in}{0.522658in}}%
\pgfpathlineto{\pgfqpoint{3.455648in}{0.576315in}}%
\pgfpathlineto{\pgfqpoint{3.415372in}{0.576315in}}%
\pgfpathlineto{\pgfqpoint{3.415372in}{0.522658in}}%
\pgfusepath{stroke,fill}%
\end{pgfscope}%
\begin{pgfscope}%
\pgfpathrectangle{\pgfqpoint{0.636356in}{0.440955in}}{\pgfqpoint{3.222048in}{0.270720in}} %
\pgfusepath{clip}%
\pgfsetbuttcap%
\pgfsetmiterjoin%
\definecolor{currentfill}{rgb}{0.333333,0.333333,0.333333}%
\pgfsetfillcolor{currentfill}%
\pgfsetlinewidth{0.501875pt}%
\definecolor{currentstroke}{rgb}{0.000000,0.000000,0.000000}%
\pgfsetstrokecolor{currentstroke}%
\pgfsetdash{}{0pt}%
\pgfpathmoveto{\pgfqpoint{3.455648in}{0.576315in}}%
\pgfpathlineto{\pgfqpoint{3.495923in}{0.576315in}}%
\pgfpathlineto{\pgfqpoint{3.495923in}{0.598556in}}%
\pgfpathlineto{\pgfqpoint{3.455648in}{0.598556in}}%
\pgfpathlineto{\pgfqpoint{3.455648in}{0.576315in}}%
\pgfusepath{stroke,fill}%
\end{pgfscope}%
\begin{pgfscope}%
\pgfpathrectangle{\pgfqpoint{0.636356in}{0.440955in}}{\pgfqpoint{3.222048in}{0.270720in}} %
\pgfusepath{clip}%
\pgfsetbuttcap%
\pgfsetmiterjoin%
\definecolor{currentfill}{rgb}{0.333333,0.333333,0.333333}%
\pgfsetfillcolor{currentfill}%
\pgfsetlinewidth{0.501875pt}%
\definecolor{currentstroke}{rgb}{0.000000,0.000000,0.000000}%
\pgfsetstrokecolor{currentstroke}%
\pgfsetdash{}{0pt}%
\pgfpathmoveto{\pgfqpoint{3.495923in}{0.569224in}}%
\pgfpathlineto{\pgfqpoint{3.536199in}{0.569224in}}%
\pgfpathlineto{\pgfqpoint{3.536199in}{0.576315in}}%
\pgfpathlineto{\pgfqpoint{3.495923in}{0.576315in}}%
\pgfpathlineto{\pgfqpoint{3.495923in}{0.569224in}}%
\pgfusepath{stroke,fill}%
\end{pgfscope}%
\begin{pgfscope}%
\pgfpathrectangle{\pgfqpoint{0.636356in}{0.440955in}}{\pgfqpoint{3.222048in}{0.270720in}} %
\pgfusepath{clip}%
\pgfsetbuttcap%
\pgfsetmiterjoin%
\definecolor{currentfill}{rgb}{0.333333,0.333333,0.333333}%
\pgfsetfillcolor{currentfill}%
\pgfsetlinewidth{0.501875pt}%
\definecolor{currentstroke}{rgb}{0.000000,0.000000,0.000000}%
\pgfsetstrokecolor{currentstroke}%
\pgfsetdash{}{0pt}%
\pgfpathmoveto{\pgfqpoint{3.536199in}{0.569259in}}%
\pgfpathlineto{\pgfqpoint{3.576474in}{0.569259in}}%
\pgfpathlineto{\pgfqpoint{3.576474in}{0.576315in}}%
\pgfpathlineto{\pgfqpoint{3.536199in}{0.576315in}}%
\pgfpathlineto{\pgfqpoint{3.536199in}{0.569259in}}%
\pgfusepath{stroke,fill}%
\end{pgfscope}%
\begin{pgfscope}%
\pgfpathrectangle{\pgfqpoint{0.636356in}{0.440955in}}{\pgfqpoint{3.222048in}{0.270720in}} %
\pgfusepath{clip}%
\pgfsetbuttcap%
\pgfsetmiterjoin%
\definecolor{currentfill}{rgb}{0.333333,0.333333,0.333333}%
\pgfsetfillcolor{currentfill}%
\pgfsetlinewidth{0.501875pt}%
\definecolor{currentstroke}{rgb}{0.000000,0.000000,0.000000}%
\pgfsetstrokecolor{currentstroke}%
\pgfsetdash{}{0pt}%
\pgfpathmoveto{\pgfqpoint{3.576474in}{0.569294in}}%
\pgfpathlineto{\pgfqpoint{3.616750in}{0.569294in}}%
\pgfpathlineto{\pgfqpoint{3.616750in}{0.576315in}}%
\pgfpathlineto{\pgfqpoint{3.576474in}{0.576315in}}%
\pgfpathlineto{\pgfqpoint{3.576474in}{0.569294in}}%
\pgfusepath{stroke,fill}%
\end{pgfscope}%
\begin{pgfscope}%
\pgfpathrectangle{\pgfqpoint{0.636356in}{0.440955in}}{\pgfqpoint{3.222048in}{0.270720in}} %
\pgfusepath{clip}%
\pgfsetbuttcap%
\pgfsetmiterjoin%
\definecolor{currentfill}{rgb}{0.333333,0.333333,0.333333}%
\pgfsetfillcolor{currentfill}%
\pgfsetlinewidth{0.501875pt}%
\definecolor{currentstroke}{rgb}{0.000000,0.000000,0.000000}%
\pgfsetstrokecolor{currentstroke}%
\pgfsetdash{}{0pt}%
\pgfpathmoveto{\pgfqpoint{3.616750in}{0.576315in}}%
\pgfpathlineto{\pgfqpoint{3.657026in}{0.576315in}}%
\pgfpathlineto{\pgfqpoint{3.657026in}{0.621747in}}%
\pgfpathlineto{\pgfqpoint{3.616750in}{0.621747in}}%
\pgfpathlineto{\pgfqpoint{3.616750in}{0.576315in}}%
\pgfusepath{stroke,fill}%
\end{pgfscope}%
\begin{pgfscope}%
\pgfpathrectangle{\pgfqpoint{0.636356in}{0.440955in}}{\pgfqpoint{3.222048in}{0.270720in}} %
\pgfusepath{clip}%
\pgfsetbuttcap%
\pgfsetmiterjoin%
\definecolor{currentfill}{rgb}{0.333333,0.333333,0.333333}%
\pgfsetfillcolor{currentfill}%
\pgfsetlinewidth{0.501875pt}%
\definecolor{currentstroke}{rgb}{0.000000,0.000000,0.000000}%
\pgfsetstrokecolor{currentstroke}%
\pgfsetdash{}{0pt}%
\pgfpathmoveto{\pgfqpoint{3.657026in}{0.523084in}}%
\pgfpathlineto{\pgfqpoint{3.697301in}{0.523084in}}%
\pgfpathlineto{\pgfqpoint{3.697301in}{0.576315in}}%
\pgfpathlineto{\pgfqpoint{3.657026in}{0.576315in}}%
\pgfpathlineto{\pgfqpoint{3.657026in}{0.523084in}}%
\pgfusepath{stroke,fill}%
\end{pgfscope}%
\begin{pgfscope}%
\pgfpathrectangle{\pgfqpoint{0.636356in}{0.440955in}}{\pgfqpoint{3.222048in}{0.270720in}} %
\pgfusepath{clip}%
\pgfsetbuttcap%
\pgfsetmiterjoin%
\definecolor{currentfill}{rgb}{0.333333,0.333333,0.333333}%
\pgfsetfillcolor{currentfill}%
\pgfsetlinewidth{0.501875pt}%
\definecolor{currentstroke}{rgb}{0.000000,0.000000,0.000000}%
\pgfsetstrokecolor{currentstroke}%
\pgfsetdash{}{0pt}%
\pgfpathmoveto{\pgfqpoint{3.697301in}{0.523154in}}%
\pgfpathlineto{\pgfqpoint{3.737577in}{0.523154in}}%
\pgfpathlineto{\pgfqpoint{3.737577in}{0.576315in}}%
\pgfpathlineto{\pgfqpoint{3.697301in}{0.576315in}}%
\pgfpathlineto{\pgfqpoint{3.697301in}{0.523154in}}%
\pgfusepath{stroke,fill}%
\end{pgfscope}%
\begin{pgfscope}%
\pgfpathrectangle{\pgfqpoint{0.636356in}{0.440955in}}{\pgfqpoint{3.222048in}{0.270720in}} %
\pgfusepath{clip}%
\pgfsetbuttcap%
\pgfsetmiterjoin%
\definecolor{currentfill}{rgb}{0.333333,0.333333,0.333333}%
\pgfsetfillcolor{currentfill}%
\pgfsetlinewidth{0.501875pt}%
\definecolor{currentstroke}{rgb}{0.000000,0.000000,0.000000}%
\pgfsetstrokecolor{currentstroke}%
\pgfsetdash{}{0pt}%
\pgfpathmoveto{\pgfqpoint{3.737577in}{0.523224in}}%
\pgfpathlineto{\pgfqpoint{3.777852in}{0.523224in}}%
\pgfpathlineto{\pgfqpoint{3.777852in}{0.576315in}}%
\pgfpathlineto{\pgfqpoint{3.737577in}{0.576315in}}%
\pgfpathlineto{\pgfqpoint{3.737577in}{0.523224in}}%
\pgfusepath{stroke,fill}%
\end{pgfscope}%
\begin{pgfscope}%
\pgfpathrectangle{\pgfqpoint{0.636356in}{0.440955in}}{\pgfqpoint{3.222048in}{0.270720in}} %
\pgfusepath{clip}%
\pgfsetbuttcap%
\pgfsetmiterjoin%
\definecolor{currentfill}{rgb}{0.333333,0.333333,0.333333}%
\pgfsetfillcolor{currentfill}%
\pgfsetlinewidth{0.501875pt}%
\definecolor{currentstroke}{rgb}{0.000000,0.000000,0.000000}%
\pgfsetstrokecolor{currentstroke}%
\pgfsetdash{}{0pt}%
\pgfpathmoveto{\pgfqpoint{3.777852in}{0.576315in}}%
\pgfpathlineto{\pgfqpoint{3.818128in}{0.576315in}}%
\pgfpathlineto{\pgfqpoint{3.818128in}{0.599057in}}%
\pgfpathlineto{\pgfqpoint{3.777852in}{0.599057in}}%
\pgfpathlineto{\pgfqpoint{3.777852in}{0.576315in}}%
\pgfusepath{stroke,fill}%
\end{pgfscope}%
\begin{pgfscope}%
\pgfpathrectangle{\pgfqpoint{0.636356in}{0.440955in}}{\pgfqpoint{3.222048in}{0.270720in}} %
\pgfusepath{clip}%
\pgfsetbuttcap%
\pgfsetmiterjoin%
\definecolor{currentfill}{rgb}{0.333333,0.333333,0.333333}%
\pgfsetfillcolor{currentfill}%
\pgfsetlinewidth{0.501875pt}%
\definecolor{currentstroke}{rgb}{0.000000,0.000000,0.000000}%
\pgfsetstrokecolor{currentstroke}%
\pgfsetdash{}{0pt}%
\pgfpathmoveto{\pgfqpoint{3.818128in}{0.576315in}}%
\pgfpathlineto{\pgfqpoint{3.858404in}{0.576315in}}%
\pgfpathlineto{\pgfqpoint{3.858404in}{0.599119in}}%
\pgfpathlineto{\pgfqpoint{3.818128in}{0.599119in}}%
\pgfpathlineto{\pgfqpoint{3.818128in}{0.576315in}}%
\pgfusepath{stroke,fill}%
\end{pgfscope}%
\begin{pgfscope}%
\pgfsetrectcap%
\pgfsetmiterjoin%
\pgfsetlinewidth{1.003750pt}%
\definecolor{currentstroke}{rgb}{0.000000,0.000000,0.000000}%
\pgfsetstrokecolor{currentstroke}%
\pgfsetdash{}{0pt}%
\pgfpathmoveto{\pgfqpoint{0.636356in}{0.711675in}}%
\pgfpathlineto{\pgfqpoint{3.858404in}{0.711675in}}%
\pgfusepath{stroke}%
\end{pgfscope}%
\begin{pgfscope}%
\pgfsetrectcap%
\pgfsetmiterjoin%
\pgfsetlinewidth{1.003750pt}%
\definecolor{currentstroke}{rgb}{0.000000,0.000000,0.000000}%
\pgfsetstrokecolor{currentstroke}%
\pgfsetdash{}{0pt}%
\pgfpathmoveto{\pgfqpoint{3.858404in}{0.440955in}}%
\pgfpathlineto{\pgfqpoint{3.858404in}{0.711675in}}%
\pgfusepath{stroke}%
\end{pgfscope}%
\begin{pgfscope}%
\pgfsetrectcap%
\pgfsetmiterjoin%
\pgfsetlinewidth{1.003750pt}%
\definecolor{currentstroke}{rgb}{0.000000,0.000000,0.000000}%
\pgfsetstrokecolor{currentstroke}%
\pgfsetdash{}{0pt}%
\pgfpathmoveto{\pgfqpoint{0.636356in}{0.440955in}}%
\pgfpathlineto{\pgfqpoint{3.858404in}{0.440955in}}%
\pgfusepath{stroke}%
\end{pgfscope}%
\begin{pgfscope}%
\pgfsetrectcap%
\pgfsetmiterjoin%
\pgfsetlinewidth{1.003750pt}%
\definecolor{currentstroke}{rgb}{0.000000,0.000000,0.000000}%
\pgfsetstrokecolor{currentstroke}%
\pgfsetdash{}{0pt}%
\pgfpathmoveto{\pgfqpoint{0.636356in}{0.440955in}}%
\pgfpathlineto{\pgfqpoint{0.636356in}{0.711675in}}%
\pgfusepath{stroke}%
\end{pgfscope}%
\begin{pgfscope}%
\pgfsetbuttcap%
\pgfsetroundjoin%
\definecolor{currentfill}{rgb}{0.000000,0.000000,0.000000}%
\pgfsetfillcolor{currentfill}%
\pgfsetlinewidth{0.501875pt}%
\definecolor{currentstroke}{rgb}{0.000000,0.000000,0.000000}%
\pgfsetstrokecolor{currentstroke}%
\pgfsetdash{}{0pt}%
\pgfsys@defobject{currentmarker}{\pgfqpoint{0.000000in}{0.000000in}}{\pgfqpoint{0.000000in}{0.069444in}}{%
\pgfpathmoveto{\pgfqpoint{0.000000in}{0.000000in}}%
\pgfpathlineto{\pgfqpoint{0.000000in}{0.069444in}}%
\pgfusepath{stroke,fill}%
}%
\begin{pgfscope}%
\pgfsys@transformshift{0.636356in}{0.440955in}%
\pgfsys@useobject{currentmarker}{}%
\end{pgfscope}%
\end{pgfscope}%
\begin{pgfscope}%
\pgfsetbuttcap%
\pgfsetroundjoin%
\definecolor{currentfill}{rgb}{0.000000,0.000000,0.000000}%
\pgfsetfillcolor{currentfill}%
\pgfsetlinewidth{0.501875pt}%
\definecolor{currentstroke}{rgb}{0.000000,0.000000,0.000000}%
\pgfsetstrokecolor{currentstroke}%
\pgfsetdash{}{0pt}%
\pgfsys@defobject{currentmarker}{\pgfqpoint{0.000000in}{-0.069444in}}{\pgfqpoint{0.000000in}{0.000000in}}{%
\pgfpathmoveto{\pgfqpoint{0.000000in}{0.000000in}}%
\pgfpathlineto{\pgfqpoint{0.000000in}{-0.069444in}}%
\pgfusepath{stroke,fill}%
}%
\begin{pgfscope}%
\pgfsys@transformshift{0.636356in}{0.711675in}%
\pgfsys@useobject{currentmarker}{}%
\end{pgfscope}%
\end{pgfscope}%
\begin{pgfscope}%
\pgftext[x=0.636356in,y=0.371511in,,top]{\rmfamily\fontsize{8.000000}{9.600000}\selectfont 1800}%
\end{pgfscope}%
\begin{pgfscope}%
\pgfsetbuttcap%
\pgfsetroundjoin%
\definecolor{currentfill}{rgb}{0.000000,0.000000,0.000000}%
\pgfsetfillcolor{currentfill}%
\pgfsetlinewidth{0.501875pt}%
\definecolor{currentstroke}{rgb}{0.000000,0.000000,0.000000}%
\pgfsetstrokecolor{currentstroke}%
\pgfsetdash{}{0pt}%
\pgfsys@defobject{currentmarker}{\pgfqpoint{0.000000in}{0.000000in}}{\pgfqpoint{0.000000in}{0.069444in}}{%
\pgfpathmoveto{\pgfqpoint{0.000000in}{0.000000in}}%
\pgfpathlineto{\pgfqpoint{0.000000in}{0.069444in}}%
\pgfusepath{stroke,fill}%
}%
\begin{pgfscope}%
\pgfsys@transformshift{1.132055in}{0.440955in}%
\pgfsys@useobject{currentmarker}{}%
\end{pgfscope}%
\end{pgfscope}%
\begin{pgfscope}%
\pgfsetbuttcap%
\pgfsetroundjoin%
\definecolor{currentfill}{rgb}{0.000000,0.000000,0.000000}%
\pgfsetfillcolor{currentfill}%
\pgfsetlinewidth{0.501875pt}%
\definecolor{currentstroke}{rgb}{0.000000,0.000000,0.000000}%
\pgfsetstrokecolor{currentstroke}%
\pgfsetdash{}{0pt}%
\pgfsys@defobject{currentmarker}{\pgfqpoint{0.000000in}{-0.069444in}}{\pgfqpoint{0.000000in}{0.000000in}}{%
\pgfpathmoveto{\pgfqpoint{0.000000in}{0.000000in}}%
\pgfpathlineto{\pgfqpoint{0.000000in}{-0.069444in}}%
\pgfusepath{stroke,fill}%
}%
\begin{pgfscope}%
\pgfsys@transformshift{1.132055in}{0.711675in}%
\pgfsys@useobject{currentmarker}{}%
\end{pgfscope}%
\end{pgfscope}%
\begin{pgfscope}%
\pgftext[x=1.132055in,y=0.371511in,,top]{\rmfamily\fontsize{8.000000}{9.600000}\selectfont 1820}%
\end{pgfscope}%
\begin{pgfscope}%
\pgfsetbuttcap%
\pgfsetroundjoin%
\definecolor{currentfill}{rgb}{0.000000,0.000000,0.000000}%
\pgfsetfillcolor{currentfill}%
\pgfsetlinewidth{0.501875pt}%
\definecolor{currentstroke}{rgb}{0.000000,0.000000,0.000000}%
\pgfsetstrokecolor{currentstroke}%
\pgfsetdash{}{0pt}%
\pgfsys@defobject{currentmarker}{\pgfqpoint{0.000000in}{0.000000in}}{\pgfqpoint{0.000000in}{0.069444in}}{%
\pgfpathmoveto{\pgfqpoint{0.000000in}{0.000000in}}%
\pgfpathlineto{\pgfqpoint{0.000000in}{0.069444in}}%
\pgfusepath{stroke,fill}%
}%
\begin{pgfscope}%
\pgfsys@transformshift{1.627755in}{0.440955in}%
\pgfsys@useobject{currentmarker}{}%
\end{pgfscope}%
\end{pgfscope}%
\begin{pgfscope}%
\pgfsetbuttcap%
\pgfsetroundjoin%
\definecolor{currentfill}{rgb}{0.000000,0.000000,0.000000}%
\pgfsetfillcolor{currentfill}%
\pgfsetlinewidth{0.501875pt}%
\definecolor{currentstroke}{rgb}{0.000000,0.000000,0.000000}%
\pgfsetstrokecolor{currentstroke}%
\pgfsetdash{}{0pt}%
\pgfsys@defobject{currentmarker}{\pgfqpoint{0.000000in}{-0.069444in}}{\pgfqpoint{0.000000in}{0.000000in}}{%
\pgfpathmoveto{\pgfqpoint{0.000000in}{0.000000in}}%
\pgfpathlineto{\pgfqpoint{0.000000in}{-0.069444in}}%
\pgfusepath{stroke,fill}%
}%
\begin{pgfscope}%
\pgfsys@transformshift{1.627755in}{0.711675in}%
\pgfsys@useobject{currentmarker}{}%
\end{pgfscope}%
\end{pgfscope}%
\begin{pgfscope}%
\pgftext[x=1.627755in,y=0.371511in,,top]{\rmfamily\fontsize{8.000000}{9.600000}\selectfont 1840}%
\end{pgfscope}%
\begin{pgfscope}%
\pgfsetbuttcap%
\pgfsetroundjoin%
\definecolor{currentfill}{rgb}{0.000000,0.000000,0.000000}%
\pgfsetfillcolor{currentfill}%
\pgfsetlinewidth{0.501875pt}%
\definecolor{currentstroke}{rgb}{0.000000,0.000000,0.000000}%
\pgfsetstrokecolor{currentstroke}%
\pgfsetdash{}{0pt}%
\pgfsys@defobject{currentmarker}{\pgfqpoint{0.000000in}{0.000000in}}{\pgfqpoint{0.000000in}{0.069444in}}{%
\pgfpathmoveto{\pgfqpoint{0.000000in}{0.000000in}}%
\pgfpathlineto{\pgfqpoint{0.000000in}{0.069444in}}%
\pgfusepath{stroke,fill}%
}%
\begin{pgfscope}%
\pgfsys@transformshift{2.123455in}{0.440955in}%
\pgfsys@useobject{currentmarker}{}%
\end{pgfscope}%
\end{pgfscope}%
\begin{pgfscope}%
\pgfsetbuttcap%
\pgfsetroundjoin%
\definecolor{currentfill}{rgb}{0.000000,0.000000,0.000000}%
\pgfsetfillcolor{currentfill}%
\pgfsetlinewidth{0.501875pt}%
\definecolor{currentstroke}{rgb}{0.000000,0.000000,0.000000}%
\pgfsetstrokecolor{currentstroke}%
\pgfsetdash{}{0pt}%
\pgfsys@defobject{currentmarker}{\pgfqpoint{0.000000in}{-0.069444in}}{\pgfqpoint{0.000000in}{0.000000in}}{%
\pgfpathmoveto{\pgfqpoint{0.000000in}{0.000000in}}%
\pgfpathlineto{\pgfqpoint{0.000000in}{-0.069444in}}%
\pgfusepath{stroke,fill}%
}%
\begin{pgfscope}%
\pgfsys@transformshift{2.123455in}{0.711675in}%
\pgfsys@useobject{currentmarker}{}%
\end{pgfscope}%
\end{pgfscope}%
\begin{pgfscope}%
\pgftext[x=2.123455in,y=0.371511in,,top]{\rmfamily\fontsize{8.000000}{9.600000}\selectfont 1860}%
\end{pgfscope}%
\begin{pgfscope}%
\pgfsetbuttcap%
\pgfsetroundjoin%
\definecolor{currentfill}{rgb}{0.000000,0.000000,0.000000}%
\pgfsetfillcolor{currentfill}%
\pgfsetlinewidth{0.501875pt}%
\definecolor{currentstroke}{rgb}{0.000000,0.000000,0.000000}%
\pgfsetstrokecolor{currentstroke}%
\pgfsetdash{}{0pt}%
\pgfsys@defobject{currentmarker}{\pgfqpoint{0.000000in}{0.000000in}}{\pgfqpoint{0.000000in}{0.069444in}}{%
\pgfpathmoveto{\pgfqpoint{0.000000in}{0.000000in}}%
\pgfpathlineto{\pgfqpoint{0.000000in}{0.069444in}}%
\pgfusepath{stroke,fill}%
}%
\begin{pgfscope}%
\pgfsys@transformshift{2.619154in}{0.440955in}%
\pgfsys@useobject{currentmarker}{}%
\end{pgfscope}%
\end{pgfscope}%
\begin{pgfscope}%
\pgfsetbuttcap%
\pgfsetroundjoin%
\definecolor{currentfill}{rgb}{0.000000,0.000000,0.000000}%
\pgfsetfillcolor{currentfill}%
\pgfsetlinewidth{0.501875pt}%
\definecolor{currentstroke}{rgb}{0.000000,0.000000,0.000000}%
\pgfsetstrokecolor{currentstroke}%
\pgfsetdash{}{0pt}%
\pgfsys@defobject{currentmarker}{\pgfqpoint{0.000000in}{-0.069444in}}{\pgfqpoint{0.000000in}{0.000000in}}{%
\pgfpathmoveto{\pgfqpoint{0.000000in}{0.000000in}}%
\pgfpathlineto{\pgfqpoint{0.000000in}{-0.069444in}}%
\pgfusepath{stroke,fill}%
}%
\begin{pgfscope}%
\pgfsys@transformshift{2.619154in}{0.711675in}%
\pgfsys@useobject{currentmarker}{}%
\end{pgfscope}%
\end{pgfscope}%
\begin{pgfscope}%
\pgftext[x=2.619154in,y=0.371511in,,top]{\rmfamily\fontsize{8.000000}{9.600000}\selectfont 1880}%
\end{pgfscope}%
\begin{pgfscope}%
\pgfsetbuttcap%
\pgfsetroundjoin%
\definecolor{currentfill}{rgb}{0.000000,0.000000,0.000000}%
\pgfsetfillcolor{currentfill}%
\pgfsetlinewidth{0.501875pt}%
\definecolor{currentstroke}{rgb}{0.000000,0.000000,0.000000}%
\pgfsetstrokecolor{currentstroke}%
\pgfsetdash{}{0pt}%
\pgfsys@defobject{currentmarker}{\pgfqpoint{0.000000in}{0.000000in}}{\pgfqpoint{0.000000in}{0.069444in}}{%
\pgfpathmoveto{\pgfqpoint{0.000000in}{0.000000in}}%
\pgfpathlineto{\pgfqpoint{0.000000in}{0.069444in}}%
\pgfusepath{stroke,fill}%
}%
\begin{pgfscope}%
\pgfsys@transformshift{3.114854in}{0.440955in}%
\pgfsys@useobject{currentmarker}{}%
\end{pgfscope}%
\end{pgfscope}%
\begin{pgfscope}%
\pgfsetbuttcap%
\pgfsetroundjoin%
\definecolor{currentfill}{rgb}{0.000000,0.000000,0.000000}%
\pgfsetfillcolor{currentfill}%
\pgfsetlinewidth{0.501875pt}%
\definecolor{currentstroke}{rgb}{0.000000,0.000000,0.000000}%
\pgfsetstrokecolor{currentstroke}%
\pgfsetdash{}{0pt}%
\pgfsys@defobject{currentmarker}{\pgfqpoint{0.000000in}{-0.069444in}}{\pgfqpoint{0.000000in}{0.000000in}}{%
\pgfpathmoveto{\pgfqpoint{0.000000in}{0.000000in}}%
\pgfpathlineto{\pgfqpoint{0.000000in}{-0.069444in}}%
\pgfusepath{stroke,fill}%
}%
\begin{pgfscope}%
\pgfsys@transformshift{3.114854in}{0.711675in}%
\pgfsys@useobject{currentmarker}{}%
\end{pgfscope}%
\end{pgfscope}%
\begin{pgfscope}%
\pgftext[x=3.114854in,y=0.371511in,,top]{\rmfamily\fontsize{8.000000}{9.600000}\selectfont 1900}%
\end{pgfscope}%
\begin{pgfscope}%
\pgfsetbuttcap%
\pgfsetroundjoin%
\definecolor{currentfill}{rgb}{0.000000,0.000000,0.000000}%
\pgfsetfillcolor{currentfill}%
\pgfsetlinewidth{0.501875pt}%
\definecolor{currentstroke}{rgb}{0.000000,0.000000,0.000000}%
\pgfsetstrokecolor{currentstroke}%
\pgfsetdash{}{0pt}%
\pgfsys@defobject{currentmarker}{\pgfqpoint{0.000000in}{0.000000in}}{\pgfqpoint{0.000000in}{0.069444in}}{%
\pgfpathmoveto{\pgfqpoint{0.000000in}{0.000000in}}%
\pgfpathlineto{\pgfqpoint{0.000000in}{0.069444in}}%
\pgfusepath{stroke,fill}%
}%
\begin{pgfscope}%
\pgfsys@transformshift{3.610554in}{0.440955in}%
\pgfsys@useobject{currentmarker}{}%
\end{pgfscope}%
\end{pgfscope}%
\begin{pgfscope}%
\pgfsetbuttcap%
\pgfsetroundjoin%
\definecolor{currentfill}{rgb}{0.000000,0.000000,0.000000}%
\pgfsetfillcolor{currentfill}%
\pgfsetlinewidth{0.501875pt}%
\definecolor{currentstroke}{rgb}{0.000000,0.000000,0.000000}%
\pgfsetstrokecolor{currentstroke}%
\pgfsetdash{}{0pt}%
\pgfsys@defobject{currentmarker}{\pgfqpoint{0.000000in}{-0.069444in}}{\pgfqpoint{0.000000in}{0.000000in}}{%
\pgfpathmoveto{\pgfqpoint{0.000000in}{0.000000in}}%
\pgfpathlineto{\pgfqpoint{0.000000in}{-0.069444in}}%
\pgfusepath{stroke,fill}%
}%
\begin{pgfscope}%
\pgfsys@transformshift{3.610554in}{0.711675in}%
\pgfsys@useobject{currentmarker}{}%
\end{pgfscope}%
\end{pgfscope}%
\begin{pgfscope}%
\pgftext[x=3.610554in,y=0.371511in,,top]{\rmfamily\fontsize{8.000000}{9.600000}\selectfont 1920}%
\end{pgfscope}%
\begin{pgfscope}%
\pgftext[x=2.247380in,y=0.194536in,,top]{\rmfamily\fontsize{9.000000}{10.800000}\selectfont \(\displaystyle m(K^+\!\pi^-)\)}%
\end{pgfscope}%
\begin{pgfscope}%
\pgfsetbuttcap%
\pgfsetroundjoin%
\definecolor{currentfill}{rgb}{0.000000,0.000000,0.000000}%
\pgfsetfillcolor{currentfill}%
\pgfsetlinewidth{0.501875pt}%
\definecolor{currentstroke}{rgb}{0.000000,0.000000,0.000000}%
\pgfsetstrokecolor{currentstroke}%
\pgfsetdash{}{0pt}%
\pgfsys@defobject{currentmarker}{\pgfqpoint{0.000000in}{0.000000in}}{\pgfqpoint{0.069444in}{0.000000in}}{%
\pgfpathmoveto{\pgfqpoint{0.000000in}{0.000000in}}%
\pgfpathlineto{\pgfqpoint{0.069444in}{0.000000in}}%
\pgfusepath{stroke,fill}%
}%
\begin{pgfscope}%
\pgfsys@transformshift{0.636356in}{0.440955in}%
\pgfsys@useobject{currentmarker}{}%
\end{pgfscope}%
\end{pgfscope}%
\begin{pgfscope}%
\pgfsetbuttcap%
\pgfsetroundjoin%
\definecolor{currentfill}{rgb}{0.000000,0.000000,0.000000}%
\pgfsetfillcolor{currentfill}%
\pgfsetlinewidth{0.501875pt}%
\definecolor{currentstroke}{rgb}{0.000000,0.000000,0.000000}%
\pgfsetstrokecolor{currentstroke}%
\pgfsetdash{}{0pt}%
\pgfsys@defobject{currentmarker}{\pgfqpoint{-0.069444in}{0.000000in}}{\pgfqpoint{0.000000in}{0.000000in}}{%
\pgfpathmoveto{\pgfqpoint{0.000000in}{0.000000in}}%
\pgfpathlineto{\pgfqpoint{-0.069444in}{0.000000in}}%
\pgfusepath{stroke,fill}%
}%
\begin{pgfscope}%
\pgfsys@transformshift{3.858404in}{0.440955in}%
\pgfsys@useobject{currentmarker}{}%
\end{pgfscope}%
\end{pgfscope}%
\begin{pgfscope}%
\pgftext[x=0.566911in,y=0.440955in,right,]{\rmfamily\fontsize{8.000000}{9.600000}\selectfont −3}%
\end{pgfscope}%
\begin{pgfscope}%
\pgfsetbuttcap%
\pgfsetroundjoin%
\definecolor{currentfill}{rgb}{0.000000,0.000000,0.000000}%
\pgfsetfillcolor{currentfill}%
\pgfsetlinewidth{0.501875pt}%
\definecolor{currentstroke}{rgb}{0.000000,0.000000,0.000000}%
\pgfsetstrokecolor{currentstroke}%
\pgfsetdash{}{0pt}%
\pgfsys@defobject{currentmarker}{\pgfqpoint{0.000000in}{0.000000in}}{\pgfqpoint{0.069444in}{0.000000in}}{%
\pgfpathmoveto{\pgfqpoint{0.000000in}{0.000000in}}%
\pgfpathlineto{\pgfqpoint{0.069444in}{0.000000in}}%
\pgfusepath{stroke,fill}%
}%
\begin{pgfscope}%
\pgfsys@transformshift{0.636356in}{0.576315in}%
\pgfsys@useobject{currentmarker}{}%
\end{pgfscope}%
\end{pgfscope}%
\begin{pgfscope}%
\pgfsetbuttcap%
\pgfsetroundjoin%
\definecolor{currentfill}{rgb}{0.000000,0.000000,0.000000}%
\pgfsetfillcolor{currentfill}%
\pgfsetlinewidth{0.501875pt}%
\definecolor{currentstroke}{rgb}{0.000000,0.000000,0.000000}%
\pgfsetstrokecolor{currentstroke}%
\pgfsetdash{}{0pt}%
\pgfsys@defobject{currentmarker}{\pgfqpoint{-0.069444in}{0.000000in}}{\pgfqpoint{0.000000in}{0.000000in}}{%
\pgfpathmoveto{\pgfqpoint{0.000000in}{0.000000in}}%
\pgfpathlineto{\pgfqpoint{-0.069444in}{0.000000in}}%
\pgfusepath{stroke,fill}%
}%
\begin{pgfscope}%
\pgfsys@transformshift{3.858404in}{0.576315in}%
\pgfsys@useobject{currentmarker}{}%
\end{pgfscope}%
\end{pgfscope}%
\begin{pgfscope}%
\pgftext[x=0.566911in,y=0.576315in,right,]{\rmfamily\fontsize{8.000000}{9.600000}\selectfont 0}%
\end{pgfscope}%
\begin{pgfscope}%
\pgfsetbuttcap%
\pgfsetroundjoin%
\definecolor{currentfill}{rgb}{0.000000,0.000000,0.000000}%
\pgfsetfillcolor{currentfill}%
\pgfsetlinewidth{0.501875pt}%
\definecolor{currentstroke}{rgb}{0.000000,0.000000,0.000000}%
\pgfsetstrokecolor{currentstroke}%
\pgfsetdash{}{0pt}%
\pgfsys@defobject{currentmarker}{\pgfqpoint{0.000000in}{0.000000in}}{\pgfqpoint{0.069444in}{0.000000in}}{%
\pgfpathmoveto{\pgfqpoint{0.000000in}{0.000000in}}%
\pgfpathlineto{\pgfqpoint{0.069444in}{0.000000in}}%
\pgfusepath{stroke,fill}%
}%
\begin{pgfscope}%
\pgfsys@transformshift{0.636356in}{0.711675in}%
\pgfsys@useobject{currentmarker}{}%
\end{pgfscope}%
\end{pgfscope}%
\begin{pgfscope}%
\pgfsetbuttcap%
\pgfsetroundjoin%
\definecolor{currentfill}{rgb}{0.000000,0.000000,0.000000}%
\pgfsetfillcolor{currentfill}%
\pgfsetlinewidth{0.501875pt}%
\definecolor{currentstroke}{rgb}{0.000000,0.000000,0.000000}%
\pgfsetstrokecolor{currentstroke}%
\pgfsetdash{}{0pt}%
\pgfsys@defobject{currentmarker}{\pgfqpoint{-0.069444in}{0.000000in}}{\pgfqpoint{0.000000in}{0.000000in}}{%
\pgfpathmoveto{\pgfqpoint{0.000000in}{0.000000in}}%
\pgfpathlineto{\pgfqpoint{-0.069444in}{0.000000in}}%
\pgfusepath{stroke,fill}%
}%
\begin{pgfscope}%
\pgfsys@transformshift{3.858404in}{0.711675in}%
\pgfsys@useobject{currentmarker}{}%
\end{pgfscope}%
\end{pgfscope}%
\begin{pgfscope}%
\pgftext[x=0.566911in,y=0.711675in,right,]{\rmfamily\fontsize{8.000000}{9.600000}\selectfont 3}%
\end{pgfscope}%
\begin{pgfscope}%
\pgftext[x=0.333676in,y=0.576315in,,bottom,rotate=90.000000]{\rmfamily\fontsize{9.000000}{10.800000}\selectfont \(\displaystyle \frac{\hat{n}_i -  n_i}{\sigma(n_i)}\)}%
\end{pgfscope}%
\begin{pgfscope}%
\pgfsetbuttcap%
\pgfsetmiterjoin%
\definecolor{currentfill}{rgb}{1.000000,1.000000,1.000000}%
\pgfsetfillcolor{currentfill}%
\pgfsetlinewidth{0.000000pt}%
\definecolor{currentstroke}{rgb}{0.000000,0.000000,0.000000}%
\pgfsetstrokecolor{currentstroke}%
\pgfsetstrokeopacity{0.000000}%
\pgfsetdash{}{0pt}%
\pgfpathmoveto{\pgfqpoint{0.636356in}{0.874107in}}%
\pgfpathlineto{\pgfqpoint{3.858404in}{0.874107in}}%
\pgfpathlineto{\pgfqpoint{3.858404in}{2.769145in}}%
\pgfpathlineto{\pgfqpoint{0.636356in}{2.769145in}}%
\pgfpathclose%
\pgfusepath{fill}%
\end{pgfscope}%
\begin{pgfscope}%
\pgfpathrectangle{\pgfqpoint{0.636356in}{0.874107in}}{\pgfqpoint{3.222048in}{1.895038in}} %
\pgfusepath{clip}%
\pgfsetbuttcap%
\pgfsetroundjoin%
\pgfsetlinewidth{1.003750pt}%
\definecolor{currentstroke}{rgb}{1.000000,0.000000,0.000000}%
\pgfsetstrokecolor{currentstroke}%
\pgfsetdash{{8.000000pt}{3.000000pt}}{0.000000pt}%
\pgfpathmoveto{\pgfqpoint{0.652547in}{1.015787in}}%
\pgfpathlineto{\pgfqpoint{3.842212in}{1.001731in}}%
\pgfpathlineto{\pgfqpoint{3.842212in}{1.001731in}}%
\pgfusepath{stroke}%
\end{pgfscope}%
\begin{pgfscope}%
\pgfpathrectangle{\pgfqpoint{0.636356in}{0.874107in}}{\pgfqpoint{3.222048in}{1.895038in}} %
\pgfusepath{clip}%
\pgfsetbuttcap%
\pgfsetroundjoin%
\pgfsetlinewidth{1.003750pt}%
\definecolor{currentstroke}{rgb}{0.000000,0.000000,1.000000}%
\pgfsetstrokecolor{currentstroke}%
\pgfsetdash{{8.000000pt}{3.000000pt}}{0.000000pt}%
\pgfpathmoveto{\pgfqpoint{0.652547in}{0.874107in}}%
\pgfpathlineto{\pgfqpoint{1.445916in}{0.875156in}}%
\pgfpathlineto{\pgfqpoint{1.526872in}{0.877742in}}%
\pgfpathlineto{\pgfqpoint{1.575445in}{0.881228in}}%
\pgfpathlineto{\pgfqpoint{1.624019in}{0.887660in}}%
\pgfpathlineto{\pgfqpoint{1.656401in}{0.894323in}}%
\pgfpathlineto{\pgfqpoint{1.688783in}{0.903682in}}%
\pgfpathlineto{\pgfqpoint{1.704975in}{0.909481in}}%
\pgfpathlineto{\pgfqpoint{1.737357in}{0.924405in}}%
\pgfpathlineto{\pgfqpoint{1.769739in}{0.944244in}}%
\pgfpathlineto{\pgfqpoint{1.802122in}{0.970022in}}%
\pgfpathlineto{\pgfqpoint{1.834504in}{1.002742in}}%
\pgfpathlineto{\pgfqpoint{1.866887in}{1.042886in}}%
\pgfpathlineto{\pgfqpoint{1.883078in}{1.066307in}}%
\pgfpathlineto{\pgfqpoint{1.899269in}{1.091914in}}%
\pgfpathlineto{\pgfqpoint{1.931651in}{1.149759in}}%
\pgfpathlineto{\pgfqpoint{1.947843in}{1.181952in}}%
\pgfpathlineto{\pgfqpoint{1.980225in}{1.253027in}}%
\pgfpathlineto{\pgfqpoint{2.012607in}{1.330614in}}%
\pgfpathlineto{\pgfqpoint{2.044990in}{1.413508in}}%
\pgfpathlineto{\pgfqpoint{2.142137in}{1.666030in}}%
\pgfpathlineto{\pgfqpoint{2.174519in}{1.740548in}}%
\pgfpathlineto{\pgfqpoint{2.190710in}{1.773545in}}%
\pgfpathlineto{\pgfqpoint{2.206902in}{1.803261in}}%
\pgfpathlineto{\pgfqpoint{2.223093in}{1.829295in}}%
\pgfpathlineto{\pgfqpoint{2.239284in}{1.851286in}}%
\pgfpathlineto{\pgfqpoint{2.255475in}{1.868925in}}%
\pgfpathlineto{\pgfqpoint{2.271666in}{1.881961in}}%
\pgfpathlineto{\pgfqpoint{2.287858in}{1.890206in}}%
\pgfpathlineto{\pgfqpoint{2.304049in}{1.893540in}}%
\pgfpathlineto{\pgfqpoint{2.320240in}{1.891915in}}%
\pgfpathlineto{\pgfqpoint{2.336431in}{1.885354in}}%
\pgfpathlineto{\pgfqpoint{2.352622in}{1.873953in}}%
\pgfpathlineto{\pgfqpoint{2.368814in}{1.857877in}}%
\pgfpathlineto{\pgfqpoint{2.385005in}{1.837355in}}%
\pgfpathlineto{\pgfqpoint{2.401196in}{1.812678in}}%
\pgfpathlineto{\pgfqpoint{2.417387in}{1.784188in}}%
\pgfpathlineto{\pgfqpoint{2.433578in}{1.752275in}}%
\pgfpathlineto{\pgfqpoint{2.465961in}{1.679903in}}%
\pgfpathlineto{\pgfqpoint{2.498343in}{1.598935in}}%
\pgfpathlineto{\pgfqpoint{2.611682in}{1.304533in}}%
\pgfpathlineto{\pgfqpoint{2.644064in}{1.229122in}}%
\pgfpathlineto{\pgfqpoint{2.676446in}{1.160611in}}%
\pgfpathlineto{\pgfqpoint{2.708829in}{1.101248in}}%
\pgfpathlineto{\pgfqpoint{2.725020in}{1.074879in}}%
\pgfpathlineto{\pgfqpoint{2.741211in}{1.050708in}}%
\pgfpathlineto{\pgfqpoint{2.757402in}{1.028692in}}%
\pgfpathlineto{\pgfqpoint{2.789785in}{0.991156in}}%
\pgfpathlineto{\pgfqpoint{2.822167in}{0.960821in}}%
\pgfpathlineto{\pgfqpoint{2.854549in}{0.937107in}}%
\pgfpathlineto{\pgfqpoint{2.886932in}{0.918995in}}%
\pgfpathlineto{\pgfqpoint{2.919314in}{0.905472in}}%
\pgfpathlineto{\pgfqpoint{2.951697in}{0.895600in}}%
\pgfpathlineto{\pgfqpoint{2.984079in}{0.888550in}}%
\pgfpathlineto{\pgfqpoint{3.032653in}{0.881846in}}%
\pgfpathlineto{\pgfqpoint{3.081226in}{0.878006in}}%
\pgfpathlineto{\pgfqpoint{3.145991in}{0.875584in}}%
\pgfpathlineto{\pgfqpoint{3.259329in}{0.874335in}}%
\pgfpathlineto{\pgfqpoint{3.615536in}{0.874107in}}%
\pgfpathlineto{\pgfqpoint{3.842212in}{0.874107in}}%
\pgfpathlineto{\pgfqpoint{3.842212in}{0.874107in}}%
\pgfusepath{stroke}%
\end{pgfscope}%
\begin{pgfscope}%
\pgfsetrectcap%
\pgfsetmiterjoin%
\pgfsetlinewidth{1.003750pt}%
\definecolor{currentstroke}{rgb}{0.000000,0.000000,0.000000}%
\pgfsetstrokecolor{currentstroke}%
\pgfsetdash{}{0pt}%
\pgfpathmoveto{\pgfqpoint{0.636356in}{2.769145in}}%
\pgfpathlineto{\pgfqpoint{3.858404in}{2.769145in}}%
\pgfusepath{stroke}%
\end{pgfscope}%
\begin{pgfscope}%
\pgfsetrectcap%
\pgfsetmiterjoin%
\pgfsetlinewidth{1.003750pt}%
\definecolor{currentstroke}{rgb}{0.000000,0.000000,0.000000}%
\pgfsetstrokecolor{currentstroke}%
\pgfsetdash{}{0pt}%
\pgfpathmoveto{\pgfqpoint{3.858404in}{0.874107in}}%
\pgfpathlineto{\pgfqpoint{3.858404in}{2.769145in}}%
\pgfusepath{stroke}%
\end{pgfscope}%
\begin{pgfscope}%
\pgfsetrectcap%
\pgfsetmiterjoin%
\pgfsetlinewidth{1.003750pt}%
\definecolor{currentstroke}{rgb}{0.000000,0.000000,0.000000}%
\pgfsetstrokecolor{currentstroke}%
\pgfsetdash{}{0pt}%
\pgfpathmoveto{\pgfqpoint{0.636356in}{0.874107in}}%
\pgfpathlineto{\pgfqpoint{3.858404in}{0.874107in}}%
\pgfusepath{stroke}%
\end{pgfscope}%
\begin{pgfscope}%
\pgfsetrectcap%
\pgfsetmiterjoin%
\pgfsetlinewidth{1.003750pt}%
\definecolor{currentstroke}{rgb}{0.000000,0.000000,0.000000}%
\pgfsetstrokecolor{currentstroke}%
\pgfsetdash{}{0pt}%
\pgfpathmoveto{\pgfqpoint{0.636356in}{0.874107in}}%
\pgfpathlineto{\pgfqpoint{0.636356in}{2.769145in}}%
\pgfusepath{stroke}%
\end{pgfscope}%
\begin{pgfscope}%
\pgfsetbuttcap%
\pgfsetroundjoin%
\definecolor{currentfill}{rgb}{0.000000,0.000000,0.000000}%
\pgfsetfillcolor{currentfill}%
\pgfsetlinewidth{0.501875pt}%
\definecolor{currentstroke}{rgb}{0.000000,0.000000,0.000000}%
\pgfsetstrokecolor{currentstroke}%
\pgfsetdash{}{0pt}%
\pgfsys@defobject{currentmarker}{\pgfqpoint{0.000000in}{0.000000in}}{\pgfqpoint{0.000000in}{0.069444in}}{%
\pgfpathmoveto{\pgfqpoint{0.000000in}{0.000000in}}%
\pgfpathlineto{\pgfqpoint{0.000000in}{0.069444in}}%
\pgfusepath{stroke,fill}%
}%
\begin{pgfscope}%
\pgfsys@transformshift{0.636356in}{0.874107in}%
\pgfsys@useobject{currentmarker}{}%
\end{pgfscope}%
\end{pgfscope}%
\begin{pgfscope}%
\pgfsetbuttcap%
\pgfsetroundjoin%
\definecolor{currentfill}{rgb}{0.000000,0.000000,0.000000}%
\pgfsetfillcolor{currentfill}%
\pgfsetlinewidth{0.501875pt}%
\definecolor{currentstroke}{rgb}{0.000000,0.000000,0.000000}%
\pgfsetstrokecolor{currentstroke}%
\pgfsetdash{}{0pt}%
\pgfsys@defobject{currentmarker}{\pgfqpoint{0.000000in}{-0.069444in}}{\pgfqpoint{0.000000in}{0.000000in}}{%
\pgfpathmoveto{\pgfqpoint{0.000000in}{0.000000in}}%
\pgfpathlineto{\pgfqpoint{0.000000in}{-0.069444in}}%
\pgfusepath{stroke,fill}%
}%
\begin{pgfscope}%
\pgfsys@transformshift{0.636356in}{2.769145in}%
\pgfsys@useobject{currentmarker}{}%
\end{pgfscope}%
\end{pgfscope}%
\begin{pgfscope}%
\pgfsetbuttcap%
\pgfsetroundjoin%
\definecolor{currentfill}{rgb}{0.000000,0.000000,0.000000}%
\pgfsetfillcolor{currentfill}%
\pgfsetlinewidth{0.501875pt}%
\definecolor{currentstroke}{rgb}{0.000000,0.000000,0.000000}%
\pgfsetstrokecolor{currentstroke}%
\pgfsetdash{}{0pt}%
\pgfsys@defobject{currentmarker}{\pgfqpoint{0.000000in}{0.000000in}}{\pgfqpoint{0.000000in}{0.069444in}}{%
\pgfpathmoveto{\pgfqpoint{0.000000in}{0.000000in}}%
\pgfpathlineto{\pgfqpoint{0.000000in}{0.069444in}}%
\pgfusepath{stroke,fill}%
}%
\begin{pgfscope}%
\pgfsys@transformshift{1.132055in}{0.874107in}%
\pgfsys@useobject{currentmarker}{}%
\end{pgfscope}%
\end{pgfscope}%
\begin{pgfscope}%
\pgfsetbuttcap%
\pgfsetroundjoin%
\definecolor{currentfill}{rgb}{0.000000,0.000000,0.000000}%
\pgfsetfillcolor{currentfill}%
\pgfsetlinewidth{0.501875pt}%
\definecolor{currentstroke}{rgb}{0.000000,0.000000,0.000000}%
\pgfsetstrokecolor{currentstroke}%
\pgfsetdash{}{0pt}%
\pgfsys@defobject{currentmarker}{\pgfqpoint{0.000000in}{-0.069444in}}{\pgfqpoint{0.000000in}{0.000000in}}{%
\pgfpathmoveto{\pgfqpoint{0.000000in}{0.000000in}}%
\pgfpathlineto{\pgfqpoint{0.000000in}{-0.069444in}}%
\pgfusepath{stroke,fill}%
}%
\begin{pgfscope}%
\pgfsys@transformshift{1.132055in}{2.769145in}%
\pgfsys@useobject{currentmarker}{}%
\end{pgfscope}%
\end{pgfscope}%
\begin{pgfscope}%
\pgfsetbuttcap%
\pgfsetroundjoin%
\definecolor{currentfill}{rgb}{0.000000,0.000000,0.000000}%
\pgfsetfillcolor{currentfill}%
\pgfsetlinewidth{0.501875pt}%
\definecolor{currentstroke}{rgb}{0.000000,0.000000,0.000000}%
\pgfsetstrokecolor{currentstroke}%
\pgfsetdash{}{0pt}%
\pgfsys@defobject{currentmarker}{\pgfqpoint{0.000000in}{0.000000in}}{\pgfqpoint{0.000000in}{0.069444in}}{%
\pgfpathmoveto{\pgfqpoint{0.000000in}{0.000000in}}%
\pgfpathlineto{\pgfqpoint{0.000000in}{0.069444in}}%
\pgfusepath{stroke,fill}%
}%
\begin{pgfscope}%
\pgfsys@transformshift{1.627755in}{0.874107in}%
\pgfsys@useobject{currentmarker}{}%
\end{pgfscope}%
\end{pgfscope}%
\begin{pgfscope}%
\pgfsetbuttcap%
\pgfsetroundjoin%
\definecolor{currentfill}{rgb}{0.000000,0.000000,0.000000}%
\pgfsetfillcolor{currentfill}%
\pgfsetlinewidth{0.501875pt}%
\definecolor{currentstroke}{rgb}{0.000000,0.000000,0.000000}%
\pgfsetstrokecolor{currentstroke}%
\pgfsetdash{}{0pt}%
\pgfsys@defobject{currentmarker}{\pgfqpoint{0.000000in}{-0.069444in}}{\pgfqpoint{0.000000in}{0.000000in}}{%
\pgfpathmoveto{\pgfqpoint{0.000000in}{0.000000in}}%
\pgfpathlineto{\pgfqpoint{0.000000in}{-0.069444in}}%
\pgfusepath{stroke,fill}%
}%
\begin{pgfscope}%
\pgfsys@transformshift{1.627755in}{2.769145in}%
\pgfsys@useobject{currentmarker}{}%
\end{pgfscope}%
\end{pgfscope}%
\begin{pgfscope}%
\pgfsetbuttcap%
\pgfsetroundjoin%
\definecolor{currentfill}{rgb}{0.000000,0.000000,0.000000}%
\pgfsetfillcolor{currentfill}%
\pgfsetlinewidth{0.501875pt}%
\definecolor{currentstroke}{rgb}{0.000000,0.000000,0.000000}%
\pgfsetstrokecolor{currentstroke}%
\pgfsetdash{}{0pt}%
\pgfsys@defobject{currentmarker}{\pgfqpoint{0.000000in}{0.000000in}}{\pgfqpoint{0.000000in}{0.069444in}}{%
\pgfpathmoveto{\pgfqpoint{0.000000in}{0.000000in}}%
\pgfpathlineto{\pgfqpoint{0.000000in}{0.069444in}}%
\pgfusepath{stroke,fill}%
}%
\begin{pgfscope}%
\pgfsys@transformshift{2.123455in}{0.874107in}%
\pgfsys@useobject{currentmarker}{}%
\end{pgfscope}%
\end{pgfscope}%
\begin{pgfscope}%
\pgfsetbuttcap%
\pgfsetroundjoin%
\definecolor{currentfill}{rgb}{0.000000,0.000000,0.000000}%
\pgfsetfillcolor{currentfill}%
\pgfsetlinewidth{0.501875pt}%
\definecolor{currentstroke}{rgb}{0.000000,0.000000,0.000000}%
\pgfsetstrokecolor{currentstroke}%
\pgfsetdash{}{0pt}%
\pgfsys@defobject{currentmarker}{\pgfqpoint{0.000000in}{-0.069444in}}{\pgfqpoint{0.000000in}{0.000000in}}{%
\pgfpathmoveto{\pgfqpoint{0.000000in}{0.000000in}}%
\pgfpathlineto{\pgfqpoint{0.000000in}{-0.069444in}}%
\pgfusepath{stroke,fill}%
}%
\begin{pgfscope}%
\pgfsys@transformshift{2.123455in}{2.769145in}%
\pgfsys@useobject{currentmarker}{}%
\end{pgfscope}%
\end{pgfscope}%
\begin{pgfscope}%
\pgfsetbuttcap%
\pgfsetroundjoin%
\definecolor{currentfill}{rgb}{0.000000,0.000000,0.000000}%
\pgfsetfillcolor{currentfill}%
\pgfsetlinewidth{0.501875pt}%
\definecolor{currentstroke}{rgb}{0.000000,0.000000,0.000000}%
\pgfsetstrokecolor{currentstroke}%
\pgfsetdash{}{0pt}%
\pgfsys@defobject{currentmarker}{\pgfqpoint{0.000000in}{0.000000in}}{\pgfqpoint{0.000000in}{0.069444in}}{%
\pgfpathmoveto{\pgfqpoint{0.000000in}{0.000000in}}%
\pgfpathlineto{\pgfqpoint{0.000000in}{0.069444in}}%
\pgfusepath{stroke,fill}%
}%
\begin{pgfscope}%
\pgfsys@transformshift{2.619154in}{0.874107in}%
\pgfsys@useobject{currentmarker}{}%
\end{pgfscope}%
\end{pgfscope}%
\begin{pgfscope}%
\pgfsetbuttcap%
\pgfsetroundjoin%
\definecolor{currentfill}{rgb}{0.000000,0.000000,0.000000}%
\pgfsetfillcolor{currentfill}%
\pgfsetlinewidth{0.501875pt}%
\definecolor{currentstroke}{rgb}{0.000000,0.000000,0.000000}%
\pgfsetstrokecolor{currentstroke}%
\pgfsetdash{}{0pt}%
\pgfsys@defobject{currentmarker}{\pgfqpoint{0.000000in}{-0.069444in}}{\pgfqpoint{0.000000in}{0.000000in}}{%
\pgfpathmoveto{\pgfqpoint{0.000000in}{0.000000in}}%
\pgfpathlineto{\pgfqpoint{0.000000in}{-0.069444in}}%
\pgfusepath{stroke,fill}%
}%
\begin{pgfscope}%
\pgfsys@transformshift{2.619154in}{2.769145in}%
\pgfsys@useobject{currentmarker}{}%
\end{pgfscope}%
\end{pgfscope}%
\begin{pgfscope}%
\pgfsetbuttcap%
\pgfsetroundjoin%
\definecolor{currentfill}{rgb}{0.000000,0.000000,0.000000}%
\pgfsetfillcolor{currentfill}%
\pgfsetlinewidth{0.501875pt}%
\definecolor{currentstroke}{rgb}{0.000000,0.000000,0.000000}%
\pgfsetstrokecolor{currentstroke}%
\pgfsetdash{}{0pt}%
\pgfsys@defobject{currentmarker}{\pgfqpoint{0.000000in}{0.000000in}}{\pgfqpoint{0.000000in}{0.069444in}}{%
\pgfpathmoveto{\pgfqpoint{0.000000in}{0.000000in}}%
\pgfpathlineto{\pgfqpoint{0.000000in}{0.069444in}}%
\pgfusepath{stroke,fill}%
}%
\begin{pgfscope}%
\pgfsys@transformshift{3.114854in}{0.874107in}%
\pgfsys@useobject{currentmarker}{}%
\end{pgfscope}%
\end{pgfscope}%
\begin{pgfscope}%
\pgfsetbuttcap%
\pgfsetroundjoin%
\definecolor{currentfill}{rgb}{0.000000,0.000000,0.000000}%
\pgfsetfillcolor{currentfill}%
\pgfsetlinewidth{0.501875pt}%
\definecolor{currentstroke}{rgb}{0.000000,0.000000,0.000000}%
\pgfsetstrokecolor{currentstroke}%
\pgfsetdash{}{0pt}%
\pgfsys@defobject{currentmarker}{\pgfqpoint{0.000000in}{-0.069444in}}{\pgfqpoint{0.000000in}{0.000000in}}{%
\pgfpathmoveto{\pgfqpoint{0.000000in}{0.000000in}}%
\pgfpathlineto{\pgfqpoint{0.000000in}{-0.069444in}}%
\pgfusepath{stroke,fill}%
}%
\begin{pgfscope}%
\pgfsys@transformshift{3.114854in}{2.769145in}%
\pgfsys@useobject{currentmarker}{}%
\end{pgfscope}%
\end{pgfscope}%
\begin{pgfscope}%
\pgfsetbuttcap%
\pgfsetroundjoin%
\definecolor{currentfill}{rgb}{0.000000,0.000000,0.000000}%
\pgfsetfillcolor{currentfill}%
\pgfsetlinewidth{0.501875pt}%
\definecolor{currentstroke}{rgb}{0.000000,0.000000,0.000000}%
\pgfsetstrokecolor{currentstroke}%
\pgfsetdash{}{0pt}%
\pgfsys@defobject{currentmarker}{\pgfqpoint{0.000000in}{0.000000in}}{\pgfqpoint{0.000000in}{0.069444in}}{%
\pgfpathmoveto{\pgfqpoint{0.000000in}{0.000000in}}%
\pgfpathlineto{\pgfqpoint{0.000000in}{0.069444in}}%
\pgfusepath{stroke,fill}%
}%
\begin{pgfscope}%
\pgfsys@transformshift{3.610554in}{0.874107in}%
\pgfsys@useobject{currentmarker}{}%
\end{pgfscope}%
\end{pgfscope}%
\begin{pgfscope}%
\pgfsetbuttcap%
\pgfsetroundjoin%
\definecolor{currentfill}{rgb}{0.000000,0.000000,0.000000}%
\pgfsetfillcolor{currentfill}%
\pgfsetlinewidth{0.501875pt}%
\definecolor{currentstroke}{rgb}{0.000000,0.000000,0.000000}%
\pgfsetstrokecolor{currentstroke}%
\pgfsetdash{}{0pt}%
\pgfsys@defobject{currentmarker}{\pgfqpoint{0.000000in}{-0.069444in}}{\pgfqpoint{0.000000in}{0.000000in}}{%
\pgfpathmoveto{\pgfqpoint{0.000000in}{0.000000in}}%
\pgfpathlineto{\pgfqpoint{0.000000in}{-0.069444in}}%
\pgfusepath{stroke,fill}%
}%
\begin{pgfscope}%
\pgfsys@transformshift{3.610554in}{2.769145in}%
\pgfsys@useobject{currentmarker}{}%
\end{pgfscope}%
\end{pgfscope}%
\begin{pgfscope}%
\pgfsetbuttcap%
\pgfsetroundjoin%
\definecolor{currentfill}{rgb}{0.000000,0.000000,0.000000}%
\pgfsetfillcolor{currentfill}%
\pgfsetlinewidth{0.501875pt}%
\definecolor{currentstroke}{rgb}{0.000000,0.000000,0.000000}%
\pgfsetstrokecolor{currentstroke}%
\pgfsetdash{}{0pt}%
\pgfsys@defobject{currentmarker}{\pgfqpoint{0.000000in}{0.000000in}}{\pgfqpoint{0.069444in}{0.000000in}}{%
\pgfpathmoveto{\pgfqpoint{0.000000in}{0.000000in}}%
\pgfpathlineto{\pgfqpoint{0.069444in}{0.000000in}}%
\pgfusepath{stroke,fill}%
}%
\begin{pgfscope}%
\pgfsys@transformshift{0.636356in}{0.874107in}%
\pgfsys@useobject{currentmarker}{}%
\end{pgfscope}%
\end{pgfscope}%
\begin{pgfscope}%
\pgfsetbuttcap%
\pgfsetroundjoin%
\definecolor{currentfill}{rgb}{0.000000,0.000000,0.000000}%
\pgfsetfillcolor{currentfill}%
\pgfsetlinewidth{0.501875pt}%
\definecolor{currentstroke}{rgb}{0.000000,0.000000,0.000000}%
\pgfsetstrokecolor{currentstroke}%
\pgfsetdash{}{0pt}%
\pgfsys@defobject{currentmarker}{\pgfqpoint{-0.069444in}{0.000000in}}{\pgfqpoint{0.000000in}{0.000000in}}{%
\pgfpathmoveto{\pgfqpoint{0.000000in}{0.000000in}}%
\pgfpathlineto{\pgfqpoint{-0.069444in}{0.000000in}}%
\pgfusepath{stroke,fill}%
}%
\begin{pgfscope}%
\pgfsys@transformshift{3.858404in}{0.874107in}%
\pgfsys@useobject{currentmarker}{}%
\end{pgfscope}%
\end{pgfscope}%
\begin{pgfscope}%
\pgftext[x=0.566911in,y=0.874107in,right,]{\rmfamily\fontsize{8.000000}{9.600000}\selectfont 0}%
\end{pgfscope}%
\begin{pgfscope}%
\pgfsetbuttcap%
\pgfsetroundjoin%
\definecolor{currentfill}{rgb}{0.000000,0.000000,0.000000}%
\pgfsetfillcolor{currentfill}%
\pgfsetlinewidth{0.501875pt}%
\definecolor{currentstroke}{rgb}{0.000000,0.000000,0.000000}%
\pgfsetstrokecolor{currentstroke}%
\pgfsetdash{}{0pt}%
\pgfsys@defobject{currentmarker}{\pgfqpoint{0.000000in}{0.000000in}}{\pgfqpoint{0.069444in}{0.000000in}}{%
\pgfpathmoveto{\pgfqpoint{0.000000in}{0.000000in}}%
\pgfpathlineto{\pgfqpoint{0.069444in}{0.000000in}}%
\pgfusepath{stroke,fill}%
}%
\begin{pgfscope}%
\pgfsys@transformshift{0.636356in}{1.347866in}%
\pgfsys@useobject{currentmarker}{}%
\end{pgfscope}%
\end{pgfscope}%
\begin{pgfscope}%
\pgfsetbuttcap%
\pgfsetroundjoin%
\definecolor{currentfill}{rgb}{0.000000,0.000000,0.000000}%
\pgfsetfillcolor{currentfill}%
\pgfsetlinewidth{0.501875pt}%
\definecolor{currentstroke}{rgb}{0.000000,0.000000,0.000000}%
\pgfsetstrokecolor{currentstroke}%
\pgfsetdash{}{0pt}%
\pgfsys@defobject{currentmarker}{\pgfqpoint{-0.069444in}{0.000000in}}{\pgfqpoint{0.000000in}{0.000000in}}{%
\pgfpathmoveto{\pgfqpoint{0.000000in}{0.000000in}}%
\pgfpathlineto{\pgfqpoint{-0.069444in}{0.000000in}}%
\pgfusepath{stroke,fill}%
}%
\begin{pgfscope}%
\pgfsys@transformshift{3.858404in}{1.347866in}%
\pgfsys@useobject{currentmarker}{}%
\end{pgfscope}%
\end{pgfscope}%
\begin{pgfscope}%
\pgftext[x=0.566911in,y=1.347866in,right,]{\rmfamily\fontsize{8.000000}{9.600000}\selectfont 5}%
\end{pgfscope}%
\begin{pgfscope}%
\pgfsetbuttcap%
\pgfsetroundjoin%
\definecolor{currentfill}{rgb}{0.000000,0.000000,0.000000}%
\pgfsetfillcolor{currentfill}%
\pgfsetlinewidth{0.501875pt}%
\definecolor{currentstroke}{rgb}{0.000000,0.000000,0.000000}%
\pgfsetstrokecolor{currentstroke}%
\pgfsetdash{}{0pt}%
\pgfsys@defobject{currentmarker}{\pgfqpoint{0.000000in}{0.000000in}}{\pgfqpoint{0.069444in}{0.000000in}}{%
\pgfpathmoveto{\pgfqpoint{0.000000in}{0.000000in}}%
\pgfpathlineto{\pgfqpoint{0.069444in}{0.000000in}}%
\pgfusepath{stroke,fill}%
}%
\begin{pgfscope}%
\pgfsys@transformshift{0.636356in}{1.821626in}%
\pgfsys@useobject{currentmarker}{}%
\end{pgfscope}%
\end{pgfscope}%
\begin{pgfscope}%
\pgfsetbuttcap%
\pgfsetroundjoin%
\definecolor{currentfill}{rgb}{0.000000,0.000000,0.000000}%
\pgfsetfillcolor{currentfill}%
\pgfsetlinewidth{0.501875pt}%
\definecolor{currentstroke}{rgb}{0.000000,0.000000,0.000000}%
\pgfsetstrokecolor{currentstroke}%
\pgfsetdash{}{0pt}%
\pgfsys@defobject{currentmarker}{\pgfqpoint{-0.069444in}{0.000000in}}{\pgfqpoint{0.000000in}{0.000000in}}{%
\pgfpathmoveto{\pgfqpoint{0.000000in}{0.000000in}}%
\pgfpathlineto{\pgfqpoint{-0.069444in}{0.000000in}}%
\pgfusepath{stroke,fill}%
}%
\begin{pgfscope}%
\pgfsys@transformshift{3.858404in}{1.821626in}%
\pgfsys@useobject{currentmarker}{}%
\end{pgfscope}%
\end{pgfscope}%
\begin{pgfscope}%
\pgftext[x=0.566911in,y=1.821626in,right,]{\rmfamily\fontsize{8.000000}{9.600000}\selectfont 10}%
\end{pgfscope}%
\begin{pgfscope}%
\pgfsetbuttcap%
\pgfsetroundjoin%
\definecolor{currentfill}{rgb}{0.000000,0.000000,0.000000}%
\pgfsetfillcolor{currentfill}%
\pgfsetlinewidth{0.501875pt}%
\definecolor{currentstroke}{rgb}{0.000000,0.000000,0.000000}%
\pgfsetstrokecolor{currentstroke}%
\pgfsetdash{}{0pt}%
\pgfsys@defobject{currentmarker}{\pgfqpoint{0.000000in}{0.000000in}}{\pgfqpoint{0.069444in}{0.000000in}}{%
\pgfpathmoveto{\pgfqpoint{0.000000in}{0.000000in}}%
\pgfpathlineto{\pgfqpoint{0.069444in}{0.000000in}}%
\pgfusepath{stroke,fill}%
}%
\begin{pgfscope}%
\pgfsys@transformshift{0.636356in}{2.295385in}%
\pgfsys@useobject{currentmarker}{}%
\end{pgfscope}%
\end{pgfscope}%
\begin{pgfscope}%
\pgfsetbuttcap%
\pgfsetroundjoin%
\definecolor{currentfill}{rgb}{0.000000,0.000000,0.000000}%
\pgfsetfillcolor{currentfill}%
\pgfsetlinewidth{0.501875pt}%
\definecolor{currentstroke}{rgb}{0.000000,0.000000,0.000000}%
\pgfsetstrokecolor{currentstroke}%
\pgfsetdash{}{0pt}%
\pgfsys@defobject{currentmarker}{\pgfqpoint{-0.069444in}{0.000000in}}{\pgfqpoint{0.000000in}{0.000000in}}{%
\pgfpathmoveto{\pgfqpoint{0.000000in}{0.000000in}}%
\pgfpathlineto{\pgfqpoint{-0.069444in}{0.000000in}}%
\pgfusepath{stroke,fill}%
}%
\begin{pgfscope}%
\pgfsys@transformshift{3.858404in}{2.295385in}%
\pgfsys@useobject{currentmarker}{}%
\end{pgfscope}%
\end{pgfscope}%
\begin{pgfscope}%
\pgftext[x=0.566911in,y=2.295385in,right,]{\rmfamily\fontsize{8.000000}{9.600000}\selectfont 15}%
\end{pgfscope}%
\begin{pgfscope}%
\pgfsetbuttcap%
\pgfsetroundjoin%
\definecolor{currentfill}{rgb}{0.000000,0.000000,0.000000}%
\pgfsetfillcolor{currentfill}%
\pgfsetlinewidth{0.501875pt}%
\definecolor{currentstroke}{rgb}{0.000000,0.000000,0.000000}%
\pgfsetstrokecolor{currentstroke}%
\pgfsetdash{}{0pt}%
\pgfsys@defobject{currentmarker}{\pgfqpoint{0.000000in}{0.000000in}}{\pgfqpoint{0.069444in}{0.000000in}}{%
\pgfpathmoveto{\pgfqpoint{0.000000in}{0.000000in}}%
\pgfpathlineto{\pgfqpoint{0.069444in}{0.000000in}}%
\pgfusepath{stroke,fill}%
}%
\begin{pgfscope}%
\pgfsys@transformshift{0.636356in}{2.769145in}%
\pgfsys@useobject{currentmarker}{}%
\end{pgfscope}%
\end{pgfscope}%
\begin{pgfscope}%
\pgfsetbuttcap%
\pgfsetroundjoin%
\definecolor{currentfill}{rgb}{0.000000,0.000000,0.000000}%
\pgfsetfillcolor{currentfill}%
\pgfsetlinewidth{0.501875pt}%
\definecolor{currentstroke}{rgb}{0.000000,0.000000,0.000000}%
\pgfsetstrokecolor{currentstroke}%
\pgfsetdash{}{0pt}%
\pgfsys@defobject{currentmarker}{\pgfqpoint{-0.069444in}{0.000000in}}{\pgfqpoint{0.000000in}{0.000000in}}{%
\pgfpathmoveto{\pgfqpoint{0.000000in}{0.000000in}}%
\pgfpathlineto{\pgfqpoint{-0.069444in}{0.000000in}}%
\pgfusepath{stroke,fill}%
}%
\begin{pgfscope}%
\pgfsys@transformshift{3.858404in}{2.769145in}%
\pgfsys@useobject{currentmarker}{}%
\end{pgfscope}%
\end{pgfscope}%
\begin{pgfscope}%
\pgftext[x=0.566911in,y=2.769145in,right,]{\rmfamily\fontsize{8.000000}{9.600000}\selectfont 20}%
\end{pgfscope}%
\begin{pgfscope}%
\pgftext[x=0.356082in,y=1.821626in,,bottom,rotate=90.000000]{\rmfamily\fontsize{9.000000}{10.800000}\selectfont Candidates}%
\end{pgfscope}%
\begin{pgfscope}%
\pgfsetrectcap%
\pgfsetroundjoin%
\pgfsetlinewidth{1.003750pt}%
\definecolor{currentstroke}{rgb}{1.000000,0.000000,0.000000}%
\pgfsetstrokecolor{currentstroke}%
\pgfsetdash{}{0pt}%
\pgfpathmoveto{\pgfqpoint{0.652547in}{1.015787in}}%
\pgfpathlineto{\pgfqpoint{1.445916in}{1.013202in}}%
\pgfpathlineto{\pgfqpoint{1.526872in}{1.015422in}}%
\pgfpathlineto{\pgfqpoint{1.575445in}{1.018689in}}%
\pgfpathlineto{\pgfqpoint{1.624019in}{1.024903in}}%
\pgfpathlineto{\pgfqpoint{1.656401in}{1.031421in}}%
\pgfpathlineto{\pgfqpoint{1.688783in}{1.040634in}}%
\pgfpathlineto{\pgfqpoint{1.704975in}{1.046361in}}%
\pgfpathlineto{\pgfqpoint{1.737357in}{1.061139in}}%
\pgfpathlineto{\pgfqpoint{1.769739in}{1.080834in}}%
\pgfpathlineto{\pgfqpoint{1.802122in}{1.106466in}}%
\pgfpathlineto{\pgfqpoint{1.834504in}{1.139042in}}%
\pgfpathlineto{\pgfqpoint{1.866887in}{1.179041in}}%
\pgfpathlineto{\pgfqpoint{1.883078in}{1.202390in}}%
\pgfpathlineto{\pgfqpoint{1.899269in}{1.227925in}}%
\pgfpathlineto{\pgfqpoint{1.915460in}{1.255672in}}%
\pgfpathlineto{\pgfqpoint{1.947843in}{1.317746in}}%
\pgfpathlineto{\pgfqpoint{1.980225in}{1.388678in}}%
\pgfpathlineto{\pgfqpoint{2.012607in}{1.466121in}}%
\pgfpathlineto{\pgfqpoint{2.044990in}{1.548871in}}%
\pgfpathlineto{\pgfqpoint{2.142137in}{1.800963in}}%
\pgfpathlineto{\pgfqpoint{2.174519in}{1.875338in}}%
\pgfpathlineto{\pgfqpoint{2.190710in}{1.908264in}}%
\pgfpathlineto{\pgfqpoint{2.206902in}{1.937908in}}%
\pgfpathlineto{\pgfqpoint{2.223093in}{1.963870in}}%
\pgfpathlineto{\pgfqpoint{2.239284in}{1.985790in}}%
\pgfpathlineto{\pgfqpoint{2.255475in}{2.003358in}}%
\pgfpathlineto{\pgfqpoint{2.271666in}{2.016323in}}%
\pgfpathlineto{\pgfqpoint{2.287858in}{2.024496in}}%
\pgfpathlineto{\pgfqpoint{2.304049in}{2.027759in}}%
\pgfpathlineto{\pgfqpoint{2.320240in}{2.026063in}}%
\pgfpathlineto{\pgfqpoint{2.336431in}{2.019431in}}%
\pgfpathlineto{\pgfqpoint{2.352622in}{2.007959in}}%
\pgfpathlineto{\pgfqpoint{2.368814in}{1.991812in}}%
\pgfpathlineto{\pgfqpoint{2.385005in}{1.971219in}}%
\pgfpathlineto{\pgfqpoint{2.401196in}{1.946471in}}%
\pgfpathlineto{\pgfqpoint{2.417387in}{1.917910in}}%
\pgfpathlineto{\pgfqpoint{2.433578in}{1.885926in}}%
\pgfpathlineto{\pgfqpoint{2.465961in}{1.813413in}}%
\pgfpathlineto{\pgfqpoint{2.498343in}{1.732303in}}%
\pgfpathlineto{\pgfqpoint{2.611682in}{1.437406in}}%
\pgfpathlineto{\pgfqpoint{2.644064in}{1.361855in}}%
\pgfpathlineto{\pgfqpoint{2.676446in}{1.293203in}}%
\pgfpathlineto{\pgfqpoint{2.708829in}{1.233699in}}%
\pgfpathlineto{\pgfqpoint{2.725020in}{1.207260in}}%
\pgfpathlineto{\pgfqpoint{2.741211in}{1.183019in}}%
\pgfpathlineto{\pgfqpoint{2.757402in}{1.160933in}}%
\pgfpathlineto{\pgfqpoint{2.789785in}{1.123256in}}%
\pgfpathlineto{\pgfqpoint{2.822167in}{1.092781in}}%
\pgfpathlineto{\pgfqpoint{2.854549in}{1.068928in}}%
\pgfpathlineto{\pgfqpoint{2.886932in}{1.050676in}}%
\pgfpathlineto{\pgfqpoint{2.919314in}{1.037014in}}%
\pgfpathlineto{\pgfqpoint{2.951697in}{1.027002in}}%
\pgfpathlineto{\pgfqpoint{2.984079in}{1.019813in}}%
\pgfpathlineto{\pgfqpoint{3.032653in}{1.012900in}}%
\pgfpathlineto{\pgfqpoint{3.081226in}{1.008851in}}%
\pgfpathlineto{\pgfqpoint{3.145991in}{1.006152in}}%
\pgfpathlineto{\pgfqpoint{3.259329in}{1.004420in}}%
\pgfpathlineto{\pgfqpoint{3.599344in}{1.002751in}}%
\pgfpathlineto{\pgfqpoint{3.842212in}{1.001731in}}%
\pgfpathlineto{\pgfqpoint{3.842212in}{1.001731in}}%
\pgfusepath{stroke}%
\end{pgfscope}%
\begin{pgfscope}%
\pgfpathrectangle{\pgfqpoint{0.636356in}{0.874107in}}{\pgfqpoint{3.222048in}{1.895038in}} %
\pgfusepath{clip}%
\pgfsetbuttcap%
\pgfsetroundjoin%
\pgfsetlinewidth{0.501875pt}%
\definecolor{currentstroke}{rgb}{0.000000,0.000000,0.000000}%
\pgfsetstrokecolor{currentstroke}%
\pgfsetdash{}{0pt}%
\pgfpathmoveto{\pgfqpoint{0.656494in}{1.003661in}}%
\pgfpathlineto{\pgfqpoint{0.656494in}{1.434866in}}%
\pgfusepath{stroke}%
\end{pgfscope}%
\begin{pgfscope}%
\pgfpathrectangle{\pgfqpoint{0.636356in}{0.874107in}}{\pgfqpoint{3.222048in}{1.895038in}} %
\pgfusepath{clip}%
\pgfsetbuttcap%
\pgfsetroundjoin%
\pgfsetlinewidth{0.501875pt}%
\definecolor{currentstroke}{rgb}{0.000000,0.000000,0.000000}%
\pgfsetstrokecolor{currentstroke}%
\pgfsetdash{}{0pt}%
\pgfpathmoveto{\pgfqpoint{0.696769in}{0.941209in}}%
\pgfpathlineto{\pgfqpoint{0.696769in}{1.313553in}}%
\pgfusepath{stroke}%
\end{pgfscope}%
\begin{pgfscope}%
\pgfpathrectangle{\pgfqpoint{0.636356in}{0.874107in}}{\pgfqpoint{3.222048in}{1.895038in}} %
\pgfusepath{clip}%
\pgfsetbuttcap%
\pgfsetroundjoin%
\pgfsetlinewidth{0.501875pt}%
\definecolor{currentstroke}{rgb}{0.000000,0.000000,0.000000}%
\pgfsetstrokecolor{currentstroke}%
\pgfsetdash{}{0pt}%
\pgfpathmoveto{\pgfqpoint{0.737045in}{0.941209in}}%
\pgfpathlineto{\pgfqpoint{0.737045in}{1.313553in}}%
\pgfusepath{stroke}%
\end{pgfscope}%
\begin{pgfscope}%
\pgfpathrectangle{\pgfqpoint{0.636356in}{0.874107in}}{\pgfqpoint{3.222048in}{1.895038in}} %
\pgfusepath{clip}%
\pgfsetbuttcap%
\pgfsetroundjoin%
\pgfsetlinewidth{0.501875pt}%
\definecolor{currentstroke}{rgb}{0.000000,0.000000,0.000000}%
\pgfsetstrokecolor{currentstroke}%
\pgfsetdash{}{0pt}%
\pgfpathmoveto{\pgfqpoint{0.777320in}{0.890476in}}%
\pgfpathlineto{\pgfqpoint{0.777320in}{1.186743in}}%
\pgfusepath{stroke}%
\end{pgfscope}%
\begin{pgfscope}%
\pgfpathrectangle{\pgfqpoint{0.636356in}{0.874107in}}{\pgfqpoint{3.222048in}{1.895038in}} %
\pgfusepath{clip}%
\pgfsetbuttcap%
\pgfsetroundjoin%
\pgfsetlinewidth{0.501875pt}%
\definecolor{currentstroke}{rgb}{0.000000,0.000000,0.000000}%
\pgfsetstrokecolor{currentstroke}%
\pgfsetdash{}{0pt}%
\pgfpathmoveto{\pgfqpoint{0.817596in}{0.874107in}}%
\pgfpathlineto{\pgfqpoint{0.817596in}{0.982870in}}%
\pgfusepath{stroke}%
\end{pgfscope}%
\begin{pgfscope}%
\pgfpathrectangle{\pgfqpoint{0.636356in}{0.874107in}}{\pgfqpoint{3.222048in}{1.895038in}} %
\pgfusepath{clip}%
\pgfsetbuttcap%
\pgfsetroundjoin%
\pgfsetlinewidth{0.501875pt}%
\definecolor{currentstroke}{rgb}{0.000000,0.000000,0.000000}%
\pgfsetstrokecolor{currentstroke}%
\pgfsetdash{}{0pt}%
\pgfpathmoveto{\pgfqpoint{0.857872in}{0.874107in}}%
\pgfpathlineto{\pgfqpoint{0.857872in}{0.982870in}}%
\pgfusepath{stroke}%
\end{pgfscope}%
\begin{pgfscope}%
\pgfpathrectangle{\pgfqpoint{0.636356in}{0.874107in}}{\pgfqpoint{3.222048in}{1.895038in}} %
\pgfusepath{clip}%
\pgfsetbuttcap%
\pgfsetroundjoin%
\pgfsetlinewidth{0.501875pt}%
\definecolor{currentstroke}{rgb}{0.000000,0.000000,0.000000}%
\pgfsetstrokecolor{currentstroke}%
\pgfsetdash{}{0pt}%
\pgfpathmoveto{\pgfqpoint{0.898147in}{0.890476in}}%
\pgfpathlineto{\pgfqpoint{0.898147in}{1.186743in}}%
\pgfusepath{stroke}%
\end{pgfscope}%
\begin{pgfscope}%
\pgfpathrectangle{\pgfqpoint{0.636356in}{0.874107in}}{\pgfqpoint{3.222048in}{1.895038in}} %
\pgfusepath{clip}%
\pgfsetbuttcap%
\pgfsetroundjoin%
\pgfsetlinewidth{0.501875pt}%
\definecolor{currentstroke}{rgb}{0.000000,0.000000,0.000000}%
\pgfsetstrokecolor{currentstroke}%
\pgfsetdash{}{0pt}%
\pgfpathmoveto{\pgfqpoint{0.938423in}{0.941209in}}%
\pgfpathlineto{\pgfqpoint{0.938423in}{1.313553in}}%
\pgfusepath{stroke}%
\end{pgfscope}%
\begin{pgfscope}%
\pgfpathrectangle{\pgfqpoint{0.636356in}{0.874107in}}{\pgfqpoint{3.222048in}{1.895038in}} %
\pgfusepath{clip}%
\pgfsetbuttcap%
\pgfsetroundjoin%
\pgfsetlinewidth{0.501875pt}%
\definecolor{currentstroke}{rgb}{0.000000,0.000000,0.000000}%
\pgfsetstrokecolor{currentstroke}%
\pgfsetdash{}{0pt}%
\pgfpathmoveto{\pgfqpoint{0.978698in}{0.890476in}}%
\pgfpathlineto{\pgfqpoint{0.978698in}{1.186743in}}%
\pgfusepath{stroke}%
\end{pgfscope}%
\begin{pgfscope}%
\pgfpathrectangle{\pgfqpoint{0.636356in}{0.874107in}}{\pgfqpoint{3.222048in}{1.895038in}} %
\pgfusepath{clip}%
\pgfsetbuttcap%
\pgfsetroundjoin%
\pgfsetlinewidth{0.501875pt}%
\definecolor{currentstroke}{rgb}{0.000000,0.000000,0.000000}%
\pgfsetstrokecolor{currentstroke}%
\pgfsetdash{}{0pt}%
\pgfpathmoveto{\pgfqpoint{1.018974in}{0.874107in}}%
\pgfpathlineto{\pgfqpoint{1.018974in}{0.982870in}}%
\pgfusepath{stroke}%
\end{pgfscope}%
\begin{pgfscope}%
\pgfpathrectangle{\pgfqpoint{0.636356in}{0.874107in}}{\pgfqpoint{3.222048in}{1.895038in}} %
\pgfusepath{clip}%
\pgfsetbuttcap%
\pgfsetroundjoin%
\pgfsetlinewidth{0.501875pt}%
\definecolor{currentstroke}{rgb}{0.000000,0.000000,0.000000}%
\pgfsetstrokecolor{currentstroke}%
\pgfsetdash{}{0pt}%
\pgfpathmoveto{\pgfqpoint{1.059250in}{0.941209in}}%
\pgfpathlineto{\pgfqpoint{1.059250in}{1.313553in}}%
\pgfusepath{stroke}%
\end{pgfscope}%
\begin{pgfscope}%
\pgfpathrectangle{\pgfqpoint{0.636356in}{0.874107in}}{\pgfqpoint{3.222048in}{1.895038in}} %
\pgfusepath{clip}%
\pgfsetbuttcap%
\pgfsetroundjoin%
\pgfsetlinewidth{0.501875pt}%
\definecolor{currentstroke}{rgb}{0.000000,0.000000,0.000000}%
\pgfsetstrokecolor{currentstroke}%
\pgfsetdash{}{0pt}%
\pgfpathmoveto{\pgfqpoint{1.099525in}{0.941209in}}%
\pgfpathlineto{\pgfqpoint{1.099525in}{1.313553in}}%
\pgfusepath{stroke}%
\end{pgfscope}%
\begin{pgfscope}%
\pgfpathrectangle{\pgfqpoint{0.636356in}{0.874107in}}{\pgfqpoint{3.222048in}{1.895038in}} %
\pgfusepath{clip}%
\pgfsetbuttcap%
\pgfsetroundjoin%
\pgfsetlinewidth{0.501875pt}%
\definecolor{currentstroke}{rgb}{0.000000,0.000000,0.000000}%
\pgfsetstrokecolor{currentstroke}%
\pgfsetdash{}{0pt}%
\pgfpathmoveto{\pgfqpoint{1.139801in}{0.890476in}}%
\pgfpathlineto{\pgfqpoint{1.139801in}{1.186743in}}%
\pgfusepath{stroke}%
\end{pgfscope}%
\begin{pgfscope}%
\pgfpathrectangle{\pgfqpoint{0.636356in}{0.874107in}}{\pgfqpoint{3.222048in}{1.895038in}} %
\pgfusepath{clip}%
\pgfsetbuttcap%
\pgfsetroundjoin%
\pgfsetlinewidth{0.501875pt}%
\definecolor{currentstroke}{rgb}{0.000000,0.000000,0.000000}%
\pgfsetstrokecolor{currentstroke}%
\pgfsetdash{}{0pt}%
\pgfpathmoveto{\pgfqpoint{1.180076in}{0.890476in}}%
\pgfpathlineto{\pgfqpoint{1.180076in}{1.186743in}}%
\pgfusepath{stroke}%
\end{pgfscope}%
\begin{pgfscope}%
\pgfpathrectangle{\pgfqpoint{0.636356in}{0.874107in}}{\pgfqpoint{3.222048in}{1.895038in}} %
\pgfusepath{clip}%
\pgfsetbuttcap%
\pgfsetroundjoin%
\pgfsetlinewidth{0.501875pt}%
\definecolor{currentstroke}{rgb}{0.000000,0.000000,0.000000}%
\pgfsetstrokecolor{currentstroke}%
\pgfsetdash{}{0pt}%
\pgfpathmoveto{\pgfqpoint{1.220352in}{0.890476in}}%
\pgfpathlineto{\pgfqpoint{1.220352in}{1.186743in}}%
\pgfusepath{stroke}%
\end{pgfscope}%
\begin{pgfscope}%
\pgfpathrectangle{\pgfqpoint{0.636356in}{0.874107in}}{\pgfqpoint{3.222048in}{1.895038in}} %
\pgfusepath{clip}%
\pgfsetbuttcap%
\pgfsetroundjoin%
\pgfsetlinewidth{0.501875pt}%
\definecolor{currentstroke}{rgb}{0.000000,0.000000,0.000000}%
\pgfsetstrokecolor{currentstroke}%
\pgfsetdash{}{0pt}%
\pgfpathmoveto{\pgfqpoint{1.260628in}{0.941209in}}%
\pgfpathlineto{\pgfqpoint{1.260628in}{1.313553in}}%
\pgfusepath{stroke}%
\end{pgfscope}%
\begin{pgfscope}%
\pgfpathrectangle{\pgfqpoint{0.636356in}{0.874107in}}{\pgfqpoint{3.222048in}{1.895038in}} %
\pgfusepath{clip}%
\pgfsetbuttcap%
\pgfsetroundjoin%
\pgfsetlinewidth{0.501875pt}%
\definecolor{currentstroke}{rgb}{0.000000,0.000000,0.000000}%
\pgfsetstrokecolor{currentstroke}%
\pgfsetdash{}{0pt}%
\pgfpathmoveto{\pgfqpoint{1.300903in}{0.874107in}}%
\pgfpathlineto{\pgfqpoint{1.300903in}{0.982870in}}%
\pgfusepath{stroke}%
\end{pgfscope}%
\begin{pgfscope}%
\pgfpathrectangle{\pgfqpoint{0.636356in}{0.874107in}}{\pgfqpoint{3.222048in}{1.895038in}} %
\pgfusepath{clip}%
\pgfsetbuttcap%
\pgfsetroundjoin%
\pgfsetlinewidth{0.501875pt}%
\definecolor{currentstroke}{rgb}{0.000000,0.000000,0.000000}%
\pgfsetstrokecolor{currentstroke}%
\pgfsetdash{}{0pt}%
\pgfpathmoveto{\pgfqpoint{1.341179in}{0.941209in}}%
\pgfpathlineto{\pgfqpoint{1.341179in}{1.313553in}}%
\pgfusepath{stroke}%
\end{pgfscope}%
\begin{pgfscope}%
\pgfpathrectangle{\pgfqpoint{0.636356in}{0.874107in}}{\pgfqpoint{3.222048in}{1.895038in}} %
\pgfusepath{clip}%
\pgfsetbuttcap%
\pgfsetroundjoin%
\pgfsetlinewidth{0.501875pt}%
\definecolor{currentstroke}{rgb}{0.000000,0.000000,0.000000}%
\pgfsetstrokecolor{currentstroke}%
\pgfsetdash{}{0pt}%
\pgfpathmoveto{\pgfqpoint{1.381454in}{1.071727in}}%
\pgfpathlineto{\pgfqpoint{1.381454in}{1.552791in}}%
\pgfusepath{stroke}%
\end{pgfscope}%
\begin{pgfscope}%
\pgfpathrectangle{\pgfqpoint{0.636356in}{0.874107in}}{\pgfqpoint{3.222048in}{1.895038in}} %
\pgfusepath{clip}%
\pgfsetbuttcap%
\pgfsetroundjoin%
\pgfsetlinewidth{0.501875pt}%
\definecolor{currentstroke}{rgb}{0.000000,0.000000,0.000000}%
\pgfsetstrokecolor{currentstroke}%
\pgfsetdash{}{0pt}%
\pgfpathmoveto{\pgfqpoint{1.421730in}{0.890476in}}%
\pgfpathlineto{\pgfqpoint{1.421730in}{1.186743in}}%
\pgfusepath{stroke}%
\end{pgfscope}%
\begin{pgfscope}%
\pgfpathrectangle{\pgfqpoint{0.636356in}{0.874107in}}{\pgfqpoint{3.222048in}{1.895038in}} %
\pgfusepath{clip}%
\pgfsetbuttcap%
\pgfsetroundjoin%
\pgfsetlinewidth{0.501875pt}%
\definecolor{currentstroke}{rgb}{0.000000,0.000000,0.000000}%
\pgfsetstrokecolor{currentstroke}%
\pgfsetdash{}{0pt}%
\pgfpathmoveto{\pgfqpoint{1.462006in}{0.890476in}}%
\pgfpathlineto{\pgfqpoint{1.462006in}{1.186743in}}%
\pgfusepath{stroke}%
\end{pgfscope}%
\begin{pgfscope}%
\pgfpathrectangle{\pgfqpoint{0.636356in}{0.874107in}}{\pgfqpoint{3.222048in}{1.895038in}} %
\pgfusepath{clip}%
\pgfsetbuttcap%
\pgfsetroundjoin%
\pgfsetlinewidth{0.501875pt}%
\definecolor{currentstroke}{rgb}{0.000000,0.000000,0.000000}%
\pgfsetstrokecolor{currentstroke}%
\pgfsetdash{}{0pt}%
\pgfpathmoveto{\pgfqpoint{1.502281in}{0.941209in}}%
\pgfpathlineto{\pgfqpoint{1.502281in}{1.313553in}}%
\pgfusepath{stroke}%
\end{pgfscope}%
\begin{pgfscope}%
\pgfpathrectangle{\pgfqpoint{0.636356in}{0.874107in}}{\pgfqpoint{3.222048in}{1.895038in}} %
\pgfusepath{clip}%
\pgfsetbuttcap%
\pgfsetroundjoin%
\pgfsetlinewidth{0.501875pt}%
\definecolor{currentstroke}{rgb}{0.000000,0.000000,0.000000}%
\pgfsetstrokecolor{currentstroke}%
\pgfsetdash{}{0pt}%
\pgfpathmoveto{\pgfqpoint{1.542557in}{0.874107in}}%
\pgfpathlineto{\pgfqpoint{1.542557in}{0.982870in}}%
\pgfusepath{stroke}%
\end{pgfscope}%
\begin{pgfscope}%
\pgfpathrectangle{\pgfqpoint{0.636356in}{0.874107in}}{\pgfqpoint{3.222048in}{1.895038in}} %
\pgfusepath{clip}%
\pgfsetbuttcap%
\pgfsetroundjoin%
\pgfsetlinewidth{0.501875pt}%
\definecolor{currentstroke}{rgb}{0.000000,0.000000,0.000000}%
\pgfsetstrokecolor{currentstroke}%
\pgfsetdash{}{0pt}%
\pgfpathmoveto{\pgfqpoint{1.582832in}{0.874107in}}%
\pgfpathlineto{\pgfqpoint{1.582832in}{0.982870in}}%
\pgfusepath{stroke}%
\end{pgfscope}%
\begin{pgfscope}%
\pgfpathrectangle{\pgfqpoint{0.636356in}{0.874107in}}{\pgfqpoint{3.222048in}{1.895038in}} %
\pgfusepath{clip}%
\pgfsetbuttcap%
\pgfsetroundjoin%
\pgfsetlinewidth{0.501875pt}%
\definecolor{currentstroke}{rgb}{0.000000,0.000000,0.000000}%
\pgfsetstrokecolor{currentstroke}%
\pgfsetdash{}{0pt}%
\pgfpathmoveto{\pgfqpoint{1.623108in}{0.890476in}}%
\pgfpathlineto{\pgfqpoint{1.623108in}{1.186743in}}%
\pgfusepath{stroke}%
\end{pgfscope}%
\begin{pgfscope}%
\pgfpathrectangle{\pgfqpoint{0.636356in}{0.874107in}}{\pgfqpoint{3.222048in}{1.895038in}} %
\pgfusepath{clip}%
\pgfsetbuttcap%
\pgfsetroundjoin%
\pgfsetlinewidth{0.501875pt}%
\definecolor{currentstroke}{rgb}{0.000000,0.000000,0.000000}%
\pgfsetstrokecolor{currentstroke}%
\pgfsetdash{}{0pt}%
\pgfpathmoveto{\pgfqpoint{1.663384in}{0.874107in}}%
\pgfpathlineto{\pgfqpoint{1.663384in}{0.982870in}}%
\pgfusepath{stroke}%
\end{pgfscope}%
\begin{pgfscope}%
\pgfpathrectangle{\pgfqpoint{0.636356in}{0.874107in}}{\pgfqpoint{3.222048in}{1.895038in}} %
\pgfusepath{clip}%
\pgfsetbuttcap%
\pgfsetroundjoin%
\pgfsetlinewidth{0.501875pt}%
\definecolor{currentstroke}{rgb}{0.000000,0.000000,0.000000}%
\pgfsetstrokecolor{currentstroke}%
\pgfsetdash{}{0pt}%
\pgfpathmoveto{\pgfqpoint{1.703659in}{1.071727in}}%
\pgfpathlineto{\pgfqpoint{1.703659in}{1.552791in}}%
\pgfusepath{stroke}%
\end{pgfscope}%
\begin{pgfscope}%
\pgfpathrectangle{\pgfqpoint{0.636356in}{0.874107in}}{\pgfqpoint{3.222048in}{1.895038in}} %
\pgfusepath{clip}%
\pgfsetbuttcap%
\pgfsetroundjoin%
\pgfsetlinewidth{0.501875pt}%
\definecolor{currentstroke}{rgb}{0.000000,0.000000,0.000000}%
\pgfsetstrokecolor{currentstroke}%
\pgfsetdash{}{0pt}%
\pgfpathmoveto{\pgfqpoint{1.743935in}{0.890476in}}%
\pgfpathlineto{\pgfqpoint{1.743935in}{1.186743in}}%
\pgfusepath{stroke}%
\end{pgfscope}%
\begin{pgfscope}%
\pgfpathrectangle{\pgfqpoint{0.636356in}{0.874107in}}{\pgfqpoint{3.222048in}{1.895038in}} %
\pgfusepath{clip}%
\pgfsetbuttcap%
\pgfsetroundjoin%
\pgfsetlinewidth{0.501875pt}%
\definecolor{currentstroke}{rgb}{0.000000,0.000000,0.000000}%
\pgfsetstrokecolor{currentstroke}%
\pgfsetdash{}{0pt}%
\pgfpathmoveto{\pgfqpoint{1.784210in}{0.941209in}}%
\pgfpathlineto{\pgfqpoint{1.784210in}{1.313553in}}%
\pgfusepath{stroke}%
\end{pgfscope}%
\begin{pgfscope}%
\pgfpathrectangle{\pgfqpoint{0.636356in}{0.874107in}}{\pgfqpoint{3.222048in}{1.895038in}} %
\pgfusepath{clip}%
\pgfsetbuttcap%
\pgfsetroundjoin%
\pgfsetlinewidth{0.501875pt}%
\definecolor{currentstroke}{rgb}{0.000000,0.000000,0.000000}%
\pgfsetstrokecolor{currentstroke}%
\pgfsetdash{}{0pt}%
\pgfpathmoveto{\pgfqpoint{1.824486in}{1.071727in}}%
\pgfpathlineto{\pgfqpoint{1.824486in}{1.552791in}}%
\pgfusepath{stroke}%
\end{pgfscope}%
\begin{pgfscope}%
\pgfpathrectangle{\pgfqpoint{0.636356in}{0.874107in}}{\pgfqpoint{3.222048in}{1.895038in}} %
\pgfusepath{clip}%
\pgfsetbuttcap%
\pgfsetroundjoin%
\pgfsetlinewidth{0.501875pt}%
\definecolor{currentstroke}{rgb}{0.000000,0.000000,0.000000}%
\pgfsetstrokecolor{currentstroke}%
\pgfsetdash{}{0pt}%
\pgfpathmoveto{\pgfqpoint{1.864762in}{1.003661in}}%
\pgfpathlineto{\pgfqpoint{1.864762in}{1.434866in}}%
\pgfusepath{stroke}%
\end{pgfscope}%
\begin{pgfscope}%
\pgfpathrectangle{\pgfqpoint{0.636356in}{0.874107in}}{\pgfqpoint{3.222048in}{1.895038in}} %
\pgfusepath{clip}%
\pgfsetbuttcap%
\pgfsetroundjoin%
\pgfsetlinewidth{0.501875pt}%
\definecolor{currentstroke}{rgb}{0.000000,0.000000,0.000000}%
\pgfsetstrokecolor{currentstroke}%
\pgfsetdash{}{0pt}%
\pgfpathmoveto{\pgfqpoint{1.905037in}{1.143232in}}%
\pgfpathlineto{\pgfqpoint{1.905037in}{1.668362in}}%
\pgfusepath{stroke}%
\end{pgfscope}%
\begin{pgfscope}%
\pgfpathrectangle{\pgfqpoint{0.636356in}{0.874107in}}{\pgfqpoint{3.222048in}{1.895038in}} %
\pgfusepath{clip}%
\pgfsetbuttcap%
\pgfsetroundjoin%
\pgfsetlinewidth{0.501875pt}%
\definecolor{currentstroke}{rgb}{0.000000,0.000000,0.000000}%
\pgfsetstrokecolor{currentstroke}%
\pgfsetdash{}{0pt}%
\pgfpathmoveto{\pgfqpoint{1.945313in}{0.941209in}}%
\pgfpathlineto{\pgfqpoint{1.945313in}{1.313553in}}%
\pgfusepath{stroke}%
\end{pgfscope}%
\begin{pgfscope}%
\pgfpathrectangle{\pgfqpoint{0.636356in}{0.874107in}}{\pgfqpoint{3.222048in}{1.895038in}} %
\pgfusepath{clip}%
\pgfsetbuttcap%
\pgfsetroundjoin%
\pgfsetlinewidth{0.501875pt}%
\definecolor{currentstroke}{rgb}{0.000000,0.000000,0.000000}%
\pgfsetstrokecolor{currentstroke}%
\pgfsetdash{}{0pt}%
\pgfpathmoveto{\pgfqpoint{1.985588in}{1.143232in}}%
\pgfpathlineto{\pgfqpoint{1.985588in}{1.668362in}}%
\pgfusepath{stroke}%
\end{pgfscope}%
\begin{pgfscope}%
\pgfpathrectangle{\pgfqpoint{0.636356in}{0.874107in}}{\pgfqpoint{3.222048in}{1.895038in}} %
\pgfusepath{clip}%
\pgfsetbuttcap%
\pgfsetroundjoin%
\pgfsetlinewidth{0.501875pt}%
\definecolor{currentstroke}{rgb}{0.000000,0.000000,0.000000}%
\pgfsetstrokecolor{currentstroke}%
\pgfsetdash{}{0pt}%
\pgfpathmoveto{\pgfqpoint{2.025864in}{1.292771in}}%
\pgfpathlineto{\pgfqpoint{2.025864in}{1.894611in}}%
\pgfusepath{stroke}%
\end{pgfscope}%
\begin{pgfscope}%
\pgfpathrectangle{\pgfqpoint{0.636356in}{0.874107in}}{\pgfqpoint{3.222048in}{1.895038in}} %
\pgfusepath{clip}%
\pgfsetbuttcap%
\pgfsetroundjoin%
\pgfsetlinewidth{0.501875pt}%
\definecolor{currentstroke}{rgb}{0.000000,0.000000,0.000000}%
\pgfsetstrokecolor{currentstroke}%
\pgfsetdash{}{0pt}%
\pgfpathmoveto{\pgfqpoint{2.066139in}{0.941209in}}%
\pgfpathlineto{\pgfqpoint{2.066139in}{1.313553in}}%
\pgfusepath{stroke}%
\end{pgfscope}%
\begin{pgfscope}%
\pgfpathrectangle{\pgfqpoint{0.636356in}{0.874107in}}{\pgfqpoint{3.222048in}{1.895038in}} %
\pgfusepath{clip}%
\pgfsetbuttcap%
\pgfsetroundjoin%
\pgfsetlinewidth{0.501875pt}%
\definecolor{currentstroke}{rgb}{0.000000,0.000000,0.000000}%
\pgfsetstrokecolor{currentstroke}%
\pgfsetdash{}{0pt}%
\pgfpathmoveto{\pgfqpoint{2.106415in}{1.369812in}}%
\pgfpathlineto{\pgfqpoint{2.106415in}{2.005932in}}%
\pgfusepath{stroke}%
\end{pgfscope}%
\begin{pgfscope}%
\pgfpathrectangle{\pgfqpoint{0.636356in}{0.874107in}}{\pgfqpoint{3.222048in}{1.895038in}} %
\pgfusepath{clip}%
\pgfsetbuttcap%
\pgfsetroundjoin%
\pgfsetlinewidth{0.501875pt}%
\definecolor{currentstroke}{rgb}{0.000000,0.000000,0.000000}%
\pgfsetstrokecolor{currentstroke}%
\pgfsetdash{}{0pt}%
\pgfpathmoveto{\pgfqpoint{2.146691in}{1.447975in}}%
\pgfpathlineto{\pgfqpoint{2.146691in}{2.116324in}}%
\pgfusepath{stroke}%
\end{pgfscope}%
\begin{pgfscope}%
\pgfpathrectangle{\pgfqpoint{0.636356in}{0.874107in}}{\pgfqpoint{3.222048in}{1.895038in}} %
\pgfusepath{clip}%
\pgfsetbuttcap%
\pgfsetroundjoin%
\pgfsetlinewidth{0.501875pt}%
\definecolor{currentstroke}{rgb}{0.000000,0.000000,0.000000}%
\pgfsetstrokecolor{currentstroke}%
\pgfsetdash{}{0pt}%
\pgfpathmoveto{\pgfqpoint{2.186966in}{1.768692in}}%
\pgfpathlineto{\pgfqpoint{2.186966in}{2.550985in}}%
\pgfusepath{stroke}%
\end{pgfscope}%
\begin{pgfscope}%
\pgfpathrectangle{\pgfqpoint{0.636356in}{0.874107in}}{\pgfqpoint{3.222048in}{1.895038in}} %
\pgfusepath{clip}%
\pgfsetbuttcap%
\pgfsetroundjoin%
\pgfsetlinewidth{0.501875pt}%
\definecolor{currentstroke}{rgb}{0.000000,0.000000,0.000000}%
\pgfsetstrokecolor{currentstroke}%
\pgfsetdash{}{0pt}%
\pgfpathmoveto{\pgfqpoint{2.227242in}{1.447975in}}%
\pgfpathlineto{\pgfqpoint{2.227242in}{2.116324in}}%
\pgfusepath{stroke}%
\end{pgfscope}%
\begin{pgfscope}%
\pgfpathrectangle{\pgfqpoint{0.636356in}{0.874107in}}{\pgfqpoint{3.222048in}{1.895038in}} %
\pgfusepath{clip}%
\pgfsetbuttcap%
\pgfsetroundjoin%
\pgfsetlinewidth{0.501875pt}%
\definecolor{currentstroke}{rgb}{0.000000,0.000000,0.000000}%
\pgfsetstrokecolor{currentstroke}%
\pgfsetdash{}{0pt}%
\pgfpathmoveto{\pgfqpoint{2.267517in}{1.932544in}}%
\pgfpathlineto{\pgfqpoint{2.267517in}{2.765235in}}%
\pgfusepath{stroke}%
\end{pgfscope}%
\begin{pgfscope}%
\pgfpathrectangle{\pgfqpoint{0.636356in}{0.874107in}}{\pgfqpoint{3.222048in}{1.895038in}} %
\pgfusepath{clip}%
\pgfsetbuttcap%
\pgfsetroundjoin%
\pgfsetlinewidth{0.501875pt}%
\definecolor{currentstroke}{rgb}{0.000000,0.000000,0.000000}%
\pgfsetstrokecolor{currentstroke}%
\pgfsetdash{}{0pt}%
\pgfpathmoveto{\pgfqpoint{2.307793in}{1.606958in}}%
\pgfpathlineto{\pgfqpoint{2.307793in}{2.334851in}}%
\pgfusepath{stroke}%
\end{pgfscope}%
\begin{pgfscope}%
\pgfpathrectangle{\pgfqpoint{0.636356in}{0.874107in}}{\pgfqpoint{3.222048in}{1.895038in}} %
\pgfusepath{clip}%
\pgfsetbuttcap%
\pgfsetroundjoin%
\pgfsetlinewidth{0.501875pt}%
\definecolor{currentstroke}{rgb}{0.000000,0.000000,0.000000}%
\pgfsetstrokecolor{currentstroke}%
\pgfsetdash{}{0pt}%
\pgfpathmoveto{\pgfqpoint{2.348069in}{1.369812in}}%
\pgfpathlineto{\pgfqpoint{2.348069in}{2.005932in}}%
\pgfusepath{stroke}%
\end{pgfscope}%
\begin{pgfscope}%
\pgfpathrectangle{\pgfqpoint{0.636356in}{0.874107in}}{\pgfqpoint{3.222048in}{1.895038in}} %
\pgfusepath{clip}%
\pgfsetbuttcap%
\pgfsetroundjoin%
\pgfsetlinewidth{0.501875pt}%
\definecolor{currentstroke}{rgb}{0.000000,0.000000,0.000000}%
\pgfsetstrokecolor{currentstroke}%
\pgfsetdash{}{0pt}%
\pgfpathmoveto{\pgfqpoint{2.388344in}{1.606958in}}%
\pgfpathlineto{\pgfqpoint{2.388344in}{2.334851in}}%
\pgfusepath{stroke}%
\end{pgfscope}%
\begin{pgfscope}%
\pgfpathrectangle{\pgfqpoint{0.636356in}{0.874107in}}{\pgfqpoint{3.222048in}{1.895038in}} %
\pgfusepath{clip}%
\pgfsetbuttcap%
\pgfsetroundjoin%
\pgfsetlinewidth{0.501875pt}%
\definecolor{currentstroke}{rgb}{0.000000,0.000000,0.000000}%
\pgfsetstrokecolor{currentstroke}%
\pgfsetdash{}{0pt}%
\pgfpathmoveto{\pgfqpoint{2.428620in}{1.850383in}}%
\pgfpathlineto{\pgfqpoint{2.428620in}{2.658321in}}%
\pgfusepath{stroke}%
\end{pgfscope}%
\begin{pgfscope}%
\pgfpathrectangle{\pgfqpoint{0.636356in}{0.874107in}}{\pgfqpoint{3.222048in}{1.895038in}} %
\pgfusepath{clip}%
\pgfsetbuttcap%
\pgfsetroundjoin%
\pgfsetlinewidth{0.501875pt}%
\definecolor{currentstroke}{rgb}{0.000000,0.000000,0.000000}%
\pgfsetstrokecolor{currentstroke}%
\pgfsetdash{}{0pt}%
\pgfpathmoveto{\pgfqpoint{2.468895in}{1.071727in}}%
\pgfpathlineto{\pgfqpoint{2.468895in}{1.552791in}}%
\pgfusepath{stroke}%
\end{pgfscope}%
\begin{pgfscope}%
\pgfpathrectangle{\pgfqpoint{0.636356in}{0.874107in}}{\pgfqpoint{3.222048in}{1.895038in}} %
\pgfusepath{clip}%
\pgfsetbuttcap%
\pgfsetroundjoin%
\pgfsetlinewidth{0.501875pt}%
\definecolor{currentstroke}{rgb}{0.000000,0.000000,0.000000}%
\pgfsetstrokecolor{currentstroke}%
\pgfsetdash{}{0pt}%
\pgfpathmoveto{\pgfqpoint{2.509171in}{1.447975in}}%
\pgfpathlineto{\pgfqpoint{2.509171in}{2.116324in}}%
\pgfusepath{stroke}%
\end{pgfscope}%
\begin{pgfscope}%
\pgfpathrectangle{\pgfqpoint{0.636356in}{0.874107in}}{\pgfqpoint{3.222048in}{1.895038in}} %
\pgfusepath{clip}%
\pgfsetbuttcap%
\pgfsetroundjoin%
\pgfsetlinewidth{0.501875pt}%
\definecolor{currentstroke}{rgb}{0.000000,0.000000,0.000000}%
\pgfsetstrokecolor{currentstroke}%
\pgfsetdash{}{0pt}%
\pgfpathmoveto{\pgfqpoint{2.549447in}{1.606958in}}%
\pgfpathlineto{\pgfqpoint{2.549447in}{2.334851in}}%
\pgfusepath{stroke}%
\end{pgfscope}%
\begin{pgfscope}%
\pgfpathrectangle{\pgfqpoint{0.636356in}{0.874107in}}{\pgfqpoint{3.222048in}{1.895038in}} %
\pgfusepath{clip}%
\pgfsetbuttcap%
\pgfsetroundjoin%
\pgfsetlinewidth{0.501875pt}%
\definecolor{currentstroke}{rgb}{0.000000,0.000000,0.000000}%
\pgfsetstrokecolor{currentstroke}%
\pgfsetdash{}{0pt}%
\pgfpathmoveto{\pgfqpoint{2.589722in}{1.143232in}}%
\pgfpathlineto{\pgfqpoint{2.589722in}{1.668362in}}%
\pgfusepath{stroke}%
\end{pgfscope}%
\begin{pgfscope}%
\pgfpathrectangle{\pgfqpoint{0.636356in}{0.874107in}}{\pgfqpoint{3.222048in}{1.895038in}} %
\pgfusepath{clip}%
\pgfsetbuttcap%
\pgfsetroundjoin%
\pgfsetlinewidth{0.501875pt}%
\definecolor{currentstroke}{rgb}{0.000000,0.000000,0.000000}%
\pgfsetstrokecolor{currentstroke}%
\pgfsetdash{}{0pt}%
\pgfpathmoveto{\pgfqpoint{2.629998in}{1.003661in}}%
\pgfpathlineto{\pgfqpoint{2.629998in}{1.434866in}}%
\pgfusepath{stroke}%
\end{pgfscope}%
\begin{pgfscope}%
\pgfpathrectangle{\pgfqpoint{0.636356in}{0.874107in}}{\pgfqpoint{3.222048in}{1.895038in}} %
\pgfusepath{clip}%
\pgfsetbuttcap%
\pgfsetroundjoin%
\pgfsetlinewidth{0.501875pt}%
\definecolor{currentstroke}{rgb}{0.000000,0.000000,0.000000}%
\pgfsetstrokecolor{currentstroke}%
\pgfsetdash{}{0pt}%
\pgfpathmoveto{\pgfqpoint{2.670273in}{1.003661in}}%
\pgfpathlineto{\pgfqpoint{2.670273in}{1.434866in}}%
\pgfusepath{stroke}%
\end{pgfscope}%
\begin{pgfscope}%
\pgfpathrectangle{\pgfqpoint{0.636356in}{0.874107in}}{\pgfqpoint{3.222048in}{1.895038in}} %
\pgfusepath{clip}%
\pgfsetbuttcap%
\pgfsetroundjoin%
\pgfsetlinewidth{0.501875pt}%
\definecolor{currentstroke}{rgb}{0.000000,0.000000,0.000000}%
\pgfsetstrokecolor{currentstroke}%
\pgfsetdash{}{0pt}%
\pgfpathmoveto{\pgfqpoint{2.710549in}{1.003661in}}%
\pgfpathlineto{\pgfqpoint{2.710549in}{1.434866in}}%
\pgfusepath{stroke}%
\end{pgfscope}%
\begin{pgfscope}%
\pgfpathrectangle{\pgfqpoint{0.636356in}{0.874107in}}{\pgfqpoint{3.222048in}{1.895038in}} %
\pgfusepath{clip}%
\pgfsetbuttcap%
\pgfsetroundjoin%
\pgfsetlinewidth{0.501875pt}%
\definecolor{currentstroke}{rgb}{0.000000,0.000000,0.000000}%
\pgfsetstrokecolor{currentstroke}%
\pgfsetdash{}{0pt}%
\pgfpathmoveto{\pgfqpoint{2.750825in}{1.003661in}}%
\pgfpathlineto{\pgfqpoint{2.750825in}{1.434866in}}%
\pgfusepath{stroke}%
\end{pgfscope}%
\begin{pgfscope}%
\pgfpathrectangle{\pgfqpoint{0.636356in}{0.874107in}}{\pgfqpoint{3.222048in}{1.895038in}} %
\pgfusepath{clip}%
\pgfsetbuttcap%
\pgfsetroundjoin%
\pgfsetlinewidth{0.501875pt}%
\definecolor{currentstroke}{rgb}{0.000000,0.000000,0.000000}%
\pgfsetstrokecolor{currentstroke}%
\pgfsetdash{}{0pt}%
\pgfpathmoveto{\pgfqpoint{2.791100in}{0.890476in}}%
\pgfpathlineto{\pgfqpoint{2.791100in}{1.186743in}}%
\pgfusepath{stroke}%
\end{pgfscope}%
\begin{pgfscope}%
\pgfpathrectangle{\pgfqpoint{0.636356in}{0.874107in}}{\pgfqpoint{3.222048in}{1.895038in}} %
\pgfusepath{clip}%
\pgfsetbuttcap%
\pgfsetroundjoin%
\pgfsetlinewidth{0.501875pt}%
\definecolor{currentstroke}{rgb}{0.000000,0.000000,0.000000}%
\pgfsetstrokecolor{currentstroke}%
\pgfsetdash{}{0pt}%
\pgfpathmoveto{\pgfqpoint{2.831376in}{0.890476in}}%
\pgfpathlineto{\pgfqpoint{2.831376in}{1.186743in}}%
\pgfusepath{stroke}%
\end{pgfscope}%
\begin{pgfscope}%
\pgfpathrectangle{\pgfqpoint{0.636356in}{0.874107in}}{\pgfqpoint{3.222048in}{1.895038in}} %
\pgfusepath{clip}%
\pgfsetbuttcap%
\pgfsetroundjoin%
\pgfsetlinewidth{0.501875pt}%
\definecolor{currentstroke}{rgb}{0.000000,0.000000,0.000000}%
\pgfsetstrokecolor{currentstroke}%
\pgfsetdash{}{0pt}%
\pgfpathmoveto{\pgfqpoint{2.871651in}{1.003661in}}%
\pgfpathlineto{\pgfqpoint{2.871651in}{1.434866in}}%
\pgfusepath{stroke}%
\end{pgfscope}%
\begin{pgfscope}%
\pgfpathrectangle{\pgfqpoint{0.636356in}{0.874107in}}{\pgfqpoint{3.222048in}{1.895038in}} %
\pgfusepath{clip}%
\pgfsetbuttcap%
\pgfsetroundjoin%
\pgfsetlinewidth{0.501875pt}%
\definecolor{currentstroke}{rgb}{0.000000,0.000000,0.000000}%
\pgfsetstrokecolor{currentstroke}%
\pgfsetdash{}{0pt}%
\pgfpathmoveto{\pgfqpoint{2.911927in}{1.071727in}}%
\pgfpathlineto{\pgfqpoint{2.911927in}{1.552791in}}%
\pgfusepath{stroke}%
\end{pgfscope}%
\begin{pgfscope}%
\pgfpathrectangle{\pgfqpoint{0.636356in}{0.874107in}}{\pgfqpoint{3.222048in}{1.895038in}} %
\pgfusepath{clip}%
\pgfsetbuttcap%
\pgfsetroundjoin%
\pgfsetlinewidth{0.501875pt}%
\definecolor{currentstroke}{rgb}{0.000000,0.000000,0.000000}%
\pgfsetstrokecolor{currentstroke}%
\pgfsetdash{}{0pt}%
\pgfpathmoveto{\pgfqpoint{2.952203in}{0.941209in}}%
\pgfpathlineto{\pgfqpoint{2.952203in}{1.313553in}}%
\pgfusepath{stroke}%
\end{pgfscope}%
\begin{pgfscope}%
\pgfpathrectangle{\pgfqpoint{0.636356in}{0.874107in}}{\pgfqpoint{3.222048in}{1.895038in}} %
\pgfusepath{clip}%
\pgfsetbuttcap%
\pgfsetroundjoin%
\pgfsetlinewidth{0.501875pt}%
\definecolor{currentstroke}{rgb}{0.000000,0.000000,0.000000}%
\pgfsetstrokecolor{currentstroke}%
\pgfsetdash{}{0pt}%
\pgfpathmoveto{\pgfqpoint{2.992478in}{0.941209in}}%
\pgfpathlineto{\pgfqpoint{2.992478in}{1.313553in}}%
\pgfusepath{stroke}%
\end{pgfscope}%
\begin{pgfscope}%
\pgfpathrectangle{\pgfqpoint{0.636356in}{0.874107in}}{\pgfqpoint{3.222048in}{1.895038in}} %
\pgfusepath{clip}%
\pgfsetbuttcap%
\pgfsetroundjoin%
\pgfsetlinewidth{0.501875pt}%
\definecolor{currentstroke}{rgb}{0.000000,0.000000,0.000000}%
\pgfsetstrokecolor{currentstroke}%
\pgfsetdash{}{0pt}%
\pgfpathmoveto{\pgfqpoint{3.032754in}{0.890476in}}%
\pgfpathlineto{\pgfqpoint{3.032754in}{1.186743in}}%
\pgfusepath{stroke}%
\end{pgfscope}%
\begin{pgfscope}%
\pgfpathrectangle{\pgfqpoint{0.636356in}{0.874107in}}{\pgfqpoint{3.222048in}{1.895038in}} %
\pgfusepath{clip}%
\pgfsetbuttcap%
\pgfsetroundjoin%
\pgfsetlinewidth{0.501875pt}%
\definecolor{currentstroke}{rgb}{0.000000,0.000000,0.000000}%
\pgfsetstrokecolor{currentstroke}%
\pgfsetdash{}{0pt}%
\pgfpathmoveto{\pgfqpoint{3.073029in}{1.143232in}}%
\pgfpathlineto{\pgfqpoint{3.073029in}{1.668362in}}%
\pgfusepath{stroke}%
\end{pgfscope}%
\begin{pgfscope}%
\pgfpathrectangle{\pgfqpoint{0.636356in}{0.874107in}}{\pgfqpoint{3.222048in}{1.895038in}} %
\pgfusepath{clip}%
\pgfsetbuttcap%
\pgfsetroundjoin%
\pgfsetlinewidth{0.501875pt}%
\definecolor{currentstroke}{rgb}{0.000000,0.000000,0.000000}%
\pgfsetstrokecolor{currentstroke}%
\pgfsetdash{}{0pt}%
\pgfpathmoveto{\pgfqpoint{3.113305in}{0.941209in}}%
\pgfpathlineto{\pgfqpoint{3.113305in}{1.313553in}}%
\pgfusepath{stroke}%
\end{pgfscope}%
\begin{pgfscope}%
\pgfpathrectangle{\pgfqpoint{0.636356in}{0.874107in}}{\pgfqpoint{3.222048in}{1.895038in}} %
\pgfusepath{clip}%
\pgfsetbuttcap%
\pgfsetroundjoin%
\pgfsetlinewidth{0.501875pt}%
\definecolor{currentstroke}{rgb}{0.000000,0.000000,0.000000}%
\pgfsetstrokecolor{currentstroke}%
\pgfsetdash{}{0pt}%
\pgfpathmoveto{\pgfqpoint{3.153581in}{0.874107in}}%
\pgfpathlineto{\pgfqpoint{3.153581in}{0.982870in}}%
\pgfusepath{stroke}%
\end{pgfscope}%
\begin{pgfscope}%
\pgfpathrectangle{\pgfqpoint{0.636356in}{0.874107in}}{\pgfqpoint{3.222048in}{1.895038in}} %
\pgfusepath{clip}%
\pgfsetbuttcap%
\pgfsetroundjoin%
\pgfsetlinewidth{0.501875pt}%
\definecolor{currentstroke}{rgb}{0.000000,0.000000,0.000000}%
\pgfsetstrokecolor{currentstroke}%
\pgfsetdash{}{0pt}%
\pgfpathmoveto{\pgfqpoint{3.193856in}{0.874107in}}%
\pgfpathlineto{\pgfqpoint{3.193856in}{0.982870in}}%
\pgfusepath{stroke}%
\end{pgfscope}%
\begin{pgfscope}%
\pgfpathrectangle{\pgfqpoint{0.636356in}{0.874107in}}{\pgfqpoint{3.222048in}{1.895038in}} %
\pgfusepath{clip}%
\pgfsetbuttcap%
\pgfsetroundjoin%
\pgfsetlinewidth{0.501875pt}%
\definecolor{currentstroke}{rgb}{0.000000,0.000000,0.000000}%
\pgfsetstrokecolor{currentstroke}%
\pgfsetdash{}{0pt}%
\pgfpathmoveto{\pgfqpoint{3.234132in}{0.890476in}}%
\pgfpathlineto{\pgfqpoint{3.234132in}{1.186743in}}%
\pgfusepath{stroke}%
\end{pgfscope}%
\begin{pgfscope}%
\pgfpathrectangle{\pgfqpoint{0.636356in}{0.874107in}}{\pgfqpoint{3.222048in}{1.895038in}} %
\pgfusepath{clip}%
\pgfsetbuttcap%
\pgfsetroundjoin%
\pgfsetlinewidth{0.501875pt}%
\definecolor{currentstroke}{rgb}{0.000000,0.000000,0.000000}%
\pgfsetstrokecolor{currentstroke}%
\pgfsetdash{}{0pt}%
\pgfpathmoveto{\pgfqpoint{3.274407in}{0.890476in}}%
\pgfpathlineto{\pgfqpoint{3.274407in}{1.186743in}}%
\pgfusepath{stroke}%
\end{pgfscope}%
\begin{pgfscope}%
\pgfpathrectangle{\pgfqpoint{0.636356in}{0.874107in}}{\pgfqpoint{3.222048in}{1.895038in}} %
\pgfusepath{clip}%
\pgfsetbuttcap%
\pgfsetroundjoin%
\pgfsetlinewidth{0.501875pt}%
\definecolor{currentstroke}{rgb}{0.000000,0.000000,0.000000}%
\pgfsetstrokecolor{currentstroke}%
\pgfsetdash{}{0pt}%
\pgfpathmoveto{\pgfqpoint{3.314683in}{0.941209in}}%
\pgfpathlineto{\pgfqpoint{3.314683in}{1.313553in}}%
\pgfusepath{stroke}%
\end{pgfscope}%
\begin{pgfscope}%
\pgfpathrectangle{\pgfqpoint{0.636356in}{0.874107in}}{\pgfqpoint{3.222048in}{1.895038in}} %
\pgfusepath{clip}%
\pgfsetbuttcap%
\pgfsetroundjoin%
\pgfsetlinewidth{0.501875pt}%
\definecolor{currentstroke}{rgb}{0.000000,0.000000,0.000000}%
\pgfsetstrokecolor{currentstroke}%
\pgfsetdash{}{0pt}%
\pgfpathmoveto{\pgfqpoint{3.354959in}{0.874107in}}%
\pgfpathlineto{\pgfqpoint{3.354959in}{0.982870in}}%
\pgfusepath{stroke}%
\end{pgfscope}%
\begin{pgfscope}%
\pgfpathrectangle{\pgfqpoint{0.636356in}{0.874107in}}{\pgfqpoint{3.222048in}{1.895038in}} %
\pgfusepath{clip}%
\pgfsetbuttcap%
\pgfsetroundjoin%
\pgfsetlinewidth{0.501875pt}%
\definecolor{currentstroke}{rgb}{0.000000,0.000000,0.000000}%
\pgfsetstrokecolor{currentstroke}%
\pgfsetdash{}{0pt}%
\pgfpathmoveto{\pgfqpoint{3.395234in}{0.890476in}}%
\pgfpathlineto{\pgfqpoint{3.395234in}{1.186743in}}%
\pgfusepath{stroke}%
\end{pgfscope}%
\begin{pgfscope}%
\pgfpathrectangle{\pgfqpoint{0.636356in}{0.874107in}}{\pgfqpoint{3.222048in}{1.895038in}} %
\pgfusepath{clip}%
\pgfsetbuttcap%
\pgfsetroundjoin%
\pgfsetlinewidth{0.501875pt}%
\definecolor{currentstroke}{rgb}{0.000000,0.000000,0.000000}%
\pgfsetstrokecolor{currentstroke}%
\pgfsetdash{}{0pt}%
\pgfpathmoveto{\pgfqpoint{3.435510in}{0.874107in}}%
\pgfpathlineto{\pgfqpoint{3.435510in}{0.982870in}}%
\pgfusepath{stroke}%
\end{pgfscope}%
\begin{pgfscope}%
\pgfpathrectangle{\pgfqpoint{0.636356in}{0.874107in}}{\pgfqpoint{3.222048in}{1.895038in}} %
\pgfusepath{clip}%
\pgfsetbuttcap%
\pgfsetroundjoin%
\pgfsetlinewidth{0.501875pt}%
\definecolor{currentstroke}{rgb}{0.000000,0.000000,0.000000}%
\pgfsetstrokecolor{currentstroke}%
\pgfsetdash{}{0pt}%
\pgfpathmoveto{\pgfqpoint{3.475785in}{0.941209in}}%
\pgfpathlineto{\pgfqpoint{3.475785in}{1.313553in}}%
\pgfusepath{stroke}%
\end{pgfscope}%
\begin{pgfscope}%
\pgfpathrectangle{\pgfqpoint{0.636356in}{0.874107in}}{\pgfqpoint{3.222048in}{1.895038in}} %
\pgfusepath{clip}%
\pgfsetbuttcap%
\pgfsetroundjoin%
\pgfsetlinewidth{0.501875pt}%
\definecolor{currentstroke}{rgb}{0.000000,0.000000,0.000000}%
\pgfsetstrokecolor{currentstroke}%
\pgfsetdash{}{0pt}%
\pgfpathmoveto{\pgfqpoint{3.516061in}{0.890476in}}%
\pgfpathlineto{\pgfqpoint{3.516061in}{1.186743in}}%
\pgfusepath{stroke}%
\end{pgfscope}%
\begin{pgfscope}%
\pgfpathrectangle{\pgfqpoint{0.636356in}{0.874107in}}{\pgfqpoint{3.222048in}{1.895038in}} %
\pgfusepath{clip}%
\pgfsetbuttcap%
\pgfsetroundjoin%
\pgfsetlinewidth{0.501875pt}%
\definecolor{currentstroke}{rgb}{0.000000,0.000000,0.000000}%
\pgfsetstrokecolor{currentstroke}%
\pgfsetdash{}{0pt}%
\pgfpathmoveto{\pgfqpoint{3.556337in}{0.890476in}}%
\pgfpathlineto{\pgfqpoint{3.556337in}{1.186743in}}%
\pgfusepath{stroke}%
\end{pgfscope}%
\begin{pgfscope}%
\pgfpathrectangle{\pgfqpoint{0.636356in}{0.874107in}}{\pgfqpoint{3.222048in}{1.895038in}} %
\pgfusepath{clip}%
\pgfsetbuttcap%
\pgfsetroundjoin%
\pgfsetlinewidth{0.501875pt}%
\definecolor{currentstroke}{rgb}{0.000000,0.000000,0.000000}%
\pgfsetstrokecolor{currentstroke}%
\pgfsetdash{}{0pt}%
\pgfpathmoveto{\pgfqpoint{3.596612in}{0.890476in}}%
\pgfpathlineto{\pgfqpoint{3.596612in}{1.186743in}}%
\pgfusepath{stroke}%
\end{pgfscope}%
\begin{pgfscope}%
\pgfpathrectangle{\pgfqpoint{0.636356in}{0.874107in}}{\pgfqpoint{3.222048in}{1.895038in}} %
\pgfusepath{clip}%
\pgfsetbuttcap%
\pgfsetroundjoin%
\pgfsetlinewidth{0.501875pt}%
\definecolor{currentstroke}{rgb}{0.000000,0.000000,0.000000}%
\pgfsetstrokecolor{currentstroke}%
\pgfsetdash{}{0pt}%
\pgfpathmoveto{\pgfqpoint{3.636888in}{1.003661in}}%
\pgfpathlineto{\pgfqpoint{3.636888in}{1.434866in}}%
\pgfusepath{stroke}%
\end{pgfscope}%
\begin{pgfscope}%
\pgfpathrectangle{\pgfqpoint{0.636356in}{0.874107in}}{\pgfqpoint{3.222048in}{1.895038in}} %
\pgfusepath{clip}%
\pgfsetbuttcap%
\pgfsetroundjoin%
\pgfsetlinewidth{0.501875pt}%
\definecolor{currentstroke}{rgb}{0.000000,0.000000,0.000000}%
\pgfsetstrokecolor{currentstroke}%
\pgfsetdash{}{0pt}%
\pgfpathmoveto{\pgfqpoint{3.677163in}{0.874107in}}%
\pgfpathlineto{\pgfqpoint{3.677163in}{0.982870in}}%
\pgfusepath{stroke}%
\end{pgfscope}%
\begin{pgfscope}%
\pgfpathrectangle{\pgfqpoint{0.636356in}{0.874107in}}{\pgfqpoint{3.222048in}{1.895038in}} %
\pgfusepath{clip}%
\pgfsetbuttcap%
\pgfsetroundjoin%
\pgfsetlinewidth{0.501875pt}%
\definecolor{currentstroke}{rgb}{0.000000,0.000000,0.000000}%
\pgfsetstrokecolor{currentstroke}%
\pgfsetdash{}{0pt}%
\pgfpathmoveto{\pgfqpoint{3.717439in}{0.874107in}}%
\pgfpathlineto{\pgfqpoint{3.717439in}{0.982870in}}%
\pgfusepath{stroke}%
\end{pgfscope}%
\begin{pgfscope}%
\pgfpathrectangle{\pgfqpoint{0.636356in}{0.874107in}}{\pgfqpoint{3.222048in}{1.895038in}} %
\pgfusepath{clip}%
\pgfsetbuttcap%
\pgfsetroundjoin%
\pgfsetlinewidth{0.501875pt}%
\definecolor{currentstroke}{rgb}{0.000000,0.000000,0.000000}%
\pgfsetstrokecolor{currentstroke}%
\pgfsetdash{}{0pt}%
\pgfpathmoveto{\pgfqpoint{3.757715in}{0.874107in}}%
\pgfpathlineto{\pgfqpoint{3.757715in}{0.982870in}}%
\pgfusepath{stroke}%
\end{pgfscope}%
\begin{pgfscope}%
\pgfpathrectangle{\pgfqpoint{0.636356in}{0.874107in}}{\pgfqpoint{3.222048in}{1.895038in}} %
\pgfusepath{clip}%
\pgfsetbuttcap%
\pgfsetroundjoin%
\pgfsetlinewidth{0.501875pt}%
\definecolor{currentstroke}{rgb}{0.000000,0.000000,0.000000}%
\pgfsetstrokecolor{currentstroke}%
\pgfsetdash{}{0pt}%
\pgfpathmoveto{\pgfqpoint{3.797990in}{0.941209in}}%
\pgfpathlineto{\pgfqpoint{3.797990in}{1.313553in}}%
\pgfusepath{stroke}%
\end{pgfscope}%
\begin{pgfscope}%
\pgfpathrectangle{\pgfqpoint{0.636356in}{0.874107in}}{\pgfqpoint{3.222048in}{1.895038in}} %
\pgfusepath{clip}%
\pgfsetbuttcap%
\pgfsetroundjoin%
\pgfsetlinewidth{0.501875pt}%
\definecolor{currentstroke}{rgb}{0.000000,0.000000,0.000000}%
\pgfsetstrokecolor{currentstroke}%
\pgfsetdash{}{0pt}%
\pgfpathmoveto{\pgfqpoint{3.838266in}{0.941209in}}%
\pgfpathlineto{\pgfqpoint{3.838266in}{1.313553in}}%
\pgfusepath{stroke}%
\end{pgfscope}%
\begin{pgfscope}%
\pgfsetbuttcap%
\pgfsetroundjoin%
\definecolor{currentfill}{rgb}{0.000000,0.000000,0.000000}%
\pgfsetfillcolor{currentfill}%
\pgfsetlinewidth{1.003750pt}%
\definecolor{currentstroke}{rgb}{0.000000,0.000000,0.000000}%
\pgfsetstrokecolor{currentstroke}%
\pgfsetdash{}{0pt}%
\pgfsys@defobject{currentmarker}{\pgfqpoint{-0.006944in}{-0.006944in}}{\pgfqpoint{0.006944in}{0.006944in}}{%
\pgfpathmoveto{\pgfqpoint{0.000000in}{-0.006944in}}%
\pgfpathcurveto{\pgfqpoint{0.001842in}{-0.006944in}}{\pgfqpoint{0.003608in}{-0.006213in}}{\pgfqpoint{0.004910in}{-0.004910in}}%
\pgfpathcurveto{\pgfqpoint{0.006213in}{-0.003608in}}{\pgfqpoint{0.006944in}{-0.001842in}}{\pgfqpoint{0.006944in}{0.000000in}}%
\pgfpathcurveto{\pgfqpoint{0.006944in}{0.001842in}}{\pgfqpoint{0.006213in}{0.003608in}}{\pgfqpoint{0.004910in}{0.004910in}}%
\pgfpathcurveto{\pgfqpoint{0.003608in}{0.006213in}}{\pgfqpoint{0.001842in}{0.006944in}}{\pgfqpoint{0.000000in}{0.006944in}}%
\pgfpathcurveto{\pgfqpoint{-0.001842in}{0.006944in}}{\pgfqpoint{-0.003608in}{0.006213in}}{\pgfqpoint{-0.004910in}{0.004910in}}%
\pgfpathcurveto{\pgfqpoint{-0.006213in}{0.003608in}}{\pgfqpoint{-0.006944in}{0.001842in}}{\pgfqpoint{-0.006944in}{0.000000in}}%
\pgfpathcurveto{\pgfqpoint{-0.006944in}{-0.001842in}}{\pgfqpoint{-0.006213in}{-0.003608in}}{\pgfqpoint{-0.004910in}{-0.004910in}}%
\pgfpathcurveto{\pgfqpoint{-0.003608in}{-0.006213in}}{\pgfqpoint{-0.001842in}{-0.006944in}}{\pgfqpoint{0.000000in}{-0.006944in}}%
\pgfpathclose%
\pgfusepath{stroke,fill}%
}%
\begin{pgfscope}%
\pgfsys@transformshift{0.656494in}{1.158363in}%
\pgfsys@useobject{currentmarker}{}%
\end{pgfscope}%
\begin{pgfscope}%
\pgfsys@transformshift{0.696769in}{1.063611in}%
\pgfsys@useobject{currentmarker}{}%
\end{pgfscope}%
\begin{pgfscope}%
\pgfsys@transformshift{0.737045in}{1.063611in}%
\pgfsys@useobject{currentmarker}{}%
\end{pgfscope}%
\begin{pgfscope}%
\pgfsys@transformshift{0.777320in}{0.968859in}%
\pgfsys@useobject{currentmarker}{}%
\end{pgfscope}%
\begin{pgfscope}%
\pgfsys@transformshift{0.817596in}{0.874107in}%
\pgfsys@useobject{currentmarker}{}%
\end{pgfscope}%
\begin{pgfscope}%
\pgfsys@transformshift{0.857872in}{0.874107in}%
\pgfsys@useobject{currentmarker}{}%
\end{pgfscope}%
\begin{pgfscope}%
\pgfsys@transformshift{0.898147in}{0.968859in}%
\pgfsys@useobject{currentmarker}{}%
\end{pgfscope}%
\begin{pgfscope}%
\pgfsys@transformshift{0.938423in}{1.063611in}%
\pgfsys@useobject{currentmarker}{}%
\end{pgfscope}%
\begin{pgfscope}%
\pgfsys@transformshift{0.978698in}{0.968859in}%
\pgfsys@useobject{currentmarker}{}%
\end{pgfscope}%
\begin{pgfscope}%
\pgfsys@transformshift{1.018974in}{0.874107in}%
\pgfsys@useobject{currentmarker}{}%
\end{pgfscope}%
\begin{pgfscope}%
\pgfsys@transformshift{1.059250in}{1.063611in}%
\pgfsys@useobject{currentmarker}{}%
\end{pgfscope}%
\begin{pgfscope}%
\pgfsys@transformshift{1.099525in}{1.063611in}%
\pgfsys@useobject{currentmarker}{}%
\end{pgfscope}%
\begin{pgfscope}%
\pgfsys@transformshift{1.139801in}{0.968859in}%
\pgfsys@useobject{currentmarker}{}%
\end{pgfscope}%
\begin{pgfscope}%
\pgfsys@transformshift{1.180076in}{0.968859in}%
\pgfsys@useobject{currentmarker}{}%
\end{pgfscope}%
\begin{pgfscope}%
\pgfsys@transformshift{1.220352in}{0.968859in}%
\pgfsys@useobject{currentmarker}{}%
\end{pgfscope}%
\begin{pgfscope}%
\pgfsys@transformshift{1.260628in}{1.063611in}%
\pgfsys@useobject{currentmarker}{}%
\end{pgfscope}%
\begin{pgfscope}%
\pgfsys@transformshift{1.300903in}{0.874107in}%
\pgfsys@useobject{currentmarker}{}%
\end{pgfscope}%
\begin{pgfscope}%
\pgfsys@transformshift{1.341179in}{1.063611in}%
\pgfsys@useobject{currentmarker}{}%
\end{pgfscope}%
\begin{pgfscope}%
\pgfsys@transformshift{1.381454in}{1.253114in}%
\pgfsys@useobject{currentmarker}{}%
\end{pgfscope}%
\begin{pgfscope}%
\pgfsys@transformshift{1.421730in}{0.968859in}%
\pgfsys@useobject{currentmarker}{}%
\end{pgfscope}%
\begin{pgfscope}%
\pgfsys@transformshift{1.462006in}{0.968859in}%
\pgfsys@useobject{currentmarker}{}%
\end{pgfscope}%
\begin{pgfscope}%
\pgfsys@transformshift{1.502281in}{1.063611in}%
\pgfsys@useobject{currentmarker}{}%
\end{pgfscope}%
\begin{pgfscope}%
\pgfsys@transformshift{1.542557in}{0.874107in}%
\pgfsys@useobject{currentmarker}{}%
\end{pgfscope}%
\begin{pgfscope}%
\pgfsys@transformshift{1.582832in}{0.874107in}%
\pgfsys@useobject{currentmarker}{}%
\end{pgfscope}%
\begin{pgfscope}%
\pgfsys@transformshift{1.623108in}{0.968859in}%
\pgfsys@useobject{currentmarker}{}%
\end{pgfscope}%
\begin{pgfscope}%
\pgfsys@transformshift{1.663384in}{0.874107in}%
\pgfsys@useobject{currentmarker}{}%
\end{pgfscope}%
\begin{pgfscope}%
\pgfsys@transformshift{1.703659in}{1.253114in}%
\pgfsys@useobject{currentmarker}{}%
\end{pgfscope}%
\begin{pgfscope}%
\pgfsys@transformshift{1.743935in}{0.968859in}%
\pgfsys@useobject{currentmarker}{}%
\end{pgfscope}%
\begin{pgfscope}%
\pgfsys@transformshift{1.784210in}{1.063611in}%
\pgfsys@useobject{currentmarker}{}%
\end{pgfscope}%
\begin{pgfscope}%
\pgfsys@transformshift{1.824486in}{1.253114in}%
\pgfsys@useobject{currentmarker}{}%
\end{pgfscope}%
\begin{pgfscope}%
\pgfsys@transformshift{1.864762in}{1.158363in}%
\pgfsys@useobject{currentmarker}{}%
\end{pgfscope}%
\begin{pgfscope}%
\pgfsys@transformshift{1.905037in}{1.347866in}%
\pgfsys@useobject{currentmarker}{}%
\end{pgfscope}%
\begin{pgfscope}%
\pgfsys@transformshift{1.945313in}{1.063611in}%
\pgfsys@useobject{currentmarker}{}%
\end{pgfscope}%
\begin{pgfscope}%
\pgfsys@transformshift{1.985588in}{1.347866in}%
\pgfsys@useobject{currentmarker}{}%
\end{pgfscope}%
\begin{pgfscope}%
\pgfsys@transformshift{2.025864in}{1.537370in}%
\pgfsys@useobject{currentmarker}{}%
\end{pgfscope}%
\begin{pgfscope}%
\pgfsys@transformshift{2.066139in}{1.063611in}%
\pgfsys@useobject{currentmarker}{}%
\end{pgfscope}%
\begin{pgfscope}%
\pgfsys@transformshift{2.106415in}{1.632122in}%
\pgfsys@useobject{currentmarker}{}%
\end{pgfscope}%
\begin{pgfscope}%
\pgfsys@transformshift{2.146691in}{1.726874in}%
\pgfsys@useobject{currentmarker}{}%
\end{pgfscope}%
\begin{pgfscope}%
\pgfsys@transformshift{2.186966in}{2.105881in}%
\pgfsys@useobject{currentmarker}{}%
\end{pgfscope}%
\begin{pgfscope}%
\pgfsys@transformshift{2.227242in}{1.726874in}%
\pgfsys@useobject{currentmarker}{}%
\end{pgfscope}%
\begin{pgfscope}%
\pgfsys@transformshift{2.267517in}{2.295385in}%
\pgfsys@useobject{currentmarker}{}%
\end{pgfscope}%
\begin{pgfscope}%
\pgfsys@transformshift{2.307793in}{1.916378in}%
\pgfsys@useobject{currentmarker}{}%
\end{pgfscope}%
\begin{pgfscope}%
\pgfsys@transformshift{2.348069in}{1.632122in}%
\pgfsys@useobject{currentmarker}{}%
\end{pgfscope}%
\begin{pgfscope}%
\pgfsys@transformshift{2.388344in}{1.916378in}%
\pgfsys@useobject{currentmarker}{}%
\end{pgfscope}%
\begin{pgfscope}%
\pgfsys@transformshift{2.428620in}{2.200633in}%
\pgfsys@useobject{currentmarker}{}%
\end{pgfscope}%
\begin{pgfscope}%
\pgfsys@transformshift{2.468895in}{1.253114in}%
\pgfsys@useobject{currentmarker}{}%
\end{pgfscope}%
\begin{pgfscope}%
\pgfsys@transformshift{2.509171in}{1.726874in}%
\pgfsys@useobject{currentmarker}{}%
\end{pgfscope}%
\begin{pgfscope}%
\pgfsys@transformshift{2.549447in}{1.916378in}%
\pgfsys@useobject{currentmarker}{}%
\end{pgfscope}%
\begin{pgfscope}%
\pgfsys@transformshift{2.589722in}{1.347866in}%
\pgfsys@useobject{currentmarker}{}%
\end{pgfscope}%
\begin{pgfscope}%
\pgfsys@transformshift{2.629998in}{1.158363in}%
\pgfsys@useobject{currentmarker}{}%
\end{pgfscope}%
\begin{pgfscope}%
\pgfsys@transformshift{2.670273in}{1.158363in}%
\pgfsys@useobject{currentmarker}{}%
\end{pgfscope}%
\begin{pgfscope}%
\pgfsys@transformshift{2.710549in}{1.158363in}%
\pgfsys@useobject{currentmarker}{}%
\end{pgfscope}%
\begin{pgfscope}%
\pgfsys@transformshift{2.750825in}{1.158363in}%
\pgfsys@useobject{currentmarker}{}%
\end{pgfscope}%
\begin{pgfscope}%
\pgfsys@transformshift{2.791100in}{0.968859in}%
\pgfsys@useobject{currentmarker}{}%
\end{pgfscope}%
\begin{pgfscope}%
\pgfsys@transformshift{2.831376in}{0.968859in}%
\pgfsys@useobject{currentmarker}{}%
\end{pgfscope}%
\begin{pgfscope}%
\pgfsys@transformshift{2.871651in}{1.158363in}%
\pgfsys@useobject{currentmarker}{}%
\end{pgfscope}%
\begin{pgfscope}%
\pgfsys@transformshift{2.911927in}{1.253114in}%
\pgfsys@useobject{currentmarker}{}%
\end{pgfscope}%
\begin{pgfscope}%
\pgfsys@transformshift{2.952203in}{1.063611in}%
\pgfsys@useobject{currentmarker}{}%
\end{pgfscope}%
\begin{pgfscope}%
\pgfsys@transformshift{2.992478in}{1.063611in}%
\pgfsys@useobject{currentmarker}{}%
\end{pgfscope}%
\begin{pgfscope}%
\pgfsys@transformshift{3.032754in}{0.968859in}%
\pgfsys@useobject{currentmarker}{}%
\end{pgfscope}%
\begin{pgfscope}%
\pgfsys@transformshift{3.073029in}{1.347866in}%
\pgfsys@useobject{currentmarker}{}%
\end{pgfscope}%
\begin{pgfscope}%
\pgfsys@transformshift{3.113305in}{1.063611in}%
\pgfsys@useobject{currentmarker}{}%
\end{pgfscope}%
\begin{pgfscope}%
\pgfsys@transformshift{3.153581in}{0.874107in}%
\pgfsys@useobject{currentmarker}{}%
\end{pgfscope}%
\begin{pgfscope}%
\pgfsys@transformshift{3.193856in}{0.874107in}%
\pgfsys@useobject{currentmarker}{}%
\end{pgfscope}%
\begin{pgfscope}%
\pgfsys@transformshift{3.234132in}{0.968859in}%
\pgfsys@useobject{currentmarker}{}%
\end{pgfscope}%
\begin{pgfscope}%
\pgfsys@transformshift{3.274407in}{0.968859in}%
\pgfsys@useobject{currentmarker}{}%
\end{pgfscope}%
\begin{pgfscope}%
\pgfsys@transformshift{3.314683in}{1.063611in}%
\pgfsys@useobject{currentmarker}{}%
\end{pgfscope}%
\begin{pgfscope}%
\pgfsys@transformshift{3.354959in}{0.874107in}%
\pgfsys@useobject{currentmarker}{}%
\end{pgfscope}%
\begin{pgfscope}%
\pgfsys@transformshift{3.395234in}{0.968859in}%
\pgfsys@useobject{currentmarker}{}%
\end{pgfscope}%
\begin{pgfscope}%
\pgfsys@transformshift{3.435510in}{0.874107in}%
\pgfsys@useobject{currentmarker}{}%
\end{pgfscope}%
\begin{pgfscope}%
\pgfsys@transformshift{3.475785in}{1.063611in}%
\pgfsys@useobject{currentmarker}{}%
\end{pgfscope}%
\begin{pgfscope}%
\pgfsys@transformshift{3.516061in}{0.968859in}%
\pgfsys@useobject{currentmarker}{}%
\end{pgfscope}%
\begin{pgfscope}%
\pgfsys@transformshift{3.556337in}{0.968859in}%
\pgfsys@useobject{currentmarker}{}%
\end{pgfscope}%
\begin{pgfscope}%
\pgfsys@transformshift{3.596612in}{0.968859in}%
\pgfsys@useobject{currentmarker}{}%
\end{pgfscope}%
\begin{pgfscope}%
\pgfsys@transformshift{3.636888in}{1.158363in}%
\pgfsys@useobject{currentmarker}{}%
\end{pgfscope}%
\begin{pgfscope}%
\pgfsys@transformshift{3.677163in}{0.874107in}%
\pgfsys@useobject{currentmarker}{}%
\end{pgfscope}%
\begin{pgfscope}%
\pgfsys@transformshift{3.717439in}{0.874107in}%
\pgfsys@useobject{currentmarker}{}%
\end{pgfscope}%
\begin{pgfscope}%
\pgfsys@transformshift{3.757715in}{0.874107in}%
\pgfsys@useobject{currentmarker}{}%
\end{pgfscope}%
\begin{pgfscope}%
\pgfsys@transformshift{3.797990in}{1.063611in}%
\pgfsys@useobject{currentmarker}{}%
\end{pgfscope}%
\begin{pgfscope}%
\pgfsys@transformshift{3.838266in}{1.063611in}%
\pgfsys@useobject{currentmarker}{}%
\end{pgfscope}%
\end{pgfscope}%
\end{pgfpicture}%
\makeatother%
\endgroup%

  \end{subfigure}
  \caption{
    Fit to the background distribution in the $\PBzero\to\APDzero\APmuon\Pmuon$ dataset.
    Both the reconstructed \PBzero mass (top) and reconstructed \APDzero mass (bottom) are described by the model.
    The total background model (red, solid) is plotted together with the flat (red, dashed) and peaking (blue, dashed) components.
  }
  \label{fig:bkgfitb}
\end{figure}

\begin{table}
  \centering
  \caption{
    Estimated parameters of the background model from a fit to the lower and upper mass sidebands of the $\PBzero\to\APDzero\APmuon\Pmuon$ dataset.
  }
  \begin{tabular}{l l S[table-format=4.4,table-figures-uncertainty=1]}
    \toprule
    Dim. & Parameter & {Estimate} \\
    \midrule
    $B^0$ & $λ_\text{flat}$ & 0.0030 \pm 0.0004 \\
          & $λ_\text{peaking}$ & 0.0039 \pm 0.0005 \\
    \midrule
    $\overline{D}^0$ & $λ_\text{flat}$ & 0.0008 \pm 0.0027 \\
                     & $\mu_\text{peaking}$ & 1867 \pm 1 \\
                     & $\sigma_\text{peaking}$ & 9 \pm 1 \\
    \midrule
    shared & $f_\text{flat/peaking}$ & 0.42 \pm 0.05 \\
    \bottomrule
  \end{tabular}
  \label{tab:bkgfit}
\end{table}

\section{Normalization decay}
\label{normfit}

The normalization channel $B^0\to\PJpsi\PKstar$ is also analyzed statistically.
This is needed to extract the normalization yield and also to obtain a pure data sample of the normalization decay for studying systematic uncertainties of the analysis.
The pure normalization sample is obtained by applying the sPlot method \cite{SPlot} to the fitted mass distribution.
This is used to obtain weights (called \emph{sweights}) that are used to project out the normalization distribution in other variables.

In contrast to the model for the signal decay channel, the model for the normalization decay only makes use of the reconstructed $B^0$ mass $m(\PKplus\Ppiminus\APmuon\Pmuon)$.
A version of the reconstructed $B^0$ mass is used in which the dimuon mass has been constrained to the nominal \PJpsi mass during the decay tree fit.
This improves the mass resolution, which allows one to avoid describing the $B^0_s\to\PJpsi\PKstar$ background (to the right of the signal peak) and partially reconstructed backgrounds (left of the peak).

The signal model consists of a sum of two \emph{Crystal Ball} (CB) functions\cite{CrystalBall}.
The Crystal Ball function is a modified normal distribution (with parameters $\mu$ and $\sigma$) that changes into a power law at a switchpoint $\alpha$.
The power law is parameterized using the parameter $n$.
One of the two CB functions is used with a left-sided tail ($\alpha > 0$) in order to describe the radiative tail of the $\PBzero\to\PJpsi\PKstar$ mass distribution, while the other CB function is right-sided ($\alpha < 0$) in order to take into account the effect of the event-dependent mass resolution more precisely.

A fit of the signal model to simulated $\PBzero\to\PJpsi\PKstar$ decays after preselection is conducted in order to fix the tail parameters of the model, as well as the fraction between the two CB functions.
See figure \ref{fig:normmcfit} for a plot of the fit.
The determined parameters are given in table \ref{tab:normmcfit}

The background model consists of a single exponential function.
An Extended Likelihood fit to the normalization data sample has been performed (see figure \ref{fig:normdatafit}).
Here, the parameters $n$ and $\alpha$ of the two CB functions, as well as the fraction between the two components is fixed using the values in table \ref{tab:normmcfit}.
The resulting parameters, which include the normalization yield needed for calculating the normalization constant, are given in table \ref{tab:normdatafit}.

\begin{figure}
  \centering
  %% Creator: Matplotlib, PGF backend
%%
%% To include the figure in your LaTeX document, write
%%   \input{<filename>.pgf}
%%
%% Make sure the required packages are loaded in your preamble
%%   \usepackage{pgf}
%%
%% Figures using additional raster images can only be included by \input if
%% they are in the same directory as the main LaTeX file. For loading figures
%% from other directories you can use the `import` package
%%   \usepackage{import}
%% and then include the figures with
%%   \import{<path to file>}{<filename>.pgf}
%%
%% Matplotlib used the following preamble
%%   \usepackage{fontspec}
%%   \setmainfont{DejaVu Serif}
%%   \setsansfont{DejaVu Sans}
%%   \setmonofont{DejaVu Sans Mono}
%%
\begingroup%
\makeatletter%
\begin{pgfpicture}%
\pgfpathrectangle{\pgfpointorigin}{\pgfqpoint{3.556457in}{2.771463in}}%
\pgfusepath{use as bounding box, clip}%
\begin{pgfscope}%
\pgfsetbuttcap%
\pgfsetmiterjoin%
\definecolor{currentfill}{rgb}{1.000000,1.000000,1.000000}%
\pgfsetfillcolor{currentfill}%
\pgfsetlinewidth{0.000000pt}%
\definecolor{currentstroke}{rgb}{1.000000,1.000000,1.000000}%
\pgfsetstrokecolor{currentstroke}%
\pgfsetdash{}{0pt}%
\pgfpathmoveto{\pgfqpoint{0.000000in}{0.000000in}}%
\pgfpathlineto{\pgfqpoint{3.556457in}{0.000000in}}%
\pgfpathlineto{\pgfqpoint{3.556457in}{2.771463in}}%
\pgfpathlineto{\pgfqpoint{0.000000in}{2.771463in}}%
\pgfpathclose%
\pgfusepath{fill}%
\end{pgfscope}%
\begin{pgfscope}%
\pgfsetbuttcap%
\pgfsetmiterjoin%
\definecolor{currentfill}{rgb}{1.000000,1.000000,1.000000}%
\pgfsetfillcolor{currentfill}%
\pgfsetlinewidth{0.000000pt}%
\definecolor{currentstroke}{rgb}{0.000000,0.000000,0.000000}%
\pgfsetstrokecolor{currentstroke}%
\pgfsetstrokeopacity{0.000000}%
\pgfsetdash{}{0pt}%
\pgfpathmoveto{\pgfqpoint{0.485787in}{0.226975in}}%
\pgfpathlineto{\pgfqpoint{3.506457in}{0.226975in}}%
\pgfpathlineto{\pgfqpoint{3.506457in}{2.721463in}}%
\pgfpathlineto{\pgfqpoint{0.485787in}{2.721463in}}%
\pgfpathclose%
\pgfusepath{fill}%
\end{pgfscope}%
\begin{pgfscope}%
\pgfpathrectangle{\pgfqpoint{0.485787in}{0.226975in}}{\pgfqpoint{3.020670in}{2.494489in}} %
\pgfusepath{clip}%
\pgfsetbuttcap%
\pgfsetroundjoin%
\pgfsetlinewidth{1.003750pt}%
\definecolor{currentstroke}{rgb}{0.000000,0.000000,1.000000}%
\pgfsetstrokecolor{currentstroke}%
\pgfsetdash{{8.000000pt}{3.000000pt}}{0.000000pt}%
\pgfpathmoveto{\pgfqpoint{0.485787in}{0.540301in}}%
\pgfpathlineto{\pgfqpoint{0.637580in}{0.572648in}}%
\pgfpathlineto{\pgfqpoint{0.789372in}{0.607979in}}%
\pgfpathlineto{\pgfqpoint{0.925985in}{0.642864in}}%
\pgfpathlineto{\pgfqpoint{1.017061in}{0.667994in}}%
\pgfpathlineto{\pgfqpoint{1.108136in}{0.694889in}}%
\pgfpathlineto{\pgfqpoint{1.199212in}{0.723815in}}%
\pgfpathlineto{\pgfqpoint{1.275108in}{0.749670in}}%
\pgfpathlineto{\pgfqpoint{1.366184in}{0.783247in}}%
\pgfpathlineto{\pgfqpoint{1.442080in}{0.813709in}}%
\pgfpathlineto{\pgfqpoint{1.502797in}{0.839915in}}%
\pgfpathlineto{\pgfqpoint{1.563514in}{0.868046in}}%
\pgfpathlineto{\pgfqpoint{1.624231in}{0.898409in}}%
\pgfpathlineto{\pgfqpoint{1.684948in}{0.931388in}}%
\pgfpathlineto{\pgfqpoint{1.730485in}{0.958096in}}%
\pgfpathlineto{\pgfqpoint{1.776023in}{0.986884in}}%
\pgfpathlineto{\pgfqpoint{1.821561in}{1.017948in}}%
\pgfpathlineto{\pgfqpoint{1.867099in}{1.051811in}}%
\pgfpathlineto{\pgfqpoint{1.912636in}{1.088935in}}%
\pgfpathlineto{\pgfqpoint{1.958174in}{1.130050in}}%
\pgfpathlineto{\pgfqpoint{1.988533in}{1.160141in}}%
\pgfpathlineto{\pgfqpoint{2.018891in}{1.192804in}}%
\pgfpathlineto{\pgfqpoint{2.049250in}{1.228520in}}%
\pgfpathlineto{\pgfqpoint{2.079608in}{1.267923in}}%
\pgfpathlineto{\pgfqpoint{2.109967in}{1.311864in}}%
\pgfpathlineto{\pgfqpoint{2.140325in}{1.361532in}}%
\pgfpathlineto{\pgfqpoint{2.170684in}{1.418653in}}%
\pgfpathlineto{\pgfqpoint{2.201042in}{1.485887in}}%
\pgfpathlineto{\pgfqpoint{2.231400in}{1.567638in}}%
\pgfpathlineto{\pgfqpoint{2.246580in}{1.613671in}}%
\pgfpathlineto{\pgfqpoint{2.261759in}{1.668791in}}%
\pgfpathlineto{\pgfqpoint{2.276938in}{1.733319in}}%
\pgfpathlineto{\pgfqpoint{2.292117in}{1.807129in}}%
\pgfpathlineto{\pgfqpoint{2.337655in}{2.093199in}}%
\pgfpathlineto{\pgfqpoint{2.352834in}{2.176665in}}%
\pgfpathlineto{\pgfqpoint{2.368014in}{2.252016in}}%
\pgfpathlineto{\pgfqpoint{2.383193in}{2.319254in}}%
\pgfpathlineto{\pgfqpoint{2.398372in}{2.378651in}}%
\pgfpathlineto{\pgfqpoint{2.413551in}{2.429400in}}%
\pgfpathlineto{\pgfqpoint{2.428731in}{2.472085in}}%
\pgfpathlineto{\pgfqpoint{2.443910in}{2.507089in}}%
\pgfpathlineto{\pgfqpoint{2.459089in}{2.533773in}}%
\pgfpathlineto{\pgfqpoint{2.474268in}{2.552350in}}%
\pgfpathlineto{\pgfqpoint{2.489448in}{2.562818in}}%
\pgfpathlineto{\pgfqpoint{2.504627in}{2.565180in}}%
\pgfpathlineto{\pgfqpoint{2.519806in}{2.559435in}}%
\pgfpathlineto{\pgfqpoint{2.534985in}{2.545584in}}%
\pgfpathlineto{\pgfqpoint{2.550165in}{2.523628in}}%
\pgfpathlineto{\pgfqpoint{2.565344in}{2.493567in}}%
\pgfpathlineto{\pgfqpoint{2.580523in}{2.455225in}}%
\pgfpathlineto{\pgfqpoint{2.595702in}{2.409154in}}%
\pgfpathlineto{\pgfqpoint{2.610882in}{2.355009in}}%
\pgfpathlineto{\pgfqpoint{2.626061in}{2.292273in}}%
\pgfpathlineto{\pgfqpoint{2.641240in}{2.221689in}}%
\pgfpathlineto{\pgfqpoint{2.656419in}{2.143000in}}%
\pgfpathlineto{\pgfqpoint{2.671599in}{2.056208in}}%
\pgfpathlineto{\pgfqpoint{2.686778in}{1.961310in}}%
\pgfpathlineto{\pgfqpoint{2.717136in}{1.749851in}}%
\pgfpathlineto{\pgfqpoint{2.747495in}{1.504224in}}%
\pgfpathlineto{\pgfqpoint{2.793033in}{1.100269in}}%
\pgfpathlineto{\pgfqpoint{2.808212in}{0.929583in}}%
\pgfpathlineto{\pgfqpoint{2.823391in}{0.779558in}}%
\pgfpathlineto{\pgfqpoint{2.838570in}{0.590114in}}%
\pgfpathlineto{\pgfqpoint{2.853750in}{0.426991in}}%
\pgfpathlineto{\pgfqpoint{2.869064in}{0.216975in}}%
\pgfpathlineto{\pgfqpoint{2.869064in}{0.216975in}}%
\pgfusepath{stroke}%
\end{pgfscope}%
\begin{pgfscope}%
\pgfpathrectangle{\pgfqpoint{0.485787in}{0.226975in}}{\pgfqpoint{3.020670in}{2.494489in}} %
\pgfusepath{clip}%
\pgfsetbuttcap%
\pgfsetroundjoin%
\pgfsetlinewidth{1.003750pt}%
\definecolor{currentstroke}{rgb}{0.000000,0.500000,0.000000}%
\pgfsetstrokecolor{currentstroke}%
\pgfsetdash{{8.000000pt}{3.000000pt}}{0.000000pt}%
\pgfpathmoveto{\pgfqpoint{1.933329in}{0.216975in}}%
\pgfpathlineto{\pgfqpoint{1.958174in}{0.398235in}}%
\pgfpathlineto{\pgfqpoint{1.988533in}{0.609444in}}%
\pgfpathlineto{\pgfqpoint{2.003712in}{0.712814in}}%
\pgfpathlineto{\pgfqpoint{2.049250in}{0.995473in}}%
\pgfpathlineto{\pgfqpoint{2.064429in}{1.086293in}}%
\pgfpathlineto{\pgfqpoint{2.109967in}{1.332977in}}%
\pgfpathlineto{\pgfqpoint{2.155504in}{1.551388in}}%
\pgfpathlineto{\pgfqpoint{2.185863in}{1.684327in}}%
\pgfpathlineto{\pgfqpoint{2.216221in}{1.805067in}}%
\pgfpathlineto{\pgfqpoint{2.246580in}{1.913608in}}%
\pgfpathlineto{\pgfqpoint{2.276938in}{2.009949in}}%
\pgfpathlineto{\pgfqpoint{2.307297in}{2.094091in}}%
\pgfpathlineto{\pgfqpoint{2.337655in}{2.166201in}}%
\pgfpathlineto{\pgfqpoint{2.352834in}{2.197523in}}%
\pgfpathlineto{\pgfqpoint{2.368014in}{2.225548in}}%
\pgfpathlineto{\pgfqpoint{2.383193in}{2.250947in}}%
\pgfpathlineto{\pgfqpoint{2.398372in}{2.273317in}}%
\pgfpathlineto{\pgfqpoint{2.413551in}{2.292512in}}%
\pgfpathlineto{\pgfqpoint{2.428731in}{2.308657in}}%
\pgfpathlineto{\pgfqpoint{2.443910in}{2.321971in}}%
\pgfpathlineto{\pgfqpoint{2.459089in}{2.332032in}}%
\pgfpathlineto{\pgfqpoint{2.474268in}{2.339035in}}%
\pgfpathlineto{\pgfqpoint{2.489448in}{2.342982in}}%
\pgfpathlineto{\pgfqpoint{2.504627in}{2.343872in}}%
\pgfpathlineto{\pgfqpoint{2.519806in}{2.341706in}}%
\pgfpathlineto{\pgfqpoint{2.534985in}{2.336483in}}%
\pgfpathlineto{\pgfqpoint{2.550165in}{2.328204in}}%
\pgfpathlineto{\pgfqpoint{2.565344in}{2.316869in}}%
\pgfpathlineto{\pgfqpoint{2.580523in}{2.302306in}}%
\pgfpathlineto{\pgfqpoint{2.595702in}{2.284888in}}%
\pgfpathlineto{\pgfqpoint{2.610882in}{2.264418in}}%
\pgfpathlineto{\pgfqpoint{2.626061in}{2.240896in}}%
\pgfpathlineto{\pgfqpoint{2.641240in}{2.214296in}}%
\pgfpathlineto{\pgfqpoint{2.656419in}{2.184682in}}%
\pgfpathlineto{\pgfqpoint{2.671599in}{2.152013in}}%
\pgfpathlineto{\pgfqpoint{2.686778in}{2.116279in}}%
\pgfpathlineto{\pgfqpoint{2.701957in}{2.077492in}}%
\pgfpathlineto{\pgfqpoint{2.732316in}{1.990756in}}%
\pgfpathlineto{\pgfqpoint{2.762674in}{1.891804in}}%
\pgfpathlineto{\pgfqpoint{2.838570in}{1.623829in}}%
\pgfpathlineto{\pgfqpoint{2.868929in}{1.531898in}}%
\pgfpathlineto{\pgfqpoint{2.884108in}{1.492138in}}%
\pgfpathlineto{\pgfqpoint{2.899287in}{1.449135in}}%
\pgfpathlineto{\pgfqpoint{2.960004in}{1.301207in}}%
\pgfpathlineto{\pgfqpoint{3.020721in}{1.173666in}}%
\pgfpathlineto{\pgfqpoint{3.081438in}{1.061553in}}%
\pgfpathlineto{\pgfqpoint{3.142155in}{0.961528in}}%
\pgfpathlineto{\pgfqpoint{3.218051in}{0.850759in}}%
\pgfpathlineto{\pgfqpoint{3.233231in}{0.829173in}}%
\pgfpathlineto{\pgfqpoint{3.324306in}{0.713334in}}%
\pgfpathlineto{\pgfqpoint{3.415382in}{0.610341in}}%
\pgfpathlineto{\pgfqpoint{3.506457in}{0.517624in}}%
\pgfpathlineto{\pgfqpoint{3.506457in}{0.517624in}}%
\pgfusepath{stroke}%
\end{pgfscope}%
\begin{pgfscope}%
\pgfsetrectcap%
\pgfsetmiterjoin%
\pgfsetlinewidth{1.003750pt}%
\definecolor{currentstroke}{rgb}{0.000000,0.000000,0.000000}%
\pgfsetstrokecolor{currentstroke}%
\pgfsetdash{}{0pt}%
\pgfpathmoveto{\pgfqpoint{0.485787in}{2.721463in}}%
\pgfpathlineto{\pgfqpoint{3.506457in}{2.721463in}}%
\pgfusepath{stroke}%
\end{pgfscope}%
\begin{pgfscope}%
\pgfsetrectcap%
\pgfsetmiterjoin%
\pgfsetlinewidth{1.003750pt}%
\definecolor{currentstroke}{rgb}{0.000000,0.000000,0.000000}%
\pgfsetstrokecolor{currentstroke}%
\pgfsetdash{}{0pt}%
\pgfpathmoveto{\pgfqpoint{3.506457in}{0.226975in}}%
\pgfpathlineto{\pgfqpoint{3.506457in}{2.721463in}}%
\pgfusepath{stroke}%
\end{pgfscope}%
\begin{pgfscope}%
\pgfsetrectcap%
\pgfsetmiterjoin%
\pgfsetlinewidth{1.003750pt}%
\definecolor{currentstroke}{rgb}{0.000000,0.000000,0.000000}%
\pgfsetstrokecolor{currentstroke}%
\pgfsetdash{}{0pt}%
\pgfpathmoveto{\pgfqpoint{0.485787in}{0.226975in}}%
\pgfpathlineto{\pgfqpoint{3.506457in}{0.226975in}}%
\pgfusepath{stroke}%
\end{pgfscope}%
\begin{pgfscope}%
\pgfsetrectcap%
\pgfsetmiterjoin%
\pgfsetlinewidth{1.003750pt}%
\definecolor{currentstroke}{rgb}{0.000000,0.000000,0.000000}%
\pgfsetstrokecolor{currentstroke}%
\pgfsetdash{}{0pt}%
\pgfpathmoveto{\pgfqpoint{0.485787in}{0.226975in}}%
\pgfpathlineto{\pgfqpoint{0.485787in}{2.721463in}}%
\pgfusepath{stroke}%
\end{pgfscope}%
\begin{pgfscope}%
\pgfsetbuttcap%
\pgfsetroundjoin%
\definecolor{currentfill}{rgb}{0.000000,0.000000,0.000000}%
\pgfsetfillcolor{currentfill}%
\pgfsetlinewidth{0.501875pt}%
\definecolor{currentstroke}{rgb}{0.000000,0.000000,0.000000}%
\pgfsetstrokecolor{currentstroke}%
\pgfsetdash{}{0pt}%
\pgfsys@defobject{currentmarker}{\pgfqpoint{0.000000in}{0.000000in}}{\pgfqpoint{0.000000in}{0.069444in}}{%
\pgfpathmoveto{\pgfqpoint{0.000000in}{0.000000in}}%
\pgfpathlineto{\pgfqpoint{0.000000in}{0.069444in}}%
\pgfusepath{stroke,fill}%
}%
\begin{pgfscope}%
\pgfsys@transformshift{0.485787in}{0.226975in}%
\pgfsys@useobject{currentmarker}{}%
\end{pgfscope}%
\end{pgfscope}%
\begin{pgfscope}%
\pgfsetbuttcap%
\pgfsetroundjoin%
\definecolor{currentfill}{rgb}{0.000000,0.000000,0.000000}%
\pgfsetfillcolor{currentfill}%
\pgfsetlinewidth{0.501875pt}%
\definecolor{currentstroke}{rgb}{0.000000,0.000000,0.000000}%
\pgfsetstrokecolor{currentstroke}%
\pgfsetdash{}{0pt}%
\pgfsys@defobject{currentmarker}{\pgfqpoint{0.000000in}{-0.069444in}}{\pgfqpoint{0.000000in}{0.000000in}}{%
\pgfpathmoveto{\pgfqpoint{0.000000in}{0.000000in}}%
\pgfpathlineto{\pgfqpoint{0.000000in}{-0.069444in}}%
\pgfusepath{stroke,fill}%
}%
\begin{pgfscope}%
\pgfsys@transformshift{0.485787in}{2.721463in}%
\pgfsys@useobject{currentmarker}{}%
\end{pgfscope}%
\end{pgfscope}%
\begin{pgfscope}%
\pgftext[x=0.485787in,y=0.157530in,,top]{\rmfamily\fontsize{8.000000}{9.600000}\selectfont 5150}%
\end{pgfscope}%
\begin{pgfscope}%
\pgfsetbuttcap%
\pgfsetroundjoin%
\definecolor{currentfill}{rgb}{0.000000,0.000000,0.000000}%
\pgfsetfillcolor{currentfill}%
\pgfsetlinewidth{0.501875pt}%
\definecolor{currentstroke}{rgb}{0.000000,0.000000,0.000000}%
\pgfsetstrokecolor{currentstroke}%
\pgfsetdash{}{0pt}%
\pgfsys@defobject{currentmarker}{\pgfqpoint{0.000000in}{0.000000in}}{\pgfqpoint{0.000000in}{0.069444in}}{%
\pgfpathmoveto{\pgfqpoint{0.000000in}{0.000000in}}%
\pgfpathlineto{\pgfqpoint{0.000000in}{0.069444in}}%
\pgfusepath{stroke,fill}%
}%
\begin{pgfscope}%
\pgfsys@transformshift{1.260318in}{0.226975in}%
\pgfsys@useobject{currentmarker}{}%
\end{pgfscope}%
\end{pgfscope}%
\begin{pgfscope}%
\pgfsetbuttcap%
\pgfsetroundjoin%
\definecolor{currentfill}{rgb}{0.000000,0.000000,0.000000}%
\pgfsetfillcolor{currentfill}%
\pgfsetlinewidth{0.501875pt}%
\definecolor{currentstroke}{rgb}{0.000000,0.000000,0.000000}%
\pgfsetstrokecolor{currentstroke}%
\pgfsetdash{}{0pt}%
\pgfsys@defobject{currentmarker}{\pgfqpoint{0.000000in}{-0.069444in}}{\pgfqpoint{0.000000in}{0.000000in}}{%
\pgfpathmoveto{\pgfqpoint{0.000000in}{0.000000in}}%
\pgfpathlineto{\pgfqpoint{0.000000in}{-0.069444in}}%
\pgfusepath{stroke,fill}%
}%
\begin{pgfscope}%
\pgfsys@transformshift{1.260318in}{2.721463in}%
\pgfsys@useobject{currentmarker}{}%
\end{pgfscope}%
\end{pgfscope}%
\begin{pgfscope}%
\pgftext[x=1.260318in,y=0.157530in,,top]{\rmfamily\fontsize{8.000000}{9.600000}\selectfont 5200}%
\end{pgfscope}%
\begin{pgfscope}%
\pgfsetbuttcap%
\pgfsetroundjoin%
\definecolor{currentfill}{rgb}{0.000000,0.000000,0.000000}%
\pgfsetfillcolor{currentfill}%
\pgfsetlinewidth{0.501875pt}%
\definecolor{currentstroke}{rgb}{0.000000,0.000000,0.000000}%
\pgfsetstrokecolor{currentstroke}%
\pgfsetdash{}{0pt}%
\pgfsys@defobject{currentmarker}{\pgfqpoint{0.000000in}{0.000000in}}{\pgfqpoint{0.000000in}{0.069444in}}{%
\pgfpathmoveto{\pgfqpoint{0.000000in}{0.000000in}}%
\pgfpathlineto{\pgfqpoint{0.000000in}{0.069444in}}%
\pgfusepath{stroke,fill}%
}%
\begin{pgfscope}%
\pgfsys@transformshift{2.034849in}{0.226975in}%
\pgfsys@useobject{currentmarker}{}%
\end{pgfscope}%
\end{pgfscope}%
\begin{pgfscope}%
\pgfsetbuttcap%
\pgfsetroundjoin%
\definecolor{currentfill}{rgb}{0.000000,0.000000,0.000000}%
\pgfsetfillcolor{currentfill}%
\pgfsetlinewidth{0.501875pt}%
\definecolor{currentstroke}{rgb}{0.000000,0.000000,0.000000}%
\pgfsetstrokecolor{currentstroke}%
\pgfsetdash{}{0pt}%
\pgfsys@defobject{currentmarker}{\pgfqpoint{0.000000in}{-0.069444in}}{\pgfqpoint{0.000000in}{0.000000in}}{%
\pgfpathmoveto{\pgfqpoint{0.000000in}{0.000000in}}%
\pgfpathlineto{\pgfqpoint{0.000000in}{-0.069444in}}%
\pgfusepath{stroke,fill}%
}%
\begin{pgfscope}%
\pgfsys@transformshift{2.034849in}{2.721463in}%
\pgfsys@useobject{currentmarker}{}%
\end{pgfscope}%
\end{pgfscope}%
\begin{pgfscope}%
\pgftext[x=2.034849in,y=0.157530in,,top]{\rmfamily\fontsize{8.000000}{9.600000}\selectfont 5250}%
\end{pgfscope}%
\begin{pgfscope}%
\pgfsetbuttcap%
\pgfsetroundjoin%
\definecolor{currentfill}{rgb}{0.000000,0.000000,0.000000}%
\pgfsetfillcolor{currentfill}%
\pgfsetlinewidth{0.501875pt}%
\definecolor{currentstroke}{rgb}{0.000000,0.000000,0.000000}%
\pgfsetstrokecolor{currentstroke}%
\pgfsetdash{}{0pt}%
\pgfsys@defobject{currentmarker}{\pgfqpoint{0.000000in}{0.000000in}}{\pgfqpoint{0.000000in}{0.069444in}}{%
\pgfpathmoveto{\pgfqpoint{0.000000in}{0.000000in}}%
\pgfpathlineto{\pgfqpoint{0.000000in}{0.069444in}}%
\pgfusepath{stroke,fill}%
}%
\begin{pgfscope}%
\pgfsys@transformshift{2.809379in}{0.226975in}%
\pgfsys@useobject{currentmarker}{}%
\end{pgfscope}%
\end{pgfscope}%
\begin{pgfscope}%
\pgfsetbuttcap%
\pgfsetroundjoin%
\definecolor{currentfill}{rgb}{0.000000,0.000000,0.000000}%
\pgfsetfillcolor{currentfill}%
\pgfsetlinewidth{0.501875pt}%
\definecolor{currentstroke}{rgb}{0.000000,0.000000,0.000000}%
\pgfsetstrokecolor{currentstroke}%
\pgfsetdash{}{0pt}%
\pgfsys@defobject{currentmarker}{\pgfqpoint{0.000000in}{-0.069444in}}{\pgfqpoint{0.000000in}{0.000000in}}{%
\pgfpathmoveto{\pgfqpoint{0.000000in}{0.000000in}}%
\pgfpathlineto{\pgfqpoint{0.000000in}{-0.069444in}}%
\pgfusepath{stroke,fill}%
}%
\begin{pgfscope}%
\pgfsys@transformshift{2.809379in}{2.721463in}%
\pgfsys@useobject{currentmarker}{}%
\end{pgfscope}%
\end{pgfscope}%
\begin{pgfscope}%
\pgftext[x=2.809379in,y=0.157530in,,top]{\rmfamily\fontsize{8.000000}{9.600000}\selectfont 5300}%
\end{pgfscope}%
\begin{pgfscope}%
\pgfsetbuttcap%
\pgfsetroundjoin%
\definecolor{currentfill}{rgb}{0.000000,0.000000,0.000000}%
\pgfsetfillcolor{currentfill}%
\pgfsetlinewidth{0.501875pt}%
\definecolor{currentstroke}{rgb}{0.000000,0.000000,0.000000}%
\pgfsetstrokecolor{currentstroke}%
\pgfsetdash{}{0pt}%
\pgfsys@defobject{currentmarker}{\pgfqpoint{0.000000in}{0.000000in}}{\pgfqpoint{0.069444in}{0.000000in}}{%
\pgfpathmoveto{\pgfqpoint{0.000000in}{0.000000in}}%
\pgfpathlineto{\pgfqpoint{0.069444in}{0.000000in}}%
\pgfusepath{stroke,fill}%
}%
\begin{pgfscope}%
\pgfsys@transformshift{0.485787in}{0.308085in}%
\pgfsys@useobject{currentmarker}{}%
\end{pgfscope}%
\end{pgfscope}%
\begin{pgfscope}%
\pgfsetbuttcap%
\pgfsetroundjoin%
\definecolor{currentfill}{rgb}{0.000000,0.000000,0.000000}%
\pgfsetfillcolor{currentfill}%
\pgfsetlinewidth{0.501875pt}%
\definecolor{currentstroke}{rgb}{0.000000,0.000000,0.000000}%
\pgfsetstrokecolor{currentstroke}%
\pgfsetdash{}{0pt}%
\pgfsys@defobject{currentmarker}{\pgfqpoint{-0.069444in}{0.000000in}}{\pgfqpoint{0.000000in}{0.000000in}}{%
\pgfpathmoveto{\pgfqpoint{0.000000in}{0.000000in}}%
\pgfpathlineto{\pgfqpoint{-0.069444in}{0.000000in}}%
\pgfusepath{stroke,fill}%
}%
\begin{pgfscope}%
\pgfsys@transformshift{3.506457in}{0.308085in}%
\pgfsys@useobject{currentmarker}{}%
\end{pgfscope}%
\end{pgfscope}%
\begin{pgfscope}%
\pgftext[x=0.416343in,y=0.308085in,right,]{\rmfamily\fontsize{8.000000}{9.600000}\selectfont \(\displaystyle {10^{1}}\)}%
\end{pgfscope}%
\begin{pgfscope}%
\pgfsetbuttcap%
\pgfsetroundjoin%
\definecolor{currentfill}{rgb}{0.000000,0.000000,0.000000}%
\pgfsetfillcolor{currentfill}%
\pgfsetlinewidth{0.501875pt}%
\definecolor{currentstroke}{rgb}{0.000000,0.000000,0.000000}%
\pgfsetstrokecolor{currentstroke}%
\pgfsetdash{}{0pt}%
\pgfsys@defobject{currentmarker}{\pgfqpoint{0.000000in}{0.000000in}}{\pgfqpoint{0.069444in}{0.000000in}}{%
\pgfpathmoveto{\pgfqpoint{0.000000in}{0.000000in}}%
\pgfpathlineto{\pgfqpoint{0.069444in}{0.000000in}}%
\pgfusepath{stroke,fill}%
}%
\begin{pgfscope}%
\pgfsys@transformshift{0.485787in}{0.957280in}%
\pgfsys@useobject{currentmarker}{}%
\end{pgfscope}%
\end{pgfscope}%
\begin{pgfscope}%
\pgfsetbuttcap%
\pgfsetroundjoin%
\definecolor{currentfill}{rgb}{0.000000,0.000000,0.000000}%
\pgfsetfillcolor{currentfill}%
\pgfsetlinewidth{0.501875pt}%
\definecolor{currentstroke}{rgb}{0.000000,0.000000,0.000000}%
\pgfsetstrokecolor{currentstroke}%
\pgfsetdash{}{0pt}%
\pgfsys@defobject{currentmarker}{\pgfqpoint{-0.069444in}{0.000000in}}{\pgfqpoint{0.000000in}{0.000000in}}{%
\pgfpathmoveto{\pgfqpoint{0.000000in}{0.000000in}}%
\pgfpathlineto{\pgfqpoint{-0.069444in}{0.000000in}}%
\pgfusepath{stroke,fill}%
}%
\begin{pgfscope}%
\pgfsys@transformshift{3.506457in}{0.957280in}%
\pgfsys@useobject{currentmarker}{}%
\end{pgfscope}%
\end{pgfscope}%
\begin{pgfscope}%
\pgftext[x=0.416343in,y=0.957280in,right,]{\rmfamily\fontsize{8.000000}{9.600000}\selectfont \(\displaystyle {10^{2}}\)}%
\end{pgfscope}%
\begin{pgfscope}%
\pgfsetbuttcap%
\pgfsetroundjoin%
\definecolor{currentfill}{rgb}{0.000000,0.000000,0.000000}%
\pgfsetfillcolor{currentfill}%
\pgfsetlinewidth{0.501875pt}%
\definecolor{currentstroke}{rgb}{0.000000,0.000000,0.000000}%
\pgfsetstrokecolor{currentstroke}%
\pgfsetdash{}{0pt}%
\pgfsys@defobject{currentmarker}{\pgfqpoint{0.000000in}{0.000000in}}{\pgfqpoint{0.069444in}{0.000000in}}{%
\pgfpathmoveto{\pgfqpoint{0.000000in}{0.000000in}}%
\pgfpathlineto{\pgfqpoint{0.069444in}{0.000000in}}%
\pgfusepath{stroke,fill}%
}%
\begin{pgfscope}%
\pgfsys@transformshift{0.485787in}{1.606476in}%
\pgfsys@useobject{currentmarker}{}%
\end{pgfscope}%
\end{pgfscope}%
\begin{pgfscope}%
\pgfsetbuttcap%
\pgfsetroundjoin%
\definecolor{currentfill}{rgb}{0.000000,0.000000,0.000000}%
\pgfsetfillcolor{currentfill}%
\pgfsetlinewidth{0.501875pt}%
\definecolor{currentstroke}{rgb}{0.000000,0.000000,0.000000}%
\pgfsetstrokecolor{currentstroke}%
\pgfsetdash{}{0pt}%
\pgfsys@defobject{currentmarker}{\pgfqpoint{-0.069444in}{0.000000in}}{\pgfqpoint{0.000000in}{0.000000in}}{%
\pgfpathmoveto{\pgfqpoint{0.000000in}{0.000000in}}%
\pgfpathlineto{\pgfqpoint{-0.069444in}{0.000000in}}%
\pgfusepath{stroke,fill}%
}%
\begin{pgfscope}%
\pgfsys@transformshift{3.506457in}{1.606476in}%
\pgfsys@useobject{currentmarker}{}%
\end{pgfscope}%
\end{pgfscope}%
\begin{pgfscope}%
\pgftext[x=0.416343in,y=1.606476in,right,]{\rmfamily\fontsize{8.000000}{9.600000}\selectfont \(\displaystyle {10^{3}}\)}%
\end{pgfscope}%
\begin{pgfscope}%
\pgfsetbuttcap%
\pgfsetroundjoin%
\definecolor{currentfill}{rgb}{0.000000,0.000000,0.000000}%
\pgfsetfillcolor{currentfill}%
\pgfsetlinewidth{0.501875pt}%
\definecolor{currentstroke}{rgb}{0.000000,0.000000,0.000000}%
\pgfsetstrokecolor{currentstroke}%
\pgfsetdash{}{0pt}%
\pgfsys@defobject{currentmarker}{\pgfqpoint{0.000000in}{0.000000in}}{\pgfqpoint{0.069444in}{0.000000in}}{%
\pgfpathmoveto{\pgfqpoint{0.000000in}{0.000000in}}%
\pgfpathlineto{\pgfqpoint{0.069444in}{0.000000in}}%
\pgfusepath{stroke,fill}%
}%
\begin{pgfscope}%
\pgfsys@transformshift{0.485787in}{2.255671in}%
\pgfsys@useobject{currentmarker}{}%
\end{pgfscope}%
\end{pgfscope}%
\begin{pgfscope}%
\pgfsetbuttcap%
\pgfsetroundjoin%
\definecolor{currentfill}{rgb}{0.000000,0.000000,0.000000}%
\pgfsetfillcolor{currentfill}%
\pgfsetlinewidth{0.501875pt}%
\definecolor{currentstroke}{rgb}{0.000000,0.000000,0.000000}%
\pgfsetstrokecolor{currentstroke}%
\pgfsetdash{}{0pt}%
\pgfsys@defobject{currentmarker}{\pgfqpoint{-0.069444in}{0.000000in}}{\pgfqpoint{0.000000in}{0.000000in}}{%
\pgfpathmoveto{\pgfqpoint{0.000000in}{0.000000in}}%
\pgfpathlineto{\pgfqpoint{-0.069444in}{0.000000in}}%
\pgfusepath{stroke,fill}%
}%
\begin{pgfscope}%
\pgfsys@transformshift{3.506457in}{2.255671in}%
\pgfsys@useobject{currentmarker}{}%
\end{pgfscope}%
\end{pgfscope}%
\begin{pgfscope}%
\pgftext[x=0.416343in,y=2.255671in,right,]{\rmfamily\fontsize{8.000000}{9.600000}\selectfont \(\displaystyle {10^{4}}\)}%
\end{pgfscope}%
\begin{pgfscope}%
\pgfsetbuttcap%
\pgfsetroundjoin%
\definecolor{currentfill}{rgb}{0.000000,0.000000,0.000000}%
\pgfsetfillcolor{currentfill}%
\pgfsetlinewidth{0.501875pt}%
\definecolor{currentstroke}{rgb}{0.000000,0.000000,0.000000}%
\pgfsetstrokecolor{currentstroke}%
\pgfsetdash{}{0pt}%
\pgfsys@defobject{currentmarker}{\pgfqpoint{0.000000in}{0.000000in}}{\pgfqpoint{0.027778in}{0.000000in}}{%
\pgfpathmoveto{\pgfqpoint{0.000000in}{0.000000in}}%
\pgfpathlineto{\pgfqpoint{0.027778in}{0.000000in}}%
\pgfusepath{stroke,fill}%
}%
\begin{pgfscope}%
\pgfsys@transformshift{0.485787in}{0.245171in}%
\pgfsys@useobject{currentmarker}{}%
\end{pgfscope}%
\end{pgfscope}%
\begin{pgfscope}%
\pgfsetbuttcap%
\pgfsetroundjoin%
\definecolor{currentfill}{rgb}{0.000000,0.000000,0.000000}%
\pgfsetfillcolor{currentfill}%
\pgfsetlinewidth{0.501875pt}%
\definecolor{currentstroke}{rgb}{0.000000,0.000000,0.000000}%
\pgfsetstrokecolor{currentstroke}%
\pgfsetdash{}{0pt}%
\pgfsys@defobject{currentmarker}{\pgfqpoint{-0.027778in}{0.000000in}}{\pgfqpoint{0.000000in}{0.000000in}}{%
\pgfpathmoveto{\pgfqpoint{0.000000in}{0.000000in}}%
\pgfpathlineto{\pgfqpoint{-0.027778in}{0.000000in}}%
\pgfusepath{stroke,fill}%
}%
\begin{pgfscope}%
\pgfsys@transformshift{3.506457in}{0.245171in}%
\pgfsys@useobject{currentmarker}{}%
\end{pgfscope}%
\end{pgfscope}%
\begin{pgfscope}%
\pgfsetbuttcap%
\pgfsetroundjoin%
\definecolor{currentfill}{rgb}{0.000000,0.000000,0.000000}%
\pgfsetfillcolor{currentfill}%
\pgfsetlinewidth{0.501875pt}%
\definecolor{currentstroke}{rgb}{0.000000,0.000000,0.000000}%
\pgfsetstrokecolor{currentstroke}%
\pgfsetdash{}{0pt}%
\pgfsys@defobject{currentmarker}{\pgfqpoint{0.000000in}{0.000000in}}{\pgfqpoint{0.027778in}{0.000000in}}{%
\pgfpathmoveto{\pgfqpoint{0.000000in}{0.000000in}}%
\pgfpathlineto{\pgfqpoint{0.027778in}{0.000000in}}%
\pgfusepath{stroke,fill}%
}%
\begin{pgfscope}%
\pgfsys@transformshift{0.485787in}{0.278379in}%
\pgfsys@useobject{currentmarker}{}%
\end{pgfscope}%
\end{pgfscope}%
\begin{pgfscope}%
\pgfsetbuttcap%
\pgfsetroundjoin%
\definecolor{currentfill}{rgb}{0.000000,0.000000,0.000000}%
\pgfsetfillcolor{currentfill}%
\pgfsetlinewidth{0.501875pt}%
\definecolor{currentstroke}{rgb}{0.000000,0.000000,0.000000}%
\pgfsetstrokecolor{currentstroke}%
\pgfsetdash{}{0pt}%
\pgfsys@defobject{currentmarker}{\pgfqpoint{-0.027778in}{0.000000in}}{\pgfqpoint{0.000000in}{0.000000in}}{%
\pgfpathmoveto{\pgfqpoint{0.000000in}{0.000000in}}%
\pgfpathlineto{\pgfqpoint{-0.027778in}{0.000000in}}%
\pgfusepath{stroke,fill}%
}%
\begin{pgfscope}%
\pgfsys@transformshift{3.506457in}{0.278379in}%
\pgfsys@useobject{currentmarker}{}%
\end{pgfscope}%
\end{pgfscope}%
\begin{pgfscope}%
\pgfsetbuttcap%
\pgfsetroundjoin%
\definecolor{currentfill}{rgb}{0.000000,0.000000,0.000000}%
\pgfsetfillcolor{currentfill}%
\pgfsetlinewidth{0.501875pt}%
\definecolor{currentstroke}{rgb}{0.000000,0.000000,0.000000}%
\pgfsetstrokecolor{currentstroke}%
\pgfsetdash{}{0pt}%
\pgfsys@defobject{currentmarker}{\pgfqpoint{0.000000in}{0.000000in}}{\pgfqpoint{0.027778in}{0.000000in}}{%
\pgfpathmoveto{\pgfqpoint{0.000000in}{0.000000in}}%
\pgfpathlineto{\pgfqpoint{0.027778in}{0.000000in}}%
\pgfusepath{stroke,fill}%
}%
\begin{pgfscope}%
\pgfsys@transformshift{0.485787in}{0.503512in}%
\pgfsys@useobject{currentmarker}{}%
\end{pgfscope}%
\end{pgfscope}%
\begin{pgfscope}%
\pgfsetbuttcap%
\pgfsetroundjoin%
\definecolor{currentfill}{rgb}{0.000000,0.000000,0.000000}%
\pgfsetfillcolor{currentfill}%
\pgfsetlinewidth{0.501875pt}%
\definecolor{currentstroke}{rgb}{0.000000,0.000000,0.000000}%
\pgfsetstrokecolor{currentstroke}%
\pgfsetdash{}{0pt}%
\pgfsys@defobject{currentmarker}{\pgfqpoint{-0.027778in}{0.000000in}}{\pgfqpoint{0.000000in}{0.000000in}}{%
\pgfpathmoveto{\pgfqpoint{0.000000in}{0.000000in}}%
\pgfpathlineto{\pgfqpoint{-0.027778in}{0.000000in}}%
\pgfusepath{stroke,fill}%
}%
\begin{pgfscope}%
\pgfsys@transformshift{3.506457in}{0.503512in}%
\pgfsys@useobject{currentmarker}{}%
\end{pgfscope}%
\end{pgfscope}%
\begin{pgfscope}%
\pgfsetbuttcap%
\pgfsetroundjoin%
\definecolor{currentfill}{rgb}{0.000000,0.000000,0.000000}%
\pgfsetfillcolor{currentfill}%
\pgfsetlinewidth{0.501875pt}%
\definecolor{currentstroke}{rgb}{0.000000,0.000000,0.000000}%
\pgfsetstrokecolor{currentstroke}%
\pgfsetdash{}{0pt}%
\pgfsys@defobject{currentmarker}{\pgfqpoint{0.000000in}{0.000000in}}{\pgfqpoint{0.027778in}{0.000000in}}{%
\pgfpathmoveto{\pgfqpoint{0.000000in}{0.000000in}}%
\pgfpathlineto{\pgfqpoint{0.027778in}{0.000000in}}%
\pgfusepath{stroke,fill}%
}%
\begin{pgfscope}%
\pgfsys@transformshift{0.485787in}{0.617830in}%
\pgfsys@useobject{currentmarker}{}%
\end{pgfscope}%
\end{pgfscope}%
\begin{pgfscope}%
\pgfsetbuttcap%
\pgfsetroundjoin%
\definecolor{currentfill}{rgb}{0.000000,0.000000,0.000000}%
\pgfsetfillcolor{currentfill}%
\pgfsetlinewidth{0.501875pt}%
\definecolor{currentstroke}{rgb}{0.000000,0.000000,0.000000}%
\pgfsetstrokecolor{currentstroke}%
\pgfsetdash{}{0pt}%
\pgfsys@defobject{currentmarker}{\pgfqpoint{-0.027778in}{0.000000in}}{\pgfqpoint{0.000000in}{0.000000in}}{%
\pgfpathmoveto{\pgfqpoint{0.000000in}{0.000000in}}%
\pgfpathlineto{\pgfqpoint{-0.027778in}{0.000000in}}%
\pgfusepath{stroke,fill}%
}%
\begin{pgfscope}%
\pgfsys@transformshift{3.506457in}{0.617830in}%
\pgfsys@useobject{currentmarker}{}%
\end{pgfscope}%
\end{pgfscope}%
\begin{pgfscope}%
\pgfsetbuttcap%
\pgfsetroundjoin%
\definecolor{currentfill}{rgb}{0.000000,0.000000,0.000000}%
\pgfsetfillcolor{currentfill}%
\pgfsetlinewidth{0.501875pt}%
\definecolor{currentstroke}{rgb}{0.000000,0.000000,0.000000}%
\pgfsetstrokecolor{currentstroke}%
\pgfsetdash{}{0pt}%
\pgfsys@defobject{currentmarker}{\pgfqpoint{0.000000in}{0.000000in}}{\pgfqpoint{0.027778in}{0.000000in}}{%
\pgfpathmoveto{\pgfqpoint{0.000000in}{0.000000in}}%
\pgfpathlineto{\pgfqpoint{0.027778in}{0.000000in}}%
\pgfusepath{stroke,fill}%
}%
\begin{pgfscope}%
\pgfsys@transformshift{0.485787in}{0.698939in}%
\pgfsys@useobject{currentmarker}{}%
\end{pgfscope}%
\end{pgfscope}%
\begin{pgfscope}%
\pgfsetbuttcap%
\pgfsetroundjoin%
\definecolor{currentfill}{rgb}{0.000000,0.000000,0.000000}%
\pgfsetfillcolor{currentfill}%
\pgfsetlinewidth{0.501875pt}%
\definecolor{currentstroke}{rgb}{0.000000,0.000000,0.000000}%
\pgfsetstrokecolor{currentstroke}%
\pgfsetdash{}{0pt}%
\pgfsys@defobject{currentmarker}{\pgfqpoint{-0.027778in}{0.000000in}}{\pgfqpoint{0.000000in}{0.000000in}}{%
\pgfpathmoveto{\pgfqpoint{0.000000in}{0.000000in}}%
\pgfpathlineto{\pgfqpoint{-0.027778in}{0.000000in}}%
\pgfusepath{stroke,fill}%
}%
\begin{pgfscope}%
\pgfsys@transformshift{3.506457in}{0.698939in}%
\pgfsys@useobject{currentmarker}{}%
\end{pgfscope}%
\end{pgfscope}%
\begin{pgfscope}%
\pgfsetbuttcap%
\pgfsetroundjoin%
\definecolor{currentfill}{rgb}{0.000000,0.000000,0.000000}%
\pgfsetfillcolor{currentfill}%
\pgfsetlinewidth{0.501875pt}%
\definecolor{currentstroke}{rgb}{0.000000,0.000000,0.000000}%
\pgfsetstrokecolor{currentstroke}%
\pgfsetdash{}{0pt}%
\pgfsys@defobject{currentmarker}{\pgfqpoint{0.000000in}{0.000000in}}{\pgfqpoint{0.027778in}{0.000000in}}{%
\pgfpathmoveto{\pgfqpoint{0.000000in}{0.000000in}}%
\pgfpathlineto{\pgfqpoint{0.027778in}{0.000000in}}%
\pgfusepath{stroke,fill}%
}%
\begin{pgfscope}%
\pgfsys@transformshift{0.485787in}{0.761853in}%
\pgfsys@useobject{currentmarker}{}%
\end{pgfscope}%
\end{pgfscope}%
\begin{pgfscope}%
\pgfsetbuttcap%
\pgfsetroundjoin%
\definecolor{currentfill}{rgb}{0.000000,0.000000,0.000000}%
\pgfsetfillcolor{currentfill}%
\pgfsetlinewidth{0.501875pt}%
\definecolor{currentstroke}{rgb}{0.000000,0.000000,0.000000}%
\pgfsetstrokecolor{currentstroke}%
\pgfsetdash{}{0pt}%
\pgfsys@defobject{currentmarker}{\pgfqpoint{-0.027778in}{0.000000in}}{\pgfqpoint{0.000000in}{0.000000in}}{%
\pgfpathmoveto{\pgfqpoint{0.000000in}{0.000000in}}%
\pgfpathlineto{\pgfqpoint{-0.027778in}{0.000000in}}%
\pgfusepath{stroke,fill}%
}%
\begin{pgfscope}%
\pgfsys@transformshift{3.506457in}{0.761853in}%
\pgfsys@useobject{currentmarker}{}%
\end{pgfscope}%
\end{pgfscope}%
\begin{pgfscope}%
\pgfsetbuttcap%
\pgfsetroundjoin%
\definecolor{currentfill}{rgb}{0.000000,0.000000,0.000000}%
\pgfsetfillcolor{currentfill}%
\pgfsetlinewidth{0.501875pt}%
\definecolor{currentstroke}{rgb}{0.000000,0.000000,0.000000}%
\pgfsetstrokecolor{currentstroke}%
\pgfsetdash{}{0pt}%
\pgfsys@defobject{currentmarker}{\pgfqpoint{0.000000in}{0.000000in}}{\pgfqpoint{0.027778in}{0.000000in}}{%
\pgfpathmoveto{\pgfqpoint{0.000000in}{0.000000in}}%
\pgfpathlineto{\pgfqpoint{0.027778in}{0.000000in}}%
\pgfusepath{stroke,fill}%
}%
\begin{pgfscope}%
\pgfsys@transformshift{0.485787in}{0.813257in}%
\pgfsys@useobject{currentmarker}{}%
\end{pgfscope}%
\end{pgfscope}%
\begin{pgfscope}%
\pgfsetbuttcap%
\pgfsetroundjoin%
\definecolor{currentfill}{rgb}{0.000000,0.000000,0.000000}%
\pgfsetfillcolor{currentfill}%
\pgfsetlinewidth{0.501875pt}%
\definecolor{currentstroke}{rgb}{0.000000,0.000000,0.000000}%
\pgfsetstrokecolor{currentstroke}%
\pgfsetdash{}{0pt}%
\pgfsys@defobject{currentmarker}{\pgfqpoint{-0.027778in}{0.000000in}}{\pgfqpoint{0.000000in}{0.000000in}}{%
\pgfpathmoveto{\pgfqpoint{0.000000in}{0.000000in}}%
\pgfpathlineto{\pgfqpoint{-0.027778in}{0.000000in}}%
\pgfusepath{stroke,fill}%
}%
\begin{pgfscope}%
\pgfsys@transformshift{3.506457in}{0.813257in}%
\pgfsys@useobject{currentmarker}{}%
\end{pgfscope}%
\end{pgfscope}%
\begin{pgfscope}%
\pgfsetbuttcap%
\pgfsetroundjoin%
\definecolor{currentfill}{rgb}{0.000000,0.000000,0.000000}%
\pgfsetfillcolor{currentfill}%
\pgfsetlinewidth{0.501875pt}%
\definecolor{currentstroke}{rgb}{0.000000,0.000000,0.000000}%
\pgfsetstrokecolor{currentstroke}%
\pgfsetdash{}{0pt}%
\pgfsys@defobject{currentmarker}{\pgfqpoint{0.000000in}{0.000000in}}{\pgfqpoint{0.027778in}{0.000000in}}{%
\pgfpathmoveto{\pgfqpoint{0.000000in}{0.000000in}}%
\pgfpathlineto{\pgfqpoint{0.027778in}{0.000000in}}%
\pgfusepath{stroke,fill}%
}%
\begin{pgfscope}%
\pgfsys@transformshift{0.485787in}{0.856719in}%
\pgfsys@useobject{currentmarker}{}%
\end{pgfscope}%
\end{pgfscope}%
\begin{pgfscope}%
\pgfsetbuttcap%
\pgfsetroundjoin%
\definecolor{currentfill}{rgb}{0.000000,0.000000,0.000000}%
\pgfsetfillcolor{currentfill}%
\pgfsetlinewidth{0.501875pt}%
\definecolor{currentstroke}{rgb}{0.000000,0.000000,0.000000}%
\pgfsetstrokecolor{currentstroke}%
\pgfsetdash{}{0pt}%
\pgfsys@defobject{currentmarker}{\pgfqpoint{-0.027778in}{0.000000in}}{\pgfqpoint{0.000000in}{0.000000in}}{%
\pgfpathmoveto{\pgfqpoint{0.000000in}{0.000000in}}%
\pgfpathlineto{\pgfqpoint{-0.027778in}{0.000000in}}%
\pgfusepath{stroke,fill}%
}%
\begin{pgfscope}%
\pgfsys@transformshift{3.506457in}{0.856719in}%
\pgfsys@useobject{currentmarker}{}%
\end{pgfscope}%
\end{pgfscope}%
\begin{pgfscope}%
\pgfsetbuttcap%
\pgfsetroundjoin%
\definecolor{currentfill}{rgb}{0.000000,0.000000,0.000000}%
\pgfsetfillcolor{currentfill}%
\pgfsetlinewidth{0.501875pt}%
\definecolor{currentstroke}{rgb}{0.000000,0.000000,0.000000}%
\pgfsetstrokecolor{currentstroke}%
\pgfsetdash{}{0pt}%
\pgfsys@defobject{currentmarker}{\pgfqpoint{0.000000in}{0.000000in}}{\pgfqpoint{0.027778in}{0.000000in}}{%
\pgfpathmoveto{\pgfqpoint{0.000000in}{0.000000in}}%
\pgfpathlineto{\pgfqpoint{0.027778in}{0.000000in}}%
\pgfusepath{stroke,fill}%
}%
\begin{pgfscope}%
\pgfsys@transformshift{0.485787in}{0.894367in}%
\pgfsys@useobject{currentmarker}{}%
\end{pgfscope}%
\end{pgfscope}%
\begin{pgfscope}%
\pgfsetbuttcap%
\pgfsetroundjoin%
\definecolor{currentfill}{rgb}{0.000000,0.000000,0.000000}%
\pgfsetfillcolor{currentfill}%
\pgfsetlinewidth{0.501875pt}%
\definecolor{currentstroke}{rgb}{0.000000,0.000000,0.000000}%
\pgfsetstrokecolor{currentstroke}%
\pgfsetdash{}{0pt}%
\pgfsys@defobject{currentmarker}{\pgfqpoint{-0.027778in}{0.000000in}}{\pgfqpoint{0.000000in}{0.000000in}}{%
\pgfpathmoveto{\pgfqpoint{0.000000in}{0.000000in}}%
\pgfpathlineto{\pgfqpoint{-0.027778in}{0.000000in}}%
\pgfusepath{stroke,fill}%
}%
\begin{pgfscope}%
\pgfsys@transformshift{3.506457in}{0.894367in}%
\pgfsys@useobject{currentmarker}{}%
\end{pgfscope}%
\end{pgfscope}%
\begin{pgfscope}%
\pgfsetbuttcap%
\pgfsetroundjoin%
\definecolor{currentfill}{rgb}{0.000000,0.000000,0.000000}%
\pgfsetfillcolor{currentfill}%
\pgfsetlinewidth{0.501875pt}%
\definecolor{currentstroke}{rgb}{0.000000,0.000000,0.000000}%
\pgfsetstrokecolor{currentstroke}%
\pgfsetdash{}{0pt}%
\pgfsys@defobject{currentmarker}{\pgfqpoint{0.000000in}{0.000000in}}{\pgfqpoint{0.027778in}{0.000000in}}{%
\pgfpathmoveto{\pgfqpoint{0.000000in}{0.000000in}}%
\pgfpathlineto{\pgfqpoint{0.027778in}{0.000000in}}%
\pgfusepath{stroke,fill}%
}%
\begin{pgfscope}%
\pgfsys@transformshift{0.485787in}{0.927575in}%
\pgfsys@useobject{currentmarker}{}%
\end{pgfscope}%
\end{pgfscope}%
\begin{pgfscope}%
\pgfsetbuttcap%
\pgfsetroundjoin%
\definecolor{currentfill}{rgb}{0.000000,0.000000,0.000000}%
\pgfsetfillcolor{currentfill}%
\pgfsetlinewidth{0.501875pt}%
\definecolor{currentstroke}{rgb}{0.000000,0.000000,0.000000}%
\pgfsetstrokecolor{currentstroke}%
\pgfsetdash{}{0pt}%
\pgfsys@defobject{currentmarker}{\pgfqpoint{-0.027778in}{0.000000in}}{\pgfqpoint{0.000000in}{0.000000in}}{%
\pgfpathmoveto{\pgfqpoint{0.000000in}{0.000000in}}%
\pgfpathlineto{\pgfqpoint{-0.027778in}{0.000000in}}%
\pgfusepath{stroke,fill}%
}%
\begin{pgfscope}%
\pgfsys@transformshift{3.506457in}{0.927575in}%
\pgfsys@useobject{currentmarker}{}%
\end{pgfscope}%
\end{pgfscope}%
\begin{pgfscope}%
\pgfsetbuttcap%
\pgfsetroundjoin%
\definecolor{currentfill}{rgb}{0.000000,0.000000,0.000000}%
\pgfsetfillcolor{currentfill}%
\pgfsetlinewidth{0.501875pt}%
\definecolor{currentstroke}{rgb}{0.000000,0.000000,0.000000}%
\pgfsetstrokecolor{currentstroke}%
\pgfsetdash{}{0pt}%
\pgfsys@defobject{currentmarker}{\pgfqpoint{0.000000in}{0.000000in}}{\pgfqpoint{0.027778in}{0.000000in}}{%
\pgfpathmoveto{\pgfqpoint{0.000000in}{0.000000in}}%
\pgfpathlineto{\pgfqpoint{0.027778in}{0.000000in}}%
\pgfusepath{stroke,fill}%
}%
\begin{pgfscope}%
\pgfsys@transformshift{0.485787in}{1.152708in}%
\pgfsys@useobject{currentmarker}{}%
\end{pgfscope}%
\end{pgfscope}%
\begin{pgfscope}%
\pgfsetbuttcap%
\pgfsetroundjoin%
\definecolor{currentfill}{rgb}{0.000000,0.000000,0.000000}%
\pgfsetfillcolor{currentfill}%
\pgfsetlinewidth{0.501875pt}%
\definecolor{currentstroke}{rgb}{0.000000,0.000000,0.000000}%
\pgfsetstrokecolor{currentstroke}%
\pgfsetdash{}{0pt}%
\pgfsys@defobject{currentmarker}{\pgfqpoint{-0.027778in}{0.000000in}}{\pgfqpoint{0.000000in}{0.000000in}}{%
\pgfpathmoveto{\pgfqpoint{0.000000in}{0.000000in}}%
\pgfpathlineto{\pgfqpoint{-0.027778in}{0.000000in}}%
\pgfusepath{stroke,fill}%
}%
\begin{pgfscope}%
\pgfsys@transformshift{3.506457in}{1.152708in}%
\pgfsys@useobject{currentmarker}{}%
\end{pgfscope}%
\end{pgfscope}%
\begin{pgfscope}%
\pgfsetbuttcap%
\pgfsetroundjoin%
\definecolor{currentfill}{rgb}{0.000000,0.000000,0.000000}%
\pgfsetfillcolor{currentfill}%
\pgfsetlinewidth{0.501875pt}%
\definecolor{currentstroke}{rgb}{0.000000,0.000000,0.000000}%
\pgfsetstrokecolor{currentstroke}%
\pgfsetdash{}{0pt}%
\pgfsys@defobject{currentmarker}{\pgfqpoint{0.000000in}{0.000000in}}{\pgfqpoint{0.027778in}{0.000000in}}{%
\pgfpathmoveto{\pgfqpoint{0.000000in}{0.000000in}}%
\pgfpathlineto{\pgfqpoint{0.027778in}{0.000000in}}%
\pgfusepath{stroke,fill}%
}%
\begin{pgfscope}%
\pgfsys@transformshift{0.485787in}{1.267025in}%
\pgfsys@useobject{currentmarker}{}%
\end{pgfscope}%
\end{pgfscope}%
\begin{pgfscope}%
\pgfsetbuttcap%
\pgfsetroundjoin%
\definecolor{currentfill}{rgb}{0.000000,0.000000,0.000000}%
\pgfsetfillcolor{currentfill}%
\pgfsetlinewidth{0.501875pt}%
\definecolor{currentstroke}{rgb}{0.000000,0.000000,0.000000}%
\pgfsetstrokecolor{currentstroke}%
\pgfsetdash{}{0pt}%
\pgfsys@defobject{currentmarker}{\pgfqpoint{-0.027778in}{0.000000in}}{\pgfqpoint{0.000000in}{0.000000in}}{%
\pgfpathmoveto{\pgfqpoint{0.000000in}{0.000000in}}%
\pgfpathlineto{\pgfqpoint{-0.027778in}{0.000000in}}%
\pgfusepath{stroke,fill}%
}%
\begin{pgfscope}%
\pgfsys@transformshift{3.506457in}{1.267025in}%
\pgfsys@useobject{currentmarker}{}%
\end{pgfscope}%
\end{pgfscope}%
\begin{pgfscope}%
\pgfsetbuttcap%
\pgfsetroundjoin%
\definecolor{currentfill}{rgb}{0.000000,0.000000,0.000000}%
\pgfsetfillcolor{currentfill}%
\pgfsetlinewidth{0.501875pt}%
\definecolor{currentstroke}{rgb}{0.000000,0.000000,0.000000}%
\pgfsetstrokecolor{currentstroke}%
\pgfsetdash{}{0pt}%
\pgfsys@defobject{currentmarker}{\pgfqpoint{0.000000in}{0.000000in}}{\pgfqpoint{0.027778in}{0.000000in}}{%
\pgfpathmoveto{\pgfqpoint{0.000000in}{0.000000in}}%
\pgfpathlineto{\pgfqpoint{0.027778in}{0.000000in}}%
\pgfusepath{stroke,fill}%
}%
\begin{pgfscope}%
\pgfsys@transformshift{0.485787in}{1.348135in}%
\pgfsys@useobject{currentmarker}{}%
\end{pgfscope}%
\end{pgfscope}%
\begin{pgfscope}%
\pgfsetbuttcap%
\pgfsetroundjoin%
\definecolor{currentfill}{rgb}{0.000000,0.000000,0.000000}%
\pgfsetfillcolor{currentfill}%
\pgfsetlinewidth{0.501875pt}%
\definecolor{currentstroke}{rgb}{0.000000,0.000000,0.000000}%
\pgfsetstrokecolor{currentstroke}%
\pgfsetdash{}{0pt}%
\pgfsys@defobject{currentmarker}{\pgfqpoint{-0.027778in}{0.000000in}}{\pgfqpoint{0.000000in}{0.000000in}}{%
\pgfpathmoveto{\pgfqpoint{0.000000in}{0.000000in}}%
\pgfpathlineto{\pgfqpoint{-0.027778in}{0.000000in}}%
\pgfusepath{stroke,fill}%
}%
\begin{pgfscope}%
\pgfsys@transformshift{3.506457in}{1.348135in}%
\pgfsys@useobject{currentmarker}{}%
\end{pgfscope}%
\end{pgfscope}%
\begin{pgfscope}%
\pgfsetbuttcap%
\pgfsetroundjoin%
\definecolor{currentfill}{rgb}{0.000000,0.000000,0.000000}%
\pgfsetfillcolor{currentfill}%
\pgfsetlinewidth{0.501875pt}%
\definecolor{currentstroke}{rgb}{0.000000,0.000000,0.000000}%
\pgfsetstrokecolor{currentstroke}%
\pgfsetdash{}{0pt}%
\pgfsys@defobject{currentmarker}{\pgfqpoint{0.000000in}{0.000000in}}{\pgfqpoint{0.027778in}{0.000000in}}{%
\pgfpathmoveto{\pgfqpoint{0.000000in}{0.000000in}}%
\pgfpathlineto{\pgfqpoint{0.027778in}{0.000000in}}%
\pgfusepath{stroke,fill}%
}%
\begin{pgfscope}%
\pgfsys@transformshift{0.485787in}{1.411048in}%
\pgfsys@useobject{currentmarker}{}%
\end{pgfscope}%
\end{pgfscope}%
\begin{pgfscope}%
\pgfsetbuttcap%
\pgfsetroundjoin%
\definecolor{currentfill}{rgb}{0.000000,0.000000,0.000000}%
\pgfsetfillcolor{currentfill}%
\pgfsetlinewidth{0.501875pt}%
\definecolor{currentstroke}{rgb}{0.000000,0.000000,0.000000}%
\pgfsetstrokecolor{currentstroke}%
\pgfsetdash{}{0pt}%
\pgfsys@defobject{currentmarker}{\pgfqpoint{-0.027778in}{0.000000in}}{\pgfqpoint{0.000000in}{0.000000in}}{%
\pgfpathmoveto{\pgfqpoint{0.000000in}{0.000000in}}%
\pgfpathlineto{\pgfqpoint{-0.027778in}{0.000000in}}%
\pgfusepath{stroke,fill}%
}%
\begin{pgfscope}%
\pgfsys@transformshift{3.506457in}{1.411048in}%
\pgfsys@useobject{currentmarker}{}%
\end{pgfscope}%
\end{pgfscope}%
\begin{pgfscope}%
\pgfsetbuttcap%
\pgfsetroundjoin%
\definecolor{currentfill}{rgb}{0.000000,0.000000,0.000000}%
\pgfsetfillcolor{currentfill}%
\pgfsetlinewidth{0.501875pt}%
\definecolor{currentstroke}{rgb}{0.000000,0.000000,0.000000}%
\pgfsetstrokecolor{currentstroke}%
\pgfsetdash{}{0pt}%
\pgfsys@defobject{currentmarker}{\pgfqpoint{0.000000in}{0.000000in}}{\pgfqpoint{0.027778in}{0.000000in}}{%
\pgfpathmoveto{\pgfqpoint{0.000000in}{0.000000in}}%
\pgfpathlineto{\pgfqpoint{0.027778in}{0.000000in}}%
\pgfusepath{stroke,fill}%
}%
\begin{pgfscope}%
\pgfsys@transformshift{0.485787in}{1.462453in}%
\pgfsys@useobject{currentmarker}{}%
\end{pgfscope}%
\end{pgfscope}%
\begin{pgfscope}%
\pgfsetbuttcap%
\pgfsetroundjoin%
\definecolor{currentfill}{rgb}{0.000000,0.000000,0.000000}%
\pgfsetfillcolor{currentfill}%
\pgfsetlinewidth{0.501875pt}%
\definecolor{currentstroke}{rgb}{0.000000,0.000000,0.000000}%
\pgfsetstrokecolor{currentstroke}%
\pgfsetdash{}{0pt}%
\pgfsys@defobject{currentmarker}{\pgfqpoint{-0.027778in}{0.000000in}}{\pgfqpoint{0.000000in}{0.000000in}}{%
\pgfpathmoveto{\pgfqpoint{0.000000in}{0.000000in}}%
\pgfpathlineto{\pgfqpoint{-0.027778in}{0.000000in}}%
\pgfusepath{stroke,fill}%
}%
\begin{pgfscope}%
\pgfsys@transformshift{3.506457in}{1.462453in}%
\pgfsys@useobject{currentmarker}{}%
\end{pgfscope}%
\end{pgfscope}%
\begin{pgfscope}%
\pgfsetbuttcap%
\pgfsetroundjoin%
\definecolor{currentfill}{rgb}{0.000000,0.000000,0.000000}%
\pgfsetfillcolor{currentfill}%
\pgfsetlinewidth{0.501875pt}%
\definecolor{currentstroke}{rgb}{0.000000,0.000000,0.000000}%
\pgfsetstrokecolor{currentstroke}%
\pgfsetdash{}{0pt}%
\pgfsys@defobject{currentmarker}{\pgfqpoint{0.000000in}{0.000000in}}{\pgfqpoint{0.027778in}{0.000000in}}{%
\pgfpathmoveto{\pgfqpoint{0.000000in}{0.000000in}}%
\pgfpathlineto{\pgfqpoint{0.027778in}{0.000000in}}%
\pgfusepath{stroke,fill}%
}%
\begin{pgfscope}%
\pgfsys@transformshift{0.485787in}{1.505914in}%
\pgfsys@useobject{currentmarker}{}%
\end{pgfscope}%
\end{pgfscope}%
\begin{pgfscope}%
\pgfsetbuttcap%
\pgfsetroundjoin%
\definecolor{currentfill}{rgb}{0.000000,0.000000,0.000000}%
\pgfsetfillcolor{currentfill}%
\pgfsetlinewidth{0.501875pt}%
\definecolor{currentstroke}{rgb}{0.000000,0.000000,0.000000}%
\pgfsetstrokecolor{currentstroke}%
\pgfsetdash{}{0pt}%
\pgfsys@defobject{currentmarker}{\pgfqpoint{-0.027778in}{0.000000in}}{\pgfqpoint{0.000000in}{0.000000in}}{%
\pgfpathmoveto{\pgfqpoint{0.000000in}{0.000000in}}%
\pgfpathlineto{\pgfqpoint{-0.027778in}{0.000000in}}%
\pgfusepath{stroke,fill}%
}%
\begin{pgfscope}%
\pgfsys@transformshift{3.506457in}{1.505914in}%
\pgfsys@useobject{currentmarker}{}%
\end{pgfscope}%
\end{pgfscope}%
\begin{pgfscope}%
\pgfsetbuttcap%
\pgfsetroundjoin%
\definecolor{currentfill}{rgb}{0.000000,0.000000,0.000000}%
\pgfsetfillcolor{currentfill}%
\pgfsetlinewidth{0.501875pt}%
\definecolor{currentstroke}{rgb}{0.000000,0.000000,0.000000}%
\pgfsetstrokecolor{currentstroke}%
\pgfsetdash{}{0pt}%
\pgfsys@defobject{currentmarker}{\pgfqpoint{0.000000in}{0.000000in}}{\pgfqpoint{0.027778in}{0.000000in}}{%
\pgfpathmoveto{\pgfqpoint{0.000000in}{0.000000in}}%
\pgfpathlineto{\pgfqpoint{0.027778in}{0.000000in}}%
\pgfusepath{stroke,fill}%
}%
\begin{pgfscope}%
\pgfsys@transformshift{0.485787in}{1.543562in}%
\pgfsys@useobject{currentmarker}{}%
\end{pgfscope}%
\end{pgfscope}%
\begin{pgfscope}%
\pgfsetbuttcap%
\pgfsetroundjoin%
\definecolor{currentfill}{rgb}{0.000000,0.000000,0.000000}%
\pgfsetfillcolor{currentfill}%
\pgfsetlinewidth{0.501875pt}%
\definecolor{currentstroke}{rgb}{0.000000,0.000000,0.000000}%
\pgfsetstrokecolor{currentstroke}%
\pgfsetdash{}{0pt}%
\pgfsys@defobject{currentmarker}{\pgfqpoint{-0.027778in}{0.000000in}}{\pgfqpoint{0.000000in}{0.000000in}}{%
\pgfpathmoveto{\pgfqpoint{0.000000in}{0.000000in}}%
\pgfpathlineto{\pgfqpoint{-0.027778in}{0.000000in}}%
\pgfusepath{stroke,fill}%
}%
\begin{pgfscope}%
\pgfsys@transformshift{3.506457in}{1.543562in}%
\pgfsys@useobject{currentmarker}{}%
\end{pgfscope}%
\end{pgfscope}%
\begin{pgfscope}%
\pgfsetbuttcap%
\pgfsetroundjoin%
\definecolor{currentfill}{rgb}{0.000000,0.000000,0.000000}%
\pgfsetfillcolor{currentfill}%
\pgfsetlinewidth{0.501875pt}%
\definecolor{currentstroke}{rgb}{0.000000,0.000000,0.000000}%
\pgfsetstrokecolor{currentstroke}%
\pgfsetdash{}{0pt}%
\pgfsys@defobject{currentmarker}{\pgfqpoint{0.000000in}{0.000000in}}{\pgfqpoint{0.027778in}{0.000000in}}{%
\pgfpathmoveto{\pgfqpoint{0.000000in}{0.000000in}}%
\pgfpathlineto{\pgfqpoint{0.027778in}{0.000000in}}%
\pgfusepath{stroke,fill}%
}%
\begin{pgfscope}%
\pgfsys@transformshift{0.485787in}{1.576770in}%
\pgfsys@useobject{currentmarker}{}%
\end{pgfscope}%
\end{pgfscope}%
\begin{pgfscope}%
\pgfsetbuttcap%
\pgfsetroundjoin%
\definecolor{currentfill}{rgb}{0.000000,0.000000,0.000000}%
\pgfsetfillcolor{currentfill}%
\pgfsetlinewidth{0.501875pt}%
\definecolor{currentstroke}{rgb}{0.000000,0.000000,0.000000}%
\pgfsetstrokecolor{currentstroke}%
\pgfsetdash{}{0pt}%
\pgfsys@defobject{currentmarker}{\pgfqpoint{-0.027778in}{0.000000in}}{\pgfqpoint{0.000000in}{0.000000in}}{%
\pgfpathmoveto{\pgfqpoint{0.000000in}{0.000000in}}%
\pgfpathlineto{\pgfqpoint{-0.027778in}{0.000000in}}%
\pgfusepath{stroke,fill}%
}%
\begin{pgfscope}%
\pgfsys@transformshift{3.506457in}{1.576770in}%
\pgfsys@useobject{currentmarker}{}%
\end{pgfscope}%
\end{pgfscope}%
\begin{pgfscope}%
\pgfsetbuttcap%
\pgfsetroundjoin%
\definecolor{currentfill}{rgb}{0.000000,0.000000,0.000000}%
\pgfsetfillcolor{currentfill}%
\pgfsetlinewidth{0.501875pt}%
\definecolor{currentstroke}{rgb}{0.000000,0.000000,0.000000}%
\pgfsetstrokecolor{currentstroke}%
\pgfsetdash{}{0pt}%
\pgfsys@defobject{currentmarker}{\pgfqpoint{0.000000in}{0.000000in}}{\pgfqpoint{0.027778in}{0.000000in}}{%
\pgfpathmoveto{\pgfqpoint{0.000000in}{0.000000in}}%
\pgfpathlineto{\pgfqpoint{0.027778in}{0.000000in}}%
\pgfusepath{stroke,fill}%
}%
\begin{pgfscope}%
\pgfsys@transformshift{0.485787in}{1.801903in}%
\pgfsys@useobject{currentmarker}{}%
\end{pgfscope}%
\end{pgfscope}%
\begin{pgfscope}%
\pgfsetbuttcap%
\pgfsetroundjoin%
\definecolor{currentfill}{rgb}{0.000000,0.000000,0.000000}%
\pgfsetfillcolor{currentfill}%
\pgfsetlinewidth{0.501875pt}%
\definecolor{currentstroke}{rgb}{0.000000,0.000000,0.000000}%
\pgfsetstrokecolor{currentstroke}%
\pgfsetdash{}{0pt}%
\pgfsys@defobject{currentmarker}{\pgfqpoint{-0.027778in}{0.000000in}}{\pgfqpoint{0.000000in}{0.000000in}}{%
\pgfpathmoveto{\pgfqpoint{0.000000in}{0.000000in}}%
\pgfpathlineto{\pgfqpoint{-0.027778in}{0.000000in}}%
\pgfusepath{stroke,fill}%
}%
\begin{pgfscope}%
\pgfsys@transformshift{3.506457in}{1.801903in}%
\pgfsys@useobject{currentmarker}{}%
\end{pgfscope}%
\end{pgfscope}%
\begin{pgfscope}%
\pgfsetbuttcap%
\pgfsetroundjoin%
\definecolor{currentfill}{rgb}{0.000000,0.000000,0.000000}%
\pgfsetfillcolor{currentfill}%
\pgfsetlinewidth{0.501875pt}%
\definecolor{currentstroke}{rgb}{0.000000,0.000000,0.000000}%
\pgfsetstrokecolor{currentstroke}%
\pgfsetdash{}{0pt}%
\pgfsys@defobject{currentmarker}{\pgfqpoint{0.000000in}{0.000000in}}{\pgfqpoint{0.027778in}{0.000000in}}{%
\pgfpathmoveto{\pgfqpoint{0.000000in}{0.000000in}}%
\pgfpathlineto{\pgfqpoint{0.027778in}{0.000000in}}%
\pgfusepath{stroke,fill}%
}%
\begin{pgfscope}%
\pgfsys@transformshift{0.485787in}{1.916221in}%
\pgfsys@useobject{currentmarker}{}%
\end{pgfscope}%
\end{pgfscope}%
\begin{pgfscope}%
\pgfsetbuttcap%
\pgfsetroundjoin%
\definecolor{currentfill}{rgb}{0.000000,0.000000,0.000000}%
\pgfsetfillcolor{currentfill}%
\pgfsetlinewidth{0.501875pt}%
\definecolor{currentstroke}{rgb}{0.000000,0.000000,0.000000}%
\pgfsetstrokecolor{currentstroke}%
\pgfsetdash{}{0pt}%
\pgfsys@defobject{currentmarker}{\pgfqpoint{-0.027778in}{0.000000in}}{\pgfqpoint{0.000000in}{0.000000in}}{%
\pgfpathmoveto{\pgfqpoint{0.000000in}{0.000000in}}%
\pgfpathlineto{\pgfqpoint{-0.027778in}{0.000000in}}%
\pgfusepath{stroke,fill}%
}%
\begin{pgfscope}%
\pgfsys@transformshift{3.506457in}{1.916221in}%
\pgfsys@useobject{currentmarker}{}%
\end{pgfscope}%
\end{pgfscope}%
\begin{pgfscope}%
\pgfsetbuttcap%
\pgfsetroundjoin%
\definecolor{currentfill}{rgb}{0.000000,0.000000,0.000000}%
\pgfsetfillcolor{currentfill}%
\pgfsetlinewidth{0.501875pt}%
\definecolor{currentstroke}{rgb}{0.000000,0.000000,0.000000}%
\pgfsetstrokecolor{currentstroke}%
\pgfsetdash{}{0pt}%
\pgfsys@defobject{currentmarker}{\pgfqpoint{0.000000in}{0.000000in}}{\pgfqpoint{0.027778in}{0.000000in}}{%
\pgfpathmoveto{\pgfqpoint{0.000000in}{0.000000in}}%
\pgfpathlineto{\pgfqpoint{0.027778in}{0.000000in}}%
\pgfusepath{stroke,fill}%
}%
\begin{pgfscope}%
\pgfsys@transformshift{0.485787in}{1.997331in}%
\pgfsys@useobject{currentmarker}{}%
\end{pgfscope}%
\end{pgfscope}%
\begin{pgfscope}%
\pgfsetbuttcap%
\pgfsetroundjoin%
\definecolor{currentfill}{rgb}{0.000000,0.000000,0.000000}%
\pgfsetfillcolor{currentfill}%
\pgfsetlinewidth{0.501875pt}%
\definecolor{currentstroke}{rgb}{0.000000,0.000000,0.000000}%
\pgfsetstrokecolor{currentstroke}%
\pgfsetdash{}{0pt}%
\pgfsys@defobject{currentmarker}{\pgfqpoint{-0.027778in}{0.000000in}}{\pgfqpoint{0.000000in}{0.000000in}}{%
\pgfpathmoveto{\pgfqpoint{0.000000in}{0.000000in}}%
\pgfpathlineto{\pgfqpoint{-0.027778in}{0.000000in}}%
\pgfusepath{stroke,fill}%
}%
\begin{pgfscope}%
\pgfsys@transformshift{3.506457in}{1.997331in}%
\pgfsys@useobject{currentmarker}{}%
\end{pgfscope}%
\end{pgfscope}%
\begin{pgfscope}%
\pgfsetbuttcap%
\pgfsetroundjoin%
\definecolor{currentfill}{rgb}{0.000000,0.000000,0.000000}%
\pgfsetfillcolor{currentfill}%
\pgfsetlinewidth{0.501875pt}%
\definecolor{currentstroke}{rgb}{0.000000,0.000000,0.000000}%
\pgfsetstrokecolor{currentstroke}%
\pgfsetdash{}{0pt}%
\pgfsys@defobject{currentmarker}{\pgfqpoint{0.000000in}{0.000000in}}{\pgfqpoint{0.027778in}{0.000000in}}{%
\pgfpathmoveto{\pgfqpoint{0.000000in}{0.000000in}}%
\pgfpathlineto{\pgfqpoint{0.027778in}{0.000000in}}%
\pgfusepath{stroke,fill}%
}%
\begin{pgfscope}%
\pgfsys@transformshift{0.485787in}{2.060244in}%
\pgfsys@useobject{currentmarker}{}%
\end{pgfscope}%
\end{pgfscope}%
\begin{pgfscope}%
\pgfsetbuttcap%
\pgfsetroundjoin%
\definecolor{currentfill}{rgb}{0.000000,0.000000,0.000000}%
\pgfsetfillcolor{currentfill}%
\pgfsetlinewidth{0.501875pt}%
\definecolor{currentstroke}{rgb}{0.000000,0.000000,0.000000}%
\pgfsetstrokecolor{currentstroke}%
\pgfsetdash{}{0pt}%
\pgfsys@defobject{currentmarker}{\pgfqpoint{-0.027778in}{0.000000in}}{\pgfqpoint{0.000000in}{0.000000in}}{%
\pgfpathmoveto{\pgfqpoint{0.000000in}{0.000000in}}%
\pgfpathlineto{\pgfqpoint{-0.027778in}{0.000000in}}%
\pgfusepath{stroke,fill}%
}%
\begin{pgfscope}%
\pgfsys@transformshift{3.506457in}{2.060244in}%
\pgfsys@useobject{currentmarker}{}%
\end{pgfscope}%
\end{pgfscope}%
\begin{pgfscope}%
\pgfsetbuttcap%
\pgfsetroundjoin%
\definecolor{currentfill}{rgb}{0.000000,0.000000,0.000000}%
\pgfsetfillcolor{currentfill}%
\pgfsetlinewidth{0.501875pt}%
\definecolor{currentstroke}{rgb}{0.000000,0.000000,0.000000}%
\pgfsetstrokecolor{currentstroke}%
\pgfsetdash{}{0pt}%
\pgfsys@defobject{currentmarker}{\pgfqpoint{0.000000in}{0.000000in}}{\pgfqpoint{0.027778in}{0.000000in}}{%
\pgfpathmoveto{\pgfqpoint{0.000000in}{0.000000in}}%
\pgfpathlineto{\pgfqpoint{0.027778in}{0.000000in}}%
\pgfusepath{stroke,fill}%
}%
\begin{pgfscope}%
\pgfsys@transformshift{0.485787in}{2.111648in}%
\pgfsys@useobject{currentmarker}{}%
\end{pgfscope}%
\end{pgfscope}%
\begin{pgfscope}%
\pgfsetbuttcap%
\pgfsetroundjoin%
\definecolor{currentfill}{rgb}{0.000000,0.000000,0.000000}%
\pgfsetfillcolor{currentfill}%
\pgfsetlinewidth{0.501875pt}%
\definecolor{currentstroke}{rgb}{0.000000,0.000000,0.000000}%
\pgfsetstrokecolor{currentstroke}%
\pgfsetdash{}{0pt}%
\pgfsys@defobject{currentmarker}{\pgfqpoint{-0.027778in}{0.000000in}}{\pgfqpoint{0.000000in}{0.000000in}}{%
\pgfpathmoveto{\pgfqpoint{0.000000in}{0.000000in}}%
\pgfpathlineto{\pgfqpoint{-0.027778in}{0.000000in}}%
\pgfusepath{stroke,fill}%
}%
\begin{pgfscope}%
\pgfsys@transformshift{3.506457in}{2.111648in}%
\pgfsys@useobject{currentmarker}{}%
\end{pgfscope}%
\end{pgfscope}%
\begin{pgfscope}%
\pgfsetbuttcap%
\pgfsetroundjoin%
\definecolor{currentfill}{rgb}{0.000000,0.000000,0.000000}%
\pgfsetfillcolor{currentfill}%
\pgfsetlinewidth{0.501875pt}%
\definecolor{currentstroke}{rgb}{0.000000,0.000000,0.000000}%
\pgfsetstrokecolor{currentstroke}%
\pgfsetdash{}{0pt}%
\pgfsys@defobject{currentmarker}{\pgfqpoint{0.000000in}{0.000000in}}{\pgfqpoint{0.027778in}{0.000000in}}{%
\pgfpathmoveto{\pgfqpoint{0.000000in}{0.000000in}}%
\pgfpathlineto{\pgfqpoint{0.027778in}{0.000000in}}%
\pgfusepath{stroke,fill}%
}%
\begin{pgfscope}%
\pgfsys@transformshift{0.485787in}{2.155110in}%
\pgfsys@useobject{currentmarker}{}%
\end{pgfscope}%
\end{pgfscope}%
\begin{pgfscope}%
\pgfsetbuttcap%
\pgfsetroundjoin%
\definecolor{currentfill}{rgb}{0.000000,0.000000,0.000000}%
\pgfsetfillcolor{currentfill}%
\pgfsetlinewidth{0.501875pt}%
\definecolor{currentstroke}{rgb}{0.000000,0.000000,0.000000}%
\pgfsetstrokecolor{currentstroke}%
\pgfsetdash{}{0pt}%
\pgfsys@defobject{currentmarker}{\pgfqpoint{-0.027778in}{0.000000in}}{\pgfqpoint{0.000000in}{0.000000in}}{%
\pgfpathmoveto{\pgfqpoint{0.000000in}{0.000000in}}%
\pgfpathlineto{\pgfqpoint{-0.027778in}{0.000000in}}%
\pgfusepath{stroke,fill}%
}%
\begin{pgfscope}%
\pgfsys@transformshift{3.506457in}{2.155110in}%
\pgfsys@useobject{currentmarker}{}%
\end{pgfscope}%
\end{pgfscope}%
\begin{pgfscope}%
\pgfsetbuttcap%
\pgfsetroundjoin%
\definecolor{currentfill}{rgb}{0.000000,0.000000,0.000000}%
\pgfsetfillcolor{currentfill}%
\pgfsetlinewidth{0.501875pt}%
\definecolor{currentstroke}{rgb}{0.000000,0.000000,0.000000}%
\pgfsetstrokecolor{currentstroke}%
\pgfsetdash{}{0pt}%
\pgfsys@defobject{currentmarker}{\pgfqpoint{0.000000in}{0.000000in}}{\pgfqpoint{0.027778in}{0.000000in}}{%
\pgfpathmoveto{\pgfqpoint{0.000000in}{0.000000in}}%
\pgfpathlineto{\pgfqpoint{0.027778in}{0.000000in}}%
\pgfusepath{stroke,fill}%
}%
\begin{pgfscope}%
\pgfsys@transformshift{0.485787in}{2.192758in}%
\pgfsys@useobject{currentmarker}{}%
\end{pgfscope}%
\end{pgfscope}%
\begin{pgfscope}%
\pgfsetbuttcap%
\pgfsetroundjoin%
\definecolor{currentfill}{rgb}{0.000000,0.000000,0.000000}%
\pgfsetfillcolor{currentfill}%
\pgfsetlinewidth{0.501875pt}%
\definecolor{currentstroke}{rgb}{0.000000,0.000000,0.000000}%
\pgfsetstrokecolor{currentstroke}%
\pgfsetdash{}{0pt}%
\pgfsys@defobject{currentmarker}{\pgfqpoint{-0.027778in}{0.000000in}}{\pgfqpoint{0.000000in}{0.000000in}}{%
\pgfpathmoveto{\pgfqpoint{0.000000in}{0.000000in}}%
\pgfpathlineto{\pgfqpoint{-0.027778in}{0.000000in}}%
\pgfusepath{stroke,fill}%
}%
\begin{pgfscope}%
\pgfsys@transformshift{3.506457in}{2.192758in}%
\pgfsys@useobject{currentmarker}{}%
\end{pgfscope}%
\end{pgfscope}%
\begin{pgfscope}%
\pgfsetbuttcap%
\pgfsetroundjoin%
\definecolor{currentfill}{rgb}{0.000000,0.000000,0.000000}%
\pgfsetfillcolor{currentfill}%
\pgfsetlinewidth{0.501875pt}%
\definecolor{currentstroke}{rgb}{0.000000,0.000000,0.000000}%
\pgfsetstrokecolor{currentstroke}%
\pgfsetdash{}{0pt}%
\pgfsys@defobject{currentmarker}{\pgfqpoint{0.000000in}{0.000000in}}{\pgfqpoint{0.027778in}{0.000000in}}{%
\pgfpathmoveto{\pgfqpoint{0.000000in}{0.000000in}}%
\pgfpathlineto{\pgfqpoint{0.027778in}{0.000000in}}%
\pgfusepath{stroke,fill}%
}%
\begin{pgfscope}%
\pgfsys@transformshift{0.485787in}{2.225966in}%
\pgfsys@useobject{currentmarker}{}%
\end{pgfscope}%
\end{pgfscope}%
\begin{pgfscope}%
\pgfsetbuttcap%
\pgfsetroundjoin%
\definecolor{currentfill}{rgb}{0.000000,0.000000,0.000000}%
\pgfsetfillcolor{currentfill}%
\pgfsetlinewidth{0.501875pt}%
\definecolor{currentstroke}{rgb}{0.000000,0.000000,0.000000}%
\pgfsetstrokecolor{currentstroke}%
\pgfsetdash{}{0pt}%
\pgfsys@defobject{currentmarker}{\pgfqpoint{-0.027778in}{0.000000in}}{\pgfqpoint{0.000000in}{0.000000in}}{%
\pgfpathmoveto{\pgfqpoint{0.000000in}{0.000000in}}%
\pgfpathlineto{\pgfqpoint{-0.027778in}{0.000000in}}%
\pgfusepath{stroke,fill}%
}%
\begin{pgfscope}%
\pgfsys@transformshift{3.506457in}{2.225966in}%
\pgfsys@useobject{currentmarker}{}%
\end{pgfscope}%
\end{pgfscope}%
\begin{pgfscope}%
\pgfsetbuttcap%
\pgfsetroundjoin%
\definecolor{currentfill}{rgb}{0.000000,0.000000,0.000000}%
\pgfsetfillcolor{currentfill}%
\pgfsetlinewidth{0.501875pt}%
\definecolor{currentstroke}{rgb}{0.000000,0.000000,0.000000}%
\pgfsetstrokecolor{currentstroke}%
\pgfsetdash{}{0pt}%
\pgfsys@defobject{currentmarker}{\pgfqpoint{0.000000in}{0.000000in}}{\pgfqpoint{0.027778in}{0.000000in}}{%
\pgfpathmoveto{\pgfqpoint{0.000000in}{0.000000in}}%
\pgfpathlineto{\pgfqpoint{0.027778in}{0.000000in}}%
\pgfusepath{stroke,fill}%
}%
\begin{pgfscope}%
\pgfsys@transformshift{0.485787in}{2.451099in}%
\pgfsys@useobject{currentmarker}{}%
\end{pgfscope}%
\end{pgfscope}%
\begin{pgfscope}%
\pgfsetbuttcap%
\pgfsetroundjoin%
\definecolor{currentfill}{rgb}{0.000000,0.000000,0.000000}%
\pgfsetfillcolor{currentfill}%
\pgfsetlinewidth{0.501875pt}%
\definecolor{currentstroke}{rgb}{0.000000,0.000000,0.000000}%
\pgfsetstrokecolor{currentstroke}%
\pgfsetdash{}{0pt}%
\pgfsys@defobject{currentmarker}{\pgfqpoint{-0.027778in}{0.000000in}}{\pgfqpoint{0.000000in}{0.000000in}}{%
\pgfpathmoveto{\pgfqpoint{0.000000in}{0.000000in}}%
\pgfpathlineto{\pgfqpoint{-0.027778in}{0.000000in}}%
\pgfusepath{stroke,fill}%
}%
\begin{pgfscope}%
\pgfsys@transformshift{3.506457in}{2.451099in}%
\pgfsys@useobject{currentmarker}{}%
\end{pgfscope}%
\end{pgfscope}%
\begin{pgfscope}%
\pgfsetbuttcap%
\pgfsetroundjoin%
\definecolor{currentfill}{rgb}{0.000000,0.000000,0.000000}%
\pgfsetfillcolor{currentfill}%
\pgfsetlinewidth{0.501875pt}%
\definecolor{currentstroke}{rgb}{0.000000,0.000000,0.000000}%
\pgfsetstrokecolor{currentstroke}%
\pgfsetdash{}{0pt}%
\pgfsys@defobject{currentmarker}{\pgfqpoint{0.000000in}{0.000000in}}{\pgfqpoint{0.027778in}{0.000000in}}{%
\pgfpathmoveto{\pgfqpoint{0.000000in}{0.000000in}}%
\pgfpathlineto{\pgfqpoint{0.027778in}{0.000000in}}%
\pgfusepath{stroke,fill}%
}%
\begin{pgfscope}%
\pgfsys@transformshift{0.485787in}{2.565417in}%
\pgfsys@useobject{currentmarker}{}%
\end{pgfscope}%
\end{pgfscope}%
\begin{pgfscope}%
\pgfsetbuttcap%
\pgfsetroundjoin%
\definecolor{currentfill}{rgb}{0.000000,0.000000,0.000000}%
\pgfsetfillcolor{currentfill}%
\pgfsetlinewidth{0.501875pt}%
\definecolor{currentstroke}{rgb}{0.000000,0.000000,0.000000}%
\pgfsetstrokecolor{currentstroke}%
\pgfsetdash{}{0pt}%
\pgfsys@defobject{currentmarker}{\pgfqpoint{-0.027778in}{0.000000in}}{\pgfqpoint{0.000000in}{0.000000in}}{%
\pgfpathmoveto{\pgfqpoint{0.000000in}{0.000000in}}%
\pgfpathlineto{\pgfqpoint{-0.027778in}{0.000000in}}%
\pgfusepath{stroke,fill}%
}%
\begin{pgfscope}%
\pgfsys@transformshift{3.506457in}{2.565417in}%
\pgfsys@useobject{currentmarker}{}%
\end{pgfscope}%
\end{pgfscope}%
\begin{pgfscope}%
\pgfsetbuttcap%
\pgfsetroundjoin%
\definecolor{currentfill}{rgb}{0.000000,0.000000,0.000000}%
\pgfsetfillcolor{currentfill}%
\pgfsetlinewidth{0.501875pt}%
\definecolor{currentstroke}{rgb}{0.000000,0.000000,0.000000}%
\pgfsetstrokecolor{currentstroke}%
\pgfsetdash{}{0pt}%
\pgfsys@defobject{currentmarker}{\pgfqpoint{0.000000in}{0.000000in}}{\pgfqpoint{0.027778in}{0.000000in}}{%
\pgfpathmoveto{\pgfqpoint{0.000000in}{0.000000in}}%
\pgfpathlineto{\pgfqpoint{0.027778in}{0.000000in}}%
\pgfusepath{stroke,fill}%
}%
\begin{pgfscope}%
\pgfsys@transformshift{0.485787in}{2.646526in}%
\pgfsys@useobject{currentmarker}{}%
\end{pgfscope}%
\end{pgfscope}%
\begin{pgfscope}%
\pgfsetbuttcap%
\pgfsetroundjoin%
\definecolor{currentfill}{rgb}{0.000000,0.000000,0.000000}%
\pgfsetfillcolor{currentfill}%
\pgfsetlinewidth{0.501875pt}%
\definecolor{currentstroke}{rgb}{0.000000,0.000000,0.000000}%
\pgfsetstrokecolor{currentstroke}%
\pgfsetdash{}{0pt}%
\pgfsys@defobject{currentmarker}{\pgfqpoint{-0.027778in}{0.000000in}}{\pgfqpoint{0.000000in}{0.000000in}}{%
\pgfpathmoveto{\pgfqpoint{0.000000in}{0.000000in}}%
\pgfpathlineto{\pgfqpoint{-0.027778in}{0.000000in}}%
\pgfusepath{stroke,fill}%
}%
\begin{pgfscope}%
\pgfsys@transformshift{3.506457in}{2.646526in}%
\pgfsys@useobject{currentmarker}{}%
\end{pgfscope}%
\end{pgfscope}%
\begin{pgfscope}%
\pgfsetbuttcap%
\pgfsetroundjoin%
\definecolor{currentfill}{rgb}{0.000000,0.000000,0.000000}%
\pgfsetfillcolor{currentfill}%
\pgfsetlinewidth{0.501875pt}%
\definecolor{currentstroke}{rgb}{0.000000,0.000000,0.000000}%
\pgfsetstrokecolor{currentstroke}%
\pgfsetdash{}{0pt}%
\pgfsys@defobject{currentmarker}{\pgfqpoint{0.000000in}{0.000000in}}{\pgfqpoint{0.027778in}{0.000000in}}{%
\pgfpathmoveto{\pgfqpoint{0.000000in}{0.000000in}}%
\pgfpathlineto{\pgfqpoint{0.027778in}{0.000000in}}%
\pgfusepath{stroke,fill}%
}%
\begin{pgfscope}%
\pgfsys@transformshift{0.485787in}{2.709440in}%
\pgfsys@useobject{currentmarker}{}%
\end{pgfscope}%
\end{pgfscope}%
\begin{pgfscope}%
\pgfsetbuttcap%
\pgfsetroundjoin%
\definecolor{currentfill}{rgb}{0.000000,0.000000,0.000000}%
\pgfsetfillcolor{currentfill}%
\pgfsetlinewidth{0.501875pt}%
\definecolor{currentstroke}{rgb}{0.000000,0.000000,0.000000}%
\pgfsetstrokecolor{currentstroke}%
\pgfsetdash{}{0pt}%
\pgfsys@defobject{currentmarker}{\pgfqpoint{-0.027778in}{0.000000in}}{\pgfqpoint{0.000000in}{0.000000in}}{%
\pgfpathmoveto{\pgfqpoint{0.000000in}{0.000000in}}%
\pgfpathlineto{\pgfqpoint{-0.027778in}{0.000000in}}%
\pgfusepath{stroke,fill}%
}%
\begin{pgfscope}%
\pgfsys@transformshift{3.506457in}{2.709440in}%
\pgfsys@useobject{currentmarker}{}%
\end{pgfscope}%
\end{pgfscope}%
\begin{pgfscope}%
\pgftext[x=0.170972in,y=1.474219in,,bottom,rotate=90.000000]{\rmfamily\fontsize{9.000000}{10.800000}\selectfont Candidates}%
\end{pgfscope}%
\begin{pgfscope}%
\pgfsetrectcap%
\pgfsetroundjoin%
\pgfsetlinewidth{1.003750pt}%
\definecolor{currentstroke}{rgb}{1.000000,0.000000,0.000000}%
\pgfsetstrokecolor{currentstroke}%
\pgfsetdash{}{0pt}%
\pgfpathmoveto{\pgfqpoint{0.485787in}{0.540301in}}%
\pgfpathlineto{\pgfqpoint{0.637580in}{0.572648in}}%
\pgfpathlineto{\pgfqpoint{0.789372in}{0.607979in}}%
\pgfpathlineto{\pgfqpoint{0.925985in}{0.642864in}}%
\pgfpathlineto{\pgfqpoint{1.017061in}{0.667994in}}%
\pgfpathlineto{\pgfqpoint{1.108136in}{0.694889in}}%
\pgfpathlineto{\pgfqpoint{1.199212in}{0.723815in}}%
\pgfpathlineto{\pgfqpoint{1.275108in}{0.749670in}}%
\pgfpathlineto{\pgfqpoint{1.366184in}{0.783247in}}%
\pgfpathlineto{\pgfqpoint{1.442080in}{0.813709in}}%
\pgfpathlineto{\pgfqpoint{1.502797in}{0.839915in}}%
\pgfpathlineto{\pgfqpoint{1.563514in}{0.868046in}}%
\pgfpathlineto{\pgfqpoint{1.624231in}{0.898410in}}%
\pgfpathlineto{\pgfqpoint{1.684948in}{0.931396in}}%
\pgfpathlineto{\pgfqpoint{1.730485in}{0.958132in}}%
\pgfpathlineto{\pgfqpoint{1.776023in}{0.987042in}}%
\pgfpathlineto{\pgfqpoint{1.821561in}{1.018586in}}%
\pgfpathlineto{\pgfqpoint{1.851919in}{1.041688in}}%
\pgfpathlineto{\pgfqpoint{1.882278in}{1.067182in}}%
\pgfpathlineto{\pgfqpoint{1.912636in}{1.096211in}}%
\pgfpathlineto{\pgfqpoint{1.927816in}{1.112479in}}%
\pgfpathlineto{\pgfqpoint{1.958174in}{1.150335in}}%
\pgfpathlineto{\pgfqpoint{1.973353in}{1.172839in}}%
\pgfpathlineto{\pgfqpoint{1.988533in}{1.197532in}}%
\pgfpathlineto{\pgfqpoint{2.003712in}{1.226011in}}%
\pgfpathlineto{\pgfqpoint{2.018891in}{1.257045in}}%
\pgfpathlineto{\pgfqpoint{2.049250in}{1.330847in}}%
\pgfpathlineto{\pgfqpoint{2.079608in}{1.418740in}}%
\pgfpathlineto{\pgfqpoint{2.109967in}{1.518046in}}%
\pgfpathlineto{\pgfqpoint{2.246580in}{1.997202in}}%
\pgfpathlineto{\pgfqpoint{2.292117in}{2.151639in}}%
\pgfpathlineto{\pgfqpoint{2.352834in}{2.382656in}}%
\pgfpathlineto{\pgfqpoint{2.368014in}{2.435121in}}%
\pgfpathlineto{\pgfqpoint{2.383193in}{2.482972in}}%
\pgfpathlineto{\pgfqpoint{2.398372in}{2.526376in}}%
\pgfpathlineto{\pgfqpoint{2.413551in}{2.564648in}}%
\pgfpathlineto{\pgfqpoint{2.428731in}{2.597478in}}%
\pgfpathlineto{\pgfqpoint{2.443910in}{2.624885in}}%
\pgfpathlineto{\pgfqpoint{2.459089in}{2.646002in}}%
\pgfpathlineto{\pgfqpoint{2.474268in}{2.660830in}}%
\pgfpathlineto{\pgfqpoint{2.489448in}{2.669232in}}%
\pgfpathlineto{\pgfqpoint{2.504627in}{2.671132in}}%
\pgfpathlineto{\pgfqpoint{2.519806in}{2.666513in}}%
\pgfpathlineto{\pgfqpoint{2.534985in}{2.655417in}}%
\pgfpathlineto{\pgfqpoint{2.550165in}{2.637947in}}%
\pgfpathlineto{\pgfqpoint{2.565344in}{2.614267in}}%
\pgfpathlineto{\pgfqpoint{2.580523in}{2.584436in}}%
\pgfpathlineto{\pgfqpoint{2.595702in}{2.549240in}}%
\pgfpathlineto{\pgfqpoint{2.610882in}{2.508764in}}%
\pgfpathlineto{\pgfqpoint{2.626061in}{2.463462in}}%
\pgfpathlineto{\pgfqpoint{2.641240in}{2.413865in}}%
\pgfpathlineto{\pgfqpoint{2.671599in}{2.303549in}}%
\pgfpathlineto{\pgfqpoint{2.717136in}{2.122940in}}%
\pgfpathlineto{\pgfqpoint{2.777853in}{1.868262in}}%
\pgfpathlineto{\pgfqpoint{2.808212in}{1.741383in}}%
\pgfpathlineto{\pgfqpoint{2.823391in}{1.689148in}}%
\pgfpathlineto{\pgfqpoint{2.838570in}{1.632374in}}%
\pgfpathlineto{\pgfqpoint{2.868929in}{1.535674in}}%
\pgfpathlineto{\pgfqpoint{2.914467in}{1.412174in}}%
\pgfpathlineto{\pgfqpoint{2.929646in}{1.372461in}}%
\pgfpathlineto{\pgfqpoint{2.990363in}{1.235259in}}%
\pgfpathlineto{\pgfqpoint{3.051080in}{1.115919in}}%
\pgfpathlineto{\pgfqpoint{3.111797in}{1.010195in}}%
\pgfpathlineto{\pgfqpoint{3.172514in}{0.915283in}}%
\pgfpathlineto{\pgfqpoint{3.248410in}{0.809585in}}%
\pgfpathlineto{\pgfqpoint{3.278768in}{0.770158in}}%
\pgfpathlineto{\pgfqpoint{3.309127in}{0.732336in}}%
\pgfpathlineto{\pgfqpoint{3.339485in}{0.695993in}}%
\pgfpathlineto{\pgfqpoint{3.369844in}{0.661018in}}%
\pgfpathlineto{\pgfqpoint{3.400202in}{0.627311in}}%
\pgfpathlineto{\pgfqpoint{3.430561in}{0.594783in}}%
\pgfpathlineto{\pgfqpoint{3.460919in}{0.563355in}}%
\pgfpathlineto{\pgfqpoint{3.491278in}{0.532954in}}%
\pgfpathlineto{\pgfqpoint{3.506457in}{0.517624in}}%
\pgfpathlineto{\pgfqpoint{3.506457in}{0.517624in}}%
\pgfusepath{stroke}%
\end{pgfscope}%
\begin{pgfscope}%
\pgfpathrectangle{\pgfqpoint{0.485787in}{0.226975in}}{\pgfqpoint{3.020670in}{2.494489in}} %
\pgfusepath{clip}%
\pgfsetbuttcap%
\pgfsetroundjoin%
\pgfsetlinewidth{0.501875pt}%
\definecolor{currentstroke}{rgb}{0.000000,0.000000,0.000000}%
\pgfsetstrokecolor{currentstroke}%
\pgfsetdash{}{0pt}%
\pgfpathmoveto{\pgfqpoint{0.500891in}{0.463364in}}%
\pgfpathlineto{\pgfqpoint{0.500891in}{0.595964in}}%
\pgfusepath{stroke}%
\end{pgfscope}%
\begin{pgfscope}%
\pgfpathrectangle{\pgfqpoint{0.485787in}{0.226975in}}{\pgfqpoint{3.020670in}{2.494489in}} %
\pgfusepath{clip}%
\pgfsetbuttcap%
\pgfsetroundjoin%
\pgfsetlinewidth{0.501875pt}%
\definecolor{currentstroke}{rgb}{0.000000,0.000000,0.000000}%
\pgfsetstrokecolor{currentstroke}%
\pgfsetdash{}{0pt}%
\pgfpathmoveto{\pgfqpoint{0.531097in}{0.339296in}}%
\pgfpathlineto{\pgfqpoint{0.531097in}{0.502930in}}%
\pgfusepath{stroke}%
\end{pgfscope}%
\begin{pgfscope}%
\pgfpathrectangle{\pgfqpoint{0.485787in}{0.226975in}}{\pgfqpoint{3.020670in}{2.494489in}} %
\pgfusepath{clip}%
\pgfsetbuttcap%
\pgfsetroundjoin%
\pgfsetlinewidth{0.501875pt}%
\definecolor{currentstroke}{rgb}{0.000000,0.000000,0.000000}%
\pgfsetstrokecolor{currentstroke}%
\pgfsetdash{}{0pt}%
\pgfpathmoveto{\pgfqpoint{0.561304in}{0.561344in}}%
\pgfpathlineto{\pgfqpoint{0.561304in}{0.673416in}}%
\pgfusepath{stroke}%
\end{pgfscope}%
\begin{pgfscope}%
\pgfpathrectangle{\pgfqpoint{0.485787in}{0.226975in}}{\pgfqpoint{3.020670in}{2.494489in}} %
\pgfusepath{clip}%
\pgfsetbuttcap%
\pgfsetroundjoin%
\pgfsetlinewidth{0.501875pt}%
\definecolor{currentstroke}{rgb}{0.000000,0.000000,0.000000}%
\pgfsetstrokecolor{currentstroke}%
\pgfsetdash{}{0pt}%
\pgfpathmoveto{\pgfqpoint{0.591511in}{0.477530in}}%
\pgfpathlineto{\pgfqpoint{0.591511in}{0.606958in}}%
\pgfusepath{stroke}%
\end{pgfscope}%
\begin{pgfscope}%
\pgfpathrectangle{\pgfqpoint{0.485787in}{0.226975in}}{\pgfqpoint{3.020670in}{2.494489in}} %
\pgfusepath{clip}%
\pgfsetbuttcap%
\pgfsetroundjoin%
\pgfsetlinewidth{0.501875pt}%
\definecolor{currentstroke}{rgb}{0.000000,0.000000,0.000000}%
\pgfsetstrokecolor{currentstroke}%
\pgfsetdash{}{0pt}%
\pgfpathmoveto{\pgfqpoint{0.621717in}{0.448490in}}%
\pgfpathlineto{\pgfqpoint{0.621717in}{0.584500in}}%
\pgfusepath{stroke}%
\end{pgfscope}%
\begin{pgfscope}%
\pgfpathrectangle{\pgfqpoint{0.485787in}{0.226975in}}{\pgfqpoint{3.020670in}{2.494489in}} %
\pgfusepath{clip}%
\pgfsetbuttcap%
\pgfsetroundjoin%
\pgfsetlinewidth{0.501875pt}%
\definecolor{currentstroke}{rgb}{0.000000,0.000000,0.000000}%
\pgfsetstrokecolor{currentstroke}%
\pgfsetdash{}{0pt}%
\pgfpathmoveto{\pgfqpoint{0.651924in}{0.477530in}}%
\pgfpathlineto{\pgfqpoint{0.651924in}{0.606958in}}%
\pgfusepath{stroke}%
\end{pgfscope}%
\begin{pgfscope}%
\pgfpathrectangle{\pgfqpoint{0.485787in}{0.226975in}}{\pgfqpoint{3.020670in}{2.494489in}} %
\pgfusepath{clip}%
\pgfsetbuttcap%
\pgfsetroundjoin%
\pgfsetlinewidth{0.501875pt}%
\definecolor{currentstroke}{rgb}{0.000000,0.000000,0.000000}%
\pgfsetstrokecolor{currentstroke}%
\pgfsetdash{}{0pt}%
\pgfpathmoveto{\pgfqpoint{0.682131in}{0.516380in}}%
\pgfpathlineto{\pgfqpoint{0.682131in}{0.637470in}}%
\pgfusepath{stroke}%
\end{pgfscope}%
\begin{pgfscope}%
\pgfpathrectangle{\pgfqpoint{0.485787in}{0.226975in}}{\pgfqpoint{3.020670in}{2.494489in}} %
\pgfusepath{clip}%
\pgfsetbuttcap%
\pgfsetroundjoin%
\pgfsetlinewidth{0.501875pt}%
\definecolor{currentstroke}{rgb}{0.000000,0.000000,0.000000}%
\pgfsetstrokecolor{currentstroke}%
\pgfsetdash{}{0pt}%
\pgfpathmoveto{\pgfqpoint{0.712338in}{0.528277in}}%
\pgfpathlineto{\pgfqpoint{0.712338in}{0.646916in}}%
\pgfusepath{stroke}%
\end{pgfscope}%
\begin{pgfscope}%
\pgfpathrectangle{\pgfqpoint{0.485787in}{0.226975in}}{\pgfqpoint{3.020670in}{2.494489in}} %
\pgfusepath{clip}%
\pgfsetbuttcap%
\pgfsetroundjoin%
\pgfsetlinewidth{0.501875pt}%
\definecolor{currentstroke}{rgb}{0.000000,0.000000,0.000000}%
\pgfsetstrokecolor{currentstroke}%
\pgfsetdash{}{0pt}%
\pgfpathmoveto{\pgfqpoint{0.742544in}{0.550727in}}%
\pgfpathlineto{\pgfqpoint{0.742544in}{0.664869in}}%
\pgfusepath{stroke}%
\end{pgfscope}%
\begin{pgfscope}%
\pgfpathrectangle{\pgfqpoint{0.485787in}{0.226975in}}{\pgfqpoint{3.020670in}{2.494489in}} %
\pgfusepath{clip}%
\pgfsetbuttcap%
\pgfsetroundjoin%
\pgfsetlinewidth{0.501875pt}%
\definecolor{currentstroke}{rgb}{0.000000,0.000000,0.000000}%
\pgfsetstrokecolor{currentstroke}%
\pgfsetdash{}{0pt}%
\pgfpathmoveto{\pgfqpoint{0.772751in}{0.550727in}}%
\pgfpathlineto{\pgfqpoint{0.772751in}{0.664869in}}%
\pgfusepath{stroke}%
\end{pgfscope}%
\begin{pgfscope}%
\pgfpathrectangle{\pgfqpoint{0.485787in}{0.226975in}}{\pgfqpoint{3.020670in}{2.494489in}} %
\pgfusepath{clip}%
\pgfsetbuttcap%
\pgfsetroundjoin%
\pgfsetlinewidth{0.501875pt}%
\definecolor{currentstroke}{rgb}{0.000000,0.000000,0.000000}%
\pgfsetstrokecolor{currentstroke}%
\pgfsetdash{}{0pt}%
\pgfpathmoveto{\pgfqpoint{0.802958in}{0.550727in}}%
\pgfpathlineto{\pgfqpoint{0.802958in}{0.664869in}}%
\pgfusepath{stroke}%
\end{pgfscope}%
\begin{pgfscope}%
\pgfpathrectangle{\pgfqpoint{0.485787in}{0.226975in}}{\pgfqpoint{3.020670in}{2.494489in}} %
\pgfusepath{clip}%
\pgfsetbuttcap%
\pgfsetroundjoin%
\pgfsetlinewidth{0.501875pt}%
\definecolor{currentstroke}{rgb}{0.000000,0.000000,0.000000}%
\pgfsetstrokecolor{currentstroke}%
\pgfsetdash{}{0pt}%
\pgfpathmoveto{\pgfqpoint{0.833164in}{0.550727in}}%
\pgfpathlineto{\pgfqpoint{0.833164in}{0.664869in}}%
\pgfusepath{stroke}%
\end{pgfscope}%
\begin{pgfscope}%
\pgfpathrectangle{\pgfqpoint{0.485787in}{0.226975in}}{\pgfqpoint{3.020670in}{2.494489in}} %
\pgfusepath{clip}%
\pgfsetbuttcap%
\pgfsetroundjoin%
\pgfsetlinewidth{0.501875pt}%
\definecolor{currentstroke}{rgb}{0.000000,0.000000,0.000000}%
\pgfsetstrokecolor{currentstroke}%
\pgfsetdash{}{0pt}%
\pgfpathmoveto{\pgfqpoint{0.863371in}{0.463364in}}%
\pgfpathlineto{\pgfqpoint{0.863371in}{0.595964in}}%
\pgfusepath{stroke}%
\end{pgfscope}%
\begin{pgfscope}%
\pgfpathrectangle{\pgfqpoint{0.485787in}{0.226975in}}{\pgfqpoint{3.020670in}{2.494489in}} %
\pgfusepath{clip}%
\pgfsetbuttcap%
\pgfsetroundjoin%
\pgfsetlinewidth{0.501875pt}%
\definecolor{currentstroke}{rgb}{0.000000,0.000000,0.000000}%
\pgfsetstrokecolor{currentstroke}%
\pgfsetdash{}{0pt}%
\pgfpathmoveto{\pgfqpoint{0.893578in}{0.516380in}}%
\pgfpathlineto{\pgfqpoint{0.893578in}{0.637470in}}%
\pgfusepath{stroke}%
\end{pgfscope}%
\begin{pgfscope}%
\pgfpathrectangle{\pgfqpoint{0.485787in}{0.226975in}}{\pgfqpoint{3.020670in}{2.494489in}} %
\pgfusepath{clip}%
\pgfsetbuttcap%
\pgfsetroundjoin%
\pgfsetlinewidth{0.501875pt}%
\definecolor{currentstroke}{rgb}{0.000000,0.000000,0.000000}%
\pgfsetstrokecolor{currentstroke}%
\pgfsetdash{}{0pt}%
\pgfpathmoveto{\pgfqpoint{0.923784in}{0.503986in}}%
\pgfpathlineto{\pgfqpoint{0.923784in}{0.627679in}}%
\pgfusepath{stroke}%
\end{pgfscope}%
\begin{pgfscope}%
\pgfpathrectangle{\pgfqpoint{0.485787in}{0.226975in}}{\pgfqpoint{3.020670in}{2.494489in}} %
\pgfusepath{clip}%
\pgfsetbuttcap%
\pgfsetroundjoin%
\pgfsetlinewidth{0.501875pt}%
\definecolor{currentstroke}{rgb}{0.000000,0.000000,0.000000}%
\pgfsetstrokecolor{currentstroke}%
\pgfsetdash{}{0pt}%
\pgfpathmoveto{\pgfqpoint{0.953991in}{0.571593in}}%
\pgfpathlineto{\pgfqpoint{0.953991in}{0.681701in}}%
\pgfusepath{stroke}%
\end{pgfscope}%
\begin{pgfscope}%
\pgfpathrectangle{\pgfqpoint{0.485787in}{0.226975in}}{\pgfqpoint{3.020670in}{2.494489in}} %
\pgfusepath{clip}%
\pgfsetbuttcap%
\pgfsetroundjoin%
\pgfsetlinewidth{0.501875pt}%
\definecolor{currentstroke}{rgb}{0.000000,0.000000,0.000000}%
\pgfsetstrokecolor{currentstroke}%
\pgfsetdash{}{0pt}%
\pgfpathmoveto{\pgfqpoint{0.984198in}{0.634815in}}%
\pgfpathlineto{\pgfqpoint{0.984198in}{0.733513in}}%
\pgfusepath{stroke}%
\end{pgfscope}%
\begin{pgfscope}%
\pgfpathrectangle{\pgfqpoint{0.485787in}{0.226975in}}{\pgfqpoint{3.020670in}{2.494489in}} %
\pgfusepath{clip}%
\pgfsetbuttcap%
\pgfsetroundjoin%
\pgfsetlinewidth{0.501875pt}%
\definecolor{currentstroke}{rgb}{0.000000,0.000000,0.000000}%
\pgfsetstrokecolor{currentstroke}%
\pgfsetdash{}{0pt}%
\pgfpathmoveto{\pgfqpoint{1.014405in}{0.626572in}}%
\pgfpathlineto{\pgfqpoint{1.014405in}{0.726691in}}%
\pgfusepath{stroke}%
\end{pgfscope}%
\begin{pgfscope}%
\pgfpathrectangle{\pgfqpoint{0.485787in}{0.226975in}}{\pgfqpoint{3.020670in}{2.494489in}} %
\pgfusepath{clip}%
\pgfsetbuttcap%
\pgfsetroundjoin%
\pgfsetlinewidth{0.501875pt}%
\definecolor{currentstroke}{rgb}{0.000000,0.000000,0.000000}%
\pgfsetstrokecolor{currentstroke}%
\pgfsetdash{}{0pt}%
\pgfpathmoveto{\pgfqpoint{1.044611in}{0.700127in}}%
\pgfpathlineto{\pgfqpoint{1.044611in}{0.788234in}}%
\pgfusepath{stroke}%
\end{pgfscope}%
\begin{pgfscope}%
\pgfpathrectangle{\pgfqpoint{0.485787in}{0.226975in}}{\pgfqpoint{3.020670in}{2.494489in}} %
\pgfusepath{clip}%
\pgfsetbuttcap%
\pgfsetroundjoin%
\pgfsetlinewidth{0.501875pt}%
\definecolor{currentstroke}{rgb}{0.000000,0.000000,0.000000}%
\pgfsetstrokecolor{currentstroke}%
\pgfsetdash{}{0pt}%
\pgfpathmoveto{\pgfqpoint{1.074818in}{0.591081in}}%
\pgfpathlineto{\pgfqpoint{1.074818in}{0.697544in}}%
\pgfusepath{stroke}%
\end{pgfscope}%
\begin{pgfscope}%
\pgfpathrectangle{\pgfqpoint{0.485787in}{0.226975in}}{\pgfqpoint{3.020670in}{2.494489in}} %
\pgfusepath{clip}%
\pgfsetbuttcap%
\pgfsetroundjoin%
\pgfsetlinewidth{0.501875pt}%
\definecolor{currentstroke}{rgb}{0.000000,0.000000,0.000000}%
\pgfsetstrokecolor{currentstroke}%
\pgfsetdash{}{0pt}%
\pgfpathmoveto{\pgfqpoint{1.105025in}{0.561344in}}%
\pgfpathlineto{\pgfqpoint{1.105025in}{0.673416in}}%
\pgfusepath{stroke}%
\end{pgfscope}%
\begin{pgfscope}%
\pgfpathrectangle{\pgfqpoint{0.485787in}{0.226975in}}{\pgfqpoint{3.020670in}{2.494489in}} %
\pgfusepath{clip}%
\pgfsetbuttcap%
\pgfsetroundjoin%
\pgfsetlinewidth{0.501875pt}%
\definecolor{currentstroke}{rgb}{0.000000,0.000000,0.000000}%
\pgfsetstrokecolor{currentstroke}%
\pgfsetdash{}{0pt}%
\pgfpathmoveto{\pgfqpoint{1.135231in}{0.642832in}}%
\pgfpathlineto{\pgfqpoint{1.135231in}{0.740167in}}%
\pgfusepath{stroke}%
\end{pgfscope}%
\begin{pgfscope}%
\pgfpathrectangle{\pgfqpoint{0.485787in}{0.226975in}}{\pgfqpoint{3.020670in}{2.494489in}} %
\pgfusepath{clip}%
\pgfsetbuttcap%
\pgfsetroundjoin%
\pgfsetlinewidth{0.501875pt}%
\definecolor{currentstroke}{rgb}{0.000000,0.000000,0.000000}%
\pgfsetstrokecolor{currentstroke}%
\pgfsetdash{}{0pt}%
\pgfpathmoveto{\pgfqpoint{1.165438in}{0.725055in}}%
\pgfpathlineto{\pgfqpoint{1.165438in}{0.809416in}}%
\pgfusepath{stroke}%
\end{pgfscope}%
\begin{pgfscope}%
\pgfpathrectangle{\pgfqpoint{0.485787in}{0.226975in}}{\pgfqpoint{3.020670in}{2.494489in}} %
\pgfusepath{clip}%
\pgfsetbuttcap%
\pgfsetroundjoin%
\pgfsetlinewidth{0.501875pt}%
\definecolor{currentstroke}{rgb}{0.000000,0.000000,0.000000}%
\pgfsetstrokecolor{currentstroke}%
\pgfsetdash{}{0pt}%
\pgfpathmoveto{\pgfqpoint{1.195645in}{0.550727in}}%
\pgfpathlineto{\pgfqpoint{1.195645in}{0.664869in}}%
\pgfusepath{stroke}%
\end{pgfscope}%
\begin{pgfscope}%
\pgfpathrectangle{\pgfqpoint{0.485787in}{0.226975in}}{\pgfqpoint{3.020670in}{2.494489in}} %
\pgfusepath{clip}%
\pgfsetbuttcap%
\pgfsetroundjoin%
\pgfsetlinewidth{0.501875pt}%
\definecolor{currentstroke}{rgb}{0.000000,0.000000,0.000000}%
\pgfsetstrokecolor{currentstroke}%
\pgfsetdash{}{0pt}%
\pgfpathmoveto{\pgfqpoint{1.225851in}{0.719019in}}%
\pgfpathlineto{\pgfqpoint{1.225851in}{0.804273in}}%
\pgfusepath{stroke}%
\end{pgfscope}%
\begin{pgfscope}%
\pgfpathrectangle{\pgfqpoint{0.485787in}{0.226975in}}{\pgfqpoint{3.020670in}{2.494489in}} %
\pgfusepath{clip}%
\pgfsetbuttcap%
\pgfsetroundjoin%
\pgfsetlinewidth{0.501875pt}%
\definecolor{currentstroke}{rgb}{0.000000,0.000000,0.000000}%
\pgfsetstrokecolor{currentstroke}%
\pgfsetdash{}{0pt}%
\pgfpathmoveto{\pgfqpoint{1.256058in}{0.712856in}}%
\pgfpathlineto{\pgfqpoint{1.256058in}{0.799031in}}%
\pgfusepath{stroke}%
\end{pgfscope}%
\begin{pgfscope}%
\pgfpathrectangle{\pgfqpoint{0.485787in}{0.226975in}}{\pgfqpoint{3.020670in}{2.494489in}} %
\pgfusepath{clip}%
\pgfsetbuttcap%
\pgfsetroundjoin%
\pgfsetlinewidth{0.501875pt}%
\definecolor{currentstroke}{rgb}{0.000000,0.000000,0.000000}%
\pgfsetstrokecolor{currentstroke}%
\pgfsetdash{}{0pt}%
\pgfpathmoveto{\pgfqpoint{1.286265in}{0.679930in}}%
\pgfpathlineto{\pgfqpoint{1.286265in}{0.771190in}}%
\pgfusepath{stroke}%
\end{pgfscope}%
\begin{pgfscope}%
\pgfpathrectangle{\pgfqpoint{0.485787in}{0.226975in}}{\pgfqpoint{3.020670in}{2.494489in}} %
\pgfusepath{clip}%
\pgfsetbuttcap%
\pgfsetroundjoin%
\pgfsetlinewidth{0.501875pt}%
\definecolor{currentstroke}{rgb}{0.000000,0.000000,0.000000}%
\pgfsetstrokecolor{currentstroke}%
\pgfsetdash{}{0pt}%
\pgfpathmoveto{\pgfqpoint{1.316472in}{0.686818in}}%
\pgfpathlineto{\pgfqpoint{1.316472in}{0.776990in}}%
\pgfusepath{stroke}%
\end{pgfscope}%
\begin{pgfscope}%
\pgfpathrectangle{\pgfqpoint{0.485787in}{0.226975in}}{\pgfqpoint{3.020670in}{2.494489in}} %
\pgfusepath{clip}%
\pgfsetbuttcap%
\pgfsetroundjoin%
\pgfsetlinewidth{0.501875pt}%
\definecolor{currentstroke}{rgb}{0.000000,0.000000,0.000000}%
\pgfsetstrokecolor{currentstroke}%
\pgfsetdash{}{0pt}%
\pgfpathmoveto{\pgfqpoint{1.346678in}{0.742447in}}%
\pgfpathlineto{\pgfqpoint{1.346678in}{0.824287in}}%
\pgfusepath{stroke}%
\end{pgfscope}%
\begin{pgfscope}%
\pgfpathrectangle{\pgfqpoint{0.485787in}{0.226975in}}{\pgfqpoint{3.020670in}{2.494489in}} %
\pgfusepath{clip}%
\pgfsetbuttcap%
\pgfsetroundjoin%
\pgfsetlinewidth{0.501875pt}%
\definecolor{currentstroke}{rgb}{0.000000,0.000000,0.000000}%
\pgfsetstrokecolor{currentstroke}%
\pgfsetdash{}{0pt}%
\pgfpathmoveto{\pgfqpoint{1.376885in}{0.779403in}}%
\pgfpathlineto{\pgfqpoint{1.376885in}{0.856126in}}%
\pgfusepath{stroke}%
\end{pgfscope}%
\begin{pgfscope}%
\pgfpathrectangle{\pgfqpoint{0.485787in}{0.226975in}}{\pgfqpoint{3.020670in}{2.494489in}} %
\pgfusepath{clip}%
\pgfsetbuttcap%
\pgfsetroundjoin%
\pgfsetlinewidth{0.501875pt}%
\definecolor{currentstroke}{rgb}{0.000000,0.000000,0.000000}%
\pgfsetstrokecolor{currentstroke}%
\pgfsetdash{}{0pt}%
\pgfpathmoveto{\pgfqpoint{1.407092in}{0.712856in}}%
\pgfpathlineto{\pgfqpoint{1.407092in}{0.799031in}}%
\pgfusepath{stroke}%
\end{pgfscope}%
\begin{pgfscope}%
\pgfpathrectangle{\pgfqpoint{0.485787in}{0.226975in}}{\pgfqpoint{3.020670in}{2.494489in}} %
\pgfusepath{clip}%
\pgfsetbuttcap%
\pgfsetroundjoin%
\pgfsetlinewidth{0.501875pt}%
\definecolor{currentstroke}{rgb}{0.000000,0.000000,0.000000}%
\pgfsetstrokecolor{currentstroke}%
\pgfsetdash{}{0pt}%
\pgfpathmoveto{\pgfqpoint{1.437298in}{0.712856in}}%
\pgfpathlineto{\pgfqpoint{1.437298in}{0.799031in}}%
\pgfusepath{stroke}%
\end{pgfscope}%
\begin{pgfscope}%
\pgfpathrectangle{\pgfqpoint{0.485787in}{0.226975in}}{\pgfqpoint{3.020670in}{2.494489in}} %
\pgfusepath{clip}%
\pgfsetbuttcap%
\pgfsetroundjoin%
\pgfsetlinewidth{0.501875pt}%
\definecolor{currentstroke}{rgb}{0.000000,0.000000,0.000000}%
\pgfsetstrokecolor{currentstroke}%
\pgfsetdash{}{0pt}%
\pgfpathmoveto{\pgfqpoint{1.467505in}{0.712856in}}%
\pgfpathlineto{\pgfqpoint{1.467505in}{0.799031in}}%
\pgfusepath{stroke}%
\end{pgfscope}%
\begin{pgfscope}%
\pgfpathrectangle{\pgfqpoint{0.485787in}{0.226975in}}{\pgfqpoint{3.020670in}{2.494489in}} %
\pgfusepath{clip}%
\pgfsetbuttcap%
\pgfsetroundjoin%
\pgfsetlinewidth{0.501875pt}%
\definecolor{currentstroke}{rgb}{0.000000,0.000000,0.000000}%
\pgfsetstrokecolor{currentstroke}%
\pgfsetdash{}{0pt}%
\pgfpathmoveto{\pgfqpoint{1.497712in}{0.798590in}}%
\pgfpathlineto{\pgfqpoint{1.497712in}{0.872781in}}%
\pgfusepath{stroke}%
\end{pgfscope}%
\begin{pgfscope}%
\pgfpathrectangle{\pgfqpoint{0.485787in}{0.226975in}}{\pgfqpoint{3.020670in}{2.494489in}} %
\pgfusepath{clip}%
\pgfsetbuttcap%
\pgfsetroundjoin%
\pgfsetlinewidth{0.501875pt}%
\definecolor{currentstroke}{rgb}{0.000000,0.000000,0.000000}%
\pgfsetstrokecolor{currentstroke}%
\pgfsetdash{}{0pt}%
\pgfpathmoveto{\pgfqpoint{1.527918in}{0.816590in}}%
\pgfpathlineto{\pgfqpoint{1.527918in}{0.888479in}}%
\pgfusepath{stroke}%
\end{pgfscope}%
\begin{pgfscope}%
\pgfpathrectangle{\pgfqpoint{0.485787in}{0.226975in}}{\pgfqpoint{3.020670in}{2.494489in}} %
\pgfusepath{clip}%
\pgfsetbuttcap%
\pgfsetroundjoin%
\pgfsetlinewidth{0.501875pt}%
\definecolor{currentstroke}{rgb}{0.000000,0.000000,0.000000}%
\pgfsetstrokecolor{currentstroke}%
\pgfsetdash{}{0pt}%
\pgfpathmoveto{\pgfqpoint{1.558125in}{0.803196in}}%
\pgfpathlineto{\pgfqpoint{1.558125in}{0.876790in}}%
\pgfusepath{stroke}%
\end{pgfscope}%
\begin{pgfscope}%
\pgfpathrectangle{\pgfqpoint{0.485787in}{0.226975in}}{\pgfqpoint{3.020670in}{2.494489in}} %
\pgfusepath{clip}%
\pgfsetbuttcap%
\pgfsetroundjoin%
\pgfsetlinewidth{0.501875pt}%
\definecolor{currentstroke}{rgb}{0.000000,0.000000,0.000000}%
\pgfsetstrokecolor{currentstroke}%
\pgfsetdash{}{0pt}%
\pgfpathmoveto{\pgfqpoint{1.588332in}{0.825189in}}%
\pgfpathlineto{\pgfqpoint{1.588332in}{0.896003in}}%
\pgfusepath{stroke}%
\end{pgfscope}%
\begin{pgfscope}%
\pgfpathrectangle{\pgfqpoint{0.485787in}{0.226975in}}{\pgfqpoint{3.020670in}{2.494489in}} %
\pgfusepath{clip}%
\pgfsetbuttcap%
\pgfsetroundjoin%
\pgfsetlinewidth{0.501875pt}%
\definecolor{currentstroke}{rgb}{0.000000,0.000000,0.000000}%
\pgfsetstrokecolor{currentstroke}%
\pgfsetdash{}{0pt}%
\pgfpathmoveto{\pgfqpoint{1.618538in}{0.841659in}}%
\pgfpathlineto{\pgfqpoint{1.618538in}{0.910458in}}%
\pgfusepath{stroke}%
\end{pgfscope}%
\begin{pgfscope}%
\pgfpathrectangle{\pgfqpoint{0.485787in}{0.226975in}}{\pgfqpoint{3.020670in}{2.494489in}} %
\pgfusepath{clip}%
\pgfsetbuttcap%
\pgfsetroundjoin%
\pgfsetlinewidth{0.501875pt}%
\definecolor{currentstroke}{rgb}{0.000000,0.000000,0.000000}%
\pgfsetstrokecolor{currentstroke}%
\pgfsetdash{}{0pt}%
\pgfpathmoveto{\pgfqpoint{1.648745in}{0.896231in}}%
\pgfpathlineto{\pgfqpoint{1.648745in}{0.958748in}}%
\pgfusepath{stroke}%
\end{pgfscope}%
\begin{pgfscope}%
\pgfpathrectangle{\pgfqpoint{0.485787in}{0.226975in}}{\pgfqpoint{3.020670in}{2.494489in}} %
\pgfusepath{clip}%
\pgfsetbuttcap%
\pgfsetroundjoin%
\pgfsetlinewidth{0.501875pt}%
\definecolor{currentstroke}{rgb}{0.000000,0.000000,0.000000}%
\pgfsetstrokecolor{currentstroke}%
\pgfsetdash{}{0pt}%
\pgfpathmoveto{\pgfqpoint{1.678952in}{0.879157in}}%
\pgfpathlineto{\pgfqpoint{1.678952in}{0.943576in}}%
\pgfusepath{stroke}%
\end{pgfscope}%
\begin{pgfscope}%
\pgfpathrectangle{\pgfqpoint{0.485787in}{0.226975in}}{\pgfqpoint{3.020670in}{2.494489in}} %
\pgfusepath{clip}%
\pgfsetbuttcap%
\pgfsetroundjoin%
\pgfsetlinewidth{0.501875pt}%
\definecolor{currentstroke}{rgb}{0.000000,0.000000,0.000000}%
\pgfsetstrokecolor{currentstroke}%
\pgfsetdash{}{0pt}%
\pgfpathmoveto{\pgfqpoint{1.709159in}{0.892896in}}%
\pgfpathlineto{\pgfqpoint{1.709159in}{0.955780in}}%
\pgfusepath{stroke}%
\end{pgfscope}%
\begin{pgfscope}%
\pgfpathrectangle{\pgfqpoint{0.485787in}{0.226975in}}{\pgfqpoint{3.020670in}{2.494489in}} %
\pgfusepath{clip}%
\pgfsetbuttcap%
\pgfsetroundjoin%
\pgfsetlinewidth{0.501875pt}%
\definecolor{currentstroke}{rgb}{0.000000,0.000000,0.000000}%
\pgfsetstrokecolor{currentstroke}%
\pgfsetdash{}{0pt}%
\pgfpathmoveto{\pgfqpoint{1.739365in}{0.915474in}}%
\pgfpathlineto{\pgfqpoint{1.739365in}{0.975913in}}%
\pgfusepath{stroke}%
\end{pgfscope}%
\begin{pgfscope}%
\pgfpathrectangle{\pgfqpoint{0.485787in}{0.226975in}}{\pgfqpoint{3.020670in}{2.494489in}} %
\pgfusepath{clip}%
\pgfsetbuttcap%
\pgfsetroundjoin%
\pgfsetlinewidth{0.501875pt}%
\definecolor{currentstroke}{rgb}{0.000000,0.000000,0.000000}%
\pgfsetstrokecolor{currentstroke}%
\pgfsetdash{}{0pt}%
\pgfpathmoveto{\pgfqpoint{1.769572in}{0.982956in}}%
\pgfpathlineto{\pgfqpoint{1.769572in}{1.034413in}}%
\pgfusepath{stroke}%
\end{pgfscope}%
\begin{pgfscope}%
\pgfpathrectangle{\pgfqpoint{0.485787in}{0.226975in}}{\pgfqpoint{3.020670in}{2.494489in}} %
\pgfusepath{clip}%
\pgfsetbuttcap%
\pgfsetroundjoin%
\pgfsetlinewidth{0.501875pt}%
\definecolor{currentstroke}{rgb}{0.000000,0.000000,0.000000}%
\pgfsetstrokecolor{currentstroke}%
\pgfsetdash{}{0pt}%
\pgfpathmoveto{\pgfqpoint{1.799779in}{0.970403in}}%
\pgfpathlineto{\pgfqpoint{1.799779in}{1.022967in}}%
\pgfusepath{stroke}%
\end{pgfscope}%
\begin{pgfscope}%
\pgfpathrectangle{\pgfqpoint{0.485787in}{0.226975in}}{\pgfqpoint{3.020670in}{2.494489in}} %
\pgfusepath{clip}%
\pgfsetbuttcap%
\pgfsetroundjoin%
\pgfsetlinewidth{0.501875pt}%
\definecolor{currentstroke}{rgb}{0.000000,0.000000,0.000000}%
\pgfsetstrokecolor{currentstroke}%
\pgfsetdash{}{0pt}%
\pgfpathmoveto{\pgfqpoint{1.829985in}{1.032492in}}%
\pgfpathlineto{\pgfqpoint{1.829985in}{1.079798in}}%
\pgfusepath{stroke}%
\end{pgfscope}%
\begin{pgfscope}%
\pgfpathrectangle{\pgfqpoint{0.485787in}{0.226975in}}{\pgfqpoint{3.020670in}{2.494489in}} %
\pgfusepath{clip}%
\pgfsetbuttcap%
\pgfsetroundjoin%
\pgfsetlinewidth{0.501875pt}%
\definecolor{currentstroke}{rgb}{0.000000,0.000000,0.000000}%
\pgfsetstrokecolor{currentstroke}%
\pgfsetdash{}{0pt}%
\pgfpathmoveto{\pgfqpoint{1.860192in}{1.006532in}}%
\pgfpathlineto{\pgfqpoint{1.860192in}{1.055972in}}%
\pgfusepath{stroke}%
\end{pgfscope}%
\begin{pgfscope}%
\pgfpathrectangle{\pgfqpoint{0.485787in}{0.226975in}}{\pgfqpoint{3.020670in}{2.494489in}} %
\pgfusepath{clip}%
\pgfsetbuttcap%
\pgfsetroundjoin%
\pgfsetlinewidth{0.501875pt}%
\definecolor{currentstroke}{rgb}{0.000000,0.000000,0.000000}%
\pgfsetstrokecolor{currentstroke}%
\pgfsetdash{}{0pt}%
\pgfpathmoveto{\pgfqpoint{1.890399in}{1.110003in}}%
\pgfpathlineto{\pgfqpoint{1.890399in}{1.151451in}}%
\pgfusepath{stroke}%
\end{pgfscope}%
\begin{pgfscope}%
\pgfpathrectangle{\pgfqpoint{0.485787in}{0.226975in}}{\pgfqpoint{3.020670in}{2.494489in}} %
\pgfusepath{clip}%
\pgfsetbuttcap%
\pgfsetroundjoin%
\pgfsetlinewidth{0.501875pt}%
\definecolor{currentstroke}{rgb}{0.000000,0.000000,0.000000}%
\pgfsetstrokecolor{currentstroke}%
\pgfsetdash{}{0pt}%
\pgfpathmoveto{\pgfqpoint{1.920605in}{1.126877in}}%
\pgfpathlineto{\pgfqpoint{1.920605in}{1.167146in}}%
\pgfusepath{stroke}%
\end{pgfscope}%
\begin{pgfscope}%
\pgfpathrectangle{\pgfqpoint{0.485787in}{0.226975in}}{\pgfqpoint{3.020670in}{2.494489in}} %
\pgfusepath{clip}%
\pgfsetbuttcap%
\pgfsetroundjoin%
\pgfsetlinewidth{0.501875pt}%
\definecolor{currentstroke}{rgb}{0.000000,0.000000,0.000000}%
\pgfsetstrokecolor{currentstroke}%
\pgfsetdash{}{0pt}%
\pgfpathmoveto{\pgfqpoint{1.950812in}{1.188331in}}%
\pgfpathlineto{\pgfqpoint{1.950812in}{1.224572in}}%
\pgfusepath{stroke}%
\end{pgfscope}%
\begin{pgfscope}%
\pgfpathrectangle{\pgfqpoint{0.485787in}{0.226975in}}{\pgfqpoint{3.020670in}{2.494489in}} %
\pgfusepath{clip}%
\pgfsetbuttcap%
\pgfsetroundjoin%
\pgfsetlinewidth{0.501875pt}%
\definecolor{currentstroke}{rgb}{0.000000,0.000000,0.000000}%
\pgfsetstrokecolor{currentstroke}%
\pgfsetdash{}{0pt}%
\pgfpathmoveto{\pgfqpoint{1.981019in}{1.218004in}}%
\pgfpathlineto{\pgfqpoint{1.981019in}{1.252443in}}%
\pgfusepath{stroke}%
\end{pgfscope}%
\begin{pgfscope}%
\pgfpathrectangle{\pgfqpoint{0.485787in}{0.226975in}}{\pgfqpoint{3.020670in}{2.494489in}} %
\pgfusepath{clip}%
\pgfsetbuttcap%
\pgfsetroundjoin%
\pgfsetlinewidth{0.501875pt}%
\definecolor{currentstroke}{rgb}{0.000000,0.000000,0.000000}%
\pgfsetstrokecolor{currentstroke}%
\pgfsetdash{}{0pt}%
\pgfpathmoveto{\pgfqpoint{2.011226in}{1.278379in}}%
\pgfpathlineto{\pgfqpoint{2.011226in}{1.309416in}}%
\pgfusepath{stroke}%
\end{pgfscope}%
\begin{pgfscope}%
\pgfpathrectangle{\pgfqpoint{0.485787in}{0.226975in}}{\pgfqpoint{3.020670in}{2.494489in}} %
\pgfusepath{clip}%
\pgfsetbuttcap%
\pgfsetroundjoin%
\pgfsetlinewidth{0.501875pt}%
\definecolor{currentstroke}{rgb}{0.000000,0.000000,0.000000}%
\pgfsetstrokecolor{currentstroke}%
\pgfsetdash{}{0pt}%
\pgfpathmoveto{\pgfqpoint{2.041432in}{1.352908in}}%
\pgfpathlineto{\pgfqpoint{2.041432in}{1.380194in}}%
\pgfusepath{stroke}%
\end{pgfscope}%
\begin{pgfscope}%
\pgfpathrectangle{\pgfqpoint{0.485787in}{0.226975in}}{\pgfqpoint{3.020670in}{2.494489in}} %
\pgfusepath{clip}%
\pgfsetbuttcap%
\pgfsetroundjoin%
\pgfsetlinewidth{0.501875pt}%
\definecolor{currentstroke}{rgb}{0.000000,0.000000,0.000000}%
\pgfsetstrokecolor{currentstroke}%
\pgfsetdash{}{0pt}%
\pgfpathmoveto{\pgfqpoint{2.071639in}{1.382435in}}%
\pgfpathlineto{\pgfqpoint{2.071639in}{1.408360in}}%
\pgfusepath{stroke}%
\end{pgfscope}%
\begin{pgfscope}%
\pgfpathrectangle{\pgfqpoint{0.485787in}{0.226975in}}{\pgfqpoint{3.020670in}{2.494489in}} %
\pgfusepath{clip}%
\pgfsetbuttcap%
\pgfsetroundjoin%
\pgfsetlinewidth{0.501875pt}%
\definecolor{currentstroke}{rgb}{0.000000,0.000000,0.000000}%
\pgfsetstrokecolor{currentstroke}%
\pgfsetdash{}{0pt}%
\pgfpathmoveto{\pgfqpoint{2.101846in}{1.479220in}}%
\pgfpathlineto{\pgfqpoint{2.101846in}{1.501135in}}%
\pgfusepath{stroke}%
\end{pgfscope}%
\begin{pgfscope}%
\pgfpathrectangle{\pgfqpoint{0.485787in}{0.226975in}}{\pgfqpoint{3.020670in}{2.494489in}} %
\pgfusepath{clip}%
\pgfsetbuttcap%
\pgfsetroundjoin%
\pgfsetlinewidth{0.501875pt}%
\definecolor{currentstroke}{rgb}{0.000000,0.000000,0.000000}%
\pgfsetstrokecolor{currentstroke}%
\pgfsetdash{}{0pt}%
\pgfpathmoveto{\pgfqpoint{2.132052in}{1.580444in}}%
\pgfpathlineto{\pgfqpoint{2.132052in}{1.598816in}}%
\pgfusepath{stroke}%
\end{pgfscope}%
\begin{pgfscope}%
\pgfpathrectangle{\pgfqpoint{0.485787in}{0.226975in}}{\pgfqpoint{3.020670in}{2.494489in}} %
\pgfusepath{clip}%
\pgfsetbuttcap%
\pgfsetroundjoin%
\pgfsetlinewidth{0.501875pt}%
\definecolor{currentstroke}{rgb}{0.000000,0.000000,0.000000}%
\pgfsetstrokecolor{currentstroke}%
\pgfsetdash{}{0pt}%
\pgfpathmoveto{\pgfqpoint{2.162259in}{1.653059in}}%
\pgfpathlineto{\pgfqpoint{2.162259in}{1.669242in}}%
\pgfusepath{stroke}%
\end{pgfscope}%
\begin{pgfscope}%
\pgfpathrectangle{\pgfqpoint{0.485787in}{0.226975in}}{\pgfqpoint{3.020670in}{2.494489in}} %
\pgfusepath{clip}%
\pgfsetbuttcap%
\pgfsetroundjoin%
\pgfsetlinewidth{0.501875pt}%
\definecolor{currentstroke}{rgb}{0.000000,0.000000,0.000000}%
\pgfsetstrokecolor{currentstroke}%
\pgfsetdash{}{0pt}%
\pgfpathmoveto{\pgfqpoint{2.192466in}{1.775824in}}%
\pgfpathlineto{\pgfqpoint{2.192466in}{1.788877in}}%
\pgfusepath{stroke}%
\end{pgfscope}%
\begin{pgfscope}%
\pgfpathrectangle{\pgfqpoint{0.485787in}{0.226975in}}{\pgfqpoint{3.020670in}{2.494489in}} %
\pgfusepath{clip}%
\pgfsetbuttcap%
\pgfsetroundjoin%
\pgfsetlinewidth{0.501875pt}%
\definecolor{currentstroke}{rgb}{0.000000,0.000000,0.000000}%
\pgfsetstrokecolor{currentstroke}%
\pgfsetdash{}{0pt}%
\pgfpathmoveto{\pgfqpoint{2.222672in}{1.881300in}}%
\pgfpathlineto{\pgfqpoint{2.222672in}{1.892148in}}%
\pgfusepath{stroke}%
\end{pgfscope}%
\begin{pgfscope}%
\pgfpathrectangle{\pgfqpoint{0.485787in}{0.226975in}}{\pgfqpoint{3.020670in}{2.494489in}} %
\pgfusepath{clip}%
\pgfsetbuttcap%
\pgfsetroundjoin%
\pgfsetlinewidth{0.501875pt}%
\definecolor{currentstroke}{rgb}{0.000000,0.000000,0.000000}%
\pgfsetstrokecolor{currentstroke}%
\pgfsetdash{}{0pt}%
\pgfpathmoveto{\pgfqpoint{2.252879in}{2.004154in}}%
\pgfpathlineto{\pgfqpoint{2.252879in}{2.012894in}}%
\pgfusepath{stroke}%
\end{pgfscope}%
\begin{pgfscope}%
\pgfpathrectangle{\pgfqpoint{0.485787in}{0.226975in}}{\pgfqpoint{3.020670in}{2.494489in}} %
\pgfusepath{clip}%
\pgfsetbuttcap%
\pgfsetroundjoin%
\pgfsetlinewidth{0.501875pt}%
\definecolor{currentstroke}{rgb}{0.000000,0.000000,0.000000}%
\pgfsetstrokecolor{currentstroke}%
\pgfsetdash{}{0pt}%
\pgfpathmoveto{\pgfqpoint{2.283086in}{2.127691in}}%
\pgfpathlineto{\pgfqpoint{2.283086in}{2.134722in}}%
\pgfusepath{stroke}%
\end{pgfscope}%
\begin{pgfscope}%
\pgfpathrectangle{\pgfqpoint{0.485787in}{0.226975in}}{\pgfqpoint{3.020670in}{2.494489in}} %
\pgfusepath{clip}%
\pgfsetbuttcap%
\pgfsetroundjoin%
\pgfsetlinewidth{0.501875pt}%
\definecolor{currentstroke}{rgb}{0.000000,0.000000,0.000000}%
\pgfsetstrokecolor{currentstroke}%
\pgfsetdash{}{0pt}%
\pgfpathmoveto{\pgfqpoint{2.313293in}{2.239894in}}%
\pgfpathlineto{\pgfqpoint{2.313293in}{2.245663in}}%
\pgfusepath{stroke}%
\end{pgfscope}%
\begin{pgfscope}%
\pgfpathrectangle{\pgfqpoint{0.485787in}{0.226975in}}{\pgfqpoint{3.020670in}{2.494489in}} %
\pgfusepath{clip}%
\pgfsetbuttcap%
\pgfsetroundjoin%
\pgfsetlinewidth{0.501875pt}%
\definecolor{currentstroke}{rgb}{0.000000,0.000000,0.000000}%
\pgfsetstrokecolor{currentstroke}%
\pgfsetdash{}{0pt}%
\pgfpathmoveto{\pgfqpoint{2.343499in}{2.350631in}}%
\pgfpathlineto{\pgfqpoint{2.343499in}{2.355376in}}%
\pgfusepath{stroke}%
\end{pgfscope}%
\begin{pgfscope}%
\pgfpathrectangle{\pgfqpoint{0.485787in}{0.226975in}}{\pgfqpoint{3.020670in}{2.494489in}} %
\pgfusepath{clip}%
\pgfsetbuttcap%
\pgfsetroundjoin%
\pgfsetlinewidth{0.501875pt}%
\definecolor{currentstroke}{rgb}{0.000000,0.000000,0.000000}%
\pgfsetstrokecolor{currentstroke}%
\pgfsetdash{}{0pt}%
\pgfpathmoveto{\pgfqpoint{2.373706in}{2.451346in}}%
\pgfpathlineto{\pgfqpoint{2.373706in}{2.455317in}}%
\pgfusepath{stroke}%
\end{pgfscope}%
\begin{pgfscope}%
\pgfpathrectangle{\pgfqpoint{0.485787in}{0.226975in}}{\pgfqpoint{3.020670in}{2.494489in}} %
\pgfusepath{clip}%
\pgfsetbuttcap%
\pgfsetroundjoin%
\pgfsetlinewidth{0.501875pt}%
\definecolor{currentstroke}{rgb}{0.000000,0.000000,0.000000}%
\pgfsetstrokecolor{currentstroke}%
\pgfsetdash{}{0pt}%
\pgfpathmoveto{\pgfqpoint{2.403913in}{2.537751in}}%
\pgfpathlineto{\pgfqpoint{2.403913in}{2.541160in}}%
\pgfusepath{stroke}%
\end{pgfscope}%
\begin{pgfscope}%
\pgfpathrectangle{\pgfqpoint{0.485787in}{0.226975in}}{\pgfqpoint{3.020670in}{2.494489in}} %
\pgfusepath{clip}%
\pgfsetbuttcap%
\pgfsetroundjoin%
\pgfsetlinewidth{0.501875pt}%
\definecolor{currentstroke}{rgb}{0.000000,0.000000,0.000000}%
\pgfsetstrokecolor{currentstroke}%
\pgfsetdash{}{0pt}%
\pgfpathmoveto{\pgfqpoint{2.434119in}{2.606035in}}%
\pgfpathlineto{\pgfqpoint{2.434119in}{2.609056in}}%
\pgfusepath{stroke}%
\end{pgfscope}%
\begin{pgfscope}%
\pgfpathrectangle{\pgfqpoint{0.485787in}{0.226975in}}{\pgfqpoint{3.020670in}{2.494489in}} %
\pgfusepath{clip}%
\pgfsetbuttcap%
\pgfsetroundjoin%
\pgfsetlinewidth{0.501875pt}%
\definecolor{currentstroke}{rgb}{0.000000,0.000000,0.000000}%
\pgfsetstrokecolor{currentstroke}%
\pgfsetdash{}{0pt}%
\pgfpathmoveto{\pgfqpoint{2.464326in}{2.652213in}}%
\pgfpathlineto{\pgfqpoint{2.464326in}{2.654997in}}%
\pgfusepath{stroke}%
\end{pgfscope}%
\begin{pgfscope}%
\pgfpathrectangle{\pgfqpoint{0.485787in}{0.226975in}}{\pgfqpoint{3.020670in}{2.494489in}} %
\pgfusepath{clip}%
\pgfsetbuttcap%
\pgfsetroundjoin%
\pgfsetlinewidth{0.501875pt}%
\definecolor{currentstroke}{rgb}{0.000000,0.000000,0.000000}%
\pgfsetstrokecolor{currentstroke}%
\pgfsetdash{}{0pt}%
\pgfpathmoveto{\pgfqpoint{2.494533in}{2.668707in}}%
\pgfpathlineto{\pgfqpoint{2.494533in}{2.671411in}}%
\pgfusepath{stroke}%
\end{pgfscope}%
\begin{pgfscope}%
\pgfpathrectangle{\pgfqpoint{0.485787in}{0.226975in}}{\pgfqpoint{3.020670in}{2.494489in}} %
\pgfusepath{clip}%
\pgfsetbuttcap%
\pgfsetroundjoin%
\pgfsetlinewidth{0.501875pt}%
\definecolor{currentstroke}{rgb}{0.000000,0.000000,0.000000}%
\pgfsetstrokecolor{currentstroke}%
\pgfsetdash{}{0pt}%
\pgfpathmoveto{\pgfqpoint{2.524739in}{2.661138in}}%
\pgfpathlineto{\pgfqpoint{2.524739in}{2.663879in}}%
\pgfusepath{stroke}%
\end{pgfscope}%
\begin{pgfscope}%
\pgfpathrectangle{\pgfqpoint{0.485787in}{0.226975in}}{\pgfqpoint{3.020670in}{2.494489in}} %
\pgfusepath{clip}%
\pgfsetbuttcap%
\pgfsetroundjoin%
\pgfsetlinewidth{0.501875pt}%
\definecolor{currentstroke}{rgb}{0.000000,0.000000,0.000000}%
\pgfsetstrokecolor{currentstroke}%
\pgfsetdash{}{0pt}%
\pgfpathmoveto{\pgfqpoint{2.554946in}{2.630797in}}%
\pgfpathlineto{\pgfqpoint{2.554946in}{2.633688in}}%
\pgfusepath{stroke}%
\end{pgfscope}%
\begin{pgfscope}%
\pgfpathrectangle{\pgfqpoint{0.485787in}{0.226975in}}{\pgfqpoint{3.020670in}{2.494489in}} %
\pgfusepath{clip}%
\pgfsetbuttcap%
\pgfsetroundjoin%
\pgfsetlinewidth{0.501875pt}%
\definecolor{currentstroke}{rgb}{0.000000,0.000000,0.000000}%
\pgfsetstrokecolor{currentstroke}%
\pgfsetdash{}{0pt}%
\pgfpathmoveto{\pgfqpoint{2.585153in}{2.570412in}}%
\pgfpathlineto{\pgfqpoint{2.585153in}{2.573629in}}%
\pgfusepath{stroke}%
\end{pgfscope}%
\begin{pgfscope}%
\pgfpathrectangle{\pgfqpoint{0.485787in}{0.226975in}}{\pgfqpoint{3.020670in}{2.494489in}} %
\pgfusepath{clip}%
\pgfsetbuttcap%
\pgfsetroundjoin%
\pgfsetlinewidth{0.501875pt}%
\definecolor{currentstroke}{rgb}{0.000000,0.000000,0.000000}%
\pgfsetstrokecolor{currentstroke}%
\pgfsetdash{}{0pt}%
\pgfpathmoveto{\pgfqpoint{2.615359in}{2.490312in}}%
\pgfpathlineto{\pgfqpoint{2.615359in}{2.494019in}}%
\pgfusepath{stroke}%
\end{pgfscope}%
\begin{pgfscope}%
\pgfpathrectangle{\pgfqpoint{0.485787in}{0.226975in}}{\pgfqpoint{3.020670in}{2.494489in}} %
\pgfusepath{clip}%
\pgfsetbuttcap%
\pgfsetroundjoin%
\pgfsetlinewidth{0.501875pt}%
\definecolor{currentstroke}{rgb}{0.000000,0.000000,0.000000}%
\pgfsetstrokecolor{currentstroke}%
\pgfsetdash{}{0pt}%
\pgfpathmoveto{\pgfqpoint{2.645566in}{2.394409in}}%
\pgfpathlineto{\pgfqpoint{2.645566in}{2.398801in}}%
\pgfusepath{stroke}%
\end{pgfscope}%
\begin{pgfscope}%
\pgfpathrectangle{\pgfqpoint{0.485787in}{0.226975in}}{\pgfqpoint{3.020670in}{2.494489in}} %
\pgfusepath{clip}%
\pgfsetbuttcap%
\pgfsetroundjoin%
\pgfsetlinewidth{0.501875pt}%
\definecolor{currentstroke}{rgb}{0.000000,0.000000,0.000000}%
\pgfsetstrokecolor{currentstroke}%
\pgfsetdash{}{0pt}%
\pgfpathmoveto{\pgfqpoint{2.675773in}{2.281575in}}%
\pgfpathlineto{\pgfqpoint{2.675773in}{2.286936in}}%
\pgfusepath{stroke}%
\end{pgfscope}%
\begin{pgfscope}%
\pgfpathrectangle{\pgfqpoint{0.485787in}{0.226975in}}{\pgfqpoint{3.020670in}{2.494489in}} %
\pgfusepath{clip}%
\pgfsetbuttcap%
\pgfsetroundjoin%
\pgfsetlinewidth{0.501875pt}%
\definecolor{currentstroke}{rgb}{0.000000,0.000000,0.000000}%
\pgfsetstrokecolor{currentstroke}%
\pgfsetdash{}{0pt}%
\pgfpathmoveto{\pgfqpoint{2.705980in}{2.165037in}}%
\pgfpathlineto{\pgfqpoint{2.705980in}{2.171620in}}%
\pgfusepath{stroke}%
\end{pgfscope}%
\begin{pgfscope}%
\pgfpathrectangle{\pgfqpoint{0.485787in}{0.226975in}}{\pgfqpoint{3.020670in}{2.494489in}} %
\pgfusepath{clip}%
\pgfsetbuttcap%
\pgfsetroundjoin%
\pgfsetlinewidth{0.501875pt}%
\definecolor{currentstroke}{rgb}{0.000000,0.000000,0.000000}%
\pgfsetstrokecolor{currentstroke}%
\pgfsetdash{}{0pt}%
\pgfpathmoveto{\pgfqpoint{2.736186in}{2.048660in}}%
\pgfpathlineto{\pgfqpoint{2.736186in}{2.056742in}}%
\pgfusepath{stroke}%
\end{pgfscope}%
\begin{pgfscope}%
\pgfpathrectangle{\pgfqpoint{0.485787in}{0.226975in}}{\pgfqpoint{3.020670in}{2.494489in}} %
\pgfusepath{clip}%
\pgfsetbuttcap%
\pgfsetroundjoin%
\pgfsetlinewidth{0.501875pt}%
\definecolor{currentstroke}{rgb}{0.000000,0.000000,0.000000}%
\pgfsetstrokecolor{currentstroke}%
\pgfsetdash{}{0pt}%
\pgfpathmoveto{\pgfqpoint{2.766393in}{1.922687in}}%
\pgfpathlineto{\pgfqpoint{2.766393in}{1.932774in}}%
\pgfusepath{stroke}%
\end{pgfscope}%
\begin{pgfscope}%
\pgfpathrectangle{\pgfqpoint{0.485787in}{0.226975in}}{\pgfqpoint{3.020670in}{2.494489in}} %
\pgfusepath{clip}%
\pgfsetbuttcap%
\pgfsetroundjoin%
\pgfsetlinewidth{0.501875pt}%
\definecolor{currentstroke}{rgb}{0.000000,0.000000,0.000000}%
\pgfsetstrokecolor{currentstroke}%
\pgfsetdash{}{0pt}%
\pgfpathmoveto{\pgfqpoint{2.796600in}{1.806505in}}%
\pgfpathlineto{\pgfqpoint{2.796600in}{1.818875in}}%
\pgfusepath{stroke}%
\end{pgfscope}%
\begin{pgfscope}%
\pgfpathrectangle{\pgfqpoint{0.485787in}{0.226975in}}{\pgfqpoint{3.020670in}{2.494489in}} %
\pgfusepath{clip}%
\pgfsetbuttcap%
\pgfsetroundjoin%
\pgfsetlinewidth{0.501875pt}%
\definecolor{currentstroke}{rgb}{0.000000,0.000000,0.000000}%
\pgfsetstrokecolor{currentstroke}%
\pgfsetdash{}{0pt}%
\pgfpathmoveto{\pgfqpoint{2.826806in}{1.691141in}}%
\pgfpathlineto{\pgfqpoint{2.826806in}{1.706281in}}%
\pgfusepath{stroke}%
\end{pgfscope}%
\begin{pgfscope}%
\pgfpathrectangle{\pgfqpoint{0.485787in}{0.226975in}}{\pgfqpoint{3.020670in}{2.494489in}} %
\pgfusepath{clip}%
\pgfsetbuttcap%
\pgfsetroundjoin%
\pgfsetlinewidth{0.501875pt}%
\definecolor{currentstroke}{rgb}{0.000000,0.000000,0.000000}%
\pgfsetstrokecolor{currentstroke}%
\pgfsetdash{}{0pt}%
\pgfpathmoveto{\pgfqpoint{2.857013in}{1.592358in}}%
\pgfpathlineto{\pgfqpoint{2.857013in}{1.610351in}}%
\pgfusepath{stroke}%
\end{pgfscope}%
\begin{pgfscope}%
\pgfpathrectangle{\pgfqpoint{0.485787in}{0.226975in}}{\pgfqpoint{3.020670in}{2.494489in}} %
\pgfusepath{clip}%
\pgfsetbuttcap%
\pgfsetroundjoin%
\pgfsetlinewidth{0.501875pt}%
\definecolor{currentstroke}{rgb}{0.000000,0.000000,0.000000}%
\pgfsetstrokecolor{currentstroke}%
\pgfsetdash{}{0pt}%
\pgfpathmoveto{\pgfqpoint{2.887220in}{1.463145in}}%
\pgfpathlineto{\pgfqpoint{2.887220in}{1.485681in}}%
\pgfusepath{stroke}%
\end{pgfscope}%
\begin{pgfscope}%
\pgfpathrectangle{\pgfqpoint{0.485787in}{0.226975in}}{\pgfqpoint{3.020670in}{2.494489in}} %
\pgfusepath{clip}%
\pgfsetbuttcap%
\pgfsetroundjoin%
\pgfsetlinewidth{0.501875pt}%
\definecolor{currentstroke}{rgb}{0.000000,0.000000,0.000000}%
\pgfsetstrokecolor{currentstroke}%
\pgfsetdash{}{0pt}%
\pgfpathmoveto{\pgfqpoint{2.917426in}{1.390253in}}%
\pgfpathlineto{\pgfqpoint{2.917426in}{1.415830in}}%
\pgfusepath{stroke}%
\end{pgfscope}%
\begin{pgfscope}%
\pgfpathrectangle{\pgfqpoint{0.485787in}{0.226975in}}{\pgfqpoint{3.020670in}{2.494489in}} %
\pgfusepath{clip}%
\pgfsetbuttcap%
\pgfsetroundjoin%
\pgfsetlinewidth{0.501875pt}%
\definecolor{currentstroke}{rgb}{0.000000,0.000000,0.000000}%
\pgfsetstrokecolor{currentstroke}%
\pgfsetdash{}{0pt}%
\pgfpathmoveto{\pgfqpoint{2.947633in}{1.323741in}}%
\pgfpathlineto{\pgfqpoint{2.947633in}{1.352439in}}%
\pgfusepath{stroke}%
\end{pgfscope}%
\begin{pgfscope}%
\pgfpathrectangle{\pgfqpoint{0.485787in}{0.226975in}}{\pgfqpoint{3.020670in}{2.494489in}} %
\pgfusepath{clip}%
\pgfsetbuttcap%
\pgfsetroundjoin%
\pgfsetlinewidth{0.501875pt}%
\definecolor{currentstroke}{rgb}{0.000000,0.000000,0.000000}%
\pgfsetstrokecolor{currentstroke}%
\pgfsetdash{}{0pt}%
\pgfpathmoveto{\pgfqpoint{2.977840in}{1.193101in}}%
\pgfpathlineto{\pgfqpoint{2.977840in}{1.229046in}}%
\pgfusepath{stroke}%
\end{pgfscope}%
\begin{pgfscope}%
\pgfpathrectangle{\pgfqpoint{0.485787in}{0.226975in}}{\pgfqpoint{3.020670in}{2.494489in}} %
\pgfusepath{clip}%
\pgfsetbuttcap%
\pgfsetroundjoin%
\pgfsetlinewidth{0.501875pt}%
\definecolor{currentstroke}{rgb}{0.000000,0.000000,0.000000}%
\pgfsetstrokecolor{currentstroke}%
\pgfsetdash{}{0pt}%
\pgfpathmoveto{\pgfqpoint{3.008047in}{1.148397in}}%
\pgfpathlineto{\pgfqpoint{3.008047in}{1.187209in}}%
\pgfusepath{stroke}%
\end{pgfscope}%
\begin{pgfscope}%
\pgfpathrectangle{\pgfqpoint{0.485787in}{0.226975in}}{\pgfqpoint{3.020670in}{2.494489in}} %
\pgfusepath{clip}%
\pgfsetbuttcap%
\pgfsetroundjoin%
\pgfsetlinewidth{0.501875pt}%
\definecolor{currentstroke}{rgb}{0.000000,0.000000,0.000000}%
\pgfsetstrokecolor{currentstroke}%
\pgfsetdash{}{0pt}%
\pgfpathmoveto{\pgfqpoint{3.038253in}{1.078296in}}%
\pgfpathlineto{\pgfqpoint{3.038253in}{1.122051in}}%
\pgfusepath{stroke}%
\end{pgfscope}%
\begin{pgfscope}%
\pgfpathrectangle{\pgfqpoint{0.485787in}{0.226975in}}{\pgfqpoint{3.020670in}{2.494489in}} %
\pgfusepath{clip}%
\pgfsetbuttcap%
\pgfsetroundjoin%
\pgfsetlinewidth{0.501875pt}%
\definecolor{currentstroke}{rgb}{0.000000,0.000000,0.000000}%
\pgfsetstrokecolor{currentstroke}%
\pgfsetdash{}{0pt}%
\pgfpathmoveto{\pgfqpoint{3.068460in}{1.044644in}}%
\pgfpathlineto{\pgfqpoint{3.068460in}{1.090982in}}%
\pgfusepath{stroke}%
\end{pgfscope}%
\begin{pgfscope}%
\pgfpathrectangle{\pgfqpoint{0.485787in}{0.226975in}}{\pgfqpoint{3.020670in}{2.494489in}} %
\pgfusepath{clip}%
\pgfsetbuttcap%
\pgfsetroundjoin%
\pgfsetlinewidth{0.501875pt}%
\definecolor{currentstroke}{rgb}{0.000000,0.000000,0.000000}%
\pgfsetstrokecolor{currentstroke}%
\pgfsetdash{}{0pt}%
\pgfpathmoveto{\pgfqpoint{3.098667in}{0.980488in}}%
\pgfpathlineto{\pgfqpoint{3.098667in}{1.032162in}}%
\pgfusepath{stroke}%
\end{pgfscope}%
\begin{pgfscope}%
\pgfpathrectangle{\pgfqpoint{0.485787in}{0.226975in}}{\pgfqpoint{3.020670in}{2.494489in}} %
\pgfusepath{clip}%
\pgfsetbuttcap%
\pgfsetroundjoin%
\pgfsetlinewidth{0.501875pt}%
\definecolor{currentstroke}{rgb}{0.000000,0.000000,0.000000}%
\pgfsetstrokecolor{currentstroke}%
\pgfsetdash{}{0pt}%
\pgfpathmoveto{\pgfqpoint{3.128873in}{0.940703in}}%
\pgfpathlineto{\pgfqpoint{3.128873in}{0.995974in}}%
\pgfusepath{stroke}%
\end{pgfscope}%
\begin{pgfscope}%
\pgfpathrectangle{\pgfqpoint{0.485787in}{0.226975in}}{\pgfqpoint{3.020670in}{2.494489in}} %
\pgfusepath{clip}%
\pgfsetbuttcap%
\pgfsetroundjoin%
\pgfsetlinewidth{0.501875pt}%
\definecolor{currentstroke}{rgb}{0.000000,0.000000,0.000000}%
\pgfsetstrokecolor{currentstroke}%
\pgfsetdash{}{0pt}%
\pgfpathmoveto{\pgfqpoint{3.159080in}{0.872035in}}%
\pgfpathlineto{\pgfqpoint{3.159080in}{0.937265in}}%
\pgfusepath{stroke}%
\end{pgfscope}%
\begin{pgfscope}%
\pgfpathrectangle{\pgfqpoint{0.485787in}{0.226975in}}{\pgfqpoint{3.020670in}{2.494489in}} %
\pgfusepath{clip}%
\pgfsetbuttcap%
\pgfsetroundjoin%
\pgfsetlinewidth{0.501875pt}%
\definecolor{currentstroke}{rgb}{0.000000,0.000000,0.000000}%
\pgfsetstrokecolor{currentstroke}%
\pgfsetdash{}{0pt}%
\pgfpathmoveto{\pgfqpoint{3.189287in}{0.820921in}}%
\pgfpathlineto{\pgfqpoint{3.189287in}{0.892267in}}%
\pgfusepath{stroke}%
\end{pgfscope}%
\begin{pgfscope}%
\pgfpathrectangle{\pgfqpoint{0.485787in}{0.226975in}}{\pgfqpoint{3.020670in}{2.494489in}} %
\pgfusepath{clip}%
\pgfsetbuttcap%
\pgfsetroundjoin%
\pgfsetlinewidth{0.501875pt}%
\definecolor{currentstroke}{rgb}{0.000000,0.000000,0.000000}%
\pgfsetstrokecolor{currentstroke}%
\pgfsetdash{}{0pt}%
\pgfpathmoveto{\pgfqpoint{3.219493in}{0.774402in}}%
\pgfpathlineto{\pgfqpoint{3.219493in}{0.851798in}}%
\pgfusepath{stroke}%
\end{pgfscope}%
\begin{pgfscope}%
\pgfpathrectangle{\pgfqpoint{0.485787in}{0.226975in}}{\pgfqpoint{3.020670in}{2.494489in}} %
\pgfusepath{clip}%
\pgfsetbuttcap%
\pgfsetroundjoin%
\pgfsetlinewidth{0.501875pt}%
\definecolor{currentstroke}{rgb}{0.000000,0.000000,0.000000}%
\pgfsetstrokecolor{currentstroke}%
\pgfsetdash{}{0pt}%
\pgfpathmoveto{\pgfqpoint{3.249700in}{0.725055in}}%
\pgfpathlineto{\pgfqpoint{3.249700in}{0.809416in}}%
\pgfusepath{stroke}%
\end{pgfscope}%
\begin{pgfscope}%
\pgfpathrectangle{\pgfqpoint{0.485787in}{0.226975in}}{\pgfqpoint{3.020670in}{2.494489in}} %
\pgfusepath{clip}%
\pgfsetbuttcap%
\pgfsetroundjoin%
\pgfsetlinewidth{0.501875pt}%
\definecolor{currentstroke}{rgb}{0.000000,0.000000,0.000000}%
\pgfsetstrokecolor{currentstroke}%
\pgfsetdash{}{0pt}%
\pgfpathmoveto{\pgfqpoint{3.279907in}{0.807729in}}%
\pgfpathlineto{\pgfqpoint{3.279907in}{0.880742in}}%
\pgfusepath{stroke}%
\end{pgfscope}%
\begin{pgfscope}%
\pgfpathrectangle{\pgfqpoint{0.485787in}{0.226975in}}{\pgfqpoint{3.020670in}{2.494489in}} %
\pgfusepath{clip}%
\pgfsetbuttcap%
\pgfsetroundjoin%
\pgfsetlinewidth{0.501875pt}%
\definecolor{currentstroke}{rgb}{0.000000,0.000000,0.000000}%
\pgfsetstrokecolor{currentstroke}%
\pgfsetdash{}{0pt}%
\pgfpathmoveto{\pgfqpoint{3.310114in}{0.725055in}}%
\pgfpathlineto{\pgfqpoint{3.310114in}{0.809416in}}%
\pgfusepath{stroke}%
\end{pgfscope}%
\begin{pgfscope}%
\pgfpathrectangle{\pgfqpoint{0.485787in}{0.226975in}}{\pgfqpoint{3.020670in}{2.494489in}} %
\pgfusepath{clip}%
\pgfsetbuttcap%
\pgfsetroundjoin%
\pgfsetlinewidth{0.501875pt}%
\definecolor{currentstroke}{rgb}{0.000000,0.000000,0.000000}%
\pgfsetstrokecolor{currentstroke}%
\pgfsetdash{}{0pt}%
\pgfpathmoveto{\pgfqpoint{3.340320in}{0.719019in}}%
\pgfpathlineto{\pgfqpoint{3.340320in}{0.804273in}}%
\pgfusepath{stroke}%
\end{pgfscope}%
\begin{pgfscope}%
\pgfpathrectangle{\pgfqpoint{0.485787in}{0.226975in}}{\pgfqpoint{3.020670in}{2.494489in}} %
\pgfusepath{clip}%
\pgfsetbuttcap%
\pgfsetroundjoin%
\pgfsetlinewidth{0.501875pt}%
\definecolor{currentstroke}{rgb}{0.000000,0.000000,0.000000}%
\pgfsetstrokecolor{currentstroke}%
\pgfsetdash{}{0pt}%
\pgfpathmoveto{\pgfqpoint{3.370527in}{0.634815in}}%
\pgfpathlineto{\pgfqpoint{3.370527in}{0.733513in}}%
\pgfusepath{stroke}%
\end{pgfscope}%
\begin{pgfscope}%
\pgfpathrectangle{\pgfqpoint{0.485787in}{0.226975in}}{\pgfqpoint{3.020670in}{2.494489in}} %
\pgfusepath{clip}%
\pgfsetbuttcap%
\pgfsetroundjoin%
\pgfsetlinewidth{0.501875pt}%
\definecolor{currentstroke}{rgb}{0.000000,0.000000,0.000000}%
\pgfsetstrokecolor{currentstroke}%
\pgfsetdash{}{0pt}%
\pgfpathmoveto{\pgfqpoint{3.400734in}{0.539715in}}%
\pgfpathlineto{\pgfqpoint{3.400734in}{0.656042in}}%
\pgfusepath{stroke}%
\end{pgfscope}%
\begin{pgfscope}%
\pgfpathrectangle{\pgfqpoint{0.485787in}{0.226975in}}{\pgfqpoint{3.020670in}{2.494489in}} %
\pgfusepath{clip}%
\pgfsetbuttcap%
\pgfsetroundjoin%
\pgfsetlinewidth{0.501875pt}%
\definecolor{currentstroke}{rgb}{0.000000,0.000000,0.000000}%
\pgfsetstrokecolor{currentstroke}%
\pgfsetdash{}{0pt}%
\pgfpathmoveto{\pgfqpoint{3.430940in}{0.539715in}}%
\pgfpathlineto{\pgfqpoint{3.430940in}{0.656042in}}%
\pgfusepath{stroke}%
\end{pgfscope}%
\begin{pgfscope}%
\pgfpathrectangle{\pgfqpoint{0.485787in}{0.226975in}}{\pgfqpoint{3.020670in}{2.494489in}} %
\pgfusepath{clip}%
\pgfsetbuttcap%
\pgfsetroundjoin%
\pgfsetlinewidth{0.501875pt}%
\definecolor{currentstroke}{rgb}{0.000000,0.000000,0.000000}%
\pgfsetstrokecolor{currentstroke}%
\pgfsetdash{}{0pt}%
\pgfpathmoveto{\pgfqpoint{3.461147in}{0.561344in}}%
\pgfpathlineto{\pgfqpoint{3.461147in}{0.673416in}}%
\pgfusepath{stroke}%
\end{pgfscope}%
\begin{pgfscope}%
\pgfpathrectangle{\pgfqpoint{0.485787in}{0.226975in}}{\pgfqpoint{3.020670in}{2.494489in}} %
\pgfusepath{clip}%
\pgfsetbuttcap%
\pgfsetroundjoin%
\pgfsetlinewidth{0.501875pt}%
\definecolor{currentstroke}{rgb}{0.000000,0.000000,0.000000}%
\pgfsetstrokecolor{currentstroke}%
\pgfsetdash{}{0pt}%
\pgfpathmoveto{\pgfqpoint{3.491354in}{0.665649in}}%
\pgfpathlineto{\pgfqpoint{3.491354in}{0.759202in}}%
\pgfusepath{stroke}%
\end{pgfscope}%
\begin{pgfscope}%
\pgfsetbuttcap%
\pgfsetroundjoin%
\definecolor{currentfill}{rgb}{0.000000,0.000000,0.000000}%
\pgfsetfillcolor{currentfill}%
\pgfsetlinewidth{1.003750pt}%
\definecolor{currentstroke}{rgb}{0.000000,0.000000,0.000000}%
\pgfsetstrokecolor{currentstroke}%
\pgfsetdash{}{0pt}%
\pgfsys@defobject{currentmarker}{\pgfqpoint{-0.006944in}{-0.006944in}}{\pgfqpoint{0.006944in}{0.006944in}}{%
\pgfpathmoveto{\pgfqpoint{0.000000in}{-0.006944in}}%
\pgfpathcurveto{\pgfqpoint{0.001842in}{-0.006944in}}{\pgfqpoint{0.003608in}{-0.006213in}}{\pgfqpoint{0.004910in}{-0.004910in}}%
\pgfpathcurveto{\pgfqpoint{0.006213in}{-0.003608in}}{\pgfqpoint{0.006944in}{-0.001842in}}{\pgfqpoint{0.006944in}{0.000000in}}%
\pgfpathcurveto{\pgfqpoint{0.006944in}{0.001842in}}{\pgfqpoint{0.006213in}{0.003608in}}{\pgfqpoint{0.004910in}{0.004910in}}%
\pgfpathcurveto{\pgfqpoint{0.003608in}{0.006213in}}{\pgfqpoint{0.001842in}{0.006944in}}{\pgfqpoint{0.000000in}{0.006944in}}%
\pgfpathcurveto{\pgfqpoint{-0.001842in}{0.006944in}}{\pgfqpoint{-0.003608in}{0.006213in}}{\pgfqpoint{-0.004910in}{0.004910in}}%
\pgfpathcurveto{\pgfqpoint{-0.006213in}{0.003608in}}{\pgfqpoint{-0.006944in}{0.001842in}}{\pgfqpoint{-0.006944in}{0.000000in}}%
\pgfpathcurveto{\pgfqpoint{-0.006944in}{-0.001842in}}{\pgfqpoint{-0.006213in}{-0.003608in}}{\pgfqpoint{-0.004910in}{-0.004910in}}%
\pgfpathcurveto{\pgfqpoint{-0.003608in}{-0.006213in}}{\pgfqpoint{-0.001842in}{-0.006944in}}{\pgfqpoint{0.000000in}{-0.006944in}}%
\pgfpathclose%
\pgfusepath{stroke,fill}%
}%
\begin{pgfscope}%
\pgfsys@transformshift{0.500891in}{0.530384in}%
\pgfsys@useobject{currentmarker}{}%
\end{pgfscope}%
\begin{pgfscope}%
\pgfsys@transformshift{0.531097in}{0.422402in}%
\pgfsys@useobject{currentmarker}{}%
\end{pgfscope}%
\begin{pgfscope}%
\pgfsys@transformshift{0.561304in}{0.617830in}%
\pgfsys@useobject{currentmarker}{}%
\end{pgfscope}%
\begin{pgfscope}%
\pgfsys@transformshift{0.591511in}{0.542917in}%
\pgfsys@useobject{currentmarker}{}%
\end{pgfscope}%
\begin{pgfscope}%
\pgfsys@transformshift{0.621717in}{0.517268in}%
\pgfsys@useobject{currentmarker}{}%
\end{pgfscope}%
\begin{pgfscope}%
\pgfsys@transformshift{0.651924in}{0.542917in}%
\pgfsys@useobject{currentmarker}{}%
\end{pgfscope}%
\begin{pgfscope}%
\pgfsys@transformshift{0.682131in}{0.577483in}%
\pgfsys@useobject{currentmarker}{}%
\end{pgfscope}%
\begin{pgfscope}%
\pgfsys@transformshift{0.712338in}{0.588124in}%
\pgfsys@useobject{currentmarker}{}%
\end{pgfscope}%
\begin{pgfscope}%
\pgfsys@transformshift{0.742544in}{0.608271in}%
\pgfsys@useobject{currentmarker}{}%
\end{pgfscope}%
\begin{pgfscope}%
\pgfsys@transformshift{0.772751in}{0.608271in}%
\pgfsys@useobject{currentmarker}{}%
\end{pgfscope}%
\begin{pgfscope}%
\pgfsys@transformshift{0.802958in}{0.608271in}%
\pgfsys@useobject{currentmarker}{}%
\end{pgfscope}%
\begin{pgfscope}%
\pgfsys@transformshift{0.833164in}{0.608271in}%
\pgfsys@useobject{currentmarker}{}%
\end{pgfscope}%
\begin{pgfscope}%
\pgfsys@transformshift{0.863371in}{0.530384in}%
\pgfsys@useobject{currentmarker}{}%
\end{pgfscope}%
\begin{pgfscope}%
\pgfsys@transformshift{0.893578in}{0.577483in}%
\pgfsys@useobject{currentmarker}{}%
\end{pgfscope}%
\begin{pgfscope}%
\pgfsys@transformshift{0.923784in}{0.566425in}%
\pgfsys@useobject{currentmarker}{}%
\end{pgfscope}%
\begin{pgfscope}%
\pgfsys@transformshift{0.953991in}{0.627074in}%
\pgfsys@useobject{currentmarker}{}%
\end{pgfscope}%
\begin{pgfscope}%
\pgfsys@transformshift{0.984198in}{0.684478in}%
\pgfsys@useobject{currentmarker}{}%
\end{pgfscope}%
\begin{pgfscope}%
\pgfsys@transformshift{1.014405in}{0.676959in}%
\pgfsys@useobject{currentmarker}{}%
\end{pgfscope}%
\begin{pgfscope}%
\pgfsys@transformshift{1.044611in}{0.744408in}%
\pgfsys@useobject{currentmarker}{}%
\end{pgfscope}%
\begin{pgfscope}%
\pgfsys@transformshift{1.074818in}{0.644702in}%
\pgfsys@useobject{currentmarker}{}%
\end{pgfscope}%
\begin{pgfscope}%
\pgfsys@transformshift{1.105025in}{0.617830in}%
\pgfsys@useobject{currentmarker}{}%
\end{pgfscope}%
\begin{pgfscope}%
\pgfsys@transformshift{1.135231in}{0.691801in}%
\pgfsys@useobject{currentmarker}{}%
\end{pgfscope}%
\begin{pgfscope}%
\pgfsys@transformshift{1.165438in}{0.767436in}%
\pgfsys@useobject{currentmarker}{}%
\end{pgfscope}%
\begin{pgfscope}%
\pgfsys@transformshift{1.195645in}{0.608271in}%
\pgfsys@useobject{currentmarker}{}%
\end{pgfscope}%
\begin{pgfscope}%
\pgfsys@transformshift{1.225851in}{0.761853in}%
\pgfsys@useobject{currentmarker}{}%
\end{pgfscope}%
\begin{pgfscope}%
\pgfsys@transformshift{1.256058in}{0.756157in}%
\pgfsys@useobject{currentmarker}{}%
\end{pgfscope}%
\begin{pgfscope}%
\pgfsys@transformshift{1.286265in}{0.725811in}%
\pgfsys@useobject{currentmarker}{}%
\end{pgfscope}%
\begin{pgfscope}%
\pgfsys@transformshift{1.316472in}{0.732147in}%
\pgfsys@useobject{currentmarker}{}%
\end{pgfscope}%
\begin{pgfscope}%
\pgfsys@transformshift{1.346678in}{0.783551in}%
\pgfsys@useobject{currentmarker}{}%
\end{pgfscope}%
\begin{pgfscope}%
\pgfsys@transformshift{1.376885in}{0.817917in}%
\pgfsys@useobject{currentmarker}{}%
\end{pgfscope}%
\begin{pgfscope}%
\pgfsys@transformshift{1.407092in}{0.756157in}%
\pgfsys@useobject{currentmarker}{}%
\end{pgfscope}%
\begin{pgfscope}%
\pgfsys@transformshift{1.437298in}{0.756157in}%
\pgfsys@useobject{currentmarker}{}%
\end{pgfscope}%
\begin{pgfscope}%
\pgfsys@transformshift{1.467505in}{0.756157in}%
\pgfsys@useobject{currentmarker}{}%
\end{pgfscope}%
\begin{pgfscope}%
\pgfsys@transformshift{1.497712in}{0.835824in}%
\pgfsys@useobject{currentmarker}{}%
\end{pgfscope}%
\begin{pgfscope}%
\pgfsys@transformshift{1.527918in}{0.852662in}%
\pgfsys@useobject{currentmarker}{}%
\end{pgfscope}%
\begin{pgfscope}%
\pgfsys@transformshift{1.558125in}{0.840129in}%
\pgfsys@useobject{currentmarker}{}%
\end{pgfscope}%
\begin{pgfscope}%
\pgfsys@transformshift{1.588332in}{0.860718in}%
\pgfsys@useobject{currentmarker}{}%
\end{pgfscope}%
\begin{pgfscope}%
\pgfsys@transformshift{1.618538in}{0.876171in}%
\pgfsys@useobject{currentmarker}{}%
\end{pgfscope}%
\begin{pgfscope}%
\pgfsys@transformshift{1.648745in}{0.927575in}%
\pgfsys@useobject{currentmarker}{}%
\end{pgfscope}%
\begin{pgfscope}%
\pgfsys@transformshift{1.678952in}{0.911459in}%
\pgfsys@useobject{currentmarker}{}%
\end{pgfscope}%
\begin{pgfscope}%
\pgfsys@transformshift{1.709159in}{0.924424in}%
\pgfsys@useobject{currentmarker}{}%
\end{pgfscope}%
\begin{pgfscope}%
\pgfsys@transformshift{1.739365in}{0.945771in}%
\pgfsys@useobject{currentmarker}{}%
\end{pgfscope}%
\begin{pgfscope}%
\pgfsys@transformshift{1.769572in}{1.008684in}%
\pgfsys@useobject{currentmarker}{}%
\end{pgfscope}%
\begin{pgfscope}%
\pgfsys@transformshift{1.799779in}{0.996685in}%
\pgfsys@useobject{currentmarker}{}%
\end{pgfscope}%
\begin{pgfscope}%
\pgfsys@transformshift{1.829985in}{1.056145in}%
\pgfsys@useobject{currentmarker}{}%
\end{pgfscope}%
\begin{pgfscope}%
\pgfsys@transformshift{1.860192in}{1.031252in}%
\pgfsys@useobject{currentmarker}{}%
\end{pgfscope}%
\begin{pgfscope}%
\pgfsys@transformshift{1.890399in}{1.130727in}%
\pgfsys@useobject{currentmarker}{}%
\end{pgfscope}%
\begin{pgfscope}%
\pgfsys@transformshift{1.920605in}{1.147012in}%
\pgfsys@useobject{currentmarker}{}%
\end{pgfscope}%
\begin{pgfscope}%
\pgfsys@transformshift{1.950812in}{1.206451in}%
\pgfsys@useobject{currentmarker}{}%
\end{pgfscope}%
\begin{pgfscope}%
\pgfsys@transformshift{1.981019in}{1.235223in}%
\pgfsys@useobject{currentmarker}{}%
\end{pgfscope}%
\begin{pgfscope}%
\pgfsys@transformshift{2.011226in}{1.293897in}%
\pgfsys@useobject{currentmarker}{}%
\end{pgfscope}%
\begin{pgfscope}%
\pgfsys@transformshift{2.041432in}{1.366551in}%
\pgfsys@useobject{currentmarker}{}%
\end{pgfscope}%
\begin{pgfscope}%
\pgfsys@transformshift{2.071639in}{1.395397in}%
\pgfsys@useobject{currentmarker}{}%
\end{pgfscope}%
\begin{pgfscope}%
\pgfsys@transformshift{2.101846in}{1.490178in}%
\pgfsys@useobject{currentmarker}{}%
\end{pgfscope}%
\begin{pgfscope}%
\pgfsys@transformshift{2.132052in}{1.589630in}%
\pgfsys@useobject{currentmarker}{}%
\end{pgfscope}%
\begin{pgfscope}%
\pgfsys@transformshift{2.162259in}{1.661150in}%
\pgfsys@useobject{currentmarker}{}%
\end{pgfscope}%
\begin{pgfscope}%
\pgfsys@transformshift{2.192466in}{1.782351in}%
\pgfsys@useobject{currentmarker}{}%
\end{pgfscope}%
\begin{pgfscope}%
\pgfsys@transformshift{2.222672in}{1.886724in}%
\pgfsys@useobject{currentmarker}{}%
\end{pgfscope}%
\begin{pgfscope}%
\pgfsys@transformshift{2.252879in}{2.008524in}%
\pgfsys@useobject{currentmarker}{}%
\end{pgfscope}%
\begin{pgfscope}%
\pgfsys@transformshift{2.283086in}{2.131207in}%
\pgfsys@useobject{currentmarker}{}%
\end{pgfscope}%
\begin{pgfscope}%
\pgfsys@transformshift{2.313293in}{2.242778in}%
\pgfsys@useobject{currentmarker}{}%
\end{pgfscope}%
\begin{pgfscope}%
\pgfsys@transformshift{2.343499in}{2.353003in}%
\pgfsys@useobject{currentmarker}{}%
\end{pgfscope}%
\begin{pgfscope}%
\pgfsys@transformshift{2.373706in}{2.453331in}%
\pgfsys@useobject{currentmarker}{}%
\end{pgfscope}%
\begin{pgfscope}%
\pgfsys@transformshift{2.403913in}{2.539456in}%
\pgfsys@useobject{currentmarker}{}%
\end{pgfscope}%
\begin{pgfscope}%
\pgfsys@transformshift{2.434119in}{2.607546in}%
\pgfsys@useobject{currentmarker}{}%
\end{pgfscope}%
\begin{pgfscope}%
\pgfsys@transformshift{2.464326in}{2.653605in}%
\pgfsys@useobject{currentmarker}{}%
\end{pgfscope}%
\begin{pgfscope}%
\pgfsys@transformshift{2.494533in}{2.670059in}%
\pgfsys@useobject{currentmarker}{}%
\end{pgfscope}%
\begin{pgfscope}%
\pgfsys@transformshift{2.524739in}{2.662509in}%
\pgfsys@useobject{currentmarker}{}%
\end{pgfscope}%
\begin{pgfscope}%
\pgfsys@transformshift{2.554946in}{2.632242in}%
\pgfsys@useobject{currentmarker}{}%
\end{pgfscope}%
\begin{pgfscope}%
\pgfsys@transformshift{2.585153in}{2.572021in}%
\pgfsys@useobject{currentmarker}{}%
\end{pgfscope}%
\begin{pgfscope}%
\pgfsys@transformshift{2.615359in}{2.492166in}%
\pgfsys@useobject{currentmarker}{}%
\end{pgfscope}%
\begin{pgfscope}%
\pgfsys@transformshift{2.645566in}{2.396605in}%
\pgfsys@useobject{currentmarker}{}%
\end{pgfscope}%
\begin{pgfscope}%
\pgfsys@transformshift{2.675773in}{2.284256in}%
\pgfsys@useobject{currentmarker}{}%
\end{pgfscope}%
\begin{pgfscope}%
\pgfsys@transformshift{2.705980in}{2.168328in}%
\pgfsys@useobject{currentmarker}{}%
\end{pgfscope}%
\begin{pgfscope}%
\pgfsys@transformshift{2.736186in}{2.052701in}%
\pgfsys@useobject{currentmarker}{}%
\end{pgfscope}%
\begin{pgfscope}%
\pgfsys@transformshift{2.766393in}{1.927730in}%
\pgfsys@useobject{currentmarker}{}%
\end{pgfscope}%
\begin{pgfscope}%
\pgfsys@transformshift{2.796600in}{1.812690in}%
\pgfsys@useobject{currentmarker}{}%
\end{pgfscope}%
\begin{pgfscope}%
\pgfsys@transformshift{2.826806in}{1.698711in}%
\pgfsys@useobject{currentmarker}{}%
\end{pgfscope}%
\begin{pgfscope}%
\pgfsys@transformshift{2.857013in}{1.601355in}%
\pgfsys@useobject{currentmarker}{}%
\end{pgfscope}%
\begin{pgfscope}%
\pgfsys@transformshift{2.887220in}{1.474413in}%
\pgfsys@useobject{currentmarker}{}%
\end{pgfscope}%
\begin{pgfscope}%
\pgfsys@transformshift{2.917426in}{1.403041in}%
\pgfsys@useobject{currentmarker}{}%
\end{pgfscope}%
\begin{pgfscope}%
\pgfsys@transformshift{2.947633in}{1.338090in}%
\pgfsys@useobject{currentmarker}{}%
\end{pgfscope}%
\begin{pgfscope}%
\pgfsys@transformshift{2.977840in}{1.211074in}%
\pgfsys@useobject{currentmarker}{}%
\end{pgfscope}%
\begin{pgfscope}%
\pgfsys@transformshift{3.008047in}{1.167803in}%
\pgfsys@useobject{currentmarker}{}%
\end{pgfscope}%
\begin{pgfscope}%
\pgfsys@transformshift{3.038253in}{1.100173in}%
\pgfsys@useobject{currentmarker}{}%
\end{pgfscope}%
\begin{pgfscope}%
\pgfsys@transformshift{3.068460in}{1.067813in}%
\pgfsys@useobject{currentmarker}{}%
\end{pgfscope}%
\begin{pgfscope}%
\pgfsys@transformshift{3.098667in}{1.006325in}%
\pgfsys@useobject{currentmarker}{}%
\end{pgfscope}%
\begin{pgfscope}%
\pgfsys@transformshift{3.128873in}{0.968338in}%
\pgfsys@useobject{currentmarker}{}%
\end{pgfscope}%
\begin{pgfscope}%
\pgfsys@transformshift{3.159080in}{0.904746in}%
\pgfsys@useobject{currentmarker}{}%
\end{pgfscope}%
\begin{pgfscope}%
\pgfsys@transformshift{3.189287in}{0.856719in}%
\pgfsys@useobject{currentmarker}{}%
\end{pgfscope}%
\begin{pgfscope}%
\pgfsys@transformshift{3.219493in}{0.813257in}%
\pgfsys@useobject{currentmarker}{}%
\end{pgfscope}%
\begin{pgfscope}%
\pgfsys@transformshift{3.249700in}{0.767436in}%
\pgfsys@useobject{currentmarker}{}%
\end{pgfscope}%
\begin{pgfscope}%
\pgfsys@transformshift{3.279907in}{0.844369in}%
\pgfsys@useobject{currentmarker}{}%
\end{pgfscope}%
\begin{pgfscope}%
\pgfsys@transformshift{3.310114in}{0.767436in}%
\pgfsys@useobject{currentmarker}{}%
\end{pgfscope}%
\begin{pgfscope}%
\pgfsys@transformshift{3.340320in}{0.761853in}%
\pgfsys@useobject{currentmarker}{}%
\end{pgfscope}%
\begin{pgfscope}%
\pgfsys@transformshift{3.370527in}{0.684478in}%
\pgfsys@useobject{currentmarker}{}%
\end{pgfscope}%
\begin{pgfscope}%
\pgfsys@transformshift{3.400734in}{0.598378in}%
\pgfsys@useobject{currentmarker}{}%
\end{pgfscope}%
\begin{pgfscope}%
\pgfsys@transformshift{3.430940in}{0.598378in}%
\pgfsys@useobject{currentmarker}{}%
\end{pgfscope}%
\begin{pgfscope}%
\pgfsys@transformshift{3.461147in}{0.617830in}%
\pgfsys@useobject{currentmarker}{}%
\end{pgfscope}%
\begin{pgfscope}%
\pgfsys@transformshift{3.491354in}{0.712695in}%
\pgfsys@useobject{currentmarker}{}%
\end{pgfscope}%
\end{pgfscope}%
\end{pgfpicture}%
\makeatother%
\endgroup%

  \caption{
    Fit of signal normalization model to simulated $\PBzero\to\PJpsi\PKstar$ candidates.
    The total signal model (red, solid) is shown together with the left-sided (blue, dashed) and right-sided (green, dashed) Crystal Ball components.
  }
  \label{fig:normmcfit}
\end{figure}

\begin{table}
  \centering
  \caption{
    Parameters of the signal model estimated from simulated decays.
  }
  \begin{tabular}{l S[table-format=4.3,table-figures-uncertainty=1]}
    \toprule
    Parameter & {Estimate} \\
    \midrule
    $\mu$                  & 5280.1 \pm 0.013 \\
    $\sigma_\text{left}$   & 5.77   \pm 0.05 \\
    $n_\text{left}$        & 1.36   \pm 0.05 \\
    $\alpha_\text{left}$   & 2.19   \pm 0.05 \\
    $\sigma_\text{right}$  & 9.41   \pm 0.16 \\
    $n_\text{right}$       & 3.30   \pm 0.25 \\
    $\alpha_\text{right}$  & -1.97  \pm 0.04 \\
    $f_\text{left/right}$  & 0.578  \pm 0.019 \\
    \bottomrule
  \end{tabular}
  \label{tab:normmcfit}
\end{table}

\begin{figure}
  \centering
  %% Creator: Matplotlib, PGF backend
%%
%% To include the figure in your LaTeX document, write
%%   \input{<filename>.pgf}
%%
%% Make sure the required packages are loaded in your preamble
%%   \usepackage{pgf}
%%
%% Figures using additional raster images can only be included by \input if
%% they are in the same directory as the main LaTeX file. For loading figures
%% from other directories you can use the `import` package
%%   \usepackage{import}
%% and then include the figures with
%%   \import{<path to file>}{<filename>.pgf}
%%
%% Matplotlib used the following preamble
%%   \usepackage{fontspec}
%%   \setmainfont{DejaVu Serif}
%%   \setsansfont{DejaVu Sans}
%%   \setmonofont{DejaVu Sans Mono}
%%
\begingroup%
\makeatletter%
\begin{pgfpicture}%
\pgfpathrectangle{\pgfpointorigin}{\pgfqpoint{3.697842in}{2.771463in}}%
\pgfusepath{use as bounding box, clip}%
\begin{pgfscope}%
\pgfsetbuttcap%
\pgfsetmiterjoin%
\definecolor{currentfill}{rgb}{1.000000,1.000000,1.000000}%
\pgfsetfillcolor{currentfill}%
\pgfsetlinewidth{0.000000pt}%
\definecolor{currentstroke}{rgb}{1.000000,1.000000,1.000000}%
\pgfsetstrokecolor{currentstroke}%
\pgfsetdash{}{0pt}%
\pgfpathmoveto{\pgfqpoint{0.000000in}{0.000000in}}%
\pgfpathlineto{\pgfqpoint{3.697842in}{0.000000in}}%
\pgfpathlineto{\pgfqpoint{3.697842in}{2.771463in}}%
\pgfpathlineto{\pgfqpoint{0.000000in}{2.771463in}}%
\pgfpathclose%
\pgfusepath{fill}%
\end{pgfscope}%
\begin{pgfscope}%
\pgfsetbuttcap%
\pgfsetmiterjoin%
\definecolor{currentfill}{rgb}{1.000000,1.000000,1.000000}%
\pgfsetfillcolor{currentfill}%
\pgfsetlinewidth{0.000000pt}%
\definecolor{currentstroke}{rgb}{0.000000,0.000000,0.000000}%
\pgfsetstrokecolor{currentstroke}%
\pgfsetstrokeopacity{0.000000}%
\pgfsetdash{}{0pt}%
\pgfpathmoveto{\pgfqpoint{0.485787in}{0.226975in}}%
\pgfpathlineto{\pgfqpoint{3.506457in}{0.226975in}}%
\pgfpathlineto{\pgfqpoint{3.506457in}{2.721463in}}%
\pgfpathlineto{\pgfqpoint{0.485787in}{2.721463in}}%
\pgfpathclose%
\pgfusepath{fill}%
\end{pgfscope}%
\begin{pgfscope}%
\pgfpathrectangle{\pgfqpoint{0.485787in}{0.226975in}}{\pgfqpoint{3.020670in}{2.494489in}} %
\pgfusepath{clip}%
\pgfsetbuttcap%
\pgfsetroundjoin%
\pgfsetlinewidth{1.003750pt}%
\definecolor{currentstroke}{rgb}{0.000000,0.000000,1.000000}%
\pgfsetstrokecolor{currentstroke}%
\pgfsetdash{{8.000000pt}{3.000000pt}}{0.000000pt}%
\pgfpathmoveto{\pgfqpoint{1.183678in}{0.216975in}}%
\pgfpathlineto{\pgfqpoint{1.184033in}{0.218002in}}%
\pgfpathlineto{\pgfqpoint{1.244750in}{0.409180in}}%
\pgfpathlineto{\pgfqpoint{1.381363in}{0.863941in}}%
\pgfpathlineto{\pgfqpoint{1.472438in}{1.155058in}}%
\pgfpathlineto{\pgfqpoint{1.548334in}{1.381671in}}%
\pgfpathlineto{\pgfqpoint{1.654589in}{1.693387in}}%
\pgfpathlineto{\pgfqpoint{1.684948in}{1.798589in}}%
\pgfpathlineto{\pgfqpoint{1.745665in}{2.020455in}}%
\pgfpathlineto{\pgfqpoint{1.776023in}{2.126330in}}%
\pgfpathlineto{\pgfqpoint{1.806382in}{2.225878in}}%
\pgfpathlineto{\pgfqpoint{1.836740in}{2.317318in}}%
\pgfpathlineto{\pgfqpoint{1.867099in}{2.397820in}}%
\pgfpathlineto{\pgfqpoint{1.882278in}{2.433614in}}%
\pgfpathlineto{\pgfqpoint{1.897457in}{2.466532in}}%
\pgfpathlineto{\pgfqpoint{1.912636in}{2.496505in}}%
\pgfpathlineto{\pgfqpoint{1.927816in}{2.522790in}}%
\pgfpathlineto{\pgfqpoint{1.942995in}{2.545579in}}%
\pgfpathlineto{\pgfqpoint{1.958174in}{2.564784in}}%
\pgfpathlineto{\pgfqpoint{1.973353in}{2.580331in}}%
\pgfpathlineto{\pgfqpoint{1.988533in}{2.592164in}}%
\pgfpathlineto{\pgfqpoint{2.003712in}{2.600240in}}%
\pgfpathlineto{\pgfqpoint{2.018891in}{2.604530in}}%
\pgfpathlineto{\pgfqpoint{2.034070in}{2.605019in}}%
\pgfpathlineto{\pgfqpoint{2.049250in}{2.601705in}}%
\pgfpathlineto{\pgfqpoint{2.064429in}{2.594599in}}%
\pgfpathlineto{\pgfqpoint{2.079608in}{2.583728in}}%
\pgfpathlineto{\pgfqpoint{2.094787in}{2.569130in}}%
\pgfpathlineto{\pgfqpoint{2.109967in}{2.550858in}}%
\pgfpathlineto{\pgfqpoint{2.125146in}{2.528982in}}%
\pgfpathlineto{\pgfqpoint{2.140325in}{2.503585in}}%
\pgfpathlineto{\pgfqpoint{2.155504in}{2.474420in}}%
\pgfpathlineto{\pgfqpoint{2.170684in}{2.442357in}}%
\pgfpathlineto{\pgfqpoint{2.185863in}{2.407328in}}%
\pgfpathlineto{\pgfqpoint{2.216221in}{2.328272in}}%
\pgfpathlineto{\pgfqpoint{2.246580in}{2.238090in}}%
\pgfpathlineto{\pgfqpoint{2.276938in}{2.139509in}}%
\pgfpathlineto{\pgfqpoint{2.322476in}{1.979942in}}%
\pgfpathlineto{\pgfqpoint{2.504627in}{1.321128in}}%
\pgfpathlineto{\pgfqpoint{2.580523in}{1.055631in}}%
\pgfpathlineto{\pgfqpoint{2.641240in}{0.838831in}}%
\pgfpathlineto{\pgfqpoint{2.656419in}{0.785857in}}%
\pgfpathlineto{\pgfqpoint{2.701957in}{0.641725in}}%
\pgfpathlineto{\pgfqpoint{2.732316in}{0.551958in}}%
\pgfpathlineto{\pgfqpoint{2.762674in}{0.466974in}}%
\pgfpathlineto{\pgfqpoint{2.793033in}{0.386174in}}%
\pgfpathlineto{\pgfqpoint{2.823391in}{0.309083in}}%
\pgfpathlineto{\pgfqpoint{2.853750in}{0.235328in}}%
\pgfpathlineto{\pgfqpoint{2.861372in}{0.216975in}}%
\pgfpathlineto{\pgfqpoint{2.861372in}{0.216975in}}%
\pgfusepath{stroke}%
\end{pgfscope}%
\begin{pgfscope}%
\pgfpathrectangle{\pgfqpoint{0.485787in}{0.226975in}}{\pgfqpoint{3.020670in}{2.494489in}} %
\pgfusepath{clip}%
\pgfsetbuttcap%
\pgfsetroundjoin%
\pgfsetlinewidth{1.003750pt}%
\definecolor{currentstroke}{rgb}{0.000000,0.500000,0.000000}%
\pgfsetstrokecolor{currentstroke}%
\pgfsetdash{{8.000000pt}{3.000000pt}}{0.000000pt}%
\pgfpathmoveto{\pgfqpoint{0.500967in}{0.775068in}}%
\pgfpathlineto{\pgfqpoint{3.491278in}{0.598024in}}%
\pgfpathlineto{\pgfqpoint{3.491278in}{0.598024in}}%
\pgfusepath{stroke}%
\end{pgfscope}%
\begin{pgfscope}%
\pgfsetrectcap%
\pgfsetmiterjoin%
\pgfsetlinewidth{1.003750pt}%
\definecolor{currentstroke}{rgb}{0.000000,0.000000,0.000000}%
\pgfsetstrokecolor{currentstroke}%
\pgfsetdash{}{0pt}%
\pgfpathmoveto{\pgfqpoint{0.485787in}{2.721463in}}%
\pgfpathlineto{\pgfqpoint{3.506457in}{2.721463in}}%
\pgfusepath{stroke}%
\end{pgfscope}%
\begin{pgfscope}%
\pgfsetrectcap%
\pgfsetmiterjoin%
\pgfsetlinewidth{1.003750pt}%
\definecolor{currentstroke}{rgb}{0.000000,0.000000,0.000000}%
\pgfsetstrokecolor{currentstroke}%
\pgfsetdash{}{0pt}%
\pgfpathmoveto{\pgfqpoint{3.506457in}{0.226975in}}%
\pgfpathlineto{\pgfqpoint{3.506457in}{2.721463in}}%
\pgfusepath{stroke}%
\end{pgfscope}%
\begin{pgfscope}%
\pgfsetrectcap%
\pgfsetmiterjoin%
\pgfsetlinewidth{1.003750pt}%
\definecolor{currentstroke}{rgb}{0.000000,0.000000,0.000000}%
\pgfsetstrokecolor{currentstroke}%
\pgfsetdash{}{0pt}%
\pgfpathmoveto{\pgfqpoint{0.485787in}{0.226975in}}%
\pgfpathlineto{\pgfqpoint{3.506457in}{0.226975in}}%
\pgfusepath{stroke}%
\end{pgfscope}%
\begin{pgfscope}%
\pgfsetrectcap%
\pgfsetmiterjoin%
\pgfsetlinewidth{1.003750pt}%
\definecolor{currentstroke}{rgb}{0.000000,0.000000,0.000000}%
\pgfsetstrokecolor{currentstroke}%
\pgfsetdash{}{0pt}%
\pgfpathmoveto{\pgfqpoint{0.485787in}{0.226975in}}%
\pgfpathlineto{\pgfqpoint{0.485787in}{2.721463in}}%
\pgfusepath{stroke}%
\end{pgfscope}%
\begin{pgfscope}%
\pgfsetbuttcap%
\pgfsetroundjoin%
\definecolor{currentfill}{rgb}{0.000000,0.000000,0.000000}%
\pgfsetfillcolor{currentfill}%
\pgfsetlinewidth{0.501875pt}%
\definecolor{currentstroke}{rgb}{0.000000,0.000000,0.000000}%
\pgfsetstrokecolor{currentstroke}%
\pgfsetdash{}{0pt}%
\pgfsys@defobject{currentmarker}{\pgfqpoint{0.000000in}{0.000000in}}{\pgfqpoint{0.000000in}{0.069444in}}{%
\pgfpathmoveto{\pgfqpoint{0.000000in}{0.000000in}}%
\pgfpathlineto{\pgfqpoint{0.000000in}{0.069444in}}%
\pgfusepath{stroke,fill}%
}%
\begin{pgfscope}%
\pgfsys@transformshift{0.485787in}{0.226975in}%
\pgfsys@useobject{currentmarker}{}%
\end{pgfscope}%
\end{pgfscope}%
\begin{pgfscope}%
\pgfsetbuttcap%
\pgfsetroundjoin%
\definecolor{currentfill}{rgb}{0.000000,0.000000,0.000000}%
\pgfsetfillcolor{currentfill}%
\pgfsetlinewidth{0.501875pt}%
\definecolor{currentstroke}{rgb}{0.000000,0.000000,0.000000}%
\pgfsetstrokecolor{currentstroke}%
\pgfsetdash{}{0pt}%
\pgfsys@defobject{currentmarker}{\pgfqpoint{0.000000in}{-0.069444in}}{\pgfqpoint{0.000000in}{0.000000in}}{%
\pgfpathmoveto{\pgfqpoint{0.000000in}{0.000000in}}%
\pgfpathlineto{\pgfqpoint{0.000000in}{-0.069444in}}%
\pgfusepath{stroke,fill}%
}%
\begin{pgfscope}%
\pgfsys@transformshift{0.485787in}{2.721463in}%
\pgfsys@useobject{currentmarker}{}%
\end{pgfscope}%
\end{pgfscope}%
\begin{pgfscope}%
\pgftext[x=0.485787in,y=0.157530in,,top]{\rmfamily\fontsize{8.000000}{9.600000}\selectfont 5220}%
\end{pgfscope}%
\begin{pgfscope}%
\pgfsetbuttcap%
\pgfsetroundjoin%
\definecolor{currentfill}{rgb}{0.000000,0.000000,0.000000}%
\pgfsetfillcolor{currentfill}%
\pgfsetlinewidth{0.501875pt}%
\definecolor{currentstroke}{rgb}{0.000000,0.000000,0.000000}%
\pgfsetstrokecolor{currentstroke}%
\pgfsetdash{}{0pt}%
\pgfsys@defobject{currentmarker}{\pgfqpoint{0.000000in}{0.000000in}}{\pgfqpoint{0.000000in}{0.069444in}}{%
\pgfpathmoveto{\pgfqpoint{0.000000in}{0.000000in}}%
\pgfpathlineto{\pgfqpoint{0.000000in}{0.069444in}}%
\pgfusepath{stroke,fill}%
}%
\begin{pgfscope}%
\pgfsys@transformshift{0.989232in}{0.226975in}%
\pgfsys@useobject{currentmarker}{}%
\end{pgfscope}%
\end{pgfscope}%
\begin{pgfscope}%
\pgfsetbuttcap%
\pgfsetroundjoin%
\definecolor{currentfill}{rgb}{0.000000,0.000000,0.000000}%
\pgfsetfillcolor{currentfill}%
\pgfsetlinewidth{0.501875pt}%
\definecolor{currentstroke}{rgb}{0.000000,0.000000,0.000000}%
\pgfsetstrokecolor{currentstroke}%
\pgfsetdash{}{0pt}%
\pgfsys@defobject{currentmarker}{\pgfqpoint{0.000000in}{-0.069444in}}{\pgfqpoint{0.000000in}{0.000000in}}{%
\pgfpathmoveto{\pgfqpoint{0.000000in}{0.000000in}}%
\pgfpathlineto{\pgfqpoint{0.000000in}{-0.069444in}}%
\pgfusepath{stroke,fill}%
}%
\begin{pgfscope}%
\pgfsys@transformshift{0.989232in}{2.721463in}%
\pgfsys@useobject{currentmarker}{}%
\end{pgfscope}%
\end{pgfscope}%
\begin{pgfscope}%
\pgftext[x=0.989232in,y=0.157530in,,top]{\rmfamily\fontsize{8.000000}{9.600000}\selectfont 5240}%
\end{pgfscope}%
\begin{pgfscope}%
\pgfsetbuttcap%
\pgfsetroundjoin%
\definecolor{currentfill}{rgb}{0.000000,0.000000,0.000000}%
\pgfsetfillcolor{currentfill}%
\pgfsetlinewidth{0.501875pt}%
\definecolor{currentstroke}{rgb}{0.000000,0.000000,0.000000}%
\pgfsetstrokecolor{currentstroke}%
\pgfsetdash{}{0pt}%
\pgfsys@defobject{currentmarker}{\pgfqpoint{0.000000in}{0.000000in}}{\pgfqpoint{0.000000in}{0.069444in}}{%
\pgfpathmoveto{\pgfqpoint{0.000000in}{0.000000in}}%
\pgfpathlineto{\pgfqpoint{0.000000in}{0.069444in}}%
\pgfusepath{stroke,fill}%
}%
\begin{pgfscope}%
\pgfsys@transformshift{1.492677in}{0.226975in}%
\pgfsys@useobject{currentmarker}{}%
\end{pgfscope}%
\end{pgfscope}%
\begin{pgfscope}%
\pgfsetbuttcap%
\pgfsetroundjoin%
\definecolor{currentfill}{rgb}{0.000000,0.000000,0.000000}%
\pgfsetfillcolor{currentfill}%
\pgfsetlinewidth{0.501875pt}%
\definecolor{currentstroke}{rgb}{0.000000,0.000000,0.000000}%
\pgfsetstrokecolor{currentstroke}%
\pgfsetdash{}{0pt}%
\pgfsys@defobject{currentmarker}{\pgfqpoint{0.000000in}{-0.069444in}}{\pgfqpoint{0.000000in}{0.000000in}}{%
\pgfpathmoveto{\pgfqpoint{0.000000in}{0.000000in}}%
\pgfpathlineto{\pgfqpoint{0.000000in}{-0.069444in}}%
\pgfusepath{stroke,fill}%
}%
\begin{pgfscope}%
\pgfsys@transformshift{1.492677in}{2.721463in}%
\pgfsys@useobject{currentmarker}{}%
\end{pgfscope}%
\end{pgfscope}%
\begin{pgfscope}%
\pgftext[x=1.492677in,y=0.157530in,,top]{\rmfamily\fontsize{8.000000}{9.600000}\selectfont 5260}%
\end{pgfscope}%
\begin{pgfscope}%
\pgfsetbuttcap%
\pgfsetroundjoin%
\definecolor{currentfill}{rgb}{0.000000,0.000000,0.000000}%
\pgfsetfillcolor{currentfill}%
\pgfsetlinewidth{0.501875pt}%
\definecolor{currentstroke}{rgb}{0.000000,0.000000,0.000000}%
\pgfsetstrokecolor{currentstroke}%
\pgfsetdash{}{0pt}%
\pgfsys@defobject{currentmarker}{\pgfqpoint{0.000000in}{0.000000in}}{\pgfqpoint{0.000000in}{0.069444in}}{%
\pgfpathmoveto{\pgfqpoint{0.000000in}{0.000000in}}%
\pgfpathlineto{\pgfqpoint{0.000000in}{0.069444in}}%
\pgfusepath{stroke,fill}%
}%
\begin{pgfscope}%
\pgfsys@transformshift{1.996122in}{0.226975in}%
\pgfsys@useobject{currentmarker}{}%
\end{pgfscope}%
\end{pgfscope}%
\begin{pgfscope}%
\pgfsetbuttcap%
\pgfsetroundjoin%
\definecolor{currentfill}{rgb}{0.000000,0.000000,0.000000}%
\pgfsetfillcolor{currentfill}%
\pgfsetlinewidth{0.501875pt}%
\definecolor{currentstroke}{rgb}{0.000000,0.000000,0.000000}%
\pgfsetstrokecolor{currentstroke}%
\pgfsetdash{}{0pt}%
\pgfsys@defobject{currentmarker}{\pgfqpoint{0.000000in}{-0.069444in}}{\pgfqpoint{0.000000in}{0.000000in}}{%
\pgfpathmoveto{\pgfqpoint{0.000000in}{0.000000in}}%
\pgfpathlineto{\pgfqpoint{0.000000in}{-0.069444in}}%
\pgfusepath{stroke,fill}%
}%
\begin{pgfscope}%
\pgfsys@transformshift{1.996122in}{2.721463in}%
\pgfsys@useobject{currentmarker}{}%
\end{pgfscope}%
\end{pgfscope}%
\begin{pgfscope}%
\pgftext[x=1.996122in,y=0.157530in,,top]{\rmfamily\fontsize{8.000000}{9.600000}\selectfont 5280}%
\end{pgfscope}%
\begin{pgfscope}%
\pgfsetbuttcap%
\pgfsetroundjoin%
\definecolor{currentfill}{rgb}{0.000000,0.000000,0.000000}%
\pgfsetfillcolor{currentfill}%
\pgfsetlinewidth{0.501875pt}%
\definecolor{currentstroke}{rgb}{0.000000,0.000000,0.000000}%
\pgfsetstrokecolor{currentstroke}%
\pgfsetdash{}{0pt}%
\pgfsys@defobject{currentmarker}{\pgfqpoint{0.000000in}{0.000000in}}{\pgfqpoint{0.000000in}{0.069444in}}{%
\pgfpathmoveto{\pgfqpoint{0.000000in}{0.000000in}}%
\pgfpathlineto{\pgfqpoint{0.000000in}{0.069444in}}%
\pgfusepath{stroke,fill}%
}%
\begin{pgfscope}%
\pgfsys@transformshift{2.499567in}{0.226975in}%
\pgfsys@useobject{currentmarker}{}%
\end{pgfscope}%
\end{pgfscope}%
\begin{pgfscope}%
\pgfsetbuttcap%
\pgfsetroundjoin%
\definecolor{currentfill}{rgb}{0.000000,0.000000,0.000000}%
\pgfsetfillcolor{currentfill}%
\pgfsetlinewidth{0.501875pt}%
\definecolor{currentstroke}{rgb}{0.000000,0.000000,0.000000}%
\pgfsetstrokecolor{currentstroke}%
\pgfsetdash{}{0pt}%
\pgfsys@defobject{currentmarker}{\pgfqpoint{0.000000in}{-0.069444in}}{\pgfqpoint{0.000000in}{0.000000in}}{%
\pgfpathmoveto{\pgfqpoint{0.000000in}{0.000000in}}%
\pgfpathlineto{\pgfqpoint{0.000000in}{-0.069444in}}%
\pgfusepath{stroke,fill}%
}%
\begin{pgfscope}%
\pgfsys@transformshift{2.499567in}{2.721463in}%
\pgfsys@useobject{currentmarker}{}%
\end{pgfscope}%
\end{pgfscope}%
\begin{pgfscope}%
\pgftext[x=2.499567in,y=0.157530in,,top]{\rmfamily\fontsize{8.000000}{9.600000}\selectfont 5300}%
\end{pgfscope}%
\begin{pgfscope}%
\pgfsetbuttcap%
\pgfsetroundjoin%
\definecolor{currentfill}{rgb}{0.000000,0.000000,0.000000}%
\pgfsetfillcolor{currentfill}%
\pgfsetlinewidth{0.501875pt}%
\definecolor{currentstroke}{rgb}{0.000000,0.000000,0.000000}%
\pgfsetstrokecolor{currentstroke}%
\pgfsetdash{}{0pt}%
\pgfsys@defobject{currentmarker}{\pgfqpoint{0.000000in}{0.000000in}}{\pgfqpoint{0.000000in}{0.069444in}}{%
\pgfpathmoveto{\pgfqpoint{0.000000in}{0.000000in}}%
\pgfpathlineto{\pgfqpoint{0.000000in}{0.069444in}}%
\pgfusepath{stroke,fill}%
}%
\begin{pgfscope}%
\pgfsys@transformshift{3.003012in}{0.226975in}%
\pgfsys@useobject{currentmarker}{}%
\end{pgfscope}%
\end{pgfscope}%
\begin{pgfscope}%
\pgfsetbuttcap%
\pgfsetroundjoin%
\definecolor{currentfill}{rgb}{0.000000,0.000000,0.000000}%
\pgfsetfillcolor{currentfill}%
\pgfsetlinewidth{0.501875pt}%
\definecolor{currentstroke}{rgb}{0.000000,0.000000,0.000000}%
\pgfsetstrokecolor{currentstroke}%
\pgfsetdash{}{0pt}%
\pgfsys@defobject{currentmarker}{\pgfqpoint{0.000000in}{-0.069444in}}{\pgfqpoint{0.000000in}{0.000000in}}{%
\pgfpathmoveto{\pgfqpoint{0.000000in}{0.000000in}}%
\pgfpathlineto{\pgfqpoint{0.000000in}{-0.069444in}}%
\pgfusepath{stroke,fill}%
}%
\begin{pgfscope}%
\pgfsys@transformshift{3.003012in}{2.721463in}%
\pgfsys@useobject{currentmarker}{}%
\end{pgfscope}%
\end{pgfscope}%
\begin{pgfscope}%
\pgftext[x=3.003012in,y=0.157530in,,top]{\rmfamily\fontsize{8.000000}{9.600000}\selectfont 5320}%
\end{pgfscope}%
\begin{pgfscope}%
\pgfsetbuttcap%
\pgfsetroundjoin%
\definecolor{currentfill}{rgb}{0.000000,0.000000,0.000000}%
\pgfsetfillcolor{currentfill}%
\pgfsetlinewidth{0.501875pt}%
\definecolor{currentstroke}{rgb}{0.000000,0.000000,0.000000}%
\pgfsetstrokecolor{currentstroke}%
\pgfsetdash{}{0pt}%
\pgfsys@defobject{currentmarker}{\pgfqpoint{0.000000in}{0.000000in}}{\pgfqpoint{0.000000in}{0.069444in}}{%
\pgfpathmoveto{\pgfqpoint{0.000000in}{0.000000in}}%
\pgfpathlineto{\pgfqpoint{0.000000in}{0.069444in}}%
\pgfusepath{stroke,fill}%
}%
\begin{pgfscope}%
\pgfsys@transformshift{3.506457in}{0.226975in}%
\pgfsys@useobject{currentmarker}{}%
\end{pgfscope}%
\end{pgfscope}%
\begin{pgfscope}%
\pgfsetbuttcap%
\pgfsetroundjoin%
\definecolor{currentfill}{rgb}{0.000000,0.000000,0.000000}%
\pgfsetfillcolor{currentfill}%
\pgfsetlinewidth{0.501875pt}%
\definecolor{currentstroke}{rgb}{0.000000,0.000000,0.000000}%
\pgfsetstrokecolor{currentstroke}%
\pgfsetdash{}{0pt}%
\pgfsys@defobject{currentmarker}{\pgfqpoint{0.000000in}{-0.069444in}}{\pgfqpoint{0.000000in}{0.000000in}}{%
\pgfpathmoveto{\pgfqpoint{0.000000in}{0.000000in}}%
\pgfpathlineto{\pgfqpoint{0.000000in}{-0.069444in}}%
\pgfusepath{stroke,fill}%
}%
\begin{pgfscope}%
\pgfsys@transformshift{3.506457in}{2.721463in}%
\pgfsys@useobject{currentmarker}{}%
\end{pgfscope}%
\end{pgfscope}%
\begin{pgfscope}%
\pgftext[x=3.506457in,y=0.157530in,,top]{\rmfamily\fontsize{8.000000}{9.600000}\selectfont 5340}%
\end{pgfscope}%
\begin{pgfscope}%
\pgfsetbuttcap%
\pgfsetroundjoin%
\definecolor{currentfill}{rgb}{0.000000,0.000000,0.000000}%
\pgfsetfillcolor{currentfill}%
\pgfsetlinewidth{0.501875pt}%
\definecolor{currentstroke}{rgb}{0.000000,0.000000,0.000000}%
\pgfsetstrokecolor{currentstroke}%
\pgfsetdash{}{0pt}%
\pgfsys@defobject{currentmarker}{\pgfqpoint{0.000000in}{0.000000in}}{\pgfqpoint{0.069444in}{0.000000in}}{%
\pgfpathmoveto{\pgfqpoint{0.000000in}{0.000000in}}%
\pgfpathlineto{\pgfqpoint{0.069444in}{0.000000in}}%
\pgfusepath{stroke,fill}%
}%
\begin{pgfscope}%
\pgfsys@transformshift{0.485787in}{1.133100in}%
\pgfsys@useobject{currentmarker}{}%
\end{pgfscope}%
\end{pgfscope}%
\begin{pgfscope}%
\pgfsetbuttcap%
\pgfsetroundjoin%
\definecolor{currentfill}{rgb}{0.000000,0.000000,0.000000}%
\pgfsetfillcolor{currentfill}%
\pgfsetlinewidth{0.501875pt}%
\definecolor{currentstroke}{rgb}{0.000000,0.000000,0.000000}%
\pgfsetstrokecolor{currentstroke}%
\pgfsetdash{}{0pt}%
\pgfsys@defobject{currentmarker}{\pgfqpoint{-0.069444in}{0.000000in}}{\pgfqpoint{0.000000in}{0.000000in}}{%
\pgfpathmoveto{\pgfqpoint{0.000000in}{0.000000in}}%
\pgfpathlineto{\pgfqpoint{-0.069444in}{0.000000in}}%
\pgfusepath{stroke,fill}%
}%
\begin{pgfscope}%
\pgfsys@transformshift{3.506457in}{1.133100in}%
\pgfsys@useobject{currentmarker}{}%
\end{pgfscope}%
\end{pgfscope}%
\begin{pgfscope}%
\pgftext[x=0.416343in,y=1.133100in,right,]{\rmfamily\fontsize{8.000000}{9.600000}\selectfont \(\displaystyle {10^{3}}\)}%
\end{pgfscope}%
\begin{pgfscope}%
\pgfsetbuttcap%
\pgfsetroundjoin%
\definecolor{currentfill}{rgb}{0.000000,0.000000,0.000000}%
\pgfsetfillcolor{currentfill}%
\pgfsetlinewidth{0.501875pt}%
\definecolor{currentstroke}{rgb}{0.000000,0.000000,0.000000}%
\pgfsetstrokecolor{currentstroke}%
\pgfsetdash{}{0pt}%
\pgfsys@defobject{currentmarker}{\pgfqpoint{0.000000in}{0.000000in}}{\pgfqpoint{0.069444in}{0.000000in}}{%
\pgfpathmoveto{\pgfqpoint{0.000000in}{0.000000in}}%
\pgfpathlineto{\pgfqpoint{0.069444in}{0.000000in}}%
\pgfusepath{stroke,fill}%
}%
\begin{pgfscope}%
\pgfsys@transformshift{0.485787in}{2.349821in}%
\pgfsys@useobject{currentmarker}{}%
\end{pgfscope}%
\end{pgfscope}%
\begin{pgfscope}%
\pgfsetbuttcap%
\pgfsetroundjoin%
\definecolor{currentfill}{rgb}{0.000000,0.000000,0.000000}%
\pgfsetfillcolor{currentfill}%
\pgfsetlinewidth{0.501875pt}%
\definecolor{currentstroke}{rgb}{0.000000,0.000000,0.000000}%
\pgfsetstrokecolor{currentstroke}%
\pgfsetdash{}{0pt}%
\pgfsys@defobject{currentmarker}{\pgfqpoint{-0.069444in}{0.000000in}}{\pgfqpoint{0.000000in}{0.000000in}}{%
\pgfpathmoveto{\pgfqpoint{0.000000in}{0.000000in}}%
\pgfpathlineto{\pgfqpoint{-0.069444in}{0.000000in}}%
\pgfusepath{stroke,fill}%
}%
\begin{pgfscope}%
\pgfsys@transformshift{3.506457in}{2.349821in}%
\pgfsys@useobject{currentmarker}{}%
\end{pgfscope}%
\end{pgfscope}%
\begin{pgfscope}%
\pgftext[x=0.416343in,y=2.349821in,right,]{\rmfamily\fontsize{8.000000}{9.600000}\selectfont \(\displaystyle {10^{4}}\)}%
\end{pgfscope}%
\begin{pgfscope}%
\pgfsetbuttcap%
\pgfsetroundjoin%
\definecolor{currentfill}{rgb}{0.000000,0.000000,0.000000}%
\pgfsetfillcolor{currentfill}%
\pgfsetlinewidth{0.501875pt}%
\definecolor{currentstroke}{rgb}{0.000000,0.000000,0.000000}%
\pgfsetstrokecolor{currentstroke}%
\pgfsetdash{}{0pt}%
\pgfsys@defobject{currentmarker}{\pgfqpoint{0.000000in}{0.000000in}}{\pgfqpoint{0.027778in}{0.000000in}}{%
\pgfpathmoveto{\pgfqpoint{0.000000in}{0.000000in}}%
\pgfpathlineto{\pgfqpoint{0.027778in}{0.000000in}}%
\pgfusepath{stroke,fill}%
}%
\begin{pgfscope}%
\pgfsys@transformshift{0.485787in}{0.282649in}%
\pgfsys@useobject{currentmarker}{}%
\end{pgfscope}%
\end{pgfscope}%
\begin{pgfscope}%
\pgfsetbuttcap%
\pgfsetroundjoin%
\definecolor{currentfill}{rgb}{0.000000,0.000000,0.000000}%
\pgfsetfillcolor{currentfill}%
\pgfsetlinewidth{0.501875pt}%
\definecolor{currentstroke}{rgb}{0.000000,0.000000,0.000000}%
\pgfsetstrokecolor{currentstroke}%
\pgfsetdash{}{0pt}%
\pgfsys@defobject{currentmarker}{\pgfqpoint{-0.027778in}{0.000000in}}{\pgfqpoint{0.000000in}{0.000000in}}{%
\pgfpathmoveto{\pgfqpoint{0.000000in}{0.000000in}}%
\pgfpathlineto{\pgfqpoint{-0.027778in}{0.000000in}}%
\pgfusepath{stroke,fill}%
}%
\begin{pgfscope}%
\pgfsys@transformshift{3.506457in}{0.282649in}%
\pgfsys@useobject{currentmarker}{}%
\end{pgfscope}%
\end{pgfscope}%
\begin{pgfscope}%
\pgfsetbuttcap%
\pgfsetroundjoin%
\definecolor{currentfill}{rgb}{0.000000,0.000000,0.000000}%
\pgfsetfillcolor{currentfill}%
\pgfsetlinewidth{0.501875pt}%
\definecolor{currentstroke}{rgb}{0.000000,0.000000,0.000000}%
\pgfsetstrokecolor{currentstroke}%
\pgfsetdash{}{0pt}%
\pgfsys@defobject{currentmarker}{\pgfqpoint{0.000000in}{0.000000in}}{\pgfqpoint{0.027778in}{0.000000in}}{%
\pgfpathmoveto{\pgfqpoint{0.000000in}{0.000000in}}%
\pgfpathlineto{\pgfqpoint{0.027778in}{0.000000in}}%
\pgfusepath{stroke,fill}%
}%
\begin{pgfscope}%
\pgfsys@transformshift{0.485787in}{0.496903in}%
\pgfsys@useobject{currentmarker}{}%
\end{pgfscope}%
\end{pgfscope}%
\begin{pgfscope}%
\pgfsetbuttcap%
\pgfsetroundjoin%
\definecolor{currentfill}{rgb}{0.000000,0.000000,0.000000}%
\pgfsetfillcolor{currentfill}%
\pgfsetlinewidth{0.501875pt}%
\definecolor{currentstroke}{rgb}{0.000000,0.000000,0.000000}%
\pgfsetstrokecolor{currentstroke}%
\pgfsetdash{}{0pt}%
\pgfsys@defobject{currentmarker}{\pgfqpoint{-0.027778in}{0.000000in}}{\pgfqpoint{0.000000in}{0.000000in}}{%
\pgfpathmoveto{\pgfqpoint{0.000000in}{0.000000in}}%
\pgfpathlineto{\pgfqpoint{-0.027778in}{0.000000in}}%
\pgfusepath{stroke,fill}%
}%
\begin{pgfscope}%
\pgfsys@transformshift{3.506457in}{0.496903in}%
\pgfsys@useobject{currentmarker}{}%
\end{pgfscope}%
\end{pgfscope}%
\begin{pgfscope}%
\pgfsetbuttcap%
\pgfsetroundjoin%
\definecolor{currentfill}{rgb}{0.000000,0.000000,0.000000}%
\pgfsetfillcolor{currentfill}%
\pgfsetlinewidth{0.501875pt}%
\definecolor{currentstroke}{rgb}{0.000000,0.000000,0.000000}%
\pgfsetstrokecolor{currentstroke}%
\pgfsetdash{}{0pt}%
\pgfsys@defobject{currentmarker}{\pgfqpoint{0.000000in}{0.000000in}}{\pgfqpoint{0.027778in}{0.000000in}}{%
\pgfpathmoveto{\pgfqpoint{0.000000in}{0.000000in}}%
\pgfpathlineto{\pgfqpoint{0.027778in}{0.000000in}}%
\pgfusepath{stroke,fill}%
}%
\begin{pgfscope}%
\pgfsys@transformshift{0.485787in}{0.648918in}%
\pgfsys@useobject{currentmarker}{}%
\end{pgfscope}%
\end{pgfscope}%
\begin{pgfscope}%
\pgfsetbuttcap%
\pgfsetroundjoin%
\definecolor{currentfill}{rgb}{0.000000,0.000000,0.000000}%
\pgfsetfillcolor{currentfill}%
\pgfsetlinewidth{0.501875pt}%
\definecolor{currentstroke}{rgb}{0.000000,0.000000,0.000000}%
\pgfsetstrokecolor{currentstroke}%
\pgfsetdash{}{0pt}%
\pgfsys@defobject{currentmarker}{\pgfqpoint{-0.027778in}{0.000000in}}{\pgfqpoint{0.000000in}{0.000000in}}{%
\pgfpathmoveto{\pgfqpoint{0.000000in}{0.000000in}}%
\pgfpathlineto{\pgfqpoint{-0.027778in}{0.000000in}}%
\pgfusepath{stroke,fill}%
}%
\begin{pgfscope}%
\pgfsys@transformshift{3.506457in}{0.648918in}%
\pgfsys@useobject{currentmarker}{}%
\end{pgfscope}%
\end{pgfscope}%
\begin{pgfscope}%
\pgfsetbuttcap%
\pgfsetroundjoin%
\definecolor{currentfill}{rgb}{0.000000,0.000000,0.000000}%
\pgfsetfillcolor{currentfill}%
\pgfsetlinewidth{0.501875pt}%
\definecolor{currentstroke}{rgb}{0.000000,0.000000,0.000000}%
\pgfsetstrokecolor{currentstroke}%
\pgfsetdash{}{0pt}%
\pgfsys@defobject{currentmarker}{\pgfqpoint{0.000000in}{0.000000in}}{\pgfqpoint{0.027778in}{0.000000in}}{%
\pgfpathmoveto{\pgfqpoint{0.000000in}{0.000000in}}%
\pgfpathlineto{\pgfqpoint{0.027778in}{0.000000in}}%
\pgfusepath{stroke,fill}%
}%
\begin{pgfscope}%
\pgfsys@transformshift{0.485787in}{0.766831in}%
\pgfsys@useobject{currentmarker}{}%
\end{pgfscope}%
\end{pgfscope}%
\begin{pgfscope}%
\pgfsetbuttcap%
\pgfsetroundjoin%
\definecolor{currentfill}{rgb}{0.000000,0.000000,0.000000}%
\pgfsetfillcolor{currentfill}%
\pgfsetlinewidth{0.501875pt}%
\definecolor{currentstroke}{rgb}{0.000000,0.000000,0.000000}%
\pgfsetstrokecolor{currentstroke}%
\pgfsetdash{}{0pt}%
\pgfsys@defobject{currentmarker}{\pgfqpoint{-0.027778in}{0.000000in}}{\pgfqpoint{0.000000in}{0.000000in}}{%
\pgfpathmoveto{\pgfqpoint{0.000000in}{0.000000in}}%
\pgfpathlineto{\pgfqpoint{-0.027778in}{0.000000in}}%
\pgfusepath{stroke,fill}%
}%
\begin{pgfscope}%
\pgfsys@transformshift{3.506457in}{0.766831in}%
\pgfsys@useobject{currentmarker}{}%
\end{pgfscope}%
\end{pgfscope}%
\begin{pgfscope}%
\pgfsetbuttcap%
\pgfsetroundjoin%
\definecolor{currentfill}{rgb}{0.000000,0.000000,0.000000}%
\pgfsetfillcolor{currentfill}%
\pgfsetlinewidth{0.501875pt}%
\definecolor{currentstroke}{rgb}{0.000000,0.000000,0.000000}%
\pgfsetstrokecolor{currentstroke}%
\pgfsetdash{}{0pt}%
\pgfsys@defobject{currentmarker}{\pgfqpoint{0.000000in}{0.000000in}}{\pgfqpoint{0.027778in}{0.000000in}}{%
\pgfpathmoveto{\pgfqpoint{0.000000in}{0.000000in}}%
\pgfpathlineto{\pgfqpoint{0.027778in}{0.000000in}}%
\pgfusepath{stroke,fill}%
}%
\begin{pgfscope}%
\pgfsys@transformshift{0.485787in}{0.863172in}%
\pgfsys@useobject{currentmarker}{}%
\end{pgfscope}%
\end{pgfscope}%
\begin{pgfscope}%
\pgfsetbuttcap%
\pgfsetroundjoin%
\definecolor{currentfill}{rgb}{0.000000,0.000000,0.000000}%
\pgfsetfillcolor{currentfill}%
\pgfsetlinewidth{0.501875pt}%
\definecolor{currentstroke}{rgb}{0.000000,0.000000,0.000000}%
\pgfsetstrokecolor{currentstroke}%
\pgfsetdash{}{0pt}%
\pgfsys@defobject{currentmarker}{\pgfqpoint{-0.027778in}{0.000000in}}{\pgfqpoint{0.000000in}{0.000000in}}{%
\pgfpathmoveto{\pgfqpoint{0.000000in}{0.000000in}}%
\pgfpathlineto{\pgfqpoint{-0.027778in}{0.000000in}}%
\pgfusepath{stroke,fill}%
}%
\begin{pgfscope}%
\pgfsys@transformshift{3.506457in}{0.863172in}%
\pgfsys@useobject{currentmarker}{}%
\end{pgfscope}%
\end{pgfscope}%
\begin{pgfscope}%
\pgfsetbuttcap%
\pgfsetroundjoin%
\definecolor{currentfill}{rgb}{0.000000,0.000000,0.000000}%
\pgfsetfillcolor{currentfill}%
\pgfsetlinewidth{0.501875pt}%
\definecolor{currentstroke}{rgb}{0.000000,0.000000,0.000000}%
\pgfsetstrokecolor{currentstroke}%
\pgfsetdash{}{0pt}%
\pgfsys@defobject{currentmarker}{\pgfqpoint{0.000000in}{0.000000in}}{\pgfqpoint{0.027778in}{0.000000in}}{%
\pgfpathmoveto{\pgfqpoint{0.000000in}{0.000000in}}%
\pgfpathlineto{\pgfqpoint{0.027778in}{0.000000in}}%
\pgfusepath{stroke,fill}%
}%
\begin{pgfscope}%
\pgfsys@transformshift{0.485787in}{0.944628in}%
\pgfsys@useobject{currentmarker}{}%
\end{pgfscope}%
\end{pgfscope}%
\begin{pgfscope}%
\pgfsetbuttcap%
\pgfsetroundjoin%
\definecolor{currentfill}{rgb}{0.000000,0.000000,0.000000}%
\pgfsetfillcolor{currentfill}%
\pgfsetlinewidth{0.501875pt}%
\definecolor{currentstroke}{rgb}{0.000000,0.000000,0.000000}%
\pgfsetstrokecolor{currentstroke}%
\pgfsetdash{}{0pt}%
\pgfsys@defobject{currentmarker}{\pgfqpoint{-0.027778in}{0.000000in}}{\pgfqpoint{0.000000in}{0.000000in}}{%
\pgfpathmoveto{\pgfqpoint{0.000000in}{0.000000in}}%
\pgfpathlineto{\pgfqpoint{-0.027778in}{0.000000in}}%
\pgfusepath{stroke,fill}%
}%
\begin{pgfscope}%
\pgfsys@transformshift{3.506457in}{0.944628in}%
\pgfsys@useobject{currentmarker}{}%
\end{pgfscope}%
\end{pgfscope}%
\begin{pgfscope}%
\pgfsetbuttcap%
\pgfsetroundjoin%
\definecolor{currentfill}{rgb}{0.000000,0.000000,0.000000}%
\pgfsetfillcolor{currentfill}%
\pgfsetlinewidth{0.501875pt}%
\definecolor{currentstroke}{rgb}{0.000000,0.000000,0.000000}%
\pgfsetstrokecolor{currentstroke}%
\pgfsetdash{}{0pt}%
\pgfsys@defobject{currentmarker}{\pgfqpoint{0.000000in}{0.000000in}}{\pgfqpoint{0.027778in}{0.000000in}}{%
\pgfpathmoveto{\pgfqpoint{0.000000in}{0.000000in}}%
\pgfpathlineto{\pgfqpoint{0.027778in}{0.000000in}}%
\pgfusepath{stroke,fill}%
}%
\begin{pgfscope}%
\pgfsys@transformshift{0.485787in}{1.015188in}%
\pgfsys@useobject{currentmarker}{}%
\end{pgfscope}%
\end{pgfscope}%
\begin{pgfscope}%
\pgfsetbuttcap%
\pgfsetroundjoin%
\definecolor{currentfill}{rgb}{0.000000,0.000000,0.000000}%
\pgfsetfillcolor{currentfill}%
\pgfsetlinewidth{0.501875pt}%
\definecolor{currentstroke}{rgb}{0.000000,0.000000,0.000000}%
\pgfsetstrokecolor{currentstroke}%
\pgfsetdash{}{0pt}%
\pgfsys@defobject{currentmarker}{\pgfqpoint{-0.027778in}{0.000000in}}{\pgfqpoint{0.000000in}{0.000000in}}{%
\pgfpathmoveto{\pgfqpoint{0.000000in}{0.000000in}}%
\pgfpathlineto{\pgfqpoint{-0.027778in}{0.000000in}}%
\pgfusepath{stroke,fill}%
}%
\begin{pgfscope}%
\pgfsys@transformshift{3.506457in}{1.015188in}%
\pgfsys@useobject{currentmarker}{}%
\end{pgfscope}%
\end{pgfscope}%
\begin{pgfscope}%
\pgfsetbuttcap%
\pgfsetroundjoin%
\definecolor{currentfill}{rgb}{0.000000,0.000000,0.000000}%
\pgfsetfillcolor{currentfill}%
\pgfsetlinewidth{0.501875pt}%
\definecolor{currentstroke}{rgb}{0.000000,0.000000,0.000000}%
\pgfsetstrokecolor{currentstroke}%
\pgfsetdash{}{0pt}%
\pgfsys@defobject{currentmarker}{\pgfqpoint{0.000000in}{0.000000in}}{\pgfqpoint{0.027778in}{0.000000in}}{%
\pgfpathmoveto{\pgfqpoint{0.000000in}{0.000000in}}%
\pgfpathlineto{\pgfqpoint{0.027778in}{0.000000in}}%
\pgfusepath{stroke,fill}%
}%
\begin{pgfscope}%
\pgfsys@transformshift{0.485787in}{1.077426in}%
\pgfsys@useobject{currentmarker}{}%
\end{pgfscope}%
\end{pgfscope}%
\begin{pgfscope}%
\pgfsetbuttcap%
\pgfsetroundjoin%
\definecolor{currentfill}{rgb}{0.000000,0.000000,0.000000}%
\pgfsetfillcolor{currentfill}%
\pgfsetlinewidth{0.501875pt}%
\definecolor{currentstroke}{rgb}{0.000000,0.000000,0.000000}%
\pgfsetstrokecolor{currentstroke}%
\pgfsetdash{}{0pt}%
\pgfsys@defobject{currentmarker}{\pgfqpoint{-0.027778in}{0.000000in}}{\pgfqpoint{0.000000in}{0.000000in}}{%
\pgfpathmoveto{\pgfqpoint{0.000000in}{0.000000in}}%
\pgfpathlineto{\pgfqpoint{-0.027778in}{0.000000in}}%
\pgfusepath{stroke,fill}%
}%
\begin{pgfscope}%
\pgfsys@transformshift{3.506457in}{1.077426in}%
\pgfsys@useobject{currentmarker}{}%
\end{pgfscope}%
\end{pgfscope}%
\begin{pgfscope}%
\pgfsetbuttcap%
\pgfsetroundjoin%
\definecolor{currentfill}{rgb}{0.000000,0.000000,0.000000}%
\pgfsetfillcolor{currentfill}%
\pgfsetlinewidth{0.501875pt}%
\definecolor{currentstroke}{rgb}{0.000000,0.000000,0.000000}%
\pgfsetstrokecolor{currentstroke}%
\pgfsetdash{}{0pt}%
\pgfsys@defobject{currentmarker}{\pgfqpoint{0.000000in}{0.000000in}}{\pgfqpoint{0.027778in}{0.000000in}}{%
\pgfpathmoveto{\pgfqpoint{0.000000in}{0.000000in}}%
\pgfpathlineto{\pgfqpoint{0.027778in}{0.000000in}}%
\pgfusepath{stroke,fill}%
}%
\begin{pgfscope}%
\pgfsys@transformshift{0.485787in}{1.499370in}%
\pgfsys@useobject{currentmarker}{}%
\end{pgfscope}%
\end{pgfscope}%
\begin{pgfscope}%
\pgfsetbuttcap%
\pgfsetroundjoin%
\definecolor{currentfill}{rgb}{0.000000,0.000000,0.000000}%
\pgfsetfillcolor{currentfill}%
\pgfsetlinewidth{0.501875pt}%
\definecolor{currentstroke}{rgb}{0.000000,0.000000,0.000000}%
\pgfsetstrokecolor{currentstroke}%
\pgfsetdash{}{0pt}%
\pgfsys@defobject{currentmarker}{\pgfqpoint{-0.027778in}{0.000000in}}{\pgfqpoint{0.000000in}{0.000000in}}{%
\pgfpathmoveto{\pgfqpoint{0.000000in}{0.000000in}}%
\pgfpathlineto{\pgfqpoint{-0.027778in}{0.000000in}}%
\pgfusepath{stroke,fill}%
}%
\begin{pgfscope}%
\pgfsys@transformshift{3.506457in}{1.499370in}%
\pgfsys@useobject{currentmarker}{}%
\end{pgfscope}%
\end{pgfscope}%
\begin{pgfscope}%
\pgfsetbuttcap%
\pgfsetroundjoin%
\definecolor{currentfill}{rgb}{0.000000,0.000000,0.000000}%
\pgfsetfillcolor{currentfill}%
\pgfsetlinewidth{0.501875pt}%
\definecolor{currentstroke}{rgb}{0.000000,0.000000,0.000000}%
\pgfsetstrokecolor{currentstroke}%
\pgfsetdash{}{0pt}%
\pgfsys@defobject{currentmarker}{\pgfqpoint{0.000000in}{0.000000in}}{\pgfqpoint{0.027778in}{0.000000in}}{%
\pgfpathmoveto{\pgfqpoint{0.000000in}{0.000000in}}%
\pgfpathlineto{\pgfqpoint{0.027778in}{0.000000in}}%
\pgfusepath{stroke,fill}%
}%
\begin{pgfscope}%
\pgfsys@transformshift{0.485787in}{1.713624in}%
\pgfsys@useobject{currentmarker}{}%
\end{pgfscope}%
\end{pgfscope}%
\begin{pgfscope}%
\pgfsetbuttcap%
\pgfsetroundjoin%
\definecolor{currentfill}{rgb}{0.000000,0.000000,0.000000}%
\pgfsetfillcolor{currentfill}%
\pgfsetlinewidth{0.501875pt}%
\definecolor{currentstroke}{rgb}{0.000000,0.000000,0.000000}%
\pgfsetstrokecolor{currentstroke}%
\pgfsetdash{}{0pt}%
\pgfsys@defobject{currentmarker}{\pgfqpoint{-0.027778in}{0.000000in}}{\pgfqpoint{0.000000in}{0.000000in}}{%
\pgfpathmoveto{\pgfqpoint{0.000000in}{0.000000in}}%
\pgfpathlineto{\pgfqpoint{-0.027778in}{0.000000in}}%
\pgfusepath{stroke,fill}%
}%
\begin{pgfscope}%
\pgfsys@transformshift{3.506457in}{1.713624in}%
\pgfsys@useobject{currentmarker}{}%
\end{pgfscope}%
\end{pgfscope}%
\begin{pgfscope}%
\pgfsetbuttcap%
\pgfsetroundjoin%
\definecolor{currentfill}{rgb}{0.000000,0.000000,0.000000}%
\pgfsetfillcolor{currentfill}%
\pgfsetlinewidth{0.501875pt}%
\definecolor{currentstroke}{rgb}{0.000000,0.000000,0.000000}%
\pgfsetstrokecolor{currentstroke}%
\pgfsetdash{}{0pt}%
\pgfsys@defobject{currentmarker}{\pgfqpoint{0.000000in}{0.000000in}}{\pgfqpoint{0.027778in}{0.000000in}}{%
\pgfpathmoveto{\pgfqpoint{0.000000in}{0.000000in}}%
\pgfpathlineto{\pgfqpoint{0.027778in}{0.000000in}}%
\pgfusepath{stroke,fill}%
}%
\begin{pgfscope}%
\pgfsys@transformshift{0.485787in}{1.865639in}%
\pgfsys@useobject{currentmarker}{}%
\end{pgfscope}%
\end{pgfscope}%
\begin{pgfscope}%
\pgfsetbuttcap%
\pgfsetroundjoin%
\definecolor{currentfill}{rgb}{0.000000,0.000000,0.000000}%
\pgfsetfillcolor{currentfill}%
\pgfsetlinewidth{0.501875pt}%
\definecolor{currentstroke}{rgb}{0.000000,0.000000,0.000000}%
\pgfsetstrokecolor{currentstroke}%
\pgfsetdash{}{0pt}%
\pgfsys@defobject{currentmarker}{\pgfqpoint{-0.027778in}{0.000000in}}{\pgfqpoint{0.000000in}{0.000000in}}{%
\pgfpathmoveto{\pgfqpoint{0.000000in}{0.000000in}}%
\pgfpathlineto{\pgfqpoint{-0.027778in}{0.000000in}}%
\pgfusepath{stroke,fill}%
}%
\begin{pgfscope}%
\pgfsys@transformshift{3.506457in}{1.865639in}%
\pgfsys@useobject{currentmarker}{}%
\end{pgfscope}%
\end{pgfscope}%
\begin{pgfscope}%
\pgfsetbuttcap%
\pgfsetroundjoin%
\definecolor{currentfill}{rgb}{0.000000,0.000000,0.000000}%
\pgfsetfillcolor{currentfill}%
\pgfsetlinewidth{0.501875pt}%
\definecolor{currentstroke}{rgb}{0.000000,0.000000,0.000000}%
\pgfsetstrokecolor{currentstroke}%
\pgfsetdash{}{0pt}%
\pgfsys@defobject{currentmarker}{\pgfqpoint{0.000000in}{0.000000in}}{\pgfqpoint{0.027778in}{0.000000in}}{%
\pgfpathmoveto{\pgfqpoint{0.000000in}{0.000000in}}%
\pgfpathlineto{\pgfqpoint{0.027778in}{0.000000in}}%
\pgfusepath{stroke,fill}%
}%
\begin{pgfscope}%
\pgfsys@transformshift{0.485787in}{1.983552in}%
\pgfsys@useobject{currentmarker}{}%
\end{pgfscope}%
\end{pgfscope}%
\begin{pgfscope}%
\pgfsetbuttcap%
\pgfsetroundjoin%
\definecolor{currentfill}{rgb}{0.000000,0.000000,0.000000}%
\pgfsetfillcolor{currentfill}%
\pgfsetlinewidth{0.501875pt}%
\definecolor{currentstroke}{rgb}{0.000000,0.000000,0.000000}%
\pgfsetstrokecolor{currentstroke}%
\pgfsetdash{}{0pt}%
\pgfsys@defobject{currentmarker}{\pgfqpoint{-0.027778in}{0.000000in}}{\pgfqpoint{0.000000in}{0.000000in}}{%
\pgfpathmoveto{\pgfqpoint{0.000000in}{0.000000in}}%
\pgfpathlineto{\pgfqpoint{-0.027778in}{0.000000in}}%
\pgfusepath{stroke,fill}%
}%
\begin{pgfscope}%
\pgfsys@transformshift{3.506457in}{1.983552in}%
\pgfsys@useobject{currentmarker}{}%
\end{pgfscope}%
\end{pgfscope}%
\begin{pgfscope}%
\pgfsetbuttcap%
\pgfsetroundjoin%
\definecolor{currentfill}{rgb}{0.000000,0.000000,0.000000}%
\pgfsetfillcolor{currentfill}%
\pgfsetlinewidth{0.501875pt}%
\definecolor{currentstroke}{rgb}{0.000000,0.000000,0.000000}%
\pgfsetstrokecolor{currentstroke}%
\pgfsetdash{}{0pt}%
\pgfsys@defobject{currentmarker}{\pgfqpoint{0.000000in}{0.000000in}}{\pgfqpoint{0.027778in}{0.000000in}}{%
\pgfpathmoveto{\pgfqpoint{0.000000in}{0.000000in}}%
\pgfpathlineto{\pgfqpoint{0.027778in}{0.000000in}}%
\pgfusepath{stroke,fill}%
}%
\begin{pgfscope}%
\pgfsys@transformshift{0.485787in}{2.079893in}%
\pgfsys@useobject{currentmarker}{}%
\end{pgfscope}%
\end{pgfscope}%
\begin{pgfscope}%
\pgfsetbuttcap%
\pgfsetroundjoin%
\definecolor{currentfill}{rgb}{0.000000,0.000000,0.000000}%
\pgfsetfillcolor{currentfill}%
\pgfsetlinewidth{0.501875pt}%
\definecolor{currentstroke}{rgb}{0.000000,0.000000,0.000000}%
\pgfsetstrokecolor{currentstroke}%
\pgfsetdash{}{0pt}%
\pgfsys@defobject{currentmarker}{\pgfqpoint{-0.027778in}{0.000000in}}{\pgfqpoint{0.000000in}{0.000000in}}{%
\pgfpathmoveto{\pgfqpoint{0.000000in}{0.000000in}}%
\pgfpathlineto{\pgfqpoint{-0.027778in}{0.000000in}}%
\pgfusepath{stroke,fill}%
}%
\begin{pgfscope}%
\pgfsys@transformshift{3.506457in}{2.079893in}%
\pgfsys@useobject{currentmarker}{}%
\end{pgfscope}%
\end{pgfscope}%
\begin{pgfscope}%
\pgfsetbuttcap%
\pgfsetroundjoin%
\definecolor{currentfill}{rgb}{0.000000,0.000000,0.000000}%
\pgfsetfillcolor{currentfill}%
\pgfsetlinewidth{0.501875pt}%
\definecolor{currentstroke}{rgb}{0.000000,0.000000,0.000000}%
\pgfsetstrokecolor{currentstroke}%
\pgfsetdash{}{0pt}%
\pgfsys@defobject{currentmarker}{\pgfqpoint{0.000000in}{0.000000in}}{\pgfqpoint{0.027778in}{0.000000in}}{%
\pgfpathmoveto{\pgfqpoint{0.000000in}{0.000000in}}%
\pgfpathlineto{\pgfqpoint{0.027778in}{0.000000in}}%
\pgfusepath{stroke,fill}%
}%
\begin{pgfscope}%
\pgfsys@transformshift{0.485787in}{2.161349in}%
\pgfsys@useobject{currentmarker}{}%
\end{pgfscope}%
\end{pgfscope}%
\begin{pgfscope}%
\pgfsetbuttcap%
\pgfsetroundjoin%
\definecolor{currentfill}{rgb}{0.000000,0.000000,0.000000}%
\pgfsetfillcolor{currentfill}%
\pgfsetlinewidth{0.501875pt}%
\definecolor{currentstroke}{rgb}{0.000000,0.000000,0.000000}%
\pgfsetstrokecolor{currentstroke}%
\pgfsetdash{}{0pt}%
\pgfsys@defobject{currentmarker}{\pgfqpoint{-0.027778in}{0.000000in}}{\pgfqpoint{0.000000in}{0.000000in}}{%
\pgfpathmoveto{\pgfqpoint{0.000000in}{0.000000in}}%
\pgfpathlineto{\pgfqpoint{-0.027778in}{0.000000in}}%
\pgfusepath{stroke,fill}%
}%
\begin{pgfscope}%
\pgfsys@transformshift{3.506457in}{2.161349in}%
\pgfsys@useobject{currentmarker}{}%
\end{pgfscope}%
\end{pgfscope}%
\begin{pgfscope}%
\pgfsetbuttcap%
\pgfsetroundjoin%
\definecolor{currentfill}{rgb}{0.000000,0.000000,0.000000}%
\pgfsetfillcolor{currentfill}%
\pgfsetlinewidth{0.501875pt}%
\definecolor{currentstroke}{rgb}{0.000000,0.000000,0.000000}%
\pgfsetstrokecolor{currentstroke}%
\pgfsetdash{}{0pt}%
\pgfsys@defobject{currentmarker}{\pgfqpoint{0.000000in}{0.000000in}}{\pgfqpoint{0.027778in}{0.000000in}}{%
\pgfpathmoveto{\pgfqpoint{0.000000in}{0.000000in}}%
\pgfpathlineto{\pgfqpoint{0.027778in}{0.000000in}}%
\pgfusepath{stroke,fill}%
}%
\begin{pgfscope}%
\pgfsys@transformshift{0.485787in}{2.231909in}%
\pgfsys@useobject{currentmarker}{}%
\end{pgfscope}%
\end{pgfscope}%
\begin{pgfscope}%
\pgfsetbuttcap%
\pgfsetroundjoin%
\definecolor{currentfill}{rgb}{0.000000,0.000000,0.000000}%
\pgfsetfillcolor{currentfill}%
\pgfsetlinewidth{0.501875pt}%
\definecolor{currentstroke}{rgb}{0.000000,0.000000,0.000000}%
\pgfsetstrokecolor{currentstroke}%
\pgfsetdash{}{0pt}%
\pgfsys@defobject{currentmarker}{\pgfqpoint{-0.027778in}{0.000000in}}{\pgfqpoint{0.000000in}{0.000000in}}{%
\pgfpathmoveto{\pgfqpoint{0.000000in}{0.000000in}}%
\pgfpathlineto{\pgfqpoint{-0.027778in}{0.000000in}}%
\pgfusepath{stroke,fill}%
}%
\begin{pgfscope}%
\pgfsys@transformshift{3.506457in}{2.231909in}%
\pgfsys@useobject{currentmarker}{}%
\end{pgfscope}%
\end{pgfscope}%
\begin{pgfscope}%
\pgfsetbuttcap%
\pgfsetroundjoin%
\definecolor{currentfill}{rgb}{0.000000,0.000000,0.000000}%
\pgfsetfillcolor{currentfill}%
\pgfsetlinewidth{0.501875pt}%
\definecolor{currentstroke}{rgb}{0.000000,0.000000,0.000000}%
\pgfsetstrokecolor{currentstroke}%
\pgfsetdash{}{0pt}%
\pgfsys@defobject{currentmarker}{\pgfqpoint{0.000000in}{0.000000in}}{\pgfqpoint{0.027778in}{0.000000in}}{%
\pgfpathmoveto{\pgfqpoint{0.000000in}{0.000000in}}%
\pgfpathlineto{\pgfqpoint{0.027778in}{0.000000in}}%
\pgfusepath{stroke,fill}%
}%
\begin{pgfscope}%
\pgfsys@transformshift{0.485787in}{2.294147in}%
\pgfsys@useobject{currentmarker}{}%
\end{pgfscope}%
\end{pgfscope}%
\begin{pgfscope}%
\pgfsetbuttcap%
\pgfsetroundjoin%
\definecolor{currentfill}{rgb}{0.000000,0.000000,0.000000}%
\pgfsetfillcolor{currentfill}%
\pgfsetlinewidth{0.501875pt}%
\definecolor{currentstroke}{rgb}{0.000000,0.000000,0.000000}%
\pgfsetstrokecolor{currentstroke}%
\pgfsetdash{}{0pt}%
\pgfsys@defobject{currentmarker}{\pgfqpoint{-0.027778in}{0.000000in}}{\pgfqpoint{0.000000in}{0.000000in}}{%
\pgfpathmoveto{\pgfqpoint{0.000000in}{0.000000in}}%
\pgfpathlineto{\pgfqpoint{-0.027778in}{0.000000in}}%
\pgfusepath{stroke,fill}%
}%
\begin{pgfscope}%
\pgfsys@transformshift{3.506457in}{2.294147in}%
\pgfsys@useobject{currentmarker}{}%
\end{pgfscope}%
\end{pgfscope}%
\begin{pgfscope}%
\pgfsetbuttcap%
\pgfsetroundjoin%
\definecolor{currentfill}{rgb}{0.000000,0.000000,0.000000}%
\pgfsetfillcolor{currentfill}%
\pgfsetlinewidth{0.501875pt}%
\definecolor{currentstroke}{rgb}{0.000000,0.000000,0.000000}%
\pgfsetstrokecolor{currentstroke}%
\pgfsetdash{}{0pt}%
\pgfsys@defobject{currentmarker}{\pgfqpoint{0.000000in}{0.000000in}}{\pgfqpoint{0.027778in}{0.000000in}}{%
\pgfpathmoveto{\pgfqpoint{0.000000in}{0.000000in}}%
\pgfpathlineto{\pgfqpoint{0.027778in}{0.000000in}}%
\pgfusepath{stroke,fill}%
}%
\begin{pgfscope}%
\pgfsys@transformshift{0.485787in}{2.716090in}%
\pgfsys@useobject{currentmarker}{}%
\end{pgfscope}%
\end{pgfscope}%
\begin{pgfscope}%
\pgfsetbuttcap%
\pgfsetroundjoin%
\definecolor{currentfill}{rgb}{0.000000,0.000000,0.000000}%
\pgfsetfillcolor{currentfill}%
\pgfsetlinewidth{0.501875pt}%
\definecolor{currentstroke}{rgb}{0.000000,0.000000,0.000000}%
\pgfsetstrokecolor{currentstroke}%
\pgfsetdash{}{0pt}%
\pgfsys@defobject{currentmarker}{\pgfqpoint{-0.027778in}{0.000000in}}{\pgfqpoint{0.000000in}{0.000000in}}{%
\pgfpathmoveto{\pgfqpoint{0.000000in}{0.000000in}}%
\pgfpathlineto{\pgfqpoint{-0.027778in}{0.000000in}}%
\pgfusepath{stroke,fill}%
}%
\begin{pgfscope}%
\pgfsys@transformshift{3.506457in}{2.716090in}%
\pgfsys@useobject{currentmarker}{}%
\end{pgfscope}%
\end{pgfscope}%
\begin{pgfscope}%
\pgftext[x=0.170972in,y=1.474219in,,bottom,rotate=90.000000]{\rmfamily\fontsize{9.000000}{10.800000}\selectfont Candidates}%
\end{pgfscope}%
\begin{pgfscope}%
\pgfsetrectcap%
\pgfsetroundjoin%
\pgfsetlinewidth{1.003750pt}%
\definecolor{currentstroke}{rgb}{1.000000,0.000000,0.000000}%
\pgfsetstrokecolor{currentstroke}%
\pgfsetdash{}{0pt}%
\pgfpathmoveto{\pgfqpoint{0.500967in}{0.808062in}}%
\pgfpathlineto{\pgfqpoint{0.652759in}{0.805792in}}%
\pgfpathlineto{\pgfqpoint{0.759014in}{0.805908in}}%
\pgfpathlineto{\pgfqpoint{0.850089in}{0.808245in}}%
\pgfpathlineto{\pgfqpoint{0.910806in}{0.811961in}}%
\pgfpathlineto{\pgfqpoint{0.971523in}{0.818737in}}%
\pgfpathlineto{\pgfqpoint{1.017061in}{0.827109in}}%
\pgfpathlineto{\pgfqpoint{1.047419in}{0.834752in}}%
\pgfpathlineto{\pgfqpoint{1.077778in}{0.844590in}}%
\pgfpathlineto{\pgfqpoint{1.108136in}{0.857113in}}%
\pgfpathlineto{\pgfqpoint{1.123316in}{0.864283in}}%
\pgfpathlineto{\pgfqpoint{1.153674in}{0.881799in}}%
\pgfpathlineto{\pgfqpoint{1.184033in}{0.903355in}}%
\pgfpathlineto{\pgfqpoint{1.214391in}{0.929456in}}%
\pgfpathlineto{\pgfqpoint{1.244750in}{0.960510in}}%
\pgfpathlineto{\pgfqpoint{1.275108in}{0.996777in}}%
\pgfpathlineto{\pgfqpoint{1.305467in}{1.038331in}}%
\pgfpathlineto{\pgfqpoint{1.335825in}{1.085033in}}%
\pgfpathlineto{\pgfqpoint{1.366184in}{1.136538in}}%
\pgfpathlineto{\pgfqpoint{1.411721in}{1.222078in}}%
\pgfpathlineto{\pgfqpoint{1.442080in}{1.283046in}}%
\pgfpathlineto{\pgfqpoint{1.487618in}{1.378802in}}%
\pgfpathlineto{\pgfqpoint{1.563514in}{1.547737in}}%
\pgfpathlineto{\pgfqpoint{1.609051in}{1.653811in}}%
\pgfpathlineto{\pgfqpoint{1.639410in}{1.728611in}}%
\pgfpathlineto{\pgfqpoint{1.654589in}{1.769334in}}%
\pgfpathlineto{\pgfqpoint{1.669768in}{1.813455in}}%
\pgfpathlineto{\pgfqpoint{1.700127in}{1.911383in}}%
\pgfpathlineto{\pgfqpoint{1.776023in}{2.160698in}}%
\pgfpathlineto{\pgfqpoint{1.806382in}{2.254407in}}%
\pgfpathlineto{\pgfqpoint{1.836740in}{2.341335in}}%
\pgfpathlineto{\pgfqpoint{1.867099in}{2.418440in}}%
\pgfpathlineto{\pgfqpoint{1.897457in}{2.484620in}}%
\pgfpathlineto{\pgfqpoint{1.912636in}{2.513583in}}%
\pgfpathlineto{\pgfqpoint{1.927816in}{2.539024in}}%
\pgfpathlineto{\pgfqpoint{1.942995in}{2.561112in}}%
\pgfpathlineto{\pgfqpoint{1.958174in}{2.579745in}}%
\pgfpathlineto{\pgfqpoint{1.973353in}{2.594840in}}%
\pgfpathlineto{\pgfqpoint{1.988533in}{2.606333in}}%
\pgfpathlineto{\pgfqpoint{2.003712in}{2.614173in}}%
\pgfpathlineto{\pgfqpoint{2.018891in}{2.618328in}}%
\pgfpathlineto{\pgfqpoint{2.034070in}{2.618781in}}%
\pgfpathlineto{\pgfqpoint{2.049250in}{2.615529in}}%
\pgfpathlineto{\pgfqpoint{2.064429in}{2.608585in}}%
\pgfpathlineto{\pgfqpoint{2.079608in}{2.597977in}}%
\pgfpathlineto{\pgfqpoint{2.094787in}{2.583748in}}%
\pgfpathlineto{\pgfqpoint{2.109967in}{2.565958in}}%
\pgfpathlineto{\pgfqpoint{2.125146in}{2.544684in}}%
\pgfpathlineto{\pgfqpoint{2.140325in}{2.520021in}}%
\pgfpathlineto{\pgfqpoint{2.155504in}{2.491745in}}%
\pgfpathlineto{\pgfqpoint{2.170684in}{2.460716in}}%
\pgfpathlineto{\pgfqpoint{2.185863in}{2.426889in}}%
\pgfpathlineto{\pgfqpoint{2.216221in}{2.350848in}}%
\pgfpathlineto{\pgfqpoint{2.246580in}{2.264675in}}%
\pgfpathlineto{\pgfqpoint{2.276938in}{2.171280in}}%
\pgfpathlineto{\pgfqpoint{2.322476in}{2.022264in}}%
\pgfpathlineto{\pgfqpoint{2.398372in}{1.770257in}}%
\pgfpathlineto{\pgfqpoint{2.428731in}{1.675607in}}%
\pgfpathlineto{\pgfqpoint{2.459089in}{1.583562in}}%
\pgfpathlineto{\pgfqpoint{2.489448in}{1.496035in}}%
\pgfpathlineto{\pgfqpoint{2.519806in}{1.412829in}}%
\pgfpathlineto{\pgfqpoint{2.550165in}{1.333523in}}%
\pgfpathlineto{\pgfqpoint{2.580523in}{1.257705in}}%
\pgfpathlineto{\pgfqpoint{2.610882in}{1.185346in}}%
\pgfpathlineto{\pgfqpoint{2.626061in}{1.150025in}}%
\pgfpathlineto{\pgfqpoint{2.656419in}{1.087443in}}%
\pgfpathlineto{\pgfqpoint{2.686778in}{1.033268in}}%
\pgfpathlineto{\pgfqpoint{2.717136in}{0.986334in}}%
\pgfpathlineto{\pgfqpoint{2.747495in}{0.945503in}}%
\pgfpathlineto{\pgfqpoint{2.777853in}{0.909844in}}%
\pgfpathlineto{\pgfqpoint{2.808212in}{0.878590in}}%
\pgfpathlineto{\pgfqpoint{2.838570in}{0.851105in}}%
\pgfpathlineto{\pgfqpoint{2.868929in}{0.826860in}}%
\pgfpathlineto{\pgfqpoint{2.899287in}{0.805402in}}%
\pgfpathlineto{\pgfqpoint{2.944825in}{0.777805in}}%
\pgfpathlineto{\pgfqpoint{2.975183in}{0.761742in}}%
\pgfpathlineto{\pgfqpoint{3.005542in}{0.747355in}}%
\pgfpathlineto{\pgfqpoint{3.035900in}{0.734422in}}%
\pgfpathlineto{\pgfqpoint{3.081438in}{0.717233in}}%
\pgfpathlineto{\pgfqpoint{3.142155in}{0.698000in}}%
\pgfpathlineto{\pgfqpoint{3.202872in}{0.681911in}}%
\pgfpathlineto{\pgfqpoint{3.278768in}{0.665190in}}%
\pgfpathlineto{\pgfqpoint{3.339485in}{0.653795in}}%
\pgfpathlineto{\pgfqpoint{3.415382in}{0.641436in}}%
\pgfpathlineto{\pgfqpoint{3.491278in}{0.630774in}}%
\pgfpathlineto{\pgfqpoint{3.491278in}{0.630774in}}%
\pgfusepath{stroke}%
\end{pgfscope}%
\begin{pgfscope}%
\pgfpathrectangle{\pgfqpoint{0.485787in}{0.226975in}}{\pgfqpoint{3.020670in}{2.494489in}} %
\pgfusepath{clip}%
\pgfsetbuttcap%
\pgfsetroundjoin%
\pgfsetlinewidth{0.501875pt}%
\definecolor{currentstroke}{rgb}{0.000000,0.000000,0.000000}%
\pgfsetstrokecolor{currentstroke}%
\pgfsetdash{}{0pt}%
\pgfpathmoveto{\pgfqpoint{0.500891in}{0.753898in}}%
\pgfpathlineto{\pgfqpoint{0.500891in}{0.800692in}}%
\pgfusepath{stroke}%
\end{pgfscope}%
\begin{pgfscope}%
\pgfpathrectangle{\pgfqpoint{0.485787in}{0.226975in}}{\pgfqpoint{3.020670in}{2.494489in}} %
\pgfusepath{clip}%
\pgfsetbuttcap%
\pgfsetroundjoin%
\pgfsetlinewidth{0.501875pt}%
\definecolor{currentstroke}{rgb}{0.000000,0.000000,0.000000}%
\pgfsetstrokecolor{currentstroke}%
\pgfsetdash{}{0pt}%
\pgfpathmoveto{\pgfqpoint{0.531097in}{0.703981in}}%
\pgfpathlineto{\pgfqpoint{0.531097in}{0.752986in}}%
\pgfusepath{stroke}%
\end{pgfscope}%
\begin{pgfscope}%
\pgfpathrectangle{\pgfqpoint{0.485787in}{0.226975in}}{\pgfqpoint{3.020670in}{2.494489in}} %
\pgfusepath{clip}%
\pgfsetbuttcap%
\pgfsetroundjoin%
\pgfsetlinewidth{0.501875pt}%
\definecolor{currentstroke}{rgb}{0.000000,0.000000,0.000000}%
\pgfsetstrokecolor{currentstroke}%
\pgfsetdash{}{0pt}%
\pgfpathmoveto{\pgfqpoint{0.561304in}{0.774670in}}%
\pgfpathlineto{\pgfqpoint{0.561304in}{0.820572in}}%
\pgfusepath{stroke}%
\end{pgfscope}%
\begin{pgfscope}%
\pgfpathrectangle{\pgfqpoint{0.485787in}{0.226975in}}{\pgfqpoint{3.020670in}{2.494489in}} %
\pgfusepath{clip}%
\pgfsetbuttcap%
\pgfsetroundjoin%
\pgfsetlinewidth{0.501875pt}%
\definecolor{currentstroke}{rgb}{0.000000,0.000000,0.000000}%
\pgfsetstrokecolor{currentstroke}%
\pgfsetdash{}{0pt}%
\pgfpathmoveto{\pgfqpoint{0.591511in}{0.757066in}}%
\pgfpathlineto{\pgfqpoint{0.591511in}{0.803722in}}%
\pgfusepath{stroke}%
\end{pgfscope}%
\begin{pgfscope}%
\pgfpathrectangle{\pgfqpoint{0.485787in}{0.226975in}}{\pgfqpoint{3.020670in}{2.494489in}} %
\pgfusepath{clip}%
\pgfsetbuttcap%
\pgfsetroundjoin%
\pgfsetlinewidth{0.501875pt}%
\definecolor{currentstroke}{rgb}{0.000000,0.000000,0.000000}%
\pgfsetstrokecolor{currentstroke}%
\pgfsetdash{}{0pt}%
\pgfpathmoveto{\pgfqpoint{0.621717in}{0.751776in}}%
\pgfpathlineto{\pgfqpoint{0.621717in}{0.798661in}}%
\pgfusepath{stroke}%
\end{pgfscope}%
\begin{pgfscope}%
\pgfpathrectangle{\pgfqpoint{0.485787in}{0.226975in}}{\pgfqpoint{3.020670in}{2.494489in}} %
\pgfusepath{clip}%
\pgfsetbuttcap%
\pgfsetroundjoin%
\pgfsetlinewidth{0.501875pt}%
\definecolor{currentstroke}{rgb}{0.000000,0.000000,0.000000}%
\pgfsetstrokecolor{currentstroke}%
\pgfsetdash{}{0pt}%
\pgfpathmoveto{\pgfqpoint{0.651924in}{0.772629in}}%
\pgfpathlineto{\pgfqpoint{0.651924in}{0.818618in}}%
\pgfusepath{stroke}%
\end{pgfscope}%
\begin{pgfscope}%
\pgfpathrectangle{\pgfqpoint{0.485787in}{0.226975in}}{\pgfqpoint{3.020670in}{2.494489in}} %
\pgfusepath{clip}%
\pgfsetbuttcap%
\pgfsetroundjoin%
\pgfsetlinewidth{0.501875pt}%
\definecolor{currentstroke}{rgb}{0.000000,0.000000,0.000000}%
\pgfsetstrokecolor{currentstroke}%
\pgfsetdash{}{0pt}%
\pgfpathmoveto{\pgfqpoint{0.682131in}{0.772629in}}%
\pgfpathlineto{\pgfqpoint{0.682131in}{0.818618in}}%
\pgfusepath{stroke}%
\end{pgfscope}%
\begin{pgfscope}%
\pgfpathrectangle{\pgfqpoint{0.485787in}{0.226975in}}{\pgfqpoint{3.020670in}{2.494489in}} %
\pgfusepath{clip}%
\pgfsetbuttcap%
\pgfsetroundjoin%
\pgfsetlinewidth{0.501875pt}%
\definecolor{currentstroke}{rgb}{0.000000,0.000000,0.000000}%
\pgfsetstrokecolor{currentstroke}%
\pgfsetdash{}{0pt}%
\pgfpathmoveto{\pgfqpoint{0.712338in}{0.785759in}}%
\pgfpathlineto{\pgfqpoint{0.712338in}{0.831192in}}%
\pgfusepath{stroke}%
\end{pgfscope}%
\begin{pgfscope}%
\pgfpathrectangle{\pgfqpoint{0.485787in}{0.226975in}}{\pgfqpoint{3.020670in}{2.494489in}} %
\pgfusepath{clip}%
\pgfsetbuttcap%
\pgfsetroundjoin%
\pgfsetlinewidth{0.501875pt}%
\definecolor{currentstroke}{rgb}{0.000000,0.000000,0.000000}%
\pgfsetstrokecolor{currentstroke}%
\pgfsetdash{}{0pt}%
\pgfpathmoveto{\pgfqpoint{0.742544in}{0.724511in}}%
\pgfpathlineto{\pgfqpoint{0.742544in}{0.772594in}}%
\pgfusepath{stroke}%
\end{pgfscope}%
\begin{pgfscope}%
\pgfpathrectangle{\pgfqpoint{0.485787in}{0.226975in}}{\pgfqpoint{3.020670in}{2.494489in}} %
\pgfusepath{clip}%
\pgfsetbuttcap%
\pgfsetroundjoin%
\pgfsetlinewidth{0.501875pt}%
\definecolor{currentstroke}{rgb}{0.000000,0.000000,0.000000}%
\pgfsetstrokecolor{currentstroke}%
\pgfsetdash{}{0pt}%
\pgfpathmoveto{\pgfqpoint{0.772751in}{0.769552in}}%
\pgfpathlineto{\pgfqpoint{0.772751in}{0.815672in}}%
\pgfusepath{stroke}%
\end{pgfscope}%
\begin{pgfscope}%
\pgfpathrectangle{\pgfqpoint{0.485787in}{0.226975in}}{\pgfqpoint{3.020670in}{2.494489in}} %
\pgfusepath{clip}%
\pgfsetbuttcap%
\pgfsetroundjoin%
\pgfsetlinewidth{0.501875pt}%
\definecolor{currentstroke}{rgb}{0.000000,0.000000,0.000000}%
\pgfsetstrokecolor{currentstroke}%
\pgfsetdash{}{0pt}%
\pgfpathmoveto{\pgfqpoint{0.802958in}{0.776703in}}%
\pgfpathlineto{\pgfqpoint{0.802958in}{0.822519in}}%
\pgfusepath{stroke}%
\end{pgfscope}%
\begin{pgfscope}%
\pgfpathrectangle{\pgfqpoint{0.485787in}{0.226975in}}{\pgfqpoint{3.020670in}{2.494489in}} %
\pgfusepath{clip}%
\pgfsetbuttcap%
\pgfsetroundjoin%
\pgfsetlinewidth{0.501875pt}%
\definecolor{currentstroke}{rgb}{0.000000,0.000000,0.000000}%
\pgfsetstrokecolor{currentstroke}%
\pgfsetdash{}{0pt}%
\pgfpathmoveto{\pgfqpoint{0.833164in}{0.758118in}}%
\pgfpathlineto{\pgfqpoint{0.833164in}{0.804729in}}%
\pgfusepath{stroke}%
\end{pgfscope}%
\begin{pgfscope}%
\pgfpathrectangle{\pgfqpoint{0.485787in}{0.226975in}}{\pgfqpoint{3.020670in}{2.494489in}} %
\pgfusepath{clip}%
\pgfsetbuttcap%
\pgfsetroundjoin%
\pgfsetlinewidth{0.501875pt}%
\definecolor{currentstroke}{rgb}{0.000000,0.000000,0.000000}%
\pgfsetstrokecolor{currentstroke}%
\pgfsetdash{}{0pt}%
\pgfpathmoveto{\pgfqpoint{0.863371in}{0.824248in}}%
\pgfpathlineto{\pgfqpoint{0.863371in}{0.868089in}}%
\pgfusepath{stroke}%
\end{pgfscope}%
\begin{pgfscope}%
\pgfpathrectangle{\pgfqpoint{0.485787in}{0.226975in}}{\pgfqpoint{3.020670in}{2.494489in}} %
\pgfusepath{clip}%
\pgfsetbuttcap%
\pgfsetroundjoin%
\pgfsetlinewidth{0.501875pt}%
\definecolor{currentstroke}{rgb}{0.000000,0.000000,0.000000}%
\pgfsetstrokecolor{currentstroke}%
\pgfsetdash{}{0pt}%
\pgfpathmoveto{\pgfqpoint{0.893578in}{0.797600in}}%
\pgfpathlineto{\pgfqpoint{0.893578in}{0.842537in}}%
\pgfusepath{stroke}%
\end{pgfscope}%
\begin{pgfscope}%
\pgfpathrectangle{\pgfqpoint{0.485787in}{0.226975in}}{\pgfqpoint{3.020670in}{2.494489in}} %
\pgfusepath{clip}%
\pgfsetbuttcap%
\pgfsetroundjoin%
\pgfsetlinewidth{0.501875pt}%
\definecolor{currentstroke}{rgb}{0.000000,0.000000,0.000000}%
\pgfsetstrokecolor{currentstroke}%
\pgfsetdash{}{0pt}%
\pgfpathmoveto{\pgfqpoint{0.923784in}{0.816769in}}%
\pgfpathlineto{\pgfqpoint{0.923784in}{0.860915in}}%
\pgfusepath{stroke}%
\end{pgfscope}%
\begin{pgfscope}%
\pgfpathrectangle{\pgfqpoint{0.485787in}{0.226975in}}{\pgfqpoint{3.020670in}{2.494489in}} %
\pgfusepath{clip}%
\pgfsetbuttcap%
\pgfsetroundjoin%
\pgfsetlinewidth{0.501875pt}%
\definecolor{currentstroke}{rgb}{0.000000,0.000000,0.000000}%
\pgfsetstrokecolor{currentstroke}%
\pgfsetdash{}{0pt}%
\pgfpathmoveto{\pgfqpoint{0.953991in}{0.790725in}}%
\pgfpathlineto{\pgfqpoint{0.953991in}{0.835949in}}%
\pgfusepath{stroke}%
\end{pgfscope}%
\begin{pgfscope}%
\pgfpathrectangle{\pgfqpoint{0.485787in}{0.226975in}}{\pgfqpoint{3.020670in}{2.494489in}} %
\pgfusepath{clip}%
\pgfsetbuttcap%
\pgfsetroundjoin%
\pgfsetlinewidth{0.501875pt}%
\definecolor{currentstroke}{rgb}{0.000000,0.000000,0.000000}%
\pgfsetstrokecolor{currentstroke}%
\pgfsetdash{}{0pt}%
\pgfpathmoveto{\pgfqpoint{0.984198in}{0.809183in}}%
\pgfpathlineto{\pgfqpoint{0.984198in}{0.853641in}}%
\pgfusepath{stroke}%
\end{pgfscope}%
\begin{pgfscope}%
\pgfpathrectangle{\pgfqpoint{0.485787in}{0.226975in}}{\pgfqpoint{3.020670in}{2.494489in}} %
\pgfusepath{clip}%
\pgfsetbuttcap%
\pgfsetroundjoin%
\pgfsetlinewidth{0.501875pt}%
\definecolor{currentstroke}{rgb}{0.000000,0.000000,0.000000}%
\pgfsetstrokecolor{currentstroke}%
\pgfsetdash{}{0pt}%
\pgfpathmoveto{\pgfqpoint{1.014405in}{0.826101in}}%
\pgfpathlineto{\pgfqpoint{1.014405in}{0.869867in}}%
\pgfusepath{stroke}%
\end{pgfscope}%
\begin{pgfscope}%
\pgfpathrectangle{\pgfqpoint{0.485787in}{0.226975in}}{\pgfqpoint{3.020670in}{2.494489in}} %
\pgfusepath{clip}%
\pgfsetbuttcap%
\pgfsetroundjoin%
\pgfsetlinewidth{0.501875pt}%
\definecolor{currentstroke}{rgb}{0.000000,0.000000,0.000000}%
\pgfsetstrokecolor{currentstroke}%
\pgfsetdash{}{0pt}%
\pgfpathmoveto{\pgfqpoint{1.044611in}{0.842499in}}%
\pgfpathlineto{\pgfqpoint{1.044611in}{0.885605in}}%
\pgfusepath{stroke}%
\end{pgfscope}%
\begin{pgfscope}%
\pgfpathrectangle{\pgfqpoint{0.485787in}{0.226975in}}{\pgfqpoint{3.020670in}{2.494489in}} %
\pgfusepath{clip}%
\pgfsetbuttcap%
\pgfsetroundjoin%
\pgfsetlinewidth{0.501875pt}%
\definecolor{currentstroke}{rgb}{0.000000,0.000000,0.000000}%
\pgfsetstrokecolor{currentstroke}%
\pgfsetdash{}{0pt}%
\pgfpathmoveto{\pgfqpoint{1.074818in}{0.896223in}}%
\pgfpathlineto{\pgfqpoint{1.074818in}{0.937234in}}%
\pgfusepath{stroke}%
\end{pgfscope}%
\begin{pgfscope}%
\pgfpathrectangle{\pgfqpoint{0.485787in}{0.226975in}}{\pgfqpoint{3.020670in}{2.494489in}} %
\pgfusepath{clip}%
\pgfsetbuttcap%
\pgfsetroundjoin%
\pgfsetlinewidth{0.501875pt}%
\definecolor{currentstroke}{rgb}{0.000000,0.000000,0.000000}%
\pgfsetstrokecolor{currentstroke}%
\pgfsetdash{}{0pt}%
\pgfpathmoveto{\pgfqpoint{1.105025in}{0.893784in}}%
\pgfpathlineto{\pgfqpoint{1.105025in}{0.934887in}}%
\pgfusepath{stroke}%
\end{pgfscope}%
\begin{pgfscope}%
\pgfpathrectangle{\pgfqpoint{0.485787in}{0.226975in}}{\pgfqpoint{3.020670in}{2.494489in}} %
\pgfusepath{clip}%
\pgfsetbuttcap%
\pgfsetroundjoin%
\pgfsetlinewidth{0.501875pt}%
\definecolor{currentstroke}{rgb}{0.000000,0.000000,0.000000}%
\pgfsetstrokecolor{currentstroke}%
\pgfsetdash{}{0pt}%
\pgfpathmoveto{\pgfqpoint{1.135231in}{0.904275in}}%
\pgfpathlineto{\pgfqpoint{1.135231in}{0.944980in}}%
\pgfusepath{stroke}%
\end{pgfscope}%
\begin{pgfscope}%
\pgfpathrectangle{\pgfqpoint{0.485787in}{0.226975in}}{\pgfqpoint{3.020670in}{2.494489in}} %
\pgfusepath{clip}%
\pgfsetbuttcap%
\pgfsetroundjoin%
\pgfsetlinewidth{0.501875pt}%
\definecolor{currentstroke}{rgb}{0.000000,0.000000,0.000000}%
\pgfsetstrokecolor{currentstroke}%
\pgfsetdash{}{0pt}%
\pgfpathmoveto{\pgfqpoint{1.165438in}{0.909048in}}%
\pgfpathlineto{\pgfqpoint{1.165438in}{0.949573in}}%
\pgfusepath{stroke}%
\end{pgfscope}%
\begin{pgfscope}%
\pgfpathrectangle{\pgfqpoint{0.485787in}{0.226975in}}{\pgfqpoint{3.020670in}{2.494489in}} %
\pgfusepath{clip}%
\pgfsetbuttcap%
\pgfsetroundjoin%
\pgfsetlinewidth{0.501875pt}%
\definecolor{currentstroke}{rgb}{0.000000,0.000000,0.000000}%
\pgfsetstrokecolor{currentstroke}%
\pgfsetdash{}{0pt}%
\pgfpathmoveto{\pgfqpoint{1.195645in}{0.892152in}}%
\pgfpathlineto{\pgfqpoint{1.195645in}{0.933317in}}%
\pgfusepath{stroke}%
\end{pgfscope}%
\begin{pgfscope}%
\pgfpathrectangle{\pgfqpoint{0.485787in}{0.226975in}}{\pgfqpoint{3.020670in}{2.494489in}} %
\pgfusepath{clip}%
\pgfsetbuttcap%
\pgfsetroundjoin%
\pgfsetlinewidth{0.501875pt}%
\definecolor{currentstroke}{rgb}{0.000000,0.000000,0.000000}%
\pgfsetstrokecolor{currentstroke}%
\pgfsetdash{}{0pt}%
\pgfpathmoveto{\pgfqpoint{1.225851in}{0.977345in}}%
\pgfpathlineto{\pgfqpoint{1.225851in}{1.015379in}}%
\pgfusepath{stroke}%
\end{pgfscope}%
\begin{pgfscope}%
\pgfpathrectangle{\pgfqpoint{0.485787in}{0.226975in}}{\pgfqpoint{3.020670in}{2.494489in}} %
\pgfusepath{clip}%
\pgfsetbuttcap%
\pgfsetroundjoin%
\pgfsetlinewidth{0.501875pt}%
\definecolor{currentstroke}{rgb}{0.000000,0.000000,0.000000}%
\pgfsetstrokecolor{currentstroke}%
\pgfsetdash{}{0pt}%
\pgfpathmoveto{\pgfqpoint{1.256058in}{0.945769in}}%
\pgfpathlineto{\pgfqpoint{1.256058in}{0.984936in}}%
\pgfusepath{stroke}%
\end{pgfscope}%
\begin{pgfscope}%
\pgfpathrectangle{\pgfqpoint{0.485787in}{0.226975in}}{\pgfqpoint{3.020670in}{2.494489in}} %
\pgfusepath{clip}%
\pgfsetbuttcap%
\pgfsetroundjoin%
\pgfsetlinewidth{0.501875pt}%
\definecolor{currentstroke}{rgb}{0.000000,0.000000,0.000000}%
\pgfsetstrokecolor{currentstroke}%
\pgfsetdash{}{0pt}%
\pgfpathmoveto{\pgfqpoint{1.286265in}{1.025928in}}%
\pgfpathlineto{\pgfqpoint{1.286265in}{1.062282in}}%
\pgfusepath{stroke}%
\end{pgfscope}%
\begin{pgfscope}%
\pgfpathrectangle{\pgfqpoint{0.485787in}{0.226975in}}{\pgfqpoint{3.020670in}{2.494489in}} %
\pgfusepath{clip}%
\pgfsetbuttcap%
\pgfsetroundjoin%
\pgfsetlinewidth{0.501875pt}%
\definecolor{currentstroke}{rgb}{0.000000,0.000000,0.000000}%
\pgfsetstrokecolor{currentstroke}%
\pgfsetdash{}{0pt}%
\pgfpathmoveto{\pgfqpoint{1.316472in}{1.023378in}}%
\pgfpathlineto{\pgfqpoint{1.316472in}{1.059818in}}%
\pgfusepath{stroke}%
\end{pgfscope}%
\begin{pgfscope}%
\pgfpathrectangle{\pgfqpoint{0.485787in}{0.226975in}}{\pgfqpoint{3.020670in}{2.494489in}} %
\pgfusepath{clip}%
\pgfsetbuttcap%
\pgfsetroundjoin%
\pgfsetlinewidth{0.501875pt}%
\definecolor{currentstroke}{rgb}{0.000000,0.000000,0.000000}%
\pgfsetstrokecolor{currentstroke}%
\pgfsetdash{}{0pt}%
\pgfpathmoveto{\pgfqpoint{1.346678in}{1.072203in}}%
\pgfpathlineto{\pgfqpoint{1.346678in}{1.107025in}}%
\pgfusepath{stroke}%
\end{pgfscope}%
\begin{pgfscope}%
\pgfpathrectangle{\pgfqpoint{0.485787in}{0.226975in}}{\pgfqpoint{3.020670in}{2.494489in}} %
\pgfusepath{clip}%
\pgfsetbuttcap%
\pgfsetroundjoin%
\pgfsetlinewidth{0.501875pt}%
\definecolor{currentstroke}{rgb}{0.000000,0.000000,0.000000}%
\pgfsetstrokecolor{currentstroke}%
\pgfsetdash{}{0pt}%
\pgfpathmoveto{\pgfqpoint{1.376885in}{1.113161in}}%
\pgfpathlineto{\pgfqpoint{1.376885in}{1.146680in}}%
\pgfusepath{stroke}%
\end{pgfscope}%
\begin{pgfscope}%
\pgfpathrectangle{\pgfqpoint{0.485787in}{0.226975in}}{\pgfqpoint{3.020670in}{2.494489in}} %
\pgfusepath{clip}%
\pgfsetbuttcap%
\pgfsetroundjoin%
\pgfsetlinewidth{0.501875pt}%
\definecolor{currentstroke}{rgb}{0.000000,0.000000,0.000000}%
\pgfsetstrokecolor{currentstroke}%
\pgfsetdash{}{0pt}%
\pgfpathmoveto{\pgfqpoint{1.407092in}{1.172386in}}%
\pgfpathlineto{\pgfqpoint{1.407092in}{1.204105in}}%
\pgfusepath{stroke}%
\end{pgfscope}%
\begin{pgfscope}%
\pgfpathrectangle{\pgfqpoint{0.485787in}{0.226975in}}{\pgfqpoint{3.020670in}{2.494489in}} %
\pgfusepath{clip}%
\pgfsetbuttcap%
\pgfsetroundjoin%
\pgfsetlinewidth{0.501875pt}%
\definecolor{currentstroke}{rgb}{0.000000,0.000000,0.000000}%
\pgfsetstrokecolor{currentstroke}%
\pgfsetdash{}{0pt}%
\pgfpathmoveto{\pgfqpoint{1.437298in}{1.231330in}}%
\pgfpathlineto{\pgfqpoint{1.437298in}{1.261354in}}%
\pgfusepath{stroke}%
\end{pgfscope}%
\begin{pgfscope}%
\pgfpathrectangle{\pgfqpoint{0.485787in}{0.226975in}}{\pgfqpoint{3.020670in}{2.494489in}} %
\pgfusepath{clip}%
\pgfsetbuttcap%
\pgfsetroundjoin%
\pgfsetlinewidth{0.501875pt}%
\definecolor{currentstroke}{rgb}{0.000000,0.000000,0.000000}%
\pgfsetstrokecolor{currentstroke}%
\pgfsetdash{}{0pt}%
\pgfpathmoveto{\pgfqpoint{1.467505in}{1.298684in}}%
\pgfpathlineto{\pgfqpoint{1.467505in}{1.326878in}}%
\pgfusepath{stroke}%
\end{pgfscope}%
\begin{pgfscope}%
\pgfpathrectangle{\pgfqpoint{0.485787in}{0.226975in}}{\pgfqpoint{3.020670in}{2.494489in}} %
\pgfusepath{clip}%
\pgfsetbuttcap%
\pgfsetroundjoin%
\pgfsetlinewidth{0.501875pt}%
\definecolor{currentstroke}{rgb}{0.000000,0.000000,0.000000}%
\pgfsetstrokecolor{currentstroke}%
\pgfsetdash{}{0pt}%
\pgfpathmoveto{\pgfqpoint{1.497712in}{1.334068in}}%
\pgfpathlineto{\pgfqpoint{1.497712in}{1.361345in}}%
\pgfusepath{stroke}%
\end{pgfscope}%
\begin{pgfscope}%
\pgfpathrectangle{\pgfqpoint{0.485787in}{0.226975in}}{\pgfqpoint{3.020670in}{2.494489in}} %
\pgfusepath{clip}%
\pgfsetbuttcap%
\pgfsetroundjoin%
\pgfsetlinewidth{0.501875pt}%
\definecolor{currentstroke}{rgb}{0.000000,0.000000,0.000000}%
\pgfsetstrokecolor{currentstroke}%
\pgfsetdash{}{0pt}%
\pgfpathmoveto{\pgfqpoint{1.527918in}{1.434203in}}%
\pgfpathlineto{\pgfqpoint{1.527918in}{1.459043in}}%
\pgfusepath{stroke}%
\end{pgfscope}%
\begin{pgfscope}%
\pgfpathrectangle{\pgfqpoint{0.485787in}{0.226975in}}{\pgfqpoint{3.020670in}{2.494489in}} %
\pgfusepath{clip}%
\pgfsetbuttcap%
\pgfsetroundjoin%
\pgfsetlinewidth{0.501875pt}%
\definecolor{currentstroke}{rgb}{0.000000,0.000000,0.000000}%
\pgfsetstrokecolor{currentstroke}%
\pgfsetdash{}{0pt}%
\pgfpathmoveto{\pgfqpoint{1.558125in}{1.509022in}}%
\pgfpathlineto{\pgfqpoint{1.558125in}{1.532183in}}%
\pgfusepath{stroke}%
\end{pgfscope}%
\begin{pgfscope}%
\pgfpathrectangle{\pgfqpoint{0.485787in}{0.226975in}}{\pgfqpoint{3.020670in}{2.494489in}} %
\pgfusepath{clip}%
\pgfsetbuttcap%
\pgfsetroundjoin%
\pgfsetlinewidth{0.501875pt}%
\definecolor{currentstroke}{rgb}{0.000000,0.000000,0.000000}%
\pgfsetstrokecolor{currentstroke}%
\pgfsetdash{}{0pt}%
\pgfpathmoveto{\pgfqpoint{1.588332in}{1.558009in}}%
\pgfpathlineto{\pgfqpoint{1.588332in}{1.580132in}}%
\pgfusepath{stroke}%
\end{pgfscope}%
\begin{pgfscope}%
\pgfpathrectangle{\pgfqpoint{0.485787in}{0.226975in}}{\pgfqpoint{3.020670in}{2.494489in}} %
\pgfusepath{clip}%
\pgfsetbuttcap%
\pgfsetroundjoin%
\pgfsetlinewidth{0.501875pt}%
\definecolor{currentstroke}{rgb}{0.000000,0.000000,0.000000}%
\pgfsetstrokecolor{currentstroke}%
\pgfsetdash{}{0pt}%
\pgfpathmoveto{\pgfqpoint{1.618538in}{1.652110in}}%
\pgfpathlineto{\pgfqpoint{1.618538in}{1.672366in}}%
\pgfusepath{stroke}%
\end{pgfscope}%
\begin{pgfscope}%
\pgfpathrectangle{\pgfqpoint{0.485787in}{0.226975in}}{\pgfqpoint{3.020670in}{2.494489in}} %
\pgfusepath{clip}%
\pgfsetbuttcap%
\pgfsetroundjoin%
\pgfsetlinewidth{0.501875pt}%
\definecolor{currentstroke}{rgb}{0.000000,0.000000,0.000000}%
\pgfsetstrokecolor{currentstroke}%
\pgfsetdash{}{0pt}%
\pgfpathmoveto{\pgfqpoint{1.648745in}{1.752685in}}%
\pgfpathlineto{\pgfqpoint{1.648745in}{1.771118in}}%
\pgfusepath{stroke}%
\end{pgfscope}%
\begin{pgfscope}%
\pgfpathrectangle{\pgfqpoint{0.485787in}{0.226975in}}{\pgfqpoint{3.020670in}{2.494489in}} %
\pgfusepath{clip}%
\pgfsetbuttcap%
\pgfsetroundjoin%
\pgfsetlinewidth{0.501875pt}%
\definecolor{currentstroke}{rgb}{0.000000,0.000000,0.000000}%
\pgfsetstrokecolor{currentstroke}%
\pgfsetdash{}{0pt}%
\pgfpathmoveto{\pgfqpoint{1.678952in}{1.853275in}}%
\pgfpathlineto{\pgfqpoint{1.678952in}{1.870047in}}%
\pgfusepath{stroke}%
\end{pgfscope}%
\begin{pgfscope}%
\pgfpathrectangle{\pgfqpoint{0.485787in}{0.226975in}}{\pgfqpoint{3.020670in}{2.494489in}} %
\pgfusepath{clip}%
\pgfsetbuttcap%
\pgfsetroundjoin%
\pgfsetlinewidth{0.501875pt}%
\definecolor{currentstroke}{rgb}{0.000000,0.000000,0.000000}%
\pgfsetstrokecolor{currentstroke}%
\pgfsetdash{}{0pt}%
\pgfpathmoveto{\pgfqpoint{1.709159in}{1.949788in}}%
\pgfpathlineto{\pgfqpoint{1.709159in}{1.965107in}}%
\pgfusepath{stroke}%
\end{pgfscope}%
\begin{pgfscope}%
\pgfpathrectangle{\pgfqpoint{0.485787in}{0.226975in}}{\pgfqpoint{3.020670in}{2.494489in}} %
\pgfusepath{clip}%
\pgfsetbuttcap%
\pgfsetroundjoin%
\pgfsetlinewidth{0.501875pt}%
\definecolor{currentstroke}{rgb}{0.000000,0.000000,0.000000}%
\pgfsetstrokecolor{currentstroke}%
\pgfsetdash{}{0pt}%
\pgfpathmoveto{\pgfqpoint{1.739365in}{2.034853in}}%
\pgfpathlineto{\pgfqpoint{1.739365in}{2.048995in}}%
\pgfusepath{stroke}%
\end{pgfscope}%
\begin{pgfscope}%
\pgfpathrectangle{\pgfqpoint{0.485787in}{0.226975in}}{\pgfqpoint{3.020670in}{2.494489in}} %
\pgfusepath{clip}%
\pgfsetbuttcap%
\pgfsetroundjoin%
\pgfsetlinewidth{0.501875pt}%
\definecolor{currentstroke}{rgb}{0.000000,0.000000,0.000000}%
\pgfsetstrokecolor{currentstroke}%
\pgfsetdash{}{0pt}%
\pgfpathmoveto{\pgfqpoint{1.769572in}{2.146613in}}%
\pgfpathlineto{\pgfqpoint{1.769572in}{2.159345in}}%
\pgfusepath{stroke}%
\end{pgfscope}%
\begin{pgfscope}%
\pgfpathrectangle{\pgfqpoint{0.485787in}{0.226975in}}{\pgfqpoint{3.020670in}{2.494489in}} %
\pgfusepath{clip}%
\pgfsetbuttcap%
\pgfsetroundjoin%
\pgfsetlinewidth{0.501875pt}%
\definecolor{currentstroke}{rgb}{0.000000,0.000000,0.000000}%
\pgfsetstrokecolor{currentstroke}%
\pgfsetdash{}{0pt}%
\pgfpathmoveto{\pgfqpoint{1.799779in}{2.227195in}}%
\pgfpathlineto{\pgfqpoint{1.799779in}{2.238997in}}%
\pgfusepath{stroke}%
\end{pgfscope}%
\begin{pgfscope}%
\pgfpathrectangle{\pgfqpoint{0.485787in}{0.226975in}}{\pgfqpoint{3.020670in}{2.494489in}} %
\pgfusepath{clip}%
\pgfsetbuttcap%
\pgfsetroundjoin%
\pgfsetlinewidth{0.501875pt}%
\definecolor{currentstroke}{rgb}{0.000000,0.000000,0.000000}%
\pgfsetstrokecolor{currentstroke}%
\pgfsetdash{}{0pt}%
\pgfpathmoveto{\pgfqpoint{1.829985in}{2.312353in}}%
\pgfpathlineto{\pgfqpoint{1.829985in}{2.323246in}}%
\pgfusepath{stroke}%
\end{pgfscope}%
\begin{pgfscope}%
\pgfpathrectangle{\pgfqpoint{0.485787in}{0.226975in}}{\pgfqpoint{3.020670in}{2.494489in}} %
\pgfusepath{clip}%
\pgfsetbuttcap%
\pgfsetroundjoin%
\pgfsetlinewidth{0.501875pt}%
\definecolor{currentstroke}{rgb}{0.000000,0.000000,0.000000}%
\pgfsetstrokecolor{currentstroke}%
\pgfsetdash{}{0pt}%
\pgfpathmoveto{\pgfqpoint{1.860192in}{2.392824in}}%
\pgfpathlineto{\pgfqpoint{1.860192in}{2.402923in}}%
\pgfusepath{stroke}%
\end{pgfscope}%
\begin{pgfscope}%
\pgfpathrectangle{\pgfqpoint{0.485787in}{0.226975in}}{\pgfqpoint{3.020670in}{2.494489in}} %
\pgfusepath{clip}%
\pgfsetbuttcap%
\pgfsetroundjoin%
\pgfsetlinewidth{0.501875pt}%
\definecolor{currentstroke}{rgb}{0.000000,0.000000,0.000000}%
\pgfsetstrokecolor{currentstroke}%
\pgfsetdash{}{0pt}%
\pgfpathmoveto{\pgfqpoint{1.890399in}{2.474059in}}%
\pgfpathlineto{\pgfqpoint{1.890399in}{2.483413in}}%
\pgfusepath{stroke}%
\end{pgfscope}%
\begin{pgfscope}%
\pgfpathrectangle{\pgfqpoint{0.485787in}{0.226975in}}{\pgfqpoint{3.020670in}{2.494489in}} %
\pgfusepath{clip}%
\pgfsetbuttcap%
\pgfsetroundjoin%
\pgfsetlinewidth{0.501875pt}%
\definecolor{currentstroke}{rgb}{0.000000,0.000000,0.000000}%
\pgfsetstrokecolor{currentstroke}%
\pgfsetdash{}{0pt}%
\pgfpathmoveto{\pgfqpoint{1.920605in}{2.523910in}}%
\pgfpathlineto{\pgfqpoint{1.920605in}{2.532835in}}%
\pgfusepath{stroke}%
\end{pgfscope}%
\begin{pgfscope}%
\pgfpathrectangle{\pgfqpoint{0.485787in}{0.226975in}}{\pgfqpoint{3.020670in}{2.494489in}} %
\pgfusepath{clip}%
\pgfsetbuttcap%
\pgfsetroundjoin%
\pgfsetlinewidth{0.501875pt}%
\definecolor{currentstroke}{rgb}{0.000000,0.000000,0.000000}%
\pgfsetstrokecolor{currentstroke}%
\pgfsetdash{}{0pt}%
\pgfpathmoveto{\pgfqpoint{1.950812in}{2.563567in}}%
\pgfpathlineto{\pgfqpoint{1.950812in}{2.572165in}}%
\pgfusepath{stroke}%
\end{pgfscope}%
\begin{pgfscope}%
\pgfpathrectangle{\pgfqpoint{0.485787in}{0.226975in}}{\pgfqpoint{3.020670in}{2.494489in}} %
\pgfusepath{clip}%
\pgfsetbuttcap%
\pgfsetroundjoin%
\pgfsetlinewidth{0.501875pt}%
\definecolor{currentstroke}{rgb}{0.000000,0.000000,0.000000}%
\pgfsetstrokecolor{currentstroke}%
\pgfsetdash{}{0pt}%
\pgfpathmoveto{\pgfqpoint{1.981019in}{2.594597in}}%
\pgfpathlineto{\pgfqpoint{1.981019in}{2.602947in}}%
\pgfusepath{stroke}%
\end{pgfscope}%
\begin{pgfscope}%
\pgfpathrectangle{\pgfqpoint{0.485787in}{0.226975in}}{\pgfqpoint{3.020670in}{2.494489in}} %
\pgfusepath{clip}%
\pgfsetbuttcap%
\pgfsetroundjoin%
\pgfsetlinewidth{0.501875pt}%
\definecolor{currentstroke}{rgb}{0.000000,0.000000,0.000000}%
\pgfsetstrokecolor{currentstroke}%
\pgfsetdash{}{0pt}%
\pgfpathmoveto{\pgfqpoint{2.011226in}{2.604278in}}%
\pgfpathlineto{\pgfqpoint{2.011226in}{2.612552in}}%
\pgfusepath{stroke}%
\end{pgfscope}%
\begin{pgfscope}%
\pgfpathrectangle{\pgfqpoint{0.485787in}{0.226975in}}{\pgfqpoint{3.020670in}{2.494489in}} %
\pgfusepath{clip}%
\pgfsetbuttcap%
\pgfsetroundjoin%
\pgfsetlinewidth{0.501875pt}%
\definecolor{currentstroke}{rgb}{0.000000,0.000000,0.000000}%
\pgfsetstrokecolor{currentstroke}%
\pgfsetdash{}{0pt}%
\pgfpathmoveto{\pgfqpoint{2.041432in}{2.621050in}}%
\pgfpathlineto{\pgfqpoint{2.041432in}{2.629194in}}%
\pgfusepath{stroke}%
\end{pgfscope}%
\begin{pgfscope}%
\pgfpathrectangle{\pgfqpoint{0.485787in}{0.226975in}}{\pgfqpoint{3.020670in}{2.494489in}} %
\pgfusepath{clip}%
\pgfsetbuttcap%
\pgfsetroundjoin%
\pgfsetlinewidth{0.501875pt}%
\definecolor{currentstroke}{rgb}{0.000000,0.000000,0.000000}%
\pgfsetstrokecolor{currentstroke}%
\pgfsetdash{}{0pt}%
\pgfpathmoveto{\pgfqpoint{2.071639in}{2.605902in}}%
\pgfpathlineto{\pgfqpoint{2.071639in}{2.614163in}}%
\pgfusepath{stroke}%
\end{pgfscope}%
\begin{pgfscope}%
\pgfpathrectangle{\pgfqpoint{0.485787in}{0.226975in}}{\pgfqpoint{3.020670in}{2.494489in}} %
\pgfusepath{clip}%
\pgfsetbuttcap%
\pgfsetroundjoin%
\pgfsetlinewidth{0.501875pt}%
\definecolor{currentstroke}{rgb}{0.000000,0.000000,0.000000}%
\pgfsetstrokecolor{currentstroke}%
\pgfsetdash{}{0pt}%
\pgfpathmoveto{\pgfqpoint{2.101846in}{2.577807in}}%
\pgfpathlineto{\pgfqpoint{2.101846in}{2.586291in}}%
\pgfusepath{stroke}%
\end{pgfscope}%
\begin{pgfscope}%
\pgfpathrectangle{\pgfqpoint{0.485787in}{0.226975in}}{\pgfqpoint{3.020670in}{2.494489in}} %
\pgfusepath{clip}%
\pgfsetbuttcap%
\pgfsetroundjoin%
\pgfsetlinewidth{0.501875pt}%
\definecolor{currentstroke}{rgb}{0.000000,0.000000,0.000000}%
\pgfsetstrokecolor{currentstroke}%
\pgfsetdash{}{0pt}%
\pgfpathmoveto{\pgfqpoint{2.132052in}{2.525723in}}%
\pgfpathlineto{\pgfqpoint{2.132052in}{2.534633in}}%
\pgfusepath{stroke}%
\end{pgfscope}%
\begin{pgfscope}%
\pgfpathrectangle{\pgfqpoint{0.485787in}{0.226975in}}{\pgfqpoint{3.020670in}{2.494489in}} %
\pgfusepath{clip}%
\pgfsetbuttcap%
\pgfsetroundjoin%
\pgfsetlinewidth{0.501875pt}%
\definecolor{currentstroke}{rgb}{0.000000,0.000000,0.000000}%
\pgfsetstrokecolor{currentstroke}%
\pgfsetdash{}{0pt}%
\pgfpathmoveto{\pgfqpoint{2.162259in}{2.472476in}}%
\pgfpathlineto{\pgfqpoint{2.162259in}{2.481845in}}%
\pgfusepath{stroke}%
\end{pgfscope}%
\begin{pgfscope}%
\pgfpathrectangle{\pgfqpoint{0.485787in}{0.226975in}}{\pgfqpoint{3.020670in}{2.494489in}} %
\pgfusepath{clip}%
\pgfsetbuttcap%
\pgfsetroundjoin%
\pgfsetlinewidth{0.501875pt}%
\definecolor{currentstroke}{rgb}{0.000000,0.000000,0.000000}%
\pgfsetstrokecolor{currentstroke}%
\pgfsetdash{}{0pt}%
\pgfpathmoveto{\pgfqpoint{2.192466in}{2.397842in}}%
\pgfpathlineto{\pgfqpoint{2.192466in}{2.407893in}}%
\pgfusepath{stroke}%
\end{pgfscope}%
\begin{pgfscope}%
\pgfpathrectangle{\pgfqpoint{0.485787in}{0.226975in}}{\pgfqpoint{3.020670in}{2.494489in}} %
\pgfusepath{clip}%
\pgfsetbuttcap%
\pgfsetroundjoin%
\pgfsetlinewidth{0.501875pt}%
\definecolor{currentstroke}{rgb}{0.000000,0.000000,0.000000}%
\pgfsetstrokecolor{currentstroke}%
\pgfsetdash{}{0pt}%
\pgfpathmoveto{\pgfqpoint{2.222672in}{2.327758in}}%
\pgfpathlineto{\pgfqpoint{2.222672in}{2.338495in}}%
\pgfusepath{stroke}%
\end{pgfscope}%
\begin{pgfscope}%
\pgfpathrectangle{\pgfqpoint{0.485787in}{0.226975in}}{\pgfqpoint{3.020670in}{2.494489in}} %
\pgfusepath{clip}%
\pgfsetbuttcap%
\pgfsetroundjoin%
\pgfsetlinewidth{0.501875pt}%
\definecolor{currentstroke}{rgb}{0.000000,0.000000,0.000000}%
\pgfsetstrokecolor{currentstroke}%
\pgfsetdash{}{0pt}%
\pgfpathmoveto{\pgfqpoint{2.252879in}{2.242029in}}%
\pgfpathlineto{\pgfqpoint{2.252879in}{2.253668in}}%
\pgfusepath{stroke}%
\end{pgfscope}%
\begin{pgfscope}%
\pgfpathrectangle{\pgfqpoint{0.485787in}{0.226975in}}{\pgfqpoint{3.020670in}{2.494489in}} %
\pgfusepath{clip}%
\pgfsetbuttcap%
\pgfsetroundjoin%
\pgfsetlinewidth{0.501875pt}%
\definecolor{currentstroke}{rgb}{0.000000,0.000000,0.000000}%
\pgfsetstrokecolor{currentstroke}%
\pgfsetdash{}{0pt}%
\pgfpathmoveto{\pgfqpoint{2.283086in}{2.139623in}}%
\pgfpathlineto{\pgfqpoint{2.283086in}{2.152439in}}%
\pgfusepath{stroke}%
\end{pgfscope}%
\begin{pgfscope}%
\pgfpathrectangle{\pgfqpoint{0.485787in}{0.226975in}}{\pgfqpoint{3.020670in}{2.494489in}} %
\pgfusepath{clip}%
\pgfsetbuttcap%
\pgfsetroundjoin%
\pgfsetlinewidth{0.501875pt}%
\definecolor{currentstroke}{rgb}{0.000000,0.000000,0.000000}%
\pgfsetstrokecolor{currentstroke}%
\pgfsetdash{}{0pt}%
\pgfpathmoveto{\pgfqpoint{2.313293in}{2.056597in}}%
\pgfpathlineto{\pgfqpoint{2.313293in}{2.070454in}}%
\pgfusepath{stroke}%
\end{pgfscope}%
\begin{pgfscope}%
\pgfpathrectangle{\pgfqpoint{0.485787in}{0.226975in}}{\pgfqpoint{3.020670in}{2.494489in}} %
\pgfusepath{clip}%
\pgfsetbuttcap%
\pgfsetroundjoin%
\pgfsetlinewidth{0.501875pt}%
\definecolor{currentstroke}{rgb}{0.000000,0.000000,0.000000}%
\pgfsetstrokecolor{currentstroke}%
\pgfsetdash{}{0pt}%
\pgfpathmoveto{\pgfqpoint{2.343499in}{1.934586in}}%
\pgfpathlineto{\pgfqpoint{2.343499in}{1.950125in}}%
\pgfusepath{stroke}%
\end{pgfscope}%
\begin{pgfscope}%
\pgfpathrectangle{\pgfqpoint{0.485787in}{0.226975in}}{\pgfqpoint{3.020670in}{2.494489in}} %
\pgfusepath{clip}%
\pgfsetbuttcap%
\pgfsetroundjoin%
\pgfsetlinewidth{0.501875pt}%
\definecolor{currentstroke}{rgb}{0.000000,0.000000,0.000000}%
\pgfsetstrokecolor{currentstroke}%
\pgfsetdash{}{0pt}%
\pgfpathmoveto{\pgfqpoint{2.373706in}{1.830663in}}%
\pgfpathlineto{\pgfqpoint{2.373706in}{1.847796in}}%
\pgfusepath{stroke}%
\end{pgfscope}%
\begin{pgfscope}%
\pgfpathrectangle{\pgfqpoint{0.485787in}{0.226975in}}{\pgfqpoint{3.020670in}{2.494489in}} %
\pgfusepath{clip}%
\pgfsetbuttcap%
\pgfsetroundjoin%
\pgfsetlinewidth{0.501875pt}%
\definecolor{currentstroke}{rgb}{0.000000,0.000000,0.000000}%
\pgfsetstrokecolor{currentstroke}%
\pgfsetdash{}{0pt}%
\pgfpathmoveto{\pgfqpoint{2.403913in}{1.739052in}}%
\pgfpathlineto{\pgfqpoint{2.403913in}{1.757722in}}%
\pgfusepath{stroke}%
\end{pgfscope}%
\begin{pgfscope}%
\pgfpathrectangle{\pgfqpoint{0.485787in}{0.226975in}}{\pgfqpoint{3.020670in}{2.494489in}} %
\pgfusepath{clip}%
\pgfsetbuttcap%
\pgfsetroundjoin%
\pgfsetlinewidth{0.501875pt}%
\definecolor{currentstroke}{rgb}{0.000000,0.000000,0.000000}%
\pgfsetstrokecolor{currentstroke}%
\pgfsetdash{}{0pt}%
\pgfpathmoveto{\pgfqpoint{2.434119in}{1.659699in}}%
\pgfpathlineto{\pgfqpoint{2.434119in}{1.679811in}}%
\pgfusepath{stroke}%
\end{pgfscope}%
\begin{pgfscope}%
\pgfpathrectangle{\pgfqpoint{0.485787in}{0.226975in}}{\pgfqpoint{3.020670in}{2.494489in}} %
\pgfusepath{clip}%
\pgfsetbuttcap%
\pgfsetroundjoin%
\pgfsetlinewidth{0.501875pt}%
\definecolor{currentstroke}{rgb}{0.000000,0.000000,0.000000}%
\pgfsetstrokecolor{currentstroke}%
\pgfsetdash{}{0pt}%
\pgfpathmoveto{\pgfqpoint{2.464326in}{1.535546in}}%
\pgfpathlineto{\pgfqpoint{2.464326in}{1.558139in}}%
\pgfusepath{stroke}%
\end{pgfscope}%
\begin{pgfscope}%
\pgfpathrectangle{\pgfqpoint{0.485787in}{0.226975in}}{\pgfqpoint{3.020670in}{2.494489in}} %
\pgfusepath{clip}%
\pgfsetbuttcap%
\pgfsetroundjoin%
\pgfsetlinewidth{0.501875pt}%
\definecolor{currentstroke}{rgb}{0.000000,0.000000,0.000000}%
\pgfsetstrokecolor{currentstroke}%
\pgfsetdash{}{0pt}%
\pgfpathmoveto{\pgfqpoint{2.494533in}{1.466018in}}%
\pgfpathlineto{\pgfqpoint{2.494533in}{1.490130in}}%
\pgfusepath{stroke}%
\end{pgfscope}%
\begin{pgfscope}%
\pgfpathrectangle{\pgfqpoint{0.485787in}{0.226975in}}{\pgfqpoint{3.020670in}{2.494489in}} %
\pgfusepath{clip}%
\pgfsetbuttcap%
\pgfsetroundjoin%
\pgfsetlinewidth{0.501875pt}%
\definecolor{currentstroke}{rgb}{0.000000,0.000000,0.000000}%
\pgfsetstrokecolor{currentstroke}%
\pgfsetdash{}{0pt}%
\pgfpathmoveto{\pgfqpoint{2.524739in}{1.386651in}}%
\pgfpathlineto{\pgfqpoint{2.524739in}{1.412620in}}%
\pgfusepath{stroke}%
\end{pgfscope}%
\begin{pgfscope}%
\pgfpathrectangle{\pgfqpoint{0.485787in}{0.226975in}}{\pgfqpoint{3.020670in}{2.494489in}} %
\pgfusepath{clip}%
\pgfsetbuttcap%
\pgfsetroundjoin%
\pgfsetlinewidth{0.501875pt}%
\definecolor{currentstroke}{rgb}{0.000000,0.000000,0.000000}%
\pgfsetstrokecolor{currentstroke}%
\pgfsetdash{}{0pt}%
\pgfpathmoveto{\pgfqpoint{2.554946in}{1.300206in}}%
\pgfpathlineto{\pgfqpoint{2.554946in}{1.328360in}}%
\pgfusepath{stroke}%
\end{pgfscope}%
\begin{pgfscope}%
\pgfpathrectangle{\pgfqpoint{0.485787in}{0.226975in}}{\pgfqpoint{3.020670in}{2.494489in}} %
\pgfusepath{clip}%
\pgfsetbuttcap%
\pgfsetroundjoin%
\pgfsetlinewidth{0.501875pt}%
\definecolor{currentstroke}{rgb}{0.000000,0.000000,0.000000}%
\pgfsetstrokecolor{currentstroke}%
\pgfsetdash{}{0pt}%
\pgfpathmoveto{\pgfqpoint{2.585153in}{1.250447in}}%
\pgfpathlineto{\pgfqpoint{2.585153in}{1.279940in}}%
\pgfusepath{stroke}%
\end{pgfscope}%
\begin{pgfscope}%
\pgfpathrectangle{\pgfqpoint{0.485787in}{0.226975in}}{\pgfqpoint{3.020670in}{2.494489in}} %
\pgfusepath{clip}%
\pgfsetbuttcap%
\pgfsetroundjoin%
\pgfsetlinewidth{0.501875pt}%
\definecolor{currentstroke}{rgb}{0.000000,0.000000,0.000000}%
\pgfsetstrokecolor{currentstroke}%
\pgfsetdash{}{0pt}%
\pgfpathmoveto{\pgfqpoint{2.615359in}{1.185275in}}%
\pgfpathlineto{\pgfqpoint{2.615359in}{1.216616in}}%
\pgfusepath{stroke}%
\end{pgfscope}%
\begin{pgfscope}%
\pgfpathrectangle{\pgfqpoint{0.485787in}{0.226975in}}{\pgfqpoint{3.020670in}{2.494489in}} %
\pgfusepath{clip}%
\pgfsetbuttcap%
\pgfsetroundjoin%
\pgfsetlinewidth{0.501875pt}%
\definecolor{currentstroke}{rgb}{0.000000,0.000000,0.000000}%
\pgfsetstrokecolor{currentstroke}%
\pgfsetdash{}{0pt}%
\pgfpathmoveto{\pgfqpoint{2.645566in}{1.118534in}}%
\pgfpathlineto{\pgfqpoint{2.645566in}{1.151886in}}%
\pgfusepath{stroke}%
\end{pgfscope}%
\begin{pgfscope}%
\pgfpathrectangle{\pgfqpoint{0.485787in}{0.226975in}}{\pgfqpoint{3.020670in}{2.494489in}} %
\pgfusepath{clip}%
\pgfsetbuttcap%
\pgfsetroundjoin%
\pgfsetlinewidth{0.501875pt}%
\definecolor{currentstroke}{rgb}{0.000000,0.000000,0.000000}%
\pgfsetstrokecolor{currentstroke}%
\pgfsetdash{}{0pt}%
\pgfpathmoveto{\pgfqpoint{2.675773in}{1.039742in}}%
\pgfpathlineto{\pgfqpoint{2.675773in}{1.075632in}}%
\pgfusepath{stroke}%
\end{pgfscope}%
\begin{pgfscope}%
\pgfpathrectangle{\pgfqpoint{0.485787in}{0.226975in}}{\pgfqpoint{3.020670in}{2.494489in}} %
\pgfusepath{clip}%
\pgfsetbuttcap%
\pgfsetroundjoin%
\pgfsetlinewidth{0.501875pt}%
\definecolor{currentstroke}{rgb}{0.000000,0.000000,0.000000}%
\pgfsetstrokecolor{currentstroke}%
\pgfsetdash{}{0pt}%
\pgfpathmoveto{\pgfqpoint{2.705980in}{1.027199in}}%
\pgfpathlineto{\pgfqpoint{2.705980in}{1.063510in}}%
\pgfusepath{stroke}%
\end{pgfscope}%
\begin{pgfscope}%
\pgfpathrectangle{\pgfqpoint{0.485787in}{0.226975in}}{\pgfqpoint{3.020670in}{2.494489in}} %
\pgfusepath{clip}%
\pgfsetbuttcap%
\pgfsetroundjoin%
\pgfsetlinewidth{0.501875pt}%
\definecolor{currentstroke}{rgb}{0.000000,0.000000,0.000000}%
\pgfsetstrokecolor{currentstroke}%
\pgfsetdash{}{0pt}%
\pgfpathmoveto{\pgfqpoint{2.736186in}{0.982890in}}%
\pgfpathlineto{\pgfqpoint{2.736186in}{1.020729in}}%
\pgfusepath{stroke}%
\end{pgfscope}%
\begin{pgfscope}%
\pgfpathrectangle{\pgfqpoint{0.485787in}{0.226975in}}{\pgfqpoint{3.020670in}{2.494489in}} %
\pgfusepath{clip}%
\pgfsetbuttcap%
\pgfsetroundjoin%
\pgfsetlinewidth{0.501875pt}%
\definecolor{currentstroke}{rgb}{0.000000,0.000000,0.000000}%
\pgfsetstrokecolor{currentstroke}%
\pgfsetdash{}{0pt}%
\pgfpathmoveto{\pgfqpoint{2.766393in}{0.947983in}}%
\pgfpathlineto{\pgfqpoint{2.766393in}{0.987069in}}%
\pgfusepath{stroke}%
\end{pgfscope}%
\begin{pgfscope}%
\pgfpathrectangle{\pgfqpoint{0.485787in}{0.226975in}}{\pgfqpoint{3.020670in}{2.494489in}} %
\pgfusepath{clip}%
\pgfsetbuttcap%
\pgfsetroundjoin%
\pgfsetlinewidth{0.501875pt}%
\definecolor{currentstroke}{rgb}{0.000000,0.000000,0.000000}%
\pgfsetstrokecolor{currentstroke}%
\pgfsetdash{}{0pt}%
\pgfpathmoveto{\pgfqpoint{2.796600in}{0.914563in}}%
\pgfpathlineto{\pgfqpoint{2.796600in}{0.954881in}}%
\pgfusepath{stroke}%
\end{pgfscope}%
\begin{pgfscope}%
\pgfpathrectangle{\pgfqpoint{0.485787in}{0.226975in}}{\pgfqpoint{3.020670in}{2.494489in}} %
\pgfusepath{clip}%
\pgfsetbuttcap%
\pgfsetroundjoin%
\pgfsetlinewidth{0.501875pt}%
\definecolor{currentstroke}{rgb}{0.000000,0.000000,0.000000}%
\pgfsetstrokecolor{currentstroke}%
\pgfsetdash{}{0pt}%
\pgfpathmoveto{\pgfqpoint{2.826806in}{0.876389in}}%
\pgfpathlineto{\pgfqpoint{2.826806in}{0.918161in}}%
\pgfusepath{stroke}%
\end{pgfscope}%
\begin{pgfscope}%
\pgfpathrectangle{\pgfqpoint{0.485787in}{0.226975in}}{\pgfqpoint{3.020670in}{2.494489in}} %
\pgfusepath{clip}%
\pgfsetbuttcap%
\pgfsetroundjoin%
\pgfsetlinewidth{0.501875pt}%
\definecolor{currentstroke}{rgb}{0.000000,0.000000,0.000000}%
\pgfsetstrokecolor{currentstroke}%
\pgfsetdash{}{0pt}%
\pgfpathmoveto{\pgfqpoint{2.857013in}{0.846966in}}%
\pgfpathlineto{\pgfqpoint{2.857013in}{0.889894in}}%
\pgfusepath{stroke}%
\end{pgfscope}%
\begin{pgfscope}%
\pgfpathrectangle{\pgfqpoint{0.485787in}{0.226975in}}{\pgfqpoint{3.020670in}{2.494489in}} %
\pgfusepath{clip}%
\pgfsetbuttcap%
\pgfsetroundjoin%
\pgfsetlinewidth{0.501875pt}%
\definecolor{currentstroke}{rgb}{0.000000,0.000000,0.000000}%
\pgfsetstrokecolor{currentstroke}%
\pgfsetdash{}{0pt}%
\pgfpathmoveto{\pgfqpoint{2.887220in}{0.814882in}}%
\pgfpathlineto{\pgfqpoint{2.887220in}{0.859106in}}%
\pgfusepath{stroke}%
\end{pgfscope}%
\begin{pgfscope}%
\pgfpathrectangle{\pgfqpoint{0.485787in}{0.226975in}}{\pgfqpoint{3.020670in}{2.494489in}} %
\pgfusepath{clip}%
\pgfsetbuttcap%
\pgfsetroundjoin%
\pgfsetlinewidth{0.501875pt}%
\definecolor{currentstroke}{rgb}{0.000000,0.000000,0.000000}%
\pgfsetstrokecolor{currentstroke}%
\pgfsetdash{}{0pt}%
\pgfpathmoveto{\pgfqpoint{2.917426in}{0.793682in}}%
\pgfpathlineto{\pgfqpoint{2.917426in}{0.838783in}}%
\pgfusepath{stroke}%
\end{pgfscope}%
\begin{pgfscope}%
\pgfpathrectangle{\pgfqpoint{0.485787in}{0.226975in}}{\pgfqpoint{3.020670in}{2.494489in}} %
\pgfusepath{clip}%
\pgfsetbuttcap%
\pgfsetroundjoin%
\pgfsetlinewidth{0.501875pt}%
\definecolor{currentstroke}{rgb}{0.000000,0.000000,0.000000}%
\pgfsetstrokecolor{currentstroke}%
\pgfsetdash{}{0pt}%
\pgfpathmoveto{\pgfqpoint{2.947633in}{0.735584in}}%
\pgfpathlineto{\pgfqpoint{2.947633in}{0.783177in}}%
\pgfusepath{stroke}%
\end{pgfscope}%
\begin{pgfscope}%
\pgfpathrectangle{\pgfqpoint{0.485787in}{0.226975in}}{\pgfqpoint{3.020670in}{2.494489in}} %
\pgfusepath{clip}%
\pgfsetbuttcap%
\pgfsetroundjoin%
\pgfsetlinewidth{0.501875pt}%
\definecolor{currentstroke}{rgb}{0.000000,0.000000,0.000000}%
\pgfsetstrokecolor{currentstroke}%
\pgfsetdash{}{0pt}%
\pgfpathmoveto{\pgfqpoint{2.977840in}{0.741036in}}%
\pgfpathlineto{\pgfqpoint{2.977840in}{0.788390in}}%
\pgfusepath{stroke}%
\end{pgfscope}%
\begin{pgfscope}%
\pgfpathrectangle{\pgfqpoint{0.485787in}{0.226975in}}{\pgfqpoint{3.020670in}{2.494489in}} %
\pgfusepath{clip}%
\pgfsetbuttcap%
\pgfsetroundjoin%
\pgfsetlinewidth{0.501875pt}%
\definecolor{currentstroke}{rgb}{0.000000,0.000000,0.000000}%
\pgfsetstrokecolor{currentstroke}%
\pgfsetdash{}{0pt}%
\pgfpathmoveto{\pgfqpoint{3.008047in}{0.736679in}}%
\pgfpathlineto{\pgfqpoint{3.008047in}{0.784224in}}%
\pgfusepath{stroke}%
\end{pgfscope}%
\begin{pgfscope}%
\pgfpathrectangle{\pgfqpoint{0.485787in}{0.226975in}}{\pgfqpoint{3.020670in}{2.494489in}} %
\pgfusepath{clip}%
\pgfsetbuttcap%
\pgfsetroundjoin%
\pgfsetlinewidth{0.501875pt}%
\definecolor{currentstroke}{rgb}{0.000000,0.000000,0.000000}%
\pgfsetstrokecolor{currentstroke}%
\pgfsetdash{}{0pt}%
\pgfpathmoveto{\pgfqpoint{3.038253in}{0.709763in}}%
\pgfpathlineto{\pgfqpoint{3.038253in}{0.758507in}}%
\pgfusepath{stroke}%
\end{pgfscope}%
\begin{pgfscope}%
\pgfpathrectangle{\pgfqpoint{0.485787in}{0.226975in}}{\pgfqpoint{3.020670in}{2.494489in}} %
\pgfusepath{clip}%
\pgfsetbuttcap%
\pgfsetroundjoin%
\pgfsetlinewidth{0.501875pt}%
\definecolor{currentstroke}{rgb}{0.000000,0.000000,0.000000}%
\pgfsetstrokecolor{currentstroke}%
\pgfsetdash{}{0pt}%
\pgfpathmoveto{\pgfqpoint{3.068460in}{0.675321in}}%
\pgfpathlineto{\pgfqpoint{3.068460in}{0.725642in}}%
\pgfusepath{stroke}%
\end{pgfscope}%
\begin{pgfscope}%
\pgfpathrectangle{\pgfqpoint{0.485787in}{0.226975in}}{\pgfqpoint{3.020670in}{2.494489in}} %
\pgfusepath{clip}%
\pgfsetbuttcap%
\pgfsetroundjoin%
\pgfsetlinewidth{0.501875pt}%
\definecolor{currentstroke}{rgb}{0.000000,0.000000,0.000000}%
\pgfsetstrokecolor{currentstroke}%
\pgfsetdash{}{0pt}%
\pgfpathmoveto{\pgfqpoint{3.098667in}{0.691036in}}%
\pgfpathlineto{\pgfqpoint{3.098667in}{0.740630in}}%
\pgfusepath{stroke}%
\end{pgfscope}%
\begin{pgfscope}%
\pgfpathrectangle{\pgfqpoint{0.485787in}{0.226975in}}{\pgfqpoint{3.020670in}{2.494489in}} %
\pgfusepath{clip}%
\pgfsetbuttcap%
\pgfsetroundjoin%
\pgfsetlinewidth{0.501875pt}%
\definecolor{currentstroke}{rgb}{0.000000,0.000000,0.000000}%
\pgfsetstrokecolor{currentstroke}%
\pgfsetdash{}{0pt}%
\pgfpathmoveto{\pgfqpoint{3.128873in}{0.681420in}}%
\pgfpathlineto{\pgfqpoint{3.128873in}{0.731458in}}%
\pgfusepath{stroke}%
\end{pgfscope}%
\begin{pgfscope}%
\pgfpathrectangle{\pgfqpoint{0.485787in}{0.226975in}}{\pgfqpoint{3.020670in}{2.494489in}} %
\pgfusepath{clip}%
\pgfsetbuttcap%
\pgfsetroundjoin%
\pgfsetlinewidth{0.501875pt}%
\definecolor{currentstroke}{rgb}{0.000000,0.000000,0.000000}%
\pgfsetstrokecolor{currentstroke}%
\pgfsetdash{}{0pt}%
\pgfpathmoveto{\pgfqpoint{3.159080in}{0.661654in}}%
\pgfpathlineto{\pgfqpoint{3.159080in}{0.712614in}}%
\pgfusepath{stroke}%
\end{pgfscope}%
\begin{pgfscope}%
\pgfpathrectangle{\pgfqpoint{0.485787in}{0.226975in}}{\pgfqpoint{3.020670in}{2.494489in}} %
\pgfusepath{clip}%
\pgfsetbuttcap%
\pgfsetroundjoin%
\pgfsetlinewidth{0.501875pt}%
\definecolor{currentstroke}{rgb}{0.000000,0.000000,0.000000}%
\pgfsetstrokecolor{currentstroke}%
\pgfsetdash{}{0pt}%
\pgfpathmoveto{\pgfqpoint{3.189287in}{0.637191in}}%
\pgfpathlineto{\pgfqpoint{3.189287in}{0.689316in}}%
\pgfusepath{stroke}%
\end{pgfscope}%
\begin{pgfscope}%
\pgfpathrectangle{\pgfqpoint{0.485787in}{0.226975in}}{\pgfqpoint{3.020670in}{2.494489in}} %
\pgfusepath{clip}%
\pgfsetbuttcap%
\pgfsetroundjoin%
\pgfsetlinewidth{0.501875pt}%
\definecolor{currentstroke}{rgb}{0.000000,0.000000,0.000000}%
\pgfsetstrokecolor{currentstroke}%
\pgfsetdash{}{0pt}%
\pgfpathmoveto{\pgfqpoint{3.219493in}{0.642435in}}%
\pgfpathlineto{\pgfqpoint{3.219493in}{0.694308in}}%
\pgfusepath{stroke}%
\end{pgfscope}%
\begin{pgfscope}%
\pgfpathrectangle{\pgfqpoint{0.485787in}{0.226975in}}{\pgfqpoint{3.020670in}{2.494489in}} %
\pgfusepath{clip}%
\pgfsetbuttcap%
\pgfsetroundjoin%
\pgfsetlinewidth{0.501875pt}%
\definecolor{currentstroke}{rgb}{0.000000,0.000000,0.000000}%
\pgfsetstrokecolor{currentstroke}%
\pgfsetdash{}{0pt}%
\pgfpathmoveto{\pgfqpoint{3.249700in}{0.619785in}}%
\pgfpathlineto{\pgfqpoint{3.249700in}{0.672754in}}%
\pgfusepath{stroke}%
\end{pgfscope}%
\begin{pgfscope}%
\pgfpathrectangle{\pgfqpoint{0.485787in}{0.226975in}}{\pgfqpoint{3.020670in}{2.494489in}} %
\pgfusepath{clip}%
\pgfsetbuttcap%
\pgfsetroundjoin%
\pgfsetlinewidth{0.501875pt}%
\definecolor{currentstroke}{rgb}{0.000000,0.000000,0.000000}%
\pgfsetstrokecolor{currentstroke}%
\pgfsetdash{}{0pt}%
\pgfpathmoveto{\pgfqpoint{3.279907in}{0.631896in}}%
\pgfpathlineto{\pgfqpoint{3.279907in}{0.684276in}}%
\pgfusepath{stroke}%
\end{pgfscope}%
\begin{pgfscope}%
\pgfpathrectangle{\pgfqpoint{0.485787in}{0.226975in}}{\pgfqpoint{3.020670in}{2.494489in}} %
\pgfusepath{clip}%
\pgfsetbuttcap%
\pgfsetroundjoin%
\pgfsetlinewidth{0.501875pt}%
\definecolor{currentstroke}{rgb}{0.000000,0.000000,0.000000}%
\pgfsetstrokecolor{currentstroke}%
\pgfsetdash{}{0pt}%
\pgfpathmoveto{\pgfqpoint{3.310114in}{0.571390in}}%
\pgfpathlineto{\pgfqpoint{3.310114in}{0.626777in}}%
\pgfusepath{stroke}%
\end{pgfscope}%
\begin{pgfscope}%
\pgfpathrectangle{\pgfqpoint{0.485787in}{0.226975in}}{\pgfqpoint{3.020670in}{2.494489in}} %
\pgfusepath{clip}%
\pgfsetbuttcap%
\pgfsetroundjoin%
\pgfsetlinewidth{0.501875pt}%
\definecolor{currentstroke}{rgb}{0.000000,0.000000,0.000000}%
\pgfsetstrokecolor{currentstroke}%
\pgfsetdash{}{0pt}%
\pgfpathmoveto{\pgfqpoint{3.340320in}{0.574361in}}%
\pgfpathlineto{\pgfqpoint{3.340320in}{0.629596in}}%
\pgfusepath{stroke}%
\end{pgfscope}%
\begin{pgfscope}%
\pgfpathrectangle{\pgfqpoint{0.485787in}{0.226975in}}{\pgfqpoint{3.020670in}{2.494489in}} %
\pgfusepath{clip}%
\pgfsetbuttcap%
\pgfsetroundjoin%
\pgfsetlinewidth{0.501875pt}%
\definecolor{currentstroke}{rgb}{0.000000,0.000000,0.000000}%
\pgfsetstrokecolor{currentstroke}%
\pgfsetdash{}{0pt}%
\pgfpathmoveto{\pgfqpoint{3.370527in}{0.590414in}}%
\pgfpathlineto{\pgfqpoint{3.370527in}{0.644838in}}%
\pgfusepath{stroke}%
\end{pgfscope}%
\begin{pgfscope}%
\pgfpathrectangle{\pgfqpoint{0.485787in}{0.226975in}}{\pgfqpoint{3.020670in}{2.494489in}} %
\pgfusepath{clip}%
\pgfsetbuttcap%
\pgfsetroundjoin%
\pgfsetlinewidth{0.501875pt}%
\definecolor{currentstroke}{rgb}{0.000000,0.000000,0.000000}%
\pgfsetstrokecolor{currentstroke}%
\pgfsetdash{}{0pt}%
\pgfpathmoveto{\pgfqpoint{3.400734in}{0.591850in}}%
\pgfpathlineto{\pgfqpoint{3.400734in}{0.646201in}}%
\pgfusepath{stroke}%
\end{pgfscope}%
\begin{pgfscope}%
\pgfpathrectangle{\pgfqpoint{0.485787in}{0.226975in}}{\pgfqpoint{3.020670in}{2.494489in}} %
\pgfusepath{clip}%
\pgfsetbuttcap%
\pgfsetroundjoin%
\pgfsetlinewidth{0.501875pt}%
\definecolor{currentstroke}{rgb}{0.000000,0.000000,0.000000}%
\pgfsetstrokecolor{currentstroke}%
\pgfsetdash{}{0pt}%
\pgfpathmoveto{\pgfqpoint{3.430940in}{0.569898in}}%
\pgfpathlineto{\pgfqpoint{3.430940in}{0.625361in}}%
\pgfusepath{stroke}%
\end{pgfscope}%
\begin{pgfscope}%
\pgfpathrectangle{\pgfqpoint{0.485787in}{0.226975in}}{\pgfqpoint{3.020670in}{2.494489in}} %
\pgfusepath{clip}%
\pgfsetbuttcap%
\pgfsetroundjoin%
\pgfsetlinewidth{0.501875pt}%
\definecolor{currentstroke}{rgb}{0.000000,0.000000,0.000000}%
\pgfsetstrokecolor{currentstroke}%
\pgfsetdash{}{0pt}%
\pgfpathmoveto{\pgfqpoint{3.461147in}{0.565398in}}%
\pgfpathlineto{\pgfqpoint{3.461147in}{0.621091in}}%
\pgfusepath{stroke}%
\end{pgfscope}%
\begin{pgfscope}%
\pgfpathrectangle{\pgfqpoint{0.485787in}{0.226975in}}{\pgfqpoint{3.020670in}{2.494489in}} %
\pgfusepath{clip}%
\pgfsetbuttcap%
\pgfsetroundjoin%
\pgfsetlinewidth{0.501875pt}%
\definecolor{currentstroke}{rgb}{0.000000,0.000000,0.000000}%
\pgfsetstrokecolor{currentstroke}%
\pgfsetdash{}{0pt}%
\pgfpathmoveto{\pgfqpoint{3.491354in}{0.603200in}}%
\pgfpathlineto{\pgfqpoint{3.491354in}{0.656985in}}%
\pgfusepath{stroke}%
\end{pgfscope}%
\begin{pgfscope}%
\pgfsetbuttcap%
\pgfsetroundjoin%
\definecolor{currentfill}{rgb}{0.000000,0.000000,0.000000}%
\pgfsetfillcolor{currentfill}%
\pgfsetlinewidth{1.003750pt}%
\definecolor{currentstroke}{rgb}{0.000000,0.000000,0.000000}%
\pgfsetstrokecolor{currentstroke}%
\pgfsetdash{}{0pt}%
\pgfsys@defobject{currentmarker}{\pgfqpoint{-0.006944in}{-0.006944in}}{\pgfqpoint{0.006944in}{0.006944in}}{%
\pgfpathmoveto{\pgfqpoint{0.000000in}{-0.006944in}}%
\pgfpathcurveto{\pgfqpoint{0.001842in}{-0.006944in}}{\pgfqpoint{0.003608in}{-0.006213in}}{\pgfqpoint{0.004910in}{-0.004910in}}%
\pgfpathcurveto{\pgfqpoint{0.006213in}{-0.003608in}}{\pgfqpoint{0.006944in}{-0.001842in}}{\pgfqpoint{0.006944in}{0.000000in}}%
\pgfpathcurveto{\pgfqpoint{0.006944in}{0.001842in}}{\pgfqpoint{0.006213in}{0.003608in}}{\pgfqpoint{0.004910in}{0.004910in}}%
\pgfpathcurveto{\pgfqpoint{0.003608in}{0.006213in}}{\pgfqpoint{0.001842in}{0.006944in}}{\pgfqpoint{0.000000in}{0.006944in}}%
\pgfpathcurveto{\pgfqpoint{-0.001842in}{0.006944in}}{\pgfqpoint{-0.003608in}{0.006213in}}{\pgfqpoint{-0.004910in}{0.004910in}}%
\pgfpathcurveto{\pgfqpoint{-0.006213in}{0.003608in}}{\pgfqpoint{-0.006944in}{0.001842in}}{\pgfqpoint{-0.006944in}{0.000000in}}%
\pgfpathcurveto{\pgfqpoint{-0.006944in}{-0.001842in}}{\pgfqpoint{-0.006213in}{-0.003608in}}{\pgfqpoint{-0.004910in}{-0.004910in}}%
\pgfpathcurveto{\pgfqpoint{-0.003608in}{-0.006213in}}{\pgfqpoint{-0.001842in}{-0.006944in}}{\pgfqpoint{0.000000in}{-0.006944in}}%
\pgfpathclose%
\pgfusepath{stroke,fill}%
}%
\begin{pgfscope}%
\pgfsys@transformshift{0.500891in}{0.777295in}%
\pgfsys@useobject{currentmarker}{}%
\end{pgfscope}%
\begin{pgfscope}%
\pgfsys@transformshift{0.531097in}{0.728483in}%
\pgfsys@useobject{currentmarker}{}%
\end{pgfscope}%
\begin{pgfscope}%
\pgfsys@transformshift{0.561304in}{0.797621in}%
\pgfsys@useobject{currentmarker}{}%
\end{pgfscope}%
\begin{pgfscope}%
\pgfsys@transformshift{0.591511in}{0.780394in}%
\pgfsys@useobject{currentmarker}{}%
\end{pgfscope}%
\begin{pgfscope}%
\pgfsys@transformshift{0.621717in}{0.775219in}%
\pgfsys@useobject{currentmarker}{}%
\end{pgfscope}%
\begin{pgfscope}%
\pgfsys@transformshift{0.651924in}{0.795623in}%
\pgfsys@useobject{currentmarker}{}%
\end{pgfscope}%
\begin{pgfscope}%
\pgfsys@transformshift{0.682131in}{0.795623in}%
\pgfsys@useobject{currentmarker}{}%
\end{pgfscope}%
\begin{pgfscope}%
\pgfsys@transformshift{0.712338in}{0.808476in}%
\pgfsys@useobject{currentmarker}{}%
\end{pgfscope}%
\begin{pgfscope}%
\pgfsys@transformshift{0.742544in}{0.748552in}%
\pgfsys@useobject{currentmarker}{}%
\end{pgfscope}%
\begin{pgfscope}%
\pgfsys@transformshift{0.772751in}{0.792612in}%
\pgfsys@useobject{currentmarker}{}%
\end{pgfscope}%
\begin{pgfscope}%
\pgfsys@transformshift{0.802958in}{0.799611in}%
\pgfsys@useobject{currentmarker}{}%
\end{pgfscope}%
\begin{pgfscope}%
\pgfsys@transformshift{0.833164in}{0.781423in}%
\pgfsys@useobject{currentmarker}{}%
\end{pgfscope}%
\begin{pgfscope}%
\pgfsys@transformshift{0.863371in}{0.846168in}%
\pgfsys@useobject{currentmarker}{}%
\end{pgfscope}%
\begin{pgfscope}%
\pgfsys@transformshift{0.893578in}{0.820069in}%
\pgfsys@useobject{currentmarker}{}%
\end{pgfscope}%
\begin{pgfscope}%
\pgfsys@transformshift{0.923784in}{0.838842in}%
\pgfsys@useobject{currentmarker}{}%
\end{pgfscope}%
\begin{pgfscope}%
\pgfsys@transformshift{0.953991in}{0.813337in}%
\pgfsys@useobject{currentmarker}{}%
\end{pgfscope}%
\begin{pgfscope}%
\pgfsys@transformshift{0.984198in}{0.831412in}%
\pgfsys@useobject{currentmarker}{}%
\end{pgfscope}%
\begin{pgfscope}%
\pgfsys@transformshift{1.014405in}{0.847984in}%
\pgfsys@useobject{currentmarker}{}%
\end{pgfscope}%
\begin{pgfscope}%
\pgfsys@transformshift{1.044611in}{0.864052in}%
\pgfsys@useobject{currentmarker}{}%
\end{pgfscope}%
\begin{pgfscope}%
\pgfsys@transformshift{1.074818in}{0.916728in}%
\pgfsys@useobject{currentmarker}{}%
\end{pgfscope}%
\begin{pgfscope}%
\pgfsys@transformshift{1.105025in}{0.914336in}%
\pgfsys@useobject{currentmarker}{}%
\end{pgfscope}%
\begin{pgfscope}%
\pgfsys@transformshift{1.135231in}{0.924627in}%
\pgfsys@useobject{currentmarker}{}%
\end{pgfscope}%
\begin{pgfscope}%
\pgfsys@transformshift{1.165438in}{0.929310in}%
\pgfsys@useobject{currentmarker}{}%
\end{pgfscope}%
\begin{pgfscope}%
\pgfsys@transformshift{1.195645in}{0.912734in}%
\pgfsys@useobject{currentmarker}{}%
\end{pgfscope}%
\begin{pgfscope}%
\pgfsys@transformshift{1.225851in}{0.996362in}%
\pgfsys@useobject{currentmarker}{}%
\end{pgfscope}%
\begin{pgfscope}%
\pgfsys@transformshift{1.256058in}{0.965353in}%
\pgfsys@useobject{currentmarker}{}%
\end{pgfscope}%
\begin{pgfscope}%
\pgfsys@transformshift{1.286265in}{1.044105in}%
\pgfsys@useobject{currentmarker}{}%
\end{pgfscope}%
\begin{pgfscope}%
\pgfsys@transformshift{1.316472in}{1.041598in}%
\pgfsys@useobject{currentmarker}{}%
\end{pgfscope}%
\begin{pgfscope}%
\pgfsys@transformshift{1.346678in}{1.089614in}%
\pgfsys@useobject{currentmarker}{}%
\end{pgfscope}%
\begin{pgfscope}%
\pgfsys@transformshift{1.376885in}{1.129920in}%
\pgfsys@useobject{currentmarker}{}%
\end{pgfscope}%
\begin{pgfscope}%
\pgfsys@transformshift{1.407092in}{1.188246in}%
\pgfsys@useobject{currentmarker}{}%
\end{pgfscope}%
\begin{pgfscope}%
\pgfsys@transformshift{1.437298in}{1.246342in}%
\pgfsys@useobject{currentmarker}{}%
\end{pgfscope}%
\begin{pgfscope}%
\pgfsys@transformshift{1.467505in}{1.312781in}%
\pgfsys@useobject{currentmarker}{}%
\end{pgfscope}%
\begin{pgfscope}%
\pgfsys@transformshift{1.497712in}{1.347706in}%
\pgfsys@useobject{currentmarker}{}%
\end{pgfscope}%
\begin{pgfscope}%
\pgfsys@transformshift{1.527918in}{1.446623in}%
\pgfsys@useobject{currentmarker}{}%
\end{pgfscope}%
\begin{pgfscope}%
\pgfsys@transformshift{1.558125in}{1.520602in}%
\pgfsys@useobject{currentmarker}{}%
\end{pgfscope}%
\begin{pgfscope}%
\pgfsys@transformshift{1.588332in}{1.569070in}%
\pgfsys@useobject{currentmarker}{}%
\end{pgfscope}%
\begin{pgfscope}%
\pgfsys@transformshift{1.618538in}{1.662238in}%
\pgfsys@useobject{currentmarker}{}%
\end{pgfscope}%
\begin{pgfscope}%
\pgfsys@transformshift{1.648745in}{1.761901in}%
\pgfsys@useobject{currentmarker}{}%
\end{pgfscope}%
\begin{pgfscope}%
\pgfsys@transformshift{1.678952in}{1.861661in}%
\pgfsys@useobject{currentmarker}{}%
\end{pgfscope}%
\begin{pgfscope}%
\pgfsys@transformshift{1.709159in}{1.957448in}%
\pgfsys@useobject{currentmarker}{}%
\end{pgfscope}%
\begin{pgfscope}%
\pgfsys@transformshift{1.739365in}{2.041924in}%
\pgfsys@useobject{currentmarker}{}%
\end{pgfscope}%
\begin{pgfscope}%
\pgfsys@transformshift{1.769572in}{2.152979in}%
\pgfsys@useobject{currentmarker}{}%
\end{pgfscope}%
\begin{pgfscope}%
\pgfsys@transformshift{1.799779in}{2.233096in}%
\pgfsys@useobject{currentmarker}{}%
\end{pgfscope}%
\begin{pgfscope}%
\pgfsys@transformshift{1.829985in}{2.317799in}%
\pgfsys@useobject{currentmarker}{}%
\end{pgfscope}%
\begin{pgfscope}%
\pgfsys@transformshift{1.860192in}{2.397873in}%
\pgfsys@useobject{currentmarker}{}%
\end{pgfscope}%
\begin{pgfscope}%
\pgfsys@transformshift{1.890399in}{2.478736in}%
\pgfsys@useobject{currentmarker}{}%
\end{pgfscope}%
\begin{pgfscope}%
\pgfsys@transformshift{1.920605in}{2.528372in}%
\pgfsys@useobject{currentmarker}{}%
\end{pgfscope}%
\begin{pgfscope}%
\pgfsys@transformshift{1.950812in}{2.567866in}%
\pgfsys@useobject{currentmarker}{}%
\end{pgfscope}%
\begin{pgfscope}%
\pgfsys@transformshift{1.981019in}{2.598772in}%
\pgfsys@useobject{currentmarker}{}%
\end{pgfscope}%
\begin{pgfscope}%
\pgfsys@transformshift{2.011226in}{2.608415in}%
\pgfsys@useobject{currentmarker}{}%
\end{pgfscope}%
\begin{pgfscope}%
\pgfsys@transformshift{2.041432in}{2.625122in}%
\pgfsys@useobject{currentmarker}{}%
\end{pgfscope}%
\begin{pgfscope}%
\pgfsys@transformshift{2.071639in}{2.610032in}%
\pgfsys@useobject{currentmarker}{}%
\end{pgfscope}%
\begin{pgfscope}%
\pgfsys@transformshift{2.101846in}{2.582049in}%
\pgfsys@useobject{currentmarker}{}%
\end{pgfscope}%
\begin{pgfscope}%
\pgfsys@transformshift{2.132052in}{2.530178in}%
\pgfsys@useobject{currentmarker}{}%
\end{pgfscope}%
\begin{pgfscope}%
\pgfsys@transformshift{2.162259in}{2.477160in}%
\pgfsys@useobject{currentmarker}{}%
\end{pgfscope}%
\begin{pgfscope}%
\pgfsys@transformshift{2.192466in}{2.402868in}%
\pgfsys@useobject{currentmarker}{}%
\end{pgfscope}%
\begin{pgfscope}%
\pgfsys@transformshift{2.222672in}{2.333126in}%
\pgfsys@useobject{currentmarker}{}%
\end{pgfscope}%
\begin{pgfscope}%
\pgfsys@transformshift{2.252879in}{2.247848in}%
\pgfsys@useobject{currentmarker}{}%
\end{pgfscope}%
\begin{pgfscope}%
\pgfsys@transformshift{2.283086in}{2.146031in}%
\pgfsys@useobject{currentmarker}{}%
\end{pgfscope}%
\begin{pgfscope}%
\pgfsys@transformshift{2.313293in}{2.063525in}%
\pgfsys@useobject{currentmarker}{}%
\end{pgfscope}%
\begin{pgfscope}%
\pgfsys@transformshift{2.343499in}{1.942355in}%
\pgfsys@useobject{currentmarker}{}%
\end{pgfscope}%
\begin{pgfscope}%
\pgfsys@transformshift{2.373706in}{1.839230in}%
\pgfsys@useobject{currentmarker}{}%
\end{pgfscope}%
\begin{pgfscope}%
\pgfsys@transformshift{2.403913in}{1.748387in}%
\pgfsys@useobject{currentmarker}{}%
\end{pgfscope}%
\begin{pgfscope}%
\pgfsys@transformshift{2.434119in}{1.669755in}%
\pgfsys@useobject{currentmarker}{}%
\end{pgfscope}%
\begin{pgfscope}%
\pgfsys@transformshift{2.464326in}{1.546843in}%
\pgfsys@useobject{currentmarker}{}%
\end{pgfscope}%
\begin{pgfscope}%
\pgfsys@transformshift{2.494533in}{1.478074in}%
\pgfsys@useobject{currentmarker}{}%
\end{pgfscope}%
\begin{pgfscope}%
\pgfsys@transformshift{2.524739in}{1.399635in}%
\pgfsys@useobject{currentmarker}{}%
\end{pgfscope}%
\begin{pgfscope}%
\pgfsys@transformshift{2.554946in}{1.314283in}%
\pgfsys@useobject{currentmarker}{}%
\end{pgfscope}%
\begin{pgfscope}%
\pgfsys@transformshift{2.585153in}{1.265194in}%
\pgfsys@useobject{currentmarker}{}%
\end{pgfscope}%
\begin{pgfscope}%
\pgfsys@transformshift{2.615359in}{1.200945in}%
\pgfsys@useobject{currentmarker}{}%
\end{pgfscope}%
\begin{pgfscope}%
\pgfsys@transformshift{2.645566in}{1.135210in}%
\pgfsys@useobject{currentmarker}{}%
\end{pgfscope}%
\begin{pgfscope}%
\pgfsys@transformshift{2.675773in}{1.057687in}%
\pgfsys@useobject{currentmarker}{}%
\end{pgfscope}%
\begin{pgfscope}%
\pgfsys@transformshift{2.705980in}{1.045354in}%
\pgfsys@useobject{currentmarker}{}%
\end{pgfscope}%
\begin{pgfscope}%
\pgfsys@transformshift{2.736186in}{1.001809in}%
\pgfsys@useobject{currentmarker}{}%
\end{pgfscope}%
\begin{pgfscope}%
\pgfsys@transformshift{2.766393in}{0.967526in}%
\pgfsys@useobject{currentmarker}{}%
\end{pgfscope}%
\begin{pgfscope}%
\pgfsys@transformshift{2.796600in}{0.934722in}%
\pgfsys@useobject{currentmarker}{}%
\end{pgfscope}%
\begin{pgfscope}%
\pgfsys@transformshift{2.826806in}{0.897275in}%
\pgfsys@useobject{currentmarker}{}%
\end{pgfscope}%
\begin{pgfscope}%
\pgfsys@transformshift{2.857013in}{0.868430in}%
\pgfsys@useobject{currentmarker}{}%
\end{pgfscope}%
\begin{pgfscope}%
\pgfsys@transformshift{2.887220in}{0.836994in}%
\pgfsys@useobject{currentmarker}{}%
\end{pgfscope}%
\begin{pgfscope}%
\pgfsys@transformshift{2.917426in}{0.816233in}%
\pgfsys@useobject{currentmarker}{}%
\end{pgfscope}%
\begin{pgfscope}%
\pgfsys@transformshift{2.947633in}{0.759381in}%
\pgfsys@useobject{currentmarker}{}%
\end{pgfscope}%
\begin{pgfscope}%
\pgfsys@transformshift{2.977840in}{0.764713in}%
\pgfsys@useobject{currentmarker}{}%
\end{pgfscope}%
\begin{pgfscope}%
\pgfsys@transformshift{3.008047in}{0.760451in}%
\pgfsys@useobject{currentmarker}{}%
\end{pgfscope}%
\begin{pgfscope}%
\pgfsys@transformshift{3.038253in}{0.734135in}%
\pgfsys@useobject{currentmarker}{}%
\end{pgfscope}%
\begin{pgfscope}%
\pgfsys@transformshift{3.068460in}{0.700481in}%
\pgfsys@useobject{currentmarker}{}%
\end{pgfscope}%
\begin{pgfscope}%
\pgfsys@transformshift{3.098667in}{0.715833in}%
\pgfsys@useobject{currentmarker}{}%
\end{pgfscope}%
\begin{pgfscope}%
\pgfsys@transformshift{3.128873in}{0.706439in}%
\pgfsys@useobject{currentmarker}{}%
\end{pgfscope}%
\begin{pgfscope}%
\pgfsys@transformshift{3.159080in}{0.687134in}%
\pgfsys@useobject{currentmarker}{}%
\end{pgfscope}%
\begin{pgfscope}%
\pgfsys@transformshift{3.189287in}{0.663254in}%
\pgfsys@useobject{currentmarker}{}%
\end{pgfscope}%
\begin{pgfscope}%
\pgfsys@transformshift{3.219493in}{0.668371in}%
\pgfsys@useobject{currentmarker}{}%
\end{pgfscope}%
\begin{pgfscope}%
\pgfsys@transformshift{3.249700in}{0.646270in}%
\pgfsys@useobject{currentmarker}{}%
\end{pgfscope}%
\begin{pgfscope}%
\pgfsys@transformshift{3.279907in}{0.658086in}%
\pgfsys@useobject{currentmarker}{}%
\end{pgfscope}%
\begin{pgfscope}%
\pgfsys@transformshift{3.310114in}{0.599083in}%
\pgfsys@useobject{currentmarker}{}%
\end{pgfscope}%
\begin{pgfscope}%
\pgfsys@transformshift{3.340320in}{0.601979in}%
\pgfsys@useobject{currentmarker}{}%
\end{pgfscope}%
\begin{pgfscope}%
\pgfsys@transformshift{3.370527in}{0.617626in}%
\pgfsys@useobject{currentmarker}{}%
\end{pgfscope}%
\begin{pgfscope}%
\pgfsys@transformshift{3.400734in}{0.619026in}%
\pgfsys@useobject{currentmarker}{}%
\end{pgfscope}%
\begin{pgfscope}%
\pgfsys@transformshift{3.430940in}{0.597630in}%
\pgfsys@useobject{currentmarker}{}%
\end{pgfscope}%
\begin{pgfscope}%
\pgfsys@transformshift{3.461147in}{0.593244in}%
\pgfsys@useobject{currentmarker}{}%
\end{pgfscope}%
\begin{pgfscope}%
\pgfsys@transformshift{3.491354in}{0.630092in}%
\pgfsys@useobject{currentmarker}{}%
\end{pgfscope}%
\end{pgfscope}%
\end{pgfpicture}%
\makeatother%
\endgroup%

  \caption{
    Fit of complete normalization model to $\PBzero\to\PJpsi\PKstar$ candidates.
    The total model (red, solid) is shown together with the signal (blue, dashed) and background (green, dashed) components.
  }
  \label{fig:normdatafit}
\end{figure}

\begin{table}
  \centering
  \caption{
    Parameters of the complete normalization model estimated from the normalization data sample.
    The parameters $n$ and $\alpha$, as well as $f_\text{left/right}$ have been fixed to the values from table \ref{tab:normmcfit}.
  }
  \begin{tabular}{l S[table-format=6.5,table-figures-uncertainty=1]}
    \toprule
    Parameter & {Estimate} \\
    \midrule
    $\mu$                       & 5281.3\pm 0.018 \\
    $\sigma_\text{left}$        & 6.340 \pm 0.023 \\
    $\sigma_\text{right}$       & 11.86 \pm 0.06 \\
    $\lambda_\text{background}$ & 0.00282 \pm 0.00016\\
    $N_\text{signal}$           & 273100 \pm 600 \\
    $N_\text{background}$       & 43200 \pm 400 \\
    \bottomrule
  \end{tabular}
  \label{tab:normdatafit}
\end{table}

\section{Normalization constant}
\label{normalization}

A possible approach to translating between a measured number $N$ of $\PBzero\to\APDzero\APmuon\Pmuon$ decays and the corresponding branching fraction $\text{BR}$ is given by
\begin{equation}
  N = \mathcal{L}\,\sigma_{b\overline{b}}\,2f_d\,\varepsilon_\text{total}\,\text{BR}\:.
  \label{eq:translate}
\end{equation}
Here, $\mathcal{L}$ is the luminosity delivered by the accelerator, $\sigma_{b\overline{b}}$ is the $b\overline{b}$ branching fraction at a given center of mass energy, $f_B$ is the probability for a $b$ quark to hadronize into a $B^0$ meson and $\varepsilon_\text{total}$ is the total signal efficiency of all analysis steps.

An alternative approach is to normalize the measurement to a decay channel with a known branching fraction.
Using \eqref{eq:translate} for both the signal and normalization channels yields
\begin{align}
  N_\text{sig} &= X \varepsilon_\text{total} \text{BR}_\text{sig} \\
  N_\text{norm} &= X \varepsilon_\text{total} \text{BR}_\text{norm}\:,
\end{align}
where $X$ includes $\mathcal{L}\,\sigma_{b\overline{b}}\,2f_d$ and can also represent systematic uncertainties that are identical for the two channels.
By substituting $X$ and solving for $\text{BR}_\text{sig}$, the following equation for $\text{BR}_\text{sig}$ is obtained.
\begin{equation}
  \text{BR}_\text{sig} = \frac{\text{BR}_\text{norm} \varepsilon_\text{norm}}{N_\text{norm} \varepsilon_\text{sig}} N_\text{sig}
\end{equation}
This way, a reduction in the systematic uncertainties of the estimated signal branching fraction can be achieved.

The factor
\begin{equation}
  \alpha = \frac{\text{BR}_\text{norm} \varepsilon_\text{norm}}{N_\text{norm} \varepsilon_\text{sig}}
  \label{eq:alpha}
\end{equation}
is referred to as the \emph{normalization constant}.
It translates the measured number of signal decays to the measured branching fraction.

A branching fraction of \num{1.32\pm0.06 e-3} (from previous measurements, \cite{PDG}) is assumed for $\text{BR}_\text{norm}$.
The efficiencies determined in chapter \ref{selection}, as well as the normalization channel yield $N_\text{norm}$ determined in section \ref{normfit} are repeated in table \ref{tab:alpha} together with the calculated normalization constant.

\begin{table}
  \centering
  \caption{Values used to calculate the normalization constant $\alpha$}
  \begin{tabular}{l S[table-format=1.3,table-figures-exponent=1,table-sign-exponent,table-figures-uncertainty=1]}
    \toprule
    Parameter & {Value} \\
    \midrule
    $\varepsilon_\text{total,sig}$  & 2.60 \pm 0.05 e-3 \\
    $\varepsilon_\text{total,norm}$ & 7.219 \pm 0.028 e-3 \\
    $N_\text{norm}$                 & 2.731 \pm 0.006 e5 \\
    $\mathup{BR}_\text{norm}$       & 1.32 \pm 0.06 e-3 \\
    \midrule
    $\alpha$                        & 1.34 \pm 0.07 e-8 \\
    \bottomrule
  \end{tabular}
  \label{tab:alpha}
\end{table}

\section{Expected limit on the branching ratio}

The signal and background models from sections \ref{signalmodel} and \ref{backgroundmodel} are combined into a single Extended Likelihood model
\begin{equation}
  p(m_B, m_D) = N_\text{sig} p_\text{sig}(m_B, m_D) + N_\text{bkg} p_\text{bkg}(m_B, m_D)\:.
\end{equation}

The branching fraction can be introduced as a parameter of the model by performing the substitution $N_\text{sig}\to \text{BR}_\text{sig}/\alpha$, where $\alpha$ is a new parameter representing the normalization constant.
The parameter $\alpha$ is constrained by multiplying the likelihood with a Gaussian function centered at the estimated value $\mu_\alpha$ with width $\sigma_\alpha$ equal to the value and error determined in section \ref{normalization}.

A limit on $\text{BR}_\text{sig}$ can be calculated by treating the nuisance parameters $N_\text{bkg}$ and $\alpha$ using the Profile Likelihood Ratio (PLR) method.
In order to do this, the PLR test statistic
\begin{equation}
  q(μ) = -2\mathup{ln}\left(\frac{\mathcal{L}(μ,\hat{\hat{θ}})}{\mathcal{L}(\hat{μ},\hat{θ})}\right)
\end{equation}
is calculated.
Here, $μ$ represents the parameters of interest, while $\theta$ represents the nuisance parameters.
$\mathcal{L}(μ, \hat{\hat{θ}})$ corresponds to the likelihood optimized for fixed $μ$ and $\mathcal{L}(\hat{μ},\hat{θ})$ corresponds to the likelihood optimized over both $μ$ and $θ$.
Assuming the applicability of Wilks' theorem \cite{Wilks}, this can be converted to an upper limit by treating $q(0)$ as a $\chi^2$ distributed variable and finding the value of $\text{BR}_\text{sig}$ for which the $p$-value for observing an equal or larger $q$ outcome is equal to 0.05 (for a 95\% confidence limit) or 0.1 (for a 90\% confidence limit).

For this purpose, a complete dataset, including the blinded region, is necessary.
While the dataset is blinded, it is useful to calculate an \emph{expected limit}, meaning an estimate of the limit that can be achieved under a background-only hypothesis.

In order to to do this, the blinded signal region is filled with toy simulated candidates.
An expected number of background candidates in the blinded window is estimated by integrating the background-only model over the signal window and propagating the uncertainties of the $\lambda$ parameters to the result of the integral.
In order to generate a toy sample, a value for the background rate $n$ is sampled from $\mathup{Normal}\left(n|N_\text{integ}, \sigma(N_\text{integ})\right)$, which is then used as as the rate paramter to sample a value $N_\text{toy}$ from a Poisson distribution.
A toy sample containing $N_\text{toy}$ candidates is then produced by restricting the background model to the blinded signal window and sampling from it.

A histogram showing the resulting limits of $10^5$ toy samples is given in figure \ref{fig:expected}.
The median of the distribution of calculated limits is
\begin{equation}
  \text{CL}_\text{expected}(90\%) = 6.8^{+5.1}_{-2.9}\,10^{-8}\:\,
\end{equation}
where the lower and upper errors enclose the central $68\%$ of calculated limits.

\begin{figure}
  \centering
  %% Creator: Matplotlib, PGF backend
%%
%% To include the figure in your LaTeX document, write
%%   \input{<filename>.pgf}
%%
%% Make sure the required packages are loaded in your preamble
%%   \usepackage{pgf}
%%
%% Figures using additional raster images can only be included by \input if
%% they are in the same directory as the main LaTeX file. For loading figures
%% from other directories you can use the `import` package
%%   \usepackage{import}
%% and then include the figures with
%%   \import{<path to file>}{<filename>.pgf}
%%
%% Matplotlib used the following preamble
%%   \usepackage{fontspec}
%%   \setmainfont{DejaVu Serif}
%%   \setsansfont{DejaVu Sans}
%%   \setmonofont{DejaVu Sans Mono}
%%
\begingroup%
\makeatletter%
\begin{pgfpicture}%
\pgfpathrectangle{\pgfpointorigin}{\pgfqpoint{4.195807in}{2.404275in}}%
\pgfusepath{use as bounding box, clip}%
\begin{pgfscope}%
\pgfsetbuttcap%
\pgfsetmiterjoin%
\definecolor{currentfill}{rgb}{1.000000,1.000000,1.000000}%
\pgfsetfillcolor{currentfill}%
\pgfsetlinewidth{0.000000pt}%
\definecolor{currentstroke}{rgb}{1.000000,1.000000,1.000000}%
\pgfsetstrokecolor{currentstroke}%
\pgfsetdash{}{0pt}%
\pgfpathmoveto{\pgfqpoint{0.000000in}{0.000000in}}%
\pgfpathlineto{\pgfqpoint{4.195807in}{0.000000in}}%
\pgfpathlineto{\pgfqpoint{4.195807in}{2.404275in}}%
\pgfpathlineto{\pgfqpoint{0.000000in}{2.404275in}}%
\pgfpathclose%
\pgfusepath{fill}%
\end{pgfscope}%
\begin{pgfscope}%
\pgfsetbuttcap%
\pgfsetmiterjoin%
\definecolor{currentfill}{rgb}{1.000000,1.000000,1.000000}%
\pgfsetfillcolor{currentfill}%
\pgfsetlinewidth{0.000000pt}%
\definecolor{currentstroke}{rgb}{0.000000,0.000000,0.000000}%
\pgfsetstrokecolor{currentstroke}%
\pgfsetstrokeopacity{0.000000}%
\pgfsetdash{}{0pt}%
\pgfpathmoveto{\pgfqpoint{0.402214in}{0.460229in}}%
\pgfpathlineto{\pgfqpoint{4.057455in}{0.460229in}}%
\pgfpathlineto{\pgfqpoint{4.057455in}{2.300509in}}%
\pgfpathlineto{\pgfqpoint{0.402214in}{2.300509in}}%
\pgfpathclose%
\pgfusepath{fill}%
\end{pgfscope}%
\begin{pgfscope}%
\pgfpathrectangle{\pgfqpoint{0.402214in}{0.460229in}}{\pgfqpoint{3.655242in}{1.840280in}} %
\pgfusepath{clip}%
\pgfsetbuttcap%
\pgfsetmiterjoin%
\pgfsetlinewidth{1.003750pt}%
\definecolor{currentstroke}{rgb}{0.000000,0.000000,0.000000}%
\pgfsetstrokecolor{currentstroke}%
\pgfsetdash{}{0pt}%
\pgfpathmoveto{\pgfqpoint{0.611119in}{0.460229in}}%
\pgfpathlineto{\pgfqpoint{0.611119in}{0.814483in}}%
\pgfpathlineto{\pgfqpoint{0.688965in}{0.814483in}}%
\pgfpathlineto{\pgfqpoint{0.688965in}{1.565931in}}%
\pgfpathlineto{\pgfqpoint{0.766812in}{1.565931in}}%
\pgfpathlineto{\pgfqpoint{0.766812in}{2.110347in}}%
\pgfpathlineto{\pgfqpoint{0.844658in}{2.110347in}}%
\pgfpathlineto{\pgfqpoint{0.844658in}{2.072008in}}%
\pgfpathlineto{\pgfqpoint{0.922504in}{2.072008in}}%
\pgfpathlineto{\pgfqpoint{0.922504in}{1.782164in}}%
\pgfpathlineto{\pgfqpoint{1.000351in}{1.782164in}}%
\pgfpathlineto{\pgfqpoint{1.000351in}{1.630341in}}%
\pgfpathlineto{\pgfqpoint{1.078197in}{1.630341in}}%
\pgfpathlineto{\pgfqpoint{1.078197in}{1.503055in}}%
\pgfpathlineto{\pgfqpoint{1.156043in}{1.503055in}}%
\pgfpathlineto{\pgfqpoint{1.156043in}{1.375769in}}%
\pgfpathlineto{\pgfqpoint{1.233890in}{1.375769in}}%
\pgfpathlineto{\pgfqpoint{1.233890in}{1.269953in}}%
\pgfpathlineto{\pgfqpoint{1.311736in}{1.269953in}}%
\pgfpathlineto{\pgfqpoint{1.311736in}{1.220878in}}%
\pgfpathlineto{\pgfqpoint{1.389582in}{1.220878in}}%
\pgfpathlineto{\pgfqpoint{1.389582in}{1.196341in}}%
\pgfpathlineto{\pgfqpoint{1.467429in}{1.196341in}}%
\pgfpathlineto{\pgfqpoint{1.467429in}{1.110462in}}%
\pgfpathlineto{\pgfqpoint{1.545275in}{1.110462in}}%
\pgfpathlineto{\pgfqpoint{1.545275in}{1.018448in}}%
\pgfpathlineto{\pgfqpoint{1.623122in}{1.018448in}}%
\pgfpathlineto{\pgfqpoint{1.623122in}{0.967840in}}%
\pgfpathlineto{\pgfqpoint{1.700968in}{0.967840in}}%
\pgfpathlineto{\pgfqpoint{1.700968in}{0.862024in}}%
\pgfpathlineto{\pgfqpoint{1.778814in}{0.862024in}}%
\pgfpathlineto{\pgfqpoint{1.778814in}{0.779211in}}%
\pgfpathlineto{\pgfqpoint{1.856661in}{0.779211in}}%
\pgfpathlineto{\pgfqpoint{1.856661in}{0.757741in}}%
\pgfpathlineto{\pgfqpoint{1.934507in}{0.757741in}}%
\pgfpathlineto{\pgfqpoint{1.934507in}{0.725536in}}%
\pgfpathlineto{\pgfqpoint{2.012353in}{0.725536in}}%
\pgfpathlineto{\pgfqpoint{2.012353in}{0.662660in}}%
\pgfpathlineto{\pgfqpoint{2.090200in}{0.662660in}}%
\pgfpathlineto{\pgfqpoint{2.090200in}{0.584448in}}%
\pgfpathlineto{\pgfqpoint{2.168046in}{0.584448in}}%
\pgfpathlineto{\pgfqpoint{2.168046in}{0.592116in}}%
\pgfpathlineto{\pgfqpoint{2.245892in}{0.592116in}}%
\pgfpathlineto{\pgfqpoint{2.245892in}{0.533841in}}%
\pgfpathlineto{\pgfqpoint{2.323739in}{0.533841in}}%
\pgfpathlineto{\pgfqpoint{2.323739in}{0.535374in}}%
\pgfpathlineto{\pgfqpoint{2.401585in}{0.535374in}}%
\pgfpathlineto{\pgfqpoint{2.401585in}{0.507770in}}%
\pgfpathlineto{\pgfqpoint{2.479431in}{0.507770in}}%
\pgfpathlineto{\pgfqpoint{2.479431in}{0.509304in}}%
\pgfpathlineto{\pgfqpoint{2.557278in}{0.509304in}}%
\pgfpathlineto{\pgfqpoint{2.557278in}{0.495501in}}%
\pgfpathlineto{\pgfqpoint{2.635124in}{0.495501in}}%
\pgfpathlineto{\pgfqpoint{2.635124in}{0.475565in}}%
\pgfpathlineto{\pgfqpoint{2.712970in}{0.475565in}}%
\pgfpathlineto{\pgfqpoint{2.712970in}{0.477099in}}%
\pgfpathlineto{\pgfqpoint{2.790817in}{0.477099in}}%
\pgfpathlineto{\pgfqpoint{2.790817in}{0.477099in}}%
\pgfpathlineto{\pgfqpoint{2.868663in}{0.477099in}}%
\pgfpathlineto{\pgfqpoint{2.868663in}{0.461763in}}%
\pgfpathlineto{\pgfqpoint{2.946509in}{0.461763in}}%
\pgfpathlineto{\pgfqpoint{2.946509in}{0.464830in}}%
\pgfpathlineto{\pgfqpoint{3.024356in}{0.464830in}}%
\pgfpathlineto{\pgfqpoint{3.024356in}{0.464830in}}%
\pgfpathlineto{\pgfqpoint{3.102202in}{0.464830in}}%
\pgfpathlineto{\pgfqpoint{3.102202in}{0.463297in}}%
\pgfpathlineto{\pgfqpoint{3.180049in}{0.463297in}}%
\pgfpathlineto{\pgfqpoint{3.180049in}{0.463297in}}%
\pgfpathlineto{\pgfqpoint{3.257895in}{0.463297in}}%
\pgfpathlineto{\pgfqpoint{3.257895in}{0.460229in}}%
\pgfpathlineto{\pgfqpoint{3.335741in}{0.460229in}}%
\pgfpathlineto{\pgfqpoint{3.335741in}{0.461763in}}%
\pgfpathlineto{\pgfqpoint{3.413588in}{0.461763in}}%
\pgfpathlineto{\pgfqpoint{3.413588in}{0.460229in}}%
\pgfpathlineto{\pgfqpoint{3.491434in}{0.460229in}}%
\pgfpathlineto{\pgfqpoint{3.491434in}{0.461763in}}%
\pgfpathlineto{\pgfqpoint{3.569280in}{0.461763in}}%
\pgfpathlineto{\pgfqpoint{3.569280in}{0.460229in}}%
\pgfpathlineto{\pgfqpoint{3.647127in}{0.460229in}}%
\pgfpathlineto{\pgfqpoint{3.647127in}{0.461763in}}%
\pgfpathlineto{\pgfqpoint{3.724973in}{0.461763in}}%
\pgfpathlineto{\pgfqpoint{3.724973in}{0.460229in}}%
\pgfusepath{stroke}%
\end{pgfscope}%
\begin{pgfscope}%
\pgfpathrectangle{\pgfqpoint{0.402214in}{0.460229in}}{\pgfqpoint{3.655242in}{1.840280in}} %
\pgfusepath{clip}%
\pgfsetrectcap%
\pgfsetroundjoin%
\pgfsetlinewidth{1.003750pt}%
\definecolor{currentstroke}{rgb}{1.000000,0.000000,0.000000}%
\pgfsetstrokecolor{currentstroke}%
\pgfsetdash{}{0pt}%
\pgfpathmoveto{\pgfqpoint{1.111111in}{0.460229in}}%
\pgfpathlineto{\pgfqpoint{1.111111in}{2.300509in}}%
\pgfusepath{stroke}%
\end{pgfscope}%
\begin{pgfscope}%
\pgfpathrectangle{\pgfqpoint{0.402214in}{0.460229in}}{\pgfqpoint{3.655242in}{1.840280in}} %
\pgfusepath{clip}%
\pgfsetbuttcap%
\pgfsetroundjoin%
\pgfsetlinewidth{1.003750pt}%
\definecolor{currentstroke}{rgb}{1.000000,0.000000,0.000000}%
\pgfsetstrokecolor{currentstroke}%
\pgfsetdash{{6.000000pt}{6.000000pt}}{0.000000pt}%
\pgfpathmoveto{\pgfqpoint{0.812573in}{0.460229in}}%
\pgfpathlineto{\pgfqpoint{0.812573in}{2.300509in}}%
\pgfusepath{stroke}%
\end{pgfscope}%
\begin{pgfscope}%
\pgfpathrectangle{\pgfqpoint{0.402214in}{0.460229in}}{\pgfqpoint{3.655242in}{1.840280in}} %
\pgfusepath{clip}%
\pgfsetbuttcap%
\pgfsetroundjoin%
\pgfsetlinewidth{1.003750pt}%
\definecolor{currentstroke}{rgb}{1.000000,0.000000,0.000000}%
\pgfsetstrokecolor{currentstroke}%
\pgfsetdash{{6.000000pt}{6.000000pt}}{0.000000pt}%
\pgfpathmoveto{\pgfqpoint{1.645602in}{0.460229in}}%
\pgfpathlineto{\pgfqpoint{1.645602in}{2.300509in}}%
\pgfusepath{stroke}%
\end{pgfscope}%
\begin{pgfscope}%
\pgfsetrectcap%
\pgfsetmiterjoin%
\pgfsetlinewidth{1.003750pt}%
\definecolor{currentstroke}{rgb}{0.000000,0.000000,0.000000}%
\pgfsetstrokecolor{currentstroke}%
\pgfsetdash{}{0pt}%
\pgfpathmoveto{\pgfqpoint{0.402214in}{2.300509in}}%
\pgfpathlineto{\pgfqpoint{4.057455in}{2.300509in}}%
\pgfusepath{stroke}%
\end{pgfscope}%
\begin{pgfscope}%
\pgfsetrectcap%
\pgfsetmiterjoin%
\pgfsetlinewidth{1.003750pt}%
\definecolor{currentstroke}{rgb}{0.000000,0.000000,0.000000}%
\pgfsetstrokecolor{currentstroke}%
\pgfsetdash{}{0pt}%
\pgfpathmoveto{\pgfqpoint{4.057455in}{0.460229in}}%
\pgfpathlineto{\pgfqpoint{4.057455in}{2.300509in}}%
\pgfusepath{stroke}%
\end{pgfscope}%
\begin{pgfscope}%
\pgfsetrectcap%
\pgfsetmiterjoin%
\pgfsetlinewidth{1.003750pt}%
\definecolor{currentstroke}{rgb}{0.000000,0.000000,0.000000}%
\pgfsetstrokecolor{currentstroke}%
\pgfsetdash{}{0pt}%
\pgfpathmoveto{\pgfqpoint{0.402214in}{0.460229in}}%
\pgfpathlineto{\pgfqpoint{4.057455in}{0.460229in}}%
\pgfusepath{stroke}%
\end{pgfscope}%
\begin{pgfscope}%
\pgfsetrectcap%
\pgfsetmiterjoin%
\pgfsetlinewidth{1.003750pt}%
\definecolor{currentstroke}{rgb}{0.000000,0.000000,0.000000}%
\pgfsetstrokecolor{currentstroke}%
\pgfsetdash{}{0pt}%
\pgfpathmoveto{\pgfqpoint{0.402214in}{0.460229in}}%
\pgfpathlineto{\pgfqpoint{0.402214in}{2.300509in}}%
\pgfusepath{stroke}%
\end{pgfscope}%
\begin{pgfscope}%
\pgfsetbuttcap%
\pgfsetroundjoin%
\definecolor{currentfill}{rgb}{0.000000,0.000000,0.000000}%
\pgfsetfillcolor{currentfill}%
\pgfsetlinewidth{0.501875pt}%
\definecolor{currentstroke}{rgb}{0.000000,0.000000,0.000000}%
\pgfsetstrokecolor{currentstroke}%
\pgfsetdash{}{0pt}%
\pgfsys@defobject{currentmarker}{\pgfqpoint{0.000000in}{0.000000in}}{\pgfqpoint{0.000000in}{0.069444in}}{%
\pgfpathmoveto{\pgfqpoint{0.000000in}{0.000000in}}%
\pgfpathlineto{\pgfqpoint{0.000000in}{0.069444in}}%
\pgfusepath{stroke,fill}%
}%
\begin{pgfscope}%
\pgfsys@transformshift{0.402214in}{0.460229in}%
\pgfsys@useobject{currentmarker}{}%
\end{pgfscope}%
\end{pgfscope}%
\begin{pgfscope}%
\pgfsetbuttcap%
\pgfsetroundjoin%
\definecolor{currentfill}{rgb}{0.000000,0.000000,0.000000}%
\pgfsetfillcolor{currentfill}%
\pgfsetlinewidth{0.501875pt}%
\definecolor{currentstroke}{rgb}{0.000000,0.000000,0.000000}%
\pgfsetstrokecolor{currentstroke}%
\pgfsetdash{}{0pt}%
\pgfsys@defobject{currentmarker}{\pgfqpoint{0.000000in}{-0.069444in}}{\pgfqpoint{0.000000in}{0.000000in}}{%
\pgfpathmoveto{\pgfqpoint{0.000000in}{0.000000in}}%
\pgfpathlineto{\pgfqpoint{0.000000in}{-0.069444in}}%
\pgfusepath{stroke,fill}%
}%
\begin{pgfscope}%
\pgfsys@transformshift{0.402214in}{2.300509in}%
\pgfsys@useobject{currentmarker}{}%
\end{pgfscope}%
\end{pgfscope}%
\begin{pgfscope}%
\pgftext[x=0.402214in,y=0.390785in,,top]{\rmfamily\fontsize{8.000000}{9.600000}\selectfont 0.0}%
\end{pgfscope}%
\begin{pgfscope}%
\pgfsetbuttcap%
\pgfsetroundjoin%
\definecolor{currentfill}{rgb}{0.000000,0.000000,0.000000}%
\pgfsetfillcolor{currentfill}%
\pgfsetlinewidth{0.501875pt}%
\definecolor{currentstroke}{rgb}{0.000000,0.000000,0.000000}%
\pgfsetstrokecolor{currentstroke}%
\pgfsetdash{}{0pt}%
\pgfsys@defobject{currentmarker}{\pgfqpoint{0.000000in}{0.000000in}}{\pgfqpoint{0.000000in}{0.069444in}}{%
\pgfpathmoveto{\pgfqpoint{0.000000in}{0.000000in}}%
\pgfpathlineto{\pgfqpoint{0.000000in}{0.069444in}}%
\pgfusepath{stroke,fill}%
}%
\begin{pgfscope}%
\pgfsys@transformshift{0.924391in}{0.460229in}%
\pgfsys@useobject{currentmarker}{}%
\end{pgfscope}%
\end{pgfscope}%
\begin{pgfscope}%
\pgfsetbuttcap%
\pgfsetroundjoin%
\definecolor{currentfill}{rgb}{0.000000,0.000000,0.000000}%
\pgfsetfillcolor{currentfill}%
\pgfsetlinewidth{0.501875pt}%
\definecolor{currentstroke}{rgb}{0.000000,0.000000,0.000000}%
\pgfsetstrokecolor{currentstroke}%
\pgfsetdash{}{0pt}%
\pgfsys@defobject{currentmarker}{\pgfqpoint{0.000000in}{-0.069444in}}{\pgfqpoint{0.000000in}{0.000000in}}{%
\pgfpathmoveto{\pgfqpoint{0.000000in}{0.000000in}}%
\pgfpathlineto{\pgfqpoint{0.000000in}{-0.069444in}}%
\pgfusepath{stroke,fill}%
}%
\begin{pgfscope}%
\pgfsys@transformshift{0.924391in}{2.300509in}%
\pgfsys@useobject{currentmarker}{}%
\end{pgfscope}%
\end{pgfscope}%
\begin{pgfscope}%
\pgftext[x=0.924391in,y=0.390785in,,top]{\rmfamily\fontsize{8.000000}{9.600000}\selectfont 0.5}%
\end{pgfscope}%
\begin{pgfscope}%
\pgfsetbuttcap%
\pgfsetroundjoin%
\definecolor{currentfill}{rgb}{0.000000,0.000000,0.000000}%
\pgfsetfillcolor{currentfill}%
\pgfsetlinewidth{0.501875pt}%
\definecolor{currentstroke}{rgb}{0.000000,0.000000,0.000000}%
\pgfsetstrokecolor{currentstroke}%
\pgfsetdash{}{0pt}%
\pgfsys@defobject{currentmarker}{\pgfqpoint{0.000000in}{0.000000in}}{\pgfqpoint{0.000000in}{0.069444in}}{%
\pgfpathmoveto{\pgfqpoint{0.000000in}{0.000000in}}%
\pgfpathlineto{\pgfqpoint{0.000000in}{0.069444in}}%
\pgfusepath{stroke,fill}%
}%
\begin{pgfscope}%
\pgfsys@transformshift{1.446568in}{0.460229in}%
\pgfsys@useobject{currentmarker}{}%
\end{pgfscope}%
\end{pgfscope}%
\begin{pgfscope}%
\pgfsetbuttcap%
\pgfsetroundjoin%
\definecolor{currentfill}{rgb}{0.000000,0.000000,0.000000}%
\pgfsetfillcolor{currentfill}%
\pgfsetlinewidth{0.501875pt}%
\definecolor{currentstroke}{rgb}{0.000000,0.000000,0.000000}%
\pgfsetstrokecolor{currentstroke}%
\pgfsetdash{}{0pt}%
\pgfsys@defobject{currentmarker}{\pgfqpoint{0.000000in}{-0.069444in}}{\pgfqpoint{0.000000in}{0.000000in}}{%
\pgfpathmoveto{\pgfqpoint{0.000000in}{0.000000in}}%
\pgfpathlineto{\pgfqpoint{0.000000in}{-0.069444in}}%
\pgfusepath{stroke,fill}%
}%
\begin{pgfscope}%
\pgfsys@transformshift{1.446568in}{2.300509in}%
\pgfsys@useobject{currentmarker}{}%
\end{pgfscope}%
\end{pgfscope}%
\begin{pgfscope}%
\pgftext[x=1.446568in,y=0.390785in,,top]{\rmfamily\fontsize{8.000000}{9.600000}\selectfont 1.0}%
\end{pgfscope}%
\begin{pgfscope}%
\pgfsetbuttcap%
\pgfsetroundjoin%
\definecolor{currentfill}{rgb}{0.000000,0.000000,0.000000}%
\pgfsetfillcolor{currentfill}%
\pgfsetlinewidth{0.501875pt}%
\definecolor{currentstroke}{rgb}{0.000000,0.000000,0.000000}%
\pgfsetstrokecolor{currentstroke}%
\pgfsetdash{}{0pt}%
\pgfsys@defobject{currentmarker}{\pgfqpoint{0.000000in}{0.000000in}}{\pgfqpoint{0.000000in}{0.069444in}}{%
\pgfpathmoveto{\pgfqpoint{0.000000in}{0.000000in}}%
\pgfpathlineto{\pgfqpoint{0.000000in}{0.069444in}}%
\pgfusepath{stroke,fill}%
}%
\begin{pgfscope}%
\pgfsys@transformshift{1.968746in}{0.460229in}%
\pgfsys@useobject{currentmarker}{}%
\end{pgfscope}%
\end{pgfscope}%
\begin{pgfscope}%
\pgfsetbuttcap%
\pgfsetroundjoin%
\definecolor{currentfill}{rgb}{0.000000,0.000000,0.000000}%
\pgfsetfillcolor{currentfill}%
\pgfsetlinewidth{0.501875pt}%
\definecolor{currentstroke}{rgb}{0.000000,0.000000,0.000000}%
\pgfsetstrokecolor{currentstroke}%
\pgfsetdash{}{0pt}%
\pgfsys@defobject{currentmarker}{\pgfqpoint{0.000000in}{-0.069444in}}{\pgfqpoint{0.000000in}{0.000000in}}{%
\pgfpathmoveto{\pgfqpoint{0.000000in}{0.000000in}}%
\pgfpathlineto{\pgfqpoint{0.000000in}{-0.069444in}}%
\pgfusepath{stroke,fill}%
}%
\begin{pgfscope}%
\pgfsys@transformshift{1.968746in}{2.300509in}%
\pgfsys@useobject{currentmarker}{}%
\end{pgfscope}%
\end{pgfscope}%
\begin{pgfscope}%
\pgftext[x=1.968746in,y=0.390785in,,top]{\rmfamily\fontsize{8.000000}{9.600000}\selectfont 1.5}%
\end{pgfscope}%
\begin{pgfscope}%
\pgfsetbuttcap%
\pgfsetroundjoin%
\definecolor{currentfill}{rgb}{0.000000,0.000000,0.000000}%
\pgfsetfillcolor{currentfill}%
\pgfsetlinewidth{0.501875pt}%
\definecolor{currentstroke}{rgb}{0.000000,0.000000,0.000000}%
\pgfsetstrokecolor{currentstroke}%
\pgfsetdash{}{0pt}%
\pgfsys@defobject{currentmarker}{\pgfqpoint{0.000000in}{0.000000in}}{\pgfqpoint{0.000000in}{0.069444in}}{%
\pgfpathmoveto{\pgfqpoint{0.000000in}{0.000000in}}%
\pgfpathlineto{\pgfqpoint{0.000000in}{0.069444in}}%
\pgfusepath{stroke,fill}%
}%
\begin{pgfscope}%
\pgfsys@transformshift{2.490923in}{0.460229in}%
\pgfsys@useobject{currentmarker}{}%
\end{pgfscope}%
\end{pgfscope}%
\begin{pgfscope}%
\pgfsetbuttcap%
\pgfsetroundjoin%
\definecolor{currentfill}{rgb}{0.000000,0.000000,0.000000}%
\pgfsetfillcolor{currentfill}%
\pgfsetlinewidth{0.501875pt}%
\definecolor{currentstroke}{rgb}{0.000000,0.000000,0.000000}%
\pgfsetstrokecolor{currentstroke}%
\pgfsetdash{}{0pt}%
\pgfsys@defobject{currentmarker}{\pgfqpoint{0.000000in}{-0.069444in}}{\pgfqpoint{0.000000in}{0.000000in}}{%
\pgfpathmoveto{\pgfqpoint{0.000000in}{0.000000in}}%
\pgfpathlineto{\pgfqpoint{0.000000in}{-0.069444in}}%
\pgfusepath{stroke,fill}%
}%
\begin{pgfscope}%
\pgfsys@transformshift{2.490923in}{2.300509in}%
\pgfsys@useobject{currentmarker}{}%
\end{pgfscope}%
\end{pgfscope}%
\begin{pgfscope}%
\pgftext[x=2.490923in,y=0.390785in,,top]{\rmfamily\fontsize{8.000000}{9.600000}\selectfont 2.0}%
\end{pgfscope}%
\begin{pgfscope}%
\pgfsetbuttcap%
\pgfsetroundjoin%
\definecolor{currentfill}{rgb}{0.000000,0.000000,0.000000}%
\pgfsetfillcolor{currentfill}%
\pgfsetlinewidth{0.501875pt}%
\definecolor{currentstroke}{rgb}{0.000000,0.000000,0.000000}%
\pgfsetstrokecolor{currentstroke}%
\pgfsetdash{}{0pt}%
\pgfsys@defobject{currentmarker}{\pgfqpoint{0.000000in}{0.000000in}}{\pgfqpoint{0.000000in}{0.069444in}}{%
\pgfpathmoveto{\pgfqpoint{0.000000in}{0.000000in}}%
\pgfpathlineto{\pgfqpoint{0.000000in}{0.069444in}}%
\pgfusepath{stroke,fill}%
}%
\begin{pgfscope}%
\pgfsys@transformshift{3.013101in}{0.460229in}%
\pgfsys@useobject{currentmarker}{}%
\end{pgfscope}%
\end{pgfscope}%
\begin{pgfscope}%
\pgfsetbuttcap%
\pgfsetroundjoin%
\definecolor{currentfill}{rgb}{0.000000,0.000000,0.000000}%
\pgfsetfillcolor{currentfill}%
\pgfsetlinewidth{0.501875pt}%
\definecolor{currentstroke}{rgb}{0.000000,0.000000,0.000000}%
\pgfsetstrokecolor{currentstroke}%
\pgfsetdash{}{0pt}%
\pgfsys@defobject{currentmarker}{\pgfqpoint{0.000000in}{-0.069444in}}{\pgfqpoint{0.000000in}{0.000000in}}{%
\pgfpathmoveto{\pgfqpoint{0.000000in}{0.000000in}}%
\pgfpathlineto{\pgfqpoint{0.000000in}{-0.069444in}}%
\pgfusepath{stroke,fill}%
}%
\begin{pgfscope}%
\pgfsys@transformshift{3.013101in}{2.300509in}%
\pgfsys@useobject{currentmarker}{}%
\end{pgfscope}%
\end{pgfscope}%
\begin{pgfscope}%
\pgftext[x=3.013101in,y=0.390785in,,top]{\rmfamily\fontsize{8.000000}{9.600000}\selectfont 2.5}%
\end{pgfscope}%
\begin{pgfscope}%
\pgfsetbuttcap%
\pgfsetroundjoin%
\definecolor{currentfill}{rgb}{0.000000,0.000000,0.000000}%
\pgfsetfillcolor{currentfill}%
\pgfsetlinewidth{0.501875pt}%
\definecolor{currentstroke}{rgb}{0.000000,0.000000,0.000000}%
\pgfsetstrokecolor{currentstroke}%
\pgfsetdash{}{0pt}%
\pgfsys@defobject{currentmarker}{\pgfqpoint{0.000000in}{0.000000in}}{\pgfqpoint{0.000000in}{0.069444in}}{%
\pgfpathmoveto{\pgfqpoint{0.000000in}{0.000000in}}%
\pgfpathlineto{\pgfqpoint{0.000000in}{0.069444in}}%
\pgfusepath{stroke,fill}%
}%
\begin{pgfscope}%
\pgfsys@transformshift{3.535278in}{0.460229in}%
\pgfsys@useobject{currentmarker}{}%
\end{pgfscope}%
\end{pgfscope}%
\begin{pgfscope}%
\pgfsetbuttcap%
\pgfsetroundjoin%
\definecolor{currentfill}{rgb}{0.000000,0.000000,0.000000}%
\pgfsetfillcolor{currentfill}%
\pgfsetlinewidth{0.501875pt}%
\definecolor{currentstroke}{rgb}{0.000000,0.000000,0.000000}%
\pgfsetstrokecolor{currentstroke}%
\pgfsetdash{}{0pt}%
\pgfsys@defobject{currentmarker}{\pgfqpoint{0.000000in}{-0.069444in}}{\pgfqpoint{0.000000in}{0.000000in}}{%
\pgfpathmoveto{\pgfqpoint{0.000000in}{0.000000in}}%
\pgfpathlineto{\pgfqpoint{0.000000in}{-0.069444in}}%
\pgfusepath{stroke,fill}%
}%
\begin{pgfscope}%
\pgfsys@transformshift{3.535278in}{2.300509in}%
\pgfsys@useobject{currentmarker}{}%
\end{pgfscope}%
\end{pgfscope}%
\begin{pgfscope}%
\pgftext[x=3.535278in,y=0.390785in,,top]{\rmfamily\fontsize{8.000000}{9.600000}\selectfont 3.0}%
\end{pgfscope}%
\begin{pgfscope}%
\pgfsetbuttcap%
\pgfsetroundjoin%
\definecolor{currentfill}{rgb}{0.000000,0.000000,0.000000}%
\pgfsetfillcolor{currentfill}%
\pgfsetlinewidth{0.501875pt}%
\definecolor{currentstroke}{rgb}{0.000000,0.000000,0.000000}%
\pgfsetstrokecolor{currentstroke}%
\pgfsetdash{}{0pt}%
\pgfsys@defobject{currentmarker}{\pgfqpoint{0.000000in}{0.000000in}}{\pgfqpoint{0.000000in}{0.069444in}}{%
\pgfpathmoveto{\pgfqpoint{0.000000in}{0.000000in}}%
\pgfpathlineto{\pgfqpoint{0.000000in}{0.069444in}}%
\pgfusepath{stroke,fill}%
}%
\begin{pgfscope}%
\pgfsys@transformshift{4.057455in}{0.460229in}%
\pgfsys@useobject{currentmarker}{}%
\end{pgfscope}%
\end{pgfscope}%
\begin{pgfscope}%
\pgfsetbuttcap%
\pgfsetroundjoin%
\definecolor{currentfill}{rgb}{0.000000,0.000000,0.000000}%
\pgfsetfillcolor{currentfill}%
\pgfsetlinewidth{0.501875pt}%
\definecolor{currentstroke}{rgb}{0.000000,0.000000,0.000000}%
\pgfsetstrokecolor{currentstroke}%
\pgfsetdash{}{0pt}%
\pgfsys@defobject{currentmarker}{\pgfqpoint{0.000000in}{-0.069444in}}{\pgfqpoint{0.000000in}{0.000000in}}{%
\pgfpathmoveto{\pgfqpoint{0.000000in}{0.000000in}}%
\pgfpathlineto{\pgfqpoint{0.000000in}{-0.069444in}}%
\pgfusepath{stroke,fill}%
}%
\begin{pgfscope}%
\pgfsys@transformshift{4.057455in}{2.300509in}%
\pgfsys@useobject{currentmarker}{}%
\end{pgfscope}%
\end{pgfscope}%
\begin{pgfscope}%
\pgftext[x=4.057455in,y=0.390785in,,top]{\rmfamily\fontsize{8.000000}{9.600000}\selectfont 3.5}%
\end{pgfscope}%
\begin{pgfscope}%
\pgftext[x=2.229834in,y=0.213810in,,top]{\rmfamily\fontsize{9.000000}{10.800000}\selectfont \(\displaystyle 90\%\ \mathrm{upper}\ \mathrm{CL}\ \mathrm{on}\ \mathrm{BR}(B^0\to \overline{D}^0\mu^+\mu^-)\)}%
\end{pgfscope}%
\begin{pgfscope}%
\pgftext[x=4.057455in,y=0.241588in,right,top]{\rmfamily\fontsize{8.000000}{9.600000}\selectfont 1e−7}%
\end{pgfscope}%
\begin{pgfscope}%
\pgfsetbuttcap%
\pgfsetroundjoin%
\definecolor{currentfill}{rgb}{0.000000,0.000000,0.000000}%
\pgfsetfillcolor{currentfill}%
\pgfsetlinewidth{0.501875pt}%
\definecolor{currentstroke}{rgb}{0.000000,0.000000,0.000000}%
\pgfsetstrokecolor{currentstroke}%
\pgfsetdash{}{0pt}%
\pgfsys@defobject{currentmarker}{\pgfqpoint{0.000000in}{0.000000in}}{\pgfqpoint{0.069444in}{0.000000in}}{%
\pgfpathmoveto{\pgfqpoint{0.000000in}{0.000000in}}%
\pgfpathlineto{\pgfqpoint{0.069444in}{0.000000in}}%
\pgfusepath{stroke,fill}%
}%
\begin{pgfscope}%
\pgfsys@transformshift{0.402214in}{0.460229in}%
\pgfsys@useobject{currentmarker}{}%
\end{pgfscope}%
\end{pgfscope}%
\begin{pgfscope}%
\pgfsetbuttcap%
\pgfsetroundjoin%
\definecolor{currentfill}{rgb}{0.000000,0.000000,0.000000}%
\pgfsetfillcolor{currentfill}%
\pgfsetlinewidth{0.501875pt}%
\definecolor{currentstroke}{rgb}{0.000000,0.000000,0.000000}%
\pgfsetstrokecolor{currentstroke}%
\pgfsetdash{}{0pt}%
\pgfsys@defobject{currentmarker}{\pgfqpoint{-0.069444in}{0.000000in}}{\pgfqpoint{0.000000in}{0.000000in}}{%
\pgfpathmoveto{\pgfqpoint{0.000000in}{0.000000in}}%
\pgfpathlineto{\pgfqpoint{-0.069444in}{0.000000in}}%
\pgfusepath{stroke,fill}%
}%
\begin{pgfscope}%
\pgfsys@transformshift{4.057455in}{0.460229in}%
\pgfsys@useobject{currentmarker}{}%
\end{pgfscope}%
\end{pgfscope}%
\begin{pgfscope}%
\pgftext[x=0.332769in,y=0.460229in,right,]{\rmfamily\fontsize{8.000000}{9.600000}\selectfont 0}%
\end{pgfscope}%
\begin{pgfscope}%
\pgfsetbuttcap%
\pgfsetroundjoin%
\definecolor{currentfill}{rgb}{0.000000,0.000000,0.000000}%
\pgfsetfillcolor{currentfill}%
\pgfsetlinewidth{0.501875pt}%
\definecolor{currentstroke}{rgb}{0.000000,0.000000,0.000000}%
\pgfsetstrokecolor{currentstroke}%
\pgfsetdash{}{0pt}%
\pgfsys@defobject{currentmarker}{\pgfqpoint{0.000000in}{0.000000in}}{\pgfqpoint{0.069444in}{0.000000in}}{%
\pgfpathmoveto{\pgfqpoint{0.000000in}{0.000000in}}%
\pgfpathlineto{\pgfqpoint{0.069444in}{0.000000in}}%
\pgfusepath{stroke,fill}%
}%
\begin{pgfscope}%
\pgfsys@transformshift{0.402214in}{0.766943in}%
\pgfsys@useobject{currentmarker}{}%
\end{pgfscope}%
\end{pgfscope}%
\begin{pgfscope}%
\pgfsetbuttcap%
\pgfsetroundjoin%
\definecolor{currentfill}{rgb}{0.000000,0.000000,0.000000}%
\pgfsetfillcolor{currentfill}%
\pgfsetlinewidth{0.501875pt}%
\definecolor{currentstroke}{rgb}{0.000000,0.000000,0.000000}%
\pgfsetstrokecolor{currentstroke}%
\pgfsetdash{}{0pt}%
\pgfsys@defobject{currentmarker}{\pgfqpoint{-0.069444in}{0.000000in}}{\pgfqpoint{0.000000in}{0.000000in}}{%
\pgfpathmoveto{\pgfqpoint{0.000000in}{0.000000in}}%
\pgfpathlineto{\pgfqpoint{-0.069444in}{0.000000in}}%
\pgfusepath{stroke,fill}%
}%
\begin{pgfscope}%
\pgfsys@transformshift{4.057455in}{0.766943in}%
\pgfsys@useobject{currentmarker}{}%
\end{pgfscope}%
\end{pgfscope}%
\begin{pgfscope}%
\pgftext[x=0.332769in,y=0.766943in,right,]{\rmfamily\fontsize{8.000000}{9.600000}\selectfont 200}%
\end{pgfscope}%
\begin{pgfscope}%
\pgfsetbuttcap%
\pgfsetroundjoin%
\definecolor{currentfill}{rgb}{0.000000,0.000000,0.000000}%
\pgfsetfillcolor{currentfill}%
\pgfsetlinewidth{0.501875pt}%
\definecolor{currentstroke}{rgb}{0.000000,0.000000,0.000000}%
\pgfsetstrokecolor{currentstroke}%
\pgfsetdash{}{0pt}%
\pgfsys@defobject{currentmarker}{\pgfqpoint{0.000000in}{0.000000in}}{\pgfqpoint{0.069444in}{0.000000in}}{%
\pgfpathmoveto{\pgfqpoint{0.000000in}{0.000000in}}%
\pgfpathlineto{\pgfqpoint{0.069444in}{0.000000in}}%
\pgfusepath{stroke,fill}%
}%
\begin{pgfscope}%
\pgfsys@transformshift{0.402214in}{1.073656in}%
\pgfsys@useobject{currentmarker}{}%
\end{pgfscope}%
\end{pgfscope}%
\begin{pgfscope}%
\pgfsetbuttcap%
\pgfsetroundjoin%
\definecolor{currentfill}{rgb}{0.000000,0.000000,0.000000}%
\pgfsetfillcolor{currentfill}%
\pgfsetlinewidth{0.501875pt}%
\definecolor{currentstroke}{rgb}{0.000000,0.000000,0.000000}%
\pgfsetstrokecolor{currentstroke}%
\pgfsetdash{}{0pt}%
\pgfsys@defobject{currentmarker}{\pgfqpoint{-0.069444in}{0.000000in}}{\pgfqpoint{0.000000in}{0.000000in}}{%
\pgfpathmoveto{\pgfqpoint{0.000000in}{0.000000in}}%
\pgfpathlineto{\pgfqpoint{-0.069444in}{0.000000in}}%
\pgfusepath{stroke,fill}%
}%
\begin{pgfscope}%
\pgfsys@transformshift{4.057455in}{1.073656in}%
\pgfsys@useobject{currentmarker}{}%
\end{pgfscope}%
\end{pgfscope}%
\begin{pgfscope}%
\pgftext[x=0.332769in,y=1.073656in,right,]{\rmfamily\fontsize{8.000000}{9.600000}\selectfont 400}%
\end{pgfscope}%
\begin{pgfscope}%
\pgfsetbuttcap%
\pgfsetroundjoin%
\definecolor{currentfill}{rgb}{0.000000,0.000000,0.000000}%
\pgfsetfillcolor{currentfill}%
\pgfsetlinewidth{0.501875pt}%
\definecolor{currentstroke}{rgb}{0.000000,0.000000,0.000000}%
\pgfsetstrokecolor{currentstroke}%
\pgfsetdash{}{0pt}%
\pgfsys@defobject{currentmarker}{\pgfqpoint{0.000000in}{0.000000in}}{\pgfqpoint{0.069444in}{0.000000in}}{%
\pgfpathmoveto{\pgfqpoint{0.000000in}{0.000000in}}%
\pgfpathlineto{\pgfqpoint{0.069444in}{0.000000in}}%
\pgfusepath{stroke,fill}%
}%
\begin{pgfscope}%
\pgfsys@transformshift{0.402214in}{1.380369in}%
\pgfsys@useobject{currentmarker}{}%
\end{pgfscope}%
\end{pgfscope}%
\begin{pgfscope}%
\pgfsetbuttcap%
\pgfsetroundjoin%
\definecolor{currentfill}{rgb}{0.000000,0.000000,0.000000}%
\pgfsetfillcolor{currentfill}%
\pgfsetlinewidth{0.501875pt}%
\definecolor{currentstroke}{rgb}{0.000000,0.000000,0.000000}%
\pgfsetstrokecolor{currentstroke}%
\pgfsetdash{}{0pt}%
\pgfsys@defobject{currentmarker}{\pgfqpoint{-0.069444in}{0.000000in}}{\pgfqpoint{0.000000in}{0.000000in}}{%
\pgfpathmoveto{\pgfqpoint{0.000000in}{0.000000in}}%
\pgfpathlineto{\pgfqpoint{-0.069444in}{0.000000in}}%
\pgfusepath{stroke,fill}%
}%
\begin{pgfscope}%
\pgfsys@transformshift{4.057455in}{1.380369in}%
\pgfsys@useobject{currentmarker}{}%
\end{pgfscope}%
\end{pgfscope}%
\begin{pgfscope}%
\pgftext[x=0.332769in,y=1.380369in,right,]{\rmfamily\fontsize{8.000000}{9.600000}\selectfont 600}%
\end{pgfscope}%
\begin{pgfscope}%
\pgfsetbuttcap%
\pgfsetroundjoin%
\definecolor{currentfill}{rgb}{0.000000,0.000000,0.000000}%
\pgfsetfillcolor{currentfill}%
\pgfsetlinewidth{0.501875pt}%
\definecolor{currentstroke}{rgb}{0.000000,0.000000,0.000000}%
\pgfsetstrokecolor{currentstroke}%
\pgfsetdash{}{0pt}%
\pgfsys@defobject{currentmarker}{\pgfqpoint{0.000000in}{0.000000in}}{\pgfqpoint{0.069444in}{0.000000in}}{%
\pgfpathmoveto{\pgfqpoint{0.000000in}{0.000000in}}%
\pgfpathlineto{\pgfqpoint{0.069444in}{0.000000in}}%
\pgfusepath{stroke,fill}%
}%
\begin{pgfscope}%
\pgfsys@transformshift{0.402214in}{1.687083in}%
\pgfsys@useobject{currentmarker}{}%
\end{pgfscope}%
\end{pgfscope}%
\begin{pgfscope}%
\pgfsetbuttcap%
\pgfsetroundjoin%
\definecolor{currentfill}{rgb}{0.000000,0.000000,0.000000}%
\pgfsetfillcolor{currentfill}%
\pgfsetlinewidth{0.501875pt}%
\definecolor{currentstroke}{rgb}{0.000000,0.000000,0.000000}%
\pgfsetstrokecolor{currentstroke}%
\pgfsetdash{}{0pt}%
\pgfsys@defobject{currentmarker}{\pgfqpoint{-0.069444in}{0.000000in}}{\pgfqpoint{0.000000in}{0.000000in}}{%
\pgfpathmoveto{\pgfqpoint{0.000000in}{0.000000in}}%
\pgfpathlineto{\pgfqpoint{-0.069444in}{0.000000in}}%
\pgfusepath{stroke,fill}%
}%
\begin{pgfscope}%
\pgfsys@transformshift{4.057455in}{1.687083in}%
\pgfsys@useobject{currentmarker}{}%
\end{pgfscope}%
\end{pgfscope}%
\begin{pgfscope}%
\pgftext[x=0.332769in,y=1.687083in,right,]{\rmfamily\fontsize{8.000000}{9.600000}\selectfont 800}%
\end{pgfscope}%
\begin{pgfscope}%
\pgfsetbuttcap%
\pgfsetroundjoin%
\definecolor{currentfill}{rgb}{0.000000,0.000000,0.000000}%
\pgfsetfillcolor{currentfill}%
\pgfsetlinewidth{0.501875pt}%
\definecolor{currentstroke}{rgb}{0.000000,0.000000,0.000000}%
\pgfsetstrokecolor{currentstroke}%
\pgfsetdash{}{0pt}%
\pgfsys@defobject{currentmarker}{\pgfqpoint{0.000000in}{0.000000in}}{\pgfqpoint{0.069444in}{0.000000in}}{%
\pgfpathmoveto{\pgfqpoint{0.000000in}{0.000000in}}%
\pgfpathlineto{\pgfqpoint{0.069444in}{0.000000in}}%
\pgfusepath{stroke,fill}%
}%
\begin{pgfscope}%
\pgfsys@transformshift{0.402214in}{1.993796in}%
\pgfsys@useobject{currentmarker}{}%
\end{pgfscope}%
\end{pgfscope}%
\begin{pgfscope}%
\pgfsetbuttcap%
\pgfsetroundjoin%
\definecolor{currentfill}{rgb}{0.000000,0.000000,0.000000}%
\pgfsetfillcolor{currentfill}%
\pgfsetlinewidth{0.501875pt}%
\definecolor{currentstroke}{rgb}{0.000000,0.000000,0.000000}%
\pgfsetstrokecolor{currentstroke}%
\pgfsetdash{}{0pt}%
\pgfsys@defobject{currentmarker}{\pgfqpoint{-0.069444in}{0.000000in}}{\pgfqpoint{0.000000in}{0.000000in}}{%
\pgfpathmoveto{\pgfqpoint{0.000000in}{0.000000in}}%
\pgfpathlineto{\pgfqpoint{-0.069444in}{0.000000in}}%
\pgfusepath{stroke,fill}%
}%
\begin{pgfscope}%
\pgfsys@transformshift{4.057455in}{1.993796in}%
\pgfsys@useobject{currentmarker}{}%
\end{pgfscope}%
\end{pgfscope}%
\begin{pgfscope}%
\pgftext[x=0.332769in,y=1.993796in,right,]{\rmfamily\fontsize{8.000000}{9.600000}\selectfont 1000}%
\end{pgfscope}%
\begin{pgfscope}%
\pgfsetbuttcap%
\pgfsetroundjoin%
\definecolor{currentfill}{rgb}{0.000000,0.000000,0.000000}%
\pgfsetfillcolor{currentfill}%
\pgfsetlinewidth{0.501875pt}%
\definecolor{currentstroke}{rgb}{0.000000,0.000000,0.000000}%
\pgfsetstrokecolor{currentstroke}%
\pgfsetdash{}{0pt}%
\pgfsys@defobject{currentmarker}{\pgfqpoint{0.000000in}{0.000000in}}{\pgfqpoint{0.069444in}{0.000000in}}{%
\pgfpathmoveto{\pgfqpoint{0.000000in}{0.000000in}}%
\pgfpathlineto{\pgfqpoint{0.069444in}{0.000000in}}%
\pgfusepath{stroke,fill}%
}%
\begin{pgfscope}%
\pgfsys@transformshift{0.402214in}{2.300509in}%
\pgfsys@useobject{currentmarker}{}%
\end{pgfscope}%
\end{pgfscope}%
\begin{pgfscope}%
\pgfsetbuttcap%
\pgfsetroundjoin%
\definecolor{currentfill}{rgb}{0.000000,0.000000,0.000000}%
\pgfsetfillcolor{currentfill}%
\pgfsetlinewidth{0.501875pt}%
\definecolor{currentstroke}{rgb}{0.000000,0.000000,0.000000}%
\pgfsetstrokecolor{currentstroke}%
\pgfsetdash{}{0pt}%
\pgfsys@defobject{currentmarker}{\pgfqpoint{-0.069444in}{0.000000in}}{\pgfqpoint{0.000000in}{0.000000in}}{%
\pgfpathmoveto{\pgfqpoint{0.000000in}{0.000000in}}%
\pgfpathlineto{\pgfqpoint{-0.069444in}{0.000000in}}%
\pgfusepath{stroke,fill}%
}%
\begin{pgfscope}%
\pgfsys@transformshift{4.057455in}{2.300509in}%
\pgfsys@useobject{currentmarker}{}%
\end{pgfscope}%
\end{pgfscope}%
\begin{pgfscope}%
\pgftext[x=0.332769in,y=2.300509in,right,]{\rmfamily\fontsize{8.000000}{9.600000}\selectfont 1200}%
\end{pgfscope}%
\end{pgfpicture}%
\makeatother%
\endgroup%

  \caption{
    Histogram showing the upper $90\%$ confidence limits on $10^5$ simulated toy samples of the $B^0\to\APDzero\APmuon\Pmuon$ dataset.
    The median (solid) and lower and upper $16\%$ percentiles (dashed) of the distribution have been marked.
  }
  \label{fig:expected}
\end{figure}

\chapter{Systematic uncertainties}

Apart from the uncertainty resulting from a limited size of the data sample, the analysis is affected by several systematic uncertainties resulting from the specific methods used to analyse the data.
These have to be understood for both the signal decay channel, as well as the normalization decay channel, which directly affects the result of the analysis through the normalization constant $\alpha$.

\subsubsection{Differences between real and simulated data samples}
One systematic uncertainty is the influence of differences between simulated and real decay candidates on the estimated signal efficiencies.
The presented preselection procedures for both the signal and normalization channels exclusively rely on mass cuts and highly efficiency PID cuts.
These are expected to be only weakly affected by differences between data and simulation.
Their systematic uncertainties can be analyzed by varying the cut window.
For example, in the case of $\PBzero\to\PJpsi\PKstar$ a cut on the invariant mass of the two muons around the \PJpsi mass can be analyzed by observing the change in cut efficiency when shifting the window by the difference of the observed \PJpsi masses in data and simulation.
The normalization yield can be increased by replacing some of the mass cuts with combined mass and PID cuts.
In this case, systematic uncertainties on the cut efficiencies could be decreased by measuring them from data.

The multivariate selection has been shown to be sensitive to differences between data and simulation (see section \ref{datamc}).
In the case of the standard partial reweighting method with PID variables resampled from data, a $9\%$ difference in the efficiency between data and simulation has been measured in the normalization sample, which can be assumed as a systematic uncertainty of this method.
An alternative method using a full, classifier-driven reweighting has been presented that reduces the difference to $1\%$.
Although the data-simulation difference has been reduced significantly, it remains to be seen if applying the reweighting procedure to the simulated signal samples introduces an additional error.

\subsubsection{Difference between phase space model and actual decay}
The fact that a phase space simulated data sample has been used can lead to an uncertainty on the branching fraction, as real decay candidates are expected to be more prevalent at low values of $q^2$, given the comparably large mass of of the \PDzero meson.
This is expected to affect the efficiency of the \PJpsi mass veto cut as well as the trigger efficiency (which varies with $q^2$).
This can be avoided by generating a simulated sample according to a theoretical model of the decay.
In the absence of this, the measurement could be limited to a $q^2$ range that allows 

\subsubsection{Choice of mass range for normalization fit}
The estimate of the normalization signal yield can be affected by a systematic uncertainty if it depends on the mass range included in the fit.
This can be determined by fitting in a larger mass range, where partially reconstructed background and $B_s\to\PJpsi\PKstar$ signal have to be described as part of the fit.

\chapter{Conclusion and outlook}

The decay $\PBzero\to\APDzero\APmuon\APmuon$ has been analyzed using data taken by LHCb corresponding to an integrated luminosity of \SI{3}{fb^{-1}}.
In order to minimize systematic uncertainties, a parallel measurement of the channel $\PBzero\to\PJpsi\PKstar$ has been carried out.
After rejecting specific physical backgrounds in both the signal and normalization channels, a multivariate selection procedure has been applied to the blinded signal dataset.
The normalization yield has been determined from a fit to the normalization decay sample.
The distribution of combinatorial background in the signal decay has been estimated.
Using the previous results and assuming a background-only hypothesis, the expected limit on the $\PBzero\to\APDzero\APmuon\APmuon$ branching fraction is determined to be
\begin{equation}
  \text{CL}_\text{expected}(90\%) = 6.8^{+5.1}_{-2.9}\,10^{-8}\:.
\end{equation}

Before the dataset can be unblinded, systematic uncertainties of the measurement must be investigated further.
In particular, it has to be shown that the measured classifier efficiency is reliable.

The yields of both the signal and normalization decays can be increased by exchanging the currently used mass veto cuts with combined mass and PID cuts.

It would also be possible to investigate other $B\to D\mu^+\mu^-$ decays.
For example, the decay $\PBzero\to\PJpsi(\APmuon\Pmuon)\APDzero$ can be analyzed using the same dataset.
In this case, the veto cut on the \PJpsi mass has to be inverted.

