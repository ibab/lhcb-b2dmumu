%!TEX root = ../TiKzPics.tex
% Author: Julian Wishahi
% E-Mail: julian.wishahi@tu-dortmund.de
% Date: June 9th 2014

\definecolor{phdorangeA}{HTML}{FC5716}
\definecolor{phdorangeB}{HTML}{BE4312}
\definecolor{phdorangeC}{HTML}{85310E}
\definecolor{phdorangeD}{HTML}{261107}
\definecolor{phdblueD}{HTML}{082026}
\definecolor{phdblueC}{HTML}{0E4253}
\definecolor{phdblueB}{HTML}{1986AA}
\definecolor{phdblueA}{HTML}{30A6B2}
\definecolor{phdwhiteA}{HTML}{F5EADA}
\definecolor{phdgreenA}{HTML}{91DE23}
\definecolor{phdgreenB}{HTML}{6DAA0F}
\definecolor{phdgreenC}{HTML}{38550A}
\definecolor{phdgreenD}{HTML}{1B2706}
\definecolor{phdvioletD}{HTML}{240026}
\definecolor{phdvioletC}{HTML}{4E0053}
\definecolor{phdvioletB}{HTML}{7F0087}
\definecolor{phdvioletA}{HTML}{B400C0}
\definecolor{leptoncolor}{HTML}{6495ED}
\definecolor{quarkcolor}{HTML}{12DF53}
\definecolor{gaugecolor}{HTML}{FFA15C}
\definecolor{higgscolor}{HTML}{000000}



\newcommand{\CommonParticleTextFormat}[5]
{\begin{minipage}[t][10ex][t]{5.5em}
    \centering
    \vspace{-0.6ex}
    \begin{minipage}[t][1.8ex][b]{5.5em}
      \centering
      {{#1}}
    \end{minipage}%
    \linebreak\vspace{-0.5ex}
    \begin{minipage}[t][3.5ex][t]{0.8em}
    \raggedleft
    \vspace{0.5ex}
    {{#2}}
    \linebreak
    {{#3}}
    \end{minipage}%
    \begin{minipage}[t][3.5ex][t]{3.4em}
    \centering
    \vspace{0.5ex}
    {{#4}}
    \end{minipage}%
    \begin{minipage}[t][3.5ex][t]{1.0em}
    \phantom{\sfrac{1}{2}}
    \end{minipage}%
    \\[1.5ex]
    \begin{minipage}[t][2.5ex][b]{5.5em}
      \centering
      {{#5}}  
    \end{minipage}
  \end{minipage}%
}

% \newcommand{\CommonParticleTextFormat}[5]{%
%   \begin{minipage}{2.9cm}
%     \centering
%     \begin{tabular}{R{0.4cm}C{2cm}L{0.4cm}}
%       \multicolumn{3}{c}{{#1}}       \\
%       {#2} & \multirow{3}{*}{{#4}} & \\
%       {#3} &                       & \\
%            &                       & \\
%       %\multicolumn{3}{c}{{#5}}       \\
%     \end{tabular}
%   \end{minipage}
% }

\newcommand\MeVcc{\si{MeV}}
\newcommand\eVcc{\si{eV}}
\newcommand\GeVcc{\si{GeV}}

\newcommand{\QuarkTextFormat}[5]
{
  \CommonParticleTextFormat{\footnotesize #1}{\footnotesize #2}{\footnotesize #3}{\Huge {#4}}{\footnotesize #5}
}

\newcommand{\LeptonTextFormat}[5]
{
  \CommonParticleTextFormat{\footnotesize #1}{\footnotesize #2}{\footnotesize #3}{\Huge {#4}}{\footnotesize #5}
}

\newcommand{\BosonTextFormat}[5]
{
  \CommonParticleTextFormat{\footnotesize #1}{\footnotesize #2}{\footnotesize #3}{\Huge {#4}}{\footnotesize #5}
}


\begin{tikzpicture}[scale=0.65, transform shape]
%% Fill Color Styles
  \tikzstyle{ParticleFill}    = [fill=white]
  \tikzstyle{QuarkFill}       = [fill=white]
  \tikzstyle{LeptonFill}      = [fill=white]
  \tikzstyle{GluonFill}       = [fill=white]
  \tikzstyle{PhotonFill}      = [fill=white]
  \tikzstyle{WeakFill}        = [fill=white]
  \tikzstyle{HiggsFill}       = [fill=white]

%% Element Styles
  % \tikzstyle{Particle}  = [ParticleFill, ultra thick, rounded corners,
  %                          text centered,
  %                          minimum width=3cm, minimum height=3cm, 
  %                          node distance=3.3cm, inner sep=0]
  \tikzstyle{Particle}  = [ParticleFill, ultra thick,
                           text centered,
                           minimum width=6em, minimum height=6em, 
                           node distance=6.7em, inner sep=0]
  \tikzstyle{Quark}     = [draw=quarkcolor, Particle, QuarkFill]
  \tikzstyle{Lepton}    = [draw=leptoncolor, Particle, LeptonFill]
  \tikzstyle{Gluon}     = [draw=gaugecolor, Particle, GluonFill]  
  \tikzstyle{Photon}    = [draw=gaugecolor, Particle, PhotonFill]  
  \tikzstyle{Weak}      = [draw=gaugecolor, Particle, WeakFill]
  \tikzstyle{Higgs}     = [draw=higgscolor, Particle, HiggsFill]

  \tikzstyle{GenerationLabel} = [font={\LARGE}, minimum width=2em, node distance=4.2em]
  \tikzstyle{PeriodLabel}     = [font={\LARGE}, node distance=2.0cm]
  \tikzstyle{TitleLabel}      = [font={\Huge\bfseries}]


  % 1. Generation
  \node[name=upquark, Quark]       
       {\QuarkTextFormat{$\SI{2.3}{\MeVcc}$}
                        {$+\sfrac{2}{3}$}{$\sfrac{1}{2}$}
                        {$u$}{up}};
  \node[name=downquark, below of=upquark, Quark]     
       {\QuarkTextFormat{$\SI{4.8}{\MeVcc}$}
                        {$-\sfrac{1}{3}$}{$\sfrac{1}{2}$}
                        {$d$}{down}};
  \node[name=electron, below of=downquark, Lepton, yshift=-0.5em]
       {\LeptonTextFormat{$\SI{0.511}{\MeVcc}$}
                         {$-1$}{$\sfrac{1}{2}$}
                         {$e$}{electron}};
  \node[name=electronneutrino, below of=electron, Lepton]  
       {\LeptonTextFormat{$<\SI{2}{\eVcc}$}
                         {$0$}{$\sfrac{1}{2}$}
                         {$\nu_{e}$}{$e$ neutrino}};

  % 2. Generation
  \node[name=charmquark, right of=upquark, Quark]       
       {\QuarkTextFormat{$\SI{1275}{\MeVcc}$}
                        {$+\sfrac{2}{3}$}{$\sfrac{1}{2}$}
                        {$c$}{charm}};
  \node[name=strangequark, below of=charmquark, Quark]     
       {\QuarkTextFormat{$\SI{95}{\MeVcc}$}
                        {$-\sfrac{1}{3}$}{$\sfrac{1}{2}$}
                        {$s$}{strange}};
  \node[name=muon, below of=strangequark, Lepton, yshift=-0.5em]
       {\LeptonTextFormat{$\SI{105.7}{\MeVcc}$}
                         {$-1$}{$\sfrac{1}{2}$}
                         {$\mu$}{muon}};
  \node[name=muononneutrino, below of=muon, Lepton]  
       {\LeptonTextFormat{$<\SI{0.19}{\MeVcc}$}
                         {$0$}{$\sfrac{1}{2}$}
                         {$\nu_{\mu}$}{$\mu$ neutrino}};

  % 3. Generation
  \node[name=topquark, right of=charmquark, Quark]       
       {\QuarkTextFormat{$\SI{173.5}{\GeVcc}$}
                        {$+\sfrac{2}{3}$}{$\sfrac{1}{2}$}
                        {$t$}{top}};
  \node[name=bottomquark, below of=topquark, Quark]     
       {\QuarkTextFormat{$\SI{4650}{\MeVcc}$}
                        {$-\sfrac{1}{3}$}{$\sfrac{1}{2}$}
                        {$b$}{bottom}};
  \node[name=tau, below of=bottomquark, Lepton, yshift=-0.5em]
       {\LeptonTextFormat{$\SI{1777}{\MeVcc}$}
                         {$-1$}{$\sfrac{1}{2}$}
                         {$\tau$}{tau}};
  \node[name=tauneutrino, below of=tau, Lepton]  
       {\LeptonTextFormat{$<\SI{18.2}{\MeVcc}$}
                         {$0$}{$\sfrac{1}{2}$}
                         {$\nu_{\tau}$}{$\tau$ neutrino}};

  % Gauge Bosons
  \node[name=gluon, right of=topquark, Gluon, xshift=+0.7em]       
       {\BosonTextFormat{$0$}
                        {$0$}{$1$}
                        {$g$}{gluon}};
  \node[name=photon, below of=gluon, Photon]     
       {\BosonTextFormat{$0$}
                        {$0$}{$1$}
                        {$\gamma$}{photon}};
  \node[name=Zboson, below of=photon, Weak, yshift=-0.5em]
       {\BosonTextFormat{$\SI{91.2}{\GeVcc}$}
                        {$0$}{$1$}
                        {$Z$}{$Z$ boson}};
  \node[name=Wboson, below of=Zboson, Weak]  
       {\BosonTextFormat{$\SI{80.4}{\GeVcc}$}
                        {$\pm 1$}{$1$}
                        {$W$}{$W$ boson}};

  % Higgs
  \node[name=Higgs, right of=gluon, Higgs, xshift=+0.7em]       
       {\BosonTextFormat{$\SI{126}{\GeVcc}$}
                        {$0$}{$0$}
                        {$H$}{Higgs boson}};

  % Group LABELS
  \node[name=Family1, above of=upquark,     GenerationLabel] {1.};
  \node[name=Family2, above of=charmquark,  GenerationLabel] {2.};
  \node[name=Family2, above of=topquark,    GenerationLabel] {3.};

  \node[name=Leptons, left of=electronneutrino, node distance=1.00cm, rotate=90, text=leptoncolor, PeriodLabel, text width=6em]
       {leptons};
  \node[name=Quarks,  left of=downquark, node distance=1.00cm, rotate=90, text=quarkcolor, PeriodLabel, text width=6em] 
       {quarks};

  \coordinate (Middle) at ($(Wboson)!0.5!(Zboson)$);

  \node[name=Bosons,  right of=Middle, rotate=90, text=gaugecolor, PeriodLabel, 
        minimum width=12.7em, node distance=2cm, text width=12em] 
       {gauge bosons};

  % Description
    % Description
  \node[font=\footnotesize,color=black!40,align=left,minimum width=6em,text width=5em,inner sep=0] at (-1.3,+0.79) {mass};
  \node[font=\footnotesize,color=black!40,align=left,minimum width=6em,text width=5em,inner sep=0] at (-1.3,+0.16) {charge};
  \node[font=\footnotesize,color=black!40,align=left,minimum width=6em,text width=5em,inner sep=0] at (-1.3,-0.31) {spin};

\end{tikzpicture}
