
\chapter{The LHCb Experiment} % 6

The LHCb experiment is dedicated to the precise study of $CP$ violation and rare decays of beauty and charm hadrons at the \gls{LHC}, \gls{CERN} in Geneva, Switzerland.

In contrast to the other three large detector experiments at the LHC (ATLAS, CMS and ALICE), which are multi-purpose detectors, LHCb is a single-arm forward spectrometer specifically optimized for the precise study of $b$ and $c$ quarks.

During Run I, which occurred from 2010 to 2012, the experiment collected proton-proton collision data corresponding to a total integrated luminosity of \SI{3}{fb^{-1}}.

\section{The Large Hadron Collider}

\begin{itemize}
  \item A nice figure showing the LHC and other facilities
  \item When was the LHC built?
  \item Which are the four large experiments and what do they do?
\end{itemize}

\begin{figure}
  \centering
  \includegraphics[width=0.75\textwidth]{cern-complex.pdf}
  \caption{The CERN accelerator complex}
\end{figure}

\section{Experimental setup}

\subsection{Tracking}

A highly accurate reconstruction of particle tracks is vital for the precise study of particle decays.
For this purpose, LHCb is equipped with several components for tracking, vertexing and momentum estimation.

The \gls{VELO} is situated close to the interaction point and records $r$ and $φ$ coordinates of particle tracks..
It consists of two halves, each consisting of an array of silicon detector modules perpendicular to the beam axis.
When the LHC beam is stable, the two \gls{VELO} halves can be brought together.

% TT is a silicon strip detector, upstream (in front of magnet)
% IT is a silicon strip detector, downstream (after magnet)
% OT is a drift detector, extends acceptance
% The magnet!
% magnet field map very precise, allows to infer particle momenta

\subsection{Particle identification}

% RICH

\subsection{Trigger and data processing}

% L0 trigger
% HLT
% reconstruction software: DaVinci, ...

